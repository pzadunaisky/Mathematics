%%%%%%%%%%%%%%%%%%%%%% Generalities %%%%%%%%%%%%%%%%%%5
\documentclass[11pt,fleqn]{article}
\usepackage[paper=a4paper]
  {geometry}

\pagestyle{plain}
\pagenumbering{arabic}
%%%%%%%%%%%%%%%%%%%%%%%%%%%%%%%%
% esto tiene que estar, en este orden...
\usepackage[small]{titlesec}
\usepackage{paragraphs}
\usepackage{hyperref}
\usepackage{amsthm,thmtools}

%\linespread{1.2}
\setlength{\parskip}{1.2ex}

\usepackage[utf8,latin1]{inputenc}
\usepackage[spanish,english]{babel}
\usepackage{enumerate}
\usepackage[alphabetic,initials]{amsrefs}
\usepackage{amsfonts,amssymb,amsmath}
%\usepackage{stmaryrd}
\usepackage{mathtools}
\usepackage{graphicx}
\usepackage[poly,arrow,curve,matrix]{xy}
\usepackage{wrapfig}

%\swapnumbers




% numbered versions
\declaretheoremstyle[headformat=swapnumber, spaceabove=\paraskip,
bodyfont=\itshape]{mystyle}
\declaretheorem[name=Lema, sibling=para, style=mystyle]{Lemma}
\declaretheorem[name=Proposici\'on, sibling=para, style=mystyle]{Proposition}
\declaretheorem[name=Teorema, sibling=para, style=mystyle]{Theorem}
\declaretheorem[name=Corolario, sibling=para, style=mystyle]{Corollary}
\declaretheorem[name=Definici\'on, sibling=para, style=mystyle]{Definition}

% unnumbered versions
\declaretheoremstyle[numbered=no, spaceabove=\paraskip,
bodyfont=\itshape]{mystyle-empty}
\declaretheorem[name=Lemma, style=mystyle-empty]{Lemma*}
\declaretheorem[name=Proposition, style=mystyle-empty]{Proposition*}
\declaretheorem[name=Theorem, style=mystyle-empty]{Theorem*}
\declaretheorem[name=Corollary, style=mystyle-empty]{Corollary*}
\declaretheorem[name=Definition, style=mystyle-empty]{Definition*}

% plain style
\declaretheoremstyle[
        headformat={{\bfseries\NUMBER.}{\itshape\NAME}\NOTE\ignorespaces},
        spaceabove=\paraskip, 
        headpunct={.},
        headfont=\itshape,
        bodyfont=\normalfont
        ]{mystyle-plain}
\declaretheorem[sibling=para, style=mystyle-plain]{Example}
\declaretheorem[sibling=para, style=mystyle-plain]{Remark}

% proofs, just as in amsthm but with adapted spacing
\makeatletter
\renewenvironment{proof}[1][\proofname]{\par
  \pushQED{\qed}%
  \normalfont \topsep.75\paraskip\relax
  \trivlist
  \item[\hskip\labelsep
        \itshape
    #1\@addpunct{.}]\ignorespaces
}{%
  \popQED\endtrivlist\@endpefalse
}
\makeatother

%%%%%%%%%%%%%%%%%%%%%%%%%%% The usual stuff%%%%%%%%%%%%%%%%%%%%%%%%%
\newcommand\NN{\mathbb N}
\newcommand\CC{\mathbb C}
\newcommand\QQ{\mathbb Q}
\newcommand\RR{\mathbb R}
\newcommand\ZZ{\mathbb Z}
\newcommand\KK{\mathbb K}

\newcommand\maps{\longmapsto}
\newcommand\ot{\otimes}
\renewcommand\to{\longrightarrow}
\renewcommand\phi{\varphi}
\newcommand\stack[2]{\genfrac{}{}{0pt}{2}{#1}{#2}}
\newcommand\uu[1]{\underline{#1}}

%%%%%%%%%%%%%%%%%%%%%%%%% Specific notation %%%%%%%%%%%%%%%%%%%%%%%%%

\newcommand\A{\mathcal A}
\newcommand\B{\mathcal B}
\renewcommand\L{\mathcal L}
\newcommand\M{\mathcal M}
\newcommand\F{\mathcal F}
\newcommand\G{\mathcal G}
\newcommand\GG{\Gamma}
\renewcommand\b{\mathfrak{b}}
\newcommand\g{\mathfrak{g}}
\newcommand\n{\mathfrak{n}}
\newcommand\opp{\mathsf{opp}}
\DeclareMathOperator\Id{\mathsf{Id}}
\DeclareMathOperator\st{\mathsf{st}}

\newcommand\C{\mathsf C}
\newcommand\K{\mathsf K}
\newcommand\D{\mathsf D}

\DeclareMathOperator\Mod{\mathsf{Mod}}
\DeclareMathOperator\Hom{\mathsf{Hom}}
\DeclareMathOperator\Ext{\mathsf{Ext}}
\DeclareMathOperator\Tor{\mathsf{Tor}}

\DeclareMathOperator\GrMod{\mathsf{GrMod}}
\DeclareMathOperator\GrHom{\underline{\mathsf{Hom}}}
\DeclareMathOperator\GrExt{\underline{\mathsf{Ext}}}
\DeclareMathOperator\GrTor{\underline{\mathsf{Tor}}}

\DeclareMathOperator\Ab{\mathsf{Ab}}
\DeclareMathOperator\ShHom{\mathcal Hom}
\DeclareMathOperator\ShExt{\mathcal Ext}
\DeclareMathOperator\ShTor{\mathcal Tor}

\DeclareMathOperator\pd{pd}
\DeclareMathOperator\id{id}
\DeclareMathOperator\depth{depth}
\DeclareMathOperator\Spec{Spec}
\DeclareMathOperator\supp{supp}
\DeclareMathOperator\ann{ann}
\DeclareMathOperator\im{Im}
\DeclareMathOperator\gr{gr}

%%%%%%%%%%%%%%%%%%%%%%%%%%%%%%%%%%%%%% TITLES %%%%%%%%%%%%%%%%%%%%%%%%%%%%%%

\title{Toric Degenerations of Schubert Varieties [P. Caldero]}
\author{[caldero.tex]}
\date{26/7/12}
\begin{document}
\maketitle
These are notes on P. Caldero's paper of the same title. We follow the enummeration of
this paper.

\section{Notation and recollection on the dual cannonical basis}
\paragraph
Standard notations on Lie Groups and their Lie algebras. $P$ is the weight lattice,
$P^+ = \sum \NN \varpi_i$ is the semigroup of dominant weights. $W$ is the Weyl group, 
$w_0 \in W$ is the longest word. $(,)$ is the $W$-invariant form on $P$

\paragraph
Defines $U_q(\g)$ over $\KK = \CC(q^{1/d})$. This is the algebra referred to as
$U_q(\g,P)$ in Jantzen's book, chapter 4 (he writes $\Lambda$ for $P$). There is a
diference in the definition of the Hopf algebra structure from the $U_q(\g)$ of Jantzen,
possibly because of this. Thus in our case $U_q(\g)$ is generated over $\KK$ by
$K_\lambda, E_i, F_i$, where $\lambda \in P$, $i = 1, \ldots n$ is the rank of $\mathfrak
h$, and relations
\begin{align*}
	K_0 &= 1 & K_{\lambda}K_\mu &= K_{\lambda + \mu} \\
	K_\lambda E_i K_{-\lambda} &= q^{(\lambda, \alpha_i)}E_i &
	K_\lambda F_i K_{-\lambda} &= q^{-(\lambda, \alpha_i)}F_i \\
				&&\mbox{ Quantum Serre Relations}
\end{align*}
It has a Hopf Algebra structure given by blah.

\paragraph
For every $\lambda \in P$, let $V_q(\lambda)$ be the simple module with highest weight =
$\lambda$. There is an injective morphism $V_q(\lambda)^* \ot_k V_q(\lambda) \to
U_q(\g)^*$ given by $\chi \ot v (u) = \chi(u.v)$. For every $\lambda \in P^+$, let
$C(\lambda) \subset U_q(\g)^*$ be the the $\KK$-span of $\chi \ot v$ with $\chi \in
V_q(\lambda)^*$ and $v \in V_q(\lambda)$. If we restrict to $v$ of highest weight, we
write $C^+(\lambda)$.

Set $\CC_q[G]= \bigoplus_{\lambda \in P^+} C(\lambda)$, $\CC_q[G/U] = \bigoplus_{\lambda
\in P^+} C^+(\lambda)$. In the classic case, we have isomorphisms between these algebras
and the algebras of regular functions over $G$ and $G/U$ respectively. By the product
formula of $U_q(\g)^*$, $\CC_q[G/U]$ is a $P^+$-graded algebra.

\paragraph
There is a Hopf pairing between $U_q(\b)$ and $U_q(\b^-)$, given by
\begin{align*}
	(K_\lambda, K_\mu) &= q^{-(\lambda,mu)} & (K_\lambda,F_i) = (E_i,K_\lambda) &= 0\\
		(E_i, F_j) &= \delta_{ij}(1-q^{*_i})^{-1} 
\end{align*}
If we denote by $\overline{-}$ the automorphism of $U_q(\g)$ sending $K_\lambda$ to
$K_{-\lambda}$ and exchanging $E_i, F_i$, we see that $(X, \overline X) \neq 0$, so in
fact for any positive weight $\beta$, $(,)$ restricted to $U_q(\b)_\beta \times
U_q(\b^-)_\beta$ is nondegenerate.

This can be extended to a nondegenerate bilinear form over $U_q(\g)$
$$ \left\langle XK_\lambda Y, Y'K_\mu X' \right\rangle = (X,
S^{-1}(Y'))(S^{-1}(X'),Y)q^{-(\lambda, \mu)/2}$$

\paragraph
Define the maps
\begin{align*}
	\beta&: U_q(\b) \rightarrow U_q(\b^-)^*, & \beta(u)(v) &= (u,v); &
	\zeta&: U_q(\g) \rightarrow U_q(\g)^*, & \zeta(u)(v) &= \langle u, v \rangle
\end{align*}
Since the bilinear forms are nondegenerate, $\beta$ and $\zeta$ are injective. It follows
from the axioms of a Hopf pairing that $\beta$ is an \emph{anti}-homomorphism of algebras.

Denote by $\rho: U_q(\g)^* \to U_q(\b)^*$ the restriction morphism
\begin{Proposition*}
	The restriction of $\rho$ to $\CC_q[G/U]$ is injectve. Moreover, for all $\lambda
\in P^+$ we have: $(i)$ for all $e \in U_q(\n)$, $\rho(\zeta(eK_{-2 \lambda})) =
\beta(eK_{-\lambda})$, $(ii)$ There exists a unique subspace $E_\lambda$ of $U_q(\n)$ such
that $\zeta(E_\lambda K_{-2\lambda}) = C^+(\lambda)$
\end{Proposition*}

\begin{proof}
	Take a matrix coefficient $c = \xi \ot v_\lambda \in V(\lambda)^* \ot V(\lambda)$.
If $\rho(c) = 0$ then $\xi = 0$ over $U_q(\b^-) v_\lambda = V(\lambda)$, so $c = 0$. Now
let $c = \xi_1 \ot v_{\lambda_1} + \ldots + \xi_n \ot v_{\lambda_n}$ with $\rho(c) = 0$.
Now $c$ can be seen as the matrix coefficient $\xi \ot v = \xi_1 \oplus \ldots \oplus \xi_n \ot
(v_{\lambda_1} + \ldots + v_{\lambda_n}) \in V^* \ot V$, where $V = V(\lambda_1) \oplus
\ldots \oplus V(\lambda_n)$. By hypothesis $U_q(\b^-) v$ is contained inside the kernel of
$\xi$, so in particular $\xi (K_\lambda v) = 0$ for all $\lambda$. Now $K_\lambda v =
\sum_i q^{(\lambda,\lambda_i)} v_{\lambda_i}$. Now think of something clever. 

Item $(i)$ is simply replacing the values in the definitions. For item $(ii)$,
see 2.1 \emph{Remark}.
\end{proof}

\paragraph
The global/cannonical basis of Kashiwara/Lusztig is introduced. This is a basis $\B$ of
$U_q(\n^-)$ with a lot of magical properties that Caldero quotes freely. It would be nice
to see some proofs, but for the moment we repeat them

Since we have a Hopf pairing $(-,-): U_q(\n) \times U_q(\n^-) \to \KK$, $\B$ induces a
dual basis $\B^* \subset U_q(\n)$. 

Kashiwara has defined operators $\tilde E_i, \tilde F_i:U_q(\n^-) \to U_q(\n^-)$, such
that $\tilde E_i(b) = b' \in \B \cup \{0\} \mod q^{-1}\ZZ[q^{-1}]\B$, and similarly for
$\tilde F_i$. There is an associated map $\tilde e_i : \B \to \B \cup \{0\}$ sending $b$ to
$b'$, and analogously for $\tilde f_i$. For $i = 1, \ldots, n$ set $\varepsilon_i(b) =
\max\{r|\tilde e_i^r(b) \neq 0\}$. This are positive integers by definition, see [K, 4.2].

Let $L_i$ be the adjoint of left multiplication $u \mapsto F_iu$ for $(-,-)$, and denote
by $L_i^{(r)}= \frac{1}{[r]_i !}L_i^r$. Then $L_i$
is a quantum derivation in $U_q(\n)$, with $L_i(e_\alpha e_\beta) = L_i(e_\alpha)e_\beta +
q^{(\alpha,\alpha_i)}e_\alpha L_i(e_\beta)$.
\begin{proof}
	Let $x \in U_q(\n)$, then $(x,F_i u) = (x_1 \ot x_2, F_i \ot u) = ((x_1,F_i)x_2,u)$
so $L_i(x) = (x_1,F_i)x_2$. Hence if $e = e_\alpha$ 
\begin{align*}
	L_i(e u) 
		&= ((e u)_1,F_i) (eu)_2 
		 = (e_1 u_1, F_i) e_2 u_2
		 = ((u_1, F_i)(e_1,K_{\alpha_i}^{-1}) + (u_1,1)(e_1,F_i))e_2 u_2 \\
		&=(e_1,F_i)e_2 (u_1,1)u_2 + (e_1,K_{\alpha_i}^{-1})e_2 (u_1,F_i)u_2 
		 = L_i(e)u + q^{(\alpha_i,\alpha)} e L_i(u)
\end{align*}
[Why is $(e_1,K_{\alpha_i}^{-1})= q^*$?]
\end{proof}

A few facts on cannonical bases:
\begin{Theorem*}
	For $b,b' \in \B$: $(i)$ $bb' \in \ZZ[q,q^{-1}]\B$, $(ii)$ $b^* b'^* \in
\ZZ[q,q^{-1}]\B^*$, $(iii)$ $L_i^{(\varepsilon_i(b))}(b^*) = (\tilde
e_i^{(\varepsilon_i(b))}(b))^*$, $(iv)$ $\overline b \in \B$.
\end{Theorem*}
The first fact is standard theory of crystal bases, and the second can be found in [L,
14.4.3, 14.4.4]. According to Caldero, $(iii)$ is a consequence of [L, 14.3.2] by
dualization, though I fail to see this. The fourth one is [14.4.3, (c)]. Caldero uses
several times that (iii) also implies that $L^{(\varepsilon_i(b)+1)}(b^*) = 0$, which I
don't see either.

\begin{Theorem*}
	Fix $\lambda \in P^+$, $\lambda = \sum \lambda_i \varpi_i$, and set
		$$ \B_\lambda = \{b \in \B| \varepsilon_i(\overline b) \leq \lambda i\}$$
	Then $\B_\lambda = \{b \in \B|bv_\lambda \neq 0\}$, the assignation $b_\lambda: e 
\mapsto b v_\lambda$ is injective on $\B$, and its image is a basis of $V_q(\lambda)$.
\end{Theorem*}

\paragraph $\CC[q,q^{-1}]$-forms of modules

Now we use the properties of the basis $\B$ to define similar objects over $A =
\CC[q,q^{-1}]$. 

Let $U_A(\n^-)$ be the $A$-submodule of $U_q(\n^-)$ generated by $\B$.
Then by [J, Theorem 11.10(b)] this is the $A$-algebra generated by
$F_i^{(m)}=\frac{1}{[m]_i!}F_i^m$. 

Let $U_A$ be the $A$-subalgebra of $U_q(\g)$ generated by $E_i^{(m)}, F_i^{(m)}$.
Set $V_A = U_A(\n^-) v_\lambda \subset V_q(\lambda)$. Then $V_A(\lambda)$ is a free module
with basis $\B_\lambda v_\lambda$.

By [J, Theorem 11.19] we get that $\B$ is compatible with speacilization, that is, seeing
$\CC \cong_A A/(q-1)A$, we have $\CC \ot_A U_A(\n^-) \cong U(\n^-)$, $\CC \ot_A
V_A(\lambda) = V(\lambda)$. Also $V_A(\lambda)^*$ has a natural $U_A(\n)$-module structe
and it specializes to $V(\lambda)^*$.

\paragraph
Let $\lambda \in P^+$ and $w \in W$. $V_q(\lambda)$ verifies the Weyl character formula,
or in other words, the $\KK$-dimension of each component agree with the $\CC$-dimension of
the corresponding component of $V(\lambda)$, and in the classical case, $V(\lambda)_{w
\lambda}$ has dimension one, so there is up to scalars a unique element $v_{w\lambda}$. 
Set $V_{q,w}(\lambda)= U_q(\b)v_{w \lambda}$. From [K, Theorem 12.14] we get

\begin{Theorem*}
	There exists a subset $\B_w \subset \B$ such that $V_{q,w}(\lambda)$ is spanned by
$\B_w v_{\lambda}$. Moreover, for all $b \in \B_\lambda$, $\delta(b) \in \langle \B_
w\rangle \ot \langle K_\mu \B_w, \mu \in P\rangle$.
\end{Theorem*}

In particular, $V_{q,w}(\lambda)^\perp \subset V_q(\lambda)^*$ is generated by $B_\lambda
\setminus \B_w$, and $V_{q,w}(\lambda)^*$ by the image of $\B_w^* \subset V_q(\lambda)$
through the quotient morphism (Caldero insists on writing $\B_w \cap \B_\lambda$, but by
definition $b \in \B_w \Rightarrow b v_\lambda \neq 0$, so $\B_w \subset \B_\lambda$).

Set $V_{A,w}(\lambda)$ the $A$-module generated by $\B_\lambda \cap B_w$. It specialices
at $q = 1$ to $V_w(\lambda)$

\section{A multiplicaive property and string parametriations of the dual canonical basis}
\paragraph
For $\lambda \in P^+$, denote by $\pi_\lambda(b)^* \in V_q(\lambda)^*$ the element such
that $\pi_\lambda(b)^*(b' v_\lambda) = \delta_{b,b'}$.

\begin{Lemma*}
For all $\lambda \in P^+$ and $b \in \B_\lambda$, $\zeta(b^* K_{-2\lambda}) =
\pi_\lambda(b)^* \ot v_\lambda$, $\beta(b^*K_{-\lambda}) = \rho(\pi_\lambda(b)^* \ot
v_\lambda)$.
\end{Lemma*}
\begin{proof}
	There is a basis for $U_q(\b^-)$ formed by $\{b K_\mu| b \in \B, \mu \in P^+\}$.
Now 
\begin{align*}
	\beta(b^*K_{-\lambda})(b' K_\mu) = (b^* K_{-\lambda}, b' K_\mu) =
		(b^*,b')q^{(\lambda, \mu)} = \delta_{b,b'}q^{(\lambda,\mu)} =
\pi_\lambda(b)^* (b'K_\mu v_\lambda).
\end{align*}
I can't prove the other claim.
\end{proof}

\emph{Remark.} The set $E_\lambda$ from 1.5 is thus seen to be $\B_\lambda^*$.

Let $A_q[G/U]$ be the sub $A$-module of $\CC_q[G]$ generated by the $\pi_\lambda(b)^* \ot
v_\lambda$. 

Write $$b^* b'^* = \sum_{b'' \in \B} d_{b,b'}^{b''} b''^* $$.

\begin{Proposition*}
	\begin{enumerate}
		\item $\KK \ot_A A_q[G/U] = \CC_q[G/U]$
		\item $A_q[G/U]$ is an $A$-algebra.
		\item If $b \in \B_{\lambda}, b' \in \B_{\lambda'}$, then $d_{b,b'}^{b''}
			\neq 0$ implies $b'' \in \B_{\lambda + \lambda'}$
		\item $A_q[G/U]/(q-1)A_q[G/U] \cong \CC[G/U]$.
	\end{enumerate}
\end{Proposition*}
\begin{proof}
	The first item is obvious since $A_q[G/U]$ has an $A$ basis which is a $\KK$-basis
of $\CC_q[G/U]$.

	Since $\beta$ is an injective antihomorphism of algebras, we have that
	\begin{align*}
		\beta^{-1}(\pi_\lambda(b)^* \ot v_\lambda \cdot \pi_{\lambda'}(b')^* \ot
			v_{\lambda'}) &= b'^* K_{-\lambda'} \cdot b^* K_{-\lambda} = 
			q^{-(\lambda',w(b*))}\sum d_{b',b}^{b''} b'' K_{- \lambda' - \lambda}
	\end{align*}
	As pointed out before, $\CC_q[G/U]$ is $P^+$-graded, and therefore this product is
	in $C^+(\lambda + \lambda')$. By the previous remark, this implies 3. Applying
	$\beta$ we see that the product $(\pi_\lambda(b)^* \ot v_\lambda \cdot
	\pi_{\lambda'}(b')^* \ot v_{\lambda'}$ is in $A_q[G/U]$. Finally 4. follows from
	1.7.
\end{proof}

\begin{Corollary*}
	Suppose $d_{b,b'}^{b''} \neq 0$. Then $\varepsilon(b'') \leq \varepsilon(b) +
\varepsilon(b')$.
\end{Corollary*}
\begin{proof}
	Consider the antiautomorphism $\overline{(-)}$ defined by $K_\lambda \mapsto
	K_{-\lambda}, E_i \mapsto E_i, F_i \mapsto F_i$. We have already mentioned than
	$\B$ is stable by it. Applying it to the previous equality, we get $d_{\overline
	b', \overline b}^{\overline b''} = d_{b,b'}^{b''}$. 

	Put $\lambda = \sum \varepsilon_i(b) \varpi_i$ and $\lambda' = \sum
	\varepsilon_i(b') \varpi_i$. From 1.6, Theorem 2 we get that $\overline b \in
	\B_\lambda$ and $\overline b' \in \B_{\lambda'}$. Hence $\overline b'' \in
	\B_{\lambda + \lambda'}$. Once again by 1.6 Theorem 2, we get our inequality.
\end{proof}

\paragraph
Fix a reduced decomposition $\tilde w_0 = s_{i_1} s_{i_2} \ldots s_{i_N}$ of $w_0$, where
$N = \dim \n$. For any $u \in U_q(\n)$, $1 \leq i \leq n$, set
$$ a_i(u) = \max\{ r, L_i^r(u) \neq 0\}, \Lambda_i(u) = L_i^{(a_i(u))}(u)$$ 
For any $b \in \B$ set
$$ A_{\tilde w_0}(b) = (a_{i_N}(b^*), a_{i_{N-1}}(\Lambda_{i_N}(b^*)), \ldots,
a_{i_1}(\Lambda_{i_2} \ldots \Lambda_{i_N}(b^*))) $$
Caldero goes on to claim that this parametrization coincides with that of [Li], given just
before \emph{Remark 1.4}. That this is true (even that the parametrized objects are the
same!) requires a careful reading of [Li]. He also give a reference to a paper by
Berenstein and Zelevinsky, which follows from the fact that $\varepsilon_i(b) = a_i(b^*)$.
Next a proof of this:

\begin{proof}
	Remember from 1.6 that $L_i^{(\varepsilon_i(b))}(b^*) = \tilde
e_i^{\varepsilon_i(b)}(b)^* \neq 0$. Hence $a_i(b^*) \leq \varepsilon_i(b)$. Now
$(L^{(\varepsilon(b)+1)}(b^*),b') = c (\tilde e_i^{\varepsilon_i(b)}(b), F_i b')$, where
$c \in \KK$. Now $F_i b' = c' \tilde f_i b'$, so this product is nonzer if and only if
$\tilde f_i b' = \tilde e_i^{\varepsilon_i(b)}(b)$, in which case $b' = \tilde
e_i^{\varepsilon_i(b) + 1}(b) = 0$, a contradiction. Hence $L^{(\varepsilon(b)+1)}(b^*) =
0$, and we have the desired equality.
\end{proof} 


\begin{Theorem*}
	The map $A_{\tilde w_0}$ is injective. Denote its image by $\mathcal C$. Then
$\mathcal C$ is the set of integral points of a rational polyhedral cone in $\RR^N$.
Moreover, if $\GG_{\tilde w_0}:= \{(\lambda, \psi) \in P^+ \times \mathcal C | \psi \in
\A_{\tilde w_0}(\B_\lambda)\}$, the same is true of this new set.
\end{Theorem*}
Injectivity for the map $A_{\tilde w_0}$ is discussed in [Li, Remark 1.4], with an offhand
reference to another papaer by Littelman. The statement for $\mathcal C$ is Proposition
1.5 (a) of this same article. I have no idea how to prove the claim for $\GG_{\tilde w_0}$

\paragraph
For $\psi \in \mathcal C$ write $b^\psi = A_{\tilde w_0}^{-1}(\psi)$. Also $d_{\psi,
\psi'}^{\psi''} = d_{b^{\psi}, b^{\psi'}}^{b^{\psi ''}}$. Put the total lexicographic
order on $\NN^N$
\begin{Theorem*}
	
\end{Theorem*}
	Let $b,b', b'' \in \B$, $A_{\tilde w_0}(b) = \psi$, $A_{\tilde w_0}(b') = \psi'$.
Then $d_{b,b'}^{b''} \neq 0$ implies $A_{\tilde w_0}(b'') \leq \psi + \psi'$. Furthermore,
$d_{\psi,\psi'}^{\psi + \psi'} = q^*$.
\begin{proof}
	Since $L_i$ is a quantum derivation and $U_q(\n)$ is an integral domain, $a_i(uv) 
= a_i(u) + a_i(v)$. By definition
		$$ b^* b'^* = \sum d_{b,b'}^{b''} b''^*$$
Now since $\varepsilon_{i_1}(b) = a_{i_1}(b) = \psi_1$ and idem for $b', b''$ [see the
comentary right after Theorem 1.6], we have $\psi''_1 \leq \psi_1 + \psi'_1$ by
Corollary 2.1. 

Apply $L_{i_N}^{\psi_1 + \psi'_1}$ on both sides. On the left hand side, only one term
survives, of the form $q^? \Lambda_{i_N}(b^*) \Lambda_{i_N}(b'^*)$. On the right, only the
terms with $\psi_1'' = \psi_1 + \psi'_1$ survive, and so we have
	$$ q^? \Lambda_{i_N}(b^*) \Lambda_{i_N}(b'^*) = \sum_{\psi_1 + \psi'_1 = \psi''_1} 
		d^{b''}_{b,b'} \Lambda_{i_N}.$$
Now by Theorem 1.6 (iii), $\Lambda_{i_N}(b^*)$ is an element of the dual canonical basis 
[This one \emph{is} a clear consequence of 1.6 (iii)!]. Hence we may proceed by induction
to prove that $\psi'' \leq \psi + \psi'$ in the lexicographic ordering.

To finish, notice that at the end of the induction we are left with
$$ q^? \Lambda_{i_1} \Lambda_{i_2} \ldots \Lambda_{i_N}(b) \Lambda_{i_1} \Lambda_{i_2}
\ldots \Lambda_{i_N}(b') = \sum d_{b,b'}^{b''} \Lambda_{i_1} \Lambda_{i_2} \ldots
\Lambda_{i_N}(b'').$$
where the sum is taken over all terms such that $\psi''_k = \psi_k + \psi'_k$, that is,
the only element on the rhs is $b^{\psi + \psi''}$ [!]. Now Caldero states that from [Li,
par 1] we have that $\Lambda_{i_1} \ldots \Lambda_{i_N}(b^*) = 1$ for all elements $b^*$
of the dual cannonical basis. Hence $d_{\psi,\psi'}^{\psi+\psi'} = q^?$. I don't know
where in $[Li, par 1]$ this is result is hidden.
\end{proof}

\paragraph
From 1.8 we can deduce that these results extend to the quotients of $\CC_q[G/U]$
corresponding to Demazure modules. Let $w \in W$ and put $\B_{\overline w}$ as the
complement of $\B_w$ in $\B$. Set
	$$ I_{A,w}:= \bigoplus_{\stackrel{\lambda \in P^+}{b \in \B_{\overline w} \cap
B_\lambda}} Ab^* \ot v_\lambda = \bigoplus_\lambda V_{A,w}(\lambda)^\perp \ot v_\lambda$$

Now $I_{A,w}$ is the orthogonal of the $A$-module generated by $\B_\omega$. 1.8 Theorem
implies that this is an ideal. The quotienta algebra can be decomposed as
$$ A_q[G/U]/I_{A,w} = \bigoplus_\lambda V_{A,w}(\lambda)^* \ot v_\lambda$$


\textbf{Bibliography}

[J] Jantzen, \emph{Lectures on Quantum Groups.}

[K] Kashiwara, \emph{On Crystal Basis}, Canadian Mathematical Society Conference
Proceedings, 27/3/95.

[L] Lusztig, \emph{Introduction to Quantum Groups}.

[Li] Littelman, \emph{Cones, Crystals and Patterns}. Transformation Groups, vol. 3, No. 2,
1998.

\end{document}
