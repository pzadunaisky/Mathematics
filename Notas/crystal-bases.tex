%%%%%%%%%%%%%%%%%%%%%% Generalities %%%%%%%%%%%%%%%%%%5
\documentclass[11pt,fleqn]{article}

\usepackage[paper=a4paper]
  {geometry}

\pagestyle{plain}
\pagenumbering{arabic}
%\linespread{1.2}
%\setlength{\parskip}{1.2ex}
% esto tiene que estar, en este orden...
\usepackage[small]{titlesec}
\usepackage{paragraphs}
\usepackage{hyperref}
\usepackage{amsthm,thmtools}

%\linespread{1.2}
\setlength{\parskip}{1.2ex}

\usepackage[utf8,latin1]{inputenc}
\usepackage[spanish,english]{babel}
\usepackage{enumerate}
\usepackage[alphabetic,initials]{amsrefs}
\usepackage{amsfonts,amssymb,amsmath}
\usepackage{dsfont}
\usepackage{mathtools}
\usepackage{graphicx}
\usepackage[poly,arrow,curve,matrix]{xy}
\usepackage{wrapfig}

%\swapnumbers


%%%%%%%%%%%%%%%%%%%%%%%%%%%%%%%%%% theorems et al. %%%%%%%%%%%%%%%%%%%%%%%%%%%
% numbered versions
\declaretheoremstyle[headformat=swapnumber, spaceabove=\paraskip,
bodyfont=\itshape]{mystyle}
\declaretheorem[name=Lemma, sibling=para, style=mystyle]{Lemma}
\declaretheorem[name=Proposition, sibling=para, style=mystyle]{Proposition}
\declaretheorem[name=Theorem, sibling=para, style=mystyle]{Theorem}
\declaretheorem[name=Corollary, sibling=para, style=mystyle]{Corollary}
\declaretheorem[name=Definition, sibling=para, style=mystyle]{Definition}

% unnumbered versions
\declaretheoremstyle[numbered=no, spaceabove=\paraskip,
bodyfont=\itshape]{mystyle-empty}
\declaretheorem[name=Lemma, style=mystyle-empty]{Lemma*}
\declaretheorem[name=Proposition, style=mystyle-empty]{Proposition*}
\declaretheorem[name=Theorem, style=mystyle-empty]{Theorem*}
\declaretheorem[name=Corollary, style=mystyle-empty]{Corollary*}
\declaretheorem[name=Definition, style=mystyle-empty]{Definition*}

% plain style
\declaretheoremstyle[
        headformat={{\bfseries\NUMBER.}{\itshape\NAME}\NOTE\ignorespaces},
        spaceabove=\paraskip, 
        headpunct={.},
        headfont=\itshape,
        bodyfont=\normalfont
        ]{mystyle-plain}
\declaretheorem[sibling=para, style=mystyle-plain]{Example}
\declaretheorem[sibling=para, style=mystyle-plain]{Remark}

% proofs, just as in amsthm but with adapted spacing
\makeatletter
\renewenvironment{proof}[1][\proofname]{\par
  \pushQED{\qed}%
  \normalfont \topsep.75\paraskip\relax
  \trivlist
  \item[\hskip\labelsep
        \itshape
    #1\@addpunct{.}]\ignorespaces
}{%
  \popQED\endtrivlist\@endpefalse
}
\makeatother



%%%%%%%%%%%%%%%%%%%%%%%%%%% The usual stuff%%%%%%%%%%%%%%%%%%%%%%%%%
\newcommand\NN{\mathbb N}
\newcommand\CC{\mathbb C}
\newcommand\QQ{\mathbb Q}
\newcommand\RR{\mathbb R}
\newcommand\ZZ{\mathbb Z}

\newcommand\maps{\longmapsto}
\newcommand\ot{\otimes}
\renewcommand\to{\longrightarrow}
\renewcommand\phi{\varphi}
\newcommand\stack[2]{\genfrac{}{}{0pt}{2}{#1}{#2}}
\newcommand\uu[1]{\underline{#1}}

%%%%%%%%%%%%%%%%%%%%%%%%% Specific notation %%%%%%%%%%%%%%%%%%%%%%%%%
\newcommand\A{\mathcal A}
\newcommand\B{\mathcal B}
\newcommand\C{\mathcal C}
\newcommand\D{\mathcal D}
\newcommand\R{\mathcal R}
\renewcommand\L{\mathcal L}
\newcommand\F{\mathcal F}
\renewcommand\S{\mathcal S}
\newcommand\I{\mathcal I}
\newcommand\kk{\mathds{k}}
\renewcommand\c{\textbf{c}}
\newcommand\G{\mathcal G}
\newcommand\GG{\Gamma}
\newcommand\OO{\Phi}
\newcommand\join{\vee}
\newcommand\meet{\wedge}
\renewcommand\a{\mathfrak{a}}
\renewcommand\b{\mathfrak{b}}
\newcommand\n{\mathfrak{n}}
\newcommand\g{\mathfrak{g}}
\newcommand\norm{\mathsf{norm}}
\newcommand\opp{\mathsf{opp}}

\DeclareMathOperator\Frac{Frac}
\DeclareMathOperator\Mod{\mathsf{Mod}}
\DeclareMathOperator\Vect{\mathsf{Vect}}
\DeclareMathOperator\Id{\mathsf{Id}}
\DeclareMathOperator\im{\mathsf{im}}
\renewcommand\ker{\mathsf{ker}}

\title{Crystal Bases of finite dimensional representations of $U_q(\g)$}
\date{7/8/2012}
\author{[crysal-bases.tex]}

\begin{document}
\maketitle
Notes based on [J], contents found in [K1]
\section{General definitions}
\paragraph \textbf{Notation}
We work over the field $\kk = \QQ(q)$, all unadorned tensor products will be over this
field. We denote by $\A^0$ the localization of $\QQ[q]$ at the maximal ideal of functions 
which have a root at zero. We identify $\QQ$ and $\A^0/q\A^0$


\paragraph  
Let $\g$ denote a finite dimensional complex semisimple Lie algebra.
We denote by $U_q = U_q(\g)$ the quantum envelopping algebra of $\g$ over $\QQ(q)$. Let
$P$ be the weight lattice of $\g$, $\Pi$ be a basis of the root system with $n =
|\Pi|$, $W$ its Weyl group and $(-,-)$ the $W$-invariant form over $P$. $U_q$ is the algebra 
with generators $E_i, F_i, K_i^{\pm 1}$ for $i = 1, \ldots, n$ and relations
\begin{align*}
	K_i K_j &= K_j K_i & K_i^{-1} K_i &= 1 \\
	K_i E_j K_i^{-1} &= q^{(\alpha_i, \alpha_j)} E_j
	& K_i F_j K_i^{-1} &= q^{-(\alpha_i, \alpha_j)} F_j \\
	[E_i, F_j] &= \delta_{ij} \frac{K_i - K_i^{-1}}{q_i-q_i^{-1}} 
	&&\mbox{Quantum Serre Relations} 
\end{align*}
where $q_i = q^{(\alpha_i,\alpha_i)/2}$. For $X_i = F_i, E_i \in U_q(\g)$, we define the 
divided powers as
$$ X_i^{(r)} = \frac{1}{[r]_{q_i}!} X_i^r $$

\paragraph
$U_q$ is a Hopf algebra with comultiplication $\Delta$, antipode $S$ and counit
$\varepsilon$ defined by
\begin{align*}
	\Delta(E_i) &= E_i \ot K_i^{-1} + E \ot 1
&	\Delta(F_i) &= F_i \ot 1 + K_i \ot F_i 
&	\Delta(K_i) &= K_i \ot K_i \\
	S(E_i) &= -E_iK_i & S(F_i) &= -K_i^{-1}F_i & S(K_i) &= K_i^{-1} \\
	\varepsilon(E_i) &= 0 & \varepsilon(F_i) &= 0 & \varepsilon(K_i) &= 1 \\
\end{align*}

\paragraph
We denote by $U_i$ the sub-Hopf-algebra of $U_q$ generated by $E_i, F_i, K_i^{\pm 1}$.


\section{Crystal Bases}
\paragraph
Let $V$ be a finite dimensional representation of $U_q$. Fix $i \in \{1, \ldots, n\}$.
Then by the representation theory of $U_i$, $V$ can be decomposed as 
$$ V = \bigoplus F_i^{(r)} (V_\lambda \cap \ker E_i)$$
where the sum runs over pairs $(\lambda,r) \in P \times \NN$ such that $(\lambda,
\alpha_i) \geq r (\alpha_i,\alpha_i) /2$. 

Let $v \in V_\lambda \cap \ker E_i$. We define $\tilde e_i, \tilde f_i$ as
\begin{align*}
	\tilde e_i F_i^{(r)}v &= F_i^{(r-1)}v & \tilde f_i F_i^{(r)}v &= F_i^{(r+1)}v
\end{align*}
and extend by linearity to $V$.

Notice that for any morphism of finite dimensional representations $\phi: V \to W$ and any
element $v \in V$, we have $\phi(\tilde x_i v) = \tilde x_i \phi(v)$, where $x \in
\{e,f\}$.

\begin{Definition}
	A \emph{crystal basis} of $V$ is a pair $(L,B)$ consisting of a free
$\A^0$-module $L$ and a $\QQ$-basis of $L / qL$, such that
	\begin{enumerate}[(a)]
		\item $L$ generates $V$ over $Q(q)$.
		\item $L = \bigoplus L_\lambda$ (where $L_\lambda = L \cap
			V_\lambda$) and $B = \bigsqcup B_{\lambda}$ (where
			$B_{\lambda} = B \cap L_\lambda / q L_\lambda$)
		\item $\tilde x_i(L) \subset L$ for all $i = 1, \ldots, n$, $x = e,
			f$, and furthermore, the induced morphisms on $L / q L$ send
			$b \in B$ to an element of $B \cup \{0\}$.
		\item For any $b,b' \in B$, we have that $\tilde e_i(b) = b'$ if and 
			only if $\tilde f_i(b') = b$
	\end{enumerate}
\end{Definition}

\begin{Lemma*}
\label{sum}
	Let $V_j$ be a $U_q$-module for $j = 1, \ldots, n$. Suppose that for each $j$ you
have a pair $(L_j,B_j)$, where $L_j$ is an $\A^0$-submodule and $B_j$ is a basis of
$L_j/qL_j$. Then $\bigoplus (L_j,B_j) := (\bigoplus L_j, \bigsqcup B_j)$ is a
crystal basis of $\bigoplus_j V_j$ if and only if each $(L_j,B_j)$ is a crystal basis of
$V_j$.
\end{Lemma*}


\begin{Example}
\label{sl2}
	Suppose $\g = \mathfrak{sl}_2(\CC)$ and $V = V_q(m)$. Then there is a basis
$\B = \{v_0, \ldots, v_m\}$ of $V$ such that
	\begin{align*}
		K^{\pm 1} v_r &= q^{\pm(m - 2r)}v_r & 
		F v_r &= [r+1]_q v_{r+1} &
		E v_r &= [m-r+1]_q v_{r-1}.
	\end{align*}
	In this case, $\tilde f v_r = v_{r+1}$, $\tilde e v_r = v_{r-1}$. Set $L =
\sum \A^0 v_i$, and $B$ equal to the image of $\B$ in the quotient $L / q L$. This is
a crystal basis, so by \ref{sum}, any $U_q(\mathfrak{sl}_2)$-module has one.
\end{Example}

\paragraph 
Take $\lambda \in P^+$, and set 
\begin{align*}
	\L(\lambda) 
		&= \sum \A \tilde f_{i_1} \tilde f_{i_2} \ldots \tilde f_{i_n} v_\lambda 
			\subset V_\lambda \\
	\B(\lambda) 
		&= \{\tilde f_{i_1} \tilde f_{i_2} \ldots \tilde f_{i_n} v_\lambda \neq 0 
			\mod q\L(\lambda) \} \subset \L(\lambda)/q \L(\lambda)
\end{align*}
Let $C(\lambda)$ be the following statement:
\begin{align*}
	C(\lambda): (\L(\lambda),\B(\lambda)) \mbox{ is a crystal basis of } V_\lambda.
\end{align*}

We find the following three conjectures in [K1]

\textbf{Conjecture 1:} Any integral $U_q$-module has a crystal basis.

\textbf{Conjecture 2:} For any $\lambda \in P^+$, $C(\lambda)$ holds.

\textbf{Conjecture 3:} For any crystal basis $(L,B)$ of an integrable $U_q$-module we have
\begin{align*}
	C(L,B):& 
		\mbox{ for any $u \in B$ such that $\tilde e_i u = 0$ for all $i$, there} \\
		& \mbox{exists $u' \in L$ such that $u = u' \mod qL$ and $e_i u' = 0$ for 
			all $i$}
\end{align*}
Kashiwara's Theorem in [K1] states that Conjectures 1, 2, 3 are true in the $A, B, C$ and
$D$ cases. This was later extended to the exceptional cases.

\begin{Lemma}
\label{C(L,B)}
	$C(\lambda)$ implies $C(\L(\lambda), \B(\lambda))$
\end{Lemma}
\begin{proof}
	Let $u \in \B(\lambda)_\mu = \B(\lambda) \cap V_q(\lambda)_\mu$. If $\lambda \neq
\mu$, $u = \tilde f_{i_1} \tilde f_I v_\lambda$, su $\tilde e_{i_1}u = \tilde f_I
v_\lambda \neq 0$. Therefore, the only way that $\tilde e_i u = 0$ for all $i$ is that $u 
= \overline v_\lambda$, in which case $u' = v_\lambda$ works.
\end{proof}


\section{Polarization}
$V, V_i$, etc. are always integrable representtions of $U_q$.

\paragraph
There is an antiautomorphism $\tau: U_q \to U_q$, defined by setting
\begin{align*}
	\tau(E_i) &= q_i F_i K_i^{-1}
		& \tau(F_i) &= q_i^{-1} K_i E_i 
		& \tau(K_i) = K_i
\end{align*}

\begin{Definition}
\label{polarization}
Let $(L,B)$ be a crystal basis for $V$. A \emph{polarization} of $(L,B)$ is a $\kk$
valued bilinear form on satisfying
\begin{align}
	(x \cdot v,v') &= (v, \tau(x) \cdot v') & \mbox{for } v,v' \in V \  x \in U_q\\
	(L,L) & \subset \A^0 \\
	[u,v] &= \delta_{u,v} &\mbox{for } u,v \in B
\end{align}
where $[-,-]$ is the $\QQ$-bilinear form induced by $(-,-)$ on $L/qL$.
\end{Definition}

\begin{Lemma}
\label{polarization-details}
	Suppose $(L,B)$ is a crystal basis for $V$, and that $(-,-)$ is a polarization.
Write $V = \bigoplus_{\lambda \in P} V_\lambda$.
	\begin{itemize}
		\item Let $v_? \in V_?$. Then $(v_\lambda, v_\mu) \neq 0$ implies $\lambda
			= \mu$.
		\item Suppose, $V_q(\lambda) \oplus V_q(\mu) \subset V$. If $\lambda \neq \mu$
			then $V_q(\lambda) \perp V_q(\mu)$.
		\item We have for all $v,v' \in V$ $[\tilde e_i v, v'] = [v, \tilde f_i v']$.
		\item $L = \{v \in V| (L,v) \subset \A^0 \}$
	\end{itemize}
\end{Lemma}
\begin{proof}
	For the first item, we have $q^{(\rho,\lambda)} (v_\lambda, v_\mu) = (K_\rho
v_\lambda, v_\mu) = (v_\lambda, K_\rho v_\mu) = q^{(\rho, \mu)} (v_\lambda,v_\mu)$ for all
$\rho \in P$. Since $q$ is not a root of unity, this implies that $\lambda = \mu$.

	By the first item, $(v_\lambda,v_\mu) = 0$. Now $V_\lambda = U_q(\n^-) v_\lambda$, 
and $\tau(U_q(\n^-)) \subset U_q(\b^+)$, which immediately implies that $(v_\lambda,-) = 0$. 
Now for any two $b,b' \in U_q(\n^-)$, $(bv_\lambda, b'v_\mu) = (v_\lambda, \tau(b)b'v_\mu)
= 0$.

	For the third item, fix $i$ and write $E = E_i$, etc. The elements $v, v'$ can be 
decomposed as $v = \sum_{n,j} F^{(n)} x_j$, $v' =\sum_{n,j} F^{(n)} x'_j$, with $E x_j = 0
= E x'_j$, thus we only have to analyze the case $v = F^{(n)}x_j, v' = F^{(m)}x'_j$. Now
$x_j, x'_j$ generate each one a simple $U_q(\mathfrak{sl_2})$-module, which is stable by 
$\tilde e$, and by the previous item we may and do assume that this two modules are 
isomorphic. Hence the weight of $x_j$ and $x'_j$ are equal. Finally, $\tilde e F^{(n)}x_j$ 
is a weight vector and so is $F^{(m)}x'_j$, so they must have the same weight, otherwise 
both sides of the previous equality are zero by the first item. This implies that $m = n-1$. 

Let $r$ be the weight of $x_j$. By the equality $ EF^{(n)} x_j = [r+1-n] F^{(n-1)} x_j$, we 
have
\begin{align*}
	(\tilde e v,v')	&= (F^{(n-1)}x_j, v') = [r+1-n]^{-1}(EF^{(n)}x_j,v') \\
			&= [r+1-n]^{-1}(v,\tau(E)F^{(n-1)}x'_j) =
[r+1-n]^{-1}(v,qFK^{-1}F^{(n-1)}x'_j) \\
			&= \frac{[n]q^{2n-r-1}}{[r+1-n]}(v,\tilde f v')
= \frac{(q^n - q^{-n})q^{n}}{(q^{r+1-n}-q^{-r-1+n})q^{-n+r+1}}(v,\tilde f v') \\
			&= \frac{1-q^{2n}}{1-q^{2r-2n+2}}(v,\tilde f v')
\end{align*}
and this implies the third item.

In the last item, one inclusion is obvious by \ref{polarization} (2). On the other hand,
assume $(L,u) \subset A$. By definition, there is a minimal $n \in \NN_0$ such that $q^n u
\in L$. If $n > 0$ then $(L,q^n u) \subset q\A^0$, which implies that $[-,q^n u] = 0$. But
by item 3., $[-,-]$ is non degenerate, and hence $q^n u \in q L$, that is, $q^{n-1} u \in
L$, which contradicts the minimality of $n$. Hence $n = 0$ and $u \in L$
\end{proof}

\begin{Lemma}
	Let $\lambda \in P^+$ and assume $C(\lambda)$. Then $(\L(\lambda), \B(\lambda))$
is polarizable.
\end{Lemma}
\begin{proof}
	By a reasoning analogous to the proof of the second item in the previous
proposition, there is at most one $\tau$-symmetric bilinear form over $V_q(\lambda)$ such
that $(v_\lambda,v_\lambda) = 1$. I haven't checked that it is well defined...

	For (2), it is clear that $(\L(\lambda)_\lambda,\L(\lambda)_\lambda) \subset \A^0$.
Now if $\mu \neq \lambda$, then
$$ (\tilde f_i \L(\lambda)_{\mu + \alpha_i}, \L(\lambda)_\mu) \subset (1+q^*)(\L(\lambda)_
{\mu + \alpha_i},\tilde e_i \L(\lambda)_\mu) \subset (1+q^*)(\L(\lambda)_{\mu + \alpha_i},
\L(\lambda)_{\mu + \alpha_i}) $$
(notice that in the proof of the third item of \ref{polarization-details}, we only used
the $\tau$-symetry of the form).

We may assume by inductive hypothesis that this last set is included in $(1+q^*) \A^0 \subset 
\A^0$. Since $\L(\lambda)_\mu = \sum_i \tilde f_i \L(\lambda)_{\mu + \alpha_i}$, we are done.

	Finally for (3), once again we have that $[v_\lambda, v_\lambda] = 1$. Now if $b
\in \B(\lambda)_\mu$ with $\mu \neq \lambda$, there is an $i$ such that $\tilde e_i b \neq
0$, so $b = \tilde f_i \tilde e_i b$. Hence $[b,b'] = [\tilde e_i b, \tilde e_i b']$. By
inductive hypothesis, this will be nonzero if and only if $\tilde e_i b = \tilde e_i b'$,
which is only possible if $b = b'$. This completes the proof.
\end{proof}

\begin{Proposition}
\label{decomposition}
	Let $\lambda \in P^+$ and assume $C(\lambda)$. Let $V$ be isomorphic to
$V_q(\lambda)^{\oplus n}$ and $W$ a representation such that $W_\lambda = 0$. Suppose
there is a crystal base $(L,B)$ of $V \oplus W$, and write $L^V = L \cap V$, $L^W = L \cap
W$, $B^V = B \cap V$ and $B^W = B \cap W$. Then
	\begin{enumerate}[(i)]
		\item There is an isomorphism $V \cong V_q(\lambda)^{\oplus n}$ such that
			$(L^V,B^V) \cong (\L(\lambda),\B(\lambda))^{\oplus n}$.
		\item $L = L^V \oplus L^W$ and $B = B^V \sqcup B^W$.
		\item $(L^V, B^V)$ is a crystal basis for $V$, idem $W$.
	\end{enumerate}
\end{Proposition}
\begin{proof}
	Since $W_\lambda = 0$, $B_\lambda = B^{V}_\lambda$. Thus there is a $\kk$-linear
isomorphism $V_\lambda \cong V_q(\lambda)^{\oplus n}_\lambda$ such that $B^\lambda$ is sent 
to $\B(\lambda)^{\oplus n}_\lambda$. By extending this morphism $U_q$-linearly, we have
the desired isomorphism, say $\phi$. Denote by $(L',B')$ the preimage of $(\L(\lambda),
\B(\lambda))$ in $V$ by $\phi$. We clearly have $B'_\lambda = B_\lambda$
and $L_\lambda = L'_\lambda$. 

By the preceeding lemma, there is a polarization $(-,-)$ over $(L',B')$, and
it is clear that $(L'_\lambda,L^{V}_\lambda) \subset \A^0$. Now for any $i$, we have
$$ (\tilde f_i L'_{\mu + \alpha_i}, L^{V}_\mu) = (L'_{\mu+\alpha_i}, \tilde e_i
L^V_\mu) \subset (L'_{\mu + \alpha_i}, L^V_{\mu + \alpha_i}) $$
and by inverse induction on the weights, this is contained in $\A^0$. Hence by 
\ref{polarization-details} point 4., $L^V \subset L'$, and so $L^V / q L^V \subset L' / q L'$; 
this implies that the cardinality of $B^{V}_\mu$ is less than or equal to that of $B'_\mu$ 
(this is a finite number since each $L_\mu$ is finite dimensional) On the other hand 
$B'_\lambda = B^{V}_\lambda$, and  since crystal bases are stable by $\tilde f_i$, we have $B' 
\subset B \cap L^V$, which implies $B' = B^V$. Hence $L^V + q L^V =L'$, so by Nakayama's 
lemma, we have equality. This is point $(i)$.

It is clear that $L^V \cap L^W = \emptyset$ so we only have to prove that $L^V + L^W = L$. 
If $l \in L$, then it can be written as $l = v + w$, with $v \in V, w \in W$. Since $L$
generates $V\oplus W$ over $\QQ(q)$, there is a number $n \in \NN$ such that $q^nv, q^nw
\in L$. Thus if we prove that $qL \cap (L^v + L^W) = qL^V + qL^W$, the minimal such $n$
must be $0$, and hence $v,w \in L$.

Suppose $v+w \in qL$. Then $v \equiv -w \mod qL$. Now since $(L^V,B^V)$ is a crystal
basis of $V$ isomorphic to $(\L(\lambda),\B(\lambda))$, there is a unique expression 
$v \equiv \sum_{I \in \mathcal I} \tilde f_I v_I \mod qL^V$ where $v_I$ is 
of weight $\lambda$, $I = (i_1, \ldots, i_r)$ and $\mathcal I$ minimal. If we denote 
$\tilde e_I = \tilde e_{i_r} \ldots \tilde e_{i_1}$, we have $\tilde e_I \tilde f_I v_I
 = v_I$. Furthermore, we have that if $I \neq J, J \in \mathcal I$, then $\tilde e_J 
\tilde f_I v_I = 0$. Otherwise, $v_I = \tilde f_J \tilde e_J \tilde f_I v_I$, and this
contradicts the minimality of $\mathcal I$. Hence we have 
$$ \sum_{J \in \mathcal I} \tilde f_J \tilde e_J v \equiv \sum_{J, I \mathcal I} \tilde
f_J \tilde e_J \tilde f_I v_I \equiv \sum_{i \in \mathcal I} \tilde f_I v_I \equiv v \mod
qL^V $$
However, this equivalence implies equivalence $\mod qL$, and since $-v \equiv w \mod qL$,
we also have $w \equiv \sum_{I \in \mathcal I} \tilde f_I \tilde e_I w \mod qL$. Since $v$ 
and $w$ are in the same weight component, $\tilde e_I w \in L/qL^\lambda \cap W/qW = 0$. 
Hence $v \equiv w \equiv 0 \mod qL$.

Finally, we prove that $B^\mu = B^{V}_\mu + B^{W}_\mu$. If $\mu$ is not a weight of
$V_q(\lambda)$, then the result is clear, and it is also clear for $\lambda = \mu$. By
induction on the weights, we may assume the result is true for $\mu + \alpha_i$ for all
$i$. Take $u \in B_\mu$, and suppose $\tilde e_i u \neq 0$. Then $u \tilde f_i \tilde e_i
u \in \tilde f(B^V_{\mu+\alpha_i} \cup B^W_{\mu+\alpha_i}) \subset B^V_{\mu} \cup
B^{W}_{\mu}$. Otherwise, $\tilde e_i u = 0$ for all $i$, and if we write $u = v + w$, this
implies $\tilde e_i v = 0$ for all $i$. By lemma \ref{C(L,B)}, this implies that $v = 0$, and
hence $u \in B_\mu$. This completes the proof of $(ii)$, and $(iii)$ follows immediately.
\end{proof}

\section{Tensor products of Crystal Bases}
\paragraph
Let $(L,B)$ be a crystal basis of $V$. For $u \in B$, we define $\varepsilon_i(u) =
\max\{n|\tilde e_i^n(u) \neq 0\}$, $\varphi_i(u) = \max \{n| \tilde f_i^n(u) \neq 0\}$


\begin{Proposition}
\label{tensor-product}
Let $(L_j,B_j)$ be crystal bases of $V_j$, and assume $C(\lambda)$ for every $\lambda$
such that $(V_j)_\lambda \neq 0$.
	\begin{enumerate}[a)]
		\item Then $(L_1 \ot_{\A^0} L_2, B_1 \ot_\QQ B_2)$ is a crystal base of
			$V_1 \ot V_2$.
		\item For $u\in B_1, v \in B_2$ we have
			\begin{align*}
				\tilde f_i (u \ot v) &= 
					\begin{cases}
						\tilde f_i u \ot v &\mbox{if } \varphi_i(u)
									> \varepsilon_i(v)\\
						u \ot \tilde f_i v &\mbox{otherwise.}
					\end{cases} \\
				\tilde e_i (u \ot v) &=
					\begin{cases}
						u \ot \tilde e_i v &\mbox{if } \varepsilon(v)
									> \varphi(u) \\
						\tilde e_i u \ot v & \mbox{otherwise.}
					\end{cases}
			\end{align*}
		\item If $(-,-)_j$ is a polarization for $(L_j,B_j)$, $(u_1 \ot u_2, v_1
			\ot _2) = (u_1,v_1)_1(u_2,v_2)_2$ is a polarization of $V_1 \ot
			V_2$ with the basis we have described.
	\end{enumerate}	
\end{Proposition}
\begin{proof}
The last item follows immediately from the previous two, so we only prove these. The
hypothesis implies that we may apply repeatedly Proposition \ref{decomposition}, and so
$V_j = \bigoplus_k V_q(\lambda_{j,k})$, $\L_j = \bigoplus_i \L_{j,k}$ $B = \bigsqcup
B_{j,k}$. Hence $L_1 \ot L_2 = \bigoplus_{k,k'} L_{1,k} \ot L_{2,k'}$, $B_1 \times B_2 =
\bigsqcup B_{1,k} \times B_{2,k'}$, and it is enough to check that for fixed $k, k'$ each
component is a crystal basis. So, we reduce to the case where $V_j = V_q(\lambda_j)$.

It is obvious that $L_1 \ot_{\A^0} L_2$ is an $\A^0$-module generating $V_1 \ot V_2$ over
$\QQ(q)$. Also, since $(V_1 \ot V_2)_{\lambda} = \bigoplus_{\lambda_1 + \lambda_2 =
\lambda} V_{1,\lambda_1} \ot V_{2,\lambda_2}$, we have $(L_1 \ot L_2)_\lambda =
\bigoplus_{\lambda_ 1 + \lambda_2 = \lambda} L_{1,\lambda_1} \ot L_{2, \lambda_2}$, so we
have points (1) and (2) of the definition.

Now we want to study the action of the $\tilde f_i, \tilde e_i$. If we fix an $i$, we may
consider the algebra $U_{q_i}(\mathfrak{sl}_2) \subset U_q$ generated by $E_i, F_i,
K_i^{\pm 1}$, and regard the $V_j$ as modules over this algebra. By the discussion above,
we may reduce the analysis of the tensor product to the case when each $V_j$ is an
irreducible module over $U_{q_i}(\mathfrak{sl}_2)$. Hence from this point on we assume 
$V_1 = V_q(m), V_2 = V_q(l)$, and prove the result by induction on $m$.

We point out that from [J, 9.13 (8)] we have 
$$ \Delta(F^{(r)}) = \sum_{j=0}^r q^{-j(r-j)}F^{(r-j)}K^{j} \ot F^{(j)}$$

Let $u \in V_q(1)$, $v \in V_q(l)$ be highest weight vectors. By \ref{decomposition} (1) 
we know that we may assume the bases to be $(\L(1),\B(1))$ and $(\L(m), \B(m))$, which we
constructed explicitly for in Example \ref{sl2}. Now $V_q(1) \ot V_q(l) =
V_q(l+1) \oplus V_q(l-1)$, with highest weight vectors $x=u \ot v$ and $y=u \ot Fv - q^m [m]
Fu \ot v$. And once again we know how to contruct a crystal basis for highest weight modules 
from Example \ref{sl2}. Simple calculations give us for $r > 0$
	\begin{align*}
		F^{(r)}(x) &= q^r u \ot F^{(r)}v + Fu \ot F^{(r-1)}v \\
			&\equiv Fu \ot F^{(r-1)}v \mod q \L \\ 
		F^{(r)}(y) &= q^r[r+1]u \ot F^{(r+1)}v + q^m[m-r]Fu \ot F^{(r)}v \\ 
			&\equiv u \ot F^{(r+1)}v \mod q \L
	\end{align*}
Hence the $\A^0$-module $L'$ generated by $F^{(r)}(x), F^{(r)}(y)$ is contained in the
$\A^0$-module generated by $F^{(r)}(u) \ot v, u \ot F^{(r)}(v)$, which is none other than
$L_1 \ot L_2$. Plus, the basis $B'$ of $L'/qL'$ formed by the images of the generators of $L'$
coincides with $B_1 \ot B_2$, which implies that in fact both lattices coincide, both
bases coincide and from the formulas above we get the actions of $\tilde f$ and $\tilde e$
on the elements of the $\QQ$-basis. This completes the case $m=1$.

Now for the inductive step. Suppose we know the result is true for all weights up to
$m-1$. Set $V = V_q(1), V' = V_q(m-1), V'' = V_q(l)$, and their corresponding bases
$(L,B), (L',B'), (L'',B'')$. By inductive hypothesis, $(L \ot L' \ot L'', B \ot B' \ot
B'')$ is a crystal basis of $V \ot V' \ot V''$. However we know that $(L \ot L', B \ot B')
= (L_{m-2},B_{m-2})\oplus(L_{m},B_m)$, so $(L_{m-2},B_{m-2}) \ot (L_l,B_l) \oplus
(L_m,B_m)\ot(L_l,B_l)$ is a crystal basis of $V_q(m-2) \ot V_q(l) \oplus V_q(m) \ot
V_q(l)$. Hence each direct summand is a crystal basis of its corresponding direct summand
by \ref{sum}.

Let us now consider the formulas for the actions of $\tilde f, \tilde e$. It is easy to
see from the formula in $2.$ that 
\begin{align*}
	\phi(u \ot v) &= \max\{\phi(u)-\varepsilon(v)+\phi(v),\phi(v)\}\\
	\varepsilon(u \ot v) &=\max\{\varepsilon(v)-\phi(u)+\varepsilon(u), \varepsilon(u)\}
\end{align*}
Given a vector $v \ot v' \ot v'' \in B \ot_\QQ B' \ot_\QQ B''$, we can use the inductive
hypothesis in the cases $1, m-1$ to find the action of $\tilde f$ on this element. This
turns out to be
\begin{align*}
	\tilde f (v \ot v' \ot v'') &=
		\begin{cases}
			\tilde f v \ot v' \ot v'' & \mbox{if } \phi(v) >
				\varepsilon(v'\ot v'') \ (1)\\
			v \ot \tilde f(v' \ot v'') &=
			\begin{cases}
			v \ot \tilde f v' \ot v'' & \mbox{if not and } \phi(v') > 
				\varepsilon(v'') \ (2)\\
			v \ot v' \ot \tilde f v'' & \mbox{if not and } \phi(v') \leq 
				\varepsilon(v'') \ (3)
			\end{cases}
		\end{cases}
\end{align*}
Condition (1) is equivalent to $(1)': \phi(v) + \phi(v') - \varepsilon(v')> 
\varepsilon(v''), \ \phi(v) > \varepsilon(v')$. The second inequality implies that the
first one is $\phi(v \ot v') > \varepsilon(v'')$. Condition (2) is equivalent to 
$\varepsilon(v') \geq \phi(v), \  \phi(v') > \varepsilon(v'')$, which is in turn equivalent
to $(2)': \varepsilon(v') \geq \phi(v), \  \phi(v \ot v') > \varepsilon(v'')$. Finally,
condition (3) is equivalent to $\varepsilon(v'') \geq \phi(v'), \ \varepsilon(v'')
\geq \phi(v) - \varepsilon(v') + \phi(v')$, which is $(3)' \varepsilon(v'') \geq \phi(v
\ot v')$. Thus finally we have that
\begin{align*}
	\tilde f(v \ot v' \ot v'') &=
		\begin{cases}
			\tilde f(v \ot v') \ot v'' &=
				 \begin{cases}
					\tilde f v \ot v' \ot v'' &\mbox{in case }(1)'\\
					v \ot \tilde f v' \ot v'' &\mbox{in case }(2)'
				 \end{cases}\\
			v \ot v' \ot \tilde f(v'') &\mbox{in case }(3)'
		\end{cases}
\end{align*}



\end{proof}

\paragraph
The \emph{crystal graph} of a representation $V$ with crystal base $(L,B)$ is a colored 
graph whose vertices are the elements of $B$, and an arrow of color $i$ goes from $b$ to
$b'$ if and only if $b' = \tilde f_i b$. The previous Proposition shows how to construct the crystal graph of the tensor
products of two representations $V_1, V_2$ with respective crystal bases $(L_1, B_1),
(L_2, B_2)$. This graph, call it $G$, can be constructed as follows

\begin{enumerate}
	\item The vertices of $G$ are the elements of $B_1 \times B_2$.
	\item For each $(b_1, b_2) \in B_1 \times B_2$ such that $\phi_i(b_1) >
		\varepsilon_i(b_2)$, add an arrow of color $i$ from $(b_1, b_2)$ to 
		$\tilde (f_i b_1, b_2)$.
	\item For each $(b_1,b_2)$ such that $\phi_i(b_1) \leq \varepsilon_i(b_2)$ and
		$\phi_i(b_2) > 0$, add an arrow of color $i$ from $(b_1,b_2)$ to $(b_1, 
		\tilde f_i b_2)$. 
\end{enumerate}

\paragraph
 As an example, the standard representation of
$U_q(\sl_3)$ has a crystal graph given by $ v_1 \stackrel{1}{\to} v_2 \stackrel{2}{\to} 
v_3$. The crystal graph of the tensor product is given by

\begin{align*}
\xymatrix{
	{} & v_1 \ar[r]^-1 & v_2 \ar[r]^2 & v_3 \\
	u_1 \ar[d]_-1 & u_1 \ot v_1 \ar[d]_-1 & u_1 \ot v_2 \ar[r]^-2 & u_1 \ot v_3
\ar[d]_-1\\
	u_2 \ar[d]_-2  & u_2 \ot v_1 \ar[r]^-1 \ar[d]_-2 & u_2 \ot v_2 \ar[d]_-2 & u_2 \ot v_3 \\
	u_3  & u_3 \ot v_1 \ar[r]^-1 & u_3 \ot v_2 \ar[r]^-2 & u_3 \ot v_3 \\
}
\end{align*}
Conjecture $C(\lambda)$ implies that the crystal graph of each indecomposable module is
connected if we forget colors. Hence, from this we get the decomposition of $\CC^3 \ot_\CC
\CC^3$ as a $U_q(\sl_3(\CC))$ module, as a sum of a representation of dimension $6$ and
one of dimension $3$.


\section{Existence of crystal bases for $V_q(\lambda)$}
\begin{Lemma}
	Let $\lambda, \mu \in P^+$, and suppose $C(\lambda)$ and $C(\mu)$ hold. Suppose
furthremore that $\dim_{\QQ(q)} V_q(\lambda)_{\sigma} = 1$ for all weights $\sigma$ of
$V_q(\lambda)$. Finally, suppose $V_q(\lambda) \ot V_q(\mu) = \sum_{\sigma \in S} V_q(\mu
+ \sigma)$, where $S = \{\sigma| u \in \B(\lambda)_\sigma \Rightarrow \tilde e_i^{1+ \langle
h_i,\mu\rangle}u = 0 \mbox{ for all } i\}$. Then $C(\mu + \sigma)$ is true for all
$\sigma \in S$
\end{Lemma}
\begin{proof}
	From the hypothesis and Proposition \ref{tensor-product} we get that $V=V_q(\mu)
\ot V_q(\lambda)$ has a crystal basis given by $(L,B) = (\L(\mu) \ot_{\A^0} \L(\lambda),
\B(\mu) \ot_{\QQ} \B(\lambda))$. Furthermore, this tensor product has a polarization. Let $V =
\bigoplus V_j$ be a decomposition in irreducible modules, and put $L_j = L \cap V_j$, $B_j
= B \cap (L_j/qL_j)$. We will prove that $L = \bigoplus L_j, B = \bigsqcup B_j$, and this
implies that $(L_j,B_j)$ is a crystal basis for each $M_j$.

Once again, it is enough to prove that $q L \cap \bigoplus L_j \subset \bigoplus q L_j$. By 
hypothesis each $V_j$ has a diferent highest weight, and so they are orthogonal for the
polarization by \ref{polarization-details} (2). If $l \in qL \cap \bigoplus L_j$, then $l
= \sum l_j$, so $0 = [\overline l,\overline l] = \sum_j [\overline l_j, \overline l_j]$,
meaning $[\overline l_j,\overline l_j] = 0$. Since the bilinear form is non-degenerate,
this implies $ \overline l_j = 0$, that is, $l_j \in q L_j$. This proves the first
equality.

Let $u \in B$; then $u = x \ot_{\QQ} y \in \B(\mu) \ot \B(\lambda)$. Suppose first that
$\tilde e_i u = 0$ for all $i$. If $\varepsilon_i(y) > \phi_i(x) \geq 0$, then $\tilde e_i(
x \ot y) = x \ot \tilde e_i y \neq 0$, wo we must have $\tilde e_i x = 0$, and
$\varepsilon_i(y) \leq \phi_i(x) \leq \langle h_i,\lambda \rangle$, so $\tilde
e_i^{\langle h_i,\lambda \rangle + 1} y = 0$. The first equality implies that $x = v_\mu$
the highest weight vector, while, the other one implies the weight of $y$, say $\sigma$, 
is in $S$. In particular $u \in L(\mu)/qL(\mu)_\mu \ot \L(\lambda)/q\L(\lambda)_{\sigma - \mu}$ 
which is of dimension $1$. Write $u$ as a sum of elements of $L_j/qL_j$ of weight $\sigma$. 
Since $L_j$ is stable by $\tilde e_i$, each element of this sum is in the kernel of
$\tilde e_i$ which as we have proven has component $\sigma$ of dimension $1$. Hence $u \in
B_j$ for some $j$. Now $B$ is generated the set of elements of the form $\tilde f_I u$
with $\tilde e_i u = 0$ for all $i$, and since $B_j$ is stable by $\tilde f_i$, $B \subset
\bigsqcup B_j$. Since both are basis for the same spaces, they hav the same cardinal and
we have an equality.

Clearly each $B_j$ cotains at least one element annihilated by all $\tilde e_i$. Since $j
= 1, \ldots, |S|$, we see that there is exactly one in each. By the previous discussion,
each $B_j = \B(\lambda')$, where $\lambda'$ is the weight of said element.
\end{proof}

\paragraph
\label{existence}
Kashiwara goes on to prove with a technical lemma that the standard and spin
representations have propety $C$ \emph{and} comply with the hypothesis of the previous lemma. 
Hence, we have $C(\lambda)$ for all $V_q(\lambda)$ that appear in the irreducible
decomposition of arbitrary tensor products of said represntations. This proves
$C(\lambda)$ for all weights in the cases $A, B, C$ and $D$ for all weights in the cases
$A, B, C$ and $D$.

\section{Bibliography}
\noindent [J] Jantzen, Lectures on Quantum Groups.

\noindent [K1] Kashiwara, Crystalizing the $q$-analogue of universal eneloping algebras. 

\section{Notes}
Some things to do or re do:

- The proof that $C(\L(\lambda),\B(\lambda))$ is polarizable is missing.

- The last part of the inductive step in \ref{tensor-product} should be written more
  clearly.

- Kashiwara's argument as mentioned in \ref{existence} is a technical matter using the
  Weyl character function. I have not yet really undertood what he does.

\end{document}










