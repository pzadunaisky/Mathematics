%%%%%%%%%%%%%%%%%%%%%% Generalities %%%%%%%%%%%%%%%%%%5
\documentclass[11pt,fleqn]{article}
\usepackage[paper=a4paper]
  {geometry}

\pagestyle{plain}
\pagenumbering{arabic}
%%%%%%%%%%%%%%%%%%%%%%%%%%%%%%%%
% esto tiene que estar, en este orden...
\usepackage[small]{titlesec}
\usepackage{paragraphs}
\usepackage{hyperref}
\usepackage{amsthm,thmtools}

%\linespread{1.2}
\setlength{\parskip}{1.2ex}

\usepackage[utf8,latin1]{inputenc}
\usepackage[spanish,english]{babel}
\usepackage{enumerate}
\usepackage{mathpazo}
\usepackage{euler}
\usepackage[alphabetic,initials]{amsrefs}
\usepackage{amsfonts,amssymb,amsmath}
\usepackage{dsfont}
\usepackage{mathtools}
\usepackage{graphicx}
\usepackage[poly,arrow,curve,matrix]{xy}
\usepackage{wrapfig}
%\swapnumbers

% numbered versions
\declaretheoremstyle[headformat=swapnumber, spaceabove=\paraskip,
bodyfont=\itshape]{mystyle}
\declaretheorem[name=Lemma, sibling=para, style=mystyle]{Lemma}
\declaretheorem[name=Proposition, sibling=para, style=mystyle]{Proposition}
\declaretheorem[name=Theorem, sibling=para, style=mystyle]{Theorem}
\declaretheorem[name=Corolllary, sibling=para, style=mystyle]{Corollary}
\declaretheorem[name=Definition, sibling=para, style=mystyle]{Definition}
%\declaretheorem[name=Example, sibling=para, style=mystyle]{Example}


% unnumbered versions
\declaretheoremstyle[numbered=no, spaceabove=\paraskip,
bodyfont=\itshape]{mystyle-empty}
\declaretheorem[name=Lemma, style=mystyle-empty]{Lemma*}
\declaretheorem[name=Proposition, style=mystyle-empty]{Proposition*}
\declaretheorem[name=Theorem, style=mystyle-empty]{Theorem*}
\declaretheorem[name=Corollary, style=mystyle-empty]{Corollary*}
\declaretheorem[name=Definition, style=mystyle-empty]{Definition*}
%\declaretheorem[name=Example, style=mystyle-empty]{Example*}

% plain style
\declaretheoremstyle[
        headformat={{\bfseries\NUMBER.}{\itshape\NAME}\NOTE\ignorespaces},
        spaceabove=\paraskip, 
        headpunct={.},
        headfont=\itshape,
        bodyfont=\normalfont
        ]{mystyle-plain}
\declaretheorem[sibling=para, style=mystyle-plain]{Example}
\declaretheorem[sibling=para, style=mystyle-plain]{Remark}

% proofs, just as in amsthm but with adapted spacing
\makeatletter
\renewenvironment{proof}[1][\proofname]{\par
  \pushQED{\qed}%
  \normalfont \topsep.75\paraskip\relax
  \trivlist
  \item[\hskip\labelsep
        \itshape
    #1\@addpunct{.}]\ignorespaces
}{%
  \popQED\endtrivlist\@endpefalse
}
\makeatother

%%%%%%%%%%%%%%%%%%%%%%%%%%% The usual stuff%%%%%%%%%%%%%%%%%%%%%%%%%
\newcommand\NN{\mathbb N}
\newcommand\CC{\mathbb C}
\newcommand\QQ{\mathbb Q}
\newcommand\RR{\mathbb R}
\newcommand\ZZ{\mathbb Z}
\newcommand\KK{\mathbb K}

\newcommand\maps{\longmapsto}
\newcommand\ot{\otimes}
\renewcommand\to{\longrightarrow}
\renewcommand\phi{\varphi}
\newcommand\eps{\varepsilon}
\newcommand\uu[1]{\underline{#1}}

%%%%%%%%%%%%%%%%%%%%%%%%% Specific notation %%%%%%%%%%%%%%%%%%%%%%%%%

\newcommand\A{\mathcal A}
\newcommand\B{\mathcal B}
\renewcommand\L{\mathcal L}
\newcommand\R{\mathcal R}
\newcommand\F{\mathcal F}
\renewcommand\O{\mathcal O}
\newcommand\I{\uu I}
\renewcommand\b{\mathfrak{b}}
\newcommand\g{\mathfrak{g}}
\newcommand\h{\mathfrak h}
\newcommand\sq{\square}
\newcommand\vep{\varepsilon}
\newcommand\co{\mathsf{co}}
\DeclareMathOperator\Id{\mathsf{Id}}
\DeclareMathOperator\st{\mathsf{st}}

\newcommand\C{\mathsf C}
\renewcommand\k{\Bbbk}
\newcommand\D{\mathsf D}

\DeclareMathOperator\Mod{\mathsf{Mod}}
\DeclareMathOperator\Hom{\mathsf{Hom}}
\DeclareMathOperator\Ext{\mathsf{Ext}}
\DeclareMathOperator\Tor{\mathsf{Tor}}

\DeclareMathOperator\HOM{\underline{\mathsf{Hom}}}
\DeclareMathOperator\EXT{\underline{\mathsf{Ext}}}
\DeclareMathOperator\TOR{\underline{\mathsf{Tor}}}

\DeclareMathOperator\Ab{\mathsf{Ab}}
\DeclareMathOperator\ShHom{\mathcal Hom}
\DeclareMathOperator\ShExt{\mathcal Ext}
\DeclareMathOperator\ShTor{\mathcal Tor}

\DeclareMathOperator\pd{pd}
\DeclareMathOperator\id{id}
\DeclareMathOperator\rank{rank}
\DeclareMathOperator\Spec{Spec}
\DeclareMathOperator\supp{supp}
\DeclareMathOperator\ann{ann}
\DeclareMathOperator\im{Im}
\DeclareMathOperator\gr{gr}

%%%%%%%%%%%%%%%%%%%%%%%%%%%%%%%%%%%%%% TITLES %%%%%%%%%%%%%%%%%%%%%%%%%%%%%%
\title{Comodule structure on hom's}
\date{[hom-comodule.tex]}
\author{Pablo Zadunaisky}

\begin{document}
\maketitle
We fix a field $\k$. All algebras and tensor products are over $\k$ unless 
otherwise stated. Given a coalgebra $H$ we denote by $\Mod^H$ the
category of right $H$-comodules and write $\Hom^H$ for $\Hom_{\Mod^H}$.

\section{Hom comodules}
\paragraph
Let $H$ be a Hopf algebra over $\k$ and let $M, N$ be $H$-comodules. We define 
the $H$-comodule-hom space $\HOM^{G}(M, N)$ as the pull-back
\begin{align*}
\xymatrix{
	\Hom_\k(M, N) \ar[r]^-{\rho^{N}_*} 
		& \Hom_\k(M, N \ot H) \\
	\HOM^{H}(M, N) \ar@{-->}[u]^-\alpha \ar@{-->}[r]^-\rho 
		& \Hom_\k(M, N) \ot H \ar[u]_-{\Gamma}
}
\end{align*}
where for each $m \in M$ we define $\Gamma(f \ot h)(m) = f(m_0) \ot g m_1$.

Since
\[ 
	(1 \ot \vep)_* \circ \rho^{N}_* 
		= (1 \ot \vep \circ \rho^{N})_* = (\Id_{N})_*
\]
we know that $\rho^{N}_*$ is injective, and hence so is $\rho$. 

It is also true that $\Gamma$ is injective. Let $T: \Hom(M,N \ot H) \to 
\Hom_\k(M,N \ot H)$ be defined by $T(F)(m) = F(m_0) \cdot 1 \ot S(m_1)$. Thus 
we get a commutative diagram
\begin{align*}
\xymatrix{
	\Hom_\k(M, N \ot H) \ar[dr]^-T \\
	\Hom_\k(M, N) \ot H \ar[u]^\Gamma \ar[r]^-i 
		& \Hom_\k(M,N \ot H)
}
\end{align*}
where $i(f \ot h)(m) = f(m) \ot h$. Since $i$ is injective, so is $\Gamma$.

Hence $\HOM^G(M,N) \subset \Hom_\k(M,N)$ consists of those functions $f: M 
\to N$ for which there exist $f_1, \ldots, f_r \in \Hom_\k(M,N)$ and $h_1, 
\ldots, h_r \in H$ such that for every $m \in M$ we get $f(m)_0 \ot f(m)_1 = 
\sum_{i = 1}^r f_i(m_0) \ot h_i m_i$. 

\paragraph
For example, suppose $G$ is the group algebra of $\hat G$, in which case $M, N$ are
$\hat G$-graded modules. Then $\HOM^G(M,N)$ consists of those morphisms which are
finite sums of homogeneous morphisms of arbitrary degree, i.e. $\HOM^G(M,N) =
\bigoplus_{g \in \hat G}\Hom^G(M,N[g])$. It is a well-known result that if $M$ is finite
dimensional then $\HOM^G(M,N) = \Hom_\k(M,N)$.

On the other hand, suppose $\k$ has characteristic zero and let $G$ be the enveloping
algebra of the one dimensional Lie algebra with basis $X$. A $G$-comodule is a vector
space $M$ with a locally nilpotent endomorphism $t: M \to M$ such that $\rho^M(m) = \sum_n
\frac{t^n(m)}{n!} \ot X^n$. Suppose $M, N$ are $G$-comodules with structure maps given by
endomorphisms $t, u$. Take $F \in \Hom_\k(M,N)$, and set
\[
	F_n = \sum_{i + j = n} \frac{(-1)^{i}}{i!j!} u^{j} \circ F \circ t^i.
\]
Then $F \in \HOM^G(M, N)$ if and only if $F_n = 0$ for $n \gg 0$. In that case $\rho(F) =
\sum_n F_n \ot X^n$. Notice that $F_1 = u \circ F - F \circ t$ is zero if and only if $F$
is a comodule morphism. If $M$ is finite dimensional then so is $F(M)$; hence both $s$ and
the restriction of $t$ to the image of $F$ are nilpotent, and $F_n = 0$ for $n > 2 \dim
M$. Thus if $M$ is finite dimensional $\HOM^G(M,N) = \Hom_\k(M,N)$

\paragraph
\label{hom-G-comodule}
The $\k$-vector space $\HOM^{G}(M,N)$ is a $G$-comodule. Indeed, consider the following
diagram
\begin{align*}
	\xymatrix{
		\Hom_\k(M,N) \ar[r]^{\rho^{N}_*}
		& \Hom_\k(M,N \ot G) \ar[r]^-{(\rho^{N} \ot 1)_*}
			& \Hom_\k(M,N \ot G \ot G)\\
		\HOM^G(M,N) \ar[r]^-\rho \ar[u]^-\alpha
			&\Hom_\k(M,N) \ot G \ar[r]^-{\rho^{N}_* \ot 1} \ar[u]^-\Gamma \ar[dr]^-{1
				\ot \Delta}
			& \Hom_\k(M, N \ot G) \ot G \ar[u]_-\Gamma \\
		{} 
		&\HOM^{G}(M,N) \ot G \ar[u]^-{\alpha \ot 1} \ar[r]^-{\rho \ot 1}
			& \Hom_\k(M,N) \ot G \ot G \ar[u]_-{\Gamma \ot 1}\\
	}
\end{align*} 
Let $T = \rho_*^{N} \ot 1 \circ \rho$ and $U = \Gamma \ot 1 \circ 1 \ot \Delta \circ
\rho$. Since the lower-right square is a pull-back diagram, in order to show that the
image of $\rho$ is contained in $\HOM^G(M,N) \ot G$ it is enough to show that $T = U$.
Since $\Gamma$ is injective this is equivalent to showing that $\Gamma \circ T = \Gamma
\circ U$. By the commutativity of the first two rows of the diagram
\begin{align*}
	\Gamma \circ T(f) = \rho^{N} \ot 1 \circ \rho^{N} \circ f = 1 \ot \Delta \circ
	\rho^{N} \circ f.
\end{align*}
Using the fact that $\rho^N \circ f(m) = \sum_{i = 1}^r f_i(m_0) \ot g_im_1$, we see that
\begin{align*}
	\Gamma(T(f(m))) = \sum_i f_i(m_0) \ot (g_i)_0m_1 \ot (g_i)_1m_2 = \Gamma \circ U(f)(m), 
\end{align*}
which completes the proof.

\begin{Proposition}
	Let $M,N$ be objects of $\Mod^G$. If $M$ is finitely generated then $\HOM^G(M,N) =
	\Hom_\k(M,N)$.
\end{Proposition}
\begin{proof}
	Fix $f \in \Hom_\k(M,N)$. Choose a basis $\B$ of $G$ and for each $m \in M$ and $g \in
	\B$ let $f_g(m) \in M$ be the element uniquely determined by the formula
	\[
		\rho^M (f(m_0)) S(m_1)  = \sum_{g \in \B} f_g(m) \ot g,
	\]
	which defines linear functions $f_g: M \to N$; since $M$ is finite dimensional there
	are at most finitely many $g$ such that $f_g \neq 0$.	Now for every $m \in M$ we get
	\begin{align*}
		\Gamma(\sum_g f_g \ot g)(m) 
			&= \sum_g f_g(m_0) \ot g m_1 
		= \rho^M(f(m_0)) (S(m_1)m_2) \\
			&= \rho^M(f(m_0)) \eps(m_1) = \rho^M(f(m)),
	\end{align*}
	which shows that $f \in \HOM^G(M,N)$.
\end{proof}

\section{Relative Hom}
We now generalize the previous construction.

\paragraph
Let $G, H$ be Hopf-algebras and let $\phi: G \to H$ be a Hopf-algebra morphism. We see $G$
as a left $H$-comodule with structure morphism $\rho^G = \phi \ot 1 \circ \Delta_G: G \to
H \ot G$. Let $L = {}^{\co H} G \subset G$ be the subalgebra of coinvariants. Suppose $L
\subset G$ is cleft, i.e. there exists a convolution invertible morphism of left
$H$-comodules $\gamma: H \to G$. The fact that $\gamma$ is a comodule morphism implies
that 
\begin{align}
	\label{gamma}
	\phi(\gamma(h)_1) \ot \gamma(h)_2 = h_1 \ot \gamma(h_2)
\end{align}
Let $\gamma^{-1}$ be the convolution-inverse of $\gamma$. Since $\rho^G$ is an algebra
morphism, $\rho^G \circ \gamma$ is convolution invertible with inverse given by $\rho^G
\circ \gamma^{-1}$. On the other hand, setting $\sigma = S \ot 1 \circ \tau \circ
\Delta_H$ where $\tau$ is the flip, we see that
\begin{align*}
	\rho^G \circ \gamma * \sigma (h) 
		&= \rho^G \circ \gamma(h_1)\sigma(h_2) \\
		&= h_1 \ot \gamma(h_2) \cdot S(h_4) \ot \gamma^{-1}(h_3) \\
	 	&= h_1 S(h_4) \ot \gamma(h_2)\gamma^{-1}(h_3) = h_1 S(h_3) \eps(h_2) \ot 1 = \eps(h)
			(1 \ot 1).
\end{align*}
Since inverses are unique we get
\begin{align}
	\label{gamma-inverse}
	\rho^G \circ \gamma^{-1}(h) =  S(h_2) \ot \gamma^{-1}(h_1).
\end{align}

%\begin{Lemma}
%	For every $h \in H$ and every $l \in L$, the element $\gamma^{-1}(h_1) \gamma(h_2)_1
%	l_1 \ot \gamma(h_2)_2 l_2$ belongs to $L \ot G$.
%\end{Lemma}
%\begin{proof}
%	Consider $G \ot G$ as a left $H$-comodule with structure map $\rho^G \ot 1$. Suppose
%	$\sum g \ot g' \in {}^{\co H} G \ot G$ and assume the set of all $g'$ is linearly
%	independent. Then clearly each $g$ is in $L$, and hence ${}^{\co H}(G \ot G) = L \ot
%	G$. Thus it is enough to show that the element lies in the coinvariants of $G \ot G$.
%	
%	Notice that
%	\[
%		\gamma^{-1}(h_1) \gamma(h_2)_1 l_1 \ot \gamma(h_2)_2l_2 = \gamma^{-1}(h_1)\ot 1 \cdot
%		\Delta_G(\gamma(h_2)l)
%	\]
%	Using	\ref{gamma-inverse} we get $\rho^G \ot 1 (\gamma^{-1}(h_1)) = S(h_2) \ot
%	\gamma^{-1}(h_1) \ot 1$. On the other hand, using \ref{gamma}
%	\begin{align*}
%		\rho^G\ot 1( \Delta_G(\gamma(h_3)l)) 
%		&= \phi(\gamma(h_3)_1) \phi(l_1) \ot \gamma(h_3)_2 l_2 \ot \gamma(h_3)_3 l_3 \\
%		&= 1 \ot \Delta_G(\phi(\gamma(h_3)_1) \phi(l_1) \ot \gamma(h_3)_2 l_2)\\
%		&= 1 \ot \Delta_G(h_3 \ot \gamma(h_4)l) = h_3 \ot \gamma(h_4)_1 l_1 \ot
%		\gamma(h_4)_2 l_2.
%	\end{align*}
%	Hence
%	\begin{align*}
%		\rho^G \ot 1 &(\gamma^{-1}(h_1) \gamma(h_2)_1 l_1 \ot \gamma(h_2)_2l_2 )) \\
%		&= \rho^G \ot 1(\gamma^{-1}(h_1) \ot 1) \cdot \rho^G \ot 1(\Delta_G(\gamma(h_2)l)) \\
%		&= S(h_2) \ot \gamma^{-1}(h_1) \ot 1 \cdot h_3 \ot \gamma(h_4)_1l_1 \ot
%		\gamma(h_4)_2 l_2\\
%		&= S(h_2)h_3 \ot \gamma^{-1}(h_1)\gamma(h_4)_1l_1 \ot \gamma(h_4)_2l_2 \\
%		&= 1 \ot \gamma^{-1}(h_1) \gamma(h_2)_1 l_1 \ot \gamma(h_2)_2l_2
%	\end{align*}
%\end{proof}

\begin{Lemma}
	For every $h \in H$ the element $\gamma^{-1}(h_1) \gamma(h_2)_1\ot \gamma(h_2)_2$ lies
	in $L \ot G$.
\end{Lemma}
\begin{proof}
	Consider $G \ot G$ as a left $H$-comodule with structure map $\rho^G \ot 1$. Suppose
	$\sum g \ot g' \in {}^{\co H} G \ot G$ and assume the set of all $g'$ is linearly
	independent. Then clearly each $g$ is in $L$, and hence ${}^{\co H}(G \ot G) = L \ot
	G$. Thus it is enough to show that the element lies in the coinvariants of $G \ot G$.

	Notice that
	\[
		\gamma^{-1}(h_1) \gamma(h_2)_1 \ot \gamma(h_2)_2 = \gamma^{-1}(h_1)\ot 1 \cdot
		\Delta_G(\gamma(h_2))
	\]
	Using	\ref{gamma-inverse} we get $\rho^G \ot 1 (\gamma^{-1}(h_1) \ot 1) = S(h_2) \ot
	\gamma^{-1}(h_1) \ot 1$. On the other hand, using \ref{gamma}
	\begin{align*}
		\rho^G\ot 1(\Delta_G(\gamma(h_3))) 
		&= \phi(\gamma(h_3)_1) \ot \gamma(h_3)_2 \ot \gamma(h_3)_3 \\
		&= 1 \ot \Delta_G(\phi(\gamma(h_3)_1) \ot \gamma(h_3)_2)\\
		&= 1 \ot \Delta_G(h_3 \ot \gamma(h_4)) = h_3 \ot \gamma(h_4)_1 \ot \gamma(h_4)_2.
	\end{align*}
	Hence
	\begin{align*}
		\rho^G \ot 1 &(\gamma^{-1}(h_1) \gamma(h_2)_1 \ot \gamma(h_2)_2)) \\
		&= \rho^G \ot 1(\gamma^{-1}(h_1) \ot 1) \cdot \rho^G \ot 1(\Delta_G(\gamma(h_2))) \\
		&= S(h_2) \ot \gamma^{-1}(h_1) \ot 1 \cdot h_3 \ot \gamma(h_4)_1 \ot
		\gamma(h_4)_2 \\
		&= S(h_2)h_3 \ot \gamma^{-1}(h_1)\gamma(h_4)_1 \ot \gamma(h_4)_2 \\
		&= 1 \ot \gamma^{-1}(h_1) \gamma(h_2)_1  \ot \gamma(h_2)_2
	\end{align*}
\end{proof}

\paragraph
Given a $G$-comodule $M$, we define a map $\Gamma: \hom_\k(L,M) \ot H \to \hom_\k(L, M \ot
G)$ by
\begin{align*}
	\Gamma(f \ot h)(l) = f \ot 1 ((\gamma^{-1}(h_1)\gamma(h_2)_1 \ot \gamma(h_2)_2) \cdot
	(l_1 \ot l_2)).
\end{align*}
By the previous lemma, the argument of the function lies in $L \ot G$ so this is
well-defined. We define $\phi_*(M)$ to be the pull-back of the following diagram
\begin{align*}
	\xymatrix{
		\hom_\k(L,M) \ar[r]^-{\rho^M_*} & \hom_\k(L, M \ot G) \\
		\phi_*(M) \ar@{-->}[u] \ar@{-->}[r] & \hom_\k(L,M) \ot H \ar[u]_-\Gamma
	}
\end{align*}
In principle $\phi_*(M)$ is only a vector-space. We will show that under an extra
hypothesis it is in fact an $H$-comodule, and that the assignation is functorial.
Recall that a \emph{cross-section} of $\phi: G \to H$ is a co-algebra morphism $\gamma: H
\to G$ such that $\phi \circ \gamma = \Id_H$.
\begin{Proposition*}
	Suppose $\gamma: H \to G$ is a cross-section of $\phi$. Then for every $G$-comodule $M$
	the vector space $\phi_*(M)$ has an $H$-comodule structure.
\end{Proposition*}
\begin{proof}
	Under the hypothesis, $\Gamma(f \ot h)(l) = f(l_1) \ot \gamma(h)l_2$.
	Consider the following diagram
	\begin{align*}
	\xymatrix{	
		\hom_\k(L,M \ot G) \ar[r]^-T 
			& \hom_\k(L, M \ot G) \ar[d]^-{(1\ot \phi)_*} \\
		\hom_\k(L,M) \ot H \ar[u]^-\Gamma \ar[r]^-{i} 
			& \hom_\k(L,M \ot H) 
	}
	\end{align*}
	where $T(f)(l) = f(l_1)S(l_2)$ and $i(f \ot h)(l) = f(l) \ot h$. The fact that $\gamma$
	is a	coalgebra morphism and $\phi \circ \gamma = \Id_H$ implies that the diagram
	commutes. Since $i$ is injective, so is $\Gamma$. The rest follows as in
	\ref{hom-G-comodule}.
\end{proof}
This shows that $\phi_*(M)$ can be seen as the set of $\k$-linear morphisms $f: L \to M$
such that there exists an element $\rho^{\phi_*(M)}(f) = \sum_i f_i \ot h_i \in
\hom_\k(L,M) \ot H$ with $\rho^M(f(l)) = \sum_i f_i(l_1) \ot \gamma(h_i)l_2$.

\paragraph
Before proving that $\phi_*$ is adjoint to $\phi^*$, we make the follwoing observation: if
$\gamma: H \to G$ is a cross-section of $\phi$, then it is convolution invertible with
inverse $\gamma^{-1} = S \circ \gamma$; in particular $\gamma^{-1}$ is an anti co-algebra
morphism, i.e. $\gamma^{-1}(h)_1 \ot \gamma^{-1}(h)_2 = \gamma^{-1}(h_2) \ot
\gamma^{-1}(h_1)$.
\begin{Theorem*}
	Suppose $\phi: G \to H$ has a cross-section $\gamma$. Then the functor $\phi_*: \Mod^G
	\to \Mod^H$ is left adjoint to $\phi^*: \Mod^H \to \Mod^G$.
\end{Theorem*}
\begin{proof}
	We wish to find for every $N$ in $\Mod^H$ and every $M$ in $\Mod^G$ isomorphisms
	\begin{align*}
	\xymatrix{
		\hom^H(N, \phi_*(M)) \ar@/^18pt/[rr]^-{\alpha} 
			&{}
			& \hom^G(\phi^*(N), M) \ar@/^15pt/[ll]^-\beta.
	}
	\end{align*}
	Suppose we are given an $H$-comodule morphism $f: N \to \phi_*(M)$, i.e., for every $n
	\in N$ we get a function $f(n) \in \phi^*(M) \subset \hom_\k(L, M)$. We set $\alpha(f):
	\phi^*(N) \to M$ as 
	\[
		\alpha(f)(n \ot g) = f(n_0)(\gamma^{-1}(n_1)g) = f(n)(\gamma^{-1}(\phi(g_1))g_2).
	\]
	We must prove that $\alpha(f)$ is a $G$-module morphism, that is $\rho^M \circ
	(\alpha(f)) = \alpha(f) \ot 1 \circ \rho^{\phi^*(N)}$. By hypothesis $f(n) \in
	\phi^*(M)$, and since $f$ is an $H$-comodule morphism we get $\rho(f(n_0)) = f(n_0) \ot
	n_1$. We fix $n \ot g \in \phi^*(N)$, so
	\begin{align*}
		\rho^M(\alpha(f)(n \ot g)) 
			&= \rho^M(f(n_0)(\gamma^{-1}(n_1)g)) \\
			&= f(n_0)(\gamma^{-1}(n_2)_1g_1) \ot \gamma(n_1)\gamma^{-1}(n_2)_2g_2\\
			&= f(n_0)(\gamma^{-1}(n_3)g_1) \ot \gamma(n_1)\gamma^{-1}(n_2)g_2 \\
			&= f(n_0)(\gamma^{-1}(n_1)g_1) \ot g_2 \\
			&= \alpha(f)(n \ot g_1) \ot g_2.
	\end{align*}
 	Hence $\alpha$ is well-defined.

	Now given $f: \phi^*(N) \to M$ we set 
	\[
		\beta(f)(n)(l) = f(n_0 \ot \gamma(n_1)l)
	\]
	so $\beta(f)(n) \in \hom_\k(L,M)$. Since $f$ and $\gamma$ are comodule morphisms, 
	\begin{align*}
		\rho^M(\beta(f)(n)(l)) 
		&= \rho^M(f(n_0 \ot \gamma(n_1)l)) \\
		&= f(n_0 \ot \gamma(n_1)l_1) \ot \gamma(n_2)l_2\\
		&= \Gamma(\beta(f)(n_0) \ot n_1)(l)
	\end{align*}
	which shows that the image of $\beta(f)$ is contained in $\phi_*(M)$ and that
	$\rho(\beta(f)(n)) = \beta(f)(n_0) \ot n_1$, so it is an $H$-comodule morphism.

	The fact that $\alpha$ and $\beta$ are inverses is a straightforward computation.
\end{proof}



\paragraph
\begin{enumerate}
	\item OK for now. It should work with $\gamma$ an $H$-comodule section, i.e. $G$ an
		$H$-cleft extension.
\end{enumerate}

\end{document}
