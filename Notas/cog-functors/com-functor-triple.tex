\documentclass[11pt,fleqn]{article}

\usepackage[paper=a4paper]
  {geometry}

\pagestyle{plain}

\usepackage{paragraphs}
\usepackage{hyperref}
\usepackage{amsthm,thmtools}

\usepackage[utf8]{inputenc}
\usepackage[spanish,english]{babel}
\usepackage{enumitem}
\usepackage[osf,noBBpl]{mathpazo}
\usepackage[alphabetic,initials]{amsrefs}
\usepackage{amsfonts,amssymb,amsmath}
\usepackage{mathtools}
\usepackage{graphicx}
\usepackage[poly,arrow,curve,matrix]{xy}
\usepackage{wrapfig}
\usepackage{xcolor}
\usepackage{helvet}
\usepackage{stmaryrd}
\usepackage{showlabels}


%%%%%%%%%%%%Theorems, for paragraphs package%%%%%%%%%%%%%%%%%%%%%%%%%%
% numbered versions
\declaretheoremstyle[headformat=swapnumber, spaceabove=\paraskip,
bodyfont=\itshape]{mystyle}
\declaretheoremstyle[headformat=swapnumber, spaceabove=\paraskip,
bodyfont=\normalfont]{mystyle-plain}
\declaretheorem[name=Lemma, sibling=para, style=mystyle]{Lemma}
\declaretheorem[name=Proposition, sibling=para, style=mystyle]{Proposition}
\declaretheorem[name=Theorem, sibling=para, style=mystyle]{Theorem}
\declaretheorem[name=Corollary, sibling=para, style=mystyle]{Corollary}
\declaretheorem[name=Definition, sibling=para, style=mystyle]{Definition}

% unnumbered versions
\declaretheoremstyle[numbered=no, spaceabove=\paraskip,
bodyfont=\itshape]{mystyle-empty}
\declaretheoremstyle[numbered=no, spaceabove=\paraskip,
bodyfont=\itshape]{mystyle-empty-plain}
\declaretheorem[name=Lemma, style=mystyle-empty]{Lemma*}
\declaretheorem[name=Proposition, style=mystyle-empty]{Proposition*}
\declaretheorem[name=Theorem, style=mystyle-empty]{Theorem*}
\declaretheorem[name=Corollary, style=mystyle-empty]{Corollary*}
\declaretheorem[name=Definition, style=mystyle-empty]{Definition*}
\declaretheorem[name=Remark, style=mystyle-empty]{Remark*}

% plain style
\declaretheoremstyle[
        headformat={{\bfseries\NUMBER.}{\itshape\NAME}\NOTE\ignorespaces},
        spaceabove=\paraskip, 
        headpunct={.},
        headfont=\itshape,
        bodyfont=\normalfont
        ]{mystyle-plain}
\declaretheorem[sibling=para, style=mystyle-plain]{Example}
\declaretheorem[sibling=para, style=mystyle-plain]{Remark}

% proofs, just as in amsthm but with adapted spacing

\makeatletter
\renewenvironment{proof}[1][\textit{Proof}]{\par
  \pushQED{\qed}%
  \normalfont \topsep.75\paraskip\relax
  \trivlist
  \item[\hskip\labelsep
        \itshape
    #1\@addpunct{.}]\ignorespaces
}{%
  \popQED\endtrivlist\@endpefalse
}
\makeatother

\newcommand\note[1]{\marginpar{{
\begin{flushleft}
\tiny#1
\end{flushleft}
}}}

\renewcommand\labelitemi{-}
%%%%%%%%%%%%%%%%%%%%%%%%%%% The usual stuff%%%%%%%%%%%%%%%%%%%%%%%%%
\newcommand\NN{\mathbb N}
\newcommand\CC{\mathbb C}
\newcommand\QQ{\mathbb Q}
\newcommand\RR{\mathbb R}
\newcommand\ZZ{\mathbb Z}

\newcommand\maps{\longmapsto}
\newcommand\ot{\otimes}
\newcommand\sq{\square}
\renewcommand\to{\longrightarrow}
\renewcommand\phi{\varphi}
\renewcommand\k{\Bbbk}
\newcommand\vspan[1]{\left\langle #1 \right\rangle}

\DeclareMathOperator\Mod{\mathsf{Mod}}
\DeclareMathOperator\mmod{\mathsf{mod}}
\DeclareMathOperator\Hom{\mathsf{Hom}}
\DeclareMathOperator\Ext{\mathsf{Ext}}
\DeclareMathOperator\Tor{\mathsf{Tor}}
\DeclareMathOperator\HOM{\underline{\mathsf{Hom}}}
\DeclareMathOperator\EXT{\underline{\mathsf{Ext}}}
\DeclareMathOperator\TOR{\underline{\mathsf{Tor}}}

\DeclareMathOperator\gr{\mathsf{gr}}
\DeclareMathOperator\im{Im}
\DeclareMathOperator\id{injdim}
\DeclareMathOperator\pd{pdim}
\DeclareMathOperator\ldim{ldim}
\DeclareMathOperator\height{\mathsf{ht}}
\DeclareMathOperator\st{\mathsf{st}}
\DeclareMathOperator\depth{depth}
\DeclareMathOperator\lcd{lcd}
\DeclareMathOperator\Spec{Spec}
\DeclareMathOperator\supp{supp}
\DeclareMathOperator\Id{Id}
\DeclareMathOperator\rank{rk}
\DeclareMathOperator\rk{rk}
\DeclareMathOperator\irr{irr}
\DeclareMathOperator\GKdim{\mathsf{GKdim}}
\DeclareMathOperator\relint{\mathsf{relint}}
\DeclareMathOperator\coker{\mathsf{coker}}
\DeclareMathOperator\tr{\mathsf{tr}}
\DeclareMathOperator\SL{\mathsf{SL}}
\DeclareMathOperator\ev{\mathsf{ev}}
\DeclareMathOperator\wt{\mathsf{wt}}


%%%%%%%%%%%%%%%%%%%%%%%%%%%%%%%%%%%%%% TITLES %%%%%%%%%%%%%%%%%%%%%%%%%%%%%%
\title{
A functor triple for categories of Hopf-Doi modules
\footnote{file:[com-functor-triple.tex]}
}
\date{01-12-16}
\author{A. Solotar, P. Zadunaisky}

\begin{document}
\maketitle

\section{General definitions}
Throughout this document $\k$ denotes a field. Unadorned tensor products $\ot$
will always be over this field, and given two vector spaces $V,W$, we will 
denote by $\hom(V,W)$ the vector space of all $k$-linear maps from $V$ to $W$.

\paragraph
\label{H-comodules-category}
Given a Hopf algebra $H$ over $\k$, we will denote by $\Mod^H$ the category
of right $H$-comodules, and by ${}^H \Mod$ the category of left $H$-comodules. 
Given an $H$-comodule $M$, we will denote its structural map by $\rho^M$.
By \cite{Wis-survey}*{3} this is a Gorthendieck category with enough 
injectives.

Let $M, N$ be two objects in $\Mod^H$; we denote by $\Hom^K(M,N)$ the space
of all $H$-comodule maps from $M$ to $N$. Now let $f: M \to N$ be a $k$-linear 
map. Set $\rho(f) \in \hom(M, N \ot H)$ to be $\rho(f)(m) = f(m_{(0)})_{(0)} 
\ot f(m_{(0)})_{(1)} S(m_{(1)})$. Now since $\k$ is a field there is an 
injective map $i: \hom(M,N) \ot H \to \hom(M, N \ot H)$ given by $i(f \ot h)
(m) = f(m) \ot h$. The map $f$ is called \emph{rational} if $\rho(f) \in 
\hom(M,N) \ot H$, and the space of all rational maps from $M$ to $N$ is 
denoted by $\HOM^K(M,N)$. If $f$ is rational then $\rho(f) \in \HOM^K(M,N)$
and $(\HOM^K(M,N), \rho)$ is an $H$-comodule. By definition $\rho(f) = f_{(0)} 
\ot f_{(1)}$ if and only if $f_{(0)}(m) \ot f_1 = f(m_{(0)})_{(0)} \ot 
f(m_{(0)})_{(1)} S(m_{(1)})$. If $M$ is finite dimensional then $\HOM^K(M,N)
= \hom(M,N)$; in general, $\HOM^K(M,N)$ is the largest subspace of $\hom(M,N)$
with an $H$-comodule structure given by $\rho$.

Given $N$ a left $H$-comodule and $M$ a right $H$-comodule, their cotensor 
product $N \sq_H M$ is defined as the kernel of the map $M \ot N \to M \ot H 
\ot N$ given by $n \ot m \mapsto n_{(0)} \ot n_{(1)} \ot m - n \ot m_{(0)} \ot 
m_{(1)}$. In other words $M \sq_H N$ consists of all elements $\sum_i n^i \ot 
m^i \in N \ot M$ such that $\sum_i n^i_{(0)} \ot n^i_{(1)} \ot m^i = \sum_i 
n^i \ot m^i_{(0)} \ot m^i_{(1)}$. It follows from the definitions that $N 
\sq_K -$ and $-\sq_H M$ are right exact functors with values in the category 
of vector spaces, and that they respect arbitrary direct sums.

\paragraph
\label{A-H-hd-modules-categories}
Let $H$ be a Hopf algebra and let $A$ be a comodule algebra over $H$. We 
denote by $\Mod_A^H$ the category of $(A,H)$-Doi-Hopf modules. This too is a 
Grothendieck category with enough injectives \cite{Wis-survey}*{5.1}. 

Let $M,N$ be two $(A,H)$-Doi-Hopf modules. The space of all $A$-linear 
rational modules is denoted by $\HOM^K_A(M,N)$. It is a $K$-comodule, and
if $M$ is finitely generated as $A$-module then $\HOM^K_A(M,N) = \Hom_A(M,N)$
as vector spaces.


\section{An adjoint triple for Hopf-Doi modules}

\paragraph
Let $H,K$ be Hopf algebras over $\k$ and let $\phi: H \to K$ be a Hopf-algebra
morphism. The map $\phi$ induces a functor $\phi_!: \Mod^H \to \Mod^K$, which 
sends an object $(M, \rho^M)$ to $(M, 1 \ot \phi \circ \rho^M)$; in other 
words, given a right $H$-comodule $M$, we can see it as a $K$-comodule whose 
structural map is given by $m \mapsto m_{(0)} \ot \phi(m_{(1)})$. This is an 
exact functor. Of course the same applies for left comodules; in particular we 
can see $H$ as either a left or a right $K$-comodule. We will denote by
$L \subset H$ the subalgebra of left coinvariants of $H$.

Now consider $H$ as a left $K$-comodule with the structure induced by $\phi$, 
and a right $H$-comodule in the obvious way; then for each $K$-comodule $N$ 
the cotensor product $N \sq_K H$ is a right $H$-comodule, with structural map 
$n \sq h \mapsto n \sq h_{(0)} \ot h_{(1)}$. By 
\cite{Doi-coalg}*{Proposition 6} the functor $\phi^*: \Mod^K \to \Mod^H$ given 
by $\phi^* = - \sq_K H$ is right adjoint to $\phi_!$. 


\paragraph
Let $A$ be an $H$-comodule algebra. We can apply the functor $\phi_!$ and thus 
obtain a $K$-comodule algebra $\phi_!(A)$. We will often write $A$ instead of
$\phi_!(A)$, as the context will make it clear whether we are considering $A$ 
as an $H$-comodule or as a $K$-comodule algebra.

Suppose $(M, \cdot_A, \rho^M)$ is an object of $\Mod^H_A$; then $(M, \cdot_A,
1 \ot \phi \circ \rho^M)$ is an object of $\Mod^K_A$. This gives us a functor,
which by analogy we denote by $\phi_!$. We will make the usual abuse of 
notation, and when we write $M$ instead of $(M, \cdot_A, \rho^M)$ we will also
write $\phi_!(M)$ instead of $(M, \cdot_A, 1 \ot \phi \rho^M)$. The functor 
$\phi_!$ reflects exactness.

Now let $(N, \cdot_A, \rho^N)$ be an object of $\Mod^K_A$. We can then define 
a right $A$-module structure on the $H$-comodule $N \sq_K H$ as follows.
Clearly $N \ot H$ is a right $A \ot H$-module, so it is also an $A$-module
through the map $\rho^A: A \to A \ot H$; it is then easy to check that $N 
\sq_K H \subset N \ot H$ is a right $A$-submodule of $N \ot H$. We thus
obtain a functor $\phi^*: \Mod^K_A \ot \Mod^H_A$. 

\begin{Proposition*}
The functor $\phi_!: \Mod^H_A \to \Mod^K_A$ is left adjoint to $\phi_!: 
\Mod^K_A \to \Mod^H_A$.
\end{Proposition*}
\begin{proof}
This is the $A$-linear version of \cite{Doi-coalg}*{Proposition 6}. There
it is shown that 
\begin{itemize}
\item for each $K$-colinear map $f: \phi_!(M) \to N$ we get
an $H$-colinear map $\alpha(f): M \to N \sq_K H$ setting $\alpha(f) = f \ot 1 
\circ \rho^M$;

\item for each $K$-colinear map $g: M \to N \sq_K H$ we get a $K$-colinear map 
$\beta(g): \phi_!(M) \to N$ setting $\beta(g) = 1 \ot \phi \circ g$ and 
identifying $N \sq_K K$ with $N$;

\item and $\alpha \circ \beta(g) = g, \beta \circ \alpha (f) = f$.
\end{itemize}
Notice also that if $f$ and $g$ are $A$-linear then so are $\alpha(f)$ and 
$\beta(g)$. Thus the restriction and correstriction of $\alpha$ and $\beta$
give us bijections between $\Hom^H_A(M, \phi^*(N))$ and $\Hom_A^K(\phi_!(M), 
N)$. Naturallity of the bijections is easy to check.
\end{proof}

The following is an easy consequence of the Proposition.
\begin{Corollary*}
The functor $\phi^*$ sends injective objects to injective objects.
\end{Corollary*}

\paragraph
We now explore under which conditions $\phi^*$ has a right adjoint. Since 
$\Mod^K_A$ and $\Mod^H_A$ are Grothendieck categories, by Fryed's adjoint 
functor theorem $\phi^*$ has a right adjoint if and only if it is right exact 
and preserves arbitrary direct sums. 

Let $\{N_i: i \in I\}$ be an arbitrary family of objects in $\Mod^K_A$. Their
direct sum in the category $\Mod^K_A$ is simply their direct sum as vector
spaces with the obvious $A$-module and $K$-comodule structures. Now cotensor 
products commute with arbitrary direct sums. In other words, the obvious map 
\[
  \left( 
    \bigoplus_{i \in I} N_i
  \right) \sq_K H \to
  \left( 
    \bigoplus_{i \in I} N_i \sq_K H
  \right)  
\]
is an isomorphism of $H$-comodules, and it is easy to check that it is also 
$A$-linear, so $\phi^*$ always commutes with direct sums. Thus $\phi^*$
has a right adjoint if and only if it is exact. Since $- \sq_K H$ is exact as
a functor from $K$-comodules to $H$-comodules if and only if $H$ is injective
as $K$-comodule \cite{Doi-coalg}*{Proposition 5}, we see that if $H$ is 
injective as left $K$-comodule, then $\phi^*$ has a right adjoint.


\begin{bibdiv}
\begin{biblist}

\bibselect{biblio}


\end{biblist}
\end{bibdiv}

\end{document}