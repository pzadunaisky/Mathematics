\documentclass[11pt,fleqn]{article}

\usepackage[paper=a4paper]
  {geometry}

\pagestyle{plain}

\usepackage{paragraphs}
\usepackage{hyperref}
\usepackage{amsthm,thmtools}

\usepackage[utf8]{inputenc}
\usepackage[spanish,english]{babel}
\usepackage{enumitem}
\usepackage[osf,noBBpl]{mathpazo}
\usepackage[alphabetic,initials]{amsrefs}
\usepackage{amsfonts,amssymb,amsmath}
\usepackage{mathtools}
\usepackage{graphicx}
\usepackage[poly,arrow,curve,matrix]{xy}
\usepackage{wrapfig}
\usepackage{xcolor}
\usepackage{helvet}
\usepackage{stmaryrd}
\usepackage{showlabels}

%%%%%%%%%%%%Theorems, for paragraphs package%%%%%%%%%%%%%%%%%%%%%%%%%%
% numbered versions
\declaretheoremstyle[headformat=swapnumber, spaceabove=\paraskip,
bodyfont=\itshape]{mystyle}
\declaretheoremstyle[headformat=swapnumber, spaceabove=\paraskip,
bodyfont=\normalfont]{mystyle-plain}
\declaretheorem[name=Lemma, sibling=para, style=mystyle]{Lemma}
\declaretheorem[name=Proposition, sibling=para, style=mystyle]{Proposition}
\declaretheorem[name=Theorem, sibling=para, style=mystyle]{Theorem}
\declaretheorem[name=Corollary, sibling=para, style=mystyle]{Corollary}
\declaretheorem[name=Definition, sibling=para, style=mystyle]{Definition}

% unnumbered versions
\declaretheoremstyle[numbered=no, spaceabove=\paraskip,
bodyfont=\itshape]{mystyle-empty}
\declaretheoremstyle[numbered=no, spaceabove=\paraskip,
bodyfont=\itshape]{mystyle-empty-plain}
\declaretheorem[name=Lemma, style=mystyle-empty]{Lemma*}
\declaretheorem[name=Proposition, style=mystyle-empty]{Proposition*}
\declaretheorem[name=Theorem, style=mystyle-empty]{Theorem*}
\declaretheorem[name=Corollary, style=mystyle-empty]{Corollary*}
\declaretheorem[name=Definition, style=mystyle-empty]{Definition*}
\declaretheorem[name=Remark, style=mystyle-empty]{Remark*}

% plain style
\declaretheoremstyle[
        headformat={{\bfseries\NUMBER.}{\itshape\NAME}\NOTE\ignorespaces},
        spaceabove=\paraskip, 
        headpunct={.},
        headfont=\itshape,
        bodyfont=\normalfont
        ]{mystyle-plain}
\declaretheorem[sibling=para, style=mystyle-plain]{Example}
\declaretheorem[sibling=para, style=mystyle-plain]{Remark}

% proofs, just as in amsthm but with adapted spacing

\makeatletter
\renewenvironment{proof}[1][\textit{Proof}]{\par
  \pushQED{\qed}%
  \normalfont \topsep.75\paraskip\relax
  \trivlist
  \item[\hskip\labelsep
        \itshape
    #1\@addpunct{.}]\ignorespaces
}{%
  \popQED\endtrivlist\@endpefalse
}
\makeatother

\newcommand\note[1]{\marginpar{{
\begin{flushleft}
\tiny#1
\end{flushleft}
}}}

\renewcommand\labelitemi{-}
\setenumerate[0]{label=(\alph*)}
%%%%%%%%%%%%%%%%%%%%%%%%%%% The usual stuff%%%%%%%%%%%%%%%%%%%%%%%%%
\newcommand\NN{\mathbb N}
\newcommand\CC{\mathbb C}
\newcommand\QQ{\mathbb Q}
\newcommand\RR{\mathbb R}
\newcommand\ZZ{\mathbb Z}

\newcommand\maps{\longmapsto}
\newcommand\ot{\otimes}
\newcommand\sq{\square}
\renewcommand\to{\longrightarrow}
\renewcommand\phi{\varphi}
\renewcommand\k{\Bbbk}
\newcommand\vspan[1]{\left\langle #1 \right\rangle}
\renewcommand\O{\mathcal O}
\newcommand\R{\mathcal R}
\newcommand\op{\mathsf{op}}
\newcommand\K{\mathcal K}
\newcommand\D{\mathcal D}
\newcommand\A{\mathcal A}
\renewcommand\L{\mathcal L}
\newcommand\F{\mathcal F}

\DeclareMathOperator\Mod{\mathsf{Mod}}
\DeclareMathOperator\Hom{\mathsf{Hom}}
\DeclareMathOperator\Ext{\mathsf{Ext}}
\DeclareMathOperator\Tor{\mathsf{Tor}}
\DeclareMathOperator\HOM{\underline{\mathsf{Hom}}}
\DeclareMathOperator\EXT{\underline{\mathsf{Ext}}}
\DeclareMathOperator\TOR{\underline{\mathsf{Tor}}}

\DeclareMathOperator\im{Im}
\DeclareMathOperator\injdim{injdim}
\DeclareMathOperator\projdim{pdim}
\DeclareMathOperator\ldim{ldim}
\DeclareMathOperator\supp{supp}
\DeclareMathOperator\Id{Id}
\DeclareMathOperator\rk{rk}
\DeclareMathOperator\Res{\mathsf{Res}}
\DeclareMathOperator\coker{coker}
\DeclareMathOperator\nat{nat}

\DeclareMathOperator\SL{\mathsf{SL}}

%%%%%%%%%%%%%%%%%%%%%%%%%%%%%%%%%%%%%% TITLES %%%%%%%%%%%%%%%%%%%%%%%%%%%%%%
\title{
Change of grading, injective dimension and dualizing complexes
\footnote{file:[applications-cog-functors.tex]}
}
\date{06-01-16}
\author{A. Solotar, P. Zadunaisky}

\begin{document}
\maketitle

\section{Introduction}

Suppose $A$ is a ring graded by some group $G$, and denote by $\Mod^G A$ the 
category of $G$-graded $A$-modules. It is a well known fact that a graded 
module $P$ is graded projective if and only if it is projective as $A$-module. 
In other words, the forgetful functor $\O: \Mod^G A \to \Mod A$ sends 
projective objects to projective objects.

If we replace ``projective'' by ``injective'' then the result is not true: 
take $A = \k[x,x^{-1}]$ with the obvious $\ZZ$-grading: $A$ itself is 
injective as an object of the category of graded modules of $A$, but it is not 
an injective $A$-module. The objective of this article is to study the efect 
of the forgetful functor $\O$ on injective dimension. More generally, given a 
group morphism $\phi: G \to H$ we can turn $G$-gradings into $H$-gradings, and
we would like to know how the corresponding change of grading functor 
$\phi_!: \Mod^G A \to \Mod^H A$ affects injective dimension of objects. See
paragraph \ref{cog-functors} for the precise definition of $\phi_!$.

While this is a very natural question, there is not much literature on the
subject. A classical result of R. Fossum and H.-B. Foxby 
\cite{FF-graded}*{Theorem 4.10} states that if $A$ is $\ZZ$-graded noetherian 
and commutative then a $\ZZ$-graded-injective module has injective dimension 
at most $1$. According to M. Van den Bergh \cite{VdB-existence-dc}*{Below 
Definition 6.1}, this result extends to the noncommutative case; a proof
of this fact can be found in the preprint \cite{Yek-note}. All these results 
are obtained by the usual route of going from ungraded to graded objects 
through filtrations and spectral sequences. If the group is not $\ZZ$ then 
these techniques are not available; however, there is one more result in this 
direction, namely if $A$ is graded over a finite group then it is known that a 
graded module is graded injective if and only if it is injective, see 
\cite{NV-graded-book3}*{2.5.2}.

We have already discussed the functor $\phi_!: \Mod^G A \to \Mod^H A$ induced
by a group morphism $\phi: G \to H$. This functor is part of an adjoint triple
$(\phi_!, \phi^*, \phi_*)$ which allow to transfer information between graded 
categories in a systematic way. Using these functors we obtain bounds for the
change in the injective dimension of $G$-graded objects when seen as 
$H$-graded through $\phi$. Our results contain all the previous ones as 
particular cases. The main one is the follwing.
\begin{Theorem*}
Let $\phi: G \to H$ be a group morphism, let $L = \ker \phi$ and let $d$ be 
the cohomological dimension of $L$. Let $A$ be a $G$-graded noetherian 
algebra, and let $I$ be an injective object of $\Mod^G A$. Then the injective
dimension of $\phi_!(I)$ is at most $d$.
\end{Theorem*}

Van den Bergh's noncommutative version of Fossum and Foxby's result for 
$\ZZ$-graded algebras was stated while discussing a certain object in the 
derived category of $\Mod A$ called a noncommutative dualizing complex. 
Roughly speaking, a dualizing complex is an object $R^\bullet$ of the derived 
category of the category of $A$-bimodules satisfying certain conditions which 
guarantee that $\R\HOM_A(-, R^\bullet)$ gives a duality between $\D(\Mod A)$ 
and $\D(\Mod A^\op)$. A graded dualizing complex in principle only guarantees 
dualities at the graded level, but according to Van den Bergh a graded 
dualizing complex is also an ungraded dualizing complex, and the proof (which
he ommits) turns on the fact that a graded injective module has injective 
dimension at most $1$. We use our results to prove a slightly more precise
version of this statement.

The article is structured as follows. In section \ref{COG-FUNCTORS} we recall
the general properties of the change of grading functors. In section 
\ref{COG-INJDIM} we prove our main results on how re-grading affects
injective dimension. Finally in section \ref{COG-DC} we prove similar results
at the derived level and prove Van den Bergh's statement on regradings of 
dualizing complexes.

Throughout this document $\k$ is a commutative ring, and unadorned $\hom$ and 
tensor products are always over $\k$. 

\section{The change of grading functors}
\label{COG-FUNCTORS}

\paragraph
\label{G-graded-vector-spaces}
Let $G$ be a group. A $G$-graded $\k$-module is a $\k$-module $V$ with a fixed 
decomposition $V = \bigoplus_{g \in G} V_g$; we say that $v \in V$ is  
homogeneous of degree $g$ if $v \in V_g$, and $V_g$ is called the 
$g$-homogeneous component of $V$. We usually say graded instead of $G$-graded 
if $G$ is clear from the context.

Given two $G$-graded modules $V$ and $W$, their tensor product is also a 
$G$-graded vector space, where for each $g \in G$ 
\begin{align*}
(V \ot W)_g = \bigoplus_{g' \in G} V_ {g'} \ot W_{(g')^{-1}g}
\end{align*}
A map between graded $\k$-modules $f: V \to W$ is said to be 
\emph{$G$-homogeneous}, or simply homogeneous, if $f(V_g) \subset W_g$ for all 
$g \in G$.  By definition, a homogeneous map $f: V \to W$ induces maps 
$f_g: V_g \to W_g$ for each $g \in G$, and $f = \bigoplus_{g \in G} f_g$; 
we refer to $f_g$ as the homogeneous component of degree $g$ of $f$. The 
\emph{support} of a $G$-graded $\k$-module $V$ is $\supp V = \{g \in G \mid 
V_g \neq 0\}$.

The category $\Mod^G \k$ has $G$-graded modules as objects and homogeneous 
$\k$-linear maps as morphisms. Kernels and cokernels of homogeneous
maps between graded $\k$-modules are graded in a natural way, so a complex
\[
  0 \to V' \to V \to V'' \to 0
\]
in $\Mod^G \k$ is a short exact sequence if and only if it is a short
exact sequence of $\k$-modules, or equivalently if for each $g \in G$
the sequence formed by taking $g$-homogeneous components is exact.

Given an object $V$ in $\Mod^G \k$ and $g \in G$, we denote by $V[g]$ the
$G$-graded $\k$-module whose homogeneous component of degree $g'$ is
$V[g]_{g'} = V_{g'g}$. This gives a natural autoequivalence of $\Mod^G \k$
with itself.

The category $\Mod^G V$ has arbitrary direct sums and products. The direct sum
of graded modules is again graded in an obvious way, but this is not the case
for direct products. Given a collection of graded $\k$-modules $\{V^i \mid i 
\in I\}$, their direct product is the graded $\k$-module whose homogeneous 
decomposition is given by
\begin{align*}
\bigoplus_{g \in G} \prod_{i \in I} V^i_g.
\end{align*}
In other words, the forgetful functor $\O: \Mod^G \k \to \Mod \k$ preserves
direct sums, but not direct products.

\paragraph
\label{G-graded-algebras}
We now recall the general definitions regarding $G$-graded $\k$-algebras.
The reader is refered to \cite{NV-graded-book3}*{Chapter 2} for proofs and 
details.

A $G$-graded $\k$-algebra is a $G$-graded $\k$-module $A$ which is also a 
$\k$-algebra, such that for all $g,g' \in G$ and all $a \in A_g, a' \in A_{g'}$
we have that $aa' \in A_{gg'}$. If $A$ is a $G$-graded algebra then its
\emph{structural map} $\rho: A \to A \ot \k[G]$ is defined as $a \in A_g 
\mapsto a \ot g \in A_g \ot \k[G]_g$ for each $g \in G$; the fact that $A$ is 
a $G$-graded algebra implies that this is an algebra map.

A $G$-graded left $A$-module is a left $A$-module $M$ which is also a 
$G$-graded $\k$-module such that for each $g,g' \in G$ and all $a \in A_g, 
m \in M_{g'}$ it happens that $am \in M_{gg'}$. Once again, we usually say 
graded instead of $G$-graded. We say that $A$ is graded left noetherian if
every left graded $A$-submodule of $A$ is finitely generated, or equivalently 
if every graded $A$-submodule of a finitely generated graded $A$-module is
also finitely generated.

We denote by $\Mod^G A$ the category whose objects are $G$-graded
$A$-modules and whose morphisms are $G$-homogeneous $A$-linear maps. 
Notice that if $M$ is a graded $A$-module then the graded $\k$-module $M[g]$
is also a graded $A$-module, with the same underlying $A$-module structure,
so shifting also induces an autoequivalence of $\Mod^G A$. 

The category $\Mod^G A$ is a Grothendieck category with enough projective and 
injective objects. Given an object $M$ of $\Mod^G A$, we will denote by 
$\projdim_A^G M$ and $\injdim_A^G M$ its projective and injective dimension, 
respectively. Given two graded $A$-modules $M, N$ we denote by $\Hom^G_A(M,N)$ 
the $\k$-module of all $G$-homogeneous $A$-linear morphisms from $M$ to $N$. 
Since $\Mod^G A$ has enough injectives, we can define for each $i \geq 0$ the 
$i$-th right derived functor of $\Hom^G_A$, which we denote by $\R^i\Hom^G_A$. 

There is also an enriched homomorphism functor $\HOM_A^G$, given by
\begin{align*}
\HOM_A^G(M,N) = \bigoplus_{g \in G} \Hom_A^G(M,N[g]),
\end{align*}
which is a $G$-graded $\k$-submodule of $\Hom_\k(M,N)$. We denote its right
derived functors by $\R^i\HOM_A^G$.

\paragraph
\label{cog-functors}
Let $G,H$ be groups and let $A$ be a $G$-graded $\k$-algebra. As shown in 
\cite{RZ-twisted}*{section 1.3}, any group homomorphism $\phi: G \to H$ 
induces three functors $\phi_!, \phi_*: \Mod^G A \to \Mod^H \k$ and $\phi^*: 
\Mod^H A \to \Mod^G A$, which form an exact triple $(\phi_!, \phi^*, \phi_*)$.
We quickly review the construction for the benefit of the reader.

Let $V$ be a $G$-graded $\k$-module. We define $\phi_!(V)$ to be the 
$H$-graded $\k$-module whose homogeneous component of degree $h \in H$ is 
given by
\begin{align*}
\phi_!(V)_h
  &= \bigoplus_{\{g \in G \mid \phi(g) = h\}} V_g.
\end{align*}
Analogously given a map $f: V \to W$ between $G$-graded $\k$-modules, we
define $\phi_!(f)$ to be the $\k$-linear map whose homogeneous component of 
degree $h \in H$ is given by
\begin{align*}
\phi_!(f)_h
  &= \bigoplus_{\{g \in G \mid \phi(f) = h\}} f_g.
\end{align*}
Notice that $\phi_!(V)$ has the same underlying $\k$-module as $V$. In 
particular, $\phi_!(A)$ is an $H$-graded $\k$-algebra which is equal to $A$ as 
$\k$-module, and if $V$ is a $G$-graded $A$-module then $\phi_!(V)$ is an 
$H$-graded $\phi_!(A)$-module with the same underlying action of $A$. Since 
the action of $A$ remains unchanged, if $f$ is $A$-linear then so is 
$\phi_!(f)$. This defines the functor $\phi_!: \Mod^G A \to \Mod^H \phi_!(A)$. 
From now on we usually write $A$ instead of $\phi_!(A)$ to lighten up the 
notation, since the context will make it clear whether we are considering it 
as a $G$-graded or as an $H$-graded algebra.

We define $\phi_*(V)$ and $\phi_*(f)$, to be the $H$-graded $\k$-module,
and $H$-homogeneous map whose homogeneous components of degree $h \in H$ 
are given by
\begin{align*}
\phi_*(V)_h
  &= \prod_{\{g \in G \mid \phi(g) = h\}} V_g,
&\phi_*(f)_h
  &= \prod_{\{g \in G \mid \phi(f) = h\}} f_g,
\end{align*}
respectively. If $V$ is also an $A$-module, we define the action of a 
homogeneous element $a \in A_{g'}$ with $g' \in G$ over an element 
$(v_g)_{g \in \phi^{-1}(h)}$ as $a(v_g) = (av_g)$. With this action $\phi_*(V)$
becomes an $H$-graded $A$-module, and we have defined the functor 
$\phi_*: \Mod^G A \to \Mod^H A$.

Now let $V',W'$ be $H$-graded $\k$-modules and let $f': V' \to W'$ be an 
$H$-homogeneous map. We set $\phi^*(V') \subset V' \ot \k[G]$ to be the 
subspace generated by all elements of the form $v \ot g$ with $v \in V'$ 
homogeneous of degree $\phi(g)$, and $\phi^*(f)(v \ot g) = f(v) \ot g$. In 
other words, for each $g \in G$ the homogeneous components of $\phi^*(V')$
and $\phi(f')$ are given by
\begin{align*}
\phi^*(V')_g 
  &=  V'_{\phi(g)} \ot \langle g \rangle,
  &f_g
  &= f_{\phi(g)} \ot \Id.
\end{align*}
If $V'$ is an $H$-graded $A$-module, then $V' \ot \k[G]$ is an 
$A \ot \k[G]$-module, and it becomes an $A$-module through the structure map 
$\rho: A \to A \ot \k[G]$; it is immediate to check that with this action it 
becomes a $G$-graded $A$-module with $(V' \ot \k[G])_g = V' \ot \vspan{g}$ for 
each $g \in G$, and that $\phi^*(V') \subset V' \ot \k[G]$ is a $G$-graded 
$A$-submodule. It is also easy to check that if $f'$ is homogeneous and 
$A$-linear then so is $\phi^*(f)$. Thus we have defined a functor $\phi^*: 
\Mod^H A \to \Mod^G A$.


\paragraph
\label{P:adjoint}
We refer to $\phi_!, \phi^*$ and $\phi_*$ collectively as the \emph{change of
grading functors}. As mentioned before, the change of grading functors form an adjoint triple, meaning that $\phi_!$ is left adjoint to $\phi^*$, which in 
turn is left adjoint to $\phi_*$. A proof of this result can be found in 
\cite{RZ-twisted}*{Proposition 3.2.1}. In that reference the groups are 
assumed to be commutative, but this is not used in the proof.
It is clear from the definitions that the change of grading 
functors are exact, and that $\phi_!, \phi_*$ reflect exactness, i.e. a complex
is exact if and only if its image by any of them is also exact. The functor 
$\phi^*$ reflects exactness if and only if $\phi$ is surjective.

\section{Injective dimension and change of grading}
\label{COG-INJDIM}
Throughout this section $\phi: G \to H$ denotes a group morphism and we set 
$L = \ker \phi$. Also $A$ denotes a $G$-graded $\k$-algebra.

\paragraph
\label{hom-dim-inequalities}
It is a well known fact that a $G$-graded $A$-module is projective if and 
only if it is projective as $A$-module. In other words, if $H = \{0\}$
and $M$ is an object of $\Mod^G A$ then $\phi_!(M)$ is projective if and only 
if $M$ is projective. As stated in the introduction, there is not much 
information on how $\phi_!$ behaves with respect to injectives, and this
is the question we now study. We begin by recalling a previous result.
\begin{Proposition*}[\cite{RZ-twisted}*{Corollaries 3.2.2, 3.2.3}]
Let $M$ be an object of $\Mod^G A$. Then the following hold.
\begin{enumerate}
\item  $\projdim^G_A M = \projdim^H_A \phi_!(M)$ and $\injdim^G_A M \leq 
\injdim^H_A \phi_!(M)$. 

\item $\projdim^G_A M \leq \projdim^H_A \phi_*(M)$ and $\injdim^G_A M = 
\injdim^H_A \phi_*(M)$. 
\end{enumerate}
\end{Proposition*}

\paragraph
\label{phi-finite}
The natural inclusion of the direct sum of a family of modules into its 
product gives rise to a natural transformation $\eta: \phi_! \Rightarrow 
\phi_*$. Notice that $\eta(M): \phi_!(M) \to \phi_*(M)$ is an isomorphism if 
and only if for each $h \in H$ the set $\supp M \cap \phi^{-1}(h)$ is finite. 

\begin{Definition*}
An object $M$ of $\Mod^G A$ is called \emph{$\phi$-finite} if for each $h \in 
H$ the set $\supp M \cap \phi^{-1}(h)$ is finite. 
\end{Definition*}

For example, if $|L| < \infty$ then every $G$-graded $A$-module is 
$\phi$-finite. Also, if $A$ is $\phi$-finite then every finitely generated 
$G$-graded $A$-module is $\phi$-finite. In view of this, the following obvious 
consequence of Proposition \ref{hom-dim-inequalities} applies in all these 
situations. 
\begin{Theorem*}
If an object $M$ of $\Mod^G A$ is $\phi$-finite then $\injdim^G_A M = 
\injdim_A^H \phi_!(M)$. 
\end{Theorem*}

\begin{Remark*}
Following the ideas of \cite{Lev-ncreg}*{section 3}, one can show that if $A$ 
is $\NN$-graded and noetherian, and $N$ is a $\ZZ$-graded module such that 
$N_n = 0$ for $n \ll 0$ then the graded injective dimension of $N$ coincides 
with its injective dimension as $A$-module. This result is obtained through the
use of a spectral sequence which is not available if the grading group is
not $\ZZ$. On the other hand, we have not been able to reprove this statement
using the change of grading functors, but combining both techniques yields the 
following nice result.

Assume $A$ is $\NN^r$-graded for some $r > 0$, and let $\psi: \ZZ^r \to \ZZ$ 
be the morphism $\psi(z_1, \ldots, z_r) = z_1 + \cdots + z_r$. Then $\psi_!(A)$
is $\NN$-graded, and given a finitely generated $\ZZ^r$-graded $A$-module $M$
the previous theorem implies that $\injdim_A^{\ZZ^r} M = \injdim_A^\ZZ 
\psi_!(M)$. If $A$ is also noetherian then we see that $\injdim_A^{\ZZ^r} M$
coincides with the injective dimension of $M$ as $A$-module.
\end{Remark*}

\paragraph
\label{P:resolution}
The algebra $\k[G]$ is a $G$-graded $\k$-algebra, and hence
through $\phi$ it is also an $H$-graded algebra, so we may consider the 
category of $H$-graded $\k[G]$-modules $\Mod^H \k[G]$. The algebra 
$\k[H]$ is an object in this category with its usual $H$-grading and the 
action of $\k[G]$ induced by $\phi$. By \cite{Mont-hopf-book}*{Theorem 8.5.6}, 
the functor $- \ot \k[H]: \Mod \k[L] \to \Mod^H \k[G]$ is an equivalence of 
categories. In particular the projective dimension of $\k[H]$ in $\Mod^H \k[G]$
equals $\projdim_{\k[L]} \k$, the cohomological dimension of $L$.

Let $N$ be an $H$-graded $A$-module. Let us say that $N$ is almost $G$-graded
if there exist an $H$-graded $A$-module $N'$ and a $G$-graded 
$A$-module $M$ such that $N \oplus N' \cong \phi_!(M)$ as $H$-graded 
$A$-modules. The following proposition states that every $H$-graded module 
has a resolution by almost $G$-graded modules, and that the length of this 
resolution is bounded by the cohomological dimension of $L$.

\begin{Proposition*}
Set $d = \projdim_{\k[G]}^H \k[H] = \projdim_{\k[L]} \k$.
For every object $N$ of $\Mod^H A$ there exists a resolution of $N$ by almost 
$G$-graded modules of length at most $d$.
\end{Proposition*}
\begin{proof}
We define a functor $D_N: \Mod^H \k[G] \to \Mod^H A$.
Given objects $V, W$ and a morphism $f: V \to W$ in $\Mod^H \k[G]$, set 
$D_N(V) = \bigoplus_{h \in H} N_h \ot V_h$ with the obvious $H$-grading,
and set $D_N(f)$ as the restriction and correstriction of $\Id_N \ot f$.
Now $A \ot \k[G]$ acts on $D_N(V)$, and thus $D_N(V)$ is an $A$-module
through the structural map $\rho: A \to A \ot G$. 

It follows from the definitions that $D_N(\k[G]) \cong \phi_!(\phi^*(N))$. 
Now for each $h \in H$ we define a map $n \in N_h \mapsto n \ot h \in 
D_N(\k[H])_h$; the direct sum of these maps gives us an isomorphism $N 
\cong D_N(\k[H])$. Taking a projective resolution $P^\bullet$ of $\k[H]$ of 
length $d$ and applying $D_N$, we obtain a complex $D_N(P^\bullet) \to 
D_N(\k[H]) \cong N$; since $\k[G]$ is a free $\k$-module, projective 
$\k[G]$-modules are projective over $\k$ so this is an exact complex. To finish
the proof it is enough to show that $D_N(P^\bullet)$ is a complex of almost
$G$-graded modules.

For each $i \geq 0$ there exists a projective object $Q^i$ in $\Mod^H 
\k[G]$ such that $P^i \oplus Q^i$ is a free $\k[G]$-module. Writing $r_i$ for 
the rank of $P^i \oplus Q^i$, we get that $D_N(P^i)$ is a direct summand of 
$D_N(P^i \oplus Q^i)$, which isomorphic to the direct sum of $r_i$ copies of 
$\phi_!(\phi^*(N))$. Since $\phi_!$ has a right adjoint, it commutes with 
arbitrary direct sums and $D_N(P^i)$ is almost $G$-graded for each $i$. This 
completes the proof.
\end{proof}

\paragraph
Let $M$ be a $G$-graded $A$-module. For each $l \in L$ we have a map $M[l] \to 
\phi^*\phi_!(M)$ whose homogeneous component of degree $g \in G$ is given by 
$m \in M[l]_g \mapsto m \ot gl \in \phi^*\phi_!(M)$. Taking the direct sum 
over all $l \in L$ we obtain a natural isomorphism
\begin{align*}
\phi^*\phi_!(M) 
  &\cong \bigoplus_{l \in L} M[l].
\end{align*}
This observation is used in the following lemma.

\label{L:acyclic}
\begin{Lemma*}
Assume $A$ is left $G$-graded noetherian. Let $I$ be an injective object of 
$\Mod^G A$. If $N$ is almost $G$-graded then $\R^i\Hom_A^H(N, \phi_!(I)) = 0$
for all $i > 0$.
\end{Lemma*}
\begin{proof}
Clearly it is enough to show that the result holds for $N = \phi_!(M)$ where
$M$ is a $G$-graded $A$-module. In that case we have isomorphisms
\begin{align*}
\Hom_A^H(\phi_!(M), \phi_!(I)) 
  &\cong \Hom_A^G(M, \phi^*(\phi_!(I)))
  \cong \Hom_A^G\left(M, \bigoplus_{l \in L} I[l] \right).
\end{align*}
Since this isomorphism is natural in the first variable, we obtain for each 
$i \geq 0$ natural isomorphisms between the derived functors
\begin{align*}
\R^i \Hom_A^H(\phi_!(M), \phi_!(I)) 
  &\cong\R^i\Hom_A^G\left(M, \bigoplus_{l \in L} I[l] \right).
\end{align*}
Now by the graded version of the Bass-Papp theorem (see 
\cite{GW-noetherian-book}*{Theorem 5.23} for a proof in 
the ungraded case, which adapts easily to the graded context), the fact that 
$A$ is left $G$-graded noetherian implies that $\bigoplus_{l \in L} I[l]$ is 
injective, and hence the last isomorphism implies $\R^i \Hom_A^H(\phi_!(M), 
\phi_!(I)) = 0$.
\end{proof}

\paragraph
\label{T:main-theorem}
We are now ready to prove the main result of this section.
\begin{Theorem*}
Assume $A$ is left $G$-graded noetherian. For every object $M$ of $\Mod^G A$ 
we get $\injdim^H_A \phi_!(M) \leq \injdim^G_A M + d$, where $d = 
\projdim_{\k[L]} \k$.
\end{Theorem*}
\begin{proof}
The case where $M$ is of infinite injective dimension is trivially true, so let
us consider the case where $n = \injdim_A^G M$ is finite. 

If $n = 0$ then $M$ is injective in $\Mod^G A$. Let $N$ be an object of 
$\Mod^H A$, and let $P^\bullet \to N$ be a resolution of $N$ by almost 
$G$-graded objects of $\Mod^H A$ of length $d$ as in Proposition 
\ref{P:resolution}. By Lemma \ref{L:acyclic}, $P^\bullet$ is a resolution of 
$N$ by objects which are acyclic for the functor $\Hom_A^H(-,\phi_!(I))$, so 
\begin{align*}
\R^i\Hom_A^H(N, \phi_!(M)) 
  &\cong H^i(\Hom_A^H(P^\bullet, \phi_!(M)))
\end{align*}
for each $i \geq 0$. Thus $\R^i\Hom_A^H(N, \phi_!(M)) = 0$ for all $i > d$, 
and since $N$ was arbitrary this implies that $\injdim_A^H \phi_!(M) \leq d$.

Now assume that the result holds for all objects of $\Mod^G A$ with
injective dimension less than $n$. Let $M \to I$ be an injective envelope
of $M$ in $\Mod^G A$, and let $M'$ be its cokernel. Then $\injdim^G_A M' = 
n-1$, and so by the inductive hypothesis $\injdim^H_A \phi_!(M') \leq n-1+d$. 
Now we have an exact sequence in $\Mod^H A$ of the form
\begin{align*}
0 \to \phi_!(M) \to \phi_!(I) \to \phi_!(M')  \to 0.
\end{align*}
By standard homological algebra the injective dimension of $\phi_!(M)$ is 
bounded above by the maximum between $\injdim_A^H \phi_!(I) + 1 \leq d + 1$
and $\injdim_A^H \phi_!(M') + 1 \leq n + d$. This gives us the desired 
inequality.
\end{proof}

\section{Change of grading in derived categories and dualizing complexes}
\label{COG-DC}
Dualizing complexes for noncommutative rings were introduced by A. Yekutieli 
in the context of connected $\NN$-graded algebras in order to study their 
local cohomology; they have proven to be very useful in the study of ring 
theoretical properties of non commutative rings, see for example 
\cites{Yek-dc, Jor-lc, VdB-existence-dc, WZ-survey-dc, YZ-rigid-dc}, etc. A 
dualizing complex is essentially an object $R^\bullet$ in the derived category 
of $\Mod A^e$ such that the functor $\R\HOM_A^G(-, R^\bullet)$ is a duality between $\D^b(\Mod A)$ and $\D(\Mod A^\op)$. 

Throughout this section $\k$ is a field, $G$ is an abelian group, and $A$ is a 
$G$-graded $\k$-algebra. We denote by $A^\op$ the opposite algebra of $A$ and 
by $A^e$ the enveloping algebra $A \ot A^\op$; since $G$ is abelian these 
algebras are $G$-graded. 

\paragraph
Let us fix some notation regarding derived categories. Given an abelian 
category $\A$, we denote by $\K(A)$ the category of complexes of objects of 
$\A$ with homotopy classes of maps of complexes as morphisms, and by $\D(\A)$ 
the derived category of $\A$. As usual we denote by $\D^+(\A), \D^-(\A), 
\D^b(\A)$ the full subcategories of $\D(A)$ consisting of left bounded, right 
bounded  and bounded complexes. Recall that an injective resolution of a 
left bounded complex $R^\bullet$ is a quasi-isomorphism $R^\bullet \to 
I^\bullet$ where $I^\bullet$ is a left bounded complex formed by injective 
objects of $\A$. If $\A$ has enough injectives then every left bounded complex 
has an injective resolution. Analogous remarks apply for projective 
resolutions of right bounded complexes.

If $F: \A \to \mathcal B$ is an exact functor between abelian categories, then
by the universal property of derived categories there is an induced functor
$\D(\A) \to \D(\mathcal B)$, which by abuse of notation we will also denote by
$F$.

\paragraph
The obvious maps $A \to A^e$ and $A^\op \to A^e$ induce restriction functors
$\Res_A: \Mod^{G} A^e \to \Mod^{G} A$ and $\Res_{A^\op}: \Mod^{G} A^e \to 
\Mod^{G} A^\op$. These functors are exact and preserve projectives
and injectives, which can be proved as in \cite{Yek-dc}*{Lemma 2.1}. If
$H$ is any group and $\phi: G \to H$ is a group morphism then it is
clear that the associated change of grading functors commute with the 
restriction functors in the obvious sense. Since restriction and change of 
grading functors are exact, they induce exact functors between the 
corresponding derived categories.

\paragraph
There exists a functor
\begin{align*}
\HOM_A^{G}: \K(\Mod^{G} A^e)^\op \times \K(\Mod^{G} A^e) 
  \to \K(\Mod^{G} A^e)
\end{align*}
defined as follows. Given complexes $M^\bullet, N^\bullet$, for each $n \in 
\ZZ$ we set
\begin{align*}
  \HOM_A^{G}(N^\bullet,M^\bullet)^n 
    &= \prod_{p \in \ZZ} \HOM_A^{G}(N^p, M^{p+n}),
\end{align*}
where the product is taken in the category of $G$-graded $A^e$-modules;
this sequence of $G$-graded $A^e$-modules is made into a complex with 
differential
\begin{align*}
  d^n &= \prod_{p \in \ZZ} ((-1)^{n+1}\HOM_A^{G}(d_N^{p},M^{p+n}) + 
  \HOM_A^{G}(N^{p},d_M^{p+n})).
\end{align*}
The action of $\HOM_A^G$ on maps is defined in the obvious way.

The functor $\HOM_A^{G}$ has a right derived functor
\begin{align*}
   \R\HOM_A^{G}: 
    \D(\Mod^{G} A^e)^\op \times \D(\Mod^{G} A^e) 
    \to \D(\Mod^{G} A^e).
\end{align*}
When $M^\bullet$ is an object of $\D^+(\Mod^{G} A^e)$ such that 
$M^i$ is injective as left $A$-module for each $i \in \ZZ$, then 
\[
  \R\HOM_A^{G}(N^\bullet, M^\bullet) 
    \cong \HOM_A^{G}(N^\bullet,M^\bullet)
\] 
for every object $N^\bullet$ of $\D(\Mod^{G} A^e)$. Analogously, if
$N^\bullet$ is an object of $\D^-(\Mod^{G} A^e)$ such that $N^i$ is 
projective as left $A$-module for each $i \in \ZZ$, then 
\[
\R\HOM_A^{G}(N^\bullet, M^\bullet) 
  \cong \HOM_A^{G}(N^\bullet, M^\bullet)
\]
for every object $M^\bullet$ of $\D(\Mod^{G} A^e)$. This is proved in the case
$G = \ZZ$ in \cite{Yek-dc}*{Theorem 2.2}, and the general proof follows the 
same reasoning. There is a completely analogous functor $\HOM^{G}_{A^\op}$ 
with a derived functor $\R\HOM_A^{G}$ with similar properties.

\paragraph
\label{natural-map}
Let $R^\bullet$ be a complex of $A^e$-modules. Seeing $A^\op$ as a
complex of $A^e$-modules concentrated in homological degree $0$, there is an 
obvious map $A^\op \to \HOM_A^{G}(R^\bullet, R^\bullet)$ given by sending $a 
\in A^\op$ to right multiplication by $a$ acting on $R^\bullet$. 
Now let $P^\bullet \to R^\bullet$ be a projective resolution of $R^\bullet$, 
so there is an isomorphism 
\[
\R\HOM_A^{G}(R^\bullet, R^\bullet) 
  \cong \HOM_A^{G}(P^\bullet, P^\bullet),
\]
and we get a map $\nat_A: A^\op \to \R\HOM_A^{\ZZ^r}(R^\bullet, R^\bullet)$. 
This map is independent of the projective resolution we choose, so we refer to 
it as the \emph{natural map} from $A^\op$ to $\R\HOM_A^{\ZZ^r}(R^\bullet, 
R^\bullet)$. In the same way there is a natural map from $A$ to 
$\R\HOM_{A^\op}^{\ZZ^r}(R^\bullet, R^\bullet)$. The proof that these maps are 
independent of the chosen resolution is quite tedious but elementary; the 
reader is refered to \cite{Zad-thesis}*{Appendix A} for details.

\paragraph
\label{dc-definition}
Assume that $G = \ZZ^r$ for some $r \geq 0$. We say that $A$ is 
$\NN^r$-graded if $\supp A \subset \NN^r$, and that it is connected if $A_0 = 
\k$. If $A$ is $\NN^r$-graded then so are $A^\op$ and $A^e$, and they are 
connected if and only if $A$ is connected. 

The following definition is adapted from \cite{Yek-dc}*{Definition 3.3}.
\begin{Definition*}
Let $A$ be a connected $\NN^r$-graded noetherian algebra. A 
\emph{$\ZZ^r$-graded dualizing complex} over $A$ is a bounded complex 
$R^\bullet$ of $A^e$-modules with the following properties.

\begin{enumerate}
\item 
\label{fg-dc}
The cohomology modules of $\Res_A(R^\bullet)$ and $\Res_{A^\op}(R^\bullet)$ 
are finitely generated.

\item 
\label{inj-dc}
Both $\Res_A(R^\bullet)$ and $\Res_{A^\op}(R^\bullet)$ have finite
injective dimension.

\item 
\label{nat-dc}
The maps $\nat_A: A^\op \to \R\HOM_{A}^{\ZZ^r} (R^\bullet, 
R^\bullet)$ and $\nat_{A^\op}: A \to \mathcal R\HOM_{A^\op}^{\ZZ^r}(R^\bullet, 
R^\bullet)$ are isomorphisms in $\D(\Mod^{\ZZ^r} A^e)$.
\end{enumerate}
\end{Definition*}
A dualizing complex in the ungraded sense is an object of $\D(\Mod A^e)$ which
complies with the ungraded analogue of the previous definition. Our objective 
is to show that a $\ZZ^r$-graded dualizing complex remains a dualizing complex 
if we change (or forget) the grading. Since being finitely generated is 
independent of grading, item \ref{fg-dc} of the definition remains true if we 
change or forget the grading. To follow how item \ref{inj-dc} behaves with 
respect to change of grading requires a derived version of Theorem 
\ref{T:main-theorem}, while item \ref{nat-dc} is also invariant by change
of grading by a simple argument. We provide the details in the following 
lemmas, in a slightly more general context.

\paragraph
\label{derived-inj-dim}
Recall that given a group morphism $\phi: G \to H$, a $G$-graded $\k$-vector 
space $M$ is said to be $\phi$-finite if $\supp M \cap \phi^{-1}(h)$ is a 
finite set for each $h \in H$. 

\begin{Lemma*}
Let $\phi: G \to H$ be a group morphism and set $L = \ker \phi$. Let 
$R^\bullet$ be a bounded complex of $G$-graded $A$-modules. 
\begin{enumerate}
\item 
\label{phi-finite-derived-injdim}
If the cohomology modules of $R^\bullet$ are $\phi$-finite then
$\injdim_A^{G} R^\bullet = \injdim_A^H \phi_!(R^\bullet)$

\item 
\label{noetherian-derived-injdim}
Let $d = \projdim_{\k[L]} \k$. If $A$ is left $G$-graded noetherian then 
the following inequalities hold
\[
  \injdim_A^G R^\bullet 
    \leq \injdim_A^H \phi_!(R^\bullet) 
    \leq \injdim_A^G R^\bullet + d.
\]
\end{enumerate}
\end{Lemma*}
\begin{proof}
Let $R^\bullet \to I^\bullet$ be an injective resolution of minimal length.
It is enough to prove the statemenet with $I^\bullet$ instead of $R^\bullet$.

Suppose $I^\bullet$ has $\phi$-finite cohomology modules. Recall that there is 
a natural transformation $\eta: \phi_! \Rightarrow \phi_*$, and that $\eta(M)$ 
is an isomorphism if an only if $M$ is $\phi$-finite. The class of 
$\phi$-finite $G$-graded $A$-modules is closed by extensions, so applying 
\cite{Hart-RD}*{Proposition 7.1} (the ``thick'' in the statement means 
``closed by extensions'') we get that the map $\phi_!(I^\bullet) \to 
\phi_*(I^\bullet)$ is a quasi-isomorphism, and since $\phi_*$ preserves 
injectives it is an injective resolution, so $\injdim_A^{G} R^\bullet \geq 
\injdim_A^H \phi_!(R^\bullet)$. If the inequality is strict, then we can 
truncate $\phi_*(I^\bullet)$ to obtain a shorter complex of the form
\[
\cdots 
  \to \phi_*(I^{j-1}) 
  \to \phi_*(I^{j}) 
  \to \phi_*(\coker d^j) 
  \to 0 
  \to \cdots
\]
with $\phi_*(\coker d^j)$ an injective $H$-graded $A$-module. Since $\phi_*$
preserves injective dimension by Proposition \ref{hom-dim-inequalities}, this
contradicts the fact that $I^\bullet$ is a minimal resolution of $R^\bullet$,
so in fact $\injdim_A^{G} R^\bullet = \injdim_A^H \phi_!(R^\bullet)$. This
proves item \ref{phi-finite-derived-injdim}

For item \ref{noetherian-derived-injdim}, assume first that $I^\bullet$ is 
bounded. We proceed by induction on $s$, the length of $I^\bullet$. The case 
$s = 0$ is a special case of Theorem \ref{T:main-theorem}. Now let $t \in \ZZ$ 
be the minimal homological degree such that $I^t \neq 0$, and consider the 
exact sequence of complexes
\begin{align*}
0 \to I^{> t} \to I^\bullet \to I^t \to 0,
\end{align*}
where $I^t$ is seen as a complex concentrated in homological degree $t$ and
$I^{> t}$ is the subcomplex of $I^\bullet$ formed by all components in 
homological degree larger than $t$. Thus there is a distinguished triangle
$\phi_!(I^{> t}) \to \phi_!(I^\bullet) \to \phi_!(I^t) \to$ in $\D(\Mod^H A)$.
By the inductive hypothesis the inequality holds for the first and third 
complexes of the triangle, so a simple argument with long exact sequences 
shows that the corresponding inequality holds for $\phi_!(I^\bullet)$.

Finally, if $I^\bullet$ is not bounded then we only have to prove that 
$\phi_!(I^\bullet)$ does not have finite injective dimension. Now $\phi^*$
preserves injective dimensions, so $\phi^*\phi_!(I^\bullet) \cong 
\bigoplus_{l \in L} I[l]^\bullet$ has infinite injective dimension, and since
$I^\bullet$ is a direct summand of this complex, it must also have infinite
injective dimension.
\end{proof}

\begin{Lemma}
\label{natural-map-lemma}
Let $G,H$ be abelian groups and $\phi: G \to H$ a group morphism. Assume $A$ 
is $G$-graded noetherian. Let $S^\bullet, R^\bullet$ be bounded 
complexes of $G$-graded $A^e$-modules such that the cohomology modules of 
$R^\bullet$ are finitely generated. 
\begin{enumerate}
\item 
\label{RHOM-cog-iso}
The map
\[
  \phi_!(\R\HOM_A^G(R^\bullet, S^\bullet)) 
    \to \R\HOM_A^H(\phi_!(R^\bullet), \phi_!(S^\bullet))
\]
is an isomorphism.

\item 
\label{nat-cog}
The composition 
\begin{align*}
\xymatrix{
\phi_!(A) 
    \ar[r]^-{\phi_!(\nat_A)}
    &\phi_!(\R\HOM_A^G(R^\bullet, R^\bullet)) \ar[r]
    &\R\HOM_A^H(\phi_!(R^\bullet), \phi_!(R^\bullet))
}
\end{align*}
equals $\nat_{\phi_!(A)}: \phi_!(A) \to \R\Hom_A^H(\phi_!(R^\bullet), 
\phi_!(R^\bullet))$
\end{enumerate}
\end{Lemma}
\begin{proof}
The map from item \ref{RHOM-cog-iso} is obtained as follows. Let $P^\bullet
\to R^\bullet$ be a projective resolution. Then $\phi_!(P^\bullet) \to 
\phi_!(R^\bullet)$ is also a projective resolution since $\phi_!$ is exact and 
preserves projectives. Now by definition of $\HOM^G_A(R^\bullet, S^\bullet)$, 
we have $\phi_!(\HOM_A^{G}(P^\bullet, S^\bullet)) \subset \HOM_A^{H}
(\phi_!(P^\bullet), \phi_!(S^\bullet))$, and the desired map is the inclusion. 
Once again this map is independent of the chosen projective resolution. 
Clearly item \ref{nat-cog} follows from this.

If $R^\bullet$ and $S^\bullet$ are concentrated in homological degree $0$,
item \ref{RHOM-cog-iso} is a well-known result, see for example 
\cite{RZ-twisted}*{Proposition 1.3.7}. The general result follows by standard 
devissage arguments using \cite{Hart-RD}*{Proposition I.7.1}.
\end{proof}

\paragraph
We are now ready to prove the main result of this section. 
\begin{Theorem*}
Let $A$ be a connected $\NN^r$-graded noetherian $\k$-algebra and let
$R^\bullet$ be a $\ZZ^r$-graded dualizing complex over $A$.
\begin{enumerate}
\item 
\label{phi-connected-dc}
Let $s \in \NN^*$ and let $\phi: \ZZ^r \to \ZZ^s$ be a group
morphism such that $\phi_!(A)$ is $\NN^s$-graded connected. Then
$\phi_!(R^\bullet)$ is a $\ZZ^s$-graded dualizing complex over 
$\phi_!(A)$ of injective dimension $\injdim^{\ZZ^r}_A R^\bullet$.

\item 
\label{ungrading-dc}
Let $\O: \D(\Mod^{\ZZ^r} A^e) \to \D(\Mod A^e)$ be the
forgetful functor. Then $\O(R^\bullet)$ is a dualizing complex over $A$
in the ungraded sense, of injective dimension at most $\injdim^{\ZZ^r}_A
R^\bullet + 1$.
\end{enumerate}
\end{Theorem*}
\begin{proof}
Let us prove item \ref{phi-connected-dc}. As we have already noticed, $\phi_!$
commutes with the restriction functors and does not change the fact that a
bimodule is finitely generated as left or right $A$-module, so 
$\phi_!(R^\bullet)$ complies with item \ref{fg-dc} of Definition 
\ref{dc-definition}. Since $A$ is $\ZZ^r$-graded noetherian it is also 
$\ZZ^s$-graded noetherian, and hence $\phi_!(A)$ is locally finite; this 
implies that $A$ is $\phi$-finite, otherwise $\phi_!(A)$ would have a 
homogeneous component of infinite dimension. Since the cohomology modules of 
$R^\bullet$ are finitely generated, they are also $\phi$-finite and hence by 
item \ref{phi-finite-derived-injdim} of Lemma \ref{natural-map}
$\injdim_A^{\ZZ^s} \phi_!(R^\bullet) = \injdim_A^{\ZZ^r} R^\bullet$, so item
\ref{inj-dc} of Definition \ref{dc-definition} also holds for $R^\bullet$. 
Finally item \ref{nat-dc} of the definition follows immediately from item 
\ref{nat-cog} of Lemma \ref{natural-map-lemma}.

We noe prove item \ref{ungrading-dc}. Let $\psi: \ZZ^r \to \ZZ$ be the map 
$\psi(z_1, \ldots, z_r) = z_1 + \cdots + z_r$. Then $A$ is $\psi$-finite and 
$\psi_!(A)$ is connected $\NN$-graded, so by the first item $\psi_!(R^\bullet)$
is a $\ZZ$-graded dualizing complex over $A$ of injective dimension 
$\injdim_A^{\ZZ^r} R^\bullet$. Now a similar reasoning as the one we used for 
the first item, but this time using item \ref{noetherian-derived-injdim} of 
Lemma \ref{natural-map-lemma} shows that $\O(\psi_!(R^\bullet)) = 
\O(R^\bullet)$ is a dualizing complex and gives the bound for its injective 
dimension.
\end{proof}

\begin{bibdiv}
\begin{biblist}
\bibselect{biblio}
\end{biblist}
\end{bibdiv}

\end{document}