\documentclass[11pt,fleqn]{article}

\usepackage[paper=a4paper]
  {geometry}

\pagestyle{plain}

\usepackage{paragraphs}
\usepackage{hyperref}
\usepackage{amsthm,thmtools}

\usepackage[utf8]{inputenc}
\usepackage[spanish,english]{babel}
\usepackage{enumitem}
\usepackage[osf,noBBpl]{mathpazo}
\usepackage[alphabetic,initials]{amsrefs}
\usepackage{amsfonts,amssymb,amsmath}
\usepackage{mathtools}
\usepackage{graphicx}
\usepackage[poly,arrow,curve,matrix]{xy}
\usepackage{wrapfig}
\usepackage{xcolor}
\usepackage{helvet}
\usepackage{stmaryrd}
\usepackage{showlabels}


%%%%%%%%%%%%Theorems, for paragraphs package%%%%%%%%%%%%%%%%%%%%%%%%%%
% numbered versions
\declaretheoremstyle[headformat=swapnumber, spaceabove=\paraskip,
bodyfont=\itshape]{mystyle}
\declaretheoremstyle[headformat=swapnumber, spaceabove=\paraskip,
bodyfont=\normalfont]{mystyle-plain}
\declaretheorem[name=Lemma, sibling=para, style=mystyle]{Lemma}
\declaretheorem[name=Proposition, sibling=para, style=mystyle]{Proposition}
\declaretheorem[name=Theorem, sibling=para, style=mystyle]{Theorem}
\declaretheorem[name=Corollary, sibling=para, style=mystyle]{Corollary}
\declaretheorem[name=Definition, sibling=para, style=mystyle]{Definition}

% unnumbered versions
\declaretheoremstyle[numbered=no, spaceabove=\paraskip,
bodyfont=\itshape]{mystyle-empty}
\declaretheoremstyle[numbered=no, spaceabove=\paraskip,
bodyfont=\itshape]{mystyle-empty-plain}
\declaretheorem[name=Lemma, style=mystyle-empty]{Lemma*}
\declaretheorem[name=Proposition, style=mystyle-empty]{Proposition*}
\declaretheorem[name=Theorem, style=mystyle-empty]{Theorem*}
\declaretheorem[name=Corollary, style=mystyle-empty]{Corollary*}
\declaretheorem[name=Definition, style=mystyle-empty]{Definition*}
\declaretheorem[name=Remark, style=mystyle-empty]{Remark*}

% plain style
\declaretheoremstyle[
        headformat={{\bfseries\NUMBER.}{\itshape\NAME}\NOTE\ignorespaces},
        spaceabove=\paraskip, 
        headpunct={.},
        headfont=\itshape,
        bodyfont=\normalfont
        ]{mystyle-plain}
\declaretheorem[sibling=para, style=mystyle-plain]{Example}
\declaretheorem[sibling=para, style=mystyle-plain]{Remark}

% proofs, just as in amsthm but with adapted spacing

\makeatletter
\renewenvironment{proof}[1][\textit{Proof}]{\par
  \pushQED{\qed}%
  \normalfont \topsep.75\paraskip\relax
  \trivlist
  \item[\hskip\labelsep
        \itshape
    #1\@addpunct{.}]\ignorespaces
}{%
  \popQED\endtrivlist\@endpefalse
}
\makeatother

\newcommand\note[1]{\marginpar{{
\begin{flushleft}
\tiny#1
\end{flushleft}
}}}

\renewcommand\labelitemi{-}
%%%%%%%%%%%%%%%%%%%%%%%%%%% The usual stuff%%%%%%%%%%%%%%%%%%%%%%%%%
\newcommand\NN{\mathbb N}
\newcommand\CC{\mathbb C}
\newcommand\QQ{\mathbb Q}
\newcommand\RR{\mathbb R}
\newcommand\ZZ{\mathbb Z}

\newcommand\maps{\longmapsto}
\newcommand\ot{\otimes}
\newcommand\sq{\square}
\renewcommand\to{\longrightarrow}
\renewcommand\phi{\varphi}
\renewcommand\k{\Bbbk}
\newcommand\vspan[1]{\left\langle #1 \right\rangle}

\DeclareMathOperator\Mod{\mathsf{Mod}}
\DeclareMathOperator\Vect{\mathsf{Vect}}
\DeclareMathOperator\Hom{\mathsf{Hom}}
\DeclareMathOperator\Ext{\mathsf{Ext}}
\DeclareMathOperator\Tor{\mathsf{Tor}}
\DeclareMathOperator\HOM{\underline{\mathsf{Hom}}}
\DeclareMathOperator\EXT{\underline{\mathsf{Ext}}}
\DeclareMathOperator\TOR{\underline{\mathsf{Tor}}}

\DeclareMathOperator\gr{\mathsf{gr}}
\DeclareMathOperator\im{Im}
\DeclareMathOperator\id{injdim}
\DeclareMathOperator\pd{pdim}
\DeclareMathOperator\ldim{ldim}
\DeclareMathOperator\height{\mathsf{ht}}
\DeclareMathOperator\st{\mathsf{st}}
\DeclareMathOperator\depth{depth}
\DeclareMathOperator\lcd{lcd}
\DeclareMathOperator\Spec{Spec}
\DeclareMathOperator\supp{supp}
\DeclareMathOperator\Id{Id}
\DeclareMathOperator\rank{rk}
\DeclareMathOperator\rk{rk}
\DeclareMathOperator\irr{irr}
\DeclareMathOperator\GKdim{\mathsf{GKdim}}
\DeclareMathOperator\relint{\mathsf{relint}}
\DeclareMathOperator\coker{\mathsf{coker}}
\DeclareMathOperator\tr{\mathsf{tr}}
\DeclareMathOperator\SL{\mathsf{SL}}
\DeclareMathOperator\ev{\mathsf{ev}}
\DeclareMathOperator\wt{\mathsf{wt}}


%%%%%%%%%%%%%%%%%%%%%%%%%%%%%%%%%%%%%% TITLES %%%%%%%%%%%%%%%%%%%%%%%%%%%%%%
\title{
Further applications of change of grading functors
\footnote{file:[applications-cog-functors.tex]}
}
\date{01-12-16}
\author{A. Solotar, P. Zadunaisky}

\begin{document}
\maketitle

Throughout this document $\k$ is a field, and unadorned $\hom$ and tensor 
products are always over $\k$. 

\section{The change of grading functors}

\paragraph
\label{G-graded-vector-spaces}
Let $G$ be a group.
A $G$-graded vector space is a vector space $V$ with a fixed decomposition
$V = \bigoplus_{g \in G} V_g$; elements in $V_g$ are called homogeneous of
degree $g$, and $V_g$ is called the $g$-homogeneous component of $V$. Given 
two $G$-graded vector spaces $V, W$, their tensor product is also a $G$-graded 
vector space, where for each $g \in G$ 
\begin{align*}
(V \ot W)_g = \bigoplus_{g' \in G} V_ {g'} \ot W_{(g')^{-1}g}
\end{align*}
The group algebra $\k[G]$ is a $G$-graded vector space in an obvious way.

A map between graded vector spaces $f: V \to W$ is said to be 
\emph{$G$-homogeneous}, or simply homogeneous if the groups is clear from the 
context, if $f(V_g) \subset W_g$ for all $g \in G$. Given a 
$G$-graded vector space $V$ and $g \in G$, we denote by $V[g]$ the graded 
vector space whose homgeneous component of degree $g'$ is $V[g]_{g'} = 
V_{g'g}$. By definition, a homogeneous map $f: V \to W$ induces maps $f_g: 
V_g \to W_g$, and $f = \bigoplus_{g \in G} f_g$ for each $g \in G$; we refer 
to $f_g$ as the homogeneous component of degree $g$ of $f$.

The category $\Vect^G \k$ has $G$-graded vector spaces as objects and 
homogeneous linear maps as morphisms. Kernels and cokernels of homogeneous
maps are $G$-graded vector spaces in a natural way, so a complex
\[
  0 \to V' \to V \to V'' \to 0
\]
in $\Vect^G \k$ forms a short exact sequence if and only if it is a short
exact sequence of vector spaces, or equivalently if for each $g \in G$
the sequence formed by taking $g$-homogeneous components is exact.
Finally, $\Vect^G V$ has arbitrary direct sums and products. The direct sum
of $G$-graded vector spaces is again $G$-graded in an obvious way, but this is 
not the case for direct products. Given a collection of $G$-graded vector 
spaces $\{V^i \mid i \in I\}$, their direct product is the $G$-graded vector 
space whose homogeneous decomposition is given by
\begin{align*}
\bigoplus_{g \in G} \prod_{i \in I} V^i_g.
\end{align*}
In other words, the forgetful functor $\O: \Vect^G \k \to \Vect \k$ preserves
direct sums, but not direct products.


\paragraph
Any group homomorphism $\phi: G \to H$ induces functors $\phi_!, \phi_*: 
\Vect^G \k \to \Vect^H \k$ and $\phi^*: \Vect^H \k \to \Vect^G \k$ as follows.

Let $V,W$ be $G$-graded vector spaces and let $f: V \to W$ be a homogeneous 
map. We define $\phi_!(V)$, resp. $\phi_!(f)$, to be the $H$-graded vector 
space, resp. $H$-homogeneous map, whose homogeneous components of degree
$h \in H$ are given by
\begin{align*}
\phi_!(M)_h
  &= \bigoplus_{\{g \in G \mid \phi(g) = h\}} V_g,
&\phi_!(f)_h
  &= \bigoplus_{\{g \in G \mid \phi(f) = h\}} f_g.
\end{align*}
Analogously, we define $\phi_!(V)$, resp. $\phi_!(f)$, to be the $H$-graded 
vector space, resp. $H$-homogeneous map, whose homogeneous components of degree
$h \in H$ are given by
\begin{align*}
\phi_*(V)_h
  &= \prod_{\{g \in G \mid \phi(g) = h\}} V_g,
&\phi_*(f)_h
  &= \prod_{\{g \in G \mid \phi(f) = h\}} f_g.
\end{align*}
Since direct sums and products reflect exactness in the catgory of vector 
spaces

Now let $V',W'$ be $H$-graded vector spaces and let $f': V' \to W'$ be an 
$H$-homogeneous map. We set $\phi^*(V') \subset V' \ot \k[G]$ to be the 
subspace generated by all elements of the form $v \ot g$ with $v \in V'$ 
homogeneous of degree $\phi(g)$, and $\phi^*(f)(v \ot g) = f(v) \ot g$. In 
other words, for each $g \in G$ the homogeneous components of $\phi^*(V')$
and $\phi(f')$ are given by
\begin{align*}
\phi^*(V')_g 
  &=  V'_{\phi(g)} \ot \langle g \rangle,
  &f_g
  &= f_{\phi(g)} \ot 1.
\end{align*}

\paragraph
A $G$-graded $\k$-algebra is a $\k$-algebra $A$ which is also a $G$-graded
vector space such that for all $g,g' \in G$ and all $a \in A_g, a' \in A_{g'}$
it happens that $aa' \in A_{gg'}$. A $G$-graded (left) $A$-module is a (left)
$A$-module $M$ which is also a $G$-graded vector space such that for each 
$g,g' \in G$ and all $a \in A_g, m \in M_{g'}$ it happens that $am \in 
M_{gg'}$. We denote by $\Mod^G A$ the category whose objects are $G$-graded
$A$-modules and whose morphisms are $G$-homogeneous $A$-linear maps.

\begin{Lemma*}
Let $\phi: G \to H$ be a group morphism, and $A$ a $G$-graded $\k$-algebra.
Then the following hold.
\begin{enumerate}
\item The $H$-graded vector space $\phi_!(A)$ is an $H$-graded $\k$-algebra.
\item There exist functors $\phi_!, \phi_* : \Mod^G A \to \Mod^H \phi_!(A)$
and $\phi^*: \Mod^H \phi_!(A) \to \Mod^G A$ which make the following diagrams
commute
\begin{align*}
\xymatrix{
  \Mod^G A 
    & \Mod^H \phi_!(A) 
    & \Mod^H \phi_!(A)  
    & \Mod^G A
    & \Mod^G A 
    & \Mod^H \phi_!(A) \\
  \Vect^G \k 
    & \Vect^H \k
    & \Vect^H \k  
    & \Vect^G \k
    & \Vect^G \k
    & \Vect^H \k
}
\end{align*}
\end{enumerate}
\end{Lemma*}


\begin{Proposition}
Let $\phi: G \to H$ be a group morphism. 
\begin{enumerate}
\item The functors $\phi_!$ and $\phi_*$ are exact and reflect exactness.
\item The functor $\phi^*$ is exact, and it reflects exactness if and only if
  $\phi$ is surjective.
\item The functor $\phi^*$ is right adjoint to $\phi_!$ and left adjoint to 
  $\phi_*$.
\end{enumerate}
\end{Proposition}
\begin{proof}

\end{proof}

\end{document}