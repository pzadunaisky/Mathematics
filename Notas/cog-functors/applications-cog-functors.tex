\documentclass[11pt,fleqn]{article}

\usepackage[paper=a4paper]
  {geometry}

\pagestyle{plain}

\usepackage{paragraphs}
\usepackage{hyperref}
\usepackage{amsthm,thmtools}

\usepackage[utf8]{inputenc}
\usepackage[spanish,english]{babel}
\usepackage{enumitem}
\usepackage[osf,noBBpl]{mathpazo}
\usepackage[alphabetic,initials]{amsrefs}
\usepackage{amsfonts,amssymb,amsmath}
\usepackage{mathtools}
\usepackage{graphicx}
\usepackage[poly,arrow,curve,matrix]{xy}
\usepackage{wrapfig}
\usepackage{xcolor}
\usepackage{helvet}
\usepackage{stmaryrd}
\usepackage{showlabels}


%%%%%%%%%%%%Theorems, for paragraphs package%%%%%%%%%%%%%%%%%%%%%%%%%%
% numbered versions
\declaretheoremstyle[headformat=swapnumber, spaceabove=\paraskip,
bodyfont=\itshape]{mystyle}
\declaretheoremstyle[headformat=swapnumber, spaceabove=\paraskip,
bodyfont=\normalfont]{mystyle-plain}
\declaretheorem[name=Lemma, sibling=para, style=mystyle]{Lemma}
\declaretheorem[name=Proposition, sibling=para, style=mystyle]{Proposition}
\declaretheorem[name=Theorem, sibling=para, style=mystyle]{Theorem}
\declaretheorem[name=Corollary, sibling=para, style=mystyle]{Corollary}
\declaretheorem[name=Definition, sibling=para, style=mystyle]{Definition}

% unnumbered versions
\declaretheoremstyle[numbered=no, spaceabove=\paraskip,
bodyfont=\itshape]{mystyle-empty}
\declaretheoremstyle[numbered=no, spaceabove=\paraskip,
bodyfont=\itshape]{mystyle-empty-plain}
\declaretheorem[name=Lemma, style=mystyle-empty]{Lemma*}
\declaretheorem[name=Proposition, style=mystyle-empty]{Proposition*}
\declaretheorem[name=Theorem, style=mystyle-empty]{Theorem*}
\declaretheorem[name=Corollary, style=mystyle-empty]{Corollary*}
\declaretheorem[name=Definition, style=mystyle-empty]{Definition*}
\declaretheorem[name=Remark, style=mystyle-empty]{Remark*}

% plain style
\declaretheoremstyle[
        headformat={{\bfseries\NUMBER.}{\itshape\NAME}\NOTE\ignorespaces},
        spaceabove=\paraskip, 
        headpunct={.},
        headfont=\itshape,
        bodyfont=\normalfont
        ]{mystyle-plain}
\declaretheorem[sibling=para, style=mystyle-plain]{Example}
\declaretheorem[sibling=para, style=mystyle-plain]{Remark}

% proofs, just as in amsthm but with adapted spacing

\makeatletter
\renewenvironment{proof}[1][\textit{Proof}]{\par
  \pushQED{\qed}%
  \normalfont \topsep.75\paraskip\relax
  \trivlist
  \item[\hskip\labelsep
        \itshape
    #1\@addpunct{.}]\ignorespaces
}{%
  \popQED\endtrivlist\@endpefalse
}
\makeatother

\newcommand\note[1]{\marginpar{{
\begin{flushleft}
\tiny#1
\end{flushleft}
}}}

\renewcommand\labelitemi{-}
%%%%%%%%%%%%%%%%%%%%%%%%%%% The usual stuff%%%%%%%%%%%%%%%%%%%%%%%%%
\newcommand\NN{\mathbb N}
\newcommand\CC{\mathbb C}
\newcommand\QQ{\mathbb Q}
\newcommand\RR{\mathbb R}
\newcommand\ZZ{\mathbb Z}

\newcommand\maps{\longmapsto}
\newcommand\ot{\otimes}
\newcommand\sq{\square}
\renewcommand\to{\longrightarrow}
\renewcommand\phi{\varphi}
\renewcommand\k{\Bbbk}
\newcommand\vspan[1]{\left\langle #1 \right\rangle}
\renewcommand\O{\mathcal O}
\newcommand\R{\mathcal R}
\newcommand\op{\mathsf{op}}
\newcommand\D{\mathcal D}


\DeclareMathOperator\Mod{\mathsf{Mod}}
\DeclareMathOperator\Hom{\mathsf{Hom}}
\DeclareMathOperator\Ext{\mathsf{Ext}}
\DeclareMathOperator\Tor{\mathsf{Tor}}
\DeclareMathOperator\HOM{\underline{\mathsf{Hom}}}
\DeclareMathOperator\EXT{\underline{\mathsf{Ext}}}
\DeclareMathOperator\TOR{\underline{\mathsf{Tor}}}

\DeclareMathOperator\im{Im}
\DeclareMathOperator\injdim{injdim}
\DeclareMathOperator\projdim{pdim}
\DeclareMathOperator\ldim{ldim}
\DeclareMathOperator\supp{supp}
\DeclareMathOperator\Id{Id}
\DeclareMathOperator\rk{rk}

\DeclareMathOperator\SL{\mathsf{SL}}



%%%%%%%%%%%%%%%%%%%%%%%%%%%%%%%%%%%%%% TITLES %%%%%%%%%%%%%%%%%%%%%%%%%%%%%%
\title{
Further applications of change of grading functors
\footnote{file:[applications-cog-functors.tex]}
}
\date{01-12-16}
\author{A. Solotar, P. Zadunaisky}

\begin{document}
\maketitle

Throughout this document $\k$ is a commutative ring, and unadorned $\hom$ and 
tensor products are always over $\k$. 

\section{The change of grading functors}

\paragraph
\label{G-graded-vector-spaces}
Let $G$ be a group. A $G$-graded $\k$-module is a $\k$-module $V$ with a fixed 
decomposition $V = \bigoplus_{g \in G} V_g$; elements in $V_g$ are called 
homogeneous of degree $g$, and $V_g$ is called the $g$-homogeneous component 
of $V$. We usually say graded instead of $G$-graded if the groups is clear 
from the context.

Given two $G$-graded vector spaces $V, W$, their tensor product is 
also a $G$-graded vector space, where for each $g \in G$ 
\begin{align*}
(V \ot W)_g = \bigoplus_{g' \in G} V_ {g'} \ot W_{(g')^{-1}g}
\end{align*}
A map between graded $\k$-modules $f: V \to W$ is said to be 
\emph{$G$-homogeneous}, or simply homogeneous, if $f(V_g) \subset W_g$ for all 
$g \in G$.  By definition, a 
homogeneous map $f: V \to W$ induces maps $f_g: V_g \to W_g$, and 
$f = \bigoplus_{g \in G} f_g$ for each $g \in G$; 
we refer to $f_g$ as the homogeneous component of degree $g$ of $f$. The 
\emph{support} of a $G$-graded $\k$-module $V$ is the set $\supp V = \{g \in 
G \mid V_g \neq 0\}$.

The category $\Mod^G \k$ has $G$-graded vector spaces as objects and 
homogeneous linear maps as morphisms. Kernels and cokernels of homogeneous
maps are $G$-graded vector spaces in a natural way, so a complex
\[
  0 \to V' \to V \to V'' \to 0
\]
in $\Mod^G \k$ forms a short exact sequence if and only if it is a short
exact sequence of vector spaces, or equivalently if for each $g \in G$
the sequence formed by taking $g$-homogeneous components is exact.

Given an object $V$ in $\Mod^G \k$ and $g \in G$, we denote by $V[g]$ the
$G$-graded $\k$-module whose homogeneous component of degree $g'$ is
$V[g]_{g'} = V_{g'g}$. This gives a natural autoequivalence of $\Mod^G \k$
with itself.

The category $\Mod^G V$ has arbitrary direct sums and products. The direct sum
of $G$-graded vector spaces is again $G$-graded in an obvious way, but this is 
not the case for direct products. Given a collection of $G$-graded 
$\k$-modules $\{V^i \mid i \in I\}$, their direct product is the $G$-graded 
$\k$-module whose homogeneous decomposition is given by
\begin{align*}
\bigoplus_{g \in G} \prod_{i \in I} V^i_g.
\end{align*}
In other words, the forgetful functor $\O: \Mod^G \k \to \Mod \k$ preserves
direct sums, but not direct products.

\paragraph
\label{G-graded-algebras}
A $G$-graded $\k$-algebra is a $\k$-algebra $A$ which is also a $G$-graded
$\k$-module such that for all $g,g' \in G$ and all $a \in A_g, a' \in A_{g'}$
we have that $aa' \in A_{gg'}$. If $A$ is a $G$-graded algebra then its
\emph{structural map} $A \to A \ot \k[G]$ is defined as $a \in A_g \mapsto a 
\ot g \in A_g \ot \k[G]_g$ for each $g \in G$; the fact that $A$ is a 
$G$-graded algebra implies that this is an algebra map.

A $G$-graded left $A$-module is a left
$A$-module $M$ which is also a $G$-graded $\k$-module such that for each 
$g,g' \in G$ and all $a \in A_g, m \in M_{g'}$ it happens that $am \in 
M_{gg'}$. Notice that if $M$ is a $G$-graded $A$-module then the 
$G$-graded $\k$-module $M[g]$ is also a $G$-graded $A$-module, with the same 
underlying $A$-module structure. We say that $A$ is $G$-graded noetherian if
every left $G$-graded $A$-submodule of $A$ is finitely generated, or 
equivalently if every $G$-graded $A$-submodule of a finitely generated 
$G$-graded $A$-module is also finitely generated.

We denote by $\Mod^G A$ the category whose objects are $G$-graded
$A$-modules and whose morphisms are $G$-homogeneous $A$-linear maps. It is a 
Grothendieck category with enough projective and injective objects; the reader 
is refered to \cite{NV-graded-book3}*{Chapter 2} for proofs and details. Given 
an object $M$ of $\Mod^G A$, we will denote by $\projdim_A^G M$ and 
$\injdim_A^G M$ its projective and injective dimension, respectively. Given 
two $G$-graded $A$-modules $M, N$ we denote by $\Hom^G_A(M,N)$ the $\k$-module 
of all $G$-homogeneous $A$-linear morphisms from $M$ to $N$. Since $\Mod^G A$ 
has enough injectives, we can define for each $i \geq 0$ the $i$-th right 
derived functor of $\Hom^G_A$, which we denote by $\R^i\Hom^G_A$. 

\paragraph
\label{cog-functors}
Let $A$ be a $G$-graded $\k$-algebra.
Any group homomorphism $\phi: G \to H$ induces functors $\phi_!, \phi_*: 
\Mod^G A \to \Mod^H \k$ and $\phi^*: \Mod^H A \to \Mod^G A$ as follows.

Let $V,W$ be $G$-graded $\k$-modules and let $f: V \to W$ be a homogeneous 
map. We define $\phi_!(V)$, resp. $\phi_!(f)$, to be the $H$-graded 
$\k$-module, resp. $H$-homogeneous map, whose homogeneous components of degree
$h \in H$ are given by
\begin{align*}
\phi_!(V)_h
  &= \bigoplus_{\{g \in G \mid \phi(g) = h\}} V_g,
&\phi_!(f)_h
  &= \bigoplus_{\{g \in G \mid \phi(f) = h\}} f_g.
\end{align*}
Notice that $\phi_!(V)$ has the same underlying $\k$-module as $V$, we are
only changing its grading. In particular, $\phi_!(A)$ is an $H$-graded 
$\k$-algebra, and if $V$ was a $G$-graded $A$-module then $\phi_!(V)$ is  
an $H$-graded $\phi_!(A)$-module, with the same underlying action; since the
action of $A$ remains unchanged, if $f$ is $A$-linear then so is $\phi_!(f)$.
This defines the functor $\phi_!: \Mod^G A \to \Mod^H \phi_!(A)$.
From now on we usually write $A$ instead of $\phi_!(A)$ to lighten up the 
notation, since the context will make it clear whether we are considering it 
as $G$-graded or as $H$-graded algebra.

We define $\phi_*(V)$, resp. $\phi_*(f)$, to be the $H$-graded 
vector space, resp. $H$-homogeneous map, whose homogeneous components of degree
$h \in H$ are given by
\begin{align*}
\phi_*(V)_h
  &= \prod_{\{g \in G \mid \phi(g) = h\}} V_g,
&\phi_*(f)_h
  &= \prod_{\{g \in G \mid \phi(f) = h\}} f_g.
\end{align*}
If $V$ is also an $A$-module, we define the action of a homogeneous element 
$a \in A_{g'}$ with $g' \in G$ over an element $(v_g)_{g \in \phi^{-1}(h)}$
as $a(v_g) = (av_g)$. With this action $\phi_*(V)$ becomes an $H$-graded 
$A$-module, and we have defined the functor $\phi_*: \Mod^G A \to \Mod^H A$.

Now let $V',W'$ be $H$-graded $\k$-modules and let $f': V' \to W'$ be an 
$H$-homogeneous map. We set $\phi^*(V') \subset V' \ot \k[G]$ to be the 
subspace generated by all elements of the form $v \ot g$ with $v \in V'$ 
homogeneous of degree $\phi(g)$, and $\phi^*(f)(v \ot g) = f(v) \ot g$. In 
other words, for each $g \in G$ the homogeneous components of $\phi^*(V')$
and $\phi(f')$ are given by
\begin{align*}
\phi^*(V')_g 
  &=  V'_{\phi(g)} \ot \langle g \rangle,
  &f_g
  &= f_{\phi(g)} \ot 1.
\end{align*}
If $V'$ is an $H$-graded $A$-module, then $V' \ot \k[G]$ is an 
$A \ot \k[G]$-module, and it becomes an $A$-module through the structure map 
$\rho: A \to A \ot \k[G]$; it is immediate to check that it is a $G$-graded 
$A$-module with $(V' \ot \k[G])_g = V' \ot \vspan{g}$ for each $g \in G$.
With these definitions in place, $\phi^*(V') \subset V' \ot \k[G]$ is a
$G$-graded $A$-submodule. It is easy to check that if $f'$ is homogeneous and 
$A$-linear then so is $\phi^*(f)$. Thus we have defined a functor $\phi^*: 
\Mod^H A \to \Mod^G A$.


\begin{Remark*}
The functors $\phi_!$ and $\phi_*$ are exact and reflect 
exactness. The functor $\phi^*$ is exact, and it reflects exactness if 
and only if $\phi$ is surjective.
\end{Remark*}

\paragraph
\label{P:adjoint}
The following result can be found in \cite{RZ-twisted}*{Proposition 3.2.1}. In 
that reference the groups are assumed to be commutative, but this is not used
in the proof.
\begin{Proposition}
Let $\phi: G \to H$ be a group morphism and let $A$ be a $G$-graded $\k$-
algebra. The functor $\phi^*$ is right adjoint to $\phi_!$ and left adjoint to 
 $\phi_*$. 
\end{Proposition}




\section{Injective dimension and change of grading}
Throughout this section $\phi: G \to H$ denotes a group morphism and we set 
$L = \ker \phi$. Also $A$ denotes a $G$-graded $\k$-algebra.

\paragraph
\label{hom-dim-inequalities}
It is a well known fact that an object $M$ of $\Mod^G A$ is projective if and 
only if it is projective as $A$-module, i.e. $\phi_!(M)$ is projective. On the 
other hand, an injective object $I$ in $\Mod^G A$ need not be injective as 
$A$-module. Our next objective is to clarify how $\phi_!$ affects injective dimension. We begin by recalling a previous result.

\begin{Proposition*}[\cite{RZ-twisted}*{Corollaries 3.2.2, 3.2.3}]
Let $\phi: G \to H$ be a group morphism and let $A$ be a $G$-graded $\k$-
algebra. Let $M$ be an object of $\Mod^G A$. Then the following hold.
\begin{enumerate}
\item  $\projdim^G_A M = \projdim^H_A \phi_!(M)$ and $\injdim^G_A M \leq 
\injdim^H_A \phi_!(M)$. 

\item $\projdim^G_A M \leq \projdim^H_A \phi_*(M)$ and $\injdim^G_A M = 
\injdim^H_A \phi_*(M)$. 
\end{enumerate}
\end{Proposition*}

\paragraph
\label{phi-finite}
The natural inclusion of the direct sum of a family of vector spaces into its 
product gives rise to a natural transformation $\eta: \phi_! \to \phi_*$. 
Notice that $\eta(M): \phi_!(M) \to \phi_*(M)$ is an isomorphism if and only 
if for each $h \in H$ the set $\supp M \cap \phi^{-1}(h)$ is finite. 

\begin{Definition*}
An object $M$ of $\Mod^G A$ is called \emph{$\phi$-finite} if for each $h \in 
H$ the set $\supp M \cap \phi^{-1}(h)$ is finite. 
\end{Definition*}

For example, if $|L| < \infty$ then every $G$-graded $A$-module is 
$\phi$-finite. Also, if $A$ is $\phi$-finite then every finitely generated 
$G$-graded $A$-module is $\phi$-finite. In view of this, the following obvious 
consequence of Proposition \ref{hom-dim-inequalities} applies in all these 
situations.
\begin{Theorem*}
If $M$ is $\phi$-finite then $\injdim^G_A M = \injdim_A^H \phi_!(M)$. 
\end{Theorem*}

\paragraph
\label{dim-L}
As we noted before, the algebra $\k[G]$ is a $G$-graded $\k$-algebra, and hence
through $\phi$ it is also an $H$-graded algebra, so we may consider the 
category of $H$-graded right $\k[G]$-modules $\Mod^H \k[G]$. The algebra 
$\k[H]$ is an object in this category with its usual $H$-grading and the 
action of $G$ on $H$ induced by $\phi$. By 
\cite{Mont-hopf-book}*{Theorem 8.5.6}, the functor $- \ot \k[H]: \Mod \k[L] \to
\Mod^H \k[G]$ is an equivalence of categories. In particular the 
projective dimension of $\k[H]$ in $\Mod^H \k[G]$ equals 
$\projdim_{\k[L]} \k$ the cohomological dimension of $L$.

\paragraph
\label{P:resolution}
Let $N$ be an $H$-graded $A$-module. Let us say that $N$ is almost $G$-graded
if there exist an $H$-graded $A$-module $N'$ and a $G$-graded 
$A$-module $M$ such that $N \oplus N' \cong \phi_!(M)$ as $H$-graded 
$A$-modules. The following Proposition states that every $H$-graded module 
has a resolution by almost $G$-graded modules, and gives a uniform bound for 
the length of that resolution.

\begin{Proposition*}
Set $l = \projdim_{\k[G]}^H \k[H] = \projdim_{\k[L]} \k$.
For every object $N$ of $\Mod^H A$ there exists a resolution of length $l$ of 
$N$ by almost $G$-graded modules.
\end{Proposition*}
\begin{proof}
We define a functor $D_N: \Mod^H \k[G] \to \Mod^H A$ as follows. Given
objects $V, W$ and a morphism $f: V \to W$ in $\Mod^H \k[G]$, set 
$D_N(V) = \bigoplus_{h \in H} N_h \ot V_h$ with the obvious $H$-grading,
and set $D_N(f)$ as the resitriction and correstriction of $1 \ot f$.
Now $A \ot \k[G]$ acts on $D_N(V)$, and the action of $A$ on $D_N(V)$ is
given by correstriction through the map $\rho: A \to A \ot G$, defined as
$\rho(a) = a \ot g$ for each $g \in G$. The functor $D_N$ is exact and 
commutes with arbitrary direct sums. 

Notice that $N \cong D_N(\k[H])$ through the map that sends $n \in N_h$ to
$n \ot h \in D_N(\k[H])_h$ for each $h \in H$. Also it follows from the
definitions that $D_N(\k[G]) \cong \phi_!(\phi^*(N))$. Taking a projective
resolution $P^\bullet$ of $\k[H]$ of length $l$ and applying $D_N$, we obtain
an exact complex $D_N(P^\bullet) \to D_N(\k[H]) \cong N$. Now for each $i 
\geq 0$ there exists a projective object $Q^i$ in $\Mod^H \k[G]$ such that
$P^i \oplus Q^i$ is free of rank $r_i$. Thus $D_N(P^i)$ is a direct summand
of $D_N(P^i \oplus Q^i)$, which isomorphic to the direct sum of $r_i$ copies
of $\phi_!(\phi^*(N))$. Since $\phi_!$ is has a right adjoint, it commutes 
with arbitrary direct sums, so $D_N(P^i)$ is almost $G$-graded for each $i$.
\end{proof}

\paragraph
Let $M$ be a $G$-graded $A$-module. For each $l \in L$ we have a map $M[l] \to 
\phi^*\phi_!(M)$ whose homogeneous component of degree $g \in G$ is given by 
$m \in M[l]_g \mapsto m \ot gl \in \phi^*\phi_!(M)$. Taking the direct sum 
over all $M$ we obtain a natural isomorphism
\begin{align*}
\phi^*\phi_!(M) 
  &\cong \bigoplus_{l \in L} M[l].
\end{align*}
This observation is used in the following lemma.

\label{L:acyclic}
\begin{Lemma*}
Assume $A$ is left $G$-graded noetherian. Let $I$ be an injective object of 
$\Mod^G A$. If $N$ is almos $G$-graded then $\R^i\Hom_A^H(N, \phi_!(I)) = 0$
for all $i > 0$.
\end{Lemma*}
\begin{proof}
Clearly it is enough to show that the result holds for $N = \phi_!(M)$ where
$M$ is a $G$-graded $A$-module. In that case we have isomorphisms
\begin{align*}
\Hom_A^H(\phi_!(M), \phi_!(I)) 
  &\cong \Hom_A^G(M, \phi^*(\phi_!(I)))
  \cong \Hom_A^G\left(M, \bigoplus_{l \in L} I[l] \right).
\end{align*}
Since this isomorphism is natural in the first variable, we obtain for each 
$i \geq 0$ natural isomorphisms between the derived functors
\begin{align*}
\R^i \Hom_A^H(\phi_!(M), \phi_!(I)) 
  &\cong\R^i\Hom_A^G\left(M, \bigoplus_{l \in L} I[l] \right).
\end{align*}
Now by the graded version of the Bass-Papp theorem (see 
\cite{GW-noetherian-book}*{Theorem 5.23} for a proof in 
the ungraded case, which adapts easily to the graded context), the fact that 
$A$ is left $G$-graded noetherian implies that $\bigoplus_{l \in L} I[l]$ is 
injective, and hence the last isomorphism implies $\R^i \Hom_A^H(\phi_!(M), 
\phi_!(I)) = 0$.
\end{proof}

\begin{Theorem}
\label{T:main-theorem}
Assume $A$ is left $G$-graded noetherian.
Set $l = \projdim_{\k[L]} \k$ and let $M$ be an object of $\Mod^G A$. Then 
$\injdim^H_A \phi_!(M) \leq \injdim^G_A M + l$.
\end{Theorem}
\begin{proof}
The case where $M$ is of finite injective dimension is trivially true, so let
us consider the case where $n = \injdim_A^G M$ is finite. 

If $n = 0$ then $M$ is injective in $\Mod^G A$. Let $N$ be an object of 
$\Mod^H A$, and let $P^\bullet \to N$ be a resolution of $N$ by almost 
$G$-graded objects of $\Mod^H A$ of length $l$, as in \ref{P:resolution}.
By Lemma \ref{L:acyclic}, $P^\bullet$ is a resolution of $N$ by objects which 
are acyclic for the functor $\Hom_A^H(-,\phi_!(I))$, so 
\begin{align*}
\R^i\Hom_A^H(N, \phi_!(M)) 
  &\cong H^i(P^\bullet, \phi_!(M))
\end{align*}
for each $i \geq 0$. In particular, if $i > l$ then $\R^i\Hom_A^H(N, 
\phi_!(M)) = 0$ which implies that $\injdim_A^H \phi_!(M) \leq l$.

Now assume that the result holds for all objects of $\Mod^G A$ with
injective dimension inferior to $n$. Let $M \to I$ be an injective hull
of $M$ in $\Mod^G A$, and let $M'$ be its cokernel. Then $\injdim^G_A M' = 
n-1$, and so $\injdim^H_A \phi_!(M') \leq n-1+l$. Now we have in $\Mod^H A$
an exact sequence of the form
\begin{align*}
0 \to \phi_!(M) \to \phi_!(I) \to \phi_!(M')  \to 0.
\end{align*}
By standard homological algebra the injective dimension of $\phi_!(M)$ is 
bounded above by the maximum between $\injdim_A^H \phi_!(I) + 1 \leq l+1$ and 
$\injdim_A^H \phi_!(M') + 1 \leq n+l$. This gives us the desired inequality.
\end{proof}


\section{Dualizing complexes and change of grading}
We finish with a short section on the effect of change of grading on dualizing
complexes. Throughout this section $\k$ is a field.

\paragraph
In this section we denote by $A$ an $\NN^r$-graded algebra and write $A^e = A 
\ot A^\op$. We will also assume that $A$ is $\ZZ^r$-graded left noetherian. By 
\cite{CQ-polycylic}*{Theorem 2.2} this is equivalent to $A$ being noetherian.
We say that $A$ is connected if $A_0 = \k$.

We denote by $\D^{\ZZ^r}(A), \D^{\ZZ^r}(A^\op)$ and $\D^{\ZZ^r}(A^e)$ the 
derived categories of $\Mod^{\ZZ^r} A$, $\Mod^{\ZZ^r} A^{\op}$ and 
$\Mod^{\ZZ^r} A^e$, respectively. We write $\R\HOM^{\ZZ^r}_A$ for the right 
derived functor of the internal $\HOM$ functor of the category of complexes of 
$\ZZ^r$-graded bimodules, as defined in \cite{Yek-dc}*{section 2}. Recall that 
an injective resolution of a complex $R^\bullet$ is an isomorphism $R^\bullet 
\to I^\bullet$ in $\D^{\ZZ^r}(A)$ where all the modules of the complex 
$I^\bullet$ are injective $\ZZ^r$-graded modules, and that the injective 
dimension of $R^\bullet$ is the minimal length of an injective resolution. We 
write $\injdim^{\ZZ^r}_A R^\bullet$ for the injective dimension of $R^\bullet$.

\paragraph
\label{dc}
The following definition is adapted from \cite{Yek-dc}*{Definition 3.3}
\begin{Definition*}
Let $A$ be a connected $\NN^r$-graded noetherian algebra. A 
\emph{$\ZZ^r$-graded dualizing complex} for $A$ is a bounded complex 
$R^\bullet$ in $\D^{\ZZ^r}(A^e)$ such that:
\begin{enumerate}
\item its cohomology modules are finitely generated as left and right 
  $\ZZ^r$-graded $A$-modules;

\item it has finite injective dimension as a complex of left or right 
  $\ZZ^r$-graded $A$-modules;

\item the natural maps $A^\op \to \R\HOM_{A} (R^\bullet, R^\bullet)$
  and $A \to \mathcal R\HOM_{A^\op}(R^\bullet, R^\bullet)$ are isomorphisms
  in $\D^{\ZZ^r}(A^e)$.
\end{enumerate}
\end{Definition*}
A \emph{dualizing complex in the ungraded sense} is an object in 
the derived category of $\Mod A$ which complies with the ungraded analogues of 
properties (1), (2) and (3). Van den Bergh's excercise was originally stated 
while discussing whether a $\ZZ$-graded dualizing complex is also a dualizing 
complex in the ungraded sense. We will give a proof of this result for 
$\ZZ^r$-graded dualizing complexes. 

\begin{Remark*}
The natural maps from item 3 of the previous definition can be understood as 
follows. First, any complex complying $R^\bullet$ with conditions 1 and 2
is isomorphic to a bounded complex of bimodules $I^\bullet$ such that each
$I^i$ is injective as left and right $\ZZ^r$-graded injective module. Hence
there exist isomorphisms
\begin{align*}
\R\HOM_{A} (R^\bullet, R^\bullet) 
  &\cong \HOM_{A} (I^\bullet, I^\bullet), \\
\R\HOM_{A^\op} (R^\bullet, R^\bullet) 
  &\cong \HOM_{A^\op} (I^\bullet, I^\bullet).
\end{align*}
Thus the natural map $A^\op \to \HOM_{A} (I^\bullet, I^\bullet)$ is the map
that assings to each $a \in A$ the $A$-linear morphism given by right 
multiplication by $a$. This is an endomorphism of $I^\bullet$ because the 
differentials of this complex are bimodule morphisms.
\end{Remark*}

\paragraph
\label{derived-inj-dim}
Let $\phi: \ZZ^r \to \{0\}$ be the trivial morphism. Since $\phi_!: 
\Mod^{\ZZ^r} A \to \Mod A$ is exact, it induces an exact functor at the 
derived level $\phi_!: \D^{\ZZ^r}(A) \to \D(A)$. The other change of grading 
functors also induce functors between the corresponding derived categories. We 
will need the following lemma, which is a derived version of Theorem 
\ref{T:main-theorem}. 

\begin{Lemma*}
Assume $A$ is left noetherian. Let $R^\bullet$ be complex in $\D^{\ZZ^r}(A)$
with finite injective dimension. The following inequalities hold.
\[
  \injdim_A^{\ZZ^r} R^\bullet 
    \leq \injdim_A \O(R^\bullet) 
    \leq \injdim_A^{\ZZ^r} R^\bullet +r.
\]
\end{Lemma*}
\begin{proof}
We may replace $R^\bullet$ by an isomorphic bounded complex of injective 
$A$-modules $I^\bullet$. We proceed by induction on $\ell$, the length of 
$I^\bullet$. The case $\ell = 0$ is a special case of Theorem 
\ref{T:main-theorem}. Now let $t \in \ZZ$ be the minimal homological degree
such that $I^t \neq 0$, and consider the exact sequence of complexes
\begin{align*}
0 \to I^{> t} \to I^\bullet \to I^t \to 0,
\end{align*}
where $I^t$ is seen as a complex concentrated in homological degree $t$ and
$I^{> t}$ is the subcomplex of $I^\bullet$ formed by all components in 
homological degree larger than $t$. Thus there is a distinguished triangle
$\O(I^{> t}) \to \O(I^\bullet) \to \O(I^t) \to$ in $\D(\Mod A)$.
Since by induction the stated inequality holds for the first and third 
complexes of the triangle, a simple argument with long exact sequences shows 
that the corresponding inequality holds for $\O(I^\bullet)$.
\end{proof}

\begin{Proposition}
Suppose $A$ is a noetherian connected $\NN^r$-graded algebra.
An object $R^\bullet$ in $\D^{\ZZ^r}(A^e)$ is a $\ZZ^r$-graded dualizing 
complex if and only if $\O(R^\bullet)$ is a dualizing complex in the ungraded 
sense.
\end{Proposition}
\begin{proof}
Since $\O(H^n(R^\bullet)) = H^n(\O(R^\bullet))$, it follows that condition 1 
of Definition \ref{dc} holds for $R^\bullet$ if and only if it holds for 
$\O(R^\bullet)$. By Lemma \ref{derived-inj-dim} $R^\bullet$ has finite 
injective dimension if and only if $\O(R^\bullet)$ has finite injective 
dimension, so condition 2 holds for $R^\bullet$ if and only if it holds for 
$\O(R^\bullet)$.

Since $A$ is noetherian, for any $\ZZ^r$-graded $A$-modules $M,N$ with $M$ 
finitely generated there exist natural isomorphisms
\[
  \O(\R^i\HOM_A^{\ZZ^r}(M,N)) \to \Ext^i_A(\O(M),\O(N)),
\]
see \cite{NV-graded-book3}*{Corollary 2.4.7}. A standard devissage argument 
extends this to a natural isomorphism
\[
  \O(\R\HOM_A^{\ZZ^r}(M^\bullet, N^\bullet)) 
    \to \R\Hom_A(\O(M^\bullet), \O(N^\bullet))
\]
whenever $M^\bullet, N^\bullet$ are bounded complexes of $\ZZ^r$-graded 
$A$-modules and $M^\bullet$ having finitely generated cohomology modules. 
Assume conditions 1 and 2 hold for either $R^\bullet$ or $\O(R^\bullet)$ (and 
hence for both). Then $R^\bullet$ is isomorphic to a bounded complex with 
finitely generated cohomology modules, and so condition 3 holds for $R^\bullet$
if and only if it holds for $\O(R^\bullet)$.
\end{proof}

\begin{bibdiv}
\begin{biblist}
\bibselect{biblio}
\end{biblist}
\end{bibdiv}

\end{document}