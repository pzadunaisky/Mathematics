%%%%%%%%%%%%%%%%%%%%%% Generalities %%%%%%%%%%%%%%%%%%5
\documentclass[11pt,fleqn]{article}
\usepackage[paper=a4paper]
  {geometry}

\pagestyle{plain}
\pagenumbering{arabic}
%%%%%%%%%%%%%%%%%%%%%%%%%%%%%%%%
\usepackage{notas}

%%%%%%%%%%%%%%%%%%%%%%%%%%% The usual stuff%%%%%%%%%%%%%%%%%%%%%%%%%
\newcommand\NN{\mathbb N}
\newcommand\CC{\mathbb C}
\newcommand\QQ{\mathbb Q}
\newcommand\RR{\mathbb R}
\newcommand\ZZ{\mathbb Z}
\renewcommand\k{\Bbbk}

\newcommand\A{\mathcal A}
\newcommand\B{\mathcal B}
\newcommand\Z{\mathsf Z}

\newcommand\maps{\longmapsto}
\newcommand\ot{\otimes}
\renewcommand\to{\longrightarrow}
\renewcommand\phi{\varphi}
\newcommand\id{\mathsf{Id}}
\newcommand\im{\mathsf{im}}
\newcommand\coker{\mathsf{coker}}
%%%%%%%%%%%%%%%%%%%%%%%%% Specific notation %%%%%%%%%%%%%%%%%%%%%%%%%
\newcommand\g{\mathfrak g}
\renewcommand\sl{\mathfrak{sl}}

\DeclareMathOperator\rep{\mathsf{rep}}
\DeclareMathOperator\Rep{\mathsf{Rep}}

\DeclareMathOperator\Hom{\mathsf{Hom}}
\DeclareMathOperator\End{\mathsf{End}}

\newcommand\qbinom[2]{\genfrac{[}{]}{0pt}{0}{#1}{#2}}

%%%%%%%%%%%%%%%%%%%%%%%%%%%%%%%%%%%%%% TITLES %%%%%%%%%%%%%%%%%%%%%%%%%%%%%%
\title{Notes on the HH-book by S. Witherspoon}
\date{[Witherspoon-notes.tex]}
\begin{document}
\maketitle

\section{Chapter I}

\paragraph 
\about{Page 5, paragraph 2.} Turn comment on invariance into a more detailed 
remark? If it is done later, insert a reference.

\paragraph
\about{Page 5, examples 1.1.14, 1.1.16} No need for $k$ to be a field, since 
by definition $A$ is free. The only change necessary is at the end of the 
second example. Also, you mention the identification of $HH^0$ with the center
before establishing it.

\paragraph
\about{Page 15, Tensor products of complexes} You use that $A$ is free over $k$
without introducing the hypothesis. 

\end{document}