%%%%%%%%%%%%%%%%%%%%%% Generalities %%%%%%%%%%%%%%%%%%%
\documentclass[11pt,fleqn]{article}

\usepackage[paper=a4paper]
  {geometry}

\pagestyle{plain}

\usepackage{paragraphs}
\usepackage{hyperref}
\usepackage{amsthm,thmtools}

\usepackage[utf8]{inputenc}
\usepackage[spanish,english]{babel}
\usepackage{enumitem}
\usepackage[osf,noBBpl]{mathpazo}
\usepackage[alphabetic,initials]{amsrefs}
\usepackage{amsfonts,amssymb,amsmath}
\usepackage{mathtools}
\usepackage{graphicx}
\usepackage[poly,arrow,curve,matrix]{xy}
\usepackage{wrapfig}
\usepackage{xcolor}
\usepackage{helvet}
\usepackage{stmaryrd}
%\usepackage{showlabels}


%%%%%%%%%%%%Theorems, for paragraphs package%%%%%%%%%%%%%%%%%%%%%%%%%%
% numbered versions
\declaretheoremstyle[headformat=swapnumber, spaceabove=\paraskip,
bodyfont=\itshape]{mystyle}
\declaretheoremstyle[headformat=swapnumber, spaceabove=\paraskip,
bodyfont=\normalfont]{mystyle-plain}
\declaretheorem[name=Lemma, sibling=para, style=mystyle]{Lemma}
\declaretheorem[name=Proposition, sibling=para, style=mystyle]{Proposition}
\declaretheorem[name=Theorem, sibling=para, style=mystyle]{Theorem}
\declaretheorem[name=Corollary, sibling=para, style=mystyle]{Corollary}
\declaretheorem[name=Definition, sibling=para, style=mystyle]{Definition}

% unnumbered versions
\declaretheoremstyle[numbered=no, spaceabove=\paraskip,
bodyfont=\itshape]{mystyle-empty}
\declaretheoremstyle[numbered=no, spaceabove=\paraskip,
bodyfont=\itshape]{mystyle-empty-plain}
\declaretheorem[name=Lemma, style=mystyle-empty]{Lemma*}
\declaretheorem[name=Proposition, style=mystyle-empty]{Proposition*}
\declaretheorem[name=Theorem, style=mystyle-empty]{Theorem*}
\declaretheorem[name=Corollary, style=mystyle-empty]{Corollary*}
\declaretheorem[name=Definition, style=mystyle-empty]{Definition*}
\declaretheorem[name=Remark, style=mystyle-empty]{Remark*}

% plain style
\declaretheoremstyle[
        headformat={{\bfseries\NUMBER.}{\itshape\NAME}\NOTE\ignorespaces},
        spaceabove=\paraskip, 
        headpunct={.},
        headfont=\itshape,
        bodyfont=\normalfont
        ]{mystyle-plain}
\declaretheorem[sibling=para, style=mystyle-plain]{Example}
\declaretheorem[sibling=para, style=mystyle-plain]{Remark}

% proofs, just as in amsthm but with adapted spacing

\makeatletter
\renewenvironment{proof}[1][\textit{Proof}]{\par
  \pushQED{\qed}%
  \normalfont \topsep.75\paraskip\relax
  \trivlist
  \item[\hskip\labelsep
        \itshape
    #1\@addpunct{.}]\ignorespaces
}{%
  \popQED\endtrivlist\@endpefalse
}
\makeatother

\newcommand\note[1]{\marginpar{{
\begin{flushleft}
\tiny#1
\end{flushleft}
}}}

\renewcommand\labelitemi{-}
%%%%%%%%%%%%%%%%%%%%%%%%%%% The usual stuff%%%%%%%%%%%%%%%%%%%%%%%%%
\newcommand\NN{\mathbb N}
\newcommand\CC{\mathbb C}
\newcommand\QQ{\mathbb Q}
\newcommand\RR{\mathbb R}
\newcommand\ZZ{\mathbb Z}

\newcommand\maps{\longmapsto}
\newcommand\ot{\otimes}
\renewcommand\to{\longrightarrow}
\renewcommand\phi{\varphi}
\newcommand\stack[2]{\genfrac{}{}{0pt}{2}{#1}{#2}}
\newcommand\vspan[1]{\left\langle #1 \right\rangle}

%%%%%%%%%%%%%%%%%%%%%%%%% Specific notation %%%%%%%%%%%%%%%%%%%%%%%%%
\newcommand\A{\mathcal A}
\newcommand\B{\mathcal B}
\newcommand\C{\mathcal C}
\newcommand\D{\mathcal D}
\newcommand\I{\mathcal I}
\renewcommand\S{\mathcal S}
\newcommand\K{\mathcal K}
\newcommand\F{\mathcal F}
\newcommand\G{\mathcal G}
\newcommand\GG{\Gamma}
\renewcommand\L{\mathcal L}
\newcommand\M{\mathcal M}
\renewcommand\O{\mathcal O}
\newcommand\R{\mathcal R}
\newcommand\tphi{\tilde \phi}
\renewcommand\b{\mathbf b}
\renewcommand\k{\Bbbk}
\renewcommand\q{\mathbf{q}}
\newcommand\m{\mathfrak m}
\newcommand\n{\mathfrak n}
\newcommand\g{\mathfrak g}
\newcommand{\Rho}{\mathrm{P}}
\newcommand\opp{\circ}
\newcommand\norm{\mathsf{norm}}
\newcommand\join{\vee}
\newcommand\meet{\wedge}
\newcommand\DC{\mathsf{DC}}
\renewcommand\AA{\mathbb A}
\newcommand\lex{\mathsf{lex}}

\DeclareMathOperator\Sgrp{\mathsf{Sgrp}}
\DeclareMathOperator\Grp{\mathsf{Grp}}

\DeclareMathOperator\Mod{\mathsf{Mod}}
\DeclareMathOperator\mmod{\mathsf{mod}}
\DeclareMathOperator\Hom{\mathsf{Hom}}
\DeclareMathOperator\Ext{\mathsf{Ext}}
\DeclareMathOperator\Tor{\mathsf{Tor}}
\DeclareMathOperator\GrHom{\underline{\mathsf{Hom}}}
\DeclareMathOperator\GrExt{\underline{\mathsf{Ext}}}
\DeclareMathOperator\GrTor{\underline{\mathsf{Tor}}}


\DeclareMathOperator\gr{\mathsf{gr}}
\DeclareMathOperator\im{Im}
\DeclareMathOperator\id{injdim}
\DeclareMathOperator\pd{pdim}
\DeclareMathOperator\ldim{ldim}
\DeclareMathOperator\height{\mathsf{ht}}
\DeclareMathOperator\st{\mathsf{st}}
\DeclareMathOperator\depth{depth}
\DeclareMathOperator\lcd{lcd}
\DeclareMathOperator\Spec{Spec}
\DeclareMathOperator\supp{supp}
\DeclareMathOperator\Id{Id}
\DeclareMathOperator\rank{rk}
\DeclareMathOperator\rk{rk}
\DeclareMathOperator\irr{irr}
\DeclareMathOperator\GKdim{\mathsf{GKdim}}
\DeclareMathOperator\relint{\mathsf{relint}}
\DeclareMathOperator\coker{\mathsf{coker}}
\DeclareMathOperator\tr{\mathsf{tr}}
\DeclareMathOperator\SL{\mathsf{SL}}
\DeclareMathOperator\ev{\mathsf{ev}}


%%%%%%%%%%%%%%%%%%%%%%%%%%%%%%%%%%%%%% TITLES %%%%%%%%%%%%%%%%%%%%%%%%%%%%%%
\title{Quantum toric degeneration of quantum Schubert varieties}
\date{[lrpz3-bis.tex]}
\author{Pablo Zadunaisky}

\begin{document}
\maketitle


%\newpage 
\section{Introduction}

The article is organized as follows: section \ref{graded-connected-algebras} contains
general material on connected $\NN_0^r$-graded algebras. We adapt some well-known 
defintions for connected $\NN_0$-graded algebras to the $\NN_0^r$-graded context. The 
regularity properties of $\NN_0^r$-graded algebras are defined in terms of its category 
of $\ZZ^r$-graded modules, and since an algebra may posses several diferent gradings we 
study how these properties change when the grading changes. In the last subsection we 
prove our main transfer result: if a connected $\NN_0^r$-graded has a filtration by 
finite dimensional graded subspaces then, under mild technical conditions, the original 
algebra inherits the regularity properties of the associated graded ring, see Theorem 
\ref{transfer}. The key result is Proposition \ref{P:ext-ss}, which is an $\NN_0^r$-
graded version of the classical spectral sequence relating $\Ext$-spaces of a filtered 
algebra with those of its associated graded algebra. However the finer grading forces us 
to do some careful bookkeeping, and it turns out that property $\chi$, or rather its 
$\NN_0^{r+1}$-graded analogue, plays a delicate role, see Corollary \ref{ldim}.

In section \ref{qatv} we review the basic theory of affine semigroups and recall some 
results from our previous article \cite{RZ2} on semigroup algebras twisted by a cocycle, 
which are quantum analogues of affine toric varieties. Then we introduce a family of 
algebras with a presentation modeled on that of an affine semigroup; this definition is 
inspired in the classical definition of Hodge algebras and their descendants, namely 
algebras with a straightening law and their quantum versions. We show that these 
algebras have a natural filtration, and give  sufficient conditions for their associated 
graded rings to be quantum affine toric varieties.

Section \ref{q-flag-schubert} contains a quick review of the definitions of quantum flag
varieties and their Schubert subvarieties. We then review Caldero's results on 
cannonical bases for these varieties found in \cite{C}, use them to show that Schubert 
varieties have quantum toric degenerations, and apply the results from the previous
section to deduce some of their homological regularity properties in Theorem 
\ref{Schubert-deg}. 

We finish with a short section \ref{richardson}, where we revisit the results of our 
previous article \cite{RZ} in the context of this article.





%\newpage 
\section{Degeneration of graded connected algebras}
\label{graded-connected-algebras}
Let $r$ be a positive integer. Throughout this article $A$ denotes a noetherian connected
$\NN_0^{r}$-graded $\k$-algebra. Here connected means that the homogeneous component of 
$A$ of degree $(0, \ldots, 0)$ is isomorphic to $\k$ as a ring, so the 
ideal generated by all homogeneous elements of non-zero degree is the unique maximal 
graded ideal of $A$; we denote this ideal by $\m$. Clearly $A/\m \cong \k$ as vector 
spaces, and we will use this isomorphism to give $\k$ the structure of an $A$-bimodule.

\subsection*{Graded modules}
%\label{graded-connected-algebras}

\paragraph
\label{grmod-generalities}
We denote by $\ZZ^{r}\Mod A$ the category of $\ZZ^{r}$-graded $A$-modules with 
homogeneous morphisms of degree $0$. We review some general properties of this category; 
the reader is referred to \cite{NV}*{chapter 2} for proofs and details.

 The category 
$\ZZ^r\Mod A$ has enough projectives and injectives, so we may speak of the graded 
projective and injective dimensions of an object $M$, which we denote by $\pd_A^{\ZZ^r} 
M$ and  $\id^{\ZZ^r}_A M$, respectively. We denote by $\ZZ^{r}\mmod A$ the subcategory 
of finitely generated $\ZZ^{r}$-graded $A$-modules. Since $A$ is noetherian $\ZZ^r 
\mmod A$ is an abelian category with enough projectives. 

For every object $M$ of $\ZZ^{r}\Mod A$ and every $\xi \in \ZZ^{r}$ we denote by $M_\xi$ 
the homogeneous component of $M$ of degree $\xi$. Also, we denote by $M[\xi]$ the object
of $\ZZ^{r}\Mod A$ with the same underlying $A$-module as $M$ and with homogeneous 
components $M[\xi]_{\zeta} = M_{\zeta + \xi}$ for all $\zeta \in \ZZ^{r}$. If
$f: N \to M$ is a morphism in $\ZZ^{r}\Mod A$ then the same function defines a morphism
$f[\xi]:N[\xi] \to M[\xi]$. In this way we get an endofunctor $[\xi]: \ZZ^{r}\Mod A \to
\ZZ^{r}\Mod A$, called the $\xi$-suspension functor; it is an autoequivalence, with 
inverse $[-\xi]$.

%\paragraph
%\label{minimal-projective-resolution}
%If $M$ is a finitely generated $\ZZ^r$-graded $A$-module then $V = \k \ot_A M$ is a 
%finite dimensional graded vector space. Fix $m_1, \ldots, m_n \in M$ such that their 
%classes modulo $\m M$ form a basis of $V$; then the homogeneous $A$-linear map $p_M: A
%\ot_\k V \to M$ defined by $p_M(1 \ot_\k \overline{m_i})= m_i$ is a minimal projective
%cover of $M$ in $\ZZ^r\Mod A$. By taking a projective cover for $\ker p_M$ and iterating
%the process we obtain a minimal projective resolution of $M$ in $\ZZ^r\Mod A$.

Given $N, M$ objects of $\ZZ^{r}\Mod A$ we set 
\begin{align*} 
  \GrHom_A(N, M) = \bigoplus_{\xi \in \ZZ^{r}} \Hom_{\ZZ^{r}\Mod A}(N,M[\xi]).  
\end{align*} 
This is a $\ZZ^{r}$-graded vector space, with its component of degree $\xi \in \ZZ^{r}$
equal to the space of homogeneous $A$-linear maps of degree $\xi$ from $N$ to $M$. For
every $i \geq 0$ we denote by $\GrExt_A^i$ the $i$-th right derived functor of 
$\GrHom_A$. We point out that 
\[
  \GrHom_A(N[\xi], M) \cong \GrHom_A(N,M[-\xi]) \cong \GrHom_A(N,M)[-\xi],
\]
as $\ZZ^r$-graded vector spaces, and that these isomorphisms induce analogous ones for 
the coresponding right derived functors.

\paragraph
\label{ext-change-of-grading}
Given $\xi = (\xi_1, \ldots, \xi_r) \in \ZZ^r$ we set $|\xi| = \xi_1 + \cdots +
\xi_r$. Given a $\ZZ^{r}$-graded vector space $V$, we denote by $|V|$ the $\ZZ$-graded
vector space whose $n$-th homogeneous component is 
\begin{align*}
  |V|_n = \bigoplus_{|\xi| = n} V_\xi.
\end{align*}
In particular $|A|$ is a connected $\NN_0$-graded algebra. Also, if $M$ is a 
$\ZZ^r$-graded $A$-module then $|M|$ is a $\ZZ$-graded $|A|$-module, and this 
assignation is functorial. Since $A$ is noetherian, \cite{RZ2}*{Proposition 1.3.7} 
implies that for every $i \geq 0$ and any pair of $\ZZ^r$-graded modules $N, M$, with $N$
finitely generated, there is an isomorphism of $\ZZ$-graded modules
\[
  |\GrExt^i_A(N,M)| \cong \GrExt^i_{|A|}(|N|,|M|),
\]
natural in both variables. 

%\newpage 
\subsection*{Homological regularity properties}
%\label{homological-regularity}

In this subsection we discuss some homological properties a connected $\NN_0^r$-graded
algebra may posses. Most of the material found in this section is standard for connected
$\NN_0$-graded algebras.

\paragraph
\label{chi}
Let $M$ be a $\ZZ^r$-graded $A$-module. We say that $\chi(M)$ holds if for each $i \geq 
0$ the graded vector space $\GrExt_A^i(\k, M)$ is finite dimensional, and say that the
algebra $A$ has property $\chi$ if $\chi(M)$ holds for every finitely generated 
$\ZZ^r$-graded $A$-module $M$. Property $\chi$ was originally introduced in 
\cite{AZ}*{section 3} and plays a fundamental role in noncommutative algebraic geometry.

\paragraph
\label{torsion-functor}
Associated to $A$ and $\m$ there is a \emph{torsion functor} 
\begin{align*} 
  \GG_\m: \ZZ^{r}\Mod A &\to \ZZ^{r}\Mod A \\ 
  M &\longmapsto \{x \in M \mid \m^n x = 0 \mbox{ for } n \gg 0\}, 
\end{align*} 
which acts on morphisms by restriction and correstriction. The torsion functor is left
exact, and for each $i \geq 0$ its $i$-th right derived functor is denoted by $H_\m^i$ 
and called the \emph{$i$-th local cohomology functor} of $A$. 

There exists a natural isomorphism
\begin{align*} 
\GG_\m \cong \varinjlim_n \GrHom_A(A/\m^n, -) 
\end{align*}
which by standard homological algebra extends to natural isomorphisms
\begin{align*} 
H^i_\m \cong \varinjlim_n \GrExt^i_A(A/\m^n, -) 
\end{align*} 
for all $i \geq 1$. 

We denote by $A^\opp$ the opposite algebra of $A$, which is also
a connected $\NN_0^r$-graded algebra, and by $\m^\opp$ its maximal graded ideal. We write
$\GG_{\m^\opp}$ and $H^i_{\m^\opp}$ for the corresponding torsion and local cohomology
functors, respectively.

\paragraph
\label{depth-and-ldim}
Given an object $M$ of $\ZZ^{r}\Mod A$, its \emph{depth} and \emph{local dimension}
are defined as
\begin{align*} 
  \depth_\m M &= \inf \{i \in \NN_0 \mid \GrExt^i_A(\k, M) \neq 0\}, \\ 
  \ldim_\m M &= \sup \{i \in \NN_0 \mid H^i_\m(M) \neq 0\},
\end{align*} 
respectively. The \emph{local cohomological dimension} of $A$, denoted by $\lcd_\m A$, is
the supremum of the $\ldim_\m M$ with $M$ finitely generated.

\paragraph
\label{AS-reg}
The following definition is taken from \cite{RZ2}*{Definition 2.1.1}. It is an 
$\NN_0^r$-graded analogue of the definition of the AS-Cohen-Macaulay, AS-Gorenstein and 
AS-regular properties for connected $\NN_0$-graded algebras found in the literature, see 
for example the introduction to \cite{JZ}.
\begin{Definition*}
Let $A$ be a connected noetherian $\NN_0^{r}$-graded algebra.  
\begin{enumerate} 
\item $A$ is called \emph{AS-Cohen-Macaulay} if there exists $n \in \NN_0$ such that
  $H^i_{\m}(A) = 0$ and $H_{\m^\opp}^i(A) = 0$ for all $i \neq n$.

\item $A$ is called \emph{left AS-Gorenstein} if it has finite graded injective dimension
  $n$ and there exists $\ell \in \ZZ^{r}$, called the \emph{Gorenstein shift} of
  $A$, such that
\[ 
\GrExt_{A}^i(\k,A) \cong 
  \begin{cases} 
    \k[\ell] & \mbox{for } i = n, 
    \\ 0 & \mbox{for } i \neq n
  \end{cases} 
\] 
as $\ZZ^{r}$-graded $A^\opp$-modules. We say $A$ is \emph{right} AS-Gorenstein if 
$A^\opp$ is left AS-Gorenstein. Finally $A$ is \emph{AS-Gorenstein} if $A$ and $A^\opp$ 
are left AS-Gorenstein, with the same injective dimensions and Gorenstein shifts.

\item $A$ is called \emph{AS-regular} if it is AS-Gorenstein, and its left and right 
graded global dimensions are finite and equal.
\end{enumerate} 
\end{Definition*} 

\paragraph
\label{invariance-by-grading}
The properties discussed in paragraphs \ref{chi} to \ref{AS-reg} are defined in terms of
the category of $\ZZ^r$-graded $A$-modules. In \ref{ext-change-of-grading} we defined 
the algebra $|A|$, which is equal to $A$ as algebra but is endowed with an connected 
$\NN_0$-grading induced by the grading on $A$ and the group morphism $|\cdot|$. The 
maximal graded ideals of $A$ and $|A|$ coincide as vector spaces, so we may ask whether 
the fact that $A$ has property $\chi$, or finite local dimension, or the 
AS-Cohen-Macaulay property, etc., implies that $|A|$ has the corresponding property.

Let us say that a property $P$ does not depend on the grading of $A$ if the following 
holds: for every connected $\NN_0^t$-graded algebra $B$ with maximal ideal $\n$, which 
is isomorphic to $A$ as algebra through an isomorphism that sends $\n$ to $\m$, $A$ has 
property $P$ if and only if $B$ has property $P$. As shown in 
\cite{RZ2}*{Corollary 1.3.9}, the local dimension of $A$ does not depend on the grading 
of $A$. An analogous result is proved in \cite{RZ2}*{Remark 2.1.7} for the properties 
defined in \ref{AS-reg}. However, the situation is more delicate for property $\chi$. 
The following lemma shows that property $\chi$ is independent of the grading of $A$ 
under the hypothesis that $A$ has finite local dimension; we do not know whether this 
hypothesis can be eliminated.

\begin{Lemma*}
Suppose $\lcd_\m A < \infty$. Then the algebra $A$ has property $\chi$ if and only if
$\chi(A)$ holds.
\end{Lemma*}
\begin{proof}
If $A$ has property $\chi$ then clearly $\chi(A)$ holds. To prove the opposite
implication, assume $\chi(A)$ holds. Recall from \ref{ext-change-of-grading} that for 
every $\ZZ^r$-graded $A$-module $M$ and for every $i \geq 0$ there exists a graded 
vector space isomorphism
\[
  |\GrExt^i_A(\k, M)|
    \cong \GrExt_{|A|}^i(|\k|, |M|),
\]
so $\chi(M)$ holds if and only if $\chi(|M|)$ holds; in view of this, the hypothesis 
implies $\chi(|A|)$ holds. 

Since $\ldim_{|\m|} |A| = \ldim_\m A < \infty$, we may apply 
\cite{RZ}*{Proposition 2.2.6} and conclude that $|A|$ has property $\chi$. From this
it follows that $\chi(|M|)$ holds for every $\ZZ^r$-graded $A$-module $M$, and hence so 
does $\chi(M)$.
\end{proof}

\paragraph
\label{chi-and-local-cohomology}
We finish this subsection with a technical result on the relation between property 
$\chi$ and local cohomology. For every $n \in \NN_0$ we denote by $A_{\geq n}$ the ideal
generated by all homogeneous elements of degree $\xi$ with $|\xi| \geq n$. 

\begin{Lemma*}
Let $M$ be a finitely generated $\ZZ^r$-graded $A$-module such that $\chi(M)$ 
holds. Then for every $i \geq 0$ and every $t \in \ZZ$ there exists $n_0 \in \ZZ$ such 
that 
\begin{align*}
  \GrExt^i_A(A/A_{\geq n}, M)_\xi &\cong H^i_\m(M)_\xi
\end{align*}
for all $n \geq n_0$ and all $\xi \in \ZZ^r$ such that $|\xi| \geq t$.
\end{Lemma*}
\begin{proof}
Clearly $A_{\geq n} \subset \m^n$. Since $\m$ is finitely generated, say by elements 
$x_1, \ldots, x_n$ with degrees $\xi^1, \ldots, \xi^n$, setting $d = \max \{|\xi^i| : 
i = 1, \ldots, n\}$ we obtain $\m^n \subset A_{\geq dn}$. In other words 
$(A_{\geq n})_{n \geq 0}$ and $\{\m^n\}_{n \geq 0}$ are cofinal, so the proof of 
\cite{BS}*{Proposition 3.1.1} easily adapts to show that for every $i \geq 0$ there 
exist natural isomorphisms
\begin{align*}
  \varinjlim_n \GrExt^i_A(A/A_{\geq n}, -) \cong \varinjlim_n \GrExt^i_A(A/\m^n, -) \cong
  H^i_\m.
\end{align*} 
The statement of the lemma will follow if we show that for all $\xi$ as in the
statement, the homogeneous component of degree $\xi$ of the natural map
\begin{align*}
  \pi^n: \GrExt^i_A(A/A_{\geq n}, M) \to H^i_\m(M)
\end{align*}
is an isomorphism for $n \gg 0$.

Fixing $t$ as in the statement, \cite{AZ}*{Propositions 3.5 (1)} implies that
the natural map $\pi^n_d: \GrExt^i_{|A|}(|A|/|A_{\geq n}|, |M|)_d \to H^i_{|\m|}(|M|)_d$
is an isomorphism for all $d \geq t$ if $n$ is large enough. Now by 
\cite{RZ2}*{Propositions 1.3.7 and 1.3.8} there exist isomorphisms
\begin{align*}
\GrExt^i_{|A|}(|A|/|A_{\geq n}|, |M|)_d 
&\cong \bigoplus_{|\xi| = d}\GrExt^i_{A}(A/A_{\geq n}, M)_\xi;\\
H^i_{|\m|}(|M|)_d 
  &\cong \bigoplus_{|\xi| = d} H^i_\m(M)_\xi.
\end{align*}
Since the assignation $M \mapsto |M|$ is functorial, we also get that $\pi^n_d = 
\bigoplus_{|\xi| = d} \pi^n_\xi$. Thus for all $\xi$ such
that $|\xi| \geq t$, the map $\pi^n_\xi$ is an isomorphism if $n$ is large enough.
\end{proof}


%\newpage 
\subsection*{Transfer of regularity properties by degeneration}
\label{transfer-of-regularity}
In this subsection we prove that if $A$ has a filtration compatible with its grading, and
the associated graded algebra has property $\chi$, then the regularity properties 
discussed in the previous subsection transfer from $\gr A$ to $A$. 
All undefined terms regarding filtrations can be found in \cite{VO}*{chapter I}.

\paragraph
\label{regular-filtration}
Recall that $A$ denotes a noetherian $\NN_0^r$-graded algebra.
The general setup for the subsection is as follows: we assume that $A$ has a 
\emph{connected} filtration, that is an exhaustive filtration $\F = \{F_nA\}_{n \geq 
0}$, with $\k = F_0A \subset F_1A \subset \cdots \subset F_nA \subset \cdots 
\bigcup_{n \geq 0} F_nA = A$, such that each layer $F_nA$ is a finite dimensional graded 
vector space, and $F_nA \cdot F_m A \subset F_{n+m}A$ for all $n, m \in \NN_0$. For each 
$\xi \in \NN_0^r$ the homogeneous component $A_\xi$ has an induced
filtration $\{F_nA_\xi\}_{n \geq 0}$, where $F_nA_\xi = F_nA \cap A_\xi$. Since
$A_\xi$ is finite dimensional this filtration is finite, so the associated graded ring 
$\gr A$ is a connected and locally finite $\NN_0^{r+1}$-graded algebra. 

Given any $\ZZ^r$-graded $A$-module $M$ with a filtration whose layers are 
$\ZZ^r$-graded subspaces, we can construct the $\ZZ^{r+1}$-graded $\gr A$-module $\gr M$.
If $M$ is any $\ZZ^r$-graded $A$-module then it can be endowed with such a filtration as 
follows: fix a graded subspace $N \subset M$ that generates 
$M$ over $A$, and for each $n \geq 0$ set $F_nM = (F_nA) N$. Any such filtration is 
called \emph{standard}, and is an exhaustive and discrete filtration by graded subspaces.
If $M$ is finitely generated and $N$ is finite dimensional then the layers of this 
filtration are also finite dimensional. 



\paragraph
\label{P:ext-ss} 
The main tool used to transfer homological information from $\gr A$ to $A$ is a 
spectral sequence that we associate to any pair of $\ZZ^r$-graded $A$-modules $N, 
M$, which converges to $\GrExt^i_A(N,M)$ and whose first page consists of the 
homogeneous components of $\GrExt^i_A(\gr N, \gr M)$. The proof is straightforward, but 
relies on several graded analogues of classical constructions for filtered rings. These
constructions can be found in \cite{VO}*{Chapter I} and \cite{MR}*{Section 7.6}, and the
proofs found in the references easily adapt to the graded context, so we use them
without further comment.

In order to keep track of the extra component in the grading when passing to associated 
graded objects, we make a slight abuse of notation and given a $\ZZ^{r+1}$-graded vector 
space $V$, we denote by $V_{(\xi, p)}$ its homogeneous component of degree $(\xi_1, 
\ldots, \xi_r, p)$.

\begin{Proposition*} 
Suppose that $\gr A$ is noetherian.
Let $M, N$ be filtered $\ZZ^r$-graded $A$-modules, with $N$ finitely generated, and
suppose that the filtration on $N$ is standard and the filtration on $M$ is discrete. Then
for every $\xi \in \ZZ^r$ there exists a convergent spectral sequence
\begin{align*} 
  E(N,M)_\xi: E_{p,q}^1 = \GrExt_{\gr A}^{-p-q}(\gr N, \gr M)_{(\xi,p)}
    &\Rightarrow \GrExt_A^{-p-q}(N,M)_\xi 
    &p,q \in \ZZ, 
\end{align*} 
such that the filtration of the vector spaces on the right hand side is finite.
\end{Proposition*}
\begin{proof}
By the $\ZZ^r$-graded version of \cite{MR}*{Theorem 6.17}, there exists a projective
resolution $P^\bullet \to N$ by filtered projective $\ZZ^r$-graded $A$-modules with
filtered differentials, such that the associated graded complex $\gr P^\bullet \to \gr N$
is a $\ZZ^{r+1}$-graded projective resolution of the $\gr A$-module $\gr N$.
Using the filtration for the $\GrHom$ spaces defined in \cite{VO}*{section I.2}, the
complex $\GrHom_A(P^\bullet, M)$ is a graded complex with a filtration by graded
subcomplexes, whose differentials are filtered maps.

If we fix $\xi \in \ZZ^r$ the homogeneous component $\GrHom_A(P^\bullet, M)_\xi$ is a 
complex of filtered finite dimensional vector spaces. By \cite{W}*{5.5.1.2} there exists 
a spectral sequence with page one equal to
\begin{align*} 
  E_{p,q}^1 
  &= H_{p+q}\left(\frac{F_p \GrHom_A(P^\bullet,M)_\xi}
      {F_{p-1}  \GrHom_A(P^\bullet,M)_\xi}\right) 
    & p,q \in \ZZ,
\end{align*} 
that converges to   
\begin{align*} 
  H_{p+q}(\GrHom_{A}(P^\bullet,M)_\xi) \cong \GrExt^{-p-q}_A(N,M)_\xi.
\end{align*}
This last space is finite dimensional, and hence the filtration on it is finite. Thus 
we only need to prove that for each $p,q \in \ZZ$ there exists an isomorphism
\begin{align*} 
  H_{p+q}\left(\frac{F_p \GrHom_A(P^\bullet,M)_\xi}{F_{p-1} \GrHom_A(P^\bullet,M)_\xi}
  \right) 
    &\cong \GrExt_{\gr A}^{-p-q}(\gr N, \gr M)_{(\xi,p)}.
\end{align*} 

By \cite{VO}*{Lemma 6.4}, there exists an isomorphism of complexes $\phi(P^\bullet, M):
\gr (\GrHom_A(P^\bullet, M)) \to \GrHom_{\gr A}(\gr P^\bullet, \gr M)$, which is defined
explicitly in the reference. Direct inspection shows that the map $\phi(P^\bullet, M)$ 
is homogeneous, so looking at its component of degree $\xi$ we obtain an isomorphism
\begin{align*}
\frac{F_p \GrHom_A(P^\bullet,M)_\xi}{F_{p-1} \GrHom_A(P^\bullet,M)}_\xi \cong 
\GrHom_{\gr A}(\gr P^\bullet, \gr M)_{(\xi,p)}.
\end{align*}
Since $\gr P^\bullet$ is a $\ZZ^{r+1}$-graded projective resolution of $\gr N$, we obtain
the desired isomorphism by applying $H_{p+q}$ to both sides of the isomorphism.
\end{proof}


\begin{Corollary}
\label{C:consequences-of-ext-ss}
Suppose $\gr A$ is noetherian, and let $M, N$ be as in Proposition \ref{P:ext-ss}.
\begin{enumerate}
  \item \label{I:dims}  For each $i \geq 0$ and each $\xi \in \ZZ^{r}$
    \[
      \dim_\k \GrExt^i_A(N,M)_\xi \leq \sum_{p=-\infty}^\infty \dim_\k \GrExt^i_{\gr
      A}(\gr N, \gr M)_{(\xi, p)}.
    \]
  
  \item \label{I:pd-id} $\pd_A^{\ZZ^r} N \leq \pd_{\gr A}^{\ZZ^{r+1}} \gr N$ and  
    $\id_A^{\ZZ^r} M \leq \id_{\gr A}^{\ZZ^{r+1}} \gr M$. 
  
  \item \label{I:chi} If $\chi(\gr M)$ holds then $\chi(M)$ holds.
\end{enumerate}
\end{Corollary}

\paragraph
\label{ldim} 
We now prove a result that relates the local cohomology of a $\ZZ^r$-graded $A$-module 
$M$ with that of its associated graded module $\gr M$. We will do this by combining 
Proposition \ref{P:ext-ss} with Lemma \ref{chi-and-local-cohomology} and the formalism
of the change of grading functors introduced in \cite{RZ}*{Section 1.3}. We recall the
relevant details. Given a group morphism $\phi: \ZZ^r \to \ZZ^t$, with $t \in \NN_0$, 
there exists a functor $\phi_!: \ZZ^r\Mod \k \to \ZZ^t \Mod \k$ that sends a 
$\ZZ^r$-graded vector space $V$ to the $\ZZ^t$-graded vector space $\phi_!(V)$, whose 
homogeneous component of degree $\zeta \in \ZZ^t$ is
\[
  V_\zeta = \bigoplus_{\phi(\xi) = \zeta} V_\xi.
\]
Notice that $\phi_!$ does not change the underlying vector space, only its grading.
If $R$ is a $\ZZ^r$-graded algebra then $\phi_!(R)$ is a $\ZZ^t$-graded algebra, and if 
$M$ is a $\ZZ^r$-graded $R$-module then $\phi_!(M)$ is a $\ZZ^t$-graded 
$\phi_!(R)$-module with the same underlying $R$-module structure as $M$.

\begin{Corollary*}
Assume $\gr A$ is noetherian. Let $M$ be a filtered $\ZZ^r$-graded $A$-module with a
discrete filtration, and assume $\chi(\gr M)$ holds. Then for each $i \geq 0$ and each 
$\xi \in \ZZ^r$ 
  \[
    \dim_\k H^i_\m(M)_\xi \leq \sum_{p=- \infty}^\infty \dim_\k H^i_{\gr \m}(\gr
    M)_{(\xi, p)}.
  \]
\end{Corollary*}
\begin{proof}
Let $\pi: \ZZ^{r+1} \to \ZZ^r$ be the projection to the first $r$-coordinates, and set $B
= \pi_!(\gr A)$, so for every $\xi \in \ZZ^r$
\begin{align*}
B_\xi = \bigoplus_{p \in \ZZ} \gr A_{(\xi, p)} = \gr (A_\xi).
\end{align*}
Thus $B$ is a connected $\NN_0^r$-graded algebra. Set $\n = \pi_!(\gr \m)$, which is
the maximal graded ideal of $B$, and set $\tilde M = \pi_!(\gr M)$.

By \cite{RZ2}*{Proposition 1.3.7} $\chi(\gr M)$ implies
$\chi(\tilde M)$ and by item \ref{I:chi} of Corollary \ref{C:consequences-of-ext-ss} it 
also implies $\chi(M)$, so we may apply Lemma \ref{chi-and-local-cohomology} to both $M$ 
and $\tilde M$, and deduce that for all $\xi \in \ZZ^r$ and all $i \geq 0$, the natural 
maps 
\begin{align*}
  \GrExt^i_A(A/A_{\geq n}, M)_\xi &\to H^i_\m(M)_\xi;\\
  \GrExt^i_B(B/B_{\geq n}, \tilde M)_\xi &\to H^i_\n( \tilde M)_\xi;
\end{align*}
are isomorphisms for $n \gg 0$.

Combining this with \cite{RZ2}*{Propositions 1.3.7 and 1.3.8}, we obtain a chain of
isomorphisms
\begin{align*}
\bigoplus_{p \in \ZZ} \GrExt^i_{\gr A}\left(\frac{\gr A}{\gr (A_{\geq n})}, \gr M 
  \right)_{(\xi, p)} 
  &\cong \GrExt^i_B \left(B/B_{\geq n}, \tilde M \right)_\xi  \\
  &\cong H^i_\n(\tilde M)_\xi 
  \cong \bigoplus_{p \in \ZZ} H^i_{\gr \m}(\gr M)_{(\xi, p)} 
\end{align*}
By definition $\gr(A/A_{\geq n}) \cong \gr A / \gr (A_{\geq n})$, so applying item 
\ref{I:dims} of Corollary \ref{C:consequences-of-ext-ss} and taking $n \gg 0$ we obtain
\begin{align*}
\dim_\k H^i_\m(M)_\xi 
  &= \dim_\k \GrExt^i_A(A/A_{\geq n}, M)_\xi \\
  &\leq \sum_{p = -\infty}^\infty \dim_\k
  \GrExt^i_{\gr A}(\gr A / \gr (A_{\geq n}), M)_{(\xi, p)} \\
  &= \sum_{p = -\infty}^\infty \dim_\k H^i_{\gr \m}(\gr M)_{(\xi, p)}.
\end{align*}
\end{proof}

%\paragraph

We are now ready to prove the main result of this section. 
\begin{Theorem}
\label{transfer}
Suppose $A$ is a connected $\NN_0^r$-graded algebra endowed with a connected 
filtration, and that $\gr A$ is noetherian and has property $\chi$. Then:
\begin{enumerate}
\item $A$ is noetherian and has property $\chi$;
\item $\lcd_\m A \leq \lcd_{\gr \m} \gr A$;
\item If $\gr A$ is AS-Cohen-Macaulay, AS-Gorenstein or AS-Regular, so is $A$.
\end{enumerate}
\end{Theorem}
\begin{proof}
The fact that $A$ is noetherian is classical, see for example \cite{MR}*{1.6.9}. If $M$ 
is any finitely generated $\ZZ^r$-graded $A$-module, we may filter it using the procedure
described in \ref{regular-filtration}. Since $\chi(\gr M)$ holds by hypothesis, item 
\ref{I:chi} of Corollary \ref{C:consequences-of-ext-ss} implies $\chi(M)$ holds, which 
proves that $A$ has property $\chi$. 

Item 2 follows from Corollary \ref{ldim}, as does the fact that if $\gr A$ is
AS-Cohen-Macaulay so is $A$. If $\gr A$ is AS-Gorenstein of injective dimension $n$ and
Gorenstein shift $(\xi, p) \in \ZZ^{r+1}$, the algebra $A$ has injective dimension at 
most $n$ by item \ref{I:pd-id} of Corollary \ref{C:consequences-of-ext-ss}. Also, for 
each $\zeta \in \ZZ^r$ the spectral sequence $E(\k, A)_\zeta$ degenerates at page $1$: 
it is zero if $\zeta \neq -\xi$, while for $\zeta = - \xi$ there is a single, one 
dimensional non-zero entry in the diagonal $-p-q = n$. Hence we obtain vector space
isomorphisms
\begin{align*}
  \GrExt^i_A(\k, A)_\zeta 
    &\cong  \begin{cases}
      \k & \mbox{ if  $\zeta = -\xi$ and $i = n$;} \\
      0 & \mbox{otherwise.}
    \end{cases}
\end{align*}
Any such isomorphism is also $A^\opp$-linear, so $A$ is left AS-Gorenstein of injective 
dimension $n$ and Gorenstein shift $\xi$. The same proof applies to show that $A$ is 
right AS-Gorenstein with the same injective dimension and Gorenstein shift, so $A$ is 
AS-Gorenstein. 

Finally, assume that $\gr A$ is AS-regular. Then $A$ is AS-Gorenstein, and item
\ref{I:pd-id} of Corollary \ref{C:consequences-of-ext-ss} implies that it has finite
left and right global dimensions. These dimensions are equal to the left and right 
projective dimensions of $\k$ as $A$-module \cite{RZ2}*{Lemma 2.1.5}, which in turn equal
the left and right injective dimensions of $A$ and hence coincide, so $A$ is AS-regular.
\end{proof}


%\newpage 
\section{Quantum affine toric degenerations}
\label{qatv}
Recall that we denote by $A$ a noetherian connected $\NN_0^r$-graded algebra.
In this section we recall the homological properties of quantum affine toric varieties
proved in \cite{RZ2}, and give necessary and sufficient conditions for the algebra $A$
to have a quantum toric degeneration, that is a connected filtration such that the
associated graded ring is a quantum affine toric variety.

\subsection*{Quantum affine toric varieties}

\paragraph
\label{affine-semigroups}
Recall that an \emph{affine semigroup} is a semigroup isomorphic to a finitely generated 
submonoid (i.e. a subgroup containing the neutral element) of $(\ZZ^t, +)$ for some $t 
\geq 1$. Let $S$ be an affine semigroup and let $G(S) \subset \ZZ^t$ be the group 
generated by $S$. It is an abelian group without torsion, and since $S$ is finitely 
generated so is $G(S)$; the rank of $S$ is defined as the rank of $G(S)$. Whenever we 
work with an affine semigroup $S$, we will assume it comes with a fixed embedding $S 
\hookrightarrow \ZZ^t$, and identify $S$ with its image through this map. Since $\ZZ^t$ 
is a totally ordered group with the lexicographic order $<_{\lex}$, the semigroup $S$ 
inherits a total order.

We say that $S$ is \emph{normal} if, whenever there exist $n \in \NN$ and $z \in G(S)$
such that $n z \in S$, then $z \in S$. By \cite{BH}*{Proposition 6.1.2}, if $S$
is normal then it is the set of integral points of a rational convex cone of the 
$\RR$-vector space generated by $G(S)$. Its \emph{interior ideal} $\relint S$ is the set 
of elements of $S$ lying in the topological interior of that convex cone.

An affine semigroup is said to be positive if it is isomorphic to a subsemigroup of
$\NN_0^t$ for some $t \geq 1$. If $S$ is positive then by \cite{MS}*{Corollary 7.23} we 
may assume that $S \subset \NN_0^t$; notice that positive semigroups are called 
\emph{pointed} semigroups in the reference. By \cite{MS}*{Proposition 7.15}, a positive 
affine semigroup $S$ is generated by the set of elements $s \in S$ such that whenever $s 
= s'+s''$, then $s' = 0$ or $s'' = 0$. This is a finite set, called the \emph{Hilbert 
basis} of $S$. 

\paragraph
\label{congruence}
Let $S$ be a positive affine semigroup and let $\{s_1, \ldots, s_n\}$ be its Hilbert 
Basis. We denote by $\pi: \NN_0^n \to S$ the semigroup morphism defined by the 
assignation $e_i \mapsto s_i$ for each $1 \leq i \leq n$, where $e_i$ is the $i$-th 
vector in the canonical basis of $\NN_0^n$. We denote by $L(S) \subset \NN_0^n \times 
\NN_0^n$ the set of pairs $(p,p')$ such that $\pi(p) = \pi(p')$. The set $L(S)$ is a 
congruence, i.e. it is an equivalence relation on $S$ and $L(S) + (\xi,\xi) \subset L(S)$
for any $\xi \in \NN_0^n$. The quotient $\NN^n_0/L(S)$ has a natural semigroup structure 
induced by the addition of $\NN_0^n$, and is isomorphic to $S$ \cite{RGS}*{Theorem 1.2}. 
By the proof of R\'edei's theorem found in \cite{RGS}*{Theorem 5.12} there exists a 
finite set $P =  \{(p_1, p'_1), \ldots, (p_m,p'_m) \} \subset L(S)$, uniquely determined 
by the lexicographic order, such that $p_i <_\lex p'_i$ and $L(S)$ is the smallest 
congruence containing $P$.

\paragraph
\label{twisted-semigroup-algebras}
A $2$-cocycle over $S$ is a function $\alpha: S \times S \to \k^\times$ such that $\alpha
(s,s') \alpha(s+s',s'') = \alpha(s,s'+s'')\alpha(s',s'')$ for all $s,s',s'' \in S$. 
Given a $2$-cocycle $\alpha$ over $S$, the \emph{$\alpha$-twisted semigroup algebra} 
$\k^\alpha[S]$ is the associative $\k$-algebra whose underlying vector space has basis 
$\{X^s \mid s \in S\}$ and whose product over these generators is given by 
$X^s X^t = \alpha(s,t)X^{s+t}$. This is a noncommutative deformation of the classical 
semigroup algebra $\k[S]$. 

Let $u \geq 1$ and let $\psi: S \to \NN_0^u$ be a semigroup morphism such that
$|\psi(s_i)| > 0$ for all $1 \leq i \leq n$. We can endow the algebra $\k^\alpha[S]$ 
with a connected $\NN_0^u$-grading by setting $\deg X^s = \psi(s)$ for all $s \in S$. 
Any connected grading on $\k^\alpha[S]$ such that the elements $X^s$ are homogeneous 
and of non-zero degree arises in this way. 

\begin{Definition}
We say that the algebra $A$ is a \emph{quantum affine toric variety} if it is
isomorphic as graded algebra to a twisted semigroup algebra $\k^\alpha[S]$ endowed with 
the grading induced by a semigroup morphism $\psi: S \to \NN_0^r$ with $|\psi(s_i)| > 0$
for all $i$. We call $S$ the \emph{underlying semigroup} of the quantum affine toric 
variety.
\end{Definition}

\paragraph
\label{properties-of-qatv}
The following is a summary of the results proved in \cite{RZ2}*{section 3.2} about 
the homological properties of quantum affine semigroup algebras.
\begin{Proposition*}
Let $S$ be a positive affine semigroup and let $R$ be a quantum affine toric variety 
with underlying group $S$. Then the following hold.
\begin{enumerate}[label=(\alph*)]
  \item $R$ is noetherian and integral.
  \item $R$ has property $\chi$ and finite local dimension equal to the rank of $S$.
  \item Suppose $S$ is normal. Then $R$ is AS-Cohen-Macaulay and a maximal order in its 
    division ring of fractions. Furthermore, $R$ is AS-Gorenstein if and only if
    there exists $s \in S$ such that $\relint S = s + S$.
\end{enumerate}
\end{Proposition*}
\begin{proof}
This is proved in \cite{RZ2}*{section 3.2} in the case where the grading morphism $\psi$ 
is the inclusion of $S$ in $\NN_0^t$. The proposition follows from the fact that all the 
properties mentioned in it are independent of the grading, see paragraph
\ref{invariance-by-grading}.
\end{proof}


%\paragraph
%Before studying algebras with a toric degeneration we prove that twisted semigroup
%algebras can be presented in an obvious way. Fix a possitive affine semigroup $S$ of rank
%$r$, and consider $\k^\alpha[S]$ as a $\ZZ^r$-graded algebra in the obvious way, i.e.
%taking $t = r$ and $\psi = \Id$ in \ref{twisted-semigroup-algebras}. Recall that $\pi:
%\ZZ^n \to \ZZ^r$ is the morphism that sends the canonical vectors of $\ZZ^n$ to the
%elements of the Hilbert basis of $S$, which we denote by $s_1, \ldots, s_n$. Recall also
%the definition of the semigroup $L(S)$ in \ref{hilbert-basis}, and denote by $\{(p_1,
%p'_1), \ldots, (p_m, p'_m)\}$ its Hilbert basis.
%
%Set $F = \k \langle Y_i \mid i = 1, \ldots, n \rangle$ and let $\phi: F \to \k^\alpha[S]$
%be the morphism that maps $Y_i$ to $X^{s_i}$. We fix an $\NN_0^r$-grading on $F$ by setting
%the degree of $Y_i$ equal to $s_i$, so $\phi$ is an $\NN_0^r$-graded algebra morphism. Given
%$\xi = (\xi_1, \ldots, \xi_n) \in \NN_0^n$ we write $Y^\xi$ for the monomial $Y_1^{\xi_1}
%Y_2^{\xi_2} \ldots Y_n^{\xi_n} \in F$. Notice that $\deg Y^\xi = \pi(\xi)$.
%
%For $1 \leq i < j \leq n$ the element 
%\begin{align*} 
%   C_{i,j} &= \alpha(s_i,s_j) Y_j Y_i - \alpha_\q(s_j,s_i)Y_i Y_j 
%\end{align*} 
%is in the kernel of $\phi$. Also, for every $s \in S$ and any $p \in \pi^{-1}(s)$ there
%exists $d_p \in k^\times$ such that $\phi(Y^\xi) = \prod_i (X^{s_i})^{p_i} = d_p X^s$.
%We write $d_l = d_{p_l}$ and $d'_l = d_{p'_l}$ for all $1 \leq l \leq m$, and set
%\begin{align*} 
% S_l= d_l Y^{p'_l} - d'_l Y^{p_l}.  
%\end{align*}   
%These elements also belong to the kernel of $\phi$. 
%
%\begin{Lemma} 
%\label{L:semigroup-presentation} 
%Keep the notation from the previous paragraph. The alebra $\k^\alpha[S]$ is isomorphic to
%$F / I$ where $I \subset F$ is the ideal generated by the elements $C_{i,j}$ and $S_l$,
%with $1\leq i < j \leq n$ and $1 \leq l \leq m$.
%\end{Lemma} 
%\begin{proof} 
%Set $B = F/I$. By abuse of notation we write $Y_i$ and $Y^\xi$ for the class of $Y_i$ and
%$Y^\xi$ in $B$ for all $1 \leq i \leq n$ and all $\xi \in \NN_0^n$. Since all the elements
%$C_{i,j}$ and $S_l$ are homogeneous, the $\NN_0^r$-grading of $F$ induces an $\NN_0^r$-grading
%on $B$ whose support is contained in $S$. Denote by $\tphi: B \to \k^\alpha[S]$ the graded
%morphism induced by $\phi$. The relations imposed on $B$ imply that if $(p, p') \in L(S)$
%then $Y^p$ and $Y^{p'}$ are nonzero multiples of each other. Hence $\dim_\k B_s = 1$ for
%all $s \in S$, and $\tilde \phi$ is a surjective morphism between locally finite graded
%vector spaces whose homogeneous components have the same dimension, so it is an
%isomorphism.
%\end{proof}


%\newpage 
\subsection*{Algebras with a toric degeneration}
%\label{algebras-with-qatd}

\begin{Definition}
A \emph{quantum affine toric degeneration} of $A$ is a connected filtration on $A$ such 
that the associated graded ring is a quantum affine toric variety. 
\end{Definition}
If $A$ degenerates to a quantum affine toric variety with underlying semigroup $S$, we 
refer to $S$ as the semigroup associated to the degeneration. In view of Theorem 
\ref{transfer} and Proposition \ref{twisted-semigroup-algebras}, an algebra with a 
quantum toric degeneration is noetherian, integral, has property $\chi$ and finite local 
dimension. Furthermore, we can determine whether $A$ is AS-Cohen-Macualay or 
AS-Gorenstein by studying the semigroup $S$.

We fix some notation for the rest of this subsection. Let $t \geq 1$. We denote by $S 
\subset \NN_0^t$ a positive affine semigroup, by $\{s_1, \ldots, s_n\}$ its 
Hilbert basis, by $\pi: \NN_0^n \to \NN_0^t$ the semigroup morphism defined by 
$e_i \longmapsto s_i$, and denote by $P = \{(p_1, p'_1), \ldots, (p_m, p'_m)\} 
\subset L(S)$ the minimal generating set of $L(S)$ such that $p_i <_{\lex} p'_i$ for all
$i$, as in \ref{congruence}.

\begin{Definition}
\label{s-phi-type}
Let $\phi:~S~\to~\NN_0$ be a semigroup morphism such that $\phi(s_i) > 0$ for all $1 
\leq i \leq n$, and set $\tphi = \phi \circ \pi$.  We say that the algebra $A$ is of 
$(S, \phi)$-type if the following hold. 
\begin{enumerate}
\item \label{generation}
$A$ is generated as algebra by a finite set of homogeneous elements $\{b_1, 
\ldots, b_n\}$ of the same cardinality as the Hilbert basis of $S$. We set
$\b^\xi = \displaystyle \prod_{i = 1}^n b_i^{\xi_i}$ for each $\xi \in \NN_0^n$.

\item \label{commutation}
For each $1 \leq i < j \leq n$ and each $\xi \in \NN_0^n$ such that $\tphi(\xi) < 
\phi(s_i + s_j)$, there exist $c_{i,j} \in \k^\times$ and $c^{i,j}_\xi \in \k$ such that
\begin{align*}
  b_j b_i = c_{i,j} b_i b_j + \sum_{\tphi(\xi) < \phi(s_i + s_j)} c^{i,j}_\xi
    \b^\xi.
\end{align*}

\item \label{straightening}
For each $1 \leq i \leq m$ and each $\xi \in \NN_0^n$ such that $\tphi(\xi) <
\tphi(p_i)$ there exist $d_i \in \k^\times$ and $d^i_\xi \in \k$ such that
    \begin{align*}
      \b^{p'_i} = d_i \b^{p_i} + \sum_{\tphi(\xi) < \tphi(p_i)} d^i_\xi \b^\xi.
    \end{align*}
\end{enumerate}
\end{Definition}

\begin{Remark*}
Since $\tphi(e_i) = \phi(s_i) > 0$ for all $i$, given $l \in \NN_0$ there exist 
finitely many $\xi \in \NN_0^n$ such that $\tphi(\xi) \leq l$ and so the sums on the 
right hand side of the formulas displayed in \ref{commutation} and \ref{straightening} 
are finite.
\end{Remark*}

\paragraph
\label{type-filtration}
If $A$ is an algebra of $\phi$-type we write $F_lA = \langle \b^\xi \mid \tphi(\xi)
\leq l \rangle$ for each $l \in \NN_0$. The fact that $\tphi(e_i) > 0$ for all $i$ 
implies that this is a finite dimensional vector space, and that $F_0A = \k$. The 
following lemma shows that $\F = \{F_lA\}_{l \geq 0}$ is a connected filtration on $A$.

\begin{Lemma*}
Let $A$ be an algebra of $(S, \phi)$-type.
\begin{enumerate}
\item \label{filtration}
Given $\xi, \nu \in \NN_0^n$ there exists $c_{\xi, \nu} \in \k^\times$ such that $\b^\xi
\b^\nu \equiv c_{\xi + \nu}\b^{\xi + \nu} \mod F_{\tphi(\xi + \nu) - 1}A$. In particular 
$\F$ is a filtration on $A$.

\item \label{straight}
Given $(\xi, \nu) \in L(S)$ there exists $d_{\xi, \nu} 
\in \k^\times$ such that $\b^\xi \equiv d_{\xi, \nu} \b^\nu \mod F_{\tphi(\xi) - 1} A$.
\end{enumerate}
\end{Lemma*}
\begin{proof}
To prove item \ref{filtration} we proceed by induction on $\tphi(\xi + \nu)$, with the
$0$-th step being obvious since $F_0A = \k$ is a subalgebra of $A$. Suppose that the
result holds for all $l < \tphi(\xi + \nu)$ and let $i$ be the least integer such that
$\xi_i \neq 0$, so $\b^\xi \b^\nu = b_i (\b^{\xi - e_i} \b ^\nu)$. Using the inductive
hypothesis we obtain
\begin{align*}
b_i (\b^{\xi - e_i} \b^\nu) 
&= b_i \bigg( c_{\xi- e_i, \nu} \b^{\xi - e_i + \nu} 
  + \sum_{\tphi(\rho) < \tphi(\xi - e_i + \nu)} c_\rho \b^\rho \bigg)\\
&= c_{\xi- e_i, \nu} b_i \b^{\xi - e_i + \nu} 
  + \sum_{\tphi(\rho) < \tphi(\xi - e_i + \nu)} c_\rho b_i \b^\rho
\end{align*}
where the $c_\rho \in \k$ and $c_{\xi - e_i,\nu} \in \k^\times$. The inductive  
hypothesis also implies each product $b_i \b^\rho$ lies in $F_{\tphi(\xi + \nu) -1} A$,
so
\[
  b_i \b^{\xi - e_i} \b^\nu \equiv c_{\xi- e_i, \nu}b_i \b^{\xi - e_i + \nu} \mod
  F_{\tphi(\xi + \nu)-1}A.
\]
Now let $j$ be the least integer such that $\nu_j \neq 0$. If $i \leq j$ then $b_i 
\b^{\xi - e_i + \nu} = \b^{\xi + \nu}$ and we are finished; otherwise, using item 2 of 
Definition \ref{s-phi-type} and a similar argument as before, we obtain
\begin{align*}
  b_i \b^{\xi - e_i + \nu} = (b_i b_j)(\b^{\xi - e_i + \nu - e_j}) \equiv c_{j,i} (b_j
  b_i)(\b^{\xi - e_i + \nu - e_j}) \mod F_{\tphi(\xi + \nu) - 1}A
\end{align*}
The same reasoning applied to the product $b_j(b_i\b^{\xi - e_i + \nu - e_j})$ shows that
\begin{align*}
  b_j (b_i \b^{\xi - e_i + \nu - e_j}) 
    &\equiv c_{e_i, \xi - e_i + \nu - e_j} b_j(\b^{\xi  + \nu - e_j}) \mod
  F_{\tphi(\xi + \nu) - 1}A.
\end{align*}
Since $j < i$, the definition of $j$ implies $b_j(\b^{\xi  + \nu - e_j}) = \b^{\xi + 
\nu}$, so the proof of item \ref{filtration} is complete.

We now prove item \ref{straight}. Set
\[
  T = \{(\xi, \nu) \in \NN_0^n \times \NN_0^n \mid \b^\xi \equiv c \b^{\nu} \mod
    F_{\tphi(\xi) -1}
    \mbox{ for some } c \in \k^\times\}.
\]
We will show that $L(S) \subset T$, which clearly implies the desired result.
By definition $T$ is an equivalence relation, and item \ref{filtration} implies it is a
congruence on $\NN_0^n$. By item \ref{straightening} of Definition \ref{s-phi-type}, 
every pair $(p_i, p'_i) \in P$ lies in $T$. Since $L(S)$ is the smallest congruence 
containing $P$, we deduce that $L(S) \subset T$.
\end{proof}

\paragraph
\label{P:equivalent-qatd}
A \emph{section} of $\pi$ is a function $\tau: S \to \NN_0^n$ such that $\pi \circ \tau =
\Id_S$, that is $s = \sum_i \tau(s)_i s_i$ for every $s \in S$. If $A$ is an algebra of
$(S,\phi)$-type then Lemma \ref{type-filtration} implies that for any section $\tau$ of
$\pi$ the set $\{\b^{\tau(s)} \mid s \in S\}$ spans $A$. We now show that an algebra has 
a quantum affine toric degeneration if and only if it is of $(S,\phi)$-type for an 
adequate morphism $\phi$ and the spanning set determined by any section is linearly 
independent.
\begin{Proposition*}
The following statements are equivalent.
\begin{enumerate}
\item \label{has-qat-degeneration}
The algebra $A$ has a quantum affine toric degeneration with associated semigroup $S$.
\item \label{every-section}
There exists a semigroup morphism $\phi: S \to \NN_0$ such that $A$ is of 
$(S, \phi)$-type, and for every section $\tau: S \to \NN_0^n$ of $\pi$ the set $\{
\b^{\tau(s)} \mid s \in S\}$ is linearly independent.
\item \label{some-section}
There exist a semigroup morphism $\phi: S \to \NN_0$ such that $A$ is of 
$(S, \phi)$-type and a section $\tau: S \to \NN_0^n$ of $\pi$ such that the set $\{
\b^{\tau(s)} \mid s \in S\}$ is linearly independent.
\end{enumerate}
\end{Proposition*}
\begin{proof}
We first show that \ref{has-qat-degeneration} implies \ref{every-section}. By hypohtesis
there exists a filtration by graded subspaces $\F = \{F_lA\}_{l \geq 0}$ such that 
$\gr_\F A \cong \k^\alpha[S]$ as $\NN_0^{r+1}$-graded algebras for some $2$-cocycle 
$\alpha: S \times S \to \k^\times$, with the grading on the twisted semigroup algebra 
given by a semigroup morphism $\psi: S \to \NN_0^{r+1}$ such that $|\psi(s_i)| > 0$ for 
all $1 \leq i \leq n$. We identify $\gr_\F A$ with $\k^\alpha[S]$ through this 
isomorphism to simplify notation. 

For each $1 \leq i \leq n$, we choose homogeneous elements $b_i \in A$ such that 
$\gr b_i = X^{s_i}$. By definition of the product of an associated graded ring, for each 
$\xi \in \NN_0^n$ the element $\prod_{i = 1}^n (\gr b_i)^{\xi_i}$ equals either 
$\gr \left(\prod_{i = 1}^n b_i^{\xi_i}\right)$ or zero. Since $\k^\alpha[S]$ is an
integral ring the last possibility cannot occur, so $\gr (\b^\xi)$ equals a nonzero 
multiple of $X^{\pi(\xi)}$. Thus if $\tau: S \to \NN_0^n$ is a section of $\pi$ the set 
$\{\gr \b^{\tau(s)} \mid s \in S\}$ is a basis of $\k^\alpha[S]$, which implies that 
$\{\b^{\tau(s)} \mid s \in S\}$ is a basis of $A$. This also proves that $A$ satisfies 
item \ref{generation} of Definition \ref{s-phi-type}.

Let $\phi: S \to \NN_0$ be the additive map given by $s \mapsto \psi(s)_{r+1}$; equivalently $\phi(s)$ is the minimal $l$ such that $\b^{\tau(s)} \in F_lA$, in particular $\phi(s_i) > 0$ since $F_0A = \k$ by hypothesis. Thus $F_lA = \vspan{
\b^{\tau(s)} \mid \phi(s) \leq l}$. For each $1 \leq i,j \leq n$
and each $1 \leq k \leq m$ there exist $c_{i,j}, d_k \in \k^\times$ such that
\begin{align*}
  (\gr b_j b_i) &= c_{i,j} (\gr b_i b_j),\\
  \gr \b^{p'_k} &= d_k \gr \b^{p_k}
\end{align*}
hold in $\k^\alpha[S]$, which implies that items \ref{commutation} and 
\ref{straightening} of Definition \ref{s-phi-type} hold in $A$. Thus $A$ is of
$(S,\phi)$-type, and we have proved \ref{has-qat-degeneration} implies 
\ref{every-section}.

It is clear that \ref{every-section} implies \ref{some-section}, so let us see that
\ref{some-section} implies \ref{has-qat-degeneration}. Define the filtration 
$\{F_l A\}_{l \geq 0}$ as in \ref{type-filtration}. By item 2 of Lemma 
\ref{type-filtration}, the set $\{\b^{\tau(s)} \mid \phi(s) \leq l\}$ generates $F_lA$
for each $l \in \NN_0$, and since by hypothesis it is linearly independent, it is a 
basis of $F_lA$. Hence $\gr_\F A$ is generated by $\{\gr \b^{\tau(s)} \mid s \in S\}$. 
Once again by Lemma \ref{type-filtration} for each $s,s' \in S$ there exist 
$\beta(s,s'), \alpha(s,s') \in \k^\times$ such that
\begin{align*}
(\gr \b^{\tau(s)})(\gr \b^{\tau(s')}) = \beta(s,s') \gr \b^{\tau(s) + \tau(s')} = 
\alpha(s,s') \gr \b^{\tau(s+s')}.
\end{align*}
Associativity of the product of $\gr_\F A$ implies that $\alpha: S \times S \to 
\k^\times$ is a $2$-coycle, so we may consider the $\k$-linear map $\k^\alpha[S] \to 
\gr_\F A$ induced by the assignation $X^s \mapsto \gr \b^{\tau(s)}$. This map is 
multiplicative and hence an isomorphism of $\k$-algebras. Furthermore, the elements $\gr 
\b^{\tau(s)}$ are homogeneous and form a basis of the maximal graded ideal of 
$\gr_\F A$, so this algebra is indeed a quantum affine toric variety.
\end{proof}

\paragraph
\label{lex-degeneration}
Let $<$ be a total order on $S$ compatible with the semigroup structure, i.e. for all 
$s,s',s'' \in S$, the inequality $s<s'$ implies $s+s'' < s'+s'' $. We will say 
that an ordered basis $\mathcal B = \{b_s \mid s \in S\}$ is an \emph{$(S, <)$-basis} of 
$A$ if each $b_s$ is homogeneous, and for all $s,s',s'' \in S$ such that 
$s'' < s+s'$ there exist $c_{s,s'} \in \k^\times$ and $c_{s,s'}^{s''} \in \k$ such that
\[
  b_s b_{s'} = c_{s,s'} b_{s+s'} + \sum_{s'' < s+s'} c_{s,s'}^{s''} b_{s''}.
\]

Recall that we are considering $S$ along with a fixed embedding $i: S \to \NN_0^t$,
which induces a total order on $S$ by restriction of the lexicographic order. For this
paragraph only we consider alternative embeddings of $S$, and for each embedding
$\iota: S \to \NN_0^u$ we denote by $<^\iota$ the total order on $S$ induced by pulling 
back the lexicographic order of $\NN_0^u$ through $\iota$. Since $\iota$ is a semigroup
morphism, any such order is compatible with the semigroup structure.
\begin{Proposition*}
The algebra $A$ has a quantum affine toric degeneration with underlying semigroup $S$ if 
and only if there exists an embedding $\iota: S \to \NN_0^u$ such that $A$ has an 
ordered $(S, <^\iota)$-basis.
\end{Proposition*}
\begin{proof}
Suppose $A$ has a quantum affine toric degeneration with underlying semigroup $S$. Then 
by Proposition \ref{P:equivalent-qatd} there exists a semigroup morphism $\phi:S\to 
\NN_0$ such that $A$ is of $(S, \phi)$-type, and we may choose any section $\tau: S \to 
\NN_0^n$ to obtain a basis $\mathcal B = \{\b^{\tau(s)} \mid s \in S\}$. Recall that with
the identifications from \ref{affine-semigroups}, we are assuming that $S \subset 
\NN_0^t$; let $\iota$ be the restriction to $S$ of the semigroup morphism $\phi \times 
\Id: \NN_0^t \to \NN_0^{t+1}$, and write $<$ for $<^\iota$. Notice that $\phi(s) < 
\phi(s')$ implies $s<s'$. By Lemma 
\ref{type-filtration}, for all $s,s' \in S$ and all $s''$ such that $\phi(s'') < 
\phi(s+s')$ there exist $c_{s,s'} \in \k^\times$ and $c_{s,s'}^{s''} \in \k$ such that
\[
  \b^{\tau(s)}\b^{\tau(s')} 
    = c_{s,s'}\b^{\tau(s+s')} 
      + \sum_{\phi(s'') < \phi(s+s')} c_{s,s'}^{s''} \b^{\tau(s'')},
\]
which implies $\mathcal B$ is an ordered $S$-basis with respect to $<$.

Now assume $A$ has an ordered $S$-basis with respect to some total order $<$ induced 
by an embedding $\iota: S \to \NN_0^u$. Since $<$ is the pull-back of the lexicographic 
order through an embedding, we might as well assume $S \subset \NN_0^u$ and that $<$ is 
the lexicographic order. For each $s \in S$ set $F_sA = \langle b_{s'} \mid s' \leq s 
\rangle$ and $F_{<s}A = \langle b_{s'} \mid s' < s \rangle$. The hypothesis implies that
$\mathcal F = \{F_sA \mid s \in S\}$ is an $S$-indexed filtration on $A$ by graded 
subspaces, so we may form the associated $S$-graded ring
\[
  \gr_\F A = \bigoplus_{s \in S} \frac{F_sA}{F_{<s}A}.
\]
The set $\{\gr b_s \mid s \in S\}$ is a basis of $\gr_\F A$, and by definition
$(\gr b_s) (\gr b_{s'}) = c_{s,s'} \gr b_{s+s'}$, so the map $\alpha: S \times S \to 
\k^\times$ defined by $\alpha(s,s') = c_{s,s'}$ is a $2$-cocycle and $\gr_\F A \cong 
\k^\alpha[S]$.

For each $1 \leq i \leq n$ we put $b_i = b_{s_i}$. The set $\{\gr b_i \mid i = 1 \ldots, 
n\}$ generates $\gr_\F A$ as algebra, so the set $\{b_i \mid i = 1, \ldots, n\}$ 
generates $A$ as algebra. Given $\xi \in \NN_0^n$, the definition of the product of 
the associated graded ring implies that $\prod_i(\gr b_i)^{\xi_i}$ equals either $\gr 
(\b^\xi)$ or zero, and since $\k^\alpha[S]$ is integral the second possibility can 
not occur. This implies that for all $1 \leq i < j \leq n$ and all $\xi \in \NN_0^n$ 
such that $\pi(\xi) < s_i +s_j$ there exist $c_{i,j} \in \k^\times$ and $c^{i,j}_\xi \in 
\k$ such that
\begin{align*}
  b_j b_i = c_{i,j} b_i b_j + \sum_{\pi(\xi) < \pi(s_i + s_j)} c^{i,j}_\xi
    \b^\xi,
\end{align*}
and for each $1 \leq i \leq m$ and each $\xi \in \NN_0^n$ such that $\pi(\xi) < p_i$ 
there exist $d_i \in \k^\times$ and $d^i_\xi \in \k$ such that
\begin{align*}
  \b^{p'_i} = d_i \b^{p_i} + \sum_{\pi(\xi) < p} d^i_\xi \b^\xi.
\end{align*}
The set 
\[
  C 
    =\{\pi(\xi) \mid d^i_\xi \neq 0 \mbox{ for some } 1 \leq i \leq m, 
      \mbox{ or } c^{i,j}_\xi \neq 0 \mbox{ for some $1 \leq i < j \leq n$}\}
\]
is finite and hence is contained in a cube $[0,N+1]^u$ for $N$ large enough, so the 
morphism $\phi: \NN_0^u \to \NN$ mapping $e_i$ to $(N+1)^i$ sends each $c \in C$ to 
the natural number having $N+1$-adic expansion $c$. In particular $\phi$ respects the 
restriction of the lexicographic order to $C$, so $A$ is of $(S,\phi)$-type. 

Let $\tau: S \to \NN_0^n$ be any section of $\pi$. The algebra $\gr_\F A$ has a natural 
$S$-grading, and for each $s \in S$ the element $\gr \b^{\tau(s)}$ is of degree $s$. As 
we have already observed, this is a non-zero element so the set $\{\gr \b^{\tau(s)} \mid 
s \in S\}$ is a basis of $\gr_\F A$, which implies that $\{\b^{\tau(s)} \mid s \in S\}$ 
is a basis of $A$. Thus by Proposition \ref{P:equivalent-qatd} $A$ has a quantum affine 
toric degeneration with underlying semigroup $S$.
\end{proof}

\paragraph
We finish this section with an easy consequence of Lemma \ref{type-filtration}. It will
not be used in the sequel, but it is interesting in its own right.
\begin{Proposition*}
If $A$ is of $(S,\phi)$-type and for some section $\tau$ of $\pi$ the set
$\{\b^{\tau(s)} \mid s \in S\}$ is linearly independent, then the relations given in
items \ref{commutation} and \ref{straightening} of Definition \ref{s-phi-type} give a
presentation of $A$.
\end{Proposition*}
\begin{proof}
Since $A$ is of $(S,\phi)$-type, it is generated as algebra by homogeneous elements 
$b_1, \ldots, b_n$, and there exist constants $c_{i,j}, c^{i,j}_\xi, d_i, d^i_\xi$ such
that $A$ complies with Definition \ref{s-phi-type}. Furthermore, the relations described 
in items \ref{commutation} and \ref{straightening} of this definition are homogeneous.

Let $B$ be the free algebra generated by $X_1, \ldots, X_n$ and let $I$ be the ideal of
$B$ generated by the elements
\begin{align*}
  X_j X_i &- c_{i,j} X_i X_j - \sum_{\tphi(\xi) < \phi(s_i + s_j)} c^{i,j}_\xi
  X^\xi, 
    & \mbox{for } 1 \leq i < j \leq n;\\
  X^{p'_i} &- d_i X^{p_i} - \sum_{\tphi(\xi) < \tphi(p_i)} d^i_\xi X^\xi, 
    &\mbox{for } 1 \leq i \leq m;
\end{align*}
where $X^\xi = X_1^{\xi_1} X_2^{\xi_2} \cdots X_n^{\xi_n}$ for each $\xi \in \NN^n_0$. 
We put a grading on $B$ by setting $\deg X_i = \deg b_i$, and this induces
a grading on $B/I$. Since $B$ is a free algebra the assignation $X_i \mapsto b_i$ 
induces a morphism of graded algebras $B \to A$, which factors through $B/I$. We thus
obtain a morphism of graded algebras $f: B/I \to A$.

We denote by $Y_i$ the image of $X_i$ in $B/I$. Cleary $B/I$ is an $(S,\phi)$-algebra, 
and the algebra map $f: B/I \to A$ sends $Y_i$ to $b_i$ for all $i$. Since 
$f(Y^{\tau(s)}) = \b^{\tau(s)}$, the set $\{Y^{\tau(s)} \mid s \in S\}$ is linearly 
independent and hence a basis of $B/I$. Thus $f$ maps a basis onto a basis, so it is 
an isomorphism.
\end{proof}


%\newpage 
\section{Quantum affine toric degeneration of quantum Schubert varieties}
\label{q-flag-schubert}
In this section we apply the rsults in the previous section to study Schubert varieties 
of quantum flag varieties. We recall the definitions of quantum flag and Schubert 
varieties found with some detail in order to establish notation, and following 
Littelmann we associate to each a an affine semigroup $S$. We then use a result due to
Caldero to show that each quantum Schubert variety has an $(S, <)$-basis in the sense of 
\ref{lex-degeneration}, and hence has a quantum toric degeneration with associated 
semigroup $S$.

\subsection*{Quantum flag and Schubert varieties}

\paragraph
\label{lie-algebra-notation}
Let $\g$ be a complex semisimple Lie algebra and let $G$ be the corresponding simply 
connected Lie group. We denote by $\Lambda$ the weight lattice of $\g$, and write 
$\{\varpi_1, \ldots, \varpi_n\}$ for its set of fundamental weights. Let $\Lambda^+ =
\sum_i \NN_0 \varpi_i$ be the set of dominant weights, and $\alpha_1, \ldots,\alpha_n$ 
the positive roots of $\g$. Let $W$ be the Weyl group of $\g$, and $s_i \in W$ the
reflection corresponding to the $i$-th positive root. Given an element $w \in W$ we 
denote its length by $\ell(w)$, and set $N$ to be the length of the longest word of $W$. 
A \emph{decomposition} of $w \in W$ is a word on the generators $s_i$ that equals $w$ in 
$W$. A decomposition of the longest word of $W$ is said to be \emph{adapted} to $w$ if
it is of the form $s_{i_1} \ldots s_{i_N}$ with $s_{i_1} \ldots s_{i_{\ell(w)}} = w$.
For every element $w \in W$ there exists a decomposition of the longest word of $W$ 
adapted to $w$. In other words the longest word of $W$ is the maximum for the weak right 
Bruhat order on $W$, see \cite{BB}*{Proposition 3.1.2}.

\paragraph
\label{quantized-enveloping-algebra}
Fix $q \in \k^\times$ not a root of unity. Let $U_q(\g)$ be the quantum enveloping 
algebra of $\g$ as defined in \cite{Jan}*{Definition 4.3}. Let $U_q^+(\g), U_q^-(\g)$ be 
the positive and negative parts of $U_q(\g)$ as defined in \cite{Jan}*{4.4}. Recall that 
for each $\lambda \in \Lambda^+$ there is an irreducible highest-weight representation 
of $U_q(\g)$ of type $\mathbf{1}$, which we denote by $V_q(\lambda)$.
Each $V_q(\lambda)$ decomposes as the direct sum of weight spaces $\bigoplus_{\mu \in 
L(\lambda)} V_q(\lambda)_\mu$, where $L(\lambda)$ is a finite subset of $\Lambda$; the 
dimensions of the weight spaces are the same as in the classical case, so the Weyl and 
Demazure character formulas hold for these representations, see \cite{K}*{section 12.4}, 
or \cite{Jan}*{subsection 5.15} for the Weyl formula.

\paragraph
Since $U_q(\g)$ is a Hopf
algebra, its dual is an algebra with convolution product induced by the coproduct of
$U_q(\g)$. There is a map $V_q(\lambda)^* \ot V_q(\lambda) \to U_q(\g)^*$ defined by
sending $\phi \ot v \in V_q(\lambda)^* \ot V_q(\lambda)$ to the linear functional
$c^\lambda_{\phi, v}$, which assigns to each $u \in U_q(\g)$ the scalar $c^\lambda_{\phi,
v}(u) = \phi(uv)$. Functionals of type $c^\lambda_{\phi, v}$ are called \emph{matrix
coefficients}. The $\k$-linear span of the matrix coefficients is a subalgebra of 
$U_q(\g)^*$ denoted by $\O_q[G]$, called the quantized algebra of coordinate functions 
over the group $G$. 

\paragraph
\label{q-full-flag-varieties}
Quantum flag varieties were introduced by Soibelman in \cite{S} and by Lakshmibai and 
Reshetikhin in \cite{qLR}; we review their definition. 
Fix a maximal Borel subgroup $B$ of $G$. The \emph{full flag variety} associated to $G$ 
is $G/B$. Let $C^+_q(\lambda)$ be the vector space of matrix coefficients of the form
$c^\lambda_{\phi,v_{\lambda}}$ in $U_q(\g)^*$, where $v_\lambda$ is a highest weight
vector in $V_q(\lambda)$, and set 
\[ 
  \O_q[G/B] = \bigoplus_{\lambda \in \Lambda^+} C^+_q(\lambda) \subset \O_q[G] 
    \subset U_q(\g)^*.  
\] 
This is called the \emph{quantum full flag variety} of $G$. By definition the product of 
two matrix coefficients in $\O_q[G/B]$ is again in $\O_q[G/B]$, and its decomposition 
as a direct sum gives $\O_q[G/B]$ the structure of a $\Lambda^+$-graded 
algebra. 

\paragraph
Let $I$ be a subset of the set of fundamental weights and set $\mathcal J(I) = 
\sum_{\varpi_i \notin I} \NN_0 \varpi_i$. Denote by $W_I \subset W$ the subgroup 
generated by the reflections $s_i$ with $\varpi_i \in I$, and for each class in 
$W/W_I$ pick a representative of minimal length. We denote by $W^I$ the set of these 
representatives. Since the Demazure character formula holds, for each $w \in W^I$ and 
each $\lambda \in \mathcal J(I)$ the vector space $V_q(\lambda)_{w\lambda}$ has 
dimension $1$. The \emph{Demazure module} $V_q(\lambda)_{w}$ is the 
$U_q^+(\g)$-submodule of $V_q(\lambda)$ generated by a vector of weight $w\lambda$
in $V_q(\lambda)$.

The set $I$ determines a Lie subalgebra $\mathfrak p \subset \g$, and a parabolic 
subgroup $P_I \subset G$. The variety $G/P_I$ is the corresponding generalized flag 
variety. To this datum we associate the $\Lambda^+$-graded subalgebra of $\O_q[G/B]$ 
\[ 
  \O_q[G/P_I]:= \bigoplus_{\lambda \in \mathcal J(I)} C_q^+(\lambda) 
\] 
called the \emph{quantum partial flag variety associated to $I$}. The case 
$I = \emptyset$ corresponds to the full flag variety.

\paragraph
\label{q-Schubert-varieties}
Given vector spaces $V_2 \subset V_1$, we denote by $V_2^\perp$ the set of linear
functionals over $V_1$ which are zero on $V_2$. For every $w \in W^I$, the vector space 
\[
  J_w^I = \bigoplus_{\lambda \in \mathcal J(I)} \vspan{c_{\phi,v_\lambda}^\lambda
  \in C_q^+(\lambda) \mid \phi  \in V_q(\lambda)_w^\perp } \subset \O_q[G/P_I] 
\] 
is an ideal of $\O_q[G/P_I]$ called the \emph{Schubert ideal} associated to $w$. The
quotient algebra $\O_q[G/P_I]_w = \O_q[G/P_I]/J^I_w$ is called the \emph{quantum Schubert
variety} associated to $w$.

%\newpage 
\subsection*{Degeneration of quantum Schubert varieties}
%\label{qtad-flag}

\paragraph
In this subsection we assume that $\k$ is of characteristic zero and that $q \in 
\k^\times$ 
is trascendental over $\QQ$. Our aim is to show that under these conditions quantum 
Schubert varieties have quantum affine toric degenerations. In order to do so we work 
for a moment over the field $\QQ(v)$, where $v$ is an indeterminate over $\QQ$, and 
consider the $\QQ(v)$-algebra $U = U_v(\g)$. The general case will follow by extension 
of scalars. 

We denote by $U^+$ and $U^-$ the algebras $U^+_v(\g)$ and $U^-_v(\g)$, respectively. Let 
$\A = \ZZ[v, v^{-1}] \subset \QQ(v)$. The algebra $U$ has an $\A$-form, defined 
equivalently in \cite{Lu}*{paragraph 3.1.13} and \cite{Jan}*{section 11.1}, which we 
denote by $U_\A$. This is a graded subring of $U$, and the algebras $U^+$ and $U^-$ also 
have $\A$-forms which we denote by $U_\A^+$ and $U_\A^-$, respectively. By definition 
$U \cong \QQ(v) \ot_\A U_\A$, and analogous results hold for $U_\A^+$ and $U_\A^-$.

The algebra $U_\A^-$ has a homogeneous $\A$-basis $\B$, called the \emph{canonical 
basis} of $U^-$, discovered independently by Lusztig and Kashiwara.
For a general overview of the theory we refer to \cite{K}, and for proofs and details to
\cite{Lu}*{section II} and \cite{Jan}*{chapters 9 - 11}. The $\A$-form $U_\A^-$ is
compatible with the coalgebra structure of $U$, in the sense that $\Delta(U_\A^-) \subset
U_\A^- \ot_\A U_\A^-$, where $\Delta$ is the comultiplication of $U$, see 
\cite{Lu}*{Proposition 14.2.6 (a)}. Furthermore the highest weight modules 
$V_v(\lambda)$ have corresponding $\A$-forms, which we denote by $V_\A(\lambda)$. All 
these $\A$-forms are compatible with the weight decompositions of the original objects.

\begin{Theorem} 
\label{T:A-crystal-basis} 
Let $\lambda \in \Lambda^+$, let $I$ be a subset of the fundamental weights and let 
$w \in W^I$. There exists $\B_\lambda \subset \B$ such that $\B_\lambda v_\lambda 
\subset V_\A(\lambda)$ is a basis of weight vectors of $V_\A(\lambda)$, and furthermore, 
there exists a subset $\B_w \subset \B$ such that $(\B_w \cap \B_\lambda)v_\lambda$ is a 
basis of the Demazure module $V_\A(\lambda)_w$.
\end{Theorem} 
\begin{proof} 
See \cite{K1}*{Theorem 3.2.5}. Remark 3.2.6 of the same reference states that this
decomposition induces a decomposition of the corresponding $\A$-forms.  
\end{proof}

\paragraph
\label{L:are-semigroups}
Littelman proved in \cite{Lit}*{Proposition 1.5, a)} that for every decomposition $w_0$ 
of the longest word of $W$ there is a parametrization of the canonical basis $\B$ by a 
set $\S_{w_0} \subset \NN^N$, where $N$ is the length of the longest word of $W$.  For 
each $s \in \S_{w_0}$ let $b_{s}$ denote the corresponding element in the canonical 
basis. Thus if $\lambda$ is a dominant weight then there exists a finite set 
$S_{\lambda, w_0} \subset S_{w_0}$ such that $\B_\lambda = \{b_s \mid s \in 
S_{\lambda, w_0} \}$.

Let $I$ be a subset of the fundamental weights, let $w \in W^I$, and let $w_0$ be a 
decomposition of the longest word of $W$ adapted to $w$. Following \cite{C} we set 
\begin{align*} 
  \tilde \S_{w_0} 
    &= \{(s,\lambda) \mid s \in S_{\lambda,w_0}\} 
    \subset \NN_0^N \times \Lambda^+ \cong \NN_0^{N+n},\\ 
  \tilde \S_{w_0}^w 
    &= \{(s,\lambda) \in \tilde \S_{w_0} \mid b_s \in \B_\lambda \cap \B_w\},  \\ 
  \tilde \S_{w_0,I}^w 
    &= \{(s,\lambda) \in \tilde S_{w_0}^w \mid \lambda \in \mathcal J(I)\} 
    = \tilde S_{w_0}^w \cap (\NN_0^N \times \mathcal J(I)).  
\end{align*} 
These are semigroups with a total order given by the restriction of the lexicographic 
order of $\NN_0^{N + n}$. By abuse of notation, we denote this order by $\leq_{\lex}$.  

\begin{Lemma*} 
The sets $\tilde \S_{w_0}, \tilde \S_{w_0}^w$ and $\tilde \S_{w_0, I}^w$ are normal 
affine semigroups.  
\end{Lemma*} 
\begin{proof} 
For $\tilde \S_{w_0}$ and $\tilde \S_{w_0}^w$ see \cite{C}*{Theorem 2.2 and Theorem 2.4}
respectively. By definition $\tilde \S_{w_0,I}^w = \tilde S_{w_0}^w \cap (\NN_0^N \times
\mathcal J(I))$, and Gordan's Lemma \cite{BH}*{Proposition 6.1.2} implies that the 
intersection of normal semigroups is normal. 
%Let $D = \RR_+ \tilde S_{w_0}^w$ and let $G$ be the group of fractions of 
%$\NN_0^n \times \mathcal J(I)$. Since $\tilde S_{w_0}^w$ is normal, Gordan's Lemma 
%\cite{BH}*{Proposition 6.1.2} implies it is equal to $D \cap (\ZZ^n \times P)$. Now 
%every point of $G \cap D$ lies in $\tilde S_{w_0}^w$, and hence in $\NN_0^n \times P^+$. 
%Thus 
%\[ 
  %\tilde S^w_{w_0,I} 
    %= \tilde S_{w_0}^w \cap (\NN_0^n \times \mathcal J(I)) 
    %= D \cap (\NN_0^n \times \mathcal J(I)) 
    %= D \cap G, 
%\]
%and Gordan's Lemma implies that $D \cap G$ is a normal semigroup.  
\end{proof}

\paragraph
Let $\lambda$ be a dominant integral weight. By Theorem \ref{T:A-crystal-basis}, the set
$\B_\lambda v_\lambda$ is a basis of $V_v(\lambda)$, and hence it has a dual basis
$\B_\lambda^*$. For every $s \in S_\lambda$ there is an element $b_{s,\lambda}^* \in
\B_\lambda^*$ defined as the only functional that sends $b v_\lambda$ to $\delta_{b,b_s}$
for all $b \in \B$. Each linear functional $b^*_{s, \lambda}$ induces a matrix 
coefficient which by abuse of notation we will also denote by $b^*_{s, \lambda}$. We 
denote by $C^+_\A$ the sub $\A$-module of $C^+_v(\lambda)$ generated by these matrix 
coefficients, and set 
\[ 
  \O_\A[G/B] = \bigoplus_{\lambda \in \Lambda^+} C^+_\A(\lambda).  
\] 
Clearly $\O_\A[G/B] \subset \O_v[G/B] \subset U_v(\g)^*$. The product of $\O_v[G/B]$ is 
induced by the product of this last algebra, which in turn is induced by the coproduct 
of $U_v(\g)$. Since $U_\A$ is stable by the coproduct, $\O_\A[G/B]$ is a sub 
$\A$-algebra of $\O_q[G/B]$.

Setting 
\begin{align*} 
\B_{\lambda, w}^* 
  &= \{b_{s, \lambda}^* \mid (s, \lambda) \in \tilde S_{w_0, I}^w \}, \\ 
\B_{\lambda, w}^\perp 
  &= \{b_{s, \lambda} \in B_\lambda^* \cap V_v(\lambda)_w^\perp\} 
\end{align*} 
we see that $\B_\lambda = \B_{\lambda, w}^* \sqcup \B_{\lambda, w}^\perp$, and 
$\B_{\lambda, w}^\perp$ is a basis of $V_v(\lambda)_w^\perp$, while the restrictions of 
the functionals of $\B_{\lambda, w}^*$ to $V_v(\lambda)_w$ form a basis of 
$V_v(\lambda)_w^*$.  Hence it is possible to define $\A$-forms of arbitrary quantum 
Schubert varieties, and by definition the image of matrix coefficients $b_{s, \lambda}^*$
with $(s, \lambda) \in \tilde S_{w_0, I}^w$ form an $\A$-basis of the $\A$-algebra 
$\O_\A[G/P_I]_w$. We keep abusing notation and denote the image of $b_{s, \lambda}^*$ in 
this quotient by the same symbol.

\begin{Proposition} 
\label{P:basis-product} 
Let $I$ be a finite set of fundamental weights, let $w \in W^I$, and let $w_0$ be a 
decomposition of the longest word of $W$ adapted to $w$. Then for every $(s,
\lambda), (s', \lambda'), (s'', \lambda'') \in \tilde S_{w_0, I}^w$ such that $(s'',
\lambda'') <_{\lex} (s + s', \lambda + \lambda')$ there exist $c_{s, s'}^{s''} \in \A$ 
and $c = c((s,\lambda),(s',\lambda')) \in \ZZ$ such that 
\begin{align}
\label{product-formula} 
b_{(s, \lambda)}^* b_{(s', \lambda')}^* 
  = v^c b_{(s+s', \lambda + \lambda')}^* 
    + \sum_{s'' <_{\lex} s+s'} c_{s,s'}^{s''} b_{(s'', \lambda + \lambda')}^*.  
\end{align} 
\end{Proposition} 
\begin{proof} 
Let $\k = \CC(t^{1/d})$, where $d$ is the length of the longest root of $\g$. Then there
is an obvious map $\QQ(v) \to \k$ sending $v$ to $t$. This induces a morphism of
Hopf algebras $U_v(\g) \hookrightarrow U_t(\g)$. The canonical basis $\B$ of $U_v(\g)$
maps to a basis of $U_t(\g)$, and since Demazure modules are preserved by extension of
scalars, \ref{T:A-crystal-basis} still holds over $U_t(\g)$ (see the reference given
there). Thus there is an obvious injective $\A$-linear morphism from $C_\A^+(\lambda)$ to
$C_t^+(\lambda)$, and hence an injective $\A$-linear morphism $\O_\A[G/B] \to \O_t[G/B]$;
this morphism is multiplicative since the extensions $U_\A \hookrightarrow U_v(\g)
\hookrightarrow U_t(\g)$ are morphisms of coalgebras. Thus it is enough to establish that
the formula holds in $\O_t[G/B]$, which is done in \cite{C}*{Proposition 2.1 and Theorem 
2.3}.
\end{proof} 

\begin{Theorem}
\label{Schubert-deg}
The quantum Schubert variety $\O_q[G/P_I]_w$ degenerates to a quantum affine toric 
variety with associated semigroup $\tilde S^w_{w_0, I}$. In particular any quantum 
Schubert variety has property $\chi$, finite local dimension, the AS-Cohen-Macaulay 
property, and is a maximal order in its skew-field of fractions.
\end{Theorem}
\begin{proof}
By Proposition \ref{P:basis-product} $\O_q[G/P_I]_w$ has an ordered $\tilde 
S^w_{w_0,I}$-basis with respect to $\leq_{\lex}$,so the first part of the statement
follows from Proposition \ref{lex-degeneration}.

Since $\tilde S^w_{w_0, I}$ is a normal semigroup, we know by Proposition 
\ref{properties-of-qatv} that the associated graded ring of the quantum Schubert variety
has property $\chi$, finite local dimension and the AS-Cohen-Macaulay property, which 
$\O_q[G/P_I]_w$ inherits by Theorem \ref{transfer}. Also by Proposition 
\ref{properties-of-qatv}, the quantum affine toric variety is a maximal order in its 
ring of fractions, and it follows from \cite{Mau}*{Chapitre IV, Proposition 2.1 and 
Chapitre V, Corollaire 2.6} that $\O_q[G/P_I]_w$ is also a maximal order. 
\end{proof}


%\newpage 
\section{Quantum affine toric degeneration of an algebras with a straightening law over 
a distributive lattice}
\label{richardson}

We finish this article by relating our previous work \cite{RZ} with the material found 
here. In that article we proved that Richardson varieties in the quantum grassmanian as 
defined in \cite{RZ}*{Definition 5.2.1} are symmetric quantum graded algebras with a 
straightening law over a poset \cite{RZ}*{Definition 3.1}, and that they satisfy a 
technical ``Condition (C)'' \cite{RZ}*{Definition 4.1}. In \cite{RZ}*{Theorem 4.9} we 
proved that, in the language of the present article, all symmetric quantum graded ASL's 
satisfying condition (C) have quantum toric degenerations. We now prove a similar result
under conditions milder than (C). In order to do this, we place 
ourselves in the context of \cite{RZ2}*{section 3}, the reader is referred to it for 
definitions and unexplained notation.

\paragraph
For the rest of this subsection $\L$ denotes a finite distributive lattice. Given $x,y 
\in \L$ we denote their meet by $x \meet y$ and their join by $x \join y$, as usual. 
Recall that the rank of $\L$ is defined as the length of any maximal chain in $\L$, which
we denote by $r$. We also denote by $J(\L)$ the set of join irreducible elements of 
$\L$, so $r = |J(\L)|$. We see $J(\L)$ as a poset with the order induced by $\L$, and
extend this to a total order, so $J(\L) = \{p_1, \ldots, p_n\}$ and $p_i < p_j$ implies 
$i < j$. Birkhoff's representation theorem states that $\L$ is isomorphic to the lattice
of poset ideals of $J(\L)$, through the assignation $l \in \L \mapsto \{p \in J(I) \mid 
p \leq l\}$.

\paragraph 
As shown in \cite{RZ2}*{section 3.3}, associated to $\L$ there is a semigroup $S(\L)$, 
defined as follows: $S(\L)$ is the abelian semigroup generated by $\L$ modulo the 
relations $\{x + y \equiv (x \meet y) + (x \join y) \mid x,y \in \L\}$.
The map $\iota: \L \to \NN_0^{r}$ defined as
\[
  \iota(l) = \sum_{\{i : p_i \leq l\}} e_i,
\]
for each $l \in L$ is injective and a lattice morphism. 
By \cite{RZ2}*{Proposition 3.3.3}, $\iota$ induces an injective semigroup morphism 
$\iota: S(\L) \to \NN_0^{r}$ whose image is a full normal subsemigroup, so $S(\L)$ is a
normal affine semigroup. 

Set $n = |\L|$; write $\L = \{x_1, \ldots, x_n\}$, and set $s_i = \iota(x_i)$ for each 
$1 \leq i \leq n$. Let $\pi: \NN_0^n \to S(\L)$ be the semigroup morphism induced 
by the assignation $e_i \mapsto s_i$ for all $i$. It follows from 
\cite{RZ2}*{Lemma 3.3.2} that every element of $S(\L)$ can be 
written in a unique way as $s = \sum_{i=1}^n m_i s_i$ with the condition that the set 
$\{x_i \mid m_i \neq 0\}$ is a chain of $\L$. Setting $t(s) = (m_1, \ldots, m_n)$ we 
obtain a section $t: S(\L) \to \NN_0^n$ of $\pi$. We refer to $t$ as the \emph{standard 
section} of $\pi$. 

\paragraph
For every pair of elements $x,y \in \L$ we define
\begin{align*} 
  R(x,y) &= \{(z,z') \in \L \times \L \mid z < x, y  <z'\}.  
\end{align*} 
We write $R(x,y)^*$ for $R(x,y) \setminus (x \meet y, x \join y)$.

\paragraph
We continue to denote by $A$ an $\NN_0^r$-graded noetherian algebra.
Suppose that there is an injective function $[-]: \L \to A$. A \emph{standard monomial} 
in $A$ is a product of the form $[x_0] [x_1] \ldots [x_n]$ with $x_{i} \leq x_{i+1}$ for 
all $1 \leq i < n$. 
\begin{Definition*}
\label{D:symm-q-gr-asl} 
Let $\L$ be a finite distributive lattice, and let $[-]: \L \to A$ be an injective 
function. We say that $A$ is a 
\emph{symmetric quantum graded algebra with a straightening law over $\L$}, or symmetric 
quantum ASL for short, if the following hold.

\begin{enumerate} 
  \item The set $\{[x] \mid x \in \L\}$ consists of homogeneous elements of nonzero
    degree that generate $A$ as an algebra.

  \item The set of standard monomials is linearly independent.

  \item For every $x,y \in \L$ there exist $\{c^{z,z'}_{x,y}\}_{(z,z') \in R(x,y)^*}
  \subset \k$ and $c_{x,y} \in \k^\times$, such that 
  \begin{align*} 
    [y][x] = c_{x,y} [x][y] + \sum_{(z,z') \in R(x,y)^*} c^{z,z'}_{x,y} [z][z'].  
  \end{align*}

  \item For every pair of incomparable elements $x,y \in \L$ there exist
  $\{d^{z,z'}_{x,y}\}_{(z,z') \in R(x,y)^*} \subset \k$ and $d_{x,y} \in \k^\times$, such
  that
  \begin{align*} 
    [x][y] = d_{x,y} [x \meet y][x \join y] +
    \sum_{(z,z') \in R(x,y)^*} d^{z,z'}_{x,y} [z][z'].  \end{align*}
  \end{enumerate} 
\end{Definition*}
\begin{Remark*}
The word \emph{symmetric} is there to distinguish this class from the more general class
of quantum graded ASL's over a poset, defined in \cite{LR1}*{Definition 1.1.1}. Also
notice that the definition of a symmetric quantum ASL over a general poset $P$ given in
\cite{RZ}*{Definition 3.1} is slightly more general, as it does not require that 
$d_{x,y} \neq 0$. This simple requirement is what takes the place of the much more 
technical condition $(C)$.
\end{Remark*}

\begin{Theorem} 
\label{T:ASL-are-dominated} 
If $A$ is a symmetric quantum graded ASL over $\L$ then it has a quantum affine toric 
degeneration with associated semigroup $S(\L)$.
\end{Theorem} 
\begin{proof} 
Let $C = \{\iota(x) + \iota(y) \mid x,y \in \L\}$. This is a finite set, and so by 
reasoning as in the proof of Proposition \ref{lex-degeneration} there is a morphism 
$\phi: \NN_0^{\rk \L} \to \NN_0$ such that $\xi <_{\lex} \nu$ implies $\phi(\xi) < 
\phi(\nu)$ for all $\xi, \nu \in C$. 

Let $x,y \in \L$ and let $(z,z') \in R(x,y)^*$. Using Birkhoff's representation theorem, 
it follows from the defintion of $\iota$ that $\iota(z') - \iota(y) <_{\lex} \iota(x) - 
\iota(z)$, or equivalently $\iota(z) + \iota(z') <_{\lex} \iota(x) + \iota(y)$. Thus by 
definition $A$ is of $(S,\phi)$-type with generators $b_i = [x_i]$. Also, by 
\cite{LR1}*{Proposition 1.14}, standard monomials form a basis of $A$. The set of 
standard monomials is the generating set associated to the standard section $t$ as 
defined in \ref{P:equivalent-qatd}, so the proposition in that paragraph implies that
$A$ has a quantum toric degeneration with associated semigroup $S(\L)$.
\end{proof}

%\newpage
\begin{bibdiv} 
\begin{biblist} 
\bib{AZ}{article}{ 
  author={Artin, M.}, 
  author={Zhang, J. J.}, 
  title={Noncommutative projective schemes}, 
  journal={Adv. Math.}, 
  volume={109}, 
  date={1994}, 
  number={2},
  pages={228--287}, 
}

    
\bib{BS}{book}{ 
  author={Brodmann, M. P.}, 
  author={Sharp, R. Y.}, 
  title={Local cohomology: an algebraic introduction with geometric applications}, 
  series={Cambridge Studies in Advanced Mathematics}, 
  volume={60}, 
  publisher={Cambridge University Press},
  place={Cambridge}, 
  date={1998}, 
  pages={xvi+416}, 
}

\bib{BH}{book}{ 
  author={Bruns, Winfried}, 
  author={Herzog, J{\"u}rgen},
  title={Cohen-Macaulay rings}, 
  series={Cambridge Studies in Advanced Mathematics},
  volume={39}, 
  publisher={Cambridge University Press}, 
  place={Cambridge}, 
  date={1993},
  pages={xii+403}, 
}

\bib{BB}{book}{
   author={Bj{\"o}rner, Anders},
   author={Brenti, Francesco},
   title={Combinatorics of Coxeter groups},
   series={Graduate Texts in Mathematics},
   volume={231},
   publisher={Springer},
   place={New York},
   date={2005},
   pages={xiv+363},
}

\bib{C}{article}{ 
  author={Caldero, Philippe}, 
  title={Toric degenerations of Schubert varieties}, 
  journal={Transform. Groups}, 
  volume={7}, 
  date={2002}, 
  number={1},
  pages={51--60}, 
}



\bib{RGS}{book}{
   author={Rosales, J. C.},
   author={Garc{\'{\i}}a-S{\'a}nchez, P. A.},
   title={Finitely generated commutative monoids},
   publisher={Nova Science Publishers, Inc., Commack, NY},
   date={1999},
   pages={xiv+185},
}
    



\bib{GL}{article}{
   author={Gonciulea, N.},
   author={Lakshmibai, V.},
   title={Degenerations of flag and Schubert varieties to toric varieties},
   journal={Transform. Groups},
   volume={1},
   date={1996},
   number={3},
   pages={215--248},
}

\bib{Jan}{book}{ 
  author={Jantzen, Jens Carsten}, 
  title={Lectures on quantum groups},
  series={Graduate Studies in Mathematics}, 
  volume={6}, 
  publisher={American Mathematical Society}, 
  place={Providence, RI}, 
  date={1996}, 
  pages={viii+266}, 
}

\bib{JZ}{article}{ 
  author={J{\o}rgensen, Peter}, 
  author={Zhang, James J.},
  title={Gourmet's guide to Gorensteinness}, 
  journal={Adv. Math.}, 
  volume={151},
  date={2000}, 
  number={2}, 
  pages={313--345}, 
}

\bib{K1}{article}{ 
  author={Kashiwara, Masaki}, 
  title={The crystal base and Littelmann's refined Demazure character formula}, 
  journal={Duke Math. J.}, 
  volume={71}, 
  date={1993},
  number={3}, 
  pages={839--858}, 
}

\bib{K}{article}{ 
  author={Kashiwara, Masaki}, 
  title={On crystal bases}, 
  conference={
    title={Representations of groups}, 
    address={Banff, AB}, date={1994}, }, 
  book={
    series={CMS Conf. Proc.}, 
    volume={16}, publisher={Amer. Math. Soc.},
    place={Providence, RI}, 
    }, 
  date={1995}, 
  pages={155--197}, 
}


\bib{qLR}{article}{ 
  author={Lakshmibai, V.}, 
  author={Reshetikhin, N.}, 
  title={Quantum flag and Schubert schemes}, 
  conference={ 
    title={ physics}, 
    address={Amherst, MA},
    date={1990}, 
  }, 
  book={ 
    series={Contemp. Math.}, 
    volume={134}, 
    publisher={Amer. Math. Soc.}, 
    place={Providence, RI}, 
    }, 
  date={1992}, 
  pages={145--181}, 
}

\bib{LR1}{article}{ 
  author={Lenagan, T. H.}, 
  author={Rigal, L.}, 
  title={Quantum graded algebras with a straightening law and the AS-Cohen-Macaulay   
    property for quantum determinantal rings and quantum Grassmannians}, 
  journal={J. Algebra}, 
  volume={301},
  date={2006}, 
  number={2}, 
  pages={670--702}, 
}

\bib{Lit}{article}{ 
  author={Littelmann, P.}, 
  title={Cones, crystals, and patterns},
  journal={Transform. Groups}, 
  volume={3}, 
  date={1998}, 
  number={2}, 
  pages={145--179}, 
}

\bib{Lu}{book}{ 
  author={Lusztig, George}, 
  title={Introduction to quantum groups},
  series={Modern Birkh\"auser Classics}, 
  note={Reprint of the 1994 edition},
  publisher={Birkh\"auser/Springer, New York}, 
  date={2010}, 
  pages={xiv+346}, 
}


\bib{Mau}{book}{
   author={Maury, Guy},
   author={Raynaud, Jacques},
   title={Ordres maximaux au sens de K. Asano},
   language={French},
   series={Lecture Notes in Mathematics},
   volume={808},
   publisher={Springer},
   place={Berlin},
   date={1980},
   pages={viii+192},
}

\bib{MR}{book}{ 
  author={McConnell, J. C.}, 
  author={Robson, J. C.}, 
  title={Noncommutative Noetherian rings}, 
  series={Graduate Studies in Mathematics}, 
  volume={30},
  edition={Revised edition}, 
  note={With the cooperation of L. W. Small},
  publisher={American Mathematical Society}, 
  place={Providence, RI}, 
  date={2001},
  pages={xx+636}, 
}

\bib{MS}{book}{ 
  author={Miller, Ezra}, 
  author={Sturmfels, Bernd}, 
  title={Combinatorial commutative algebra}, 
  series={Graduate Texts in Mathematics}, 
  volume={227},
  publisher={Springer-Verlag}, 
  place={New York}, 
  date={2005}, 
  pages={xiv+417}, 
}


\bib{VO}{book}{ 
  author={N{\u{a}}st{\u{a}}sescu, Constantin}, 
  author={Van Oystaeyen, F.},
  title={Graded and filtered rings and modules}, 
  series={Lecture Notes in Mathematics},
  volume={758}, 
  publisher={Springer}, 
  place={Berlin}, 
  date={1979}, 
  pages={x+148}, 
}



\bib{NV}{book}{ 
  author={N{\u{a}}st{\u{a}}sescu, Constantin}, 
  author={Van Oystaeyen, Freddy}, 
  title={Methods of graded rings}, 
  series={Lecture Notes in Mathematics},
  volume={1836}, 
  publisher={Springer-Verlag}, 
  place={Berlin}, 
  date={2004}, 
  pages={xiv+304},
}



\bib{RZ}{article}{
  author={Rigal, L.}, 
  author={Zadunaisky, P.}, 
  title={Quantum analogues of Richardson varieties in the grassmannian and their toric
  degeneration},
  journal={J. Algebra}, 
  volume={372}, 
  date={2012}, 
  pages={293--317}, 
}

\bib{RZ2}{article}{
  author={Rigal, L.}, 
  author={Zadunaisky, P.}, 
  title={Twisted Semigroup Algebras},
  journal={Alg. Rep. Theory},
  year={2015},
  doi={10.1007/s10468-015-9525-z},
}

\bib{S}{article}{ 
  author={So{\u\i}bel{\cprime}man, Ya. S.}, 
  title={On the quantum flag manifold}, 
  language={Russian}, 
  journal={Funktsional. Anal. i Prilozhen.},
  volume={26}, 
  date={1992}, 
  number={3}, 
  pages={90--92}, 
  translation={
    journal={Funct. Anal. Appl.}, 
    volume={26}, 
    date={1992}, 
    number={3},
    pages={225--227}, 
  }, 
}


\bib{W}{book}{ 
  author={Weibel, Charles A.}, 
  title={An introduction to homological algebra}, 
  series={Cambridge Studies in Advanced Mathematics}, 
  volume={38},
  publisher={Cambridge University Press}, 
  place={Cambridge}, 
  date={1994}, 
  pages={xiv+450}, 
}
\end{biblist} 
\end{bibdiv}
\end{document}





Our objective in the present article is to adapt to the quantum setting a technique that 
has been a staple of algebraic geometry during the last thirty years, namely that of 
toric degenerations, and apply this technique to the study of quantum analogues of 
Schubert varieties in arbitrary flag varieties. This statement is made in the spirit of 
noncommutative algebraic geometry, in the sense that we see quantum varieties as defined 
implicitly by a noncommutative deformation of the homogeneous coordinate ring of a 
classical variety.

By degeneration we refer to the process of deforming an algebraic variety $V$ over a 
field $\k$ into a new variety $T$. In the simpler examples, this is done by finding a 
flat regular map $p: W \to \AA^1(\k)$ such that a generic fiber is 
isomorphic to $V$ but a special fiber, usually $p^{-1}(0)$, is isomorphic to $T$. If we 
have such a flat deformation of $V$ into $T$, then we can show that $V$ has some nice 
property by showing first that $T$ has the desired property, and then using flatness to 
show that this proprety extends to all fibers. A \emph{toric} degeneration is one in
which the target $T$ of the degeneration is a toric variety. The underlying combinatorial
structure of toric varieties is well documented, see for example \cite{MS}*{section II}, 
and has been aplied to study many of their homological and geometrical properties with 
great success, which makes them an ideal target for a degeneration.

The degeneration process can be carried out by purely algebraic means. In this case, the 
transfer of properties from $T$ to $V$ can be understood in algebraic terms as the 
classic idea of studying a ring $A$ by filtering it and passing to an associated graded 
ring. Of course this idea is not new in noncommutative algebra. Our contribution is to 
show that degeneration techniques are well suited to study the regularity properties of
homological nature and geometrical flavor usually found in noncommutative algebraic 
geometry, such as being AS-Cohen-Macaulay, having finite local dimension, etc. Once
we have proved this, we introduce the notion of an algebra with a \emph{quantum affine 
toric degeneration}, i.e. a graded algebra with a filtration by graded subspaces whose
associated graded algebra is a quantum affine toric variety. These quantum varieties 
were studied in our previous article \cite{RZ2}, where we showed that they share most 
of the combinatorial and homological properties of their classical counterparts, and 
hence are well-suited to be the target of a noncommutative degeneration. As an example, 
we show that Schubert subvarieties of quantum flag varieties have quantum affine toric 
degenerations and thus establish that they are AS-Cohen-Macaulay and maximal orders
in their skew-fields of fractions.

Since we were interested in applying these techniques to quantum flag and Schubert 
varieties, the works of Gonciulea and Lakshmibai \cite{GL} and Caldero \cite{C} served 
as guiding examples. In particular the last article uses techniques from quantum groups 
to prove that Schubert subvarieties of flag varieties have toric degenerations, and 
was the main inspiration to the present article.