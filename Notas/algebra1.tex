%%%%%%%%%%%%%%%%%%%%%% Generalities %%%%%%%%%%%%%%%%%%5
\documentclass[11pt,fleqn]{article}
\usepackage[paper=a4paper]
  {geometry}

\pagestyle{plain}
\pagenumbering{arabic}
%%%%%%%%%%%%%%%%%%%%%%%%%%%%%%%%
% esto tiene que estar, en este orden...
\usepackage[small]{titlesec}
\usepackage{paragraphs}
\usepackage{hyperref}
\usepackage{amsthm,thmtools}

%\linespread{1.2}
\setlength{\parskip}{1.2ex}

\usepackage[latin1,utf8]{inputenc}
\usepackage[spanish,english]{babel}
\usepackage{enumerate}
\usepackage{mathpazo}
\usepackage{euler}
\usepackage[alphabetic,initials]{amsrefs}
\usepackage{amsfonts,amssymb,amsmath}
\usepackage{dsfont}
\usepackage{mathtools}
\usepackage{graphicx}
\usepackage[poly,arrow,curve,matrix]{xy}
\usepackage{wrapfig}
%\swapnumbers

% numbered versions
\declaretheoremstyle[headformat=swapnumber, spaceabove=\paraskip,
bodyfont=\itshape]{mystyle}
\declaretheorem[name=Lemma, sibling=para, style=mystyle]{Lemma}
\declaretheorem[name=Proposition, sibling=para, style=mystyle]{Proposition}
\declaretheorem[name=Theorem, sibling=para, style=mystyle]{Theorem}
\declaretheorem[name=Corolllary, sibling=para, style=mystyle]{Corollary}
\declaretheorem[name=Definition, sibling=para, style=mystyle]{Definition}

% unnumbered versions
\declaretheoremstyle[numbered=no, spaceabove=\paraskip,
bodyfont=\itshape]{mystyle-empty}
\declaretheorem[name=Lemma, style=mystyle-empty]{Lemma*}
\declaretheorem[name=Proposition, style=mystyle-empty]{Proposition*}
\declaretheorem[name=Theorem, style=mystyle-empty]{Theorem*}
\declaretheorem[name=Corollary, style=mystyle-empty]{Corollary*}
\declaretheorem[name=Definition, style=mystyle-empty]{Definition*}

% plain style
\declaretheoremstyle[
        headformat={{\bfseries\NUMBER.}{\itshape\NAME}\NOTE\ignorespaces},
        spaceabove=\paraskip, 
        headpunct={.},
        headfont=\itshape,
        bodyfont=\normalfont
        ]{mystyle-plain}
\declaretheorem[sibling=para, style=mystyle-plain]{Example}
\declaretheorem[sibling=para, style=mystyle-plain]{Remark}

% proofs, just as in amsthm but with adapted spacing
\makeatletter
\renewenvironment{proof}[1][\proofname]{\par
  \pushQED{\qed}%
  \normalfont \topsep.75\paraskip\relax
  \trivlist
  \item[\hskip\labelsep
        \itshape
    #1\@addpunct{.}]\ignorespaces
}{%
  \popQED\endtrivlist\@endpefalse
}
\makeatother

%%%%%%%%%%%%%%%%%%%%%%%%%%% The usual stuff%%%%%%%%%%%%%%%%%%%%%%%%%
\newcommand\NN{\mathbb N}
\newcommand\CC{\mathbb C}
\newcommand\QQ{\mathbb Q}
\newcommand\RR{\mathbb R}
\newcommand\ZZ{\mathbb Z}
\newcommand\KK{\mathbb K}

\newcommand\maps{\longmapsto}
\newcommand\ot{\otimes}
\renewcommand\to{\longrightarrow}
\renewcommand\phi{\varphi}
\newcommand\stack[2]{\genfrac{}{}{0pt}{2}{#1}{#2}}
\newcommand\uu[1]{\underline{#1}}

%%%%%%%%%%%%%%%%%%%%%%%%% Specific notation %%%%%%%%%%%%%%%%%%%%%%%%%

\newcommand\A{\mathcal A}
\newcommand\B{\mathcal B}
\renewcommand\L{\mathcal L}
\newcommand\R{\mathcal R}
\newcommand\F{\mathcal F}
\renewcommand\O{\mathcal O}
\newcommand\I{\uu I}
\renewcommand\b{\mathfrak{b}}
\newcommand\g{\mathfrak{g}}
\newcommand\h{\mathfrak h}
\newcommand\sq{\square}
\newcommand\n{\mathfrak{n}}
\newcommand\opp{\mathsf{opp}}
\DeclareMathOperator\Id{\mathsf{Id}}
\DeclareMathOperator\st{\mathsf{st}}

\newcommand\C{\mathsf C}
\newcommand\K{\mathsf K}
\newcommand\D{\mathsf D}

\DeclareMathOperator\Mod{\mathsf{Mod}}
\DeclareMathOperator\Hom{\mathsf{Hom}}
\DeclareMathOperator\Ext{\mathsf{Ext}}
\DeclareMathOperator\Tor{\mathsf{Tor}}

\DeclareMathOperator\Gl{\mathsf{Gl}}
\DeclareMathOperator\GrHom{\underline{\mathsf{Hom}}}
\DeclareMathOperator\GrExt{\underline{\mathsf{Ext}}}
\DeclareMathOperator\GrTor{\underline{\mathsf{Tor}}}

\DeclareMathOperator\Ab{\mathsf{Ab}}
\DeclareMathOperator\ShHom{\mathcal Hom}
\DeclareMathOperator\ShExt{\mathcal Ext}
\DeclareMathOperator\ShTor{\mathcal Tor}

\DeclareMathOperator\pd{pd}
\DeclareMathOperator\id{id}
\DeclareMathOperator\rank{rank}
\DeclareMathOperator\Spec{Spec}
\DeclareMathOperator\supp{supp}
\DeclareMathOperator\ann{ann}
\DeclareMathOperator\im{Im}
\DeclareMathOperator\gr{gr}

%%%%%%%%%%%%%%%%%%%%%%%%%%%%%%%%%%%%%% TITLES %%%%%%%%%%%%%%%%%%%%%%%%%%%%%%


\title{Sobre el cambio de programa de Álgebra I}
\date{28/04/2013}
\author{PZ}
\begin{document}
\maketitle

El objetivo de este documento es dejar por escrito los puntos de acuerdo entre los
graduados respecto del proyecto de cambio de programa de Álgebra I. En general estamos de
acuerdo en agregar el taller de computación, pero estamos en contra de que saquen
contenidos de la materia. 

Quimey Vivas puso a disposición de los graduados el documento elaborado por la comisión
encargada de estudiar el cambio de programa. La mayor parte del documento se refiere a
problemas de implementación de la nueva modalidad, pero al principio se habla de
"reorganizar" los contenidos, y en la última sección se propone un programa tentativo,
detallando los contenidos por clas. 

Los cambios respecto de la organización actual son: 
\begin{enumerate}
  \item Combinatoria deja de ser un capítulo de la materia, y sus temas se distribuyen entre
  los demás. Una parte, bosones, pasa a formar parte de la introducción de la materia
  Probabilidades y Estadística.

  \item Números complejos deja de ser un capítulo de la materia y su contenido pasa a
  formar parte de la introducción al capítulo de Polinomios. El tratamiento de raíces de
  la unidad se deja para Álgebra II. 
\end{enumerate}
Los dos cambios presentan el mismo patrón, de tomar dos de los temas principales de la
organización actual de la materia, y dividir sus contenidos específicos en "los que
aparecen de forma natural" en los otros cuatro capítulos y "los que deberían dejarse para
más adelante". En ambos casos el argumento para postergar contenidos a materias futuras es
que los alumnos "tienen dificultad para comprender" estos temas, que requieren un poco más
de madurez matemática, y que de esta forma se pierde tiempo en la materia.

Nos resulta preocupante el hecho de que se proponga sacar contenidos de la materia sin una
justificación más extensa. Los argumentos que sí están presentes en el documento nos
parecen insuficientes.

Sobre el argumento de pérdida de tiempo, la mayoría de los graduados coincidimos en que
Álgebra I es una materia en la que el tiempo suele sobrar, y se dedican muchas clases a
repasar temas y a comentar "curiosidades sobre los números primos" y "chismes sobre
crterios de irreducibilidad". Por otro lado, Álgebra II y Probabilidades y Estadística son
materias en las que en general el tiempo \emph{falta}, y nos parece poco aconsejable sumar
temas a esas materias.

Acerca de la dificultad que representan para los alumnos los temas de combinatoria, y en
menor medida números complejos, no nos parece que esa sea una razón para evitar estos
contenidos. En cambio se podría aprovechar el taller de computación para explorar ciertos
temas de probabilidad, utilizando fuerza bruta, y no tan bruta, para hacer listas de
casos, distinguir casos que cumplan ciertas características, etc. Por otro lado, los
alumnos vienen trabajando con números complejos desde el CBC, así que no parece razonable
suponer

Finalmente queremos señalar que si bien muchos de los contenidos propuestos en el
programa, como discutir algoritmos de sorting tipo Fibonacci, Hanoi, etc. o acerca de
propiedades de los números primos, no figuran en el programa oficial de la materia, muchas
veces forman parte de la discusión en las clases prácticas sin generar dificultades para
terminar el programa a tiempo.
\end{document}
