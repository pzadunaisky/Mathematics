\documentclass[11pt]{amsart}
\usepackage[margin=0.5in,left=5mm,right=5mm,paperwidth=150mm, paperheight=210mm]{geometry}
\usepackage{multicol}
\usepackage{blindtext}
\usepackage{amsfonts}
\usepackage[utf8]{inputenc}
\usepackage[T1]{fontenc}
% \usepackage{lmodern} % load a font with all the characters
\usepackage{mathtools}
\usepackage{enumitem}
% \usepackage{titlesec}

%\linespread{0.85}

\setlength{\columnsep}{0.5cm}
\setlength{\columnseprule}{0.8pt}
\setlength{\parindent}{0pt}

\setlist[description]{leftmargin=5pt,labelindent=\parindent}

\newcommand{\itt}[1]{\textit{#1}}
\newcommand{\btt}[1]{\textbf{#1}}

\makeatletter
\def\subsubsection{\@startsection{subsubsection}{3}%
  \z@{.5\linespacing\@plus.7\linespacing}{.1\linespacing}%
  {\normalfont\itshape}}
\makeatother

%%%%%%%%%%%%%%%%%%%%%%%%%
%%%%%%%%%%%%%%%%%%%%%%%%%

\newcommand\RR{{\mathbb{R}}}
\newcommand\QQ{{\mathbb{Q}}}
\newcommand\ZZ{{\mathbb{Z}}}
\newcommand\NN{{\mathbb{N}}}
\newcommand\CC{{\mathbb{C}}}
\newcommand\A{\mathcal{A}}
\newcommand\B{\mathcal{B}}

\newcommand\wt[1]{{(#1)}}
\newcommand\der[1]{\tfrac{\partial}{\partial#1}}
\newcommand\maxw{\mathsf{max}}
\newcommand\minw{\mathsf{min}}
\newcommand\tip{\mathsf{tip}}
\newcommand\rw{{\mathsf{R}}}
\newcommand\gldim{\mathsf{gldim}}
\newcommand\sop{\mathsf{sop}}

\renewcommand\epsilon{\varepsilon}
\newcommand\psm[1]{\begin{psmallmatrix}#1\end{psmallmatrix}}
\newcommand\sym{\mathrm{sym}}
%%%%%%%%%%%%%%%%%%%%%%%%%
%%%%%%%%%%%%%%%%%%%%%%%%%
\begin{document}
\begin{multicols}{2}

\section{Superficies de Riemann}
\subsection*{Toro pinchado.}
Consideramos el subconjunto de $\CC^2$,
{\small\[
 S=\{(z,w)\in\CC^2: w^2 = (z^2-1)(z^2-k^2)\}
\]}
donde $k\in\CC\setminus{1,-1}$. Topol\'ogicamente, este conjunto es un 
toro pinchado en dos puntos. 

Sea $x=(1+k)(1-k)$ y consideremos una
ra\'iz cuadrada $\mathsf{r}$ definida en $U=\CC\setminus\text{Recta}$ de 
manera que $x,1\in U$. Si tomamos $\epsilon$ muy chico, podemos 
asegurar que $(1+k+\epsilon)(1-k+\epsilon) = x + \epsilon(1+\epsilon^2)$ 
caiga siempre adentro de $U$. Tomamos $z=1+\epsilon$.
\[
 \begin{aligned}
  &(z^2-1)(z^2-k^2)\\&=\epsilon(\epsilon+2)(1+k+\epsilon)(1-k+\epsilon).
 \end{aligned}
\]
Llamando 
$h(\epsilon)=\mathsf{r}((\epsilon+2)(1+k+\epsilon)(1+k-\epsilon))$, 
tenemos que $h$ es holomorfa en un entorno del origen y que el punto 
$(w(\epsilon),1+\epsilon)\in S$ para $w(\epsilon) = 
\mathsf{r(\epsilon)}h(\epsilon)$. Hay un $\epsilon_0\in\CC$ que pertenece a
$\text{Recta}$, haciendo tender a $\epsilon$ a ambos lados de 
$\epsilon_0$ obtenemos que:

\itt{Si le pegamos una pequeña vuelta al $1$, el valor de $w$ cambia de 
signo}.

Por lo tanto si en el plano $z$ eliminamos el segmento que hay entre $1$ y 
$k$ y el segmento que hay entre $-k$ y $-1$, entonces lo que queda es un 
dominio en el que para cada valor de $z$ hay una elección coherente de un 
$w$ tal que $(z,w)\in S$ y la función $w(z)$ es continua.

Como hay dos signos posibles, poniendo tubos por los tajos que cortamos 
obtenemos el toro pinchado.

\itt{Mas info en la intro de ``Riemann Surfaces'' de C. Teleman.}
\newpage
\end{multicols}

\end{document}
