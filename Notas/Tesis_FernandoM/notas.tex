

%%%%%%%%%%%%%%%%%%%%%% Generalities %%%%%%%%%%%%%%%%%%5
\documentclass[11pt,fleqn]{article}
\usepackage[paper=a4paper]
  {geometry}

\pagestyle{plain}
\pagenumbering{arabic}
%%%%%%%%%%%%%%%%%%%%%%%%%%%%%%%%
% esto tiene que estar, en este orden...
\usepackage[small]{titlesec}
\usepackage{paragraphs}
\usepackage{hyperref}
\usepackage{amsthm,thmtools}

%\linespread{1.2}
\setlength{\parskip}{1.2ex}

\usepackage[utf8]{inputenc}
\usepackage[spanish,english]{babel}
\usepackage{enumerate}
\usepackage{mathpazo}
%\usepackage{euler}
\usepackage[alphabetic,initials]{amsrefs}
\usepackage{amsfonts,amssymb,amsmath}
\usepackage{dsfont}
\usepackage{mathtools}
\usepackage{graphicx}
\usepackage[poly,arrow,curve,matrix]{xy}
\usepackage{stmaryrd}
\usepackage{showlabels}
\usepackage{ulem}
%\swapnumbers

% numbered versions
\declaretheoremstyle[headformat=swapnumber, spaceabove=\paraskip,
bodyfont=\itshape]{mystyle}
\declaretheorem[name=Lemma, sibling=para, style=mystyle]{Lemma}
\declaretheorem[name=Proposition, sibling=para, style=mystyle]{Proposition}
\declaretheorem[name=Theorem, sibling=para, style=mystyle]{Theorem}
\declaretheorem[name=Corolllary, sibling=para, style=mystyle]{Corollary}
\declaretheorem[name=Definition, sibling=para, style=mystyle]{Definition}
%\declaretheorem[name=Example, sibling=para, style=mystyle]{Example}


% unnumbered versions
\declaretheoremstyle[numbered=no, spaceabove=\paraskip,
bodyfont=\itshape]{mystyle-empty}
\declaretheorem[name=Lemma, style=mystyle-empty]{Lemma*}
\declaretheorem[name=Proposition, style=mystyle-empty]{Proposition*}
\declaretheorem[name=Theorem, style=mystyle-empty]{Theorem*}
\declaretheorem[name=Corollary, style=mystyle-empty]{Corollary*}
\declaretheorem[name=Definition, style=mystyle-empty]{Definition*}
%\declaretheorem[name=Example, style=mystyle-empty]{Example*}

% plain style
\declaretheoremstyle[
        headformat={{\bfseries\NUMBER.}{\itshape\NAME}\NOTE\ignorespaces},
        spaceabove=\paraskip, 
        headpunct={.},
        headfont=\itshape,
        bodyfont=\normalfont
        ]{mystyle-plain}
\declaretheorem[sibling=para, style=mystyle-plain]{Example}
\declaretheorem[sibling=para, style=mystyle-plain]{Remark}

% proofs, just as in amsthm but with adapted spacing
\makeatletter
\renewenvironment{proof}[1][\proofname]{\par
  \pushQED{\qed}%
  \normalfont \topsep.75\paraskip\relax
  \trivlist
  \item[\hskip\labelsep
        \itshape
    #1\@addpunct{.}]\ignorespaces
}{%
  \popQED\endtrivlist\@endpefalse
}
\makeatother

%%%%%%%%%%%%%%%%%%%%%%%%%%% The usual stuff%%%%%%%%%%%%%%%%%%%%%%%%%
\newcommand\NN{\mathbb N}
\newcommand\CC{\mathbb C}
\newcommand\QQ{\mathbb Q}
\newcommand\RR{\mathbb R}
\newcommand\ZZ{\mathbb Z}
\newcommand\KK{\mathbb K}

\newcommand\maps{\longmapsto}
\newcommand\ot{\otimes}
\renewcommand\to{\longrightarrow}
\renewcommand\phi{\varphi}
\newcommand\op{\mathsf{op}}
\newcommand\blabla[1]{\noindent\framebox[1.05\width]{\texttt{Note: #1}}}
\renewcommand\ll{\llbracket}
\newcommand\rr{\rrbracket}
%%%%%%%%%%%%%%%%%%%%%%%%% Specific notation %%%%%%%%%%%%%%%%%%%%%%%%%
\renewcommand\k{\Bbbk}

\DeclareMathOperator\Mod{\mathsf{Mod}}
\DeclareMathOperator\Hom{\mathsf{Hom}}
\DeclareMathOperator\Ext{\mathsf{Ext}}
\DeclareMathOperator\Tor{\mathsf{Tor}}

\DeclareMathOperator\HOM{\underline{\mathsf{Hom}}}
\DeclareMathOperator\EXT{\underline{\mathsf{Ext}}}
\DeclareMathOperator\TOR{\underline{\mathsf{Tor}}}

\newcommand\join{\vee}
\newcommand\meet{\wedge}
\renewcommand\q{\mathbf{q}}

%%%%%%%%%%%%%%%%%%%%%%%%%%%%%%%%%%%%%% TITLES %%%%%%%%%%%%%%%%%%%%%%%%%%%%%%
\title{Tesis de Fernando Martin}
\date{[TesisFernandoM.tex]}
\author{Pablo Zadunaisky}
\begin{document}
\maketitle
\section*{Introducción}
\paragraph Linea 3, Pag. 5. La palabra "poder" está de más.

\paragraph Penúltima linea del primer párrafo "computos" (o algo así...).

\paragraph "finito-dimensionalidad", "remarcablemente", "concreniente" son todos
anglicismos.

\paragraph Varias palabras están mal separadas: "reescrit-ura", "pub-licación",
"anál-ogo", etc.

\section*{Introduction}

\paragraph Pg. 7, Al principio del 4to párrafo. In most cases passive voice is avoided.

\paragraph Mismo párrafo, última linea. "scalars", no "scalares".

\section{Preliminaries}

\paragraph Número del capítulo?

\subsection{The path algebra of a quiver}
\paragraph Hay dos pares de dos puntos en "$\alpha: 1 \to 1:$", pero el segundo confunde.
De hecho puede ser eliminado. Más en general, uno no termina párrafos en dos puntos, 
especialmente antes de enunciar un resultado; el clásico "Tenemos el siguiente: TEOREMA"
should be avoided...

\subsection{Completions}
\paragraph Definición de $k$-algebra norm?

\paragraph Qué es "$e$" en la definición de la norma?

\paragraph "plays nicely"? Mejor "compatible with".

\paragraph 1ra línea del 1er párrafo, "we have that" hay que eliminarlo.

\paragraph Última linea de Example 1.6 "this \sout{may} can only happen"

\paragraph Example 1.7, overflow.

\subsection{The Jacobian algebra of a quiver with potetntial}

\paragraph Example 1.10, además de los dos puntos innecesarios, donde dice "As any path
of length $n$..." debería decir $2n$, ídem más adelante.

\subsection{Ideal triangulations of surfaces}
\paragraph en la definición de \emph{arc}, "contractible into $M$" es poco claro. En el
copete de Figure 1.1 "contractible [...] to the set of marked points" no tiene sentido.

\paragraph En la demostración de 1.13, no me queda claro por qué "any marked point must 
be a vertex of the decomposition", muy posiblemente porque no estoy seguro de qué es una 
\emph{cell decomposition} (no se lo digas a Gabriel que me saca el aprobado de TA).

\subsection{The QP associated to an ideal triangulation}

\paragraph En el segundo párrafo, "section 3 \sout{from} of [LF09]".

\paragraph Ya lo hablamos, no me queda claro qué quiere decir que un ciclo "circle a 
puncture".

\section{The diamond lemma}

\paragraph "Extendable" quiere decir "extensible" en el espacio o en el tiempo, como una 
escalera o un permiso de recidencia, pero no a todos los elementos de un álgebra.

\subsection{Bergman's diamond lemma}

\paragraph Nunca definís $| \cdot |$. ES obvio que es la longitud, but still...

\paragraph En la definición de ambiguity, una boludez es que $\hat r'$ debería ser 
$\widehat{r'}$. Una no-boludez es que "resolvable" es un crimen de lesa humanidad. 
Podés decir "solvable" o "soluble", pero cada vez que escribís resolvable Shakespeare 
da media vuelta en su tumba. Sí, ya sé que Bergman usaba esa palabra. Bergman es un 
criminal.

\paragraph Tampoco es tan grave, pero Andrea me contagió su odio a la frase "we have 
that". El 75\% de las veces la podés sacar y el significado es el mismo.

\paragraph Example 2.4: "...our variables \sout{such} so that $x < y$...". También "the 
\emph{set} of $S$-irreducible monomials \sout{are} is exactly...".

\paragraph En los ejemplos muchas veces escribís tus relaciones en la forma 
$f_\sigma - W_\sigma$, pero tu enunciado dice $W_\sigma - f_\sigma$.

\paragraph Example 2.7, al pié de la página la igualdad $(xy-yx, yzx^n- zx^ny) = 
(xy-yx, yz-zy)$ es falsa... hay que tomar todos los $n$ hasta un cierto punto.

\paragraph Heuristic 2.8, al final del punto 3, "steps 1 and 2 \sout{as} to produce...". 
Ídem al final de la heurística [por cierto, me cabe que heurística empiece a tomar un lugar junto a Lema, Corolario, Proposiición, Teorema, blah...].

\subsection{A diamond lemma for path algebras}

\paragraph "further relations in the path algebra \sout{a lot of} many rewriting rules...". También "in a similar vein \sout{as} to [FFG93]".

\subsection{A topological diamond lemma}

\paragraph En la última fórmula, el cociente final debería ser por $(x^{n-1} - 1)$. 

\section{Jacobian algebras...}

\paragraph Cuarto párrafo de la introducción, overflow.

\subsection{Tetrahedron}

\paragraph "...12 rewriting rules that overlap \sout{themselves} in various ways."

\paragraph Las referencias de las imágenes faltan en todos los casos.

\paragraph Pag. 36 "... a basis for the Jacobian algebra is given by the set of 
irreducible paths". Creo que entendí algo mal, o falta algo. La frase es cierta si el 
álgebra Jacobiana es de la forma $k\langle Q \rangle/ I$, pero al tomar completación 
deja de ser cierto ¿no?. Siguiendo con el mismo argumento, en la pag. 37 
hablás de que el álgebra Jacobiana hereda una graduación del álgebra de caminos, pero el 
álgebra de caminos completada no es graduada, porque un camino infinito no tiene grado 
bien definido. Finalmente en la página 38 llamás "The Jacobian algebra" al cociente

\paragraph Hay un $-$ en la serie de Hilbert de la página 38 que sobra.

\paragraph Aunque no es parte d ela definición, uno suele mencionar si la filtración 
de un álgebra es completa o no, es decir si $A = \bigcup F_nA$. Si filtrás el álgebra 
de caminos completada por el largo, los caminos de largo infinito no están en ninguna 
capa de la filtración.

\subsection{Polygonal subdivision of surfaces}
\paragraph Por su estructura, la oración "With the sole exception of Ladkani's paper 
covered" tiene por sujeto a "Ladkani's paper"; parece que estuvieras diciendo que el 
artículo de Ladkani es una excepción a algo. Si entiendo bien, quisiste decir algo así 
como "Having covered the only case not considered in [Lad]...".

\paragraph "...this situation and \sout{lets us produce} produces..."

\subsection{A family of infinite-dimensional examples}

\paragraph ¿Qué es $\Omega(n^2)$?

\paragraph No entiendo el ejemplo 3.1 En particular no entiendo el rol que juega $K_8$.

\subsection{A family of finite-dimensional examples}

Esta sección es MUY linda.

\paragraph "the classification theorem" Agregar referencia.

\paragraph Proposition 3.3: "The loops (¿cuáles? $R$, $B$ o $R \cup B$?) divide the 
surface into a disjoint (non-overlapping??) family of regions...". En la demostracíón 
LaTeX te pasó una mala jugada con la ubicación de las imagenes... 

\paragraph "where each \sout{$S_i$} $C_i$...". Pag 48. "in the same band are 
\sout{distanced} separated by at least..." Pag. 49 "contained in \sout{a} the same 
region".


\end{document}
