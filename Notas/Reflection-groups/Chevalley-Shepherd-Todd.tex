%%%%%%%%%%%%%%%%%%%%%% Generalities %%%%%%%%%%%%%%%%%%5
\documentclass[11pt,fleqn]{article}
\usepackage[paper=a4paper]
  {geometry}

\pagestyle{plain}
\pagenumbering{arabic}
%%%%%%%%%%%%%%%%%%%%%%%%%%%%%%%%
\usepackage{notas}

%%%%%%%%%%%%%%%%%%%%%%%%%%% The usual stuff%%%%%%%%%%%%%%%%%%%%%%%%%
\newcommand\NN{\mathbb N}
\newcommand\CC{\mathbb C}
\newcommand\QQ{\mathbb Q}
\newcommand\RR{\mathbb R}
\newcommand\ZZ{\mathbb Z}
\renewcommand\k{\Bbbk}

\newcommand\A{\mathcal A}
\newcommand\M{\mathcal M}
\newcommand\Z{\mathsf Z}

\newcommand\maps{\longmapsto}
\newcommand\ot{\otimes}
\renewcommand\to{\longrightarrow}
\renewcommand\phi{\varphi}
\newcommand\id{\mathsf{Id}}
\newcommand\im{\mathsf{im}}
\newcommand\coker{\mathsf{coker}}
%%%%%%%%%%%%%%%%%%%%%%%%% Specific notation %%%%%%%%%%%%%%%%%%%%%%%%%

\declaretheorem[name=Ejercicio, style=mystyle-empty]{Ejercicio}


%%%%%%%%%%%%%%%%%%%%%%%%%%%%%%%%%%%%%% TITLES %%%%%%%%%%%%%%%%%%%%%%%%%%%%%%
\title{El teorema de Chevalley-Shepherd-Todd}
\date{Septiembre 2017}
\author{Pablo Zadunaisky}
\begin{document}
\maketitle

\section{El álgebra de polinomios}
Dado un número natural $n \in \NN$ notamos por $P_n$ el anillo de polinomios 
en $n$ variables $\CC[x_1, \ldots, x_n]$. Cuando $n \leq 3$ a veces notamos 
$x = x_1, y = x_2, z = x_3$. 

\paragraph
El anillo $P_n$ es un espacio vectorial complejo de dimensión infinita. Por 
ejemplo los conjuntos $M_2, M_3$ son bases de $P_2$ y $P_3$, respectivamente.
\begin{align*}
\M_2 
	&= \{x^i y^j \mid i,j \in \NN_0\}; \\
\M_3 
	&= \{x^i y^j z^k \mid i,j,k \in \NN_0\},
\end{align*}
y en general una base de $P_n$ es
\begin{align*}
\M_n 
	&= \{x_1^{i_1} x_2^{i_2} \cdots x_n^{i_n} \mid i_1, \ldots, i_n \in 
	\NN_0\}.
\end{align*}
Los elementos de esta base se llaman \newterm{monomios}.

\begin{Ejercicio}
Probar que $\M_n$ es una base de $P_n$.
\end{Ejercicio}

\paragraph
Cada monomio tiene un grado, que es la suma de los exponentes que aparecen en 
el. Por ejemplo el grado de $x^2y^7$ es $7+2 = 9$, y el grado de $x^i y^j z^k$
es $i+j+k$. Un polinomio $p \in P_n$ se dice \newterm{homogéneo de grado $d$} 
si es una combinación lineal de monomios de grado $d$; por ejemplo $x^2y^7 + 
5 x^3y^3z^3 + (i + \pi) z^9$ es un polinomio homogéneo de grado $9$. Notar que 
$0$ es un polinomio homogéneo de grado arbitrario.

\begin{Ejercicio}
Sean $n, d \in \NN$ y sea $P_n^{(d)}$ el conjunto de todos los polinomios 
homogéneos de grado $d$. Probar que $P_n^{(d)}$ es un subespacio vectorial 
de $P_n$.
\end{Ejercicio}

\paragraph
Como dijimos, el álgebra de polinomios es un espacio vectorial complejo de 
dimensión infinita, así que no tiene sentido preguntarse cuál es su dimensión.
Sin embargo el subespacio de los polinomios homogéneos de grado $d$ en
$P_n$ tiene dimensión finita; para ver esto alcanza con observar que 
$P_n^{(d)}$ tiene como base los monomios $x_1^{i_1} \cdots x_n^{i_n}$
tales que $i_1 + \cdots + i_n = d$; como cada $i_j$ es un entero no negativo
tenemos que $0 \leq i_j \leq d$, y por lo tanto hay a lo sumo $d^n$ monomios
distintos en $P_n^{(d)}$, con lo cual este espacio tiene dimensión a lo 
sumo $d^n$.

Este argumento alcanza para probar que $\dim P_n^{(d)} \leq d^n$, pero 
claramente es muy poco eficiente. Sería mejor calcular explícitamente la 
dimensión de este espacio, o lo que es lo mismo, el número de monomios 
distintos de grado $d$. Vamos a encontrar esta dimensión de dos formas
distintas en el caso $n = 3$.

\noindent \about{Cálculo de $\dim P_3^{(d)}$, primera forma} 
Nuestro problema es encontrar todos los monomios $x^i y^j z^k$ tales que
$i+j+k = d$, o lo que es lo mismo encontrar todas las $3$-uplas de enteros
$(i,j,k)$ que suman $d$. El problema es equivalente a distribuir $d$ piedras
en $3$ grupos separados, y si representamos cada piedra con un punto y cada
separación con una linea vertical, obtenemos diagramas como los siguientes
\begin{align*}
..|...|. && (i,j,k) &= (2,3,1) & d&=6 \\
....|....| && (i,j,k) &= (4,4,0) & d&=8 \\
.....|...|....... && (i,j,k) &= (5,3,7) & d&=15 
\end{align*}
Así, el problema se reduce a encontrar de cuántas formas podemos colocar $2$
barras entre $d$ puntos. La primera barra puede colocarse en $d+1$ lugares, y
la segunda en $d+2$ (se agregó la posibilidad de colocarla antes o después de 
la primera barra). Como nos dá lo mismo el orden de las barras, el resultado 
final es $\frac{(d+1)(d+2)}{2}$. Así $\dim P_3^{(d)} = \frac{(d+1)(d+2)}{2}$.

\noindent \about{Cálculo de $\dim P_3^{(d)}$, segunda forma}
Consideremos la serie en varias variables 
\[
A(x,y,z) 
= \frac{1}{1-x} \cdot \frac{1}{1-y} \cdot \frac{1}{1-z}
= (1 + x + x^2 + \cdots)(1 + y + y^2 + \cdots) (1 + z + z^2 + \cdots) 
\]
La serie $A(x,y,z)$ es la suma de todos los monomios de $M_3$, y cada uno
aparece exactamente una vez. Evaluando esta serie en una nueva variable $t$
tenemos
\[
A(t,t,t) = \frac{1}{(1-t)^3}
\]
Ahora cada monomio $x^i y^j z^k$ con $i+j+k = d$ aporta exactamente un sumando
de la forma $t^d$, por lo tanto el coeficiente de $t^d$ en $A(t,t,t)$ es 
exactamente el número de monomios de grado $d$. Usando el teorema de Taylor
tenemos que este coeficiente es 
\begin{align*}
 \frac{1}{d!} \partial_t^d (1-t)^{-3} 
 	= \frac{(d+2)!}{2 d!} 
 	= \frac{(d+1)(d+2)}{2}
\end{align*}

\begin{Ejercicio}
Calcular la dimensión de $P^{(d)}_n$ para $d, n \in \NN$ cualquiera.
\end{Ejercicio}

\paragraph
Todo polinomio $p \in P_n$ se puede escribir como una suma $p = p_{(0)} + 
p_{(1)} + \cdots + p_{(d)}$, con $p_{(i)}$ un polinomio homogéneo de grado $i$ 
y $p_{(d)} \neq 0$. Llamamos a $d$ el \newterm{grado} de $p$ y lo notamos por 
$\deg p$. Además esta escritura es única, lo que implica que $P_n = 
\bigoplus_{d \geq 0} P_n^{(d)}$. Es facil ver que si $f \in P_n^{(d)}$ y
$g \in P_n^{(e)}$ entonces $fg \in P_n^{(d+e)}$. 

\begin{Definicion*}
La \newterm{serie de Hilbert} de $P_n$ es la serie $\displaystyle 
\sum_{d = 0}^\infty (\dim_\CC P_n^{(d)}) t^d$.
\end{Definicion*}
La serie de Hilbert es una manera cómoda de recordar que, aunque $P_n$ tiene
dimensión infinita, tiene una descomposición como suma directa de espacios
de dimensión finita muy natural. Notar que gracias a la segunda forma de 
calcular $\dim_\CC P_n^{(d)}$ sabemos que la serie de Hilbert de $P_3$ es (la 
serie de Taylor de la función) $\frac{1}{(1-t)^3}$
\begin{Ejercicio}
Hallar un cociente de polinomios cuya serie de Taylor sea igual a la serie de 
Hilbert de $P_n$.
\end{Ejercicio}

\section{Polinomios simétricos}

\paragraph
Para cada $n \in \NN$ notamos por $S_n$ al conjunto de funciones biyectivas
de $\{1, 2, \ldots, n\}$ en sí mismo. Cada función $\sigma \in S_n$ induce
una función $f_\sigma: P_n \to P_n$ que se puede expresar en la base de 
monomios como
\begin{align*}
f_\sigma(x_1^{i_1} x_2^{i_2} \cdots x_n^{i_n}) 
	&= x_{\sigma(1)}^{i_1} x_{\sigma(2)}^{i_2} \cdots x_{\sigma(n)}^{i_n}
	= x_1^{i_{\sigma^{-1}(1)}}
		x_2^{i_{\sigma^{-1}(2)}} \cdots x_n^{i_{\sigma^{-1}(n)}}
\end{align*}

Por ejemplo, si $n = 2$ entonces $S_2$ consiste de dos funciones: la identidad
y aquella que intercambia $1$ y $2$. La primera induce la función identidad, 
mientras que la segunda es la transformación lineal dada por $x^i y^j \mapsto 
x^j y^i$. Si $n = 3$ entonces $S_3$ contiene tres funciones, entre ellas la 
función $\sigma$ tal que $\sigma(1) = 3, \sigma(2) = 1, \sigma(3) = 2$. Por 
definición $f_{\sigma}(x^i y^j z^k) = x^j y^k z^i$. Es ilustrativo encontrar
la imagen de $x^i y^j z^k$ por $f_\sigma$ para cada $\sigma \in S_3$.

\begin{Ejercicio}
Sea $n \in \NN$ y $\sigma \in S_n$. Probar las siguientes afirmaciones
\begin{enumerate}
\item $f_{\sigma} = \id_{P_n}$ si y solo si $\sigma$ es la función
identidad de $\{1, 2, \ldots, n\}$.

\item Si $p,q \in P_n$ entonces $f_\sigma(pq) = f_\sigma(p) f_\sigma(q)$,
y $f_\sigma(1) = 1$. En otras palabras, $f_\sigma$ es un morfismo de anillos.
Además $f_\sigma \circ f_\tau = f_{\sigma \circ \tau}$ para cualquier $\tau 
\in S_n$. 

\item Para todo $d \in \NN$ y todo $\sigma \in S_n$ vale que
$f_\sigma(P_n^{(d)}) = P_n^{(d)}$.
\end{enumerate}
\end{Ejercicio}


Un polinomio $p \in \CC[x,y]$ se dice simétrico si $p(x,y) = p(y,x)$. En otras
palabras, si escribimos a $p$ como suma de monomios, y reemplazamos cada $x$ 
por una $y$ y cada $y$ por una $x$ entonces obtenemos el mismo polinomio.
Estos son algunos polinomios simétricos en dos variables
\begin{align*}
x+y, 
	&& x^2 + y^2, 
	&& x^2 + 2xy + y^2, 
	&& 7x^3y + 7y^3x + x^2y^2 + 5xy + 1.
\end{align*}

Los polinomios simétricos en tres variables se definen análogamente: si 
intercambiamos $x$ e $y$, o $y$ y $z$, o $z$ y $x$, debemos obtener el mismo 
polinomio. Aquí tenemos algunos ejemplos de polinomios simétricos en tres 
variables.
\begin{align*}
&\mathbf{1},
	&\mathbf{x + y +z},
	&& x^2 + y^2 + z^2,
	&&\mathbf{xy + yz + xz}, \\
& x^3 + y^3 + z^3 
	&& x^2y + y^2z + xz^2
	&& \mathbf{xyz}
\end{align*}

\begin{Definicion}
Fijamos $n \in \NN$.
Un polinomio $p \in P_n$ se dice \newterm{simétrico} si $f_\sigma(p) = p$ para
cualquier $\sigma \in S_n$. Notamos por $A_n$ al conjunto de todos los 
polinomios simétricos.
\end{Definicion}

\begin{Ejercicio}
Probar que $A_n$ es un anillo. Además, si para cada $d \in \NN$ ponemos 
$A_n^{(d)} = P_n \cap A_n^{(d)}$ entonces $A = \bigoplus_{d \geq 0} A^{(d)}$.
\end{Ejercicio}

\paragraph
A partir de ahora fijamos $n$ y notamos $P = P_n, A = A_n$.
Nuestro objetivo es comprender el anillo $A$. Por ejemplo, nos gustaría 
encontrar alguna forma de describir todos sus elementos, o mejor aún hallar
bases de cada espacio $A^{(d)}$. La respuesta es un teorema conocido como
el Teorema Fundamental de las Funciones Simétricas.

Para empezar, señalamos que hay una forma muy sencilla de construir una función
simétrica. Dado un polinomio $p \in P$, el polinomio $M(p) = \frac{1}{n!}
\sum_{\sigma \in S_n} f_\sigma(p)$ es simétrico, y si $p$ es homogéneo 
entonces $M(p)$ también lo es. 

\begin{Definicion}
Para cada $1 \leq i \leq n$, el $i$-ésimo polinomio simétrico elemental es 
\begin{align*}
s_{n,i} = M(x_1 x_2 \cdots x_i) 
	= \frac{1}{n!} \sum_{j_1 < j_2 <\cdots < j_i} x_{j_1}x_{j_2} \cdots x_{j_i}
\end{align*}
\end{Definicion}
Notar que $\deg s_{n,i} = i$, y que $s_{n,i}(x_1, \ldots, x_{n-1}, 0) = 
s_{n-1,i}$ si $i < n$. Los polinomios simétricos para $n=3$ aparecen en 
negrita en el ejemplo anterior.

\begin{Teorema}
Cualquier polinomio simétrico se escribe como una combinación lineal de 
polinomios de la forma $s_{n,1}^{r_1} s_{n,2}^{r_2} \cdots s_{n,n}^{r_n}$ con 
cada $r_i \in \NN$, y esa escritura es única.
\end{Teorema}
Otra forma de enunciar este teorema es decir que el conjunto $\{s_i \mid
1 \leq i \leq n\}$ es un conjunto algebraicamente independiente de generadores
de $A$. Dicho de otra forma, para todo polinomio $p \in A_n$ existe un 
polinomio $q \in \CC[z_1, \ldots, z_n]$ tal que $q(s_1, \ldots, s_n) = p$

\begin{proof}
Demostramos el teorema por inducción en $n$. Cuando $n = 1$ no tenemos nada 
que probar. Ahora, dado un polinomio $p \in P_n$ notamos por $p' \in A_{n-1}$
el polinomio $p(x_1, \ldots, x_{n-1}, 0)$, así que en particular $s_{n,i}' = 
s_{n-1,i}$ si $1 \leq i < n$.

Si $p \in A_n^{(d)}$ entonces $p' \in A_{n-1}$ y por hipótesis inductiva 
existe un único polinomio $q \in \CC[z_1, \ldots, z_{n-1}]$ tal que 
$q(s'_{n,1}, \ldots, s'_{n,n-1}) = p'$. Sea $f(x_1, \ldots, x_n) = p(x_1, 
\ldots, x_n) - q(s_{n,1}, \ldots, s_{n,n-1})$. Si evaluamos $x_n = 0$ en esta 
expresión obtenemos
\begin{align*}
p' - q(s'_{n,1}, \ldots, s'_{n,n-1},0) = 0,
\end{align*}
es decir que $x_n$ dvide a $f$. Como $f$ es simétrico tenemos que cada $x_i$
divide a $f$ y por lo tanto $s_n$ divide a $f$. Así $p = q(s_{n,1}, \ldots,
s_{n,n-1}) + h s_{n,n}$ con $h$ un polinomio simétrico de grado $d-n$ 
unívocamente determinado. Haciendo inducción en el grado de $h$ vemos que $h$ 
también es un polinomio en las funciones simétricas elementales de forma única.

Supongamos que $q_1, q_2 \in \CC[z_1, \ldots, z_n]$ son tales que
$p = q_1(s_{n,1}, \ldots, s_{n,n}) = _1(s_{n,1}, \ldots, s_{n,n})$. Entonces 
$q_1' = q_2' = q$ por unicidad de la escritura de $p'$. Esto implica que  
\end{proof}

\section{Un poco de teoría de Galois}
En toda esta sección $L \subset K \subset \CC$ son cuerpos. En particular 
$K$ es un $L$-espacio-vectorial.

\paragraph
Decimos que $K$ es una extensión finita de $L$ si $\dim_L K$ es finita. Vamos 
a usar sin demostración un resultado clásico de Teoría de Galois, conocido 
como el Teorema del Elemento Primitivo.

\begin{Teorema*}
Existe un elemento en $K$, digamos $\gamma$, tal que $K = L[\gamma]$.
\end{Teorema*}

El elemento $\gamma$ no es dificil de encontrar. Notar que basta con encontrar 
un elemento $\gamma$ tal que el conjunto $\{1, \gamma, \gamma^2, \ldots, 
\gamma^{n-1}\}$ sea linealmente independiente. Los elementos que cumplen esa
condición son un conjunto denso en $K$.

\paragraph
Sea $G$ un subgrupo finito del grupo de automorfismos de $K$. Notamos por $K^G$
el conjunto de todos los elementos $a \in K$ tales que $\sigma(a) = a$ para
cualquier $\sigma \in G$. Es facil probar que $K^G$ también es un cuerpo.

\begin{Proposicion*}
Sea $L = K^G$. Entonces $K$ es un $L$-espacio vectorial de dimensión $|G|$.
\end{Proposicion*}  
\begin{proof}
Sea $\alpha \in K$, y sea $p_\alpha(x) = \prod_{g \in G} (x-g(\alpha))$. 
Entonces $p_\alpha(x)$ es un polinomio de grado $|G|$ con coeficientes en $L$ 
(¡probarlo!) tal que $p_\alpha(\alpha) = 0$, por lo tanto el conjunto $\{
\alpha^i \mid 0 \leq i \leq |G|\}$ es linealmente dependiente. Ahora si $K' 
\subset K$ contiene a $L$ y es una extensión finita, entonces $K' = L(\alpha)$
y por lo tanto es finita y $\dim_L K' \leq |G|$. 

Supongamos que $\dim_L K = \infty$. Entonces podemos tomar $\alpha_1 \in K 
\setminus L $ y $K_1 = L(\alpha_1)$. Por el párrafo anterior $\dim_L K_1 \leq 
|G|$ y por lo tanto las inclusiones $L \subset K_1 \subset K$ son estrictas, y 
por lo tanto $\dim_L K_1 \geq 2$. Tomando $\alpha_2 \in K \setminus K_1$ y 
poniendo $K_2 = K_1(\alpha_2)$ obtenemos una cadena más larga $L \subset K_1 
\subset K_2 \subset K$ con $\dim_L K_2 \geq 3$, y continuando asi construimos 
$K_{|G|}$ de dimensión $|G| + 1$. Por el teorema del elemento primitivo 
$K_{|G|} = L[\alpha]$ y por lo visto antes $\dim_L L[\alpha] \leq |G|$, lo que
da una contradicción, así que $\dim_L K < \infty$ y por lo tanto $\dim_L K \leq
|G|$.

Por el teorema del elemento primitivo existe $\gamma \in K$ tal que $K = 
L[\gamma]$. Si algún $g \in G$ cumple que $g(\gamma) = \gamma$ entonces
$g = \id_K$, por lo tanto el conjunto $\{g(\gamma), g \in G\}$ tiene cardinal
$|G|$. Si $p \in L[x]$ es un polinomio tal que $p(\gamma) = 0$ entonces
$p(g(\gamma)) = g(p(\gamma)) = 0$, y por lo tanto $p_\gamma \mid p$. En otras 
palabras, el conjunto $\{\gamma^i \mid 0 \leq i \leq |G| - 1\}$ es linealmente
independiente sobre $L$, así que $\dim_L K = |G|$.
\end{proof}


\section{El teorema de Chevalley-Shepherd-Todd}

\section{Más ejemplos}
\end{document}