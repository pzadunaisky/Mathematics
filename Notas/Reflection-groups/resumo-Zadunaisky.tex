\documentclass{amsart}
\usepackage{amsfonts}
\usepackage{graphicx}
\usepackage{amscd}
\usepackage{amsmath}
\usepackage{amssymb}
\usepackage{amssymb}
\usepackage{amsfonts}
\usepackage{graphicx,color}
\usepackage{amsthm}
\usepackage{hyperref}
\setcounter{MaxMatrixCols}{30}
\theoremstyle{plain}
\usepackage[brazil]{babel}
\usepackage[utf8]{inputenc}
\usepackage[T1]{fontenc}
%\begin{document}

%%% Titulo da pagina: Favor n\~ao mudar

\newcommand{\header}{
 \hrule
 \vskip 6truept
 \centerline{\large Jornada de \'Algebra no Amazonas - terceira edi\c{c}\~ao}
  \centerline{Universidade Federal do Amazonas}
  \centerline{4 a 8 de setembro de 2017 - Parintins / AM}
 \vskip 6truept
 \hrule
 \vspace{1cm}
 }
%%%  Titulo do resumo

\title{ \header  O Teorema de Chevalley-Shephard-Todd}


%%%  Nome do colaborador

\author{Pablo Zadunaisky (USP) {\rm pzadunaisky@gmail.com}  \\
}
 
\thanks{financiamento (Supported by) FAPESP Bolsa de Postdoc 2016/25984-1.}


\thispagestyle{empty}
\begin{document}
\maketitle
%Escrever o tipo de atividade (palestra, minicurso, oficina):
\textbf{Tipo de Atividade:} Minicurso\\\\
%CARGA HORÁRIA PARA OS OUVINTES (Para Ouvintes: Palestras será de 50 a 60 minutos, Minicursos será de 4 a 6 horas:
%das oficinas fica a critério dos proponentes
\textbf{Carga hor\'aria:} 3 horas.\\\\
%PÚBLICO-ALVO (alunos de graduação em Matemática, ou alunos de pós-graduação em Matemática, ou alunos de áreas afins):
\textbf{P\'ublico-alvo:} Alumnos de maestrado en adelante.\\\\
\textbf{Resumo:} (MAXIMO 02 PAGINAS) 

Si $G$ es un subgrupo de $\mathsf{GL}(n, \mathbb C)$, entonces actúa sobre el
álgebra de polinomios en $n$ variables de forma natural. Un polinomio en $n$
variables se dice $G$-invariante si es estable por la acción de todo elemento
de $G$, y el conjunto de todos los polinomios $G$-invariantes forman una 
subálgebra de $\mathbb C[x_1, \ldots, x_n]$. 
El teorema de Chevalley-Shephard-Todd dice que si $G$ es un grupo finito
generado por reflexiones, entonces el álgebra de polinomios invariantes de 
$G$ está generada por $n$ polinomios $s_1, \ldots, s_n$ algebraicamente 
independientes. Los polinomios no son únicos, pero sus grados están 
unívocamente determinados. 

En este curso daremos varios ejemplos de este teorema y una demostración 
basada en la demostración de Chevalley. Esta demostración usa solamente 
herramientas elementales: si usted entendió el párrafo anterior está en
condiciones de entender todo el minicurso.

ADVERTENCIA: El curso será en portunhol.
\\\\
\begin{thebibliography}{99}
%\textbf{Refer\^encias:}\\\\
\bibitem{Chev} CHEVALLEY, Claude. Invariants of finite groups generated by
reflections. American Journal of Mathematics, vol 77, No. (Oct. 1955).
\end{thebibliography}
%%%%%%%%%%%%%%%%%%%%%%%%%%%%%%%%%%%%%%%%%%%
\end{document} 