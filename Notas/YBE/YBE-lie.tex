%%%%%%%%%%%%%%%%%%%%%% Generalities %%%%%%%%%%%%%%%%%%5
\documentclass[11pt,fleqn]{article}
\usepackage[paper=a4paper]
  {geometry}

\pagestyle{plain}
\pagenumbering{arabic}
%%%%%%%%%%%%%%%%%%%%%%%%%%%%%%%%
\usepackage{notas}

%%%%%%%%%%%%%%%%%%%%%%%%%%% The usual stuff%%%%%%%%%%%%%%%%%%%%%%%%%
\newcommand\NN{\mathbb N}
\newcommand\CC{\mathbb C}
\newcommand\QQ{\mathbb Q}
\newcommand\RR{\mathbb R}
\newcommand\ZZ{\mathbb Z}
\renewcommand\k{\Bbbk}

\newcommand\A{\mathcal A}
\newcommand\B{\mathcal B}
\newcommand\Z{\mathsf Z}

\newcommand\maps{\longmapsto}
\newcommand\ot{\otimes}
\renewcommand\to{\longrightarrow}
\renewcommand\phi{\varphi}
\newcommand\id{\mathsf{Id}}
\newcommand\im{\mathsf{im}}
\newcommand\coker{\mathsf{coker}}
%%%%%%%%%%%%%%%%%%%%%%%%% Specific notation %%%%%%%%%%%%%%%%%%%%%%%%%
\newcommand\g{\mathfrak g}
\renewcommand\sl{\mathfrak{sl}}

\DeclareMathOperator\rep{\mathsf{rep}}
\DeclareMathOperator\Ret{\mathsf{Ret}}

\DeclareMathOperator\Hom{\mathsf{Hom}}
\DeclareMathOperator\End{\mathsf{End}}

\newcommand\range[1]{\left\llbracket #1 \right\rrbracket}

%%%%%%%%%%%%%%%%%%%%%%%%%%%%%%%%%%%%%% TITLES %%%%%%%%%%%%%%%%%%%%%%%%%%%%%%
\title{Soluciones de YBE linealizadas}
\date{28/02/2018}
\author{[YBE-lie.tex]}
\begin{document}
\maketitle

\section{Generalidades}
\paragraph
Sea $V$ un espacio vectorial complejo de dimensión finita dotado de un 
producto interno. Entonces existe un isomorfismo $V \ot V \cong \End(V)$ dado 
por $x \ot y \mapsto  \langle y,-\rangle x$. Este a su vez induce un 
isomorfismo $\End(V) \ot \End(V) \cong \End(\End(V))$, dado por $f \ot g 
\mapsto (h \mapsto f\circ h^* \circ g^*)$. Tomar $h^*$ en lugar de $h$ 
introduce ciertas complicaciones técnicas, pero es coherente con la notación 
habitual en el estudio de soluciones conjuntistas de YBE.

Sea $\B = \{e_1, \ldots, e_n\}$ una base ortonormal de $V$. Entonces $V \ot 
V \cong M_n(\CC)$ a través del isomorfismo $e_i \ot e_j \mapsto E^i_j$;
notamos por $\B^2$ a la base formada por los $E^i_j$. Tenemos entonces que 
$\End(V \ot V) \cong M_n(\CC) \ot M_n(\CC)$. Todo endomorfismo 
$f: V \ot V \to V \ot V$ está definido por escalares $\alpha_{i,j}^{k,l}$ tales
que $f(e_i \ot e_j) = \sum_{k,l} \alpha_{i,j}^{k,l} e_k \ot e_l$, y el 
isomorfismo está dado por $f \mapsto T_{\B}(f) = \sum_{i,j,k,l} 
\alpha_{i,j}^{k,l} E^i_k \ot E^j_l$. 

Sea $\B' = \{f_1, \ldots, f_n\}$ otra base ortonormal de $V$, y sea $P$
la matriz del cambio de base de $\B'$ a $\B$. Poniendo $X^P = P X P^*$
tenemos que $T_{\B'}(f) = \sum_{i=1}^s A_i^P \ot B_i^P$.

\paragraph
Sea $(X,r)$ una solución conjuntista finita de YBE. Como de costumbre notamos
$r(x,y) = (f_x(y), g_y(x))$. Sin pérdida de generalidad podemos suponer que 
$X = \range{n}$ con $n = \# X$. Identificamos $\CC^n$ con el espacio vectorial 
complejo generado por $X$, poniendo $e_i$ como el vector correspondiente a $i 
\in X$, y $\B$ la base canónica. Notamos ademas $R \in \End(\CC^n \ot \CC^n)$ 
a la transformación lineal inducida por $r$. Se sigue de la definición que 
$T_\B(R) = \sum_{i,j \in \range{n}} E_j^{f_i(j)} \ot E_i^{g_j(i)}$.

\paragraph
Supongamos ahora que $r$ es involutiva. Imponemos la relación de equivalencia 
$x \sim y$ si y solo si $f_x = f_y$. Entonces $x \sim y$
implica que $g_x = g_y$ y que $\Ret(X,r) = (X/\! \sim, \overline r)$ es una 
solución conjuntista de YBE. Llamamos a la clase de equivalencia de $x$
el \newterm{bloque} de $x$; un $t$-bloque es una clase de equivalencia de 
cardinal $t \in \NN$. Ordenando los bloques por cardinalidad decreciente, 
notamos por $\lambda = \lambda(X,r)$ a la partición $(\lambda_1, \ldots, 
\lambda_s) \vdash n$ donde $\lambda_i$ es el cardinal del $i$-ésimo bloque 
Sin pérdida de generalidad podemos suponer que $X = \range{n}$ con $n = \# X$
y que el $k$-ésimo bloque de $X$ es $I_k = \range{\sum_{j=1}^{k-1} \lambda_j + 
1, \sum_{j=1}^k \lambda_j}$. Dado $x \in X$ notamos por $\lambda(x)$ el 
tamamaño de su bloque.

\begin{Lemma*}
Para todos $x,y \in X$ se tiene que $\lambda(y) = \lambda(f_x(y))$. En otras
palabras $f_x$ permuta los $t$-bloques de $r$.
\end{Lemma*}
\begin{proof}
La función $f_x$ es inyectiva. La buena definición de $\sim$ implica que $f_x$
envía cualquier elemento del bloque de $y$ en un elemento del bloque de 
$f_x(y)$ y esto implica que $\lambda(y) \leq \lambda(f_x(y))$. Iterando este 
argumento tenemos que $\lambda(y) \leq \lambda(f_x(y)) \leq \lambda(f_x^2(y)) 
\leq \cdots \leq \lambda(f_x^s(y))$ para todo $s \in \NN$. Pero al ser $X$ un 
conjunto finito y $f_x$ una permutación, existe un $s$ tal que $f_x^s = f_x$,
y por lo tanto la cadena de desigualdades debe ser una cadena de igualdades.
\end{proof}

\paragraph
Notamos $\sigma_k = f_x, \tau_k = g_y$ para cualquier $x, y \in I_k$. Se sigue
que
\begin{align*}
T_\B(R)
	&= \sum_{k=1}^s \sum_{l=1}^s A_k^l \ot B_l^k 
\end{align*}
donde 
\begin{align*}
A_k^l 
	&= \sum_{j \in I_k} E_j^{\sigma_l(j)}
	& B_l^k
	&= \sum_{i \in I_l} E_i^{\tau_k(i)}
\end{align*}
Como consecuencia del lema anterior, la submatriz de coordenadas $I_k \times
\sigma_l(I_k)$ de $A_k^l$ es una matriz de permutación de $\lambda_k \times 
\lambda_k$; los coeficientes por fuera de esta submatriz son todos nulos.


\end{document}
