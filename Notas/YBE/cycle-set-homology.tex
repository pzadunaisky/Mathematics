

%%%%%%%%%%%%%%%%%%%%%% Generalities %%%%%%%%%%%%%%%%%%5
\documentclass[11pt,fleqn]{article}
\usepackage[paper=a4paper]
  {geometry}

\pagestyle{plain}
\pagenumbering{arabic}
%%%%%%%%%%%%%%%%%%%%%%%%%%%%%%%%
% esto tiene que estar, en este orden...
\usepackage[small]{titlesec}
\usepackage{paragraphs}
\usepackage{hyperref}
\usepackage{amsthm,thmtools}

%\linespread{1.2}
\setlength{\parskip}{1.2ex}

\usepackage[utf8,latin1]{inputenc}
\usepackage[spanish,english]{babel}
\usepackage{enumerate}
\usepackage{mathpazo}
%\usepackage{euler}
\usepackage[alphabetic,initials]{amsrefs}
\usepackage{amsfonts,amssymb,amsmath}
\usepackage{dsfont}
\usepackage{mathtools}
\usepackage{graphicx}
\usepackage[poly,arrow,curve,matrix]{xy}
\usepackage{stmaryrd}
\usepackage{showlabels}
\usepackage{tikz}
%\swapnumbers

% numbered versions
\declaretheoremstyle[headformat=swapnumber, spaceabove=\paraskip,
bodyfont=\itshape]{mystyle}
\declaretheorem[name=Lemma, sibling=para, style=mystyle]{Lemma}
\declaretheorem[name=Proposition, sibling=para, style=mystyle]{Proposition}
\declaretheorem[name=Theorem, sibling=para, style=mystyle]{Theorem}
\declaretheorem[name=Corolllary, sibling=para, style=mystyle]{Corollary}
\declaretheorem[name=Definition, sibling=para, style=mystyle]{Definition}
%\declaretheorem[name=Example, sibling=para, style=mystyle]{Example}


% unnumbered versions
\declaretheoremstyle[numbered=no, spaceabove=\paraskip,
bodyfont=\itshape]{mystyle-empty}
\declaretheorem[name=Lemma, style=mystyle-empty]{Lemma*}
\declaretheorem[name=Proposition, style=mystyle-empty]{Proposition*}
\declaretheorem[name=Theorem, style=mystyle-empty]{Theorem*}
\declaretheorem[name=Corollary, style=mystyle-empty]{Corollary*}
\declaretheorem[name=Definition, style=mystyle-empty]{Definition*}
%\declaretheorem[name=Example, style=mystyle-empty]{Example*}

% plain style
\declaretheoremstyle[
        headformat={{\bfseries\NUMBER.}{\itshape\NAME}\NOTE\ignorespaces},
        spaceabove=\paraskip, 
        headpunct={.},
        headfont=\itshape,
        bodyfont=\normalfont
        ]{mystyle-plain}
\declaretheorem[sibling=para, style=mystyle-plain]{Example}
\declaretheorem[sibling=para, style=mystyle-plain]{Remark}

% proofs, just as in amsthm but with adapted spacing
\makeatletter
\renewenvironment{proof}[1][\proofname]{\par
  \pushQED{\qed}%
  \normalfont \topsep.75\paraskip\relax
  \trivlist
  \item[\hskip\labelsep
        \itshape
    #1\@addpunct{.}]\ignorespaces
}{%
  \popQED\endtrivlist\@endpefalse
}
\makeatother

%%%%%%%%%%%%%%%%%%%%%%%%%%% The usual stuff%%%%%%%%%%%%%%%%%%%%%%%%%
\newcommand\NN{\mathbb N}
\newcommand\CC{\mathbb C}
\newcommand\QQ{\mathbb Q}
\newcommand\RR{\mathbb R}
\newcommand\ZZ{\mathbb Z}
\newcommand\KK{\mathbb K}
\newcommand\maps{\longmapsto}
\newcommand\ot{\otimes}
\renewcommand\to{\longrightarrow}
\renewcommand\phi{\varphi}
\newcommand\blabla[1]{\noindent\framebox[1.05\width]{\texttt{Note: #1}}}

%%%%%%%%%%%%%%%%%%%%%%%%% Specific notation %%%%%%%%%%%%%%%%%%%%%%%%%
\renewcommand\k{\Bbbk}

\DeclareMathOperator\Mod{\mathsf{Mod}}
\DeclareMathOperator\Hom{\mathsf{Hom}}
\DeclareMathOperator\Ext{\mathsf{Ext}}
\DeclareMathOperator\Tor{\mathsf{Tor}}

\DeclareMathOperator\Id{\mathsf{Id}}
\DeclareMathOperator\Fun{\mathsf{Fun}}
\DeclareMathOperator\TOR{\underline{\mathsf{Tor}}}

\newcommand\join{\vee}
\newcommand\meet{\wedge}
\renewcommand\q{\mathbf{q}}

%%%%%%%%%%%%%%%%%%%%%%%%%%%%%%%%%%%%%% TITLES %%%%%%%%%%%%%%%%%%%%%%%%%%%%%%
\title{Cycle set cohomology}
\date{[cycle-set-cohomology.tex]}
%\author{Pablo Zadunaisky}
\begin{document}
\maketitle

\section{Cycle sets}
\paragraph
\label{cycle-set-def}
A cycle set is a set $X$ with an binary operation $\cdot$ such that:
\begin{enumerate}[(a)]
\item for each $a \in X$ the operation $x \mapsto a \cdot x$ is bijective, and
\item for all $a,b,c \in X$ the following relation holds:
\[
  (a \cdot b) \cdot (a \cdot c) = (b \cdot a) \cdot (b \cdot c).
\]
\end{enumerate}
We denote by $a \ast -$ the inverse operation to $a \cdot -$, so $a \ast b = c$
if and only if $b = a \cdot c$.

\paragraph
\label{cycle-set-ybe}
Let $(X, \cdot)$ be a cycle set. Then 
\begin{align*}
\sigma: X \times X &\to X \times X \\
(a,b) &\maps ((b \ast a) \cdot b, b \ast a)
\end{align*}
is a solution to the YBE.

\begin{Lemma}
\label{cycle-set-ybe-props}
Let $(X, \cdot)$ be a cycle set, and let $\sigma$ be the associated solution to YBE.
Then $\sigma^2 = \Id_{X \times X}$, and for each $b,d \in X$ there exist unique $a,c \in 
X$ such that $\sigma(a,b) = (c,d)$. 
\end{Lemma}
\begin{proof}
First
\begin{align*}
\sigma^2(a,b) 
  &= \sigma((b \ast a) \cdot b, b \ast a) \\
  &= (((b \ast a) \ast ((b \ast a) \cdot b)) \cdot (b \ast a), 
    (b \ast a) \ast ((b \ast a) \cdot b)) \\
  &= (b \cdot (b \ast a), b) = (a,b).
\end{align*}

To prove the second assertion we need to show that $b,d$ determine $a,c$ uniquely in the 
equation $((b \ast a)\cdot b, b \ast a) = (c,d)$. Clearly $a = b \cdot d$ and $c = d 
\cdot b$, so the result follows.
\end{proof}

\paragraph
\label{ybe-semigroup}
The associated semigroup to a solution of the YBE $\sigma: X^{\times 2} \to X^{\times 2}$
is the quotient of the free semigroup on $X$ by the relations $ab = cd$ whenever 
$\sigma(a,b) = (c,d)$. It is denoted by $S_\sigma$. Its enveloping group is denoted by 
$G_\sigma$.
\begin{Proposition*}
Let $(X, \cdot)$ be a cycle-set and let $S_\sigma$ be the semigroup associated to the 
corresponding solution of the YBE. Then the following hold.
\begin{itemize}
\item $S_\sigma$ is cancellative.
\item The natural map $S_\sigma \hookrightarrow G_\sigma$ is injective. 
\item $G_\sigma$ is a group of fractions of $S_\sigma$.
\end{itemize}
\end{Proposition*}
\begin{proof}
In view of Lemma \ref{cycle-set-ybe-props} and \cite{GIVdB}*{Theorem 1.4}, $S_\sigma$ is 
a $I$-type semigroup in the sense defined below \cite{GIVdB}*{Theorem 1.2}. Then
\cite{GIVdB}*{Corollary 1.6} says that $S_\sigma$ is cancellative. By 
\cite{CP}*{Theorem 1.24} the next two items hold if and only if for each $s, s' \in 
S_\sigma$ the right ideals $S_\sigma s$ and $S_\sigma s'$ have nonzero intersection.

The semigroup $S_\sigma$ is naturally graded by length. The fact that it is a $I$-type
semigroup implies that $(S_\sigma)_k = \binom{n+k-1}{n-1} = \frac{k^{n-1}}{(n-1)!} + 
o(k^{n-2})$. Let $s, s' \in S_\sigma$, both of length $l$. Since $S$ is cancellative for 
$k \geq l$ we get
\begin{align*}
(S_\sigma s)_k 
  = (S_\sigma s')_k 
  = \binom{n+k-l-1}{n-1} 
  = \frac{k^{n-1}}{(n-1)!} + o(k^{n-2})
\end{align*}
Now $(S_\sigma s \cap S_\sigma s')_k \geq |S_\sigma s|_k + |S_\sigma s'|_k - |S|_k = 
\frac{k^{n-1}}{(n-1)!} + o (k^{n-2})$ which is a positive number for $k \gg 0$, so 
$S_\sigma s \cap S_\sigma s' \neq \emptyset$. If $s$ and $s'$ have diferent length,
say $s$ is longer than $s'$, we find $s''$ such that $s''s'$ has the same length as $s$
and since $S_\sigma s'' s' \cap S_\sigma s \subset S_\sigma s' \cap S_\sigma s$ the 
result follows.
\end{proof}

\paragraph
\label{h2-and-hh1}
We recall some facts from \cite{LV}*{section 11}. Let $(X,\cdot)$ be a cycle set and 
let $A$ be any abelian group. Denote by $\Fun(X,A)$ the set of all functions $X \to A$.
Then $\Fun(X,A)$ is a $G_\sigma$-set with action defined by
\begin{align*}
f \cdot a (b) = f (a \ast b) 
\end{align*}
for all $a,b \in X$ and $f \in \Fun(X,A)$. Of course this also induces an $S_\sigma$ 
action on $\Fun(X,A)$ by restriction. The main theorem of \cite{LV}*{section 11} states
that
\[
  H^2(X, A) \cong H^1(G_\sigma, \Fun(X,A)).
\]
By Proposition \ref{ybe-semigroup} the map $S_\sigma \to G_\sigma$ is injective and 
$G_\sigma$ is a group of fractions of $S_\sigma$. Thus we may apply 
\cite{CE}*{Chapter XS, Proposition 4.1} and conclude that the group $H^1(S_X, \Fun(X,A))$
is isomorphic to both cohomology groups above.

Now let $R = \ZZ[S_X]$ be the semigroup ring of $S_X$. Then $\Fun(X,A)$ is an $R$-module
with right action given as above, which can be turned into an $R$-bimodule by setting a 
trivial left action. In that case, as shown in \cite{CE}*{Chapter X, section 2}, there is
a further isomorphism $H^1(S_\sigma, \Fun(X,A)) \cong HH^1(\ZZ[S_\sigma], \Fun(X,A))$. 
Thus we may apply techniques from Hochschild cohomology theory to calculate $H^2(X, A)$.
This is particularly useful since $R$ is a homogeneous algebra, and hence the first 
three terms of the minimal resolution of $R$ as $R$-bimodule are known, and these are 
precisely the terms needed to calculate the first Hochschild cohomology group.

\begin{bibdiv}
\begin{biblist}
\bib{CE}{book}{
   author={Cartan, Henri},
   author={Eilenberg, Samuel},
   title={Homological algebra},
   series={Princeton Landmarks in Mathematics},
   note={With an appendix by David A. Buchsbaum;
   Reprint of the 1956 original},
   publisher={Princeton University Press},
   place={Princeton, NJ},
   date={1999},
   pages={xvi+390},
}

\bib{CP}{book}{
   author={Clifford, A. H.},
   author={Preston, G. B.},
   title={The algebraic theory of semigroups. Vol. I},
   series={Mathematical Surveys, No. 7},
   publisher={American Mathematical Society},
   place={Providence, R.I.},
   date={1961},
   pages={xv+224},
}

\bib{GIVdB}{article}{
   author={Gateva-Ivanova, Tatiana},
   author={Van den Bergh, Michel},
   title={Semigroups of $I$-type},
   journal={J. Algebra},
   volume={206},
   date={1998},
   number={1},
   pages={97--112},
}

\bib{LV}{article}{
  author={Lebed, V.},
  author={Vendramin, L.},
  title={Homology of left non-degenerate set-theoretic solutions to the Yang-Baxter Equation},
  url={arXiv:1509.07067},
}

\end{biblist}
\end{bibdiv}
\end{document}
































