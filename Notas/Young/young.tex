%%%%%%%%%%%%%%%%%%%%%% Generalities %%%%%%%%%%%%%%%%%%5
\documentclass[11pt,fleqn]{article}
\usepackage[paper=a4paper]
  {geometry}

\pagestyle{plain}
\pagenumbering{arabic}
%%%%%%%%%%%%%%%%%%%%%%%%%%%%%%%%
\usepackage{notas}

\usepackage[centertableaux]{ytableau}
%%%%%%%%%%%%%%%%%%%%%%%%%%% The usual stuff%%%%%%%%%%%%%%%%%%%%%%%%%
\newcommand\NN{\mathbb N}
\newcommand\CC{\mathbb C}
\newcommand\QQ{\mathbb Q}
\newcommand\RR{\mathbb R}
\newcommand\ZZ{\mathbb Z}

\newcommand\maps{\longmapsto}
\newcommand\ot{\otimes}
\renewcommand\to{\longrightarrow}
\renewcommand\phi{\varphi}
\newcommand\eps{\varepsilon}


\DeclareMathOperator\Sym{\mathsf{Sym}}
\DeclareMathOperator\Alt{\mathsf{Alt}}
\DeclareMathOperator\sg{\mathsf{sg}}
\DeclarePairedDelimiter\gen{\langle}{\rangle}

%%%%%%%%%%%%%%%%%%%%%%%%% Specific notation %%%%%%%%%%%%%%%%%%%%%%%%%


%%%%%%%%%%%%%%%%%%%%%%%%%%%%%%%%%%%%%% TITLES %%%%%%%%%%%%%%%%%%%%%%%%%%%%%%
\title{Tableros de Young}
\date{[young.tex]}
\author{Pablo Zadunaisky}

\begin{document}
\maketitle

Estas notas siguen los capítulos 4 y 6 del libro de Fulton y Harris sobre teoría de
representaciones.

\section{Definiciones generales}
Notamos por $\SS_d$ al $d$-ésimo grupo de permutaciones. Dado $\sigma \in \SS_d$ notamos
por $e_\sigma$ al elemento correspondiente en el álgebra $\CC[\SS_d]$.
\begin{align*}
	a &= \sum_{\sigma \in S_d} e_\sigma & b &= \sum_{\sigma \in S_d} \sg(\sigma) e_\sigma
\end{align*}

\begin{Definition}
	Sea $V$ un espacio vectorial. Dado $d \in \NN$ definimos los siguientes subespacios de
	la potencia tensorial $V^{\ot d}$
	\begin{align*}
		\Sym^n(V) &= \gen{\sum_{\sigma \in S_d} v_{\sigma(1)} \ot \ldots \ot v_{\sigma(d)}
	\mid v_1 \ot \ldots \ot v_d \in V^{\ot d}}\\
	\Alt^n(V) &= \gen{\sum_{\sigma \in S_d} \sg(\sigma) v_{\sigma(1)} \ot \ldots \ot
	v_{\sigma(d)} \mid v_1 \ot \ldots \ot v_d \in V^{\ot d}}
	\end{align*}
\end{Definition}

\paragraph
Una observación interesante es que si hacemos a $S_d$ actuar sobre $V^{\ot d}$ permutando
los tensores, y ponemos $a = \sum_{\sigma \in S_d} e_\sigma$ y $b = \sum_{\sigma \in S_d}
\sg(\sigma) e_\sigma$ entonces $\Sym^d V = a\cdot V^{\ot d}$ y $\Alt^d V = b \cdot V^{\ot
d}$.

\paragraph
\label{diagramas-de-young}
\about{Diagramas de Young}
Fijemos $n \in \NN$. Una \newterm{partición} de $n$ es una $d$-upla de enteros $\lambda =
(\lambda_1, \ldots, \lambda_d)$ con $\lambda_1 \geq \lambda_2 \geq \ldots \geq \lambda_d$
tal que $n = \lambda_ 1 + \cdots + \lambda_d$. En este caso escribimos $|\lambda| = n$ o
$\lambda \vdash n$.

A cada partición le asociamos su \newterm{diagrama de Young}, que es un diagrama con $d$
filas de casilleros cuya $i$-ésima fila tiene $\lambda_i$ casilleros. Abajo mostramos
tres ejemplos de particiones con sus respectivos diagramas de Young.
\begin{align*}
	(2,1) & \ \ydiagram{2,1} 
		& (3,3,1) & \ \ydiagram{3, 3, 1} 
		& (5,3,2)\  & \ydiagram{5, 3, 2}
\end{align*}
Muchas veces identificamos una partición $\lambda$ con su diagrama de Young.

\begin{Proposition}
\label{hook-length-formula}
Sea $\lambda = (\lambda_1, \ldots, \lambda_k)$ una partición de $n$. Sea $\ell(i,j)$ la
longitud del gancho correspondiente al lugar $(i,j)$ en $\lambda$, y sea $l_i = \ell(i,1)
= \lambda_i + k - i$ para todo $1 \leq i \leq k$. Se tiene entonces que
\begin{align*}
	\prod_{(i,j) \in \lambda} \ell(i,j) &= \frac{l_1! l_2! \ldots l_k!}{\prod_{i<j} (l_i -
		l_j)}.
\end{align*}
\end{Proposition}
\end{document}
