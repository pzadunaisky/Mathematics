%%%%%%%%%%%%%%%%%%%%%% Generalities %%%%%%%%%%%%%%%%%%5
\documentclass[11pt,fleqn]{article}
\usepackage[paper=a4paper]
  {geometry}

\pagestyle{plain}
\pagenumbering{arabic}
%%%%%%%%%%%%%%%%%%%%%%%%%%%%%%%%
% esto tiene que estar, en este orden...
\usepackage[small]{titlesec}
\usepackage{paragraphs}
\usepackage{hyperref}
\usepackage{amsthm,thmtools}

%\linespread{1.2}
%\setlength{\parskip}{1.2ex}

\usepackage[utf8]{inputenc}
\usepackage[spanish,english]{babel}
\usepackage{enumerate}
\usepackage[alphabetic,initials]{amsrefs}
\usepackage{amsfonts,amssymb,amsmath}
%\usepackage{stmaryrd}
\usepackage{mathtools}
\usepackage{graphicx}
\usepackage[poly,arrow,curve,matrix]{xy}
\usepackage{wrapfig}

%\swapnumbers




% numbered versions
\declaretheoremstyle[headformat=swapnumber, spaceabove=\paraskip,
bodyfont=\itshape]{mystyle}
\declaretheorem[name=Lema, sibling=para, style=mystyle]{Lemma}
\declaretheorem[name=Proposici\'on, sibling=para, style=mystyle]{Proposition}
\declaretheorem[name=Teorema, sibling=para, style=mystyle]{Theorem}
\declaretheorem[name=Corolario, sibling=para, style=mystyle]{Corollary}
\declaretheorem[name=Definici\'on, sibling=para, style=mystyle]{Definition}

% unnumbered versions
\declaretheoremstyle[numbered=no, spaceabove=\paraskip,
bodyfont=\itshape]{mystyle-empty}
\declaretheorem[name=Lema, style=mystyle-empty]{Lemma*}
\declaretheorem[name=Proposici\'on, style=mystyle-empty]{Proposition*}
\declaretheorem[name=Teorema, style=mystyle-empty]{Theorem*}
\declaretheorem[name=Corolario, style=mystyle-empty]{Corollary*}
\declaretheorem[name=Definici\'on, style=mystyle-empty]{Definition*}

% plain style
\declaretheoremstyle[
        headformat={{\bfseries\NUMBER.}{\itshape\NAME}\NOTE\ignorespaces},
        spaceabove=\paraskip, 
        headpunct={.},
        headfont=\itshape,
        bodyfont=\normalfont
        ]{mystyle-plain}
\declaretheorem[sibling=para, style=mystyle-plain]{Example}
\declaretheorem[sibling=para, style=mystyle-plain]{Remark}

% proofs, just as in amsthm but with adapted spacing
\makeatletter
\renewenvironment{proof}[1][\proofname]{\par
  \pushQED{\qed}%
  \normalfont \topsep.75\paraskip\relax
  \trivlist
  \item[\hskip\labelsep
        \itshape
    #1\@addpunct{.}]\ignorespaces
}{%
  \popQED\endtrivlist\@endpefalse
}
\makeatother

%%%%%%%%%%%%%%%%%%%%%%%%%%% The usual stuff%%%%%%%%%%%%%%%%%%%%%%%%%
\newcommand\NN{\mathbb N}
\newcommand\CC{\mathbb C}
\newcommand\QQ{\mathbb Q}
\newcommand\RR{\mathbb R}
\newcommand\ZZ{\mathbb Z}

\newcommand\maps{\longmapsto}
\newcommand\ot{\otimes}
\renewcommand\to{\longrightarrow}
\renewcommand\phi{\varphi}
\newcommand\stack[2]{\genfrac{}{}{0pt}{2}{#1}{#2}}
\newcommand\uu[1]{\underline{#1}}

%%%%%%%%%%%%%%%%%%%%%%%%% Specific notation %%%%%%%%%%%%%%%%%%%%%%%%%

\newcommand\C{\mathcal C}
\newcommand\B{\mathcal B}


\DeclareMathOperator\Id{\mathsf{Id}}

\DeclareMathOperator\im{Im}
\DeclareMathOperator\GL{\mathsf{GL}}
\DeclareMathOperator\SO{\mathsf{SO}}
\DeclareMathOperator\sen{sen}

%%%%%%%%%%%%%%%%%%%%%%%%%%%%%%%%%%%%%% TITLES %%%%%%%%%%%%%%%%%%%%%%%%%%%%%%

\title{Grupos desde el \'algebra lineal}
\author{[grupos.tex]}
\date{17/02/2015}
\begin{document}
\maketitle
\section{Grupos de matrices}
\begin{Definition}
	Un \emph{grupo de matrices} (de tamaño $n \in \NN$) es un conjunto $G \subset M_n(\CC)$
	que cumple las siguientes condiciones:
	\begin{enumerate}
		\item La matriz identidad pertenece a $G$.
		\item Si $A, B \in G$ entonces $AB \in G$.
		\item Si $A \in G$ entonces $A^{-1} \in G$.
	\end{enumerate}
\end{Definition}
Para que una idea se útil en matemáticas, tiene que tener a la vez muchos ejemplos
interesantes (¿a quién le importa que el conjunto $\{\Id_n\}$ sea un grupo?), y a la vez
tiene que permitir decir muchas cosas sobre todos los ejemplos a la vez (Podríamos definir
un "grupex" como \emph{cualquier} subconjunto de matrices inversibles, pero entonces lo
único que vamos a saber sobre un grupex es que es un conjunto de matrices
inversibles\footnote{O no, como podría argumentar un reciente ganador del premio
Ramanujan.}).


\paragraph \textbf{Ejemplos:}
\begin{enumerate}
	\item Claramente $\GL_n(\CC)$ es un grupo, y por el punto 3 de la definición, todo
		grupo está contenido en $\GL_n(\CC)$. Más en general, todos los conjuntos de la
		siguiente cadena son grupos: $\GL_n(\ZZ) \subset \GL_n(\QQ) \subset \GL_n(\RR)
		\subset \GL_n(\CC)$. ¿Es cierto que el conjunto de las matrices con coeficientes en
		$\NN$ forma un grupo?

	\item El conjunto de matrices de la forma 
			$R_\theta = \begin{pmatrix} 
								\cos \theta & -\sen \theta \\
								\sen \theta & \cos \theta
							\end{pmatrix}$
			con $\theta \in \RR$ forma un grupo. Notar que $R_\theta R_\varphi = R_{\theta +
			\varphi}$. Tradicionalmente este conjunto se denota $\SO(2)$.

	\item Sea $A \in \GL_n(\CC)$. Entonces el conjunto $\langle A \rangle = \{A^n \mid n
		\in \ZZ\}$ forma un grupo. Por ejemplo, si $\theta = \pi/2$, entonces $R_\theta^4 =
		\Id_2$, así que
		\begin{align*}
			\langle R_\theta \rangle 
				&= \{\Id_2, R_\theta, R_\theta^2, R_\theta^3\}
				= \left\{ 
						\begin{pmatrix} 1 & 0 \\ 0 & 1\end{pmatrix},
						\begin{pmatrix} 0 & -1 \\ 1 & 0\end{pmatrix},						
						\begin{pmatrix} -1 & 0 \\ 0 & -1\end{pmatrix},						
						\begin{pmatrix} 0 & 1 \\ -1 & 0\end{pmatrix}
					\right\}
		\end{align*}

	\item El conjunto de matrices de la forma
			$T_n = \begin{pmatrix} 
					1 & 0 \\
					n & 1
				 \end{pmatrix}$
			con $n \in \ZZ$ es un grupo. Notar que $T_n T_m = T_{n+m}$.

	\item Sea $G$ un grupo. Entonces $S(G) = \{A \in G \mid \det A = 1\}$ también es un
		grupo. Como $S(G) \subset G$, decimos que $S(G)$ es un \emph{subgrupo} de $G$.
\end{enumerate}

Veamos una manera bastante barata de construir más ejemplos.
\begin{Proposition*}
	Sea $X \subset \CC^n$. Entonces el conjunto $\GL(X) = \{A \in \GL_n(\CC) \mid A.X =
	X\}$ es un grupo.
\end{Proposition*}
\begin{proof}
	Claramente $\Id_n \in E(X)$. Si $A,B \in E(X)$ entonces $(AB).X = A.(B.X) = A.X = X$.
	Finalmente, si $A.X = X$ multiplicando a ambos lados por $A^{-1}$ tenemos que $X =
	A^{-1}.X$.
\end{proof}
Por supuesto, podemos cambiar $\CC$ por $\RR$, o por $\QQ$ o por $\ZZ$ en el resultado
anterior y todo funciona igual. \textbf{Moraleja:} Hay muchos conjuntos interesantes que
son grupos.

\paragraph
Ya que vimos que hay muchos conjuntos interesantes que son grupos, vamos a ver qué
propiedades interesantes tienen. Para eso vamos a hacer una pequeña simplificación; vamos
a suponer que nuestros grupos de matrices son \emph{finitos}. Casi todos los ejemplos que
dimos son infinitos, así que damos más y más ejemplos, ahora de grupos finitos.
\\

\noindent\textbf{Ejemplos}
\begin{enumerate}
	\item Si $\theta = \frac{2\pi}{n}$, entonces el grupo $\langle R_\theta \rangle$ es
		finito.

	\item Sea $\Delta \subset \RR^2$ un triángulo equilátero con centro en el orígen de
		coordenadas. Entonces $\GL(\Delta) = \{\Id, R_{2\pi/3}, R_{4\pi/3}, S, S R_{2\pi/3},
		SR_{2\pi/3}\}$, donde $S = \begin{pmatrix} 0 & 1 \\ 1 & 0\end{pmatrix}$. 

	\item Más aún, si $\Delta \subset \RR^2$ es un $n$-ágono regular centrado en el orígen,
		$\GL(\Delta) = \{R_{2\pi/n}^i, SR_{2\pi/n}^i \mid i = 0, 1, \ldots, n-1\}$.

	\item El conjunto de matrices de $n \times n$ con exactamente un $1$ por fila y un $1$
		por columna y ceros en el resto de sus entradas es un grupo finito, llamado el
		\emph{grupo simétrico}. Si nos quedamos con las matrices de esta forma de
		determinante $1$, obtenemos el \emph{grupo alternante}.

	\item Si $Q \subset \RR^3$ es el cubo cuyos vértices son los puntos de la forma
		$(a,b,c)$ con $a,b,c \in \{\pm 1\}$, entonces $\GL(Q)$ es un grupo de 48 elementos.

	\item Si $G$ es un grupo finito y $T \in \GL_n(\CC)$ entonces $G^T = \{TAT^{-1} \mid A
		\in G\}$ también es un grupo. Decimos que dos grupos $G, G'$ son \emph{conjugados}
		si existe una matriz $T \in \GL_n(\CC)$ tal que $G' = G^T$.
\end{enumerate}

\paragraph
Vamos a probar nuestro primer resultado general 
\begin{Proposition*}
	Sea $G \subset M_n(\CC)$ un grupo finito. Entonces para todo $A \in G$ existe $m \in
	\NN$ tal que $A^m = \Id$.
\end{Proposition*}
\begin{proof}
	Consideremos el conjunto $X = \{A^i \mid i \in \NN\}$. Como $G$ es un grupo, $X \subset
	G$, en particular $X$ es finito. Eso quiere decir que existen $s \leq t$ tales que $A^s
	= A^t$, y tomando $m = t-s$, tenemos que $A^m = A^{s}A^{-t} = A^t A^{-t} = \Id_n$.
\end{proof}
De esto se deduce que existe un entero $e$, llamado el \emph{exponente del grupo} tal que
$A^e = \Id_n$ para todo $A \in G$. Basta con tomar $e$ igual al MCM de los órdenes de cada
matriz. 

\begin{Corollary*}
	Si $G$ es un grupo finito entonces cada $A \in G$ es diagonalizable, y sus autovalores
	son raíces $e$-ésimas de la unidad.
\end{Corollary*}
\begin{proof}
	Cualquier $A \in G$ satisface el polinomio $X^e - 1$. Luego el polinomio minimal $m_A$
	divide a $X^e - 1$, que por supuesto tiene como raíces a las raíces $e$-ésimas de la
	unidad, todas simples. Luego las raíces de $m_A$ son simples, lo que implica que $A$ es
	diagonalizable. Además son todas raíces de la unidad, así que los autovalores de $A$
	son raíces de la unidad.
\end{proof}

\paragraph
Acá reunimos varios hechos aislados sobre grupos. no son cosas particularmente importantes
por sí mismas, simplemente son cosas que uno usa una y otra vez cuando habla de grupos.
Pueden tomarse como ejercicios, y a partir de ahora vamos a dar estas cosas por sabidas.

\begin{enumerate}
	\item Si $G$ es un grupo y $A \in G$, entonces la función $f_A: G \to G$ dada por
		$f_A(B) = AB$ es biyectiva. Lo mismo para la función $g_A(B) = BA$, y para $h_A(B) =
		ABA^{-1}$.
\end{enumerate}

\paragraph
\textbf{Grupos finitos conmutativos.}
Recordemos que si $G$ es un grupo y $T \in \GL_n(\CC)$, entonces $G^T = \{TAT^{-1} \mid a
\in G\}$ también es un grupo. El corolario anterior dice que si $G$ es finito, entonces
para todo $A \in G$ podemos encontrar una matriz inversible $T$ tal que $TAT^{-1}$ es
diagonal. Como las matrices diagonales son más fáciles de manipular que matrices
cualquiera, sería interesante encontrar una matriz $T$ tal que $G^T$ consistiera solo de
matrices diagonales.

Una gran característica de las matrices diagonales es que conmutan entre sí, es decir, si
$D, D'$ son matrices diagonales, entonces $DD' = D'D$. Ahora supongamos que $G$ es
conjugado a un grupo de matrices diagonales, donde la conjugación está dad por una matriz
inversible $T$. Entonces, dadas $A, B \in G$ tenemos que $D = TAT^{-1}$ y $D' = TBT^{-1}$
conmutan, es decir que
\begin{align*}
	T(AB)T^{-1} 
		&= (TAT^{-1})(TBT^{-1}) = DD' = D'D 
		\\&= (TBT^{-1})(TAT^{-1}) = T(BA)T^{-1}.
\end{align*}
Conjugando a ambos lados por $T$, tenemos que $AB = BA$. Es decir que si $G$ es conjugado
a un grupo de matrices diagonales, entonces las matrices de $G$ necesariamente conmutan
entre sí; estos grupos se llaman \emph{conmutativos} o \emph{abelianos}. Resulta que
también vale la vuelta, es decir, que si un grupo $G$ consiste de matrices que conmutan
entre sí, entonces es conjugado a un grupo de matrices diagonales. Esto se deduce del
siguiente resultado general.

\begin{Proposition*}
	Sea $X$ un conjunto de matrices que conmutan todas entre sí. Si toda matriz de $X$ es
	diagonalizable, entonces existe una base de $\CC^n$ de formada por autovectores de
	todas las matrices de $X$.
\end{Proposition*}
\begin{proof}
	Ejercicio.
\end{proof}
Entre los grupos que mencionamos más arriba, el grupo de las rotaciones $\langle R_\theta
\rangle$ cumple la hipótesis de la proposición (¡demostrarlo!). Entonces debe existir una
base en la que todos sus miembros son diagonales. Esta no es dificil de encontrar, y es un
buen ejercicio encontrar esa base.

\paragraph

\begin{Proposition*}
	Sea $G$ un grupo finito. Entonces existe una matriz inversible $T$ tal que $G^T$
	consiste de matrices ortogonales de determinante $1$.
\end{Proposition*}
\begin{proof}
	Consideremos el producto interno habitual en $\CC^n$, que notamos por
	$\alngle-,-\rangle$. Definimos un \emph{nuevo} producto interno, de la siguiente
	manera: dados $v,w \in \CC^n$, tomamos
	\begin{align*}
		(v,w) &= \frac{1}{|G|} \sum_{A \in G} \langle Av, Aw\rangle.
	\end{align*}
	Es facil ver que $(-,-)$ define un nuevo producto interno. Además, vale que si $A \in
	G$, entonces $(Av,Aw) = (v,w)$, o en otras palabras la adjunta de $A$ respecto de
	este producto interno es $A^{-1}$. En otras palabras, la transformación lineal $L_A:
	\CC^n \to \CC^n$ dada por $v \mapsto Av$ es ortogonal (es inversible y su inversa
	coincide con su adjunta).
	
	Ahora sea $B$ una base ortonormal para el nuevo	producto interno, y sea $T$ la matriz
	de cambio de base de la base canónica en $B$. La matriz $A^T$ es simplemente la matriz
	de la transformación $L_A$ en la base $B$, por lo que $A^T$ debe ser una matriz
	ortogonal.
\end{proof}


\section{Grupos de transformaciones lineales}

\section{Grupos abstractos}

\end{document}

\textbf{Grupos finitos conmutativos II.}
Nuestro objetivo ahora es entender \emph{todos} los grupos de matrices conmutativos. Para
eso vamos a usar una idea muy común en matemática, y muy sencilla: vamos a dar una lista
de grupos abelianos finitos, una lista ``corta'' en algún sentido, y vamos a mostrar que
si queremos estudiar un grupo abeliano $G$ cualquiera, entonces alcanza con elegir el
grupo correcto de nuestra lista, digamos $G'$, que es una copia fiel del grupo $G$
original.

La idea es muy parecida la de la forma de Jordan. Para saber si dos matrices son
conjugadas, alcanza con llevar cada una a su forma de Jordan, y comparar ambas. Salvo
permutación de bloques (que equivale a reordenar las bases), deben ser iguales. Esta
condición es otra vez \emph{necesaria} y \emph{suficiente}. El procedimiento puede
describirse así: a cada matriz le asignamos una, y solo una, matriz en forma de Jordan
(digamos que ordenamos los bloques de mayor a menor, y bloques de igual tamaño según
módulo y argumento del autovalor correspondiente); luego dos matrices son conjugadas si y
solo si tienen la misma forma de Jordan.

Vamos a llevar adelante un procedimiento parecido. Ya vimos que todo grupo finito abeliano
es conjugado a un grupo de matrices diagonales con raíces de la unidad como autovalores.
Ahora vamos a dar una lista de grupos de matrices que cumplan con esa característica, y de
forma tal que todo grupo abeliano sea conjugado a \emph{uno y solo uno de ellos}. 

\begin{Definition*}
	Sean $n, d_1, d_2, \ldots, d_n \in \NN$. Llamamos $\C(d_1, \ldots, d_n)$ al conjunto de
	matrices diagonales que tienen en el $i$-ésimo lugar de la diagonal una raíz
	$d_i$-ésima de la unidad.
\end{Definition*}
Por ejemplo 
\begin{align*}
	\C(4,2) 
		= \bigg\{ 
			&\begin{pmatrix}
				1 & 0 \\ 0 & 1
			\end{pmatrix},
			\begin{pmatrix}
				1 & 0 \\ 0 & -1
			\end{pmatrix},
			\begin{pmatrix}
				i & 0 \\ 0 & 1
			\end{pmatrix},
			\begin{pmatrix}
				i & 0 \\ 0 & -1
			\end{pmatrix},\\
			&\begin{pmatrix}
				-1 & 0 \\ 0 & 1
			\end{pmatrix},
			\begin{pmatrix}
				-1 & 0 \\ 0 & -1
			\end{pmatrix},
			\begin{pmatrix}
				-i & 0 \\ 0 & 1
			\end{pmatrix},
			\begin{pmatrix}
				-i & 0 \\ 0 & -1
			\end{pmatrix}
		\bigg\}
\end{align*}

Nuestro resultado es el siguiente. Vamos a decir que una $n$-upla $(d_1, d_2 \ldots,
d_n) \in \NN^n$ es decreciente si $d_1 \geq d_2 \geq \cdots \geq d_n$2.
\begin{Theorem*}
	Si $G \subset M_n(\CC)$ es un grupo abeliano de matrices, entonces existe una única
	$n$-upla descendiente $(d_1, \ldots	d_n) \in \NN^n$ tal que $G$ es conjugado a $\C(d_1,
	\ldots, d_n)$
\end{Theorem*}
\begin{proof}
	Llamemos $d$ al menor número tal que $A^d = \Id_n$ para todo $A \in G$. Entonces
	también vale que $A^d = \Id_n$ para todo $A \in G^T$, y como las matrices de $G^T$ son
	diagonales, eso quiere decir que cada matriz tiene en su diagonal raíces $d$-ésimas de
	la unidad. En otras palabras $G^T \subset \C(d, d, \ldots, d)$.
\end{proof}



