\documentclass[smaller,usepdftitle=false]{beamer}
\usepackage[latin1,utf8]{inputenc}
\usepackage[T1]{fontenc}
\usepackage{tikz}
\usetikzlibrary{calc}

\usepackage{mathdots}

\linespread{1.2}

\usetikzlibrary{topaths}


\mode<presentation>
\usetheme{Warsaw}
\setbeamertemplate{navigation symbols}{}
\setbeamertemplate{headline}{}
\setbeamertemplate{footline}{}
\usepackage[matrix,arrow,graph,curve]{xy}
\usepackage{amssymb}
\usepackage{leftidx}
\usepackage{graphics}
\usepackage{cancel}
\usepackage{xcolor}

\newcommand\NN{\mathbb N}
\newcommand\CC{\mathbb C}
\newcommand\QQ{\mathbb Q}
\newcommand\RR{\mathbb R}
\newcommand\ZZ{\mathbb Z}

\newcommand\maps{\longmapsto}
\newcommand\ot{\otimes}
\renewcommand\to{\longrightarrow}
\renewcommand\phi{\varphi}
\newcommand\stack[2]{\genfrac{}{}{0pt}{2}{#1}{#2}}
\newcommand\vectspan[1]{\left\langle #1 \right\rangle}

\newcommand\gl{\mathfrak{gl}}

\DeclareMathOperator\sym{\mathsf{sym}}
\newtheorem*{Proposition}{Proposition}


\def\pausa{\pause \bigskip}


\title[]{Gelfand-Tsetlin modules of $\gl(n,\CC)$ with arbitrary characters}
\author[]{Pablo Zadunaisky - USP, S\~ao Paulo\footnote{FAPESP Postdoc 
grant 2016/25984-1} \\
(Joint work with L.E. Ramírez)}
\date{Dubrovnik, June 2017}

\begin{document}
% Remove this to get the pauses...
%\let\pause\relax

\begin{frame}
\titlepage
\end{frame}


\begin{frame}
\frametitle{Gelfand-Tsetlin tableaux}
A GT-tableau of height $n$ is an array $T(v)$ of $\binom{n}{2}$ complex numbers
\bigskip

\begin{tikzpicture}
\node (n1) at (-3,3) {$v_{n,1}$};
\node (n2) at (-1.5,3) {$v_{n,2}$};
\node (ndots) at (0,3) {$\cdots$};
\node (nn-1) at (1.5,3) {$v_{n,n-1}$};
\node (nn) at (3,3) {$v_{n,n}$};

\node (n-11) at (-2.25,2.25) {$v_{n-1,1}$};
\node (n-1dots) at (0,2.25) {$\cdots$};
\node (n-1n-1) at (2.25,2.25) {$v_{n-1,n-1}$};

\node (dots1) at (-1.5,1.5) {$\ddots$};
\node (dots2) at (0,1.5) {$\cdots$};
\node (dots3) at (1.5,1.5) {$\iddots$};

\node (21) at (-.75,.75) {$v_{2,1}$};
\node (22) at (.75,.75) {$v_{2,2}$};
\node (11) at (0,0) {$v_{1,1}$};

\node (A) at (-4.2, 2.75) {};
\node (B) at (4.2, 2.75) {};
\node (C) at (0,0) {};

\draw[->] (n1) -- (n-11);
\draw[->] (n-11) -- (n2);
\draw[->] (n-11) -- (-1.65,1.7) node{$\phantom{v_{n,n}}$};
\draw[->] (n2) -- (-0.75,2.5);
\draw[->] (nn-1) -- (n-1n-1);
\draw[->] (n-1n-1) -- (nn);
\draw[->] (21) -- (11);
\draw[->] (11) -- (22);
\end{tikzpicture}

\textbf{Elementary:} $\delta^{k,i}$ is the tableaux with $v_{k,i} = 1$, other 
$v_{r,s} = 0$.

\textbf{Standard:} $v_{k,i} - v_{k-1,i} \in \ZZ_{\geq 0}$ and $v_{k-1,i} - 
v_{k,i+1} \in \ZZ_{>0}$.

\textbf{Integral:} $v \in \ZZ \times \ZZ^2 \times \cdots \times \ZZ^n$.

\textbf{Generic:} $v_{k,i} - v_{k,j} \notin \ZZ, 1 \leq k < n$.
\end{frame}

\begin{frame}
\frametitle{A theorem of Gelfand and Tsetlin}

\begin{Theorem}[Gelfand-Tsetlin -'50]
Let $\lambda = (\lambda_1, \ldots, \lambda_n)$ with $\lambda_i - \lambda_{i+1} 
\in \ZZ_{\geq 0}$, and
$$V(\lambda) = \langle T(v) \mid v_{n,i} = \lambda_i - i + 1 \mbox{ and }
T(v) \mbox{ standard }\rangle_\CC.$$ 
The space $V(\lambda)$ is a $\gl(n,\CC)$-module of highest weight $\lambda$
with
\begin{align*}
E_{k,k+1} T(v) &= - \sum_{i=1}^k e^+_{k,i}(v) T(v+\delta^{k,i}) \\
E_{k+1,k} T(v) &= \sum_{i=1}^k e^-_{k,i}(v) T(v-\delta^{k,i}) \\
E_{k,k} T(v) &= e_k(v) T(v)
\end{align*}
where the $e^\pm_{k,i}$ are rational functions, the $e_k$ are symmetric 
polynomials.
\end{Theorem}
\end{frame}

\begin{frame}
\frametitle{Gelfand-Tsetlin subalgebra}

\begin{align*}
\gl(1,\CC) 
	\subset \gl(2,\CC) &\subset \cdots \subset \gl(n,\CC) \\
U(\gl(1,\CC)) 
	\subset U(\gl(2,\CC)) &\subset \cdots \subset U(\gl(n,\CC)) \\
\Gamma 
	= \mbox{ Algebra generated by } &\bigcup_k Z(U(\gl(k,\CC))) 
\subset U(\gl(n,\CC))
\end{align*}

\pause

By Harish-Chandra + Zhelobenko
\begin{align*}
\Gamma 
	&\cong \CC[\lambda_{k,i} \mid 1 \leq i \leq k 
		\leq n]^{S_1 \times \cdots S_n} \\
c &\longmapsto \gamma 
\end{align*}
\end{frame}

\begin{frame}
\frametitle{GT-Tableaux and characters of $\Gamma$}
\begin{align*}
\xymatrix{
	\mbox{Tableaux modulo } S_1 \times S_2 \times \cdots \times S_n
		\ar@/^18pt/[r] & \mbox{Characters of } \Gamma \ar@/^18pt/[l] \\
	\overline{T(v)} \ar@{<~>}[r]
		& \chi_v = \mbox{evaluation at } v
}
\end{align*}
\pause 
\begin{Proposition}[Zhelobenko]
If $T(v) \in V(\lambda)$ then $c T(v) = \gamma(v) T(v)$. In other words
$T(v)$ is the unique eigenvector of eigenvalue $\chi_v$ in $V(\lambda)$.
\end{Proposition}
\end{frame}

\begin{frame}
\frametitle{Gelfand-Tsetlin modules}
\begin{Definition}
A \emph{GT-module} is a $U(\gl(n,\CC))$-module $V$ decomposable as $V = 
\bigoplus_\chi V(\chi)$, with $V(\chi)$ the set of all vectors of
generalized $\Gamma$-eigenvalue $\chi$.
\end{Definition}
\pause

Explicit examples:
\begin{itemize}
\item Finite dimensional modules (Zhelobenko). 

Characters: standard tableaux. Multiplicity: $1$.

\item Generic modules (Drozd, Futorny and Ovsienko). 

Characters: generic. Multiplicity: $1$.

\item Index $2$ modules (Futorny, Grantcharov and Ramírez). 

Characters: pairs of singular entries. Multiplicities: $1, 2$.
\end{itemize}
\end{frame}

\begin{frame}
\frametitle{Generic GT-modules}
\vspace{-1cm}
\begin{gather*}
A = \mbox{Regular functions over generic tableaux}  \\
(\mbox{Localization of }\CC[\lambda_{k,i} \mid 1 \leq i \leq k \leq n], 
\mbox{contains } e^\pm_{k,i}, e_k) 
\end{gather*}

\begin{Theorem}
Let $V_A$ be the free $A$-module over the set $\{T(z) \mid z_{k,i} 
\in \ZZ, z_{n,i} = 0\}$. The module $V_A$ is a $U(\gl(n,\CC))$-module with 
action
\vspace{-.3cm}
\begin{align*}
E_{k,k+1} T(z) &= - \sum_{i=1}^k e^+_{k,i}(\lambda + z) 
	T(z+\delta^{k,i}); \\
E_{k+1,k} T(z) &= \sum_{i=1}^k e^-_{k,i}(\lambda + z) 
	T(z -\delta^{k,i}); \\
E_{k,k} T(z) &= e_k(\lambda + z) T(z).
\end{align*}
Also $c T(z) = \gamma(\lambda + z) T(z)$.
\end{Theorem}
\end{frame}

\begin{frame}
\frametitle{Generic GT-modules}
\begin{Corollary}[Drozd, Futorny, Ovsienko '92]
Let $v$ be a generic tableau, $\CC_v$ the corresponding $1$-dimensional 
$A$-module. Then $V(T(v)) = V_A \ot_A \CC_v$ is a GT-module with $
\dim V(\chi_{v+z}) =1$ for all $z \in \ZZ^N_0$. 
\end{Corollary}
\begin{proof}
The module $V$ has a $\CC$-basis $T(v + z) = T(z) \ot_A 1_v$. The 
action of $U$ is obtained evaluating the GT-formulas at $v$. In particular
$c T(v+z) = \gamma(v+z) T(v+z)$.
\end{proof}
\pause

This clearly does not work if $v$ is not generic.
\end{frame}

\begin{frame}
\frametitle{GT-modules of index $2$ (easy case)}
$B =$ Regular functions in $A$ that can be evaluated at an index $2$ tableau.

(Localization of $\CC[\lambda_{k,i} \mid 1 \leq i \leq k \leq n]$, 
contains $e^\pm_{k,i}, e_k \mbox{ except } e^\pm_{r,s}, 
	e^{\pm}_{r,t}$) 
\pause
\begin{Theorem}[Z, '17]
Let $\tau$ be the involution interchanging coordinates $(r,s)$ and $(r,t)$, 
and let 
\begin{align*}
D_e(z)
	&= \frac{T(z) + T(\tau(z))}{2} = \sym_{S_2}(T(z)),\\
D_\tau(z)
	&= \frac{T(z) - T(\tau(z))}
	{2(\lambda_{r,s} - \lambda_{r,t})} = \sym_{S_2}(\partial_\tau T(z)).
\end{align*}
The $B$-module $L_B \subset V_A$ generated by all the $D_e(z), D_\tau(z)$ is a 
$U$-submodule of $V_A$. 
\end{Theorem}
\end{frame}

\begin{frame}
\frametitle{GT-modules of index $2$ (easy case)}
\begin{Corollary}
Let $v$ be of index at most $2$, let $\CC_v$ be the 
corresponding $B$-module, and let $V(T(v)) = L_B \ot_B \CC_v$.
\begin{itemize}
\item If $v$ is generic then $V(T(v)) \cong V_A \ot_A \CC_v$.

\item If $v$ is critical then $V(T(v))$ is a GT-module with 
$V(T(v))(\chi_{v+z}) = 1$ or $2$. 
\end{itemize}
\end{Corollary}

\begin{proof}[Proof of the theorem and the corollary]
Direct computations inside $V_A$ plus reduction modulo $\lambda_{k,i} - 
v_{k,i}$.
\end{proof}

\end{frame}

\begin{frame}
\frametitle{Construction of modules with arbitrary characters}
Let $v \in \CC^N$. Assume $v$ is in good form (see board).

\begin{itemize}
\item $C(v):$ functions in $A$ defined at $v$.

\item $E(v):$ stabilizer of $v$ in $S_1 \times S_2 \times \cdots \times S_n$.

\item $L(v):$ the $C(v)$-module generated by $\sym_{E(v)} (\partial_\sigma 
T(z))$ for $\sigma \in E$.
\end{itemize}
\pause

\begin{Theorem}[Ramírez, Z' - '17]
$L(v) \subset V_A$ is a $U$-submodule, and $V(T(v)) = L(v) \ot_{C(v)} \CC_v$ 
is a GT-module with $\dim V(T(v))(\chi_{v+z}) = |\frac{E(v)}{E(v) \cap 
E(v+z)}|$.
\end{Theorem}
\end{frame}

\begin{frame}
\frametitle{Example: the $3$-singular module of $\gl(4,\CC)$.}

\begin{tikzpicture}
\node (v) at (-3,0.75) {$v=$};

\node (n1) at (-2,1.5) {$\lambda_1$};
\node (n2) at (-0.75,1.5) {$\lambda_2 + 1$};
\node (ndots) at (0.75,1.5) {$\lambda_3 + 2$};
\node (nn-1) at (2,1.5) {$\lambda_4 + 3$};

\node (n-11) at (-1.5,1) {$0$};
\node (n-1dots) at (0,1) {$0$};
\node (n-1n-1) at (1.5,1) {$0$};

\node (dots1) at (-0.75,0.5) {$\alpha$};
\node (dots2) at (0.75,0.5) {$\beta$};

\node (dots3) at (0,0) {$\theta$};

\node (blabal) at (4, 0.75) {$(\alpha-\beta \notin \ZZ)$};
\end{tikzpicture}
\bigskip 

\begin{tabular}{|c|c|l|}
\hline
$z_3$ & Stabilizer & Nonzero derived tableaux \\
\hline
$(a,b,c)$
  & $(1,1,1)$
  & \parbox[c]{7cm}
    {$D_e(v+z), D_{(12)}(v+z), D_{(23)}(v+z),$\\ $D_{(123)}(v+z), 
    D_{(132)}(v+z), D_{(13)}(v+z)$} \\
\hline
$(a,a,b)$
  & $(2,1)$
  & $D_e(v+z), D_{(23)}(v+z), D_{(123)}(v+z)$ \\
\hline
$(a,b,b)$
  & $(1,2)$ 
  & $D_e(v+z), D_{(12)}(v+z), D_{(132)}(v+z)$ \\
\hline
$(a,a,a)$
  & $(3)$
  & $D_e(v+z)$ \\
\hline
\end{tabular}

where $a>b>c$.
\end{frame}

\begin{frame}
\frametitle{Example: the $3$-singular module of $\gl(4,\CC)$.}
\begin{align*}
E_{4,3} &D_{(13)}(v+z)
  = \frac{1}{2} \frac{\partial p_{3,1}^-}{\partial \lambda_{1,3}}(v+z)
    D_{(123)}(v+z-\delta^{3,1}) + \frac{1}{2} D_{(12)}(v+z-\delta^{3,1})\\
    &+ \frac{1}{4} D_{e} (v+z-\delta^{3,1}) -2 \left( 3 p^-_{2,3}(v+z) - 
    1\right) D_{(13)}(v+z-\delta^{3,2})\\
    &+ D_{(12)}(v+z-\delta^{3,2}) - 
    \frac{\partial p_{3,2}^-}{\partial\lambda_{3,2}}(v+z) D_e(v+z-
    \delta^{3,2})\\
    &+ \frac{1}{2}p_{3,3}^-(v+z) D_{(13)}(v+z-\delta^{3,3}) 
    + \frac{1}{4}p_{3,3}^-(v+z) D_{(123)}(v+z-\delta^{3,3})  \\
    &+ \frac{1}{2}\frac{\partial p_{3,3}^-}{\partial \lambda_{3,3}}(v+z) 
      D_{(132)}(v+z-\delta^{3,3}) + \frac{1}{4}
      \frac{\partial p_{3,3}^-}{\partial \lambda_{3,3}}(v+z) 
      D_{(23)}(v+z-\delta^{3,3}) \\
    &+ \frac{1}{2}D_{(12)}(v+z-\delta^{3,3}) + \frac{1}{4}
    D_{e}(v+z-\delta^{3,3})
\end{align*}
\end{frame}

\begin{frame}
\begin{center}
\huge
Thank you!
\end{center}

\end{frame}
\end{document}