%%%%%%%%%%%%%%%%%%%%%% Generalities %%%%%%%%%%%%%%%%%%5
\documentclass[11pt,fleqn]{article}
\usepackage[paper=a4paper]
  {geometry}

\pagestyle{plain}
\pagenumbering{arabic}
%%%%%%%%%%%%%%%%%%%%%%%%%%%%%%%%
\usepackage{notas}
\usepackage{tikz}
\usepackage{mathdots}

%%%%%%%%%%%%%%%%%%%%%%%%%%% The usual stuff%%%%%%%%%%%%%%%%%%%%%%%%%
\newcommand\NN{\mathbb N}
\newcommand\CC{\mathbb C}
\newcommand\QQ{\mathbb Q}
\newcommand\RR{\mathbb R}
\newcommand\ZZ{\mathbb Z}
\renewcommand\k{\Bbbk}

\newcommand\B{\mathcal B}
\newcommand\F{\mathcal F}
\newcommand\V{\mathcal V}
\newcommand\D{\mathcal D}
\renewcommand\H{\mathcal H}
\renewcommand\O{\mathcal O}

\newcommand\maps{\longmapsto}
\newcommand\ot{\otimes}
\renewcommand\to{\longrightarrow}
\renewcommand\phi{\varphi}
\newcommand\Id{\mathsf{Id}}
\newcommand\im{\mathsf{im}}
\newcommand\coker{\mathsf{coker}}
%%%%%%%%%%%%%%%%%%%%%%%%% Specific notation %%%%%%%%%%%%%%%%%%%%%%%%%
\newcommand\g{\mathfrak g}
\newcommand\gl{\mathfrak{gl}}
\newcommand\gen{\mathsf{gen}}
\newcommand\std{\mathsf{std}}
\newcommand\crit{\mathsf{crit}}
\newcommand\ceq{\hat =}

\newcommand\vectspan[1]{\left\langle #1 \right\rangle}

\renewcommand\ll{\llbracket}
\newcommand\rr{\rrbracket}

\DeclareMathOperator\End{End}
\DeclareMathOperator\Frac{\mathsf{Frac}}
\DeclareMathOperator\Sym{\mathsf{Sym}}
\DeclareMathOperator\Ant{\mathsf{Ant}}
\DeclareMathOperator\ev{ev}
\DeclareMathOperator\sg{sg}
\DeclareMathOperator\sh{sh}
%%%%%%%%%%%%%%%%%%%%%%%%%%%%%%%%%%%%%% TITLES %%%%%%%%%%%%%%%%%%%%%%%%%%%%%%
\title{Fully Singular Gelfand-Tsetlin modules over $\gl(n,\CC)$}
\author{[construction-fully-singular.tex]}
\date{21/04/2017}
\begin{document}
\maketitle

\section{Notation on tableaux}
\textbf{Notation:} For each $n,m \in \NN$ we set $\ll m,n \rr = \{i \in 
\NN \mid m \leq i \leq n\}$; also $\ll n \rr = \ll 1, n\rr$. 

\paragraph
Fix $n \in \NN_{\geq 2}$, and set $N = \frac{n(n+1)}{2}$. We think of 
GT-tableaux as arrays with $N$ complex entries of the form

\begin{tikzpicture}
\node (n1) at (-2,2.5) {$v_{n,1}$};
\node (n2) at (-1,2.5) {$v_{n,2}$};
\node (ndots) at (0,2.5) {$\cdots$};
\node (nn-1) at (1,2.5) {$v_{n,n-1}$};
\node (nn) at (2,2.5) {$v_{n,n}$};

\node (n-11) at (-1.5,2) {$v_{n-1,1}$};
\node (n-1dots) at (0,2) {$\cdots$};
\node (n-1n-11) at (1.5,2) {$v_{n-1,n-1}$};

\node (dots1) at (-1,1.625) {$\ddots$};
\node (dots2) at (0,1.5) {$\cdots$};
\node (dots3) at (1,1.625) {$\iddots$};

\node (21) at (-.5,1) {$v_{2,1}$};
\node (22) at (.5,1) {$v_{2,2}$};
\node (11) at (0,.5) {$v_{1,1}$};

\node (A) at (-3.5, 2.75) {};
\node (B) at (3.5, 2.75) {};
\node (C) at (0,0) {};
%\draw (A) -- (B) -- (C) -- (A);
\end{tikzpicture}

The set of all GT-tableaux can be identified with $\CC^N$ by indexing the 
entries of a point $v \in \CC^N$ by $\{(k,i) \mid 1 \leq i \leq k \leq n\}$. 
We will denote by $T(v)$ the tableau corresponding to $v \in \CC^N$. We denote
by $\ZZ^N_0$ the set of all tableaux with $z_{k,j} \in \ZZ$ and all $z_{n,i} 
= 0$ (i.e. talbeaux with integral entries and top row $0$).

\begin{Definition}
\label{D:various-tableaux}
We say that a point $v \in \CC^N$, or the corresponding tableau $T(v)$, is:
\begin{itemize}
\item \emph{integral} if $v \in \ZZ^{N}$;

\item \emph{standard} if for all $1 \leq i < j \leq k \leq n$ the difference 
$v_{k,i}-v_{k-1,i} \in \ZZ_{\geq 0}$ and $v_{k-1,i}-v_{k,i+1} \in \ZZ_{>0}$;

\item \emph{generic} if for all $1 \leq i < j \leq k < n$ the difference 
$v_{k,i}-v_{k,j}$ is not in $\ZZ$;

\item \emph{singular} if it is a non-generic tableau, i.e. there is a pair of 
entries in the same row differing by an integer; if there is exactly one such 
pair then we say that the tableau is \emph{$1$-singular};

\item \emph{critical} if there are two entries in the same row which are 
equal; if there is exactly one such pair then we say the tableau is
\emph{$1$-critical}.
\end{itemize}
\end{Definition}
We denote the set of standard points by $\CC^N_\std$, and the set of generic 
points by $\CC^N_\gen$. Any pair of two entries differing by an integer will
be called a \emph{singular pair}, and if they are equal they will be called
a \emph{critical pair}. 

\paragraph
Let $m \in \ll n \rr$. The group $S_m$ acts on the space of GT-tableaux by 
permuting the entries of the $m$-th row. Let $v \in \CC^N$. Its 
\newterm{$m$-shape} is a partition $\sh_m(v) = (\lambda_1, \ldots, \lambda_r) 
\vdash m$, where $\lambda_i$ is the frequency of the $i$-th most common value 
of the entries of $v$. For example, $\sh((1,4,1,1,4,1,6)) = (4,2,1)$. In this 
case the orbit of $v$ under the action of $S_m$ generates an $S_m$-module. 
Setting $S_v = S_{\lambda_1} \times S_{\lambda_2} \times \cdots S_{\lambda_r}
\subset S_m$, the orbit of $v$ is isomorphic to the induced representation
$\operatorname{Ind}_{S_v}^{S_m} \mathbf{1}$, where $\mathbf 1$ is the trivial
representation of $S_v$. In particular the dimension of the orbit of $v$ is 
$m!/\lambda!$, where $\lambda! = \lambda_1! \lambda_2! \cdots \lambda_r!$.

\section{Auxiliary results on invariant polynomials}

\paragraph
\about{Notation}
\label{L:symmetric-varia}
We denote by $\Lambda$ the polynomial ring $\CC[\lambda_{k,i} \mid 1 \leq i
\leq k \leq n]$. We set $G = S_1 \times S_2 \times \cdots \times S_n$, which
acts on $\Lambda$, and write $\Gamma = \Lambda^G$.

Fix $m \in \NN$. We set $x_i = \lambda_{m,i}$ for $1 \leq i \leq m$, and
$\Lambda_m = \CC[x_1, \ldots, x_m] \subset \Lambda$. Also set $\Gamma_m =
\Lambda_m^{S_m} \subset \Gamma$. Given $f \in \Lambda_m$ we will sometimes 
write $f^\sigma$ for the action of $\sigma$ on $f$. 

Set
\begin{align*}
F &= \Frac(\Lambda)
	&E &= \Frac(\Gamma) \\
F_m &= \Frac(\Lambda_m)
	&E_m &= \Frac(\Gamma_m)
\end{align*}
Given $f \in F$ we set
\begin{align*}
\Sym(f) 
	&= \frac{1}{m!}\sum_{\sigma \in S_m} f^\sigma
&\Ant(f)
	&= \frac{1}{m!}\sum_{\sigma \in S_m} \sg(\sigma) f^\sigma
\end{align*}
We also set $\Delta = \prod_{i<j}(x_i-x_j)$. The following lemma will be very 
useful in the sequel.

\begin{Lemma*}
Let $f \in A$ where $\Lambda_m \subset A \subset F$ is a ring stable by the 
action of the symmetric group. Then the following hold. 
\begin{enumerate}[(a)]
\item $\Sym(f)^\sigma = \Sym(f)$ and $\Ant(f)^\sigma = \sg(\sigma) \Ant(f)$.

\item The polynomial $f -f^{(i,j)}$ is divisible by $x_i - x_j$ in $A$.

\item If $f^\sigma = \sg (\sigma) f$ for each $\sigma \in S_m$ then $\Delta$ 
divides $f$ in $A$ and $f/\Delta$ is an invariant element of $A$. In particular
$\Delta$ divides $\Ant(f)$, and $\Ant(f)/\Delta$ is invariant.
\end{enumerate}
\end{Lemma*}

\paragraph
\label{L:basis-lemma}
The algebra $\Gamma_m$ is a graded subalgebra of $\Lambda_m$, isomorphic to a 
polynomial algebra in $m$-variables and its $i$-th generator in degree $i$; 
consequently, its Hilbert series is $\prod_{i = 1}^m \frac{1}{1-t^i}$, see 
\cite{Macd-symmetric-hall}*{(2.4)}. 

Let $J \subset \Lambda_m$ be the ideal generated by symmetric polynomials of 
positive degree. This is a graded ideal stable by the action of $S_m$, and by 
\cite{Chev-reflection-groups}*{Theorem (B)} the quotient $\Lambda_m/J$ is
a graded version of the regular representation of $S_m$, with Hilbert series
$\prod_{i=1}^m \frac{1-t^i}{1-t}$. By Nakayama's Lemma a subset of $\Lambda_m$
is a minimal set of generators of $\Lambda_m$ over $\Gamma_m$ if and only if
its projection on $\Lambda_m/J$ is a basis of this vector space. Thus if the 
generating set is formed by homogeneous elements then the degrees of its 
elements are fixed, and looking at the Hilbert series of the algebras we see 
that such a minimal set is free over $\Gamma_m$. From now on when we talk 
about a basis of $\Lambda_m$ we will mean a homogeneous basis over $\Gamma_m$.

Bases of $\Lambda_m$ are usually indexed by elements of the symmetric group,
with $\deg f_\sigma = \ell(\sigma)$; here $\ell$ is the usual length function 
of $S_m$ with respect to the generating set $\{(i,i+1) \mid 1 \leq i \leq m-1
\}$. Two well known examples are Schubert polynomials  
\cite{Man-symm-book}*{Definition 2.3.4} and a family of harmonic polynomials 
\cite{Man-symm-book}*{Lemma 2.5.9}. Our bases will always follow these 
conventions. From the discussion above the following lemma is clear.

\begin{Lemma*}
Let $\B = \{f_\sigma \mid \sigma \in S_m\}$ be a basis of $\Lambda_m$, and let
$\B' = \{g_\sigma \mid \sigma \in S_m\} \subset \Lambda_m$ with each $g_\sigma$
homogeneous of degree $\ell(\sigma)$. The following are equivalent: 
\begin{enumerate}[(a)]
\item \label{B'-basis}
$\B'$ is a basis of $\Lambda_m$.

\item 
\label{projection-basis}
The projection of $\B'$ to the quotient $\Lambda_m/J$ is a basis.

\item
\label{coefficients-basis}
There exist $\{p_{\sigma,\tau} \mid \ell(\sigma) \leq \ell(\tau), \sigma, \tau 
\in S_m\} \subset \Gamma_m$ such that 
\begin{enumerate}[(i)]
\item $g_\tau = \sum_{\ell(\sigma) \leq \ell(\tau)} p_{\sigma,\tau}f_\sigma$;
\item $\deg p_{\sigma, \tau} = \ell(\tau) - \ell(\sigma)$;
\item for each $i \in \ll m \rr$ the square matrix $[p_{\sigma,\tau}]$ with 
$\ell(\sigma) = \ell(\tau) = i$ is invertible.
\end{enumerate}
\end{enumerate}
\end{Lemma*}
Notice that the $p_{\sigma,\tau}$ of item $(c)$ form the matrix of coordinates 
of the $g_\tau$ in the basis of $f_\sigma$, and in particular the matrices 
from subitem $(iii)$ are formed by the degree-wise coordinates of the classes
of $g_\tau$ in $\Lambda_m/J$.


\paragraph
We define a $\Gamma_m$-bilinear form $\phi: \Lambda_m \ot_{\Gamma_m} 
\Lambda_m \to \Gamma_m$, given by $\phi(f,g) = \frac{1}{\Delta}\Ant(fg)$. This 
is well defined thanks to Lemma \ref{L:symmetric-varia}. By definition
\begin{align*}
\phi(f^\tau, g)
	&= \frac{1}{\Delta} \sum_{\sigma \in S_m} \sg(\sigma) (f^\tau g)^\sigma
	= \frac{1}{\Delta} \sum_{\sigma \in S_m} \sg(\sigma) 
		(f g^{\tau^{-1}})^{\tau\sigma} \\
	&= \frac{\sg(\tau^{-1})}{\Delta} \sum_{\sigma \in S_m} \sg(\sigma) 
		(f g)^{\tau^{-1}\sigma}
	= \sg(\tau^{-1})\phi(f, g^{\tau^{-1}})  
\end{align*}
The bilinear form introduced in \cite{Man-symm-book}*{2.5.3} is 
$\phi(f,g)(0)$. Both forms induce the same bilinear pairing over $\Gamma_m/J$,
which by \cite{Man-symm-book}*{Proposition 2.5.7} is non-degenerate, 
although it is not a scalar product since for example $\phi(\Delta,\Delta) = 
0$.

We say that two bases $\{f_\sigma \mid \sigma \in S_m\}$ and $\{g_\sigma \mid 
\sigma \in S_m\}$ are dual with respect to $\phi$ if $\phi(f_\sigma, g_\nu) = 
\delta_{\sigma\nu, w_0}$, where $w_0$ is the longest word of $S_m$.

\begin{Proposition*}
Every homogeneous basis of $\Lambda_m$ has a dual basis with respect to 
$\phi$. 
\end{Proposition*}
\begin{proof}
Take the projection of a basis $\B = \{f_\sigma \mid \sigma \in S_m \}$ onto 
$\Lambda_m/J$, and denote it $\pi(\B)$. Since $\phi$ induces a nondegenerate 
bilinear form over this vector space, $\pi(\B)$ has a dual basis with respect
to the induced form. Fix a set $\tilde \B = \{\tilde f_\sigma \mid \sigma \in 
S_m\}$ of homogeneous polynomials whose projection to the quotient gives the dual bases, labelling them so $\phi(f_\sigma, \tilde f_{\nu}) = 
\delta_{\sigma \nu, w_0} + p$, where $p$ is a polynomial of strictly positive 
degree. Since its projection is a basis, $\tilde \B$ is a homogeneous basis 
of $\Lambda_m$ over $\Gamma_m$.

Let $X$ be the matrix with rows and columns indexed by $S_m$ with $X_{\sigma, 
\nu} = \phi(f_\sigma, \tilde f_{\nu^{-1} w_0})$. By definition, $\phi(f,g)$ is 
either zero or a homogeneous symetric polynomial of degree $\deg f + \deg g - 
\deg \Delta$; this forces that $\deg f_{\nu^{-1} w_0} = \deg \Delta - \deg f
= \ell(w_0) - \ell(\nu) = \ell(\nu^{-1}w_0)$, and hence $\tilde B$ is 
compatible with our conventions regarding the degrees. Furthermore
\begin{align*}
\phi(f_\sigma, \tilde f_\nu) = \begin{cases}
		0 & \mbox{ if } \deg f_\sigma + \deg \tilde f_\nu < \ell(w_0) \\
		\delta_{\sigma\nu, w_0} 
			& \mbox{ if } \deg f_\sigma + \deg \tilde f_\nu = \ell(w_0);  \\
		p \in (\Gamma_m)_{>0} 
			& \mbox{ if } \deg f_\sigma + \deg \tilde f_\nu > \ell(w_0).
	\end{cases}
\end{align*}
Hence if we order the elements of $S_m$ by length then $X$ is a lower 
triangular matrix with $1$'s in the diagonal, so there exists a matrix
$M$ with coefficients in $\Gamma_m$ such that $X M = \Id$, i.e.
\begin{align*}
\sum_{\nu \in S_m} (f_\sigma,  M_{\nu,\tau} \tilde f_{\nu^{-1}w_0})
	&= \delta_{\sigma, \tau}.
\end{align*}
Setting $f^*_\sigma = \sum_{\nu} M_{\nu, \sigma^{-1}w_0} \tilde 
f_{\nu^{-1}w_0}$ we obtain the dual basis $\B^* = \{f^*_\sigma \mid \sigma \in
S_m\}$.
\end{proof}

\paragraph
\label{homogeneous-basis}
Let $J_i$ be the homogeneous component of degree $i$ of $J$, which sits inside
$(\Lambda_m)_i$. Since both are $S_m$-modules, $J_i$ has an $S_m$ equivariant
complement (in fact many) which we denote by $H_i$. Set $H = \bigoplus_i H_i$,
which is clearly isomorphic to $\Lambda_m / J$ as a graded $S_m$-module. Thus
$H$ is a graded version of the regular representation of $S_m$, and in 
particular we can choose a basis of $\Lambda_m$ consisting of homogeneous 
generators of $H$. In that case we say that the basis realizes the regular 
representation. Its distinguishing property is that $f_\sigma^\tau = \sum_\rho
\lambda^\rho_{\sigma, \tau} f_\rho$, where the $\lambda_{\sigma,\tau}^\rho$
are all complex coefficients (i.e. symmetric polynomials of degree $0$) with 
$\ell(\rho) = \ell(\sigma)$. 

\textcolor{red}{QUESTION: Can the basis be chosen so that it is self-dual with 
respect to $\phi$, or at least so that its dual basis is easily described in 
terms of the original? I found several self-dual bases for $m = 3$, but 
Schubert polynomials are not one of them. They simplify many things.}

The following are a few self-dual bases (up to constants) for $m = 3$. Here
$\omega$ is a primitive third root of unity.
\begin{align*}
f_1 &=1 
	&f_{(12)}&=2 \left(x_1-x_2\right)
	&f_{(23)}&=2 \left(x_2-x_3\right) \\
f_{(132)} &= x_2 x_3-x_1 x_3
	&f_{(123)} &= x_1 x_2-x_1 x_3
	&f_{(13)} &= \Delta; \\ \\
g_1 &=1
	&g_{(12)}&=2 \left(x_1-x_2\right)
	&g_{(23)}&=2 \left(x_2-x_3\right) \\
g_{(132)} &= 2(x_1^2 - x_2^2)
	&g_{(123)} &= 2(x_2^2 - x_3^2)
	&g_{(13)} &= \Delta \\ \\
h_1 &=1
	&h_{(12)}&=x_1 + \omega x_2 + \omega^2 x_3
	&h_{(23)}&=x_1 + \omega^2 x_2 + \omega x_3\\
h_{(132)} &= x_1^2 + \omega x_2^2 + \omega^2 x_3^2
	&h_{(123)} &= x_1 + \omega^2 x_2^2 + \omega x_3^2
	&h_{(13)} &= \Delta \\  \\
t_1 &=1 
	&t_{(12)}&= \left(x_1-x_2\right)/3
	&t_{(23)}&= \left(x_2-x_3\right)/3 \\
t_{(132)} &= 2(2 x_1-x_2-x_3)(x_2-x_3) 
	&t_{(123)} &= 2(x_1+x_2-2 x_3)(x_1-x_2)
	&t_{(13)} &= \Delta
\end{align*}


\section{The universal fully critical GT-module}

\paragraph
Let $A$ be the algebra of regular functions defined over $\CC^N_\gen$; in other
words, $A$ is the localization of $\Lambda = \CC[\lambda_{k,i} \mid 1 \leq i 
\leq k \leq n]$ at the infinite set $\{\lambda_{k,i} - \lambda_{k,j} - a \mid
1 \leq i < j \leq k < n, a \in \ZZ\}$. The algebra $A$ contains all the 
rational functions appearing in the Gelfand-Tsetlin formulas. Thus if $v \in
\CC_\gen^N$ then evaluation at $v$ is defined for every rational function 
$f/g \in A$ and defines a $1$-dimensional representation of $A$, which we
denote by $\CC_v$. Notice that $\Lambda \subset A \subset \Frac(\Lambda)$, and 
that $A$ is stable by the action of $S_m$. Consider the action of $\ZZ^N_0$ 
over $\CC^N$, given by $v^z = v+z$. This induces an action on $A$, given by 
$(z \cdot f)(v) = f(v + z)$. We will sometimes write $f(\lambda^z)$ for 
$(z \cdot f)$. 

\paragraph
Let $V_\CC$ be the $\CC$ vector space with basis $\{T(z) \mid z \in \ZZ^N_0\}$.
The symmetric group $S_m$ acts on $V_\CC$ through its action on the $m$-th row
a tableau $z$. Set $V_A = A \ot_\CC V_\CC$; we write $fT(z)$ for the 
elementary tensor $f \ot_\CC T(z)$ where $f \in A$ and $z \in \ZZ_0^N$. The 
symmetric group acts on $V_A$ by a diagonal action, so $\sigma \cdot (f T(z)) 
= f^\sigma T(\sigma(z))$. 

The Gelfand-Tsetlin subalgebra of $U$ is isomorphic to $\Gamma$ by a 
Harish-Chandra homomorphism. It has a standard family of generators $c_{k,i}$ 
with $1 \leq i \leq k \leq n$, and the Harish-Chandra isomorphism maps them to 
polynoamis $\gamma_{k,i}$. The following proposition was proved in 
\cite{Zad-1-sing}.

\begin{Proposition*}
\label{P:universal-generic-GT-module}
The $A$-module $V_A$ can be endowed with the structure of a $U$-module 
with the action of the canonical generators given by
\begin{align*}
E_{k,k+1} T(z) 
	&= - \sum_{i=1}^k \frac{p^+_{k,i}(\lambda^z)}{q_{k,i}(\lambda^z)} 
		T(z + \delta^{k,i}); \\
E_{k+1,k} T(z) 
	&= \sum_{i=1}^k \frac{p^-_{k,i}(\lambda^z)}{q_{k,i}(\lambda^z)} 
		T(z - \delta^{k,i}); \\
E_{k,k} T(z)
	&= (|\lambda^z|_k - |\lambda^z|_{k-1} + k -1) T(z),
\end{align*}
and for each $1 \leq i \leq k \leq n$, we have $c_{k,i} T(z) = 
\gamma_{k,i}(\lambda^z) T(z)$. The action of $U$ thus defined is 
$S_m$-equivariant.

Furthermore, given $v \in \CC^N_\gen$, the Gelfand-Tsetlin module associated
to the generic tableaux $T(v)$ is isomorphic to $V_A \ot_A \CC_v$.
\end{Proposition*}

\paragraph
\label{derived-tableaux}
Let $B \subset A$ be the subalgebra formed by rational functions $f/g \in A$
such that $(x_i - x_j) \nmid g$ for $1 \leq i < j \leq k$.
These are precisely the regular functions on $\CC_\gen^N$ that can be extended 
to the space of tableaux with fully-critical $m$-th row (recall that $x_i = 
\lambda_{m,i}$).

Let $\B = \{f_\sigma \mid \sigma \in S_m\}$ be a basis of $\Lambda_m$ over 
$\Gamma_m$. For each $v \in \ZZ^N_0$ set
\begin{align*}
D_\sigma^\B T(v)
	&= \frac{1}{m!} \sum_{\tau \in S_m} 
		\tau \left(\frac{f_\sigma}{\Delta}T(v)\right)
	= \frac{1}{m!}
		\sum_{\tau \in S_m} \sg(\tau) \frac{f_\sigma^\tau}{\Delta} T(\tau(v))
		\in V_A.
\end{align*}
By construction $\tau \cdot D_\sigma^\B T(v) = D_\sigma^\B T(v)$ for all 
$\tau \in S_m$. Let $g \in \Lambda_m$. Then there exist $p_\sigma \in 
\Gamma_m$ such that $g = \sum_\sigma p_\sigma f_\sigma$ and
\begin{align*}
\frac{1}{m!} \sum_{\tau \in S_m} 
	\tau \left(\frac{g}{\Delta}T(v)\right)
	&= \frac{1}{m!} \sum_{\tau \in S_m} 
		\tau \left(\frac{\sum_\sigma p_\sigma f_\sigma}{\Delta}T(v)\right) 
	= \sum_\sigma p_\sigma D_\sigma^\B T(v).
\end{align*}
In particular the $\Gamma_m$-module generated by the set $\{D_\sigma^\B T(v) 
\mid \sigma \in S_m\}$ inside $V_A$ is independent of $\B$. 

\begin{Definition*}
We refer to the elements of the set $\{D_\sigma^\B T(v) \mid \sigma \in S_m\}$ 
as the \newterm{derived tableaux of $T(v)$ with respect to $\B$}. For any 
algebra $\Gamma_m \subset C \subset B$ we denote by $L_C(v)$ the $C$-module 
spanned by the derived tableaux of $T(v)$, and by $L_C$ the $C$-submodule
$\sum_{v \in \ZZ^N_0} L_C(v) \subset V_A$.
\end{Definition*}

\paragraph
Let $\B^* = \{g_\sigma \mid \sigma \in S_m \}$ be the dual basis of $\B$ 
with respect to $\phi$. Let $X_{\sigma, \tau} = \frac{\sg(\tau) 
f_\sigma^\tau}{\Delta}$ and $Y_{\rho, \nu} = g_{\nu^{-1}w_0}^\rho$.
Then $\sum_\tau X_{\sigma, \tau} Y_{\tau, \nu} = \phi(f_\sigma, 
g_{\nu^{-1}w_0}) = \delta_{\sigma, \nu}$, so $X$ and $Y$ are inverses. This
implies that 
\[
	\delta_{\rho, \tau} = \sum_\nu Y_{\rho, \nu} X_{\nu, \tau}
		= \sum_\nu \frac{\sg(\tau) g_{\nu^{-1}w_0}^\rho f_\nu^\tau}{\Delta}.
\]
Thus
\begin{align*}
\sum_{\nu \in S_m} m! g_{\nu^{-1}w_0}^\rho D_\nu^\B T(v)
	&= \sum_{\tau \in S_m} \sum_\nu  
		\frac{\sg(\tau) g_{\nu^{-1}w_0}^\rho f_\nu^\tau}{\Delta}
			T(\tau(v))
	= T(\rho(v)).
\end{align*}
Hence the orbit of $T(v)$ lies inside $L_{\Lambda_m}(v)$.

\begin{Lemma}
Let $\B$ be a basis realizing the regular representation, and let $L_\CC(v)$
be the vector space generated by derived tableaux associated to $T(v)$ with
respect to $\B$. Then $\dim L_\CC(v) = m!/\sh_m(v)!$ and $L_C(v) = C \ot_\CC
L_\CC(v)$.
\end{Lemma}
\begin{proof}
For simplicity we write $D_\sigma(v)$ for $D_\sigma^\B T(v)$.
Let $H$ be the vector space generated by $\B$ in $\Lambda_m$; by defintion 
this set is stable by the action of $S_m$ and realizes the regular 
representation. As mentioned in \ref{homogeneous-basis}, the action of $S_m$
over the polynomials of the basis is given by scalars 
$\lambda_{\sigma, \tau}^\rho \in \CC^\times$ such that $f_\sigma^\tau = 
\sum_\rho \lambda_{\sigma, \tau}^\rho f_\rho$. 

Now let $F = \Frac(\Lambda_m)$. Clearly $L_F(v)$ is contained in the $F$-vector
space generated by the orbit of $T(v)$, and by the previous paragraph it must
be equal to this vector space, which is of dimension $m!/\sh(v)$. Thus there 
must be a set of $m!/\sh(v)$ derived tableaux which are linearly independent 
over $F$, and in particular over $\CC$. We will show that these tableaux are
enough to generate $L_\CC(v)$.

Let $\O(v)$ be the vector space generated by the orbit of $v$ through the 
action of $S_m$. We put on $\O(v)$ a right $S_m$-action, given by $\tau(v) 
\cdot \nu = \sg(\nu) \nu^{-1}\tau(v)$. Now let $D: \O(v) \ot H \to L_\CC(v)$
be the map $D(\tau(v) \ot f_\sigma) = D_\sigma \tau^{-1}(v)$. By definition
\begin{align*}
D_\sigma (\tau^{-1}(v))
	&= \frac{1}{m!} \sum_{\nu \in S_m} 
		\nu \left(\frac{f_\sigma}{\Delta}T(\tau^{-1}(v))\right)
	= \frac{1}{m!} \sum_{\nu \in S_m} 
		\nu \tau^{-1} \left(\frac{f_\sigma^\tau}{\Delta^\tau}T(v)\right) \\
	&= \frac{\sg(\tau^{-1})}{m!} \sum_{\nu \in S_m} 
		\nu \left(\frac{\sum_\rho \lambda^\rho_{\sigma, \tau} f_\rho}{\Delta}
			T(v)\right)
	= \sg(\tau^{-1}) \sum_\rho \lambda_{\sigma, \tau}^\rho D_\rho (v). 
\end{align*} 
Thus $D(v \cdot \tau \ot f) = D(v \ot f^\tau)$, and hence $D$ induces a
surjective map $D: \O(v) \ot_{S_m} H \to L_\CC(v)$. Now $H$ is the regular
representation, so $\O(v) \ot_{S_m} H \cong O(v)$ as vector space, which 
implies that $\dim_\CC L_\CC(v) \leq \dim \O(v) = m!/\sh(v)!$. Thus the number 
of $\CC$ linearly independent derived tableaux is at most $m!/\sh(v)!$, and we 
have seen that this set must also be $F$-linearly independent, so $L_C(v)$ is 
a free $C$-module with these tableaux as a basis.
\end{proof}


\begin{Proposition}
\label{derived-tableaux-basis}
Let $\B$ be a basis realizing the regular representation of $S_m$.
The $B$-module $L_B \subset V_A$ is free with a basis of derived tableaux.
\end{Proposition}
\begin{proof}
We have proved that each $L_B(v)$ is free over $B$ with a basis of derived 
tableaux. Now if $v = \sigma(w)$ for some $\sigma \in S_m$ then $L_B(v) = 
L_B(w)$; thus taking $\overline v \in \ZZ^N_0 / S_m$, we may set 
$L_B(\overline v) = L_B(v)$. Since $L_F(v)$ is the $F$-vector space generated 
by the orbit of $v$, the spaces $\{L_B(\overline v) \mid \overline v \in 
\ZZ^N_0\}$ are in direct sum, and $L_B = \bigoplus L_B(\overline v)$. 
\end{proof}

Let $v \in \ZZ^N_0$. Then there exists an element $\sigma \in S_m$ such that
the $m$-th row of $\sigma(v)$ is an increasing sequence. Amongst all such 
$\sigma$, there is exactly one which does not interchange two equal entries; 
an uneducated guess is that it should be of minimal length among.
Let us denote this element by $\sigma_v$. If $w = \tau(v)$, then $\sigma_w = 
\sigma_v$ implies that $w = v$, so $\tau$ is in the stabilizer of $v$. In other
words, $w \neq v$ implies that the coclasses of $\sigma_w$ and $\sigma_v$
modulo the stabilizer of $v$ are different, and hence the set 
$\{\sigma_{\tau(v)} \mid \tau (v)\}$ forms a complete collection of 
representatives of coclasses modulo de stabilizer of $v$. 

\textcolor{red}{QUESTION: Is there a basis of derived tableaux of $L(v)$, 
naturally indexed by the set of pairs $(w,\sigma_w)$ with $w = \sigma(v)$?
Enrique thinks yes, I like the idea, and the numbers match. In the end it
depends on the chosen $\B$.}


\begin{Lemma}
Let $v\in \ZZ^N_0$. Then $E_{k,k+1} T(v) \in L_B$ and $E_{k+1,k}T(v) \in L_B$
for every $k \in \ll n \rr$.
\end{Lemma}
\begin{proof}
We work with a basis of derived tableaux, which we denote by $\{D_\sigma(w)
\mid (w, \sigma) \in I\}$ where $I$ is some indexing set.

The statement is true by the definition of the action of the generators of $U$
unless $k = m$. Let $\Sigma(v) = \{(i,j) \mid v_{m,i} = v_{m,j} \mbox{ and } 
i <j\}$, and let
\[
	\Delta(v) = \prod_{(i,j) \in \Sigma(v)} x_i - x_j.
\]
By definition $\Delta(v) E_{m,m+1} T(v)$ is a $B$-linear combination of 
tableaux $T(w)$ with $w \in \ZZ^N_0$, and these tableaux lie in $L_B$. Thus we 
can write
\[
\Delta(v) E_{m,m+1} T(v)
	= \sum_{(w,\sigma)} p_{w,\sigma} D_\sigma T(w)
\]
where the sum is over $I$, and each $p_{w, \sigma} \in B$. Now if $(i,j) \in 
\Sigma(v)$ then $(\Delta(v) E_{m,m+1} T(v))^{(i,j)}= - \Delta(v) E_{m,m+1} 
T(v)$, and since $D_\rho T(w)^{(i,j)} = D_\rho T(w)$ and the writing is 
unique, we see that $p_{w, \sigma}^{(i,j)} = - p_{w, \sigma}$, which in turn 
implies that $p_{\rho,w}$ is divisible by $x_i - x_j$ in $B$ by Lemma 
\ref{L:symmetric-varia}. Since $(i,j)$ was arbitrary, we see that $\Delta(v) 
\mid p_{\rho,w}$ in $B$, and so
\[
E_{m,m+1} T(v)
	= \sum \frac{p_{\rho, w}}{\Delta(v) } D_\rho T(w) \in L_B.
\]
The proof that $E_{k+1,k} T(v) \in L_B$ is similar.
\end{proof}


\begin{Proposition*}
The $B$-module $L_B$ is a $U$-submodule of $V_A$.
\end{Proposition*}
\begin{proof}
We wish to prove that $E_{k,k+1} D_\sigma T(v) \in L_B$ for all $k \in
\ll n-1 \rr$ and all $\sigma \in S_m$. Now
\begin{align*}
\Delta E_{k,k+1} D_\sigma T(v)
	&= \Delta \sum_{\tau \in S_m} \tau \left( 
		\frac{f_\sigma}{\Delta} E_{k+1,k} T(v)
	\right)
	= \sum_{\tau \in S_m} \sg(\tau) \tau \left( 
		f_\sigma E_{k+1,k} T(v)
			\right) \in L_B
\end{align*}
and once again we can write
\begin{align*}
\Delta E_{k,k+1} D_\sigma T(v)
	&= \sum_{D} p_{\rho, w} D_{\rho} T(w).
\end{align*}
The same reasoning as above shows that $p_{\rho,w}/\Delta \in B$, so
\begin{align*}
E_{k,k+1} D_\sigma T(v)
	&= \sum_{D} \frac{p_{\rho, w}}{\Delta} D_{\rho} T(w) \in L_B.
\end{align*}
The proof that $E_{k+1,k} D_\sigma (v) \in L_B$ is analogous. Finally,
since $L_B(v)$ is contained in the $A$-module generated by the orbit
of $T(v)$, and each tableaux in the orbit of $T(v)$ is an eigenvector of
$E_{k,k}$ of eigenvalue $|\lambda + v|_k - |\lambda + v|_{k-1} + k-1$, it 
follows that every element of $L_B(v)$ is an eigenvector of the same 
eigenvalue.
\end{proof}

\begin{Definition}
Let $\overline B = B/I$, where $I$ is the ideal generated by $\{x_i - x_j \mid 
1 \leq i < j \leq m\}$. The \newterm{universal fully critical GT-module} is the
$\overline B$-module $L_{\overline B} = L_B \ot \overline B$.
\end{Definition}

\textcolor{red}{QUESTION: Can we find explicit formulas for the action of $U$ 
on $L_B$ or $L_{\overline B}$?}

Here is one idea on how to do it. Put on $V_A$ the only inner product with 
coefficients in $A$ (ok, bilinear form) such that $\langle T(v), T(w) \rangle
= \delta_{v,w}$. Now set $D^*_\sigma T(v) = \sum_\tau \tau(f_\sigma^* T(v))$, 
where the $f_\sigma^*$'s are dual to the $f_\sigma$ with respect to $\phi$. 
Then
\begin{align*}
\langle D_\sigma T(v), D^*_\nu T(v)\rangle
	&= \cdots
	= \sum_{\epsilon}\sg \epsilon \phi(f_\sigma^\epsilon, f^*_\nu)
\end{align*}
where $\epsilon$ runs over the stabilizer of $v$ in $S_m$. In particular if $v$
is noncritical (all entries are different and hence the stabilizer is trivial) 
then this is $\delta_{\sigma\nu, w_0}$, and the coefficient of $D_\sigma(T(v + 
\delta^{k,j}))$ in $E_{m,m+1} D_\tau T(v)$ can be obtained by taking the inner
product of this last guy with $D^*_\sigma T(v + \delta^{k,j})$. However, for 
this to work in general we need a basis of derived tableaux and its dual basis.
Hence Enrique's suggestion in \ref{derived-tableaux-basis}. Once we do that
we still need to see what happens when we reduce these formulas modulo $x_1 = 
x_2 = \cdots = x_m$, which in this context is equivalent to evaluating in a 
(generic) fully critical tableaux.

\paragraph
Let us consider the action of the Gelfand-Tsetlin algebra on the module $L_B$.
For each $1 \leq i \leq k \leq n$ we see
\begin{align*}
c_{k,i} D_\sigma T(z)
	&= \frac{1}{m!} \sum_{\tau \in S_m} 
		\tau \left(\frac{f_\sigma}{\Delta} c_{k,i}T(z)\right)
	= \frac{1}{m!} \sum_{\tau \in S_m} 
		\tau \left(\frac{f_\sigma}{\Delta}\gamma_{k,i}(\lambda^z)T(z)\right).
\end{align*}
Now $\tau \cdot \gamma_{k,i} = \gamma_{k,i}$, so we get that $\tau \cdot 
\gamma_{k,i}(\lambda^z) = \gamma_{k,i}(\lambda^{\tau(z)})$, so 
\begin{align*}
c_{k,i} D_\sigma T(z)
	&= \frac{1}{m!} \sum_{\tau \in S_m} \gamma_{k,i}(\lambda^{\tau(z)})
		\tau \left(\frac{f_\sigma}{\Delta}T(z)\right).
\end{align*}
Using the inner product described above
\begin{align*}
\langle D_\sigma T(z), D^*_\nu T(z) \rangle
	&= \frac{1}{m!} \sum_{\tau(z) = \rho(z)} 
		\gamma_{k,i}(\lambda^{\tau(z)})
			\left(\frac{f_\sigma}{\Delta}\right)^\tau (f_\nu^*)^\rho \\
	&= \frac{1}{m!}\sum_\epsilon \sg(\epsilon)
		\phi(\gamma_{k,i}(\lambda^{\tau(z)}) f_\sigma^\epsilon f_\nu^*),
\end{align*}
where once again $\epsilon$ runs over the stabilizer of $v$ in $S_m$.

\textcolor{red}{QUESTION: How does this reduce modulo $x_1 = x_2 = \cdots = 
x_m$?}

\paragraph
\about{Specialization of $L_B$ is a GT-module}
Let $w$ be a fully-critical tableau, let $\CC_w$ be the one dimensional 
representation of $B$ corresponding to $w$, and let $V(w) = L_B \ot_B \CC_w$.
We will prove that $V(w)$ is a Gelfand-Tsetlin module.

First notice that $V(w) = \bigoplus_{\overline v} L_B(\overline v) \ot_B 
\CC_w$ where the $\overline v$ run over all the classes of $\ZZ^N / S_m$, and 
that since $L_B(\overline v)$ is finitely generated over $B$ for
each $v \in \ZZ^N_0$, the tensor product $L_B(\overline v) \ot_B \CC_v$ is a 
finite dimensional $\CC$-vector space. Direct inspection shows that 
$L_B(\overline v)$ is a $\Gamma$-module, and hence so is $L_B(\overline v)
\ot_B \CC_w$. Thus $V(w)$ can be written as the direct sum of $\Gamma$-modules 
of finite dimension. This implies that for each vector $a \in V(w)$ the module 
$\Gamma a$ is finite dimensional, and hence $V(w)$ is a GT-module.
\begin{bibdiv}
\begin{biblist}
\bibselect{biblio}
\end{biblist}
\end{bibdiv}

\end{document}

\newpage
\appendix

\section{Appendix: calculation of the discriminant of $S_m$}
This was the original proof that $\phi$ was non-degenerate. It includes
a non-trivial calculation of the discriminant of a reflection group that
can probably be generalized. Given two elements $a, b$ of a $\CC$-algebra $A$, 
we write $a \ceq b$ if there exists $z \in \CC^\times$ suc hthat $a = z b$.

\paragraph
\label{L:equivariant-det}
We will work for some time with matrices with coefficientes in $F$
whose rows and columns are indexed by subsets $S_m$; one may interpret these 
as $F$-linear homomorphisms between subspaces of the group algebra $F[S_m]$, 
though we will not use such interpretation. If $X$ is any such matrix, we 
denote by $X_\sigma$ its $\sigma$-th column and by $X^\tau$ its $\tau$-th row. 
Consequently $X_\sigma^\tau$ is the element in row $\tau$, column $\sigma$.
The action of $S_m$ on $\Lambda_m$ induces an action on the space of matrices.
We will say that $X$ has equivariant rows, resp. colums, if $\nu X^\tau = 
X^{\nu \tau}$, resp. $\nu X_\sigma = X_{\nu \sigma}$, for each $\nu \in S_m$. 
It follows from the definitions that if a matrix is invertible and has 
equivariant rows, then its inverse has equivariant columns.

\begin{Lemma*}
Assume $X$ has equivariant rows. Then $\Delta^{m!/2} \mid \det X$.
\end{Lemma*}
\begin{proof}
We begin by proving a simpler result. Let $L$ be a set of cardinality $2r$ on 
which $\langle (1,2) \rangle$ acts without fixed points, and let $Y$ be a 
square matrix with coefficients in $\Lambda_m$ and rows and columns indexed by 
$Y$. For each $l \in L$ we denote by $Y^l$ the $l$-th row of $Y$. We claim 
that if $(1,2)Y^l = Y^{(1,2)l}$ then $(x_1-x_2)^r \mid \det Y$, and prove it 
by induction on $r$. If $r = 1$ then up to a sign $\det Y$ equals 
\begin{align*}
\left|
\begin{matrix}
f & g \\ f^{(1,2)} & g^{(1,2)}
\end{matrix}
\right| 
&= f g^{(1,2)} - f^{(1,2)} g = fg^{(1,2)} - (fg^{(1,2)})^{(1,2)}
\end{align*}
for some $f,g \in \Lambda_m$. By Lemma \ref{L:symmetric-varia} this is 
divisible by $x_1 - x_2$.

Let $l,l',l'' \in L$, and let $Y'$ be the matrix obtained from $Y$ by erasing
rows $l, (1,2)l$ and columns $l',l''$. In the polynomial $\det Y$ the terms 
$Y^{l}_{l'}Y^{(1,2)l}_{l''}$ and $Y^{l}_{l''}Y^{(1,2)l}_{l'}$ appear 
multiplied by $\det Y'$ but with opposite signs. Now $Y'$ is a matrix of size 
$r-2$ with $\langle (1,2)\rangle$-equivariant rows, so by the inductive 
hypothesis $(x_1 - x_2)^{r/2-1} \mid Y'$, and hence
\begin{align*}
(x_1 - x_2)^{r/2} 
	&\mid (Y^{l}_{l'}Y^{(1,2)l}_{l''} - Y^{l}_{l''}Y^{(1,2)l}_{l'}) \det Y'
		= (Y^{l}_{l'}Y^{(1,2)l}_{l''} - (Y^{l}_{l'}Y^{(1,2)l}_{l''})^{(1,2)}) 
		\det Y'.
\end{align*}
Now $\det Y$ is a sum of terms of this form, with the sum going over all 
pairs $(l', l'')$ with $l' \neq l''$. Thus $\det Y$ is divisible by 
$(x_1 - x_2)^{r/2}$.

Now we go back to our original goal of proving that $\Delta^{r/2} \mid \det X$.
Since $(1,2)$ acts on $S_m$ without fixed points and $X$ has equivariant rows,
it follows that $(x_1 - x_2)^{m!/2} \mid \det X$.
Let $i, j \in \ll m \rr$ with $i < j$. The transposition $(i,j)$ acts on 
$X$ by interchanging $m!/2$ pairs of rows: it sends the $\sigma$-th row
to $(i,j)\sigma$ and the $(i,j)\sigma$-th row to the $\sigma$-th row. Thus 
$(i,j) \det X = \det (i,j) X = (-1)^{m!/2} \det X$. It follows that $\sigma 
\cdot \det X = (\sg \sigma)^{m!/2} \det X$, whence it follows that 
$x_i - x_j$ divides $\det X$ as many times as $x_1 - x_2$, so 
$(x_i - x_j)^{r/2} \mid \det X$. Since $\Lambda_m$ is a unique factorization 
domain and the set $\{x_i - x_j \mid 1 \leq i < j \leq m\}$ consists of 
mutually coprime polynomials, it follows that $\Delta \mid \det X$.
\end{proof}

\paragraph
\label{P:basis-det}
Let $\B = \{f_\sigma \mid \sigma \in S_m\}$ be a basis of $\Lambda_m$. We write
$M(\B)$ for the square matrix such that $M(\B)_\sigma^\tau = f_\sigma^\tau$. 
By definition $M(\B)$ has equivariant rows. Notice that $\det M(B)$ is well
defined up to a sign.
\begin{Proposition*}
Let $\B = \{f_\sigma \mid \sigma \in S_m\}$ be a homogeneous basis of 
$\Lambda_m$ over $\Gamma_m$. Then $\det M(\B) \ceq \Delta^{m!/2}$. 
\end{Proposition*}
\begin{proof}
First, notice that each $\tau \in S_m$
defines an automorphism of $F$, that $E = F^{S_m}$ is the fraction field
of $\Gamma_m$, and that $\B$ is a basis of $F$ as $E$-module. Now $S_m$ is
naturally identified with the Galois group of the extension $F \mid E$. 
Suppose $(p_\tau)_{\tau \in S_m}$ is a vector in the kernel of $M(\B)$, so 
$\sum_\tau p_\tau \tau \cdot (f_\sigma)$. This implies that $\sum_\tau p_\tau 
\tau$ is identically zero, and by Dedekind's Lemma 
\cite{Mor-galois}*{Lemma 2.2} this implies $p_\tau = 0$ for all $\tau$, so
$\det M(\B)$ is not zero. As we observed above, the rows of $M(\B)$ are 
equivariant, and hence $\Delta^{m!/2} \mid \det M(\B)$ by Lemma 
\ref{L:equivariant-det}. Also, its degree is precisely $\sum_\sigma f_\sigma$.
Notice that the sum of the degrees can be calculated as $H'_{\Lambda_m/J}(1)$. 
Now 
\begin{align*}
H'_{\Lambda_m/J}(1)
	&= \sum_{i=1}^m \frac{d}{dt}\left( \frac{1-t^i}{1-t}\right) 
		\mid_{t = 1} 
		\prod_{j \neq i}^m \frac{1-t^j}{1-t} \mid_{t=1} \\
	&= \sum_{i=1}^m (1+2+ \cdots + i-1) \frac{m!}{i}
	= m! \sum_{i=1}^m \frac{i(i-1)}{2i}
	= \frac{m!}{2} \binom{m}{2}= \deg \Delta^{m!/2}. 
\end{align*}
Thus $\det M(\B) \ceq \Delta^{m!/2}$.
\end{proof}

\begin{Proposition}
Let $\phi$ be as above.
\begin{enumerate}[(a)]
\item 
\label{dual-basis}
Every basis of $\Lambda_m$ has a dual basis in $\Lambda_m$.

\item 
\label{quotient-bil-form}
$\phi$ induces a non-degenerate $\CC$-bilinear form on $\Lambda_m/J$
with values in $\CC$.
\end{enumerate}
\end{Proposition}
\begin{proof}
Let $\B = \{f_\sigma \mid \sigma \in S_m\}$ be a basis of $\Gamma_m$. As before
we denote by $M = M(\B)$ the matrix $M^\tau_\sigma = f_\sigma^\tau$, and by $C$
the matrix of cofactors of $M$. Now Cramer's rule says that $\sum_\tau 
M^\tau_\sigma C_\tau^\nu = \Delta^{m!/2} \delta_{\sigma, \nu}$. 

By definition $C_\rho^\nu$ is the determinant of the matrix obtained by
deleting row $\nu$ and column $\rho$ of $M$. Let $1 \leq i < j \leq m$; 
by taking the Laplace expansion of this reduced matrix along the $(i,j)\nu$-th 
row, we see that $C_\rho^\nu$ can be written as a $\Lambda_m$-linear 
combination of determinants of matrices of size $m!-2$ with 
$(i,j)$-equivariant rows, and hence as shown in the proof of Lemma 
\ref{L:equivariant-det} $(x_i-x_j)^{m!/2 -1}$ divides $C_\sigma^\tau$. Since 
$(i,j)$ was arbitrary, we get that $\Delta^{m!/2-1} \mid C_\sigma^\tau$. 

Let $w_0$ be the longest word of $S_m$. For each $\nu, \rho \in S_m$ set 
$g_{\nu,\rho} = \sg(\rho) \frac{C^{w_0\nu^{-1}}_\rho}{\Delta^{m!/2-1}} \in 
\Lambda_m$. With this notation the inverse matrix of $M$ is $(M^{-1})^\rho_\nu 
= \frac{\sg(\rho) g_{w_0\nu^{-1}, \rho}}{\Delta}$. Now since $M$ has 
equivariant rows, $M^{-1}$ has equivariant columns, which implies that 
$g_{\nu, \rho} = g_{\nu,e}^\rho$. Setting $g_\nu = g_{\nu, e}$, the fact that 
$M M^{-1} = \Id$ can be rewritten as
\begin{align*}
\delta_{\sigma, \nu}
	&= \sum_{\tau \in S_m} f_\sigma^\tau \frac{C_\tau^\nu}{\Delta^{m!/2}} 
	= \frac{1}{\Delta}\sum_{\tau \in S_m} 
		\sg(\tau)f_\sigma^\tau g_{w_0\nu^{-1}}^\tau
	= \phi(f_\sigma g_{w_0\nu^{-1}}),
\end{align*}
so $\phi(f_\sigma, g_\nu) = 1$ if and only if $\nu\sigma = w_0$, and zero 
otherwise. Thus $\B' = \{g_\nu \mid \nu \in S_m\}$ is a dual set to $\B$. 

Notice that $g_\nu$ is by definition a homogeneous polynomial, and that the 
formula above implies that $\deg f_{\nu^{-1} w_0} + \deg g_{\nu} = 
\binom{m}{2} = \ell(w_0)$, where $w_0$ is the longest word of $S_m$, so 
$\deg g_{\nu} = \ell(w_0) - \ell(\nu^{-1} w_0) = \ell(\nu)$. 

Finally, by definition $\phi$ induces a bilinear form $\overline \phi: 
\Lambda_m/J \ot \Lambda_m/J \to \CC$, and the projections of the sets $\B$
and $\B'$ on this quotient are dual bases; since dual bases exist, the form
is non-degenerate.
\end{proof}
If $\B = \{f_\sigma \mid \sigma \in S_m \}$ is basis, we denote by $\B^* = 
\{f_\tau^* \mid \tau \in S_m\}$ its dual basis. Thus $\phi(f_\sigma, f^*_\tau)
= \delta_{\tau\sigma, w_0}$.
\end{document}
