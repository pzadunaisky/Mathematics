\documentclass[smaller,usepdftitle=false]{beamer}
\usepackage[latin1,utf8]{inputenc}
\usepackage[T1]{fontenc}
\usepackage{tikz}
\usetikzlibrary{calc}

\usepackage{mathdots}

\linespread{1.2}

\usetikzlibrary{topaths}


\mode<presentation>
\usetheme{Warsaw}
\setbeamertemplate{navigation symbols}{}
\setbeamertemplate{headline}{}
\setbeamertemplate{footline}{}
\usepackage[matrix,arrow,graph,curve]{xy}
\usepackage{amssymb}
\usepackage{leftidx}
\usepackage{graphics}
\usepackage{cancel}
\usepackage{xcolor}

\newcommand\NN{\mathbb N}
\newcommand\CC{\mathbb C}
\newcommand\QQ{\mathbb Q}
\newcommand\RR{\mathbb R}
\newcommand\ZZ{\mathbb Z}

\newcommand\maps{\longmapsto}
\newcommand\ot{\otimes}
\renewcommand\to{\longrightarrow}
\renewcommand\phi{\varphi}
\newcommand\stack[2]{\genfrac{}{}{0pt}{2}{#1}{#2}}
\newcommand\vectspan[1]{\left\langle #1 \right\rangle}

\newcommand\gl{\mathfrak{gl}}

\DeclareMathOperator\sym{\mathsf{sym}}
\DeclareMathOperator\st{\mathsf{st}}
\newtheorem*{Proposition}{Proposition}


\def\pausa{\pause \bigskip}


\title[]{Gelfand-Tsetlin modules}
\author[]{Pablo Zadunaisky\footnote{FAPESP Postdoc 
grant 2016/25984-1} - USP, S\~ao Paulo \\
(Joint work with L.E. Ramírez)}
\date{Natal, october  2017}

\begin{document}
% Remove this to get the pauses...
%\let\pause\relax

\begin{frame}
\titlepage
\end{frame}
\begin{frame}
\frametitle{Gelfand-Tsetlin tableaux}
A GT-tableau of height $n$ is an array $T(v)$ of $\frac{n(n+1)}{2}$ complex 
numbers
\bigskip

\begin{tikzpicture}
\node (n1) at (-3,3) {$v_{n,1}$};
\node (n2) at (-1.5,3) {$v_{n,2}$};
\node (ndots) at (0,3) {$\cdots$};
\node (nn-1) at (1.5,3) {$v_{n,n-1}$};
\node (nn) at (3,3) {$v_{n,n}$};

\node (n-11) at (-2.25,2.25) {$v_{n-1,1}$};
\node (n-1dots) at (0,2.25) {$\cdots$};
\node (n-1n-1) at (2.25,2.25) {$v_{n-1,n-1}$};

\node (dots1) at (-1.5,1.5) {$\ddots$};
\node (dots2) at (0,1.5) {$\cdots$};
\node (dots3) at (1.5,1.5) {$\iddots$};

\node (21) at (-.75,.75) {$v_{2,1}$};
\node (22) at (.75,.75) {$v_{2,2}$};
\node (11) at (0,0) {$v_{1,1}$};

\node (A) at (-4.2, 2.75) {};
\node (B) at (4.2, 2.75) {};
\node (C) at (0,0) {};

\draw[->] (n1) -- (n-11);
\draw[->] (n-11) -- (n2);
\draw[->] (n-11) -- (-1.65,1.7) node{$\phantom{v_{n,n}}$};
\draw[->] (n2) -- (-0.75,2.5);
\draw[->] (nn-1) -- (n-1n-1);
\draw[->] (n-1n-1) -- (nn);
\draw[->] (21) -- (11);
\draw[->] (11) -- (22);
\end{tikzpicture}

Tableaux are parametrized by $v \in \CC^{\frac{n(n+1)}{2}}$. 

We identify $\ZZ^{\frac{n(n-1)}{2}}$ with integral tableaux with top 
row $0$.
\end{frame}

\begin{frame}
\frametitle{Generic Gelfand-Tsetlin modules}

Let $v \in \CC^{\frac{n(n+1)}{2}}$ and set 
\begin{align*}
V(T(v))
	&= \langle T(v+z) \mid z \in \ZZ^{\frac{n(n-1)}{2}} \rangle_\CC
\end{align*}

\begin{Theorem}[Drozd, Futorny and Ovsienko, 88-92]
If $v$ is such that $v_{k,i} - v_{k,j} \notin \ZZ, 1 \leq k < n$,
 then $V(T(v))$ can be endowed with the structure of a 
$\gl(n,\CC)$-module.
\end{Theorem}

\vspace{-.7cm}

\pause
\begin{align*}
E_{k,k+1} T(w) &= - \sum_{i=1}^k \theta^+_{k,i}(w) 
  T(w+\delta^{k,i}); \\
E_{k+1,k} T(w) &= \sum_{i=1}^k \theta^-_{k,i}(w) 
  T(w -\delta^{k,i}); \\
E_{k,k} T(w) &= \theta_k(w) T(w).
\end{align*}
The functions $\theta$ are rational functions.
\end{frame}

\begin{frame}
\frametitle{Gelfand-Tsetlin subalgebra}

\begin{align*}
\gl(1,\CC) 
	\subset \gl(2,\CC) &\subset \cdots \subset \gl(n,\CC) \\
U(\gl(1,\CC)) 
	\subset U(\gl(2,\CC)) &\subset \cdots \subset U(\gl(n,\CC)) \\
\Gamma 
	= \mbox{ Algebra generated by } &\bigcup_k Z(U(\gl(k,\CC))) 
\subset U(\gl(n,\CC))
\end{align*}

By works of Harish-Chandra + Zhelobenko
\begin{align*}
\Gamma 
	&\cong \CC[\lambda_{k,i} \mid 1 \leq i \leq k 
		\leq n]^{S_1 \times \cdots S_n} \\
c &\longmapsto \gamma_c
\end{align*}

\pause

\begin{Theorem}[DFO, continued]
If $v$ is generic, the obvious basis of $V(T(v))$ is a $\Gamma$-eigenbasis, 
with $$c T(v+z) = \gamma_c(v+z) T(v+z).$$
\end{Theorem}
\end{frame}

\begin{frame}
\frametitle{Singular Gelfand-Tsetlin modules}
A point $v \in \CC^{\frac{n(n+1)}{2}}$ is \emph{critical} if $v_{k,i} - 
v_{k,j} \in \ZZ$ implies $v_{k,i} = v_{k,j}$.

\begin{Theorem}[Ramírez, Z. -- 2017]
Let $v \in \CC^{\frac{n(n+1)}{2}}$ be critical. Then there exists a 
$\gl(n,\CC)$-module structure over $V(T(v))$ such that $(c - \gamma_c(v+z))^n 
T(v+z) = 0$ for $n \gg 0$.
\end{Theorem}

\textbf{Observation:} We have 
\begin{align*}
V(T(v))[v+z] 
	&= \{x \in V(T(v)) \mid (c - \gamma_{c}(v+z))^n x = n \gg 0\} \\
	&=\{T(v+z') \mid \sigma(v+z) = \sigma(v+z') 
		\mbox{ for some } \sigma \in G\} \\
	& = \{T(v+\sigma(z)) \mid \sigma \in G_v \subset G\} 
\end{align*}

\end{frame}

\begin{frame}
\frametitle{Examples of character spaces}
Set $v \in \CC^{\frac{n(n+1)}{2}}, z \in \ZZ^{\frac{n(n-1)}{2}}$, with $v_3 = 
(0,0,0)$ and $z_3 = (a,b,c)$ with $a > b > c$.
\begin{align*}
\xymatrix{
& T(v+(c,b,a)) & \\
	T(v+(b,c,a)) \ar@{-}[ur]& & T(v+(c,a,b)) \ar@{-}[ul]\\
	T(v+(b,a,c)) \ar@{-}[u] \ar@{-}[urr] &  & T(v+(a,c,b)) \ar@{-}[u] 
		\ar@{-}[ull]\\
	& T(v+(a,b,c)) \ar@{-}[ur] \ar@{-}[ul] & 
}
\end{align*}
\end{frame}

\begin{frame}
\frametitle{Examples of character spaces}
Set $v \in \CC^{\frac{n(n+1)}{2}}, z \in \ZZ^{\frac{n(n-1)}{2}}$, with $v_3 = 
(0,0,0)$ and $z_3 = (a,a,b)$ with $a > b$.
\begin{align*}
\xymatrix{
	T(v+(b,a,a)) \\
	T(v+(a,b,a)) \ar@{-}[u] \\
	T(v+(a,a,b)) \ar@{-}[u]
}
\end{align*}
\end{frame}

\begin{frame}
\frametitle{Examples of character spaces}
Set $v \in \CC^{\frac{n(n+1)}{2}}, z \in \ZZ^{\frac{n(n-1)}{2}}$, with $v_4 = 
(0,0,0,0)$ and $z_4 = (a,a,b,b)$ with $a > b$.
\begin{align*}
\xymatrix{
& T(z+(b,b,a,a)) & \\
& T(z+(b,a,b,a)) \ar@{-}[u]& \\
T(z+(a,b,b,a)) \ar@{-}[ur] & & T(z+(b,a,a,b)) \ar@{-}[ul] \\
& T(z+(a,b,a,b)) \ar@{-}[ul] \ar@{-}[ur]& \\
& T(z+(a,a,b,b)) \ar@{-}[u]& 
}
\end{align*}
\end{frame}

\end{document}


\begin{frame}
\end{frame}

\begin{frame}
\frametitle{GT-modules}
\begin{Definition}
A \emph{GT-module} is a $U(\gl(n,\CC))$-module $V$ decomposable as $V = 
\bigoplus_\chi V_\chi$, with $V_\chi$ the set of all vectors of
generalized $\Gamma$-eigenvalue $\chi$.
\end{Definition}

\textbf{Problem:} Construct all irreducible objects of the category of 
GT-modules.
\pause

\textbf{Conjecture:} For each $v$ there is a GT-module structure on
\begin{align*}
V(T(v)) 
  = \CC \mbox{-span} \langle T(v + z) \mid z \in \ZZ^{\binom{n}{2}} \rangle
\end{align*}
such that
\begin{itemize}
\item $V(T(v))_w \neq 0$ iff $w = v+z$ for some $z \in \ZZ^{\binom{n}{2}}$.

\item If $S$ is an irreducible GT-module with $S_{v+z} \neq 0$ then $S$ is a 
subquotient of $V(T(v))$.
\end{itemize}
\end{frame}

\begin{frame}
\frametitle{A generic construction}
\begin{align*}
F &= \CC(\lambda_{k,i} \mid 1 \leq i \leq k \leq n) &
V_\CC &= \CC\langle T(z) \mid z \in \ZZ^{\binom{n}{2}}\rangle
\end{align*}
$G = S_n \times \cdots \times S_2 \times S_1$ acts on both, and hence on $V_F 
= F \ot_\CC V_\CC$

\pausa

\begin{Theorem}
The space $V_F$ is a $U = U(\gl(n,\CC))$-module with action
\vspace{-.3cm}
\begin{align*}
E_{k,k+1} T(z) &= - \sum_{i=1}^k \theta^+_{k,i}(\lambda + z) 
  T(z+\delta^{k,i}); \\
E_{k+1,k} T(z) &= \sum_{i=1}^k \theta^-_{k,i}(\lambda + z) 
  T(z -\delta^{k,i}); \\
E_{k,k} T(z) &= \theta_k(\lambda + z) T(z).
\end{align*}
Also $c T(z) = \sigma(\lambda + z) T(z)$. The action of $U$ is $G$-equivariant.
\end{Theorem}
\end{frame}

\begin{frame}
\frametitle{$V(T(v))$ for generic tableaux}
\begin{align*}
A &= S^{-1} \CC[\lambda_{k,i}] &
S 
  &= \{\lambda_{k,i} - \lambda_{k,j} - a \mid a \in \ZZ, 
    1 \leq i < j \leq k < n\}
\end{align*}
(regular functions on the space of all generic tableaux).
\pausa 

$v$ generic tableaux $\longrightarrow$ $1$-dim $A$-module $\CC_v$.

\begin{Proposition}
$L_A = A \ot V_\CC \subset V_F$ is a $U$-submodule, and $V(T(v)) = \CC_v 
\ot_A L_A$ satisfies the conjecture. 
\end{Proposition}
In this case $\dim V(T(v))_{v+z} = 1$, so it has a basis parametrized by 
tableaux and the action is given by the GT-formulas.
\end{frame}

\begin{frame}
\frametitle{A candidate for $V(T(v))$ for singular characters}
Write $\st(v)$ for the stabilizer of $v$ inside $G$.

\textbf{Normalization of $v$}:
$v' = g(v+z)$ is a normalization of $v$ if 
\begin{align*}
\st(v') \cap S_k &= S_{\{1, \ldots, r_1\}} \times S_{\{r_1 + 1, \ldots, r_2\}}
  \times \cdots \times S_{\{r_{t-1}+1, \ldots, k\}}
\end{align*}
is as large as possible. 
\pausa

Set
\begin{align*}
\eta &= \eta(v) =\{w = w' \mid \st(w') < \st(v')\} \\
B_\eta &= \mbox{ Regular functions over $\eta$} \\
\CC_{v'} &= \mbox{$1$-dimensional $B_\eta$-module induced by $v'$}
\end{align*}
\end{frame}

\begin{frame}
\frametitle{A candidate for $V(T(v))$ for singular characters}
\begin{Theorem}[Ramírez, Z' - 17]
There exists $L_\eta \subset V_F$ which is a $B_\eta$-module and a 
$U$-submodule. The $U$-module $V(T(v')) = \CC_{v'} \ot_{B_\eta} L_\eta$ is a 
GT-module.
\end{Theorem}

We have an explicit basis of $W(T(v'))_{v+z}$ formed by "derived tableaux":
\begin{align*}
\left\{D_\sigma(v'+z) \mid \sigma \in \frac{\st(v')}{\st(v') 
\cap \st(v'+z)} \right\}
\end{align*}
In particular $\dim_\CC V(T(v'))_{v'+z} = \# \frac{\st(v')}{\st(v') 
\cap \st(v'+z)}$. We think the conjecture holds for $V(T(v'))$, but have no 
proof yet (holds in special cases by previous work of Futorny, Grantcharov and 
Ramírez).
\end{frame}


\begin{frame}
\frametitle{Example: a $3$-singular module of $\gl(4,\CC)$.}

\begin{tikzpicture}
\node (v) at (-3,0.75) {$v=$};

\node (n1) at (-2,1.5) {$\lambda_1$};
\node (n2) at (-0.75,1.5) {$\lambda_2 + 1$};
\node (ndots) at (0.75,1.5) {$\lambda_3 + 2$};
\node (nn-1) at (2,1.5) {$\lambda_4 + 3$};

\node (n-11) at (-1.5,1) {$0$};
\node (n-1dots) at (0,1) {$0$};
\node (n-1n-1) at (1.5,1) {$0$};

\node (dots1) at (-0.75,0.5) {$\alpha$};
\node (dots2) at (0.75,0.5) {$\beta$};

\node (dots3) at (0,0) {$\gamma$};

\node (blabal) at (4, 0.75) {$(\alpha-\beta \notin \ZZ)$};
\end{tikzpicture}
\bigskip 

\begin{tabular}{|c|c|l|}
\hline
$z_3$ & Stabilizer & Nonzero derived tableaux \\
\hline
$(a,b,c)$
  & $(1,1,1)$
  & \parbox[c]{7cm}
    {$D_e(v+z), D_{(12)}(v+z), D_{(23)}(v+z),$\\ $D_{(123)}(v+z), 
    D_{(132)}(v+z), D_{(13)}(v+z)$} \\
\hline
$(a,a,b)$
  & $(2,1)$
  & $D_e(v+z), D_{(23)}(v+z), D_{(123)}(v+z)$ \\
\hline
$(a,b,b)$
  & $(1,2)$ 
  & $D_e(v+z), D_{(12)}(v+z), D_{(132)}(v+z)$ \\
\hline
$(a,a,a)$
  & $(3)$
  & $D_e(v+z)$ \\
\hline
\end{tabular}

where $a>b>c$.
\end{frame}

\begin{frame}
\frametitle{Example: the $3$-singular module of $\gl(4,\CC)$.}
\begin{align*}
E_{4,3} &D_{(13)}(v+z)
  = \frac{1}{2} \frac{\partial p_{3,1}^-}{\partial \lambda_{1,3}}(v+z)
    D_{(123)}(v+z-\delta^{3,1}) + \frac{1}{2} D_{(12)}(v+z-\delta^{3,1})\\
    &+ \frac{1}{4} D_{e} (v+z-\delta^{3,1}) -2 \left( 3 p^-_{2,3}(v+z) - 
    1\right) D_{(13)}(v+z-\delta^{3,2})\\
    &+ D_{(12)}(v+z-\delta^{3,2}) - 
    \frac{\partial p_{3,2}^-}{\partial\lambda_{3,2}}(v+z) D_e(v+z-
    \delta^{3,2})\\
    &+ \frac{1}{2}p_{3,3}^-(v+z) D_{(13)}(v+z-\delta^{3,3}) 
    + \frac{1}{4}p_{3,3}^-(v+z) D_{(123)}(v+z-\delta^{3,3})  \\
    &+ \frac{1}{2}\frac{\partial p_{3,3}^-}{\partial \lambda_{3,3}}(v+z) 
      D_{(132)}(v+z-\delta^{3,3}) + \frac{1}{4}
      \frac{\partial p_{3,3}^-}{\partial \lambda_{3,3}}(v+z) 
      D_{(23)}(v+z-\delta^{3,3}) \\
    &+ \frac{1}{2}D_{(12)}(v+z-\delta^{3,3}) + \frac{1}{4}
    D_{e}(v+z-\delta^{3,3})
\end{align*}
\end{frame}


\end{document}




