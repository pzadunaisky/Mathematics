%%%%%%%%%%%%%%%%%%%%%% Generalities %%%%%%%%%%%%%%%%%%5
\documentclass[11pt,fleqn]{article}
\usepackage[paper=a4paper]
  {geometry}

\pagestyle{plain}
\pagenumbering{arabic}
%%%%%%%%%%%%%%%%%%%%%%%%%%%%%%%%
\usepackage{notas}
\usepackage{tikz}
\usepackage{mathdots}

%%%%%%%%%%%%%%%%%%%%%%%%%%% The usual stuff%%%%%%%%%%%%%%%%%%%%%%%%%
\newcommand\NN{\mathbb N}
\newcommand\CC{\mathbb C}
\newcommand\QQ{\mathbb Q}
\newcommand\RR{\mathbb R}
\newcommand\ZZ{\mathbb Z}
\renewcommand\k{\Bbbk}

\newcommand\B{\mathcal B}
\newcommand\F{\mathcal F}
\newcommand\V{\mathcal V}
\newcommand\D{\mathcal D}
\renewcommand\H{\mathcal H}
\renewcommand\O{\mathcal O}
\newcommand\DD{\mathfrak D}

\newcommand\maps{\longmapsto}
\newcommand\ot{\otimes}
\renewcommand\to{\longrightarrow}
\renewcommand\phi{\varphi}
\newcommand\Id{\mathsf{Id}}
\newcommand\im{\mathsf{im}}
\newcommand\coker{\mathsf{coker}}
%%%%%%%%%%%%%%%%%%%%%%%%% Specific notation %%%%%%%%%%%%%%%%%%%%%%%%%
\newcommand\g{\mathfrak g}
\newcommand\p{\mathfrak p}
\newcommand\m{\mathfrak m}
\newcommand\gl{\mathfrak{gl}}
\newcommand\gen{\mathsf{gen}}
\newcommand\std{\mathsf{std}}
\newcommand\sh{\mathsf{sh}}
\newcommand\OTheta{\overline \Theta}

\newcommand\vectspan[1]{\left\langle #1 \right\rangle}
\newcommand\interval[1]{\llbracket #1 \rrbracket}
\newcommand\Shuffle{\mathsf{Shuffle}}

\DeclareMathOperator\Frac{Frac}
\DeclareMathOperator\Specm{Specm}

\DeclareMathOperator\sym{sym}
\DeclareMathOperator\asym{asym}
\DeclareMathOperator\sg{sg}
\DeclareMathOperator\st{\mathsf{st}}

\newcommand\bigmodule{big GT module}

%%%%%%%%%%%%%%%%%%%%%%%%%%%%%%%%%%%%%% TITLES %%%%%%%%%%%%%%%%%%%%%%%%%%%%%%
\title{$\Gamma$-module structure of $V(T(v))$}
%\author{[gamma-structure.tex]}
\date{}

\begin{document}
\maketitle
%\vspace{-2cm}

This document is a continuation of \cite{RZ-singular-characters}, we keep the 
same notation as in that paper. We also use the same notation as in 
\cite{PS-chains-bruhat}.

\section{}
Fix $n \in \NN$, let $N = \frac{n(n+1)}{2}$ and let $\mu = (1,2, \ldots, n)$.
If $\eta, \theta$ are compositions of $N$ then $\theta < \eta$ means that 
$\theta$ is a refinement of $\eta$, i.e. $\theta = (\theta^{(1)}, \ldots,
\theta^{(t)})$ with $\theta^{(i)}$ a composition of $\eta_i$; by abuse of 
notation we also denote by $\theta$ the concatenation of the compositions 
$\theta^{(i)}$. We fix $\eta = (\eta_1, \ldots, \eta_r)$ with $\eta < \mu$.

\paragraph
We write $\CC[X_\eta] = \CC[x_{k,i} \mid (k,i) \in \Sigma(\eta)]$. For each $1 
\leq k \leq r$ and each $1 \leq i <j \leq \eta_k$ set $u^k_{i,j} = x_{k,i} - 
x_{k,j}$. We denote by $D_k = \{u^k_{i,j \mid 1 \leq i < j \leq \eta_k}\}$ and
$D = \cup_k D_k$. Set $\p_\eta$ to be the prime ideal generated by $D$, and 
$\Delta_\eta = \prod_{D} u^k_{i,j}$. Thus $\CC[X_\eta]/\p_\eta \cong \CC[Y] 
= \CC[Y_1, \ldots, Y_r]$, where $Y_k = x_{k,i} + \p_\eta$ for any $(k,i) \in 
\Sigma(\eta)$. We denote by $\pi_\eta: \CC[X_\eta] \to \CC[Y]$ the projection
to the quotient.

Now set
\begin{align*}
C_\eta = \{x_{k,i} - x_{k,j} - z \mid (k,i),(k,j) 
	\in \Sigma(\eta), z \in \ZZ \setminus \{0\}\}.
\end{align*}
and $B_\eta = C_\eta^{-1} \CC[X_\eta]\left[\frac{\Delta_\eta}{\Delta_\mu}
\right]$. Notice that $\p_\eta B_\eta$ is also a prime ideal. 
We write $I^\mu_\eta$ for the set of all pairs $(k,k')$ such
that $\eta_k$ and $\eta_{k'}$ are parts of one composition $\eta^{(i)}$.
With this notation $B_\eta / \p_\eta B_\eta \cong \CC[Y][(Y_k - Y_{k'})^{-1} 
\mid (k,k') \in I^\mu_\eta]$. By a slight abuse of notation we also write 
$\pi: B_\eta \to B_\eta /\p_\eta B_\eta$ for the projection.

\paragraph
For each $\sigma \in S_\eta$ set $s_\sigma = \partial_{\sigma^{-1} w_\eta} 
\Delta_\eta$. By definition $\deg s_\sigma = \ell(\sigma)$. The algebra 
$\CC[D]$ is closed under the action of divided differences so $s_\sigma \in 
\CC[D]$. Furthermore the set $\{s_\sigma \mid \sigma \in S_\eta\}$ is a basis 
of $\CC[X_\eta]$ as $\CC[X_\eta]^{S_\eta}$-module, a basis of $B_\eta$ as 
$B_\eta^{S_\eta}$-module, and a basis of $\CC(X_\eta)$ as 
$\CC(X_\eta)^{S_\eta}$ vector space. Given $f \in \CC(X_\eta)$ we denote by 
$f_{(\sigma)} \in \CC(X_\eta)^{S_{\eta}}$ the coordinate of $f$ in this basis 
corresponding to $s_\sigma$, so $f = \sum_{\sigma \in S_\eta} f_{(\sigma)} 
s_\sigma$. 

We set $c_{\tau, \sigma}^\nu = (s_{w_\eta \tau} s_{w_\eta \sigma})_{(w_\eta 
\nu)}$ and define the operator $D_{\sigma, \nu}^\eta: K \to K$
\[
D_{\sigma, \nu}^\eta(f) = 
		\sum_{\tau \in S_\eta} c_{\tau, \sigma}^\nu f_{(w_\eta \tau)}
\]

\begin{Lemma*}
For each $\sigma \in S_\eta$, each $z \in \ZZ^\mu_0$ and each $f \in K$ we have
\begin{align*}
D_\sigma^\eta(f T(z))
	&= \sum_{\nu \in S_\eta} (f s_{w_\eta \sigma})_{(w_\eta \nu)} 
		D_\nu^\eta(T(z)) = \sum_{\nu \in S_\eta} 
			D^\eta_{\sigma, \nu}(f) D_\nu^\eta(T(z)).
\end{align*}
\end{Lemma*}
\begin{proof}
We know that $D_\sigma^\eta = D_{w_\eta}^\eta \cdot s_{w_\eta \sigma }$, so
\begin{align*}
D_\sigma^\eta(f T(z))
	&= D_{w_\eta}^\eta((f s_{w_\eta \sigma}) T(z))\\
	&= \sum_{\nu} (f s_{w_\eta \sigma})_{(w_\eta \nu)}
		D_{w_\eta}^\eta (s_{w_\eta \nu} T(z))\\
	&= \sum_{\nu}(f s_{w_\eta \sigma})_{(w_\eta \nu)}
		D_{\nu}^\eta (T(z)).
\end{align*}
Now $f = \sum_\tau f_{(w_\eta \tau)} s_{w_\eta \tau}$ so
\begin{align*}
(f s_{w_\eta \sigma})_{(w_\eta \nu)}
	&= \sum_\tau f_{(w_\eta \tau)} (s_{w_\eta \tau} 
		s_{w_\eta \sigma})_{(w_\eta \nu)} 
	= D_{\sigma, \nu}^\eta(f).
\end{align*}
\end{proof}

\paragraph
According to \cite{RZ-singular-characters}*{4.4} for each $\sigma \in S_\eta$
there exists a polynomial $s_{\sigma}^*$ such that $\sym_{\eta} (s_\sigma 
s_\tau^*) = \delta_{\sigma, \tau}$. Furthermore, $s_\sigma^* = \frac{1}{\eta!}
s_{w_\eta \sigma} + c_\sigma$, where $c_\sigma \in \CC[D]^{S_\eta}_{\geq 1}$. 
Thus
\[
f_{(w_\eta \sigma)} 
	= \sym(f s_{w_\eta \sigma}^*)
	\equiv \frac{1}{\eta!} \sym (f s_\sigma) \mod \CC[D]^{S_\eta}_{\geq 1}.
\]


\paragraph
We denote by $\Theta: \CC[X_\eta] \to \mathsf{End}_\CC(B_\eta)$ the unique 
algebra map induced by $\Theta(x_{k,i}) = \frac{\partial}{\partial x_{k,i}}$.
Also we denote by $\OTheta_{(\sigma)}: B_\eta \to A_\eta$ the operator
$\frac{1}{\eta!}\pi \circ \Theta(\DD_\sigma)$, where $\DD_\sigma$ is defined 
as in \cite{PS-chains-bruhat}*{p. 139, (4.1)}.

\begin{Lemma*}
For each $f \in B_\eta$ and each $\sigma \in S_\eta$ we have $\pi(f_{(\sigma)})
= \OTheta(f)$.
\end{Lemma*}
\begin{proof}

\end{proof}

\begin{bibdiv}
\begin{biblist}
\bibselect{biblio}
\end{biblist}
\end{bibdiv}

\end{document}

============================================================================


We define two bilinear forms $(-,-)_{PS}, \vectspan{-,-}: \CC[X_\eta] \times 
\CC[X_\eta] \to \CC[Y]$, where for each $f,g \in \CC[X_\eta]$
\begin{align*}
(f,g)_{PS} &= \pi(\Theta(f)(g)); 
& \vectspan{f,g} &= \pi \circ \sym_\eta\left( \frac{fg}{\Delta}\right).
\end{align*}

\begin{Lemma*}
Let $f,g \in \CC[X_\eta]$. 
\begin{enumerate}[(a)]
\item $(1, f)_{PS} = \vectspan{\Delta_\eta, f} = \pi(f) = \pi(\sym_\eta(f))$.

\item For each $\sigma \in S_\eta$ we have $(f,\partial_{\sigma} g) =
(I_{\sigma^{-1}}(f), g)$, where $I_\sigma$ is the operator defined in
\cite{PS-chains-bruhat}*{(6.2) p. 144}.

\item For any $\sigma \in S_\eta$ we have $D_\sigma^\eta(f) = 
\pi (\Theta(\DD_{\sigma}) (f))$, where $\DD_\sigma$ is as in 
\cite{PS-chains-bruhat}*{p. 139 (4.1)}.
\end{enumerate}
\end{Lemma*}
\begin{proof}
Item $(a)$ follows from the definitions. For item $(b)$ we follow the proof
of \cite{PS-chains-bruhat}. Let $V = \vectspan{x_{k,i} \mid (k,i) \in 
\Sigma(\eta)}_\CC$. The restriction of $(-,-)_{PS}$ to $V$ is a nondegenerate
bilinear form with values in $\CC$. Now fix $(k,i) \in \Sigma(\eta)$ and let
$s = s^(k)_i, u = x_{k,i} - x_{k,j}, v = \sum_{j=1}^{\mu_k} x_{k,i}$. Then
$(u,v)_{SP} = 0$ and we can fix a basis $w_3, \ldots, w_N$ of the SP-orthogonal
to the vector space generated by $u$ and $v$. Notice that if $x \in V$ then 
$(v,x) = 0$ implies $s \cdot x = x$. Let $\{u^*, v^*, w_i^* \mid 3 \leq i \leq
N\}$ be the basis of $V^*$ dual to $\{u,v,w_i \mid 3 \leq i \leq N\}$, and for
each $f \in \CC[X_\eta]$ let $f(a_1, \ldots, a_N) = f(a_1 u^* + a_2 v^* + 
a_3 w_3^* + \cdots + a_N w_N^*)$. Then 
\begin{align*}
\partial_s (f)(a_1, \ldots, a_N)
	&= \frac{f(a_1, a_2, \ldots, a_N) - f(-a_1, a_2, \ldots, a_N)}{a_1}; \\
I_s (f)(a_1, \ldots, a_N)
	&= \int_{-a_1}^{a_1} f(t, a_2, \ldots, a_N).
\end{align*}
These operators are linear with respect to all variables except $u$, so it
is enough to check that $(u^c, \partial_s u^d)_{SP} = (I_s(u^c), u^d)_{SP}$ for
$c,d \in \NN$, which is immediate. Thus the result holds for $\sigma = s$, and
follows by induction on length.
Now by definition
\begin{align*}
\pi \circ D_\sigma^\eta(f) 
	&= \pi \circ \sym_\eta (\partial_\sigma f) 
	= (1, \partial_\sigma f)_{PS}
	= (I_{\sigma^{-1}}(1), f)_{PS}
	= \Theta(\DD_\sigma)(f). 
\end{align*}
\end{proof}

\paragraph
Recall that $L_\eta$ is the free $B_\eta$ submodule of $V_K$ with basis
\begin{align*}
\B = \{D_\sigma^\eta T(z) \mid 
	z \in \mathcal N(\eta), 
	\sigma \in \mathsf{Shuff}_{\epsilon(z)}^\eta\}.
\end{align*}
The main result from \cite{RZ-singular-characters} is that this is a 
$U$-submodule of $V_K$. Set $A_\eta = B_\eta / \p_\eta B_\eta$ and let 
$V(\eta) = L_\eta/\p_\eta L_\eta$. If $v \in \CC^\eta$ is fully 
$\eta$-critical then $ev_v: B_\eta \to \CC$ contains the ideal $\p_\eta 
B_\eta$, so 
\[
V(T(v)) = \CC_v \ot_{B_\eta} L_\eta = \CC_v \ot_{A_\eta} V(\eta).
\]
We introduce the intermediate module $V(\eta)$ because the coefficients of the 
action of $U$ on $V(\eta)$ lie in $A_\eta$, i.e. a finite localization of a
polynomial algebra, so they are simpler. Furthermore these coefficients reflect
the structure of $V(T(v))$ for $v$ ''generic'' among $\eta$-critical points, 
i.e. without integral differences between entries in consecutive rows.

If we write $\D^\eta_\sigma(z)$ for the 
image of $D_\sigma^\eta T(z)$ then $V(\eta)$ is a free $A_\eta$-module with
basis
\begin{align*}
\B = \{\D_\sigma^\eta T(z) \mid 
	z \in \mathcal N(\eta), 
	\sigma \in \mathsf{Shuff}_{\epsilon(z)}^\eta\}.
\end{align*}

For each $\sigma \in S_\eta$ we denote by $(\partial_\sigma \Delta_\eta)^*$
the unique polynomial such that $\langle (\partial_\sigma \Delta_\eta)^*, 
\partial_\sigma \Delta_\eta \rangle = 1$, which equals $\frac{1}{\eta!}
\partial_{\sigma^{-1} w_\eta} \Delta_\eta + c_{\sigma^{-1} w_\eta}$ with
$c_\sigma$ in the ideal generated by $\p^{S_\eta}$. It follows that
\begin{align*}
D_{w_\eta}((\partial_\sigma \Delta_\eta)^* T(z)) 
&= \sym \left( 
	\frac{(\partial_\sigma \Delta_\eta)^*}{\Delta_\eta } T(z) 
	\right) \\
&\equiv \frac{1}{\eta!}
	\sym \left( 
		\frac{\partial_{\sigma^{-1} w_\eta} \Delta_\eta}{\Delta_\eta } 
			T(z) 
	\right) = \frac{1}{\eta!} D_{w_\eta \sigma}^\eta T(z)
			\mod \p_\eta L_\eta. 
\end{align*}

\begin{Proposition}
Let $f \in K$. Then 
\begin{align*}
D_\sigma(f T(z))
	\equiv \sum_{\tau \in S_\eta} 
		\frac{1}{\eta!}
			D_{w_\eta\tau}^\eta(f \partial_{\sigma^{-1}} \Delta_\eta) 
		D^\eta_{\tau}(T(z)) \mod \p_\eta L_\eta.
\end{align*}
Furthermore, if $f \in \CC[X_\eta]$ then the coefficient of 
$D_\sigma^\eta(T(z))$ is $\sym_\eta(f)$, and for any other $\tau$
with $\ell(\tau) \geq \ell(\sigma)$ the coefficient of $D_\tau^\eta(T(z))$
is zero.
\end{Proposition}
Notice that we have taken $f \in K$. This congruence makes sense by considering
$\p_\eta L_\eta \subset V_K$ as a $B_\eta$-submodule.
\begin{proof}
We have
\begin{align*}
f \partial_{\sigma^{-1}} \Delta_\eta 
	= \frac{1}{\eta!}
	\sum_{\nu \in S_\eta} D^\eta_\nu(f \partial_{\sigma^{-1}} \Delta_\eta ) 
		(\partial_\nu \Delta_\eta)^*,
\end{align*}
so 
\begin{align*}
D_\sigma(f T(z))
	&= D_{w_\eta}^\eta (f \partial_{\sigma^{-1}}\Delta_\eta T(z)) \\
	&= \sum_{\nu \in S_\eta} D^\eta_\nu(f \partial_{\sigma^{-1}} \Delta_\eta) 
		D_{w_\eta}^\eta((\partial_\nu \Delta_\eta)^* T(z))\\
	&\equiv \frac{1}{\eta!}\sum_{\nu \in S_\eta} 
		D_{\nu}^\eta(f \partial_{\sigma^{-1}} \Delta_\eta) 
		D^\eta_{w_\eta\nu}(T(z)) \mod \p_\eta L_\eta.
\end{align*}
The formula follows by taking $\tau = w_\eta \nu$. The case $f \in \CC[X_\eta]$
is proved in \cite{RZ-singular-characters}*{5.6 Theorem}. 
\end{proof}
As a corollary we immediately get formulas for the action of $\Gamma$ on 
$V(\eta)$
\begin{Corollary*}
Let $c \in \Gamma$ and let $\gamma \in \CC[X_\mu]^{S_\mu}$ be the 
corresponding symmetric polynomial through Zhelobenkho's isomorphism.
For each $z \in \mathcal N(\eta)$ and each $\sigma \in 
\mathsf{Shuff}^\eta_{\epsilon(z)}$ we have
\begin{align*}
c \D^\eta_\sigma(z)
	&= \pi(\gamma(\lambda + z)) \D^\eta_\sigma(z)
		+ \frac{1}{\eta!}\sum_{\ell(\tau) < \ell(\sigma)} 
			\pi \circ \Theta(\DD_{w_\eta \tau})
			(\gamma(\lambda + z) \partial_{\sigma^{-1}} \Delta_\eta) )
		\D_\tau^\eta (z).
\end{align*}
\end{Corollary*}


\end{document}