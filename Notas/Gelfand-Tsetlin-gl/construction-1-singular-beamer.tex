\documentclass[smaller,usepdftitle=false]{beamer}
\usepackage[latin1,utf8]{inputenc}
\usepackage[T1]{fontenc}
\usepackage{tikz}
\usetikzlibrary{calc}

\usepackage{mathdots}

\linespread{1.2}

\usetikzlibrary{topaths}


\mode<presentation>
\usetheme{Warsaw}
\setbeamertemplate{navigation symbols}{}
\setbeamertemplate{headline}{}
\setbeamertemplate{footline}{}
\usepackage[matrix,arrow,graph,curve]{xy}
\usepackage{amssymb}
\usepackage{leftidx}
\usepackage{graphics}
\usepackage{cancel}
\usepackage{xcolor}

\newcommand\NN{\mathbb N}
\newcommand\CC{\mathbb C}
\newcommand\QQ{\mathbb Q}
\newcommand\RR{\mathbb R}
\newcommand\ZZ{\mathbb Z}

\newcommand\maps{\longmapsto}
\newcommand\ot{\otimes}
\renewcommand\to{\longrightarrow}
\renewcommand\phi{\varphi}
\newcommand\stack[2]{\genfrac{}{}{0pt}{2}{#1}{#2}}
\newcommand\vectspan[1]{\left\langle #1 \right\rangle}

\newcommand\gl{\mathfrak{gl}}
\newtheorem*{Proposition}{Proposition}


\def\pausa{\pause \bigskip}


\title[]{How to build 1-singular GT-modules}
\author[]{Pablo Zadunaisky}
\date{IME - USP}

\begin{document}
% Remove this to get the pauses...
%\let\pause\relax

\begin{frame}
\titlepage
\end{frame}

\begin{frame}
\frametitle{Gelfand-Tsetlin tableaux}
A GT-tableau is an array $T(v)$ of $\binom{n}{2}$ complex numbers

\bigskip

\begin{tikzpicture}
\node (n1) at (-3,3) {$v_{n,1}$};
\node (n2) at (-1.5,3) {$v_{n,2}$};
\node (ndots) at (0,3) {$\cdots$};
\node (nn-1) at (1.5,3) {$v_{n,n-1}$};
\node (nn) at (3,3) {$v_{n,n}$};

\node (n-11) at (-2.25,2.25) {$v_{n-1,1}$};
\node (n-1dots) at (0,2.25) {$\cdots$};
\node (n-1n-1) at (2.25,2.25) {$v_{n-1,n-1}$};

\node (dots1) at (-1.5,1.5) {$\ddots$};
\node (dots2) at (0,1.5) {$\cdots$};
\node (dots3) at (1.5,1.5) {$\iddots$};

\node (21) at (-.75,.75) {$v_{2,1}$};
\node (22) at (.75,.75) {$v_{2,2}$};
\node (11) at (0,0) {$v_{1,1}$};

\node (A) at (-4.2, 2.75) {};
\node (B) at (4.2, 2.75) {};
\node (C) at (0,0) {};

\draw[->] (n1) -- (n-11);
\draw[->] (n-11) -- (n2);
\draw[->] (n-11) -- (-1.65,1.7) node{$\phantom{v_{n,n}}$};
\draw[->] (n2) -- (-0.75,2.5);
\draw[->] (nn-1) -- (n-1n-1);
\draw[->] (n-1n-1) -- (nn);
\draw[->] (21) -- (11);
\draw[->] (11) -- (22);
\end{tikzpicture}

\pausa

\textbf{Standard:} $v_{k,i} - v_{k-1,i} \in \ZZ_{\geq 0}$ and $v_{k-1,i} - 
v_{k,i+1} \in \ZZ_{>0}$.

\textbf{Integral:} All $v_{k,i} \in \ZZ$.

\textbf{Generic:} $v_{k,i} - v_{k,j} \notin \ZZ, 1 \leq k < n$.

\textbf{Elementary:} $\delta^{k,i}$ is the tableaux with $v_{k,i} = 1$, other 
$v_{r,s} = 0$.
\end{frame}

\begin{frame}
\frametitle{A theorem of Gelfand and Tsetlin}

\begin{Theorem}[Gelfand-Tsetlin -'50]
Let $\lambda = (\lambda_1, \ldots, \lambda_n)$ with $\lambda_i - \lambda_{i+1} 
\in \ZZ_{\geq 0}$, and
$$V(\lambda) = \langle T(v) \mid v_{n,i} = \lambda_i - i + 1 \mbox{ and }
T(v) \mbox{ standard }\rangle_\CC.$$ 
The space $V(\lambda)$ is a $\gl(n,\CC)$-module of highest weight $\lambda$
with
\begin{align*}
E_{k,k+1} T(v) &= - \sum_{i=1}^k e^+_{k,i}(v) T(v+\delta^{k,i}) \\
E_{k+1,k} T(v) &= \sum_{i=1}^k e^-_{k,i}(v) T(v-\delta^{k,i}) \\
E_{k,k} T(v) &= e_k(v) T(v)
\end{align*}
where the $e^\pm_{k,i}$ are rational functions, the $e_k$ are symmetric 
polynomials.
\end{Theorem}
\end{frame}

\begin{frame}
\frametitle{Gelfand-Tsetlin subalgebra}

\begin{gather*}
\gl(1,\CC) \subset \gl(2,\CC) \subset \cdots \subset \gl(n,\CC) \\
U(\gl(1,\CC)) \subset U(\gl(2,\CC)) \subset \cdots \subset U(\gl(n,\CC)) \\
\Gamma = \mbox{ Algebra generated by } \bigcup_k Z(U(\gl(k,\CC))) 
\subset U(\gl(n,\CC))
\end{gather*}
\pause
\vspace{-.5cm}
\begin{Proposition}[Zhelobenko]
Let $c_{k,i}$ be the $i$-th Casimir operator of $Z(U(\gl(n,k)))$. If $T(v) 
\in V(\lambda)$ then $c_{k,i} T(v) = \gamma_{k,i}(v) T(v)$ for $\gamma_{k,i} 
\in \CC[\lambda_{k,1}, \ldots, \lambda_{k,k}]^{S_k}$. Furthermore, $c_{k,i}
\mapsto \gamma_{k,i}$ induces an isomorphism $\Gamma \cong \CC[\lambda_{k,i}
\mid 1 \leq i \leq k \leq n]^{S_1 \times \cdots \times S_n}$
\end{Proposition}
\end{frame}

\begin{frame}
\frametitle{GT-Tableaux and characters of $\Gamma$}
\begin{align*}
\xymatrix{
	\mbox{Tableaux modulo } S_1 \times S_2 \times \cdots \times S_n
		\ar@/^18pt/[r] & \mbox{Characters of } \Gamma \ar@/^18pt/[l] \\
	\overline{T(v)} \ar@{<~>}[r]
		& \chi_v = \mbox{evaluation at } v
}
\end{align*}

\pause 

\begin{Definition}
A \emph{GT-module} is a $U(\gl(n,\CC))$-module $V$ decomposable as $V = 
\bigoplus_\chi V(\chi)$, with $V(\chi)$ the set of all vectors of
generalized $\Gamma$-eigenvalue $\chi$.
\end{Definition}

A tableau $T(v) \in V(\lambda)$ is an eigenvector of $\Gamma$ of eigenvalue 
$\chi_v$. Thus $V(\lambda)$ is always a GT-module.
\end{frame}

\begin{frame}
\frametitle{Existence result}
\bigskip
Explicit examples:
\begin{itemize}
\item Characters corresponding to standard tableaux (by Gelfand and Tsetlin's
work). Multiplicities all equal $1$.

\item Characters corresponding to generic tableaux (Drozd, Futorny and 
Ovsienko, around 1990). Multiplicities all equal $1$.

\item Characters corresponding to certain non-generic tableaux (Futorny, 
Grantcharov and Ramírez, in 2016). Some multiplicities are $2$.
\end{itemize}

QUESTION: What characters of $\Gamma$ appear in irreducible GT-modules, and
with what multiplicity?
\pausa

ANSWER: All of them and the multiplicity is uniformly bounded, as proved by 
Ovsienko in 2012.
\end{frame}

\begin{frame}
\frametitle{The family of generic tableaux}
Let us see how to construct GT-modules associated to generic tableaux.

\center
\begin{tikzpicture}
\node (A) at (2,0) {};
\node (-A) at (-2,0) {};
\node (B) at (1,{sqrt(3)}) {};
\node (-B) at (-1,{-sqrt(3)}) {};
\node (C) at (1,{-sqrt(3)}) {};
\node (-C) at (-1,{sqrt(3)}) {};

\draw[thick] (0,0) -- ($1.2*(A)$);
\draw[thick] (0,0) -- ($1.2*(-A)$);
\draw[thick] (0,0) -- ($1.2*(B)$);
\draw[thick] (0,0) -- ($1.2*(-B)$);
\draw[thick] (0,0) -- ($1.2*(C)$);
\draw[thick] (0,0) -- ($1.2*(-C)$);

\draw ($1.2*(A) - .2*(B)$) -- ($1.2*(B) - .2*(A)$);
\draw ($1.2*(A) - .2*(C)$) -- ($1.2*(C) - .2*(A)$);
\draw ($1.2*(-A) - .2*(-C)$) -- ($1.2*(-C) - .2*(-A)$);
\draw ($1.2*(-A) - .2*(-B)$) -- ($1.2*(-B) - .2*(-A)$);
\draw ($1.2*(C) - .2*(-B)$) -- ($1.2*(-B) - .2*(C)$);
\draw ($1.2*(-C) - .2*(B)$) -- ($1.2*(B) - .2*(-C)$);

\draw[dashed,thick] (30:-2cm) -- (30:2cm);
\draw[dashed] (-1,1)+(30:2.5cm) -- +(30:-1.5cm);
\draw[dashed] (1,-1)+(30:1.5cm) -- +(30:-2.5cm);

\fill[fill=black] (0,0.9) circle (0.05cm);

\fill[fill=red] (1,0) circle (0.05cm);

\fill[fill=red] (0,0)+(30:1.73cm) circle (0.05cm);
\draw[color=red,dashed] (0,0) +(30:1.73cm) circle (0.1cm);

\fill[fill=red] (1,{(sqrt(3))}) circle (0.03cm);
\draw[color=red,thick] (1,{(sqrt(3))}) circle (0.06cm);
\draw[color=red,thick] (1,{(sqrt(3))}) circle (0.1cm);

\node at (0,-2.5) {A cross section of the family of generic tableaux};
\end{tikzpicture}
\end{frame}

\begin{frame}
\frametitle{Characters associated to generic tableaux}
\vspace{-1cm}
\begin{gather*}
A = \mbox{Regular functions over generic tableaux}  \\
(\mbox{Localization of }\CC[\lambda_{k,i} \mid 1 \leq i \leq k \leq n], 
\mbox{contains } e^\pm_{k,i}, e_k) 
\end{gather*}

\begin{Theorem}
Let $V_A$ be the free $A$-module over the set $\{T(\lambda + z) \mid z_{k,i} 
\in \ZZ, z_{n,i} = 0\}$. The module $V_A$ is a $U$-module with 
action
\vspace{-.3cm}
\begin{align*}
E_{k,k+1} T(\lambda + z) &= - \sum_{i=1}^k e^+_{k,i}(\lambda + z) 
	T(\lambda + z+\delta^{k,i}); \\
E_{k+1,k} T(\lambda + z) &= \sum_{i=1}^k e^-_{k,i}(\lambda + z) 
	T(\lambda + z -\delta^{k,i}); \\
E_{k,k} T(\lambda + z) &= e_k(\lambda + z) T(\lambda + z).
\end{align*}
Also $c_{k,i} T(\lambda + z) = \gamma_{k,i}(\lambda + z) T(\lambda + z)$.
\end{Theorem}
\end{frame}

\begin{frame}
\frametitle{Characters associated to generic tableaux}
\begin{Corollary}[Drozd, Futorny, Ovsienko '92]
Let $v$ be a generic tableau, $\CC_v$ the corresponding $1$-dimensional 
$A$-module. Then $V = V_A \ot_A \CC_v$ is a GT-module with $V(\chi_v) \neq
0$. 
\end{Corollary}
\begin{proof}
The module $V$ has a $\CC$-basis $T(v + z) = T(\lambda + z) \ot_A 1$. The 
action of $U$ is obtained evaluating the GT-formulas at $v$. In particular
$c_{k,i} T(v+z) = \gamma_{k,i}(v+z)$. Take $z=0$.
\end{proof}
All characters of the form $\chi_{v+z}$ have multiplicity $1$ in $V_v$, and
the rest have multiplicity zero.
\end{frame}

\begin{frame}
\frametitle{The family of $1$-singular tableaux}
Fix $1 \leq s < t \leq r < n$. A tableau is: 

\textbf{singular} if $v_{r,s} - v_{r,t} \in \ZZ$; 

\textbf{$1$-singular} if this is the only singular pair; 

\textbf{$1$-critical} if it is $1$-singular and $v_{r,s} = v_{r,t}$.
%\pause
\center
\begin{tikzpicture}
\node (A) at (2,0) {};
\node (-A) at (-2,0) {};
\node (B) at (1,{sqrt(3)}) {};
\node (-B) at (-1,{-sqrt(3)}) {};
\node (C) at (1,{-sqrt(3)}) {};
\node (-C) at (-1,{sqrt(3)}) {};

\draw[color=white!90!black] (0,0) -- ($1.2*(A)$);
\draw[color=white!90!black] (0,0) -- ($1.2*(-A)$);
\draw[thick] (0,0) -- ($1.2*(B)$);
\draw[thick] (0,0) -- ($1.2*(-B)$);
\draw[thick] (0,0) -- ($1.2*(C)$);
\draw[thick] (0,0) -- ($1.2*(-C)$);

\draw ($1.2*(A) - .2*(B)$) -- ($1.2*(B) - .2*(A)$);
\draw ($1.2*(A) - .2*(C)$) -- ($1.2*(C) - .2*(A)$);
\draw ($1.2*(-A) - .2*(-C)$) -- ($1.2*(-C) - .2*(-A)$);
\draw ($1.2*(-A) - .2*(-B)$) -- ($1.2*(-B) - .2*(-A)$);
\draw ($1.2*(C) - .2*(-B)$) -- ($1.2*(-B) - .2*(C)$);
\draw ($1.2*(-C) - .2*(B)$) -- ($1.2*(B) - .2*(-C)$);

\draw[dashed,thick] (30:-2cm) -- (30:2cm);
\draw[dashed] (-1,1)+(30:2.5cm) -- +(30:-1.5cm);
\draw[dashed] (1,-1)+(30:1.5cm) -- +(30:-2.5cm);

\fill[fill=black] (0,0.9) circle (0.05cm);

\fill[fill=black] (1,0) circle (0.05cm);

\fill[fill=red] (0,0)+(30:1.73cm) circle (0.05cm);
\draw[color=red,dashed] (0,0) +(30:1.73cm) circle (0.1cm);

\fill[fill=red] (1,{(sqrt(3))}) circle (0.03cm);
\draw[color=red,thick] (1,{(sqrt(3))}) circle (0.06cm);
\draw[color=red,thick] (1,{(sqrt(3))}) circle (0.1cm);

\node at (0,-2.5) {A cross section of the family of generic + $1$-critical 
tableaux};
\end{tikzpicture}
\end{frame}

\begin{frame}
\frametitle{Characters associated to $1$-singular tableaux}
$B =$ Regular functions in $A$ that can be evaluated in $1$-singular 
tableaux 

(Localization of $\CC[\lambda_{k,i} \mid 1 \leq i \leq k \leq n]$, 
contains $e^\pm_{k,i}, e_k \mbox{ except } e^\pm_{r,s}, 
	e^{\pm}_{r,t}$) 
\pause
\begin{Theorem}[Z, '17]
Let $\tau$ be the involution interchanging coordinates $(r,s)$ and $(r,t)$, 
and let 
\begin{align*}
S(\lambda + z)
	&= \frac{T(\lambda + z) + T(\lambda + \tau(z))}{2},\\
A(\lambda + z)
	&= \frac{T(\lambda + z) - T(\lambda + \tau(z))}
	{2(\lambda_{r,s} - \lambda_{r,t})}.
\end{align*}
The $B$-module $L_B \subset V_A$ generated by all the $S(z), A(z)$ is a 
$U$-submodule of $V_A$. 
\end{Theorem}
\end{frame}

\begin{frame}
\frametitle{Characters associated to $1$-singular tableaux}
\begin{Corollary}
Let $v$ be a generic or a $1$-singular tableaux, let $\CC_v$ be the 
corresponding $B$-module, and let $V_v = L_B \ot_B \CC_v$.
\begin{itemize}
\item If $v$ is $1$-critical then $V_v$ is a GT-module with 
$V_v(\chi_v) \neq 0$. 
 
\item If $v$ is $1$-singular, $v+z$ is $1$-critical for some $z$, and 
$V_{v+z}(\chi_v) \neq 0$.

\item If $v$ is generic then $V_v \cong V_A \ot_A \CC_v$.
\end{itemize}
\end{Corollary}

\begin{proof}[Proof of the theorem and the corollary]
Direct computations inside $V_A$ plus reduction modulo $\lambda_{k,i} - 
v_{k,i}$.
\end{proof}

Some characters have multiplicity two!

\end{frame}

\begin{frame}
\frametitle{The family of fully critical tableaux}
\textbf{Fully critical tableaux:} If $v_{r,s} - v_{r,t} \in \ZZ$, then
$v_{r,s} = v_{r,t}$ ($r$ fixed)
\center
\begin{tikzpicture}
\node (A) at (2,0) {};
\node (-A) at (-2,0) {};
\node (B) at (1,{sqrt(3)}) {};
\node (-B) at (-1,{-sqrt(3)}) {};
\node (C) at (1,{-sqrt(3)}) {};
\node (-C) at (-1,{sqrt(3)}) {};

\draw[color=white!90!black] (0,0) -- ($1.2*(A)$);
\draw[color=white!90!black] (0,0) -- ($1.2*(-A)$);
\draw[color=white!90!black] (0,0) -- ($1.2*(B)$);
\draw[color=white!90!black] (0,0) -- ($1.2*(-B)$);
\draw[color=white!90!black] (0,0) -- ($1.2*(C)$);
\draw[color=white!90!black] (0,0) -- ($1.2*(-C)$);

\draw ($1.2*(A) - .2*(B)$) -- ($1.2*(B) - .2*(A)$);
\draw ($1.2*(A) - .2*(C)$) -- ($1.2*(C) - .2*(A)$);
\draw ($1.2*(-A) - .2*(-C)$) -- ($1.2*(-C) - .2*(-A)$);
\draw ($1.2*(-A) - .2*(-B)$) -- ($1.2*(-B) - .2*(-A)$);
\draw ($1.2*(C) - .2*(-B)$) -- ($1.2*(-B) - .2*(C)$);
\draw ($1.2*(-C) - .2*(B)$) -- ($1.2*(B) - .2*(-C)$);

\draw[dashed,thick] (30:-2cm) -- (30:2cm);
\draw[dashed] (-1,1)+(30:2.5cm) -- +(30:-1.5cm);
\draw[dashed] (1,-1)+(30:1.5cm) -- +(30:-2.5cm);

\fill[fill=black] (0,0.9) circle (0.05cm);

\fill[fill=black] (1,0) circle (0.05cm);

\fill[fill=red] (0,0)+(30:1.73cm) circle (0.05cm);
\draw[color=red,dashed] (0,0) +(30:1.73cm) circle (0.1cm);

\fill[fill=red] (1,{(sqrt(3))}) circle (0.03cm);
\draw[color=red,thick] (1,{(sqrt(3))}) circle (0.06cm);

\node at (0,-2.5) {The space of generic and fully critical tableaux};
\end{tikzpicture}
\end{frame}

\begin{frame}
\frametitle{Characters associated to fully-critical tableaux}
\begin{gather*}
C = \mbox{Regular functions in $A$ that can be evaluated at fully critical 
tableaux} 
\end{gather*}

QUESTION: Is there a full $C$-lattice $L_C \subset V_A$ which is also a
$U$-submodule?

\pausa

ANSWER: Yes. We are working on the best way to present it. 

\pausa
\center
TO BE CONTINUED... 
\end{frame}
\end{document}