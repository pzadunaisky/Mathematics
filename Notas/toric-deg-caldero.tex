%%%%%%%%%%%%%%%%%%%%%% Generalities %%%%%%%%%%%%%%%%%%5
\documentclass[11pt,fleqn]{article}
\usepackage[paper=a4paper]
  {geometry}

\pagestyle{plain}
\pagenumbering{arabic}
%%%%%%%%%%%%%%%%%%%%%%%%%%%%%%%%
% esto tiene que estar, en este orden...
\usepackage[small]{titlesec}
\usepackage{paragraphs}
\usepackage{hyperref}
\usepackage{amsthm,thmtools}

%\linespread{1.2}
\setlength{\parskip}{1.2ex}

\usepackage[utf8,latin1]{inputenc}
\usepackage[spanish,english]{babel}
\usepackage{enumerate}
\usepackage[alphabetic,initials]{amsrefs}
\usepackage{amsfonts,amssymb,amsmath}
%\usepackage{dsfont}
\usepackage{mathtools}
\usepackage{graphicx}
\usepackage[poly,arrow,curve,matrix]{xy}
\usepackage{wrapfig}

%\swapnumbers


% numbered versions
\declaretheoremstyle[headformat=swapnumber, spaceabove=\paraskip,
bodyfont=\itshape]{mystyle}
\declaretheorem[name=Lemma, sibling=para, style=mystyle]{Lemma}
\declaretheorem[name=Propositionn, sibling=para, style=mystyle]{Proposition}
\declaretheorem[name=Theorem, sibling=para, style=mystyle]{Theorem}
\declaretheorem[name=Corollary, sibling=para, style=mystyle]{Corollary}
\declaretheorem[name=Definition, sibling=para, style=mystyle]{Definition}

% unnumbered versions
\declaretheoremstyle[numbered=no, spaceabove=\paraskip,
bodyfont=\itshape]{mystyle-empty}
\declaretheorem[name=Lemma, style=mystyle-empty]{Lemma*}
\declaretheorem[name=Proposition, style=mystyle-empty]{Proposition*}
\declaretheorem[name=Theorem, style=mystyle-empty]{Theorem*}
\declaretheorem[name=Corollary, style=mystyle-empty]{Corollary*}
\declaretheorem[name=Definition, style=mystyle-empty]{Definition*}

% plain style
\declaretheoremstyle[
        headformat={{\bfseries\NUMBER.}{\itshape\NAME}\NOTE\ignorespaces},
        spaceabove=\paraskip, 
        headpunct={.},
        headfont=\itshape,
        bodyfont=\normalfont
        ]{mystyle-plain}
\declaretheorem[sibling=para, style=mystyle-plain]{Example}
\declaretheorem[sibling=para, style=mystyle-plain]{Remark}

% proofs, just as in amsthm but with adapted spacing
\makeatletter
\renewenvironment{proof}[1][\proofname]{\par
  \pushQED{\qed}%
  \normalfont \topsep.75\paraskip\relax
  \trivlist
  \item[\hskip\labelsep
        \itshape
    #1\@addpunct{.}]\ignorespaces
}{%
  \popQED\endtrivlist\@endpefalse
}
\makeatother

%%%%%%%%%%%%%%%%%%%%%%%%%%% The usual stuff%%%%%%%%%%%%%%%%%%%%%%%%%
\newcommand\NN{\mathbb N}
\newcommand\CC{\mathbb C}
\newcommand\QQ{\mathbb Q}
\newcommand\RR{\mathbb R}
\newcommand\ZZ{\mathbb Z}
\newcommand\KK{\mathbb K}

\newcommand\maps{\longmapsto}
\newcommand\ot{\otimes}
\renewcommand\to{\longrightarrow}
\renewcommand\phi{\varphi}
\newcommand\stack[2]{\genfrac{}{}{0pt}{2}{#1}{#2}}
\newcommand\uu[1]{\underline{#1}}

%%%%%%%%%%%%%%%%%%%%%%%%% Specific notation %%%%%%%%%%%%%%%%%%%%%%%%%

\newcommand\A{\mathcal A}
\newcommand\B{\mathcal B}
\renewcommand\L{\mathcal L}
\newcommand\M{\mathcal M}
\newcommand\F{\mathcal F}
\renewcommand\O{\mathcal O}
\newcommand\GG{\Gamma}
\renewcommand\b{\mathfrak{b}}
\newcommand\g{\mathfrak{g}}
\newcommand\h{\mathfrak h}
\newcommand\kk{k}
\newcommand\n{\mathfrak{n}}
\newcommand\opp{\mathsf{opp}}
\DeclareMathOperator\Id{\mathsf{Id}}
\DeclareMathOperator\st{\mathsf{st}}

\newcommand\C{\mathcal C}
\newcommand\K{\mathsf K}
\newcommand\D{\mathsf D}

\DeclareMathOperator\Mod{\mathsf{Mod}}
\DeclareMathOperator\Hom{\mathsf{Hom}}
\DeclareMathOperator\Ext{\mathsf{Ext}}
\DeclareMathOperator\Tor{\mathsf{Tor}}

\DeclareMathOperator\GrMod{\mathsf{GrMod}}
\DeclareMathOperator\GrHom{\underline{\mathsf{Hom}}}
\DeclareMathOperator\GrExt{\underline{\mathsf{Ext}}}
\DeclareMathOperator\GrTor{\underline{\mathsf{Tor}}}

\DeclareMathOperator\Ab{\mathsf{Ab}}
\DeclareMathOperator\ShHom{\mathcal Hom}
\DeclareMathOperator\ShExt{\mathcal Ext}
\DeclareMathOperator\ShTor{\mathcal Tor}

\DeclareMathOperator\pd{pd}
\DeclareMathOperator\id{id}
\DeclareMathOperator\depth{depth}
\DeclareMathOperator\Spec{Spec}
\DeclareMathOperator\supp{supp}
\DeclareMathOperator\ann{ann}
\DeclareMathOperator\im{Im}
\DeclareMathOperator\gr{gr}

%%%%%%%%%%%%%%%%%%%%%%%%%%%%%%%%%%%%%% TITLES %%%%%%%%%%%%%%%%%%%%%%%%%%%%%%


\title{Toric Degenerations of Schubert Varieties [P. Caldero]}
\author{[toric-deg-caldero.tex]}
\date{16/8/12}
\begin{document}
\maketitle
\section{General definitions}
\paragraph \textbf{Notation}
We work over the field $\kk = \QQ(q)$, all unadorned tensor products will be over this
field. We denote by $\A^0$ the localization of $\QQ[q]$ at the maximal ideal of functions 
which have a root at zero. We identify $\QQ = \A^0/q\A^0$

\paragraph  
Let $\g$ denote a finite dimensional complex simple Lie algebra with $\h \subset \g$ its
Cartan subalgebra. Denote by $\varpi_1, \ldots, \varpi_n \in \h^*$ its fundamental weights, 
and let $P = \ZZ \varpi_1 + \ldots + \ZZ \varpi_n \subset \h^*$ be the weight lattie. Its
finite dimensional representations are parametrized by the \emph{dominant weights}, $P^+ = 
\NN \varpi_1 + \ldots + \NN \varpi_n$. Let $\alpha_1, \ldots, \alpha_n$ denote the roots of
$\g$, $\Pi$ be a basis of the root system with $n = |\Pi|$, $W$ its Weyl group and $(-,-)$ 
the $W$-invariant form over $P$. We denote by $Q \subset P$ the lattice generated by the
roots, and $Q^+ = Q \cap P^+$.

\paragraph 
We denote by $U_q = U_q(\g)$ the quantum envelopping algebra of $\g$ over $\QQ(q)$. 
$U_q$ is the algebra with generators $E_i, F_i, K_\mu$ for $i = 1, \ldots, n, \mu
\in Q$ and relations
\begin{align*}
	K_\lambda K_\mu &= K_{\lambda + \mu} \\
	K_i E_j K_i^{-1} &= q^{(\alpha_i, \alpha_j)} E_j
	& K_i F_j K_i^{-1} &= q^{-(\alpha_i, \alpha_j)} F_j \\
	[E_i, F_j] &= \delta_{ij} \frac{K_i - K_i^{-1}}{q_i-q_i^{-1}} 
	&\mbox{Quantum Serre Relations} 
\end{align*}
where $q_i = q^{(\alpha_i,\alpha_i)/2}$. For $X_i = F_i, E_i \in U_q(\g)$, we define the 
divided powers as
$$ X_i^{(r)} = \frac{1}{[r]_{q_i}!} X_i^r $$

\paragraph
$U_q$ is a Hopf algebra with comultiplication $\Delta$, antpode $S$ and counit
$\varepsilon$ defined by
\begin{align*}
	\Delta(E_i) &= E_i \ot K_i^{-1} + E \ot 1
&	\Delta(F_i) &= F_i \ot 1 + K_i \ot F_i 
&	\Delta(K_i) &= K_i \ot K_i \\
	S(E_i) &= -E_iK_i & S(F_i) &= -K_i^{-1}F_i & S(K_i) &= K_i^{-1} \\
	\varepsilon(E_i) &= 0 & \varepsilon(F_i) &= 0 & \varepsilon(K_i) &= 1 \\
\end{align*}
We use Sweedler's notation, so for any element $a \in U_q$, we write $\Delta(a) = \sum a_1
\ot a_2$.

\paragraph
We denote $U_q(\b)$ the algebra generated by all $E_i, K_i^{\pm 1}$, and by $U_q(\n)$ the
algebra generated only by the $E_i$. The same definitions apply changing $E$ by $F$ for
$U_q(\b^-), U_q(\n^-)$.

\paragraph 
Given $\lambda \in P^+$, we denote by $V_q(\lambda)$ the finite dimensional representation
of $U_q(\g)$ of highest weight $\lambda$. We fix a highest weight vector and denote it
$v_\lambda$. It is clear that $V_q(\lambda) = U_q(\n^-) v_\lambda$. 

Given an element $w \in W$, we denote by $v_{w\lambda}$ the image of $v_\lambda$ by the 
Lusztig operator $T_w$. This simpliy fixes an element in the one dimensional component of 
weight $w \cdot \lambda$ of $V_q(\lambda)$, where $\cdot$ is the action of the Weyl group 
on the weight lattice. The module $V_q(\lambda)_w := U_q(\b)v_{w\lambda}$ is called a 
Demazure module. 

\paragraph
We denote by $U_i$ the sub-Hopf-algebra of $U_q$ generated by $E_i, F_i, K_i^{\pm 1}$.

\paragraph
Since $U_q(\g)$ is a Hopf algebra, its vector space dual $U_q(\g)^*$ is an algebra, with
multiplication given by $\phi \cdot \psi (a) = \sum \phi(a_1) \psi(a_2)$. 

\section{Global and crystal bases}
\paragraph
Denote by $\B \subset U_q(\n^-)$ the canonical (or global) basis as defined in \cite{J}*{chapter 11}.
This is a basis of the $\ZZ[q,q^{-1}]$-form of $U_q(\n^-)$, the $\ZZ[q,q^{-1}]$ algebra
generated by $K_\mu$ and the divided powers $F_i^{(m)}$ for all $\mu \in Q$, $m \in \NN$.
From this one deduced immediately that for $b,b' \in \B$, $bb' \in \ZZ[q,q^{-1}]\B$. By
construction, $b \in \B$ if and only if $\overline b \in \B$, where $\overline -$ is the
antiautomorphism of $U_q$ which fixes the $E_i, F_i$ and sends $K_\lambda$ to
$K_{-\lambda}$.

\paragraph
Once again by construction, there are operators $\tilde E_i, \tilde F_i: U_q(\n^-) \to
U_q(\n^-)$ such that for all $b \in \B$, there is a $b' \in \B \cup \{0\}$ with $\tilde
E_i(b) \equiv b' \mod q^{-1}$. The assignation $b \mapsto b'$ is denoted by $e_i: \B \to
\B \cup \{0\}$. Put $\varepsilon_i(b) = \max \{k|e_i^k(b)\neq 0\}$

\begin{Theorem*}
\label{global-crystal}
	Let $\lambda \in P^+$, $\lambda = \sum_i \lambda_i \varpi_i$. Set
$$ \B_\lambda = \{b \in \B|\varepsilon_i(\overline b) \leq \lambda_i\}$$
Then the map $b \in \B_\lambda \mapsto b v_\lambda$ is bijective and its image is a basis
of $V_q(\lambda)$.
\end{Theorem*}
This is \cite{J}*{11.16 Theorem}, where $\B_\lambda$ is not given explicitly. This is
given without proof in \cite{K}*{Proposition 8.2}.

\paragraph
\textbf{String parametrization of the Cannonical basis:}
\label{string}
Fix a reduced decompostion of the longest word of the Weyl group, say $w_0 = s_{i_1} \ldots 
s_{i_N}$. Associated to any such decomposition there is a parametriation of the cannonical
basis by elements of $\NN^N$, which we now present.

Let $\Lambda_i(b) = \tilde e_i^{\varepsilon_i(b)}(b)$. Assign to each $b \in \B$ the
sequence of numbers 
$$A_{\tilde w_0} (b) = (\varepsilon_{i_N}(b),\varepsilon_{i_{N-1}}(\Lambda_{N}(b))
,\ldots, \varepsilon_{i_1}(\Lambda_{i_2} \ldots \Lambda_{i_N}(b))) \in \NN^N $$
The image of this parametrization is denoted by $\C_{\tilde w_0} \subset \NN^N$. It is the 
set of integral points of a finitely generated rational cone. For each $\psi \in \C$, we
write $b^\psi = A_{\tilde w_0}^{-1}(\psi)$.

This results are found in \cite{Lit}

\paragraph
Consider the bilinear form $(-,-): U_q(\n) \times U_q(\n^-) \to \kk$. Since it is
non-degenerate, for every $b \in B$ there is a unique element $b^* \in U_q(\n)$ such that 
$(b^*,b') = \delta_{bb'}$. The set $\B^*$ of all such elements is a basis of $U_q(\n)$,
called the \emph{dual canonical basis}. It is proven in \cite{L}*{14.4.13} that this basis
shares many traits with $\B$, for examble $b^* b'^* \in \ZZ[q,q^{-1}]\B^*$


\section{Matrix coefficients and quantum flag varieties}
\paragraph Let $V$ be a finite dimensional representation of $U_q$. There is an injective
morphism $i: V^* \ot V \to U_q(\g)^*$, such that for any $\xi \in V^*, v \in V, a \in U_q$,
$i(\xi \ot v) (a) = \xi (av)$. The image of $i$ is called the set of matrix coefficients
of $V$. It is standard to denote the matrix coefficient corresponding to $v \in
V_q(\lambda), \xi \in V_q(\lambda)^*$ by $c^\lambda_{\xi,v}$.

\begin{Lemma}
Any matrix coefficient is a linear combination of matrix coefficients of irreducible
representations. Furthermore, the $k$-span of said matrix coefficients forms a subalgebra
of $U_q(\g)^*$.
\end{Lemma}
We denote this algebra by $\O_q[G]$.

\begin{proof}
Let $c$ be a matrix coefficient, corresponding to $v \in V, \xi \in V^*$. We may decompose
$V = \bigoplus_{i = 1}^r V_q(\lambda_i)$, $v = \sum v_i, \xi = \sum \xi_i$. It is
immediate to check that $c = \sum_i c^{\lambda_i}_{\xi_i,v_i}$. 

Take $c^\lambda_{v,\xi}$ and $c^\mu_{w,\xi}$. By definition their product is given by
convolution, that is, 
$$c^\lambda_{v,\xi} \cdot c^\mu_{w, \chi} (a) = \sum c^\lambda_{v,\xi}(a_1) 
c^\mu_{w,\chi}(a_2) = \xi(a_1 v) \chi(a_2 w) = i(\chi \ot \xi, v \ot w)$$
which is again a matrix coefficient.
\end{proof}

\paragraph 
Let $C^+(\lambda)$ be the set of matrix coefficients of the form
$c^\lambda_{\xi,v_{\lambda}}$. We put $\O_q[G/U] := \sum_{\lambda \in P^+} C^+(\lambda)$

\begin{Lemma*}
$\O_q[G/U]$ is a $P^+$-graded algebra.
\end{Lemma*}
\begin{proof}
	$U_q(\g)^*$ has a natural left $U_q$-module structure, which restricts to
$C^+(\lambda)$, and the assignation $\xi \mapsto c^\lambda_{\xi,v_\lambda}$ is an
isomorphism. Hence $C^+(\lambda)$ is an irreducible $U_q$-module. Given a total order of
$P^+$, there is no highest weight vector of weight $\lambda$ in $\bigoplus_{\mu < \lambda} 
C^+(\mu)$, hence $C^+(\lambda)$ does not intersect this sum. This proves that the sum is
direct.

Now given $c_\xi \in C^+(\lambda), c_\chi \in C^+(\mu)$, we have that 
$c_\xi c_\chi = i(\xi \ot \chi,v_\lambda \ot v_\mu)$. Let $V' = U_q(\g) v_\lambda \ot
v_\mu$, then $i(\xi \ot \chi,v_\lambda \ot v_\mu) = i(\xi \ot \chi|_{V'}, v_\lambda \ot
v_\mu)$. It is easy to check that $v_\lambda \ot v_\mu$ is a highest weight vector of
weight $\lambda + \mu$, so $V' \cong V_q(\lambda + \mu)$. Through this isomorphism, we
have $c_\xi c_\chi = c^{\lambda + \mu}_{\xi \ot \chi|_{V'}, v_\lambda \ot v_\mu} \in
C^+(\lambda + \mu)$.
\end{proof}

From this lemma we deduce that given $I \subset \{\varpi_1, \ldots, \varpi_n\}$ and
setting $\mathcal J = \sum_{i \notin I} \NN \varpi_i$, we have an associated $P^+$-graded
subalgebra of $\O_q[G/U]$, 
$$\O_q[G/P_I]:= \bigoplus_{\lambda \in \mathcal J} C^+(\lambda)$$
which is the \emph{quantum flag variety associated to $I$}.

\begin{Lemma}
Let $\rho: U_q(\g)^* \to U_q(\b^-)^*$ be the restriction morphism. Then $\rho$ is
injective over $\O_q[G/U]$, and hence over $\O_q[G/P_I]$.
\end{Lemma}
\begin{proof}
	Let $c_i = c_{\xi_i,v_{\lambda_i}}^{\lambda_i}$, $c = c_1 + \ldots + c_n$ and 
suppose that $\rho(c) = 0$. To see that $c = 0$, it is enough to prove it for all
monomials of the form $F^I K_\mu E^J$. Now $c_i(F^I K_\mu E^J) = \xi_i(F^I K_\mu E^J
v_\lambda)$. If $J \neq 0$, then $E^J v_\lambda = 0$ and hence $c (F^I K_\mu E^J) = \sum
c_i(F^I K_\mu E^J) = 0$. On the other hand, if $J = 0$, $F^J K_\mu \in U_q(\b^-)$, and so
$c(F^J K_\mu) = 0$ by hypothesis. Hence $c = 0$.
\end{proof}

\paragraph
Fix $I$ a subset of the set of fundamental weights, and denote by $W_I \subset W$ the
subgroup generated by reflections $s_{\alpha_i}$ with $i \in I$. Finally, for each class
$W/W_I$ pick a representative of smallest length; the set of all this representatives is
denoted by $W^I$.

Fix $w \in W^I$, and let $V_q(\lambda)_w$ be the associated Demazure module. The space
$$J_w^I = \bigoplus_{\lambda \in \mathcal J} \kk-\mathsf{span}\{c_\xi \in C^+(\lambda)|\xi 
\in V_q(\lambda)_w^\perp \} \subset \O_q[G/P_I]$$
is called the \emph{Schubert ideal} associated to $w$. It is a two-sided ideal because the 
image of $v_\lambda \ot v_\mu$ by $w$ is $v_{w \lambda} \ot v_{w \mu}$, hence by the
product formula, if either $\xi$ or $\chi$ are zero over their respective Demazure modules,
then $c_\xi \cdot c_\chi$ will be zero over $V_q(\lambda + \mu)_w$. The quotient
$\O_q[G/P_I]_w = \O_q[G/P_I]/J^I_w$ is called the quantum Schubert variety associated to
$w$.

\section{Caldero's Formula and semigroup dominated algebras}
Following Caldero's style, we claim that the following reuslts can be found in the paper
we are reviewing. 

\begin{Theorem} 
Fix a decomposition $\tilde w_0 = s_{i_1} \ldots s_{i_N}$ of the longest word of $W$.
Let $\GG_{\tilde w_0} = \{(\lambda, \psi)|b_{\psi} \in \B_\lambda\} \subset P^+ \times
\NN^N$. Then the set $b_\psi^* K_\lambda$ is a basis of $\rho(\O_q[G/U])$.
\end{Theorem}
As we have seen before, $\rho$ is injective over the quantum flag varieties, and so from
this we get a basis of $\O_q[G/U]$ parametrized by the elements of $\GG_{\tilde w_0}$.
For each $s \in \GG_{\tilde w_0}$, we denote by $b_s$ the corresponding element of the
basis of $\O_q[G/U]$.

\paragraph
Since $P^+$ is an ordered semigroup isomorphic to $\NN^n$, $\GG_{\tilde w_0}$ can be seen
as embedded in $\NN^{n+N}$. Caldero claims that it is also a rational cone, but I haven't
checked this. His referece, as usual, is incomplete.

\begin{Theorem*}
Let $s,s' \in \GG_{\tilde w_0}$. Then we have
\begin{align*} 
\label{domination}
\tag{$\dagger$}
b_s b_{s'} &= \sum_{s'' < s + s'} d_{s,s'}^{s''} b_{s''} 
& d_{s,s'}^{s''} \in \kk 
\end{align*}
Furthermore, $d_{s,s'}^{s+s'} = q^{?}$.
\end{Theorem*}

\paragraph
Since for each $\tilde w_0$ we get a diferent parametrization, the previous formula is in
fact several diferent ones. We say that $\tilde w_0$ is \emph{adapted} to $w \in W$ if
there is a reduced decomposition of $w$ of the form $s_{i_1} \ldots s_{i_l}$, where $l =
\ell(w)$. In that case, from \cite{Lit}*{section 1} we have that $A_{\tilde w_0}(\B_w) =
\NN^l \times \NN^{N - l}$, so in particular, the basis of $\O_q[G/U]_w$ is parametrized
by $P^+ \times \NN^l$.

\paragraph
Suppose $s = (\lambda, \psi)$. Then by definition $b^*_\psi \in (B_\lambda)^* \subset
V_q(\lambda)^*$, so formula (\ref{domination}) also makes sense when restricted to
$\O_q[G/P_I]$, and by the same reduction argument as before, to Schubert varieties
$\O_q[G/P_I]_w$, only this time the parametrizing semigroup is $\mathcal J$. Therefore
formula (\ref{domination}) makes sense when retricted to arbitrary Schubert varieties of
arbitrary quantum flag varieties. In the next section we will prove, following Caldero's
idea, that algebras with a basis satisfying a relation such as this always have a toric 
degeneration.
\end{document}
\textbf{Dual Spaces and Bilinear Forms}
\paragraph
There exists a unique bilinear form $(-,-):U_q(\b) \times U_q(\b^-)\to \kk$ such that for
all $u,u' \in U_q(\b), v,v' \in U_q(\b^-), \lambda, \mu \in P^+, i,j \in \{1, \ldots,
n\}$.
\begin{align*}
	(u,vv') &= (u_1,v)(u_2,v') & (uu',v) &= (u',v_1)(u,v_2) \\
	(K_\lambda,K_\mu) &= q^{(\lambda, \mu)} & (K_\lambda, F_i)&=0 \\
	(K_\lambda, E_i) &= 0 & (E_i,F_j)&= \frac{\delta_{ij}}{1-q_i^2}.
\end{align*}
Fixing $\mu \in Q^+$, this form is non degenerate over $U_q(\b)_\mu \times
U_q(\b^-)_{-\mu}$. See \cite{J}*{Proposition 6.12} for a proof (Jantzen proves this for a
bilinear form on $U_q(\b^-) \times U_q(\b)$ with different values on the generators, but
his proof can be adapted to the present case). We point out that $(1,-) = (-,1) =
\varepsilon$, the counit of the algebra.

For any dominant weight in the root lattice, $\lambda \in Q^+$, the restriction of $(-,-)$
to $U_q(\b)_\mu \times U_q(\b^-)_{-\mu}$ is nondegenerate \cite{J}*{Proposition 6.18}.
This is the first result in this notes that uses the fact that $\mathsf{char} \kk = 0$ and
that $q$ is trascendental over $\QQ$. 

\paragraph
With this bilinear form, we may define
\begin{align*} 
	\langle-,-\rangle&: U_q \times U_q \to \kk(q^{1/2}) \\
	\langle X_1 K_\lambda S(Y_1), Y_2 K_\mu S(X_2) \rangle &=
(X_1,Y_2)(X_2,Y_1)q^{-(\lambda,\mu)/2}.
\end{align*}
where $X_i \in U_q(\b)$, $Y_i \in U_q(\b^-)$. This defines a nondegenerate bilinear form.
See \cite{J}*{6.20} for the details.

\begin{Lemma}
	Let $\beta: U_q(\b) \to U_q(\b^-)^*, \beta(u)(v) = (u,v)$, and $\zeta: U_q(\g) \to
U_q(\g)^*, \zeta(u)(v) = \langle u,v \rangle$. Then $\beta$ and $\zeta$ are injective, and
$\beta$ is an anti-homomorphism of algebras.
\end{Lemma}
\begin{proof}
Ijectivity is an immediate consequence of the fact that both bilinear forms are
nondegenerate. Now given $u,u' \in U_q(\b)$, $\beta(uu')(v) = (uu',v) = (u',v_1)(u,v_2) =
[\beta(u') \cdot \beta(u)](v)$
\end{proof}

\paragraph
\label{bilinear-Lustig}
There is an algebra isomorphism $f:U_q(\n) \to U_q(\n^-)$. The bilinear form induced by
$$ U_q(\n) \times U_q(\n) \to U_q(\n)\times U_q(\n^-)\stackrel{(-,-)}{\to} \kk$$
coincides with the bilinear form defined by Lusztig in his book.
======================================
\paragraph
Let $L_i$ be the adjoint to the operator $u \mapsto F_i u$ for the bilinear form $(-,-)$. 
Then $L_i(u) = (u_2,F_i)u_1$. This is a quantum
derivation, and in fact for any homogeneous element of degree $\alpha \in Q$, $e_\alpha
\in U_q(\n)$ we have $L_i(e_\alpha u) = L_i(e_\alpha) u + q^{(\alpha,\alpha_i)}e_\alpha
L_i(u)$. Set $L_i^{(m)} = \frac{1}{[m]_{q_i}!}L_i^m$.
\begin{Lemma*}
We have the following equality $ \varepsilon_i(b) = \max\{m|L^{(m)}(b^*) \neq 0\}$.
Furthermore, $L_i^{(\varepsilon_i(b))}(b^*) = (\tilde e_i^{(\varepsilon_i(b))}(b))^*$
\end{Lemma*}




