%%%%%%%%%%%%%%%%%%%%%% Generalities %%%%%%%%%%%%%%%%%%5
\documentclass[11pt,fleqn]{article}
\usepackage[paper=a4paper]
  {geometry}

\pagestyle{plain}
\pagenumbering{arabic}
%%%%%%%%%%%%%%%%%%%%%%%%%%%%%%%%
\usepackage{notas}
\usepackage{tikz}
%%%%%%%%%%%%%%%%%%%%%%%%%%% The usual stuff%%%%%%%%%%%%%%%%%%%%%%%%%
\newcommand\NN{\mathbb N}
\newcommand\CC{\mathbb C}
\newcommand\QQ{\mathbb Q}
\newcommand\RR{\mathbb R}
\newcommand\ZZ{\mathbb Z}
\renewcommand\k{\Bbbk}

\newcommand\F{\mathcal F}
\newcommand\V{\mathcal V}
\newcommand\D{\mathcal D}
\renewcommand\H{\mathcal H}

\newcommand\maps{\longmapsto}
\newcommand\ot{\otimes}
\renewcommand\to{\longrightarrow}
\renewcommand\phi{\varphi}
\newcommand\Id{\mathsf{Id}}
\newcommand\im{\mathsf{im}}
\newcommand\coker{\mathsf{coker}}
%%%%%%%%%%%%%%%%%%%%%%%%% Specific notation %%%%%%%%%%%%%%%%%%%%%%%%%
\newcommand\g{\mathfrak g}
\newcommand\gl{\mathfrak{gl}}
\newcommand\gen{\mathsf{gen}}
\newcommand\std{\mathsf{std}}

\DeclareMathOperator\End{End}
%%%%%%%%%%%%%%%%%%%%%%%%%%%%%%%%%%%%%% TITLES %%%%%%%%%%%%%%%%%%%%%%%%%%%%%%
\title{Singular Gelfand Tsetlin modules over $\gl$ à la Zadunaisky}
\date{[singular-GT.tex]}
\author{Pablo Zadunaisky}
\begin{document}
\maketitle

The objective of these notes is to clear up the construction of Singular
Gelfand-Tsetlin modules due to Futorny, Grantcharov and Ramirez found in 
\cite{FGR}. 

\section{GT Tableaux and GT algebras}
References for all results can be found in \cite{FGR}*{section 3}.

\paragraph
Fix $n \in \NN_{\geq 2}$, and set $N = \frac{n(n+1)}{2}$. For each $m \in \NN$ 
set $U_m = U(\gl(m, \CC))$ and $Z_m \subset U_m$ its center, and set $U = 
U_n$. We get a chain of inclusions $U_1 \subset U_2 \subset \cdots \subset U_n$
induced by the maps sending standard generators $E_{i,j} \in \gl(k,\CC)$ to 
the corresponding $E_{i,j} \in \gl(k+1, \CC)$. 

\paragraph
\label{HC-morphism}
For each $m \in \NN$ the algebra $Z_m$ is a polynomial algebra on the 
generators
\[
	c_{m,k} = \sum_{(i_1, \ldots, i_k) \in [m]^k} E_{i_1,i_2} E_{i_2,i_3} 
		\cdots E_{i_k, i_1} \qquad \qquad 1 \leq k \leq m.
\]
Let $\Lambda_m = \CC[\lambda_{m,k} \mid 1 \leq k \leq m]$ be a polynomial
algebra in $m$ variables, and set
\[
	\gamma_{m,k} = \sum_{i = 1}^m (\lambda_{m,i}+m-1)^k 
	\prod_{i \neq j} \left( 1 - \frac{1}{\lambda_{m,i} - \lambda_{m,j}}\right).
\]
Although it is not obvious, the $\gamma_{m,k}$ are algebraically independent 
polynomials and they are invariant by the obvious action of the symmetric 
group $S_m$ on $\Lambda$; in fact $\Lambda_m^{S_m} = \CC[\gamma_{m,k} \mid 
1 \leq k \leq m]$, and there is an embedding $Z_m \to \Lambda$ given by 
$c_{m,k} \mapsto \gamma_{m,k}$. 

\paragraph
\label{GT-algebra}
We write $\Gamma = \CC[c_{m,k} \mid 1 \leq k \leq m \leq n] = 
\subset U$, which is the algebra generated by $\bigcup_{k=1}^n Z_k$. The 
$c_{m,k}$ are algebraically independent and hence this is isomorphic to a 
polynomial algebra in $N$ generators.

Let $\Lambda = \CC[\lambda_{i,j} \mid 1 \leq i \leq j \leq n]$ be a polynomial
algebra in $N$ variables. The group $S_m$ acts on $\Lambda$ 
permuting the variables $\lambda_{m,k}$ with $1 \leq k \leq m$ and fixing the 
rest. This induces an action of the group $G = S_n \times S_{n-1} \times 
\cdots \times S_1$ on $\Lambda$. Composing with the embedding from 
\ref{HC-morphism} we get that $\Gamma$ is isomorphic to $\Lambda^G$.

\note{I suppose this is just the Harish-Chandra morphism of each $Z_k$ pasted
together. Actually, is there \emph{a} HC morphism? In any case restriction to 
$Z_k$ gives an isomorphism with invariants in a polynomial algebra.}

\paragraph
\label{GT-tableaux}
Recall we have fixed $n \in \NN$ and set $N = \frac{n(n+1)}{2}$. A 
\newterm{Gelfand-Tsetlin 
tableau} of size $n$ is a vector $v \in \CC^{N}$, whose coordinates are 
indexed by $\{(i,j) \mid 1 \leq i \leq j \leq n\}$. We can represent this 
graphically as [INSERT IMAGE]. For ease of reference we will sometimes write 
\[
v = (v_{n,1}, v_{n,2}, \ldots v_{n,n} 
	\mid v_{n-1,1}, \ldots, v_{n-1,n-1} 
	\mid \cdots 
	\mid v_{1,1}).
\]

\paragraph
\label{various-tableau}
We say that a GT-tableau $v$ is 
\begin{itemize}
\item \newterm{integral} if $v \in \ZZ^{N}$. 

\item \newterm{standard} if for all $1 \leq i < j \leq k < n$ the difference 
$v_{k,i}-v_{k-1,i} \in \ZZ_{\geq 0}$ and $v_{k-1,i}-v_{k,i+1} \in \ZZ_{>0}$; 
we denote the set of standard tableaux by $\CC^N_\std$.

\item \newterm{generic} if for all $1 \leq i < j \leq k 
< n$ the difference $v_{k,i}-v_{k,j}$ is not in $\ZZ$; we denote the set of 
generic tableaux by $\CC^N_\gen$. 

\item \newterm{singular} if it is non-generic tableau, i.e. there is a pair of 
entries in the same line differing by an integer. If there is exactly one 
such pair then the tableau is called \newterm{1-singular}. Finally if two 
entries in the same row are equal the tableau is called \newterm{critical}.
\end{itemize}



\section{GT-modules}

\paragraph
\label{GT-formulas}
Set for each $1 \leq i \leq k \leq n$
\begin{align*}
p_{k,i}^\pm(\lambda) 
	&= \prod_{j = 1}^{k\pm1}(\lambda_{k,i} -\lambda_{k\pm1,j}); &
q_{k,i}(\lambda)
	&= \prod_{j \neq i} (\lambda_{k,i} - \lambda_{k,j}). \\
|\lambda|_k = \lambda_{k,1} + \lambda_{k,2} + \cdots \lambda_{k,k}.
\end{align*}
Notice that $p_{k,i}^\pm(v)$ depends only on the $k, k \pm 1$-th rows of 
$v \in \CC^N$, while $q_{k,i}(v)$ depends only on the $k$-th row.

\paragraph
Set $\ZZ^N_0 = \{v \in \ZZ^N \mid v_{n,i} = 0 \mbox{ for } 1 \leq i \leq n\}$
If $w$ is a GT tableau set 
\[
V(w) = \langle T(w+z) \mid z \in \ZZ^N_0 \rangle_\CC.
\]
The $T$ is just a symbol to remind us that we have a basis indexed by 
tableaux; in particular $T(v+z) \neq T(v) + T(z)$.

The following is a classical theorem which we quote without proof.
\begin{Theorem}[Gelfand-Tsetlin, '50]
\label{GT}
Let $\lambda = (\lambda_1, \ldots, \lambda_n)$ be a dominant weight of 
$\gl(n,\CC)$, and let 
\begin{align*}
V(\lambda) 
	&= \langle T(v) \mid v \in \CC^N_{\std} 
		\mbox{ and } v_{n,1} = \lambda_1, v_{n,2} = \lambda_2 - 1, \ldots, 
		v_{n,n} = \lambda_{n} - n +1
	\rangle_\CC
\end{align*}
(by convention, if $v$ is non-standard then $T(v) = 0$ in $V(\lambda)$).
Then
\begin{enumerate}[(a)]
\item \label{GT-structure}
$V(\lambda)$ is a $U$-module with the action of the canonical generators 
given by
\begin{align*}
E_{k,k+1} T(v) 
	&= - \sum_{i=1}^k \frac{p^+_{k,i}(v)}{q_{k,i}(v)} T(v + \delta^{k,i}) \\
E_{k+1,k} T(v) 
	&= \sum_{i=1}^k \frac{p^-_{k,i}(v)}{q_{k,i}(v)} T(v - \delta^{k,i}) \\
E_{k,k} T(v)
	&= (|v|_k - |v|_{k-1} + k -1) T(v)
\end{align*}


\item \label{GT-action} 
For each $1 \leq k \leq m \leq n$, we have $c_{m,k} T(v) 
= \gamma_{m,k}(v) T(v)$.

\item With this structure, $V(\lambda)$ is a finite dimensional representation
of maximal weight $\lambda$.
\end{enumerate}
\end{Theorem}

\paragraph
\label{GT-generic}
We will now reprove the following result due to Drozd, Futorny and Ovsienko.
\begin{Theorem*}[Futorny-Droz-Ovsyenko]
Let $w \in \CC^N$. If $w$ is generic then replacing $V(\lambda)$ with $V(w)$, 
items (\ref{GT-structure}) and (\ref{GT-action}) of \ref{GT} hold mutatis 
mutandis (in particular the condition on $v$ being standard is void).
\end{Theorem*}

We give a proof of this result. In order to do that we need a lemma.

\begin{Lemma*}
Let $z \in \ZZ^N_0$ and let $S_z = \CC^N_\std \cap (\CC^N_\std + z)$. Let $F \in 
\CC(\lambda_{k,i} \mid 1 \leq i \leq k \leq n)$ be a rational function which is
well defined in a set containing $S_z$. If $F(v) = 0$ for all $v \in S_z$, then
$F = 0$.
\end{Lemma*}
\begin{proof}
Explicitly, $v$ lies in $S_z$ if and only if
\begin{align*}
v_{k,i} - v_{k-1,i} - \max\{z_{k-1,i} - z_{k,i}, 0\} +1 &\in \ZZ_{> 0} \\
v_{k-1,i} - v_{k,i+1} - \max\{z_{k,i+1} - z_{k-1,i}\}&\in \ZZ_{>0} 
\end{align*}
We enumerate the coordinate functions of a tableau as follows: start by 
$\lambda_{n,n}$, which we denote by $x_1$; then, looking at all entries with 
second coordinate $n-1$, we enumerate them by taking $x_2 = \lambda_{n-1,n-1}$, then $x_3 = \lambda_{n,n-1}$; next 
we take the elements with second coordinate $n-2$ starting by $\lambda_{n-2,n-2}$ and 
moving in the northwest direction. Explicitly, setting $\phi(i,j) = (i-j+1) + 
\frac{(n-j)(n-j+1)}{2}$ we write $x_{\phi(i,j)} = \lambda_{i,j}$. The following figure
shows the enumeration corresponding to $n = 3$; two entries are joined by an edge if
and only if they appear in one of the inequalities defining $S_z$, and in all cases
the leftmost appears with a plus sing and the rightmost with a minus.

\begin{tikzpicture}
\node (31) at (0,3) {$\lambda_{3,1}$};
\node (32) at (2,3) {$\lambda_{3,2}$};
\node (33) at (4,3) {$\lambda_{3,3}$};
\node (21) at (1,2) {$\lambda_{2,1}$};
\node (22) at (3,2) {$\lambda_{2,2}$};
\node (11) at (2,1) {$\lambda_{1,1}$};

\node (31a) at (6,3) {$x_6$};
\node (32a) at (8,3) {$x_3$};
\node (33a) at (10,3) {$x_1$};
\node (21a) at (7,2) {$x_5$};
\node (22a) at (9,2) {$x_2$};
\node (11a) at (8,1) {$x_4$};

\draw (33) -- (22) -- (11)  (32) -- (21);
\draw (22) -- (32)  (11) -- (21) -- (31);

\draw (33a) -- (22a) -- (11a)  (32a) -- (21a);
\draw (22a) -- (32a)  (11a) -- (21a) -- (31a);
\end{tikzpicture}

With this enumeration, all the equations 
defining $S_z$ read as $x_r(v) - x_s(v) - t_{r,s} \in \ZZ_{>0}$ for some 
$t_{r,s} \in \ZZ$, and in each case $r > s$. Thus if we put $x_1(v) = 0$ then 
by induction for each $s = 1, \ldots, N$ there is always an $x_s(v)$ 
satisfying all equations involving $x_r$ with $r\leq s$, so in particular $S_z$
is not empty. 

For each $s \in [N]$, let $w_s \in \ZZ^N$ be the tableau such that $x_t(w_s) 
= 1$ if $t \geq s$, and $x_t(w_s) = 0$ if $t < s$. Notice that $w_s$ does not
lie in $S_0$, yet $v \in S_z$ implies $v + r w_s \in S_z$ for all $r \in \NN$. 
This is clear in the example below: we have put full edges between entries 
whose difference from right to left lies in $\ZZ_{>0}$, and dashed ones for 
entries whose difference from right to left lies in $\ZZ_{\geq 0}$. Adding 
both figures together, we can put a full edge in the result wherever there is 
a full edge in either of the summands.

\begin{figure}[h]
\centering
\begin{tikzpicture}
\node (31) at (0,3) {$v_{3,1}$};
\node (32) at (2,3) {$v_{3,2}$};
\node (33) at (4,3) {$v_{3,3}$};
\node (21) at (1,2) {$v_{2,1}$};
\node (22) at (3,2) {$v_{2,2}$};
\node (11) at (2,1) {$v_{1,1}$};

\node (+) at (5,2) {$+$};

\node (31a) at (6,3) {$r$};
\node (32a) at (8,3) {$r$};
\node (33a) at (10,3) {$0$};
\node (21a) at (7,2) {$r$};
\node (22a) at (9,2) {$0$};
\node (11a) at (8,1) {$r$};

\draw (33) -- (22) -- (11)  (32) -- (21);
\draw[dashed] (22) -- (32)  (11) -- (21) -- (31);

\draw[dashed] (33a) -- (22a) -- (11a)  (32a) -- (21a);
\draw[dashed] (22a) -- (32a)  (11a) -- (21a) -- (31a);
\end{tikzpicture}
\end{figure}

Let $f/g$ be the reduced expression of $F$; the hypothesis implies that if 
$v \in S_z$ then $f(v) = 0$ and $g(v) \neq 0$, and all we have to prove is 
that $f = 0$; we will do this by induction on the highest $t$ such that $x_t$ 
appears in $f$. If $f$ is a polynomial on $x_1$, then taking $v \in S_z$, our 
hypothesis implies that $f(v + rw_1) = 0$ for all $r \in \NN$, which can only 
happen if $f = 0$. Assuming the result holds for all numbers less than $t$, we 
write $f$ as a polynomial in $x_t$ with coefficients in the ring of polynomial 
functions on $x_1, \ldots, x_{t-1}$, so $f = f_0 + f_1 x_t + \cdots + f_l 
x_t^l$. Fix $v \in S_z$. For each $r \in \NN$ the value of $f_i(v + rw_1)$ is 
independent of $r$, since the entries below the $t$-th are fixed. Using the 
fact that $f(v + rw_s) = 0$ for all $r \in \NN$, we see that $f_i(v) = 0$; 
since $v$ was arbitrary, this shows that each $f_i$ is zero over $S_z$, which 
implies that $f_i = 0$ for all $i$ by the inductive hypothesis.
\end{proof}

\paragraph
Consider the action of $\ZZ^N$ over $\CC^N$, given by $v^z = v+z$. This induces
an action on $\Lambda$, given by $(z \cdot f)(\lambda) = f(\lambda^z)$.
This action extends naturally to the fraction field $K = 
\operatorname{Frac}(\Lambda) = \CC(\lambda_{k,i} \mid 1 \leq i \leq k \leq 
n)$. 

For the rest of this section let $\H$ be the infinite hyperplane arrangement 
in $\CC^N$ consisting of all hyperplanes defined by the equations 
$\lambda_{k,i} - \lambda_{k,j} - z$ for all $1 \leq i < j \leq k \leq n$ and 
all $z \in \ZZ$, and let $A \subset K$ be the algebra of rational functions 
whose poles are contained in $\H$. Notice that $A$ is stable by the action of
$\ZZ^N$.


\begin{Proposition}
Let $V_A$ be the $A$-module with basis $\{T(z) \mid z \in \ZZ^n_0\}$, and let 
$V_K = K \ot_A V_A$. We see $V_A$ as $A$-submodule of $V_K$ in the obvious way.

\begin{enumerate}[(a)]
\item \label{generic-GT-structure}
The vector space $V_K$ can be endowed with the structure of a $U$-module 
with the action of the canonical generators given by
\begin{align*}
E_{k,k+1} T(z) 
	&= - \sum_{i=1}^k \frac{p^+_{k,i}(\lambda^z)}{q_{k,i}(\lambda^z)} 
		T(z + \delta^{k,i}); \\
E_{k+1,k} T(z) 
	&= \sum_{i=1}^k \frac{p^-_{k,i}(\lambda^z)}{q_{k,i}(\lambda^z)} 
		T(z - \delta^{k,i}); \\
E_{k,k} T(z)
	&= (|\lambda|_k - |\lambda|_{k-1} + k -1) T(z).
\end{align*}

\item For each $1 \leq k \leq m \leq n$, we have $c_{m,k} T(z) = 
\gamma_{m,k}(\lambda^z) T(z)$.

\item The $A$-module $V_A$ is a sub $U$-module of $V_K$.
\end{enumerate}
\end{Proposition}

\begin{proof}
Let $F$ be the free $\CC$-algebra generated by $X_{k,k+1}, X_{k+1,k}$ for 
$1 \leq k < n$, and $X_{k,k}$ for $1 \leq k \leq n$; there is an obvious map
$\phi: F \to U$. The formulas in item \ref{generic-GT-structure} define an 
$F$-module structure on $V_K$, so there is an algebra map $F \to \End_K V_K$, 
and we must show that this map factors through $U$.

Let $a \in F$ be a monomial on length $r \geq 0$. Then for each $w \in \ZZ^N_0$
there exists a rational function $f_a(-,w) \in K$ such that
\[
	aT(z) = \sum_{w \in \ZZ^N_0} f_a(z,w) T(z+w)
\]
for all $z \in \ZZ^N_0$. Furthermore if the sum of the absolute values of the
entries of $w$ is greater than $r$ then $f_a(-,w) = 0$.

If $v \in \CC^N_\std$ then there exists a weight 
\end{proof}

\paragraph
Let $\Sigma = \{(k,i,j) \mid  1 \leq i < j \leq k \leq n \}$, and for each
$(i,j,k) \in \Sigma$ let $H_{i,j}^k = \lambda_{k,i} - \lambda_{k,j}$. We 
denote by $\F$ be the localization of $\Lambda$ at $\{H^k_{i,j} + r \mid 
(k,i,j) \in \Sigma, r \in \ZZ\}$; in other words $\F$ is the set of rational 
functions whose poles are contained in the integer translates of the
hyperplanes defined by these equations. Yet another way to see it is as 
follows: take the infinite hyperplane arrangement consisting of all 
hyperplanes $H_{i,j}^k$ and their integer translates; then $\F$ is the algebra
of rational functions which are smooth in the complement of this arrangement.
Notice that the functions appearing in the GT-formulas lie in $\F$.

\paragraph
\label{universal-generic-module}
Let $\V$ be the free $\F$-module with basis $\{G(z) \mid z \in 
T_{n-1}(\CC)\}$. We define endomorphisms 
\begin{align*}
	\Theta_{k,k+1} G(z) 
		&= - \sum_{i=1}^k 
		\frac{a_{k,i}^+(\lambda^z)}
		{q_{k,i}(\lambda^z)} 
		G(z + \delta^{k,i}) & 1 \leq k \leq n-1;\\
	\Theta_{k+1,k} G(z) 
		&= \sum_{i=1}^k 
		\frac{a_{k,i}^-(\lambda^z)}
		{q_{k,i}(\lambda^z)} 
		G(z - \delta^{k,i}) & 1 \leq k \leq n-1;\\
	\Theta_{k,k} G(z) 
		&= \left(
			\sum_{i=1}^k (\lambda^z_{k,i} + i -1) - 
			\sum_{i=1}^{k-1}(\lambda^z_{k-1,i} + i - 1) \right) 
			G(z) & 1 \leq k \leq n.
\end{align*}
where $\lambda^z_{i,j} = \lambda_{i,j} + z_{i,j}$.

Notice that each $v \in \CC^{\frac{n(n+1)}{2}}$ induces a one-dimensional 
representation of $\Lambda$, and that this extends to $\F$ if and only 
if $v$ is generic. Let $v \in \CC^{\frac{n(n+1)}{2}}_\gen$, and let $\CC_v$ be 
the one dimensional representation of $\F$ determined by $v$. Then there is an
obvious vector-space isomorphism $\CC_v \ot_\F \V \to V(T(v))$, given by
$1 \ot G(z) \mapsto T(v+z)$. Furthermore, the morphisms $\Id_{\CC_v} \ot_\F 
\Theta_{k,k+1}, \Id_{\CC_v} \ot_\F \Theta_{k+1,k}, \Id_{\CC_v} \ot_\F 
\Theta_{k,k}$ are precisely those which define the action of $E_{k,k+1}, 
E_{k+1,k}, E_{k,k}$, respectively. 

\begin{Proposition*}
The map $U \to \End_\F(\V)$ given by $E_{i,j} \mapsto \Theta_{i,j}$ is an 
algebra morphism, and so $\V$ is a $U$-module.
\end{Proposition*}
\begin{proof}
Let $X = \{X_{k,k+1}, X_{k+1,k}, X_{k,k}, X_{n,n} \mid 1 \leq k \leq n-1\}$, 
let $\pi: \CC \langle X \rangle \to U$ be the obvious map, and let $\phi: \CC 
\langle X\rangle \to \End_\F(\V)$ be given by $X_{i,j} \mapsto 
\Theta_{i,j}$. We 
need to show that $R \in \ker \pi$ implies $R \in \ker \phi$, i.e. that 
$R(e)G(z) = 0$ for all $z \in T_{n-1}(\CC)$.

Set $R(e)G(z) = \sum_{w \in T_{n-1}(\CC)} f_w G(w)$. Now for each $v \in 
\CC_\gen^N$ we get
\begin{align*}
0 = \Id_{\CC_v} \ot_\F R(e) (1 \ot_\F G(z)) = \sum_{w \in T_{n-1}(\CC)} 
f_w(v)(1 \ot_\F G(w))
\end{align*}
so $f_w(v) = 0$ for all $v \in \CC^N_\gen$. Thus $f_w = 0$.
\end{proof}

\section{$1$-singular GT-modules}

In the preceeding paragraph we built a universal generic GT-module which 
specialices to the generic module corresponding to any $v \in \CC^N_\gen$. In 
\cite{FGR} the authors cook up singular modules correspoding to a 
\sout{$1$-singular} $1$-critical vector $v$. We now try to re-do their 
construction by building similar universal $1$-singular ($1$-critical?)
modules. 

Fix $(k,i,j) \in \Sigma$, and let $\Sigma^* = \Sigma \setminus \{(k,i,j)\}$.
We write $\F^*$ for the subalgebra of $\F$ consisting of rational functions 
without poles in $H^k_{i,j}$. Since $H^k_{i,j} = \lambda_{k,i} - \lambda_{k,j}$
is \emph{not} invertible in $\F^*$, we set $\F_{i,j}^k = \F^*/H^k_{i,j}$, 
which is the algebra of rational functions over $H^k_{i,j}$ whose poles are in 
the complement of the infinite hyperplane arrangement formed by the integer 
translates of the hyperplanes $H^r_{s,t} \cap H^k_{i,j}$. Set $\pi: \F^* \to 
\F_{i,j}^k$ to be the quotient map.

Our objective is to build a universal $1$-singular module for all $v \in 
H^k_{i,j}$ such that $v_{k,i} = v_{k,j}$ by endowing the free $\F_{i,j}^k$
-module with basis $\{T(z) \mid z \in T_{n-1}(\ZZ)\}$, which we denote by
$\V_{i,j}^k$, with a $\gl(n,\CC)$-module structure. For all $v \in H^k_{i,j}$
there is a $1$-dimensional $\F_{i,j}^k$-module $\CC_v$ where $f \in 
\F_{i,j}^k$ acts by multiplication by $f(v)$, and so $\CC_v \ot_{\F_{i,j}^k} 
\V^k_{i,j}$ becomes a $\gl(n,\CC)$-module.

\paragraph
\label{lemma-tau}
Let $\tau = (i,j) \in S_k$, so $\tau(z)_{k,i} =  z_{k,j}, \tau(z)_{k,j} = 
z_{k,i}$, while all other coordinates coincide. Given $f \in \F$ we write
$f^\tau$ for $f^\tau(v) = f(\tau(v))$, and say that a function is $\tau$ 
invariant if $f = f^\tau$. Finally set $d: \F \to \F$ to be the derivation
$d = \frac12 \left( \frac{\partial}{\partial v_{k,i}} - 
\frac{\partial}{\partial v_{k,j}} \right)$.

\begin{Lemma*}
\begin{enumerate}
\item If $t \neq k,k-1$ then $a_{t,s}$ is $\tau$-invariant for all $s$. 
If $t \neq k,k+1$ then $b_{t,s}$ is $\tau$-invariant for all $s$. 
If $t \neq k$ then $q_{t,s}$ is $\tau$-invariant for all $s$. 

\item If $(t,s) \notin \{(k,i),(k,j)\}$ then $a_{t,s}(\lambda + z) + 
a_{t,s}(\lambda + \tau(z))$, $b_{t,s}(\lambda + z) + b_{t,s}(\lambda + 
\tau(z))$ and $c_{t,s}(\lambda + z)$ are $\tau$-invariant.

\item If $f = f^\tau$ then $d(f) = 0$.
\end{enumerate}
\end{Lemma*}
\begin{proof}
By definition
\begin{align*}
	a_{t,s}(\lambda + z) 
		&= \prod_{r = 1}^{t+1}(\lambda_{t,s}^z - 
			\lambda_{t+1,r}^z)
\end{align*}
In the conditions of $1.$ and $2.$, this is independent of the entries $(k,i),
(k,j)$ unless $t+1 = k$, and in that case
\begin{align*}
	a_{k-1,s}&(\lambda + z) + a_{k-1,s}(\lambda + \tau(z)) \\
	&= (
		(\lambda_{k-1,s}^z - \lambda_{k,i}^z)(\lambda_{k-1,s}^z - 
		\lambda_{k,j}^z)
		+
		(\lambda_{k-1,s}^{\tau(z)} - \lambda_{k,i}^{\tau(z)})
		(\lambda_{k-1,s}^{\tau(z)} - \lambda_{k,j}^{\tau(z)}))\\
	&\prod_{r \neq i,j}(\lambda_{k-1,s}^z - 
			\lambda_{k,r}^z).
\end{align*}
The terms in the product are independent of $\tau$, and the term outside of it 
is $\tau$ invariant. The proofs for $b_{t,s}, c_{t,s}$ is similar. Item $3.$ 
is simple.
\end{proof}



\paragraph
Denote by $\V_{i,j}^* \subset \V$ the $\F^*$-lattice generated by $\{G(z) \mid 
z \in T_{n-1}(\CC)\}$, so $\F \ot_{\F^*} \V_{i,j}^* = \V$, and set $\V_{i,j} = 
\F_{i,j} \ot_{\F^*} \V_{i,j}^*$. Thus $\V_{i,j}$ is the free $\F_{i,j}$-module 
with basis $\{G(z) \mid z \in T_{n-1}(\CC)\}$; notice the abuse of notation, 
where $G(z)$ may be in $\V, \V^*_{i,j}$ or $\V_{i,j}$. 

Set
\begin{align*}
S(z) 
	&= \frac{G(z) + G(\tau(z))}{2};
	&A(z)
	&= \frac{G(z) - G(\tau(z))}{2}.
\end{align*}
Once again these elements may belong to any of three spaces.

\paragraph
We want to describe the action of $\gl(n,\CC)$ over the vectors $S(z)$ and 
$A(z)$. First notice that
\begin{align*}
\Theta_{t,t+1} S(z)
	&= \Theta_{t,t+1} \frac{G(z) + G(\tau(z))}{2} \\
	&= \frac12 \sum_{s=1}^{t}
	\left(\frac{a^+_{t,s}(\lambda^z)}{q_{t,s}(\lambda^z)}G(z + \delta^{t,s}) +
	\frac{a^+_{t,s}(\lambda^{\tau (z)})}{q_{t,s}(\lambda^{\tau(z)})}G(\tau(z) 
	+ \delta^{t,s}) \right). \\
\Theta_{t+1,t} S(z)
	&= \Theta_{t+1,t} \frac{G(z) + G(\tau(z))}{2} \\
	&= \frac12 \sum_{s=1}^{t}
	\left(\frac{a^-_{t,s}(\lambda^z)}{q_{t,s}(\lambda^z)}G(z - \delta^{t,s}) +
	\frac{a^-_{t,s}(\lambda^{\tau (z)})}{q_{t,s}(\lambda^{\tau(z)})}G(\tau(z) 
	- \delta^{t,s}) \right). 
\end{align*}
Now using the fact that $G(z) = S(z) + A(z)$ we obtain
\begin{align*}
\Theta_{t,t+1} S(z)
	&= \frac12 \sum_{s=1}^{t}
	\frac{a^+_{t,s}(\lambda^z)}{q_{t,s}(\lambda^z)}S(z + \delta^{t,s}) +
	\frac{a^+_{t,s}(\lambda^z)}{q_{t,s}(\lambda^z)}A(z + \delta^{t,s}) \\
	&+\frac{a^+_{t,s}(\lambda^{\tau (z)})}{q_{t,s}(\lambda^{\tau(z)})}
	S(\tau(z) + \delta^{t,s}) +	\frac{a^+_{t,s}(\lambda^{\tau (z)})}{q_{t,s}
	(\lambda^{\tau(z)})} A(\tau(z) + \delta^{t,s}). 
\end{align*}
Finally, since $\tau(z) + \delta^{t,s} = \tau(z + \tau(\delta^{t,s}))$ and 
$S(\tau(z)) = S(z), A(\tau(z)) = -A(z)$ obtain
\begin{align*}
\Theta_{t,t+1} S(z)
	&= \frac12 \sum_{s=1}^{t}
	\frac{a^+_{t,s}(\lambda^z)}{q_{t,s}(\lambda^z)}S(z + \delta^{t,s}) +
	\frac{a^+_{t,s}(\lambda^z)}{q_{t,s}(\lambda^z)}A(z + \delta^{t,s}) \\
	&+\frac{a^+_{t,s}(\lambda^{\tau (z)})}{q_{t,s}(\lambda^{\tau(z)})}
	S(z + \tau(\delta^{t,s})) -	\frac{a^+_{t,s}(\lambda^{\tau (z)})}{q_{t,s}
	(\lambda^{\tau(z)})} A(z + \tau(\delta^{t,s})). 
\end{align*}
It follows that there exist functions $F_{t,s}^\pm(\lambda,z), 
G_{t,s}^\pm(\lambda,z)$ such that
\begin{align*}
\Theta_{t,t+1} S(z) 
	&= \sum_{s=1}^{t} F_{t,s}^+(\lambda, z) S(z + \delta^{t,s}) 
		+ G_{t,s}^+(\lambda, z) A(z + \delta^{t,s}). \\
\Theta_{t+1,t} S(z) 
	&= \sum_{s=1}^{t} F_{t,s}^-(\lambda, z) S(z - \delta^{t,s}) 
		+ G_{t,s}^-(\lambda, z) A(z - \delta^{t,s}).
\end{align*}

Furthermore, direct inspection shows that
\begin{align*}
\Theta_{t,t+1} A(z) 
	&= \sum_{s=1}^{t} F_{t,s}^+(\lambda, z) A(z + \delta^{t,s}) 
		+ G_{t,s}^+(\lambda, z) S(z + \delta^{t,s}). \\
\Theta_{t+1,t} S(z) 
	&= \sum_{s=1}^{t} F_{t,s}^-(\lambda, z) A(z - \delta^{t,s}) 
		+ G_{t,s}^-(\lambda, z) S(z - \delta^{t,s}).
\end{align*}


We refer to these functions as the coordinate functions.

\paragraph
When $t \neq k,k-1$ the functions involved are $\tau$-invariant, and 
$\tau(z) + \delta^{t,s} = \tau(z + \delta^{t,s})$, so we get
\begin{align*}
F_{t,s}^\pm(\lambda, z) &= \frac{a^\pm_{t,s}(\lambda^z)}{q_{t,s}(\lambda^z)}, &
G_{t,s}^\pm(\lambda, z) &= 0.
\end{align*}

If $t = k-1$ then using $G(z) = S(z) + A(z)$ and $G(\tau(z)) = S(z) - A(z)$,
and the fact that $q_{k-1,s}$ is independent of the $k$-th row, we get
\begin{align*}
F_{k-1,s}^\pm(\lambda,z) 
	&= \frac{a^\pm_{k-1,s}(\lambda^z) + a^\pm_{k-1,s}(\lambda^{\tau(z)})}{q_{k-1,s}
	(\lambda^z)} \\
G_{k-1,s}^\pm(\lambda, z)
	&= \frac{a^\pm_{k-1,s}(\lambda^z) - a^\pm_{k-1,s}(\lambda^{\tau(z)})}{q_{k-1,s}
	(\lambda^z)}  
\end{align*}

Finally let us consider the case $t = k$. In that case $q_{k,s}$ is 
$\tau$-invariant for $s \neq i,j$, so the same reasoning as above gives
\begin{align*}
F_{k,s}^\pm(\lambda,z) 
	&= \frac{a^\pm_{k,s}(\lambda^z) + a^\pm_{k,s}(\lambda^{\tau(z)})}{q_{k,s}
	(\lambda^z)} &
G_{k,s}^\pm(\lambda, z)
	&= \frac{a^\pm_{k,s}(\lambda^z) - a^\pm_{k,s}(\lambda^{\tau(z)})}{q_{k,s}
	(\lambda^z)}  & \mbox{for } s \neq i,j\\
F_{k,i}^\pm(\lambda,z)
	&= \frac{a^\pm_{k,i}(\lambda^z)}{q_{k,i}(\lambda^z)} + \frac{a^\pm_{k,j}
	(\lambda^{\tau(z)})}{q_{k,j}(\lambda^{\tau(z)})} &
G_{k,i}^\pm(\lambda,z)
	&= \frac{a^\pm_{k,i}(\lambda^z)}{q_{k,i}(\lambda^z)} - \frac{a^\pm_{k,j}
	(\lambda^{\tau(z)})}{q_{k,j}(\lambda^{\tau(z)})}\\
F_{k,j}^\pm 
	&= F_{k,i}^\pm(\lambda, \tau(z)) &
G_{k,j}^\pm
	&= G_{k,i}^\pm(\lambda, \tau(z))
\end{align*}

\begin{Lemma}
Let $z \in T_{n-1}(\CC)$.
\begin{enumerate}
\item If $z \neq \tau(z)$ then the functions 
$F^\pm_{t,s}(\lambda,z), G_{t,s}^\pm(\lambda, z)$ lie in $\F$ for all $1 
\leq s \leq t \leq n$.

\item If $z = \tau(z)$ then the functions $G_{k,i}^\pm(\lambda,z)
= G_{k,j}^\pm(\lambda,z)$ have a pole of order $1$, while the rest lie in $\F$.
\end{enumerate}
\end{Lemma}
\begin{proof}
By definition $H^k_{i,j} \mid q_{t,s}(\lambda^z)$ if and only if $(t,s) \in
\{(k,j), (k,i)\}$ and $z = \tau(z)$. Item 1 and the second part of item 2 
follow from this.

Assume $z = \tau(z)$. Then $H^k_{i,j}$ divides both $q_{k,i}(\lambda^z)$ and
$q_{k,j}(\lambda^z)$, so we need to show that $(H^k_{i,j})^2$ divides
$a^+_{k,i}(\lambda^z)q_{k,j}(\lambda^z) + a^+_{k,j}(\lambda^z)
q_{k,i}(\lambda^z)$. Now by definition this equals
\begin{align*}
(\lambda_{k,j} - \lambda _{k,i})\bigg(
	\prod_r (\lambda_{k,i}^z - \lambda_{k+1,r}^z) 
	\prod_{r \neq i,j}(\lambda_{k,j}^z - \lambda_{k,r}^z)
	- \prod_r (\lambda_{k,j}^z - \lambda_{k+1,r}^z) 
	\prod_{r \neq i,j}(\lambda_{k,i}^z - \lambda_{k,r}^z)\bigg).
\end{align*}
If we assume $\lambda_{k,i} = \lambda_{k,j}$ then the two summands inside the
parentheses are equal, and hence the polynomial is divisible by $(\lambda_{k,i}
-\lambda_{k,j})$, which completes the proof for $F_{k,i}^+$. Since under the 
hypothesis $F_{k,i}^+(\lambda,z) = F_{k,j}^+(\lambda,z)$, we are halfway done.

For $G_{k,i}^+$, the denominator of the resulting function is
\begin{align*}
(\lambda_{k,j} - \lambda _{k,i})\bigg(
	\prod_r (\lambda_{k,i}^z - \lambda_{k+1,r}^z) 
	\prod_{r \neq i,j}(\lambda_{k,j}^z - \lambda_{k,r}^z)
	+ \prod_r (\lambda_{k,j}^z - \lambda_{k+1,r}^z) 
	\prod_{r \neq i,j}(\lambda_{k,i}^z - \lambda_{k,r}^z)\bigg).
\end{align*}
If we assume $\lambda_{k,i} = \lambda_{k,j}$ the polynomial in parentheses 
reduces to twice the product of linear polynomials, none of them zero. A 
similar reasoning works for $G_{k,j}^+$, so we are finished.
\end{proof}

\paragraph
We define a $\CC$-linear map $D: \V_{i,j}^* \to \V_{i,j}$ as follows. First 
let $d: \F \to \F$ be the derivation $d = \frac12 \left( 
\frac{\partial}{\partial v_{k,i}} - \frac{\partial}{\partial v_{k,j}}\right)$, 
which clearly restricts to a derivation $d: \F^* \to \F^*$. Now set 
$D(fG(z)) = \pi(d(f)) S(z) + \pi(f) A(z)$, or equivalently
\begin{align*}
D(fS(z)) & = \pi(d(f)) S(z) & D(fA(z)) &= \pi(f) A(z).
\end{align*}

\begin{Lemma*}
\label{lemma-D-Theta}

Let $z \in T_{n-1}(\CC)$. Then for all $1 \leq r,s \leq n$ the following 
equalities hold
\begin{enumerate}
\item $D((v_{k,i} - v_{k,j}) \Theta_{r,s} G(z)) 
		= D((v_{k,i} - v_{k,j}) \Theta_{r,s} G(\tau(z) ))$.
\item $D(\Theta_{r,s}G(z)) = - D(\Theta_{r,s}G(\tau(z)))$
\end{enumerate}
\end{Lemma*}
\begin{proof}
See \cite{FGR}*{Proposition 4.7}. Proof is long and ugly.
\end{proof}

\paragraph
\label{1-singular-action}
For each $1 \leq r,s \leq n$ and $z \in T_{n-1}(\CC)$, set
\begin{align*}
\Omega_{r,s} S(z) 
	&= D((v_{k,i} - v_{k,j}) \Theta_{r,s} G(z)) = D((v_{k,i} - v_{k,j}) 
	\Theta_{r,s} S(z)) \\
\Omega_{r,s} A(z)
	&= D(\Theta_{r,s}G(z)) = D(\Theta_{r,s}A(z))
\end{align*}
By the previous lemma these are well defined $\F_{i,j}$-linear endomorphisms of
$\V_{i,j}$.

\begin{Proposition*}
The map $E_{r,s} \in \gl(n,\CC) \mapsto \Omega_{r,s} \in \End_{\F_{i,j}}
(\V_{i,j})$ defines a $\gl(n,\CC)$-module structure on $V_{i,j}$.
\end{Proposition*}

IDEA: Rewrite the action of the $\Theta_{r,s}, \Omega_{r,s}$ in terms of $S(z), 
A(z)$. If everything works out then Lemma \ref{lemma-D-Theta} and with luck
Proposition \ref{1-singular-action} should be easier to prove than in the 
paper. 


\newpage
\begin{bibdiv}
\begin{biblist}
\bib{FGR}{article}{
   author={Futorny, Vyacheslav},
   author={Grantcharov, Dimitar},
   author={Ramirez, Luis Enrique},
   title={Singular Gelfand-Tsetlin modules of ${\germ{gl}}(n)$},
   journal={Adv. Math.},
   volume={290},
   date={2016},
   pages={453--482},
}

\end{biblist}
\end{bibdiv}
\end{document}
