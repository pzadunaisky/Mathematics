\documentclass[11pt,fleqn]{article}
\usepackage[paper=a4paper]
  {geometry}

\pagestyle{plain}
\pagenumbering{arabic}
%%%%%%%%%%%%%%%%%%%%%%%%%%%%%%%%
\usepackage{notas}
\usepackage{tikz}
%%%%%%%%%%%%%%%%%%%%%%%%%%% The usual stuff%%%%%%%%%%%%%%%%%%%%%%%%%
\newcommand\NN{\mathbb N}
\newcommand\CC{\mathbb C}
\newcommand\QQ{\mathbb Q}
\newcommand\RR{\mathbb R}
\newcommand\ZZ{\mathbb Z}
\renewcommand\k{\Bbbk}

\newcommand\F{\mathcal F}
\newcommand\V{\mathcal V}
\newcommand\D{\mathcal D}
\renewcommand\H{\mathcal H}

\newcommand\maps{\longmapsto}
\newcommand\ot{\otimes}
\renewcommand\to{\longrightarrow}
\renewcommand\phi{\varphi}
\newcommand\Id{\mathsf{Id}}
\newcommand\im{\mathsf{im}}
\newcommand\coker{\mathsf{coker}}
%%%%%%%%%%%%%%%%%%%%%%%%% Specific notation %%%%%%%%%%%%%%%%%%%%%%%%%
\newcommand\g{\mathfrak g}
\newcommand\gl{\mathfrak{gl}}
\newcommand\gen{\mathsf{gen}}
\newcommand\std{\mathsf{std}}

\DeclareMathOperator\End{End}
%%%%%%%%%%%%%%%%%%%%%%%%%%%%%%%%%%%%%% TITLES %%%%%%%%%%%%%%%%%%%%%%%%%%%%%%
\title{Singular Gelfand Tsetlin modules over $\gl$}
\date{[singular-GT-summary.tex]}
\author{Pablo Zadunaisky}
\begin{document}
\maketitle

All notation is borrowed from \cite{FGR}. 

\section{Summary}

\paragraph
Fix $n \in \NN$. Set $N = \frac{n(n+1)}{2}$, and write $v \in \CC^N$ as $v = 
(v_{k,j} \mid 1 \leq j \leq k \leq n)$. We also write $\ZZ^N_0$ for the set of 
all $z \in \CC^N$ such that $z_{n,i} = 0$ for all $1 \leq i \leq n$.
Let $\CC(\lambda) = \CC(\lambda_{k,j} \mid 1 \leq j \leq k \leq n)$. For each
$p \in \CC(\lambda)$ and $z \in \ZZ^N$ set $p(\lambda^z)$ for $p(\lambda + z)$.
Let $U = U(\gl(n,\CC))$.

Write $\Sigma = \{(k,i,j) \mid 1 \leq i < j \leq k \leq n\}$. Let 
$A \subset \CC(\lambda)$ be the algebra of rational functions of the form 
$f/g$ with $g^{-1}(0)$ contained in $\displaystyle \bigcup_{(k,i,j) \in 
\Sigma, z \in \ZZ} \lambda_{k,i} - \lambda_{k,j} - z$.

Set for each $1 \leq i \leq k \leq n$
\begin{align*}
p_{k,i}^\pm(\lambda) 
	&= \prod_{j = 1}^{k\pm1}(\lambda_{k,i} -\lambda_{k\pm1,j}); &
q_{k,i}(\lambda)
	&= \prod_{j \neq i} (\lambda_{k,i} - \lambda_{k,j}). \\
|\lambda|_k &= \lambda_{k,1} + \lambda_{k,2} + \cdots \lambda_{k,k}.
\end{align*}

\begin{Theorem*}
Let $V_A$ be the free $A$-module generated by $\{T(z) \mid z \in \ZZ^N_0\}$.
\begin{enumerate}[(a)]
\item 
The module $V_A$ can be endowed with the structure of a $U$-module 
with the action of the canonical generators given by
\begin{align*}
E_{k,k+1} T(z) 
	&= - \sum_{i=1}^k \frac{p^+_{k,i}(\lambda^z)}{q_{k,i}(\lambda^z)} 
		T(z + \delta^{k,i}); \\
E_{k+1,k} T(z) 
	&= \sum_{i=1}^k \frac{p^-_{k,i}(\lambda^z)}{q_{k,i}(\lambda^z)} 
		T(z - \delta^{k,i}); \\
E_{k,k} T(z)
	&= (|\lambda^z|_k - |\lambda^z|_{k-1} + k -1) T(z).
\end{align*}

\item For each $1 \leq k \leq m \leq n$, we have $c_{m,k} T(z) = 
\gamma_{m,k}(\lambda^z) T(z)$.
\end{enumerate}
\end{Theorem*}

\paragraph
Let $v \in \CC^N_\gen$. Then there is a one dimensional $A$-module $\CC_v$,
given by $1 \cdot f/g = f(v)/g(v)$. Now the space $\CC_v \ot_A V_A$ inherits 
a $U$-module structure, and it is isomorphic to the generic GT-module that
contains the tableau $v$. The element $1 \ot_A T(z)$ corresponds to the usual 
tableau $T(v+z)$.

\paragraph
Let $(k,i,j) \in \Sigma$, and let $\tau$ be the linera transformation that 
exchanges coordinates $(k,i)$ and $(k,j)$, leaving the rest fixed. Also write
$x = \lambda_{k,i}, y = \lambda_{k,j}$. Denote $A^k_{i,j} \subset A$ the 
rational functions in $A$ without poles in the hyperplane $x-y = 0$. 

Write $V^k_{i,j} \subset V_A$ for the $A^k_{i,j}$-module generated by the 
elements
\begin{align*}
S_\tau(z) 
	&= \frac{T(z) + T(\tau(z))}{2}
&A_\tau(z) 
	&= \frac{T(z) - T(\tau(z))}{2(x-y)}
\end{align*}
Notice that $T(z) = S_\tau(z) + (x-y) A_\tau(z) \in V^k_{i,j}$ for all $z \in 
\ZZ_0^N$.

Then $V^k_{i,j}$ is a $U$-submodule of $V_A$. Furthermore, for each $v \in 
\CC^N$ which is $1$-critical in coordinates $(k,i)$ and $(k,j)$, there is a
$1$-dimensional representation of $V_{i,j}^k$ which we denote by $\CC_v$.

\begin{Proposition*}
The $1$-critical GT-module corresponding to $v$ is isomorphic to $\CC_v 
\ot_{A^k_{i,j}} V^k_{i,j}$, with $1 \ot S_\tau(z)$ corresponding to $T(v+z)$ 
and $1 \ot A_\tau(z)$ corresponding to $DT(v + z)$.
\end{Proposition*}

\paragraph
Let $v \in \CC^N$ be critical in a subset $\Psi \subset \Sigma$. 
Take $V = \bigcap_{(k,i,j) \in \Psi} V^k_{i,j}$. Since this is an 
intersection of $U$-submodules of $V_A$, it is again a $U$-submodule.
It is also a module over the algebra $B = \bigcap_\Psi A^k_{i,j}$ of rational 
functions contained in $A$ without poles in the hyperplanes of the form 
$\lambda_{k.i} - \lambda_{k,j}$ where $(k,i),(k,j) \in \Psi$. 

\begin{Proposition*}
The tableau $v$ defines a one dimensional $B$-module $\CC_v$ and we get a 
$U$-module by taking $\CC_v \ot_B V$. 
\end{Proposition*}

If there are several critical sets, the argument can be suitably modified.
In particular we could take $\Sigma$ instead of $\Psi$, and define a $U$-module
for an arbitrary fully critical tableau $v$.

\paragraph
\about{Example}
Suppose that $\{(k,i),(k,j)\} \cap \{(t,r),(t,s)\} = \emptyset$. Set $V = 
V^k_{i,j} \cap V^t_{r,s}$. Set also $B = A^k_{i,j} \cap A^t_{r,s}$. 
Write $\rho$ for the permutation corresponding to $(t,r,s)$, and $a = 
\lambda_{t,r}, b = \lambda_{t,s}$. A $B$-basis of $V$ is
\begin{align*}
&\frac{T(z) + T(\tau(z)) + T(\rho(z)) + T(\rho\tau(z))}{2}
& \frac{T(z) - T(\tau(z)) + T(\rho(z)) - T(\rho\tau(z))}{2(x-y)} \\
&\frac{T(z) + T(\tau(z)) - T(\rho(z)) - T(\rho\tau(z))}{2(a-b)}
& \frac{T(z) - T(\tau(z)) - T(\rho(z)) + T(\rho\tau(z))}{2(x-y)(a-b)}
\end{align*}
with $z \in \ZZ^N_0$.
I have not checked whether this coincides with the singular module that 
appears in your last paper, but whether it does or it does not, both should be 
good news. In any case the first element which is completley symmetric will
be an eigenvector for all the elements of $\Gamma$ \emph{once we reduce modulo
$(x-y)(a-b)$}.

\paragraph
\about{Question.}
Can we find a $A^k_{i,j} \cap A^k_{i,l} \cap A^k_{l,j}$-basis for $V^k_{i,j} 
\cap V^k_{i,l} \cap V^k_{l,j}$? Can it be a GT-basis modulo $(x-y)(y-z)(x-z)$?
Is it given by the generators of the irreducible submodules of the 
$S_3$-module generated by $T(z)$?


\begin{bibdiv}
\begin{biblist}
\bib{FGR}{article}{
   author={Futorny, Vyacheslav},
   author={Grantcharov, Dimitar},
   author={Ramirez, Luis Enrique},
   title={Singular Gelfand-Tsetlin modules of ${\germ{gl}}(n)$},
   journal={Adv. Math.},
   volume={290},
   date={2016},
   pages={453--482},
}

\end{biblist}
\end{bibdiv}

\end{document}