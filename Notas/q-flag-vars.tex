%%%%%%%%%%%%%%%%%%%%% Generalities %%%%%%%%%%%%%%%%%%
\documentclass[11pt,fleqn]{article}

\usepackage[paper=a4paper]
  {geometry}

\pagestyle{plain}
\pagenumbering{arabic}
%\linespread{1.2}
\setlength{\parskip}{1.1ex}

\usepackage[latin1,utf8]{inputenc}
\usepackage[spanish,english]{babel}
\usepackage{euler}
\usepackage[osf,noBBpl]{mathpazo}
\usepackage{enumerate}
\usepackage[alphabetic,initials]{amsrefs}
\usepackage{hyperref}
\usepackage{amsfonts,amssymb,amsmath,amsthm}
\usepackage{stmaryrd}
\usepackage{mathtools}
\usepackage{graphicx}
\usepackage[poly,arrow,curve,matrix]{xy}
\usepackage{wrapfig}
\usepackage{xcomment}
%\swapnumbers
\hyphenation{ca-te-go-ry gene-rated}

%%%%%%%%%%%%%%%%%%%%%%%%%%%%%%%%%% Theorems et al. %%%%%%%%%%%%%%%%%%%%%%%%%%%

\theoremstyle{plain}
\newtheorem{Theorem}{Theorem}[section]
\newtheorem{Proposition}[Theorem]{Proposition}
\newtheorem{Excercise}[Theorem]{Excercise}
\newtheorem{Lemma}[Theorem]{Lemma}
\newtheorem{Corollary}[Theorem]{Corollary}

\newtheorem*{Theorem*}{Theorem}
\newtheorem*{Proposition*}{Proposition}
\newtheorem*{Excercise*}{Excercise}
\newtheorem*{Lemma*}{Lemma}
\newtheorem*{Corollary*}{Corollary}


\theoremstyle{remark}
\newtheorem{Remark}[Theorem]{Remark}
\newtheorem{Example}[Theorem]{Example}

\theoremstyle{definition}
\newtheorem{Definition}[Theorem]{Definition}
\renewcommand\proofname{proof}

\newcommand\note[1]{\marginpar{{
\begin{flushleft}
\footnotesize#1
\end{flushleft}
}}}
\renewcommand\labelitemi{-}

%%%%%%%%%%%%%%%%%%%%%%%%%%% The usual stuff%%%%%%%%%%%%%%%%%%%%%%%%%
\newcommand\NN{\mathbb N}
\newcommand\CC{\mathbb C}
\newcommand\QQ{\mathbb Q}
\newcommand\RR{\mathbb R}
\newcommand\ZZ{\mathbb Z}

\newcommand\maps{\longmapsto}
\newcommand\ot{\otimes}
\renewcommand\to{\longrightarrow}
\renewcommand\phi{\varphi}
\newcommand\stack[2]{\genfrac{}{}{0pt}{2}{#1}{#2}}
\newcommand\lin[1]{\left\langle #1 \right\rangle}


\newcommand\good{good }
%%%%%%%%%%%%%%%%%%%%%%%%% Specific notation %%%%%%%%%%%%%%%%%%%%%%%%%
\newcommand\A{A}
\newcommand\D{\mathcal D}
\newcommand\B{\mathcal B}
\renewcommand\S{\mathcal S}
\newcommand\F{\mathcal F}
\newcommand\I{\mathcal I}
\newcommand\GG{\Gamma}
\renewcommand\O{\mathcal O}
\newcommand\R{\mathcal R}
\newcommand\G{\mathcal G}
\newcommand\U{\mathcal U}
\renewcommand\a{\mathfrak a}
\renewcommand\b{\mathfrak b}
\newcommand\g{\mathfrak g}
\newcommand\p{p}
\newcommand\m{\mathfrak m}
\newcommand\n{\mathfrak n}
\newcommand\opp{\mathsf{opp}}
\newcommand\pdim{\mathsf{pdim}}

\DeclareMathOperator\im{Im}
\DeclareMathOperator\Ab{\mathsf{Ab}}
\DeclareMathOperator\Mod{\mathsf{Mod}}
\DeclareMathOperator\Vect{\mathsf{Vect}}
\DeclareMathOperator\Hom{\mathsf{Hom}}
\DeclareMathOperator\Ext{\mathsf{Ext}}
\DeclareMathOperator\Tor{\mathsf{Tor}}
\DeclareMathOperator\Gr{\mathsf{GrMod}}
\DeclareMathOperator\grmod{\mathsf{grmod}}
\DeclareMathOperator\GrHom{\underline{\mathsf{Hom}}}
\DeclareMathOperator\GrExt{\underline{\mathsf{Ext}}}
\DeclareMathOperator\GrTor{\underline{\mathsf{Tor}}}

\DeclareMathOperator\id{\mathsf{id}}
\DeclareMathOperator\pd{\mathsf{pd}}
\DeclareMathOperator\gr{\mathsf{gr}}
\DeclareMathOperator\ldim{\mathsf{ldim}}
\DeclareMathOperator\injdim{\mathsf{injdim}}
\DeclareMathOperator\st{\mathsf{st}}
\DeclareMathOperator\depth{\mathsf{depth}}
\DeclareMathOperator\Spec{Spec}
\DeclareMathOperator\supp{supp}
\DeclareMathOperator\Id{Id}
\DeclareMathOperator\rk{rk}
\DeclareMathOperator\GKdim{\mathsf{GKdim}}

\title{Quantum Schubert varieties are AS-Cohen Macaulay}
\author{[q-flag-vars.tex]}
\date{6/2/2013}
\xcomment{Theorem,Lemma,Corollary,Proposition,section}
\begin{document}
\maketitle
In this note we prove that quantum flag varieties and their Schubert cells
are AS Cohen Macaulay (see Definition \ref{as-regularity}). The plan of the proof
is as follows: first we show that if a connected graded algebra $A$ has a
filtration such that its associated graded ring is graded connected and AS Cohen
Macaulay, then the same holds for $A$. Afterwards we introduce a family of algebras 
which can be equipped with such a filtration. This family includes quantum flag varieties
and their Schubert cells.

Throughout this document $k$ denotes a field and $A$ a $k$-algebra. All ideals will be
two-sided ideals unless specified. Unadorned tensor products are tensor products over $k$.

\section{Graded algebras}
In this section we present some technical results on graded $k$-algebras that also have a
filtration by graded subspaces. For general reference and defintions on $\ZZ$-graded and
filtered rings see \cite{VO}. For rings graded by arbitrary groups see \cite{NV}.

Let $n \in \NN_0$ and let $A$ be an $\NN_0^n$-graded $k$-algebra. The algebra $A$ is said 
to be \emph{connected} if its component of degree $(0, \ldots, 0)$ is equal to $k$. We
denote by $\Gr A$ the category whose objects are $\ZZ^n$-graded $A$-modules and whose
morphisms are homogeneous $A$-linear morphisms of degree $(0, \ldots, 0)$. We denote by 
$\grmod A$ the full subcategory of $\Gr A$ whose objects are finitely generated graded
modules. Given two $\ZZ^n$-graded $A$-modules $N$ and $M$, we denote by $\GrHom_A(N,M)$
the $\ZZ^n$-graded vector space of homogeneous morphisms from $N$ to $M$ of arbitrary
degree. For every $i \in \NN_0$ we denote by $\GrExt^i$ the $i$-th derived functor of
$\GrHom_A$. For every $d \in \NN_0^n$ we denote by $M_d$ the $k$-vector subspace of $M$ 
formed by its homogeneous elements of degree $d$, and for every element $m \in M$ we
denote by $m_d$ its homogeneous component of degree $d$. 

A $\ZZ^n$-graded $A$-module $F$ is said to be \emph{graded-free} if it is free over $A$ and 
has a basis
consisting of homogeneous elements. The same argument as in the ungraded case shows that 
any projective object in $\Gr A$ is a direct summand of a graded-free object. Since
graded-free modules are free as $A$-modules, projective objects in $\Gr A$ are projective
$A$-modules.

The following is a standard result which we include for completeness.
\begin{Proposition}\cite{NV}*{Corollary 2.4.4}
\label{grhom-hom}
	Given $n \in \NN_0$, let $A$ be an $\NN_0^n$-graded algebra and let $N, M$ be
	$\ZZ^n$-graded $A$-modules, with $N$ finitely generated. Then
	$$ \GrHom_A(N,M) = \Hom_A(N,M).$$
	If $A$ is noetherian, then for all $i \geq 0$ there is a natural $k$-vector 
	space isomorphism
	$$ \GrExt_A^i(N,M) \cong \Ext_A^i(N,M).$$
\end{Proposition}
\begin{proof}
	Let $f \in \Hom_A(N,M)$, and for each $d,e \in \ZZ^n$ and $x \in N_{e}$ set
	$f_d(x) = f(x)_{e+d}$. This defines a homogeneous $A$-linear morphism $f_d: 
	N \to M$ of degree $d$. For every homogeneous element $x \in N$ there are finitely
	many $d$'s such that $f_d(x) \neq 0$, and $f(x) = \sum_{d \in \ZZ^n}f_d(x)$.
	
	Let $n_1, \ldots, n_r$ be a finite set of homogeneous generators of $N$, and let
	$D$ be the set of $d \in \ZZ^n$ such that there exists $1 \leq i \leq r$ for which
	$f_d(n_i) \neq 0$. The set $D$ is finite, and
	\begin{align*}
		f(n_i) &= \sum_{d \in D} f_d(n_i) & \mbox{ for all } 1 \leq i \leq r 
	\end{align*}
	which implies that $f = \sum_D f_d \in \GrHom_A(N,M)$.

	Now suppose $A$ is noetherian and let $P_\bullet \to N$ be a projective resolution
	of $N$ in $\Gr A$. Since projective objects of $\Gr A$ are also projective in 
	$\Mod A$, $P_\bullet$ is also a projective resolution of $N$ in $\Mod A$. By the first
	part of the proposition, $\GrHom_A (P_\bullet,M) = \Hom_A(P_\bullet,M)$ and hence 
	the homologies of these	two complexes coincide. 
\end{proof}

The category $\Gr A$ has arbitrary direct sums and products, and hence arbitrary direct
and inverse limits. If $\{M^i\}_{i \in \NN_0}$ is directed system in $\Gr A$, the
underlying $A$-module of its direct limit is equal to the limit of the underlying
$A$-modules. In a more categorical language, the  forgetful functor $\O: \Gr A \to
\Mod A$ commutes with direct limits. In particular, direct limits are exact over $\Gr A$.

We recall that for evey object $M$ in $\Gr A$, its graded projective dimension, denoted by
$\gr \pdim~M$, is the minimal length of a projective resolution of $M$ in $\Gr A$. The
graded injective dimension of $M$, denoted by $\gr \injdim~M$, is defined analogously. The
\emph{graded global dimension} of $A$ is the supremum of all projective dimensions of
finitely generated graded $A$-modules. If $A$ is a noetherian connected algebra, then its
global dimension is equal to the graded projective dimension of the trivial module $k$,
see \cite{RZ2}.

\subsection{Local cohomology and homological regularity}
Let $r \in \NN_0$. Throughout this section $A$ denotes a noetherian connected 
$\NN_0^r$-graded algebra. When talking about $A$-modules we will write ``graded'' instead
of ``$\ZZ^r$-graded''.
We denote by $\m$ the ideal generated by the elements of nonzero degree of $A$. Notice
that this is the only maximal graded ideal of $A$.

The torsion functor $\GG_\m: \Mod A \to \Mod A$ is defined as 
\begin{align*}
\GG_\m(M) = \{m \in M \mid \m^n m = 0 \mbox{ for } n \gg 0\} \cong \varinjlim_n
\Hom_A(A/\m^n,M)
\end{align*}
on objects and by restriction and corestriction on morphisms. This functor
is left exact so we can consider its right derived functor $\R \GG_\m : \D(\Mod A) \to
\D(\Mod A)$.

For every $i \in \NN_0$ the $i$-th \emph{local cohomology functor} of $A$ is $H_\m^i :=
\R^i \GG_\m$. Since direct limits are exact in $\Mod A$, there is a natural isomorphism
\begin{align*}
H_\m^i \cong \varinjlim_n \Ext_A^i(A/\m^n,-).
\end{align*}
We denote by $\GG_{\m^\opp}$ and $H_{\m^\opp}^i$ the torsion and local cohomology
functors of the opposite algebra $A^\opp$, respectively.

Let $M$ be a graded $A$-module. Since $\m$ is a graded ideal, $\GG_\m(M) \subset M$ is a
graded submodule. Thus the torsion functor induces a graded torsion functor $\GG_\m^{\gr}:
\Gr A \to \Gr A$. By the same argument as in the ungraded case, there is a natural
isomorphism
\begin{align*}
\GG_\m^{\gr}(M) \cong \varinjlim_n \GrHom_A(A/\m^n, M).
\end{align*}
The following lemma aims to clarify the relation between the derived functors of the
graded torsion functor and their ungraded counterparts.
\begin{Lemma}
\label{graded-torsion}
  Let $\O: \Gr A \to \Mod A$ be the  forgetful functor. Then for every $i \geq 0$
  the following diagram commutes
  \begin{align*}
  \xymatrix{
    \Gr A \ar[r]^-{\R^i \GG_\m^{\gr}} \ar[d]_-\O
      & \Gr A \ar[d]^-\O \\
    \Mod A \ar[r]^-{H^i_\m} 
      & \Mod A
  }
  \end{align*}
\end{Lemma}
\begin{proof}
  Recall that direct limits are exact in $\Gr A$, so there is a natural isomorphism
  \begin{align*}
  \R^i \GG_\m^{\gr} &\cong \varinjlim_n \GrExt^i_A(A/\m^n,-)
  \end{align*}
  Since the forgetful functor $\O$ commutes with direct limits and $A/\m^n$ is finitely
  generated for every $n \in \NN_0$, Proposition \ref{grhom-hom} implies that for every $i \geq
  0$ there is a natural isomorphism
  \begin{align*}
    \O(\varinjlim_n \GrExt^i_A(A/\m^n,M)) \cong \varinjlim_n \Ext^i_A(A/\m^n,\O(M)).
  \end{align*}
  This completes the proof.
\end{proof}
In view of Lemma \ref{graded-torsion}, we make a slight abuse of notation and write
$\GG_\m$ for the graded torsion functor and $H_\m^i$ for its $i$-th derived functor. The
context will always make it clear if we are considering the graded or ungraded versions.

Suppose $A$ is commutative of finite Krull dimension $n$. By \cite{BH}*{Excercise 2.1.27},
$A$ is Cohen Macaulay if and only if the  local algebra $A_\m$ is Cohen Macaulay. By 
Grothendieck's vanishing theorem \cite{BH}*{Theorem 3.5.7} and \cite{BH}*{Remark 3.6.18},
the local algebra $A_\m$ is Cohen Macaulay if and only if the local cohomology modules
$H^i_\m (A)$ are zero except for $i = n$. This equivalence can be used to transfer the
notion of being Cohen Macaulay to the  noncommutative setting, where it is known as AS
Cohen Macaulay. The following defintions are taken from the introduction of \cite{JZ}.
\begin{Definition} 
\label{as-regularity}
Let $A$ be a noetherian connected $\NN_0^r$-graded algebra.
\begin{enumerate}
  \item $A$ is called \emph{AS Cohen Macaulay} if there exists an $n \in \NN_0$ such that
  \begin{align*}
    H^i_{\m}(A) &= H_{\m^\opp}^i(A) = 0 &\mbox{for all } i \neq n.
  \end{align*}
	
  \item $A$ is called \emph{AS Gorenstein} if there exist $n \in \NN_0$ and $\ell \in
  \ZZ^r$ such that $\gr \injdim A= \gr \injdim A^\opp = n$ and
  \begin{align*}
    \GrExt_{A}^i(k,A) \cong \GrExt_{A^\opp}^i(k,A^\opp) 
      &\cong \begin{cases} k[\ell] & \mbox{ for } i = n \\ 0 & \mbox{ for } i
		\neq n,\end{cases}
  \end{align*}
  where the isomorphisms are of graded bimodules. Notice that it is enough to check that
  they are vector space isomorphisms.

  \item $A$ is called \emph{AS regular} if it is AS Gorenstein and it has finite left and
  right global dimensions.
\end{enumerate}
\end{Definition}

The conditions of Definition \ref{as-regularity} are usually introduced in the context
of connected $\NN_0$-graded algebras. The following remark aims to clarify the relation
between this case and the multigraded one.
\begin{Remark}
\label{AS-phi}
Let $\phi: \ZZ^r \to \ZZ$ be a group morphism such that $\phi(\NN_0^r) \subset \NN_0$. We
define $\phi_!(A)$ to be the $\NN_0$-graded algebra with underlying $k$-algebra equal to
$A$, and homogeneous components given by
\begin{align*}
\phi_!(A)_n &= \bigoplus_{\{\xi \in \ZZ^r \mid \phi(\xi) = n\}} A_\xi &\mbox{for every } n
\in \NN_0.
\end{align*}
Suppose $\phi_!(A)$ is connected. Since $A$ and $\phi_!(A)$ have the same underlying
algebra, Lemma \ref{graded-torsion} implies that for every $i \geq 0$, the $i$-th local 
cohomology modules of $A$ and $\phi_!(A)$ have the same underlying $A$-module. In
particular, one of them is zero if and only if the other is zero, so $A$ is AS Cohen
Macaulay if and only if $\phi_!(A)$ is AS Cohen Macaulay. 

By Proposition \ref{grhom-hom}, for every $i \geq 0$ the underlying $A$-modules of
$\GrExt^i_A(k,A)$ and $\GrExt^i_{\phi_!(A)}(k,\phi_!(A))$ coincide, and by \cite{RZ2} the
graded injective dimension of $A$ and that of $\phi_!(A)$ are the same, so $A$ is AS
Gorenstein if and only if $\phi_!(A)$ is AS Gorenstein. By \cite{RZ2}, the global dimension
of $A$ is equal to that of $\phi_!(A)$, so $A$ is AS regular if and only if $\phi_!(A)$ is
AS regular.
\end{Remark}

Next we define two invariants associated to local cohomology.
\begin{Definition}
Let $M$ be a graded $A$-module. The \emph{depth} of $M$ is the least integer such that
$\GrExt^i_A(k,M)$ is nonzero.
\end{Definition}

\begin{Definition}
The \emph{local dimension} of $M$, denoted $\ldim M$, is the supremum of the $i \in \NN_0$
such that $H^i_\m(M)$ is nonzero. We say that $A$ has \emph{finite local dimension as a
graded algebra} if the set 
\begin{align*}
\{\ldim M | M \mbox{ is an object of } A\} \subset \NN_0
\end{align*}
is bounded. 
\end{Definition}

The next lemma summarizes various results about the local cohomology functors.
\begin{Lemma}
\label{lc-varia}
Let $M$ be an object in $\Gr A$.
\begin{enumerate}
\item We have $\depth M = \inf \{i \in \NN_0 \mid H_\m^i(M) \neq 0\}$.

\item If $A$ is noetherian then local cohomology functors and direct limits commute.

\item If $A$ is noetherian and has finite local dimension as a graded algebra, then 
$\ldim M \leq \ldim A$.

\item Suppose $A$ is $\NN_0$-graded, and for every $n \in \NN_0$ denote by $A_{\geq n}$
the ideal generated by homogeneous elements of degree larger than or equal to $n$.
For every $d_0 \in \ZZ$ and every $i \geq 0$ there exists $n_0 \in \NN$ such that 
\begin{align*}
H^i_\m(M)_d &\cong \GrExt_A^i(A/A_{\geq n},M)_d & \mbox{for all } n \geq n_0,\ d \geq d_0.
\end{align*}
\end{enumerate}
\end{Lemma}
\begin{proof}
\begin{enumerate}
\item See \cite{W}*{4.6.3}.

\item See \cite{vdB}*{Lemma 4.3}.

\item Since $A$ has finite local dimension as a graded algebra, there is an object $N$ in
$\grmod A$ such that $n = \ldim N$ is an upper bound for the local dimensions of all
objects in $\grmod A$. Since $A$ is noetherian, there exists a finitely generated
graded-free $A$-module $F$ and an epimorphism $p: F \to N$ with finitely generated kernel
$K$. By looking at the long exact sequence for local cohomology associated to the short
exact sequence $0 \to K \to F \to N \to 0$ we obtain
\begin{align*}
H_\m^n(F) \to H_\m^n(N) \to H^{n+1}_\m(K) = 0.
\end{align*}
In particular $H_\m^n(F) \neq 0$, and since local cohomology functors commute with finite
direct sums, $\ldim F = \ldim A = n$. Thus the result is valid for finitely generated graded
$A$-modules. In particular $H^i_\m(M') = 0$ for any finitely generated graded submodule
$M' \subset M$ and any $i > \ldim A$. Since $M$ is the direct limit of its finitely generated 
graded submodules, the previous item implies that $H^i_\m(M)=0$ for $i > \ldim A$.

\item See \cite{AZ}*{Proposition 3.5 and Corollary 3.6}.
\end{enumerate}
\end{proof}

The following definition was introduced by Artin and Zhang in \cite{AZ}*{Section 3}.
It appears as a natural condition when transfering notions from algebraic
geometry to the study of connected graded algebras.

\begin{Definition}
Let $M$ be a graded $A$-module. We say that $M$ has \emph{property $\chi$} if for all $i
\in \NN_0$ the vector space $\GrExt_A^i(k,M)$ is finite dimensional. We say that $A$ has
\emph{property $\chi$ as a graded algebra} if all finitely generated graded modules have
property $\chi$.
\end{Definition}
	
Balanced dualizing complexes were introduced by A. Yekutieli in \cite{Ye} to answer a
question posed by M. Artin on the local cohomology modules of AS regular algebras. They
have proven to be a very useful tool for the study of graded connected algebras and their
homological properties.
\begin{Definition}
	Let $A^e = A \ot A^{\opp}$. A \emph{dualizing complex} $R^\bullet$ is
	an object of the derived category $\D(\Gr A^e)$ such that:
	\begin{enumerate}
		\item Its homology modules are finitely generated both as left and right
			$A$-modules.
		\item It has finite injective dimension as a complex of both left and
			right $A$-modules.
		\item The natural morphisms
		\begin{align*}
			A^\opp &\to \R\GrHom_A(R^\bullet, R^\bullet) & \mbox{and}&&  A
			&\to \R\GrHom_{A^\opp}(R^\bullet, R^\bullet)
		\end{align*}
	are isomorphisms in $\D(\Gr A^e)$.
	\end{enumerate}
	A dualizing complex is said to be \emph{balanced} if $\R\GG_\m(R^\bullet) \cong
	A'$, where $A' = \GrHom_k(A,k)$ is the Matlis dual of $A$.
\end{Definition}

The main result on the existence of balanced dualizing complexes is the following:

\begin{Proposition}\cite{vdB}*{Proposition 6.3}
\label{vdb-criterion}
  A connected $\NN_0$-graded noetherian $k$-algebra $B$ has a balanced dualizing complex
  if and only if $B$ and $B^\opp$ have property $\chi$ and finite local dimension as
  graded algebras.
\end{Proposition}

\begin{Remark}
  We keep the notation from Remark \ref{AS-phi}. As proved in \cite{RZ2}, if $A$ has
  finite local dimension as a graded algebra, then $\phi_!(A)$ has property $\chi$ as a
  graded algebra if and only if $A$ does. In particular, $\phi_!(A)$ has a dualizing
  complex if and only if $A$ and $A^\opp$ have finite local dimension as graded algebras
  and property $\chi$ as graded algebras.
\end{Remark}

\section{GF-rings and modules}
\begin{Definition}
	A \emph{GF-algebra} (as in ``graded \emph{and} filtered'') is an $\NN_0$-graded
	$k$-algebra $A$ with a filtration $\F= \{F_pA\}_{p \in \NN_0}$ such that each
	$F_pA$ is a graded $k$-subspace of $A$. We say that an $A$-module $M$ is a
	\emph{GF-module} over $A$ if it is both a $\ZZ$-graded and a $\ZZ$-filtered module 
	in the usual sense, and each layer of its filtration is a graded $k$-subspace of
	$M$.
\end{Definition}
As a trivial example we consider $k$ to be a GF-algebra with $k_0 = F_0k = k$. A GF-vector 
space is a $\ZZ$-graded vector space with a filtration by graded subspaces.

\begin{Definition}
A filtration on a vector space $M$ is said to be
\begin{enumerate}
	 \item[(E)] \emph{exhaustive} if $\displaystyle \bigcup_{p \in \ZZ}
	 	F_p M = M$.
	 \item[(B)] \emph{bounded below} if there is a $q \in \ZZ$ such that
		 $F_p M = \{0\}$ for all $p < q$.
	 \item[(D)] \emph{discrete} if $\displaystyle \bigcap_{p \in \ZZ}
	 	F_p M = \{0\}$.
\end{enumerate}
\end{Definition}
Let $A$ be a GF-algebra and let $M$ be a GF-module over $A$. Given a vector subspace $V
\subset M$, the filtration on $M$ induces a filtration on $V$ by setting $F_pV =
V \cap F_pM$ for every $p \in \ZZ$. In particular every homogeneous component of $M$ is a
filtered subspace. If the filtration on $M$ is exhaustive and discrete then for each $m \in M$
there exists $p(m) \in \ZZ$ such that $m \in F_{p(m)}M$ and $m \notin F_q M$ for $q < p(m)$.
Given a homogeneous element $m \in M$ we denote by $\delta(m)$ the ordered pair $(p(m),
\deg m)$.

We say that $M$ is a \emph{GF-free} module if it is a free $A$-module with a homogeneous 
basis $\{e_i\}_{i \in I}$ and a discrete filtration such that if $\delta(e_i) = (p_i,d_i)$
then
$$ F_p M_d = \sum_{i \in I} F_{p-p_i}A_{d-d_i} e_i.$$ 
The GF-module $M$ is \emph{locally finite} if $F_pM_d$ is a finite dimensional vector space
for all $p,d \in \ZZ$. Finally $M$ is \emph{GF-finite} if it is generated by a
finite set of homogeneous elements $m_1, \ldots, m_r$ with $\delta(m_i) = (p_i,d_i)$
such that  
$$ F_p M_d = \sum_{i = 1}^r F_{p-p_i}A_{d-d_i} m_i.$$ 

Given two GF-modules $M$ and $N$, the vector space $\Hom_A(N,M)$ is filtered in the following
way: for every $p \in \ZZ$, set
\begin{align*}
F_p\Hom_A(N,M) &= \{f \in \Hom_A(N,M)| f(F_qN) \subset F_{q+p} M \mbox{ for all }
q \in \ZZ\}.
\end{align*}
This induces a filtration on the $k$-vector subspace $\GrHom_A(N,M) \subset \Hom_A(N,M)$.
With this filtration the $\ZZ$-graded vector space $\GrHom_A(N,M)$ becomes a GF-vector
space. 

A morphism $f \in \GrHom_A(N,M)$ which is homogeneous of degree $0$ and such that $f(F_pN) 
\subset F_pM$ for every $p \in \ZZ$ is called a \emph{GF-morphism}.
Equivalently, $f$ is a GF-morphism if and only if $f \in F_0\GrHom_A(N,M)_0$. We say that
a GF-morphism $f$ is \emph{strict} if $f(F_p N) = \im f \cap F_p M$ for all $p \in \ZZ$.
Notice that this condition is strictly stronger than that of being a GF-morphism. A
GF-module $M$ is finite if and only if there is a GF-finite and free module $F$ and a
strict epimorphism $F \to M$.

Since the GF-algebra $A$ is filtered, we may consider its associated graded algebra 
	$$\gr A = \bigoplus_{p = 0}^{\infty} \frac{F_{p}A}{F_{p-1}} A = \bigoplus_{(p,d) \in
	\NN_0^2} \frac{F_pA_d}{F_{p-1}A_d}.$$
This decomposition gives $\gr A$ the structure of an $\NN_0^2$-graded algebra.
Analogously, given a GF-module $M$, its associated graded module
	$$\gr M = \bigoplus_{p \in \ZZ} \frac{F_pM}{F_{p-1}M} = \bigoplus_{(p,d) \in \ZZ^2}
	\frac{F_pM_d}{F_{p-1}M_d}$$ 
is a $\ZZ^2$-graded $\gr A$-module. If the filtration on
$M$ is exhaustive and discrete, a homogeneous element $m \in M$ defines a nonzero class
in $(\gr M)_{\delta(m)}$ which we denote by $\gr m$. If $f \in \Hom_A(N,M)$ with $\delta(f)
= (p,d)$ then $\gr f: \gr N \to \gr M$ is a homogeneous $\gr A$-linear morphism of degree
$(p,d)$. 

Now we present some technical results on GF-modules to be used in the sequel.
\begin{Lemma}
\label{filtered-complex}
Let $A$ be a GF-algebra and let $K, M$ and $N$ be GF-modules. Suppose the filtrations
on $A$ and $M$ are exhaustive and discrete. 
\begin{enumerate}
	\item Let $(*): K \stackrel{f}{\to} M \stackrel{g}{\to} N$ be a complex with $f$ and
	$g$ GF-morphisms. Its associated graded complex $\gr K \stackrel{\gr f}{\to} \gr
	M \stackrel{\gr g}{\to} \gr N$ is exact if and only if $(*)$ is exact and
	$f, g$ are strict.

	\item If $M$ is GF-free with basis $\{m_i\}_{i \in I}$ then $\gr M$ is a 
	$\ZZ^2$-graded $\gr A$-free module with basis $\{\gr m_i\}_{i \in I}$.

	\item If $\gr M$ is generated over $\gr A$ by the set $\{\gr m_i|i \in
	I\}$, then $M$ is generated over $A$ by the set $\{m_i|i \in I\}$. 
	Moreover, $M$ is GF-finite if and only if $\gr M$ is finitely generated.

	\item There exists a resolution of $M$ by GF-free modules with exact
	differentials. Furthermore, if $\gr A$ is noetherian and $\gr M$ is
	finitely generated over $\gr A$, the GF-free modules in the resolution can
	be chosen to be GF-finite.
\end{enumerate}
\end{Lemma}
\begin{proof}
\begin{enumerate}
	\item Notice that the gradings play no role in this statement, so we may
	refer to the filtered case, proven in \cite{VO}*{4.4, item 5.}.

	\item Let $p_i = p(m_i)$ and $d_i = \deg m_i$. By definition, for every $(p,d) \in 
	\ZZ^2$ 
	\begin{align*}
	  F_pM_d = \bigoplus_{i \in I} F_{p -p_i}A_{d-d_i} m_i
	\end{align*}
	so
	\begin{align*}
 	  \gr M_{(p,d)} = \bigoplus_{i \in I} \frac{F_{p-p_i}A_{d-d_i}
	  m_i}{F_{p-p_i-1}A_{d-d_i}{m_i}} \cong \bigoplus_{i \in I} (\gr
	  A)_{(p-p_i,d-d_i)} \gr m_i. 
	\end{align*}
	Thus $\gr M$ is a graded-free $\gr A$-module with basis $\{\gr m_i|i \in I\}$.

	\item Let $L$ be a GF-free module with basis $\{e_i|i \in I\}$ and
	$\delta(e_i) = \delta(m_i)$. Since the filtration on $A$ is exhaustive and
	discrete, the same holds for the filtration on $L$. Let $f: L \to M$ be the map
	defined by setting $f(e_i) = m_i$ and consider its associated graded map $\gr f: \gr
	L \to \gr M$. Since $\gr f(\gr e_i) = \gr m_i$, the map $\gr f$ is surjective by 
	hypothesis, so item 1. implies that $f$ is a strict epimorphism. In particular $M$
	is generated by the set $\{m_i|i \in I\}$. 
		
	If $\gr M$ is finitely generated then $L$ can be taken to be GF-finite and free in
	the previous argument, and so there exists a strict epimorphism from a GF-finite and 
	free module onto $M$. This is equivalent to the fact that $M$ is a GF-finite module. 
	Conversely suppose we have a GF-finite and free module $L$ and a strict epimorphism $L
	\to M$. By passing to the associated graded modules we get an epimorphism
	from $\gr L$ to $\gr M$, and since $\gr L$ is finitely generated and free by item
	2., $\gr M$ is finitely generated.

	\item The first part of item 3. shows that for every GF-module $M$ there
	exists a GF-free module $L$ and a strict epimorhism $L \to M$, so applying the
	usual procedure to construct a free resolution we obtain a resolution by GF-free
	modules with exact differentials. 
	
	If $\gr A$ is noetherian and $\gr M$ is finitely generated then $A$ is a
	noetherian algebra\footnote{The previous item has a ''filtered version'', namely
	if $A$ is a filtered algebra, $M$ is an $A$-filtered module, and $\gr M$ is finitely
	generated, then $M$ is finitely generated. Any ideal $I\subset A$ is a filtered
	ideal with the induced filtration, and if $\gr A$ is noetherian, $\gr I \subset
	\gr A$ is finitely generated, so $I$ is finitely generated, i.e. $A$ is
	noetherian} and $M$ a  	finitely generated $A$-module. The second part of item 	3.
	shows that every finitely generated module has a GF-finite and free cover, so we
	may apply the usual procedure to construct finite and free resolutions over
	noetherian algebras to obtain the desired result.
	\end{enumerate}
\end{proof}

\begin{Lemma}
\label{filtered-hom}
Let $A$ be a GF-algebra and let $M$ and $N$ be two GF-modules. Assume all
filtrations are exhaustive and discrete. 
\begin{enumerate}
	\item If $N$ is GF-finite then $\{F_p\Hom_A(N,M)\}_{p \in \ZZ}$ is an exhaustive
	filtration on $\Hom_A(N,M)$. If the filtration on $M$ is bounded below then the
	filtration on $\Hom_A(N,M)$ is also bounded below.

	\item If $N$ is GF-finite and free then there is a natural $\ZZ^2$-graded
	vector space isomorphism
	\begin{align*}
		\phi: \gr \Hom_A(N,M) \to  \GrHom_{\gr A}(\gr N, \gr M)
	\end{align*}
	given by the following assignation: for every $f \in F_{p(f)}\Hom_A(N,M)$
	and $x \in F_{p(x)}N$
	\begin{align*}
		\phi(\gr f)(\gr x) = \overline{f(x)} \in F_{p(f) + p(x)}M /
		F_{p(f)+p(x)-1} M.
	\end{align*}
\end{enumerate}
\end{Lemma}
\begin{proof}
\begin{enumerate}
	\item Let $n_1, \ldots, n_r$ be a finite set of generators of $N$ and set $p_i =
	p(n_i)$ for $1 \leq i \leq r$. Let $f \in \Hom_A(N,M)$ and let $q_i = p(f(n_i))$. 
	If $t =	\max_i\{q_i - p_i\}$ then
	\[
		f(F_pN) = \sum_i f(F_{p - p_i}A n_i) = \sum_i F_{p -p_i}A
		f(n_i) \subset \sum_i F_{p - p_i + q_i}M \subset F_{p +
		t}M.
	\]
	Hence $f \in F_t\Hom(N,M)$. If we assume the filtration on $M$ is bounded below
	there exists $q \in \ZZ$ such that $F_q M = 0$. Let $p_0 = \max\{p_i, 1 \leq i
	\leq r\}$ If $f \in F_{q - p_0}\Hom_A(N,M)$ then $f(n_i) \in F_{q - p_0 + p_1}M =
	0$, so $F_{q-p_0}\Hom_A(N,M) = 0$ and the filtration on $\Hom_A(N,M)$ is bounded
	below.

	\item It is clear from the definition that $\phi$ is a graded morphism. Since 
	GF-free modules are filtered projective in the sense of \cite{VO}*{Section I.5},
	$\phi$ is an isomorphism by \cite{VO}*{Lemma 6.4}.
\end{enumerate}
\end{proof}

The following theorem relates the $\ZZ$-graded $\Ext$-modules over $A$ with the
$\ZZ^2$-graded $\Ext$-modules over $\gr A$. It is a generalization of the spectral
sequence described in \cite{B}*{section 3}, which deals with filtered algebras. We will
use this result in section 2. to transfer some homological properties from $\gr A$ to $A$.

\begin{Theorem}
\label{ext-ss}
	 Let $A$ be a GF-algebra, and let $M$ and $N$ be two GF-modules with $N$ GF-finite.
	 Suppose that all filtrations are exhaustive and discrete, that the filtration on
	 $M$ is bounded below, and that $\gr A$ is noetherian. Then for each $d \in \ZZ$ 
	 there is a convergent spectral sequence:
	\begin{align*}
		E(N,M)_d: E_{p,q}^1 &= \GrExt_{\gr A}^{-p-q}(\gr N, \gr
		M)_{(p,d)} \Rightarrow \GrExt_A^{-p-q}(N,M)_d &p,q \in \ZZ,
	\end{align*}
	and the filtration of the $\GrExt$-group on the right hand side is
	bounded below and exhaustive. 
\end{Theorem}
\begin{proof}
 	By item 4. of Lemma \ref{filtered-complex} there is a GF-free and finite
	resolution of $N$ with exact differentials
	\begin{align*}
		 \ldots \to P_{-2} \to P_{-1} \to P_{0} \to N \to 0
	\end{align*}
	In particular each $P_i$ has a finite and bounded below filtration, so item 1. of
	Lemma \ref{filtered-hom} implies that $\GrHom_A(P_{i},M)$ is a filtered	GF-vector
	space for all $i \leq 0$, and its filtration is exhaustive and bounded below. 
	
	Fix $d \in \ZZ$. Since the differentials in the resolution of $N$ are strict
	GF-morphisms, $\GrHom_A(P_\bullet,M)_d$ is a complex of filtered vector spaces. By
	\cite{W}*{5.5.1.2} there is a spectral sequence with page $1$ equal to
	\begin{align*}
		 E_{p,q}^1 &= H_{p+q}(F_p \GrHom_A(P_\bullet,M)_d / F_{p-1}
		 \GrHom_A(P_\bullet,M)_d) & p,q \in \ZZ
	\end{align*}
	that converges to 	
	\begin{align*}
		 H_{p+q}(\GrHom_{A}(P_\bullet,M)_d) = \GrExt^{-p-q}_A(N,M)_d.
	\end{align*}
	
	By item 1. of Lemma \ref{filtered-complex} the complex
	\begin{align*}
		 \ldots \to \gr P_{-2} \to \gr P_{-1} \to \gr P_{0} \to \gr N \to 0
	\end{align*}
	is exact, and by item 2. of the same Lemma it is a free resolution of $\gr N$ as a
	$\ZZ^2$-graded $\gr A$-module. Finally by item 2. of Lemma \ref{filtered-hom}
	there is a natural isomorphism
	$$F_p\GrHom_A(P_\bullet,M)_d/F_{p-1}\GrHom_A(P_\bullet,M)_d \cong \GrHom_{\gr A}
	(\gr P_\bullet,\gr M)_{(p,d)}$$ 
	and so $E_1^{p,q} \cong \GrExt_{\gr A}^{-p-q}(\gr N, \gr M)_{(p,d)}$. This
	completes the proof.
\end{proof}

The following is an immediate corollary of the previous result
\begin{Corollary}
\label{zero-lemma}
	Suppose $M$ and $N$ are GF-modules over $A$ satisfying the hypotheses of
	Theorem \ref{ext-ss}. Fix $d \in \ZZ$ and $i \in \NN_0$. If $\GrExt_{\gr A}^i(\gr N, 
	\gr M)_{(p,d)} = 0$ for all $p \in \ZZ$, then $\GrExt_A^i(N,M)_d = 0$.
\end{Corollary}
\begin{proof}
	By hypothesis the $i$-th diagonal in the first page of $E(N,M)_d$
	has no nonzero entry and so the same is true for the infinity page. By Theorem
	\ref{ext-ss} $\GrExt_A^i(N,M)_d$ has a bounded below and exhaustive filtration such 
	that its associated graded module is zero. Hence by item 3. of Lemma
	\ref{filtered-complex}, $\GrExt_A^i(N,M)_d = 0$.
\end{proof}


\subsection{Transfer of homological properties}
We say that an algebra $A$ is \emph{GF-connected} if it is a GF-algebra and $\gr A$ is
connected. Throughout this section $A$ denotes a GF-connected algebra. Our aim is to prove
that the homological regularity properties discussed in subsection 1.2 transfer from $\gr
A$ to $A$.

Given a graded $A$-module $M$ and a set of homogeneous generators $\{m_i\}_{i \in I}$ of
$M$ we can give $M$ the structure of a GF-module, setting $F_pM = \sum_{i \in I} F_pA
m_i$ for all $p \in \ZZ$. If $M$ is finitely generated and $I$ is a finite set, the
filtration we obtain is exhaustive and discrete and $M$ is a GF-finite module. By item 3.
of Lemma \ref{filtered-complex}, $\gr M$ will be a finitely generated $\gr A$-module. This 
simple result will be important in the sequel, so we write it down as a lemma.
\begin{Lemma}
\label{GF-structure}
  Assume $A$ is GF-connected and let $M$ be a finitely generated graded $A$-module. Then
  $M$ can be given the structure of a GF-finite module and its associated graded module is
  finitely generated over $\gr A$.
\end{Lemma}

This result is used to prove the next lemma, which is the crucial step in the proof of the
transfer of homological regularity properties from $\gr A$ to $A$. 
\begin{Lemma}
\label{varia-modules}
	Let $A$ be a GF-connected algebra such that $\gr A$ is noetherian, and let $M$ be
	a finitely generated graded $A$-module. Suppose $M$ has a GF-module structure as
	in Lemma \ref{GF-structure}. The following hold:
	\begin{enumerate}
		\item $\depth \gr M \leq \depth M$.
		\item $\gr\pdim~M \leq \gr\pdim \gr M$ and $\gr\injdim M \leq \gr\injdim
			\gr M$.	
		\item If $\gr M$ has property $\chi$ then so does $M$. 
		\item If $\gr M$ has property $\chi$ then $\ldim M \leq \ldim \gr M$. 
	\end{enumerate}
\end{Lemma}
\begin{proof} Recall from Theorem \ref{ext-ss} that under the hypotheses of the lemma, for
	any finite GF-module $N$ and any $d \in \ZZ$ there is a spectral sequence 
	\begin{align*}
	E(N,M)_d: E^1_{p,q} &= \GrExt_{\gr A}^{-p-q}(\gr N, \gr M)_{(p,d)} \Rightarrow
	\GrExt_A^{-p-q}(N,M)_d & p,q \in \ZZ.
	\end{align*}

	\begin{enumerate}
	\item Let $i < \depth \gr M$. By defintion $\GrExt^i_{\gr A}(k,\gr M) = 0$,
	and since $\gr k = k$, applying Corollary \ref{zero-lemma} we see that
	$\GrExt_A^i(k,M) = 0$, so $i < \depth M$. This proves the desired inequality.

	\item Let $M$ be any finite GF-module and let $m_1, \ldots, m_r$ be a minimal set
	of generators of $M$. Let $F$ be the GF-finite and free module with basis $e_1,
	\ldots, e_r$, such that $\delta(e_i) = \delta(n_i)$ for $1 \leq i \leq r$. By
	Nakayama's Lemma for connected graded algebras, the map $p: F \to M$ given by 
	$p(e_i) = m_i$ for all $i$ is the minimal projective cover of $M$, and it is also 
	a strict morphism by definition. Thus we may repeat the construction described in
	item 4. of Lemma \ref{filtered-complex} and obtain a resolution $P_\bullet \to M$
	by GF-finite and free modules that is also a minimal resolution of $M$. In
	particular $P_i = 0$ for $i > \gr\pdim~M$, so by item 1. of the same lemma, $\gr
	P_\bullet$ is a free resolution of $\gr M$, which implies that $\gr\pdim M \leq
	\gr\pdim \gr M$.
	
	Let $N$ be a finitely generated graded $A$-module, and give it a GF-module
	struture as in Lemma \ref{GF-structure}. If $i > \gr\injdim \gr M$ then
	$\GrExt_{\gr A}^i(\gr N, \gr M) = 0$, and again by Corollary \ref{zero-lemma},
	$\GrExt_A^i(M,N) = 0$, so $\injdim~M \leq i$.
	
	\item Suppose $\gr M$ has property $\chi$, i.e. $\GrExt_{\gr A}^i(k,\gr M)$ is
	finite dimensional for all $i \geq 0$. Fix $i\in \NN_0$. By hypothesis, for all $d
	\in \ZZ$ the $i$-th diagonal of the first page of $E(k,M)_d$ has at most a finite
	number of nonzero entries, each of finite dimension. Furthermore, there are 
	\emph{no} nonzero entries except for finitely many $d$'s. The same is true of the 
	infinity page, so the associated graded vector space of $\GrExt_A^i(k,M)_d$ is
	finite dimensional, and equal to zero for all but finitely many $d$'s. By
	item 3. of Lemma \ref{filtered-complex} the same is true for $\GrExt_A^i(k,M)$,
	which implies this is a finite dimensional $k$-vector space.


	\item Let $i > \ldim \gr M$ and let $d \in \ZZ$. We will prove that $H_\m^i(M)_d =
	0$. By item 4. of Lemma \ref{lc-varia}, there exists $n_0 \in \NN$ such that 
	\begin{align*}
	  H_\m^i(M)_d &\cong \GrExt^i_A(A/A_{\geq n}, M)_d &\mbox{for all } n \geq n_0,
	\end{align*}
	so it is enough to prove that $\GrExt^i_A(A/A_{\geq n},M)_d = 0$ for $n$ large
	enough. 
	
	Let $\pi: \ZZ^2 \to \ZZ$ be the projection on the second coordinate. Then using
	the notation of Remark \ref{AS-phi}, the algebra $B= \pi_!(\gr A)$ is a connected
	$\NN_0$-graded algebra. Let $\pi_!(M)$ be the graded $B$-module with
	underlying $\gr A$-module equal to $\gr M$, and homogeneous components given by
	\begin{align*}
	  \pi_!(M)_d &= \bigoplus_{p \in \ZZ} M_{p,d} & \mbox{for every } d \in \ZZ.
	\end{align*}

	By hypothesis, $H^i_\m(\gr M) = 0$ and since $\gr M$ and $\pi_!(\gr M)$ have the
	same underlying $\gr A$-module, by Lemma \ref{graded-torsion} their cohomology
	modules have the same underlying $\gr A$-modules. Once again by item 4. of Lemma
	\ref{lc-varia}, there exists $n_1 \in \NN_0$ such that
	\begin{align*}
	  0 = H^i_\m(\pi_!(M))_d &\cong \GrExt_{B}^i(B/B_{\geq n}, \pi_!(\gr M)) &\mbox{for
	  all } n \geq n_1.
	\end{align*}
	By definition $B_{\geq n} = \gr (A_{\geq n})$, and since $\gr A / \gr (A_{\geq n})
	\cong \gr (A/A_{\geq n})$, Proposition \ref{grhom-hom} implies that
	\begin{align*}
	  \GrExt^i_{\gr A}(\gr(A/A_{\geq n}), \gr M)_d &= 0 &\mbox{for all } n \geq n_1.
	\end{align*}
	Taking $n$ larger than both $n_0$ and $n_1$, the result follows from Corollary
	\ref{zero-lemma}.
\end{enumerate}
\end{proof}

Now we prove our main transfer result. 
\begin{Theorem}
\label{AS-transfer}
Assume $\gr A$ has property $\chi$ as a graded algebra. Then $A$ has property $\chi$ as
a graded algebra, and if $\gr A$ has any of the following properties, so does $A$:
	\begin{enumerate}
		\item Having finite local dimension.
		\item Having a balanced dualizing complex.
		\item Being AS Cohen Macaulay.
		\item Being AS Gorenstein.
 		\item Being AS regular.
	\end{enumerate}
\end{Theorem}
\begin{proof}
  Let $M$ be a finitely generated graded $A$-module. Since $\gr A$ has property $\chi$ as
  an algebra, $\gr M$ has property $\chi$ and by item 3. of Lemma \ref{varia-modules}, so
  does $M$. This proves that $A$ has property $\chi$ as an algebra.
  \begin{enumerate}
    \item Since $\gr A$ has property $\chi$ as an algebra, this follows from item 4. of 
    Lemma \ref{varia-modules}. 
	
    \item By Proposition \ref{vdb-criterion} $A$ has a balanced dualizing complex if and
    only if $A$ and $A^\opp$ have property $\chi$ as graded algebras and finite local 
    dimension as graded algebras. Since $\gr A^\opp = (\gr A)^\opp$, the result follows
    from the previous item.

    \item Since the $\gr A$-modules $\gr A$ and $\gr A^\opp$ have property $\chi$ over
    $\gr A$ we apply items 1. and 4. of Lemma \ref{varia-modules} to get the series of
    inequalities
    \begin{align*} 
      \depth \gr A \leq \depth A &\leq \ldim A \leq \ldim \gr A,\\
      \depth \gr A^\opp \leq \depth A^\opp &\leq \ldim A^\opp \leq \ldim \gr A^\opp.
    \end{align*}
    If $\gr A$ is AS Cohen Macaulay then all these numbers are equal, so $A$ is AS Cohen
    Macaulay.

  \item Since $\gr A$ is AS Gorenstein, it has finite injective dimension, and there
  are vector space isomorphisms
  \begin{align*}
    \GrExt_{\gr A}^i(k,\gr A) &\cong
    \begin{cases} 
      k & \mbox{ if } i = \gr \injdim \gr A, \\ 0 & \mbox{ otherwise.}
    \end{cases}
  \end{align*}
  By Corollary \ref{zero-lemma} and item 2. of Lemma \ref{filtered-complex}, 
  \begin{align*}
    \GrExt_{A}^i(k,A) &\cong
    \begin{cases} 
      k & \mbox{ if } i = \gr \injdim \gr A, \\ 0 & \mbox{ otherwise.}
    \end{cases}
  \end{align*}
  This and item 2. of Lemma \ref{varia-modules} imply $\gr \injdim A = \gr \injdim
  \gr A$.

  \item This follows immediately from item 2. of Lemma \ref{varia-modules} and the fact
  that the global dimension of $A$ is equal to ts graded global dimension.\footnote{This
  is proved in Paul Smith's book, Theorem 2.7 of chapter 14. It will probably also be in
  \cite{RZ2}.}
  \end{enumerate}
\end{proof}

\begin{Remark}
	By \cite{Ye}*{Proposition 4.4 and Corollary 4.10}, any AS Gorenstein
	algebra has a dualizing complex and therefore has proprety $\chi$. Thus
	for the last two items one may ommit the hypothesis that $\gr A$ has
	property $\chi$.
\end{Remark}

\section{Algebras with $(\S,\alpha)$-bases}
In \cite{C} P. Caldero proved that Schubert subvarieties of full flag varieties over the
complex numbers degenerate to toric varieties. To prove this he introduces a $k$-basis
for the homogeneous coordinate ring of any such variety with good multiplicative
properties, and uses this basis to define a filtration such that its
associated graded ring is a semigroup algebra. In this section we define a class of
algebras having a basis with similar multiplicative properties,
and prove that all algebras in this class have a filtration such that their associated
graded rings are \emph{twisted} semigroup algebras, as studied by the authors in
\cite{RZ2}. We then use results due to Caldero \cite{C} and Littelmann \cite{Lit} to prove 
that \emph{quantum} Schubert
varieties, or rather quantum deformations of their homogeneous coordinate rings, lie in
this class. Finally we use the tools developed in section 2. to investigate their
homological regularity properties.

A \emph{positive affine semigroup} is a semigroup with a zero element that is isomorphic
to a finitely generated subsemigroup of $\NN_0^N$ for some $N \in \NN$. For simple facts
on positive affine semigroups the reader is refered to \cite{BH}*{chapter 6}. From now on 
we fix a positive affine semigroup $\S$ and an embedding of $\S$ in $\NN_0^N$. The
lexicographic order of $\NN_0^N$ induces a total order on $\S$, which we will denote by
$\leq$.  We say that a $k$-algebra $A$ is \emph{$\S$-graded} if it is an
$\NN_0^N$-graded algebra and $A_d = 0$ for all $d \in \NN_0^N \setminus \S$. 

An $\S$-filtration on a $k$-algebra $A$ is a family of $k$-subvector spaces of
$A$, $\F = \{F_sA\}_{s \in \S}$, such that $F_s A \cdot F_{s'}A \subset F_{s+s'} A$ for 
all $s,s' \in \S$, and $F_{s'} A \subset F_sA$ whenever $s' \leq s$. We write $F_{<s} A$
for  $\sum_{s'<s} F_{s'} A$. The associated $\S$-graded algebra of an $\S$-filtered
algebra is $\gr_\F A = \bigoplus_{s \in \S} \frac{F_sA}{F_{<s}A}$. One may define
$\S$-filtered modules and their associated $\S$-graded modules in an analogous way. We
consider $k$ to be an $\S$-filtered $k$-algebra with the trivial filtration $F_sk = k$ for
all $s \in \S$. 

Let $A$ be an $\S$-filtered $k$-algebra and let $M$ be an $\S$-filtered $A$-module, with
filtration $\G = \{G_sM\}_{s \in \S}$. For every $s \in \S$ we define $G_{<s}M = \sum_{s'
< s} G_{s'}M$. Given an element $m \in G_sM \setminus G_{<s}M$, its class in $G_sM /
G_{<s}M$ defines an element in the homogeneous component $\gr_\G M_s$, which we denote by
$\gr m$.

The following result is a refinement of item 2. of Lemma \ref{filtered-complex}.
\begin{Lemma}
\label{s-filtered-basis}
	Let $\S$ be an affine semigroup, let $V$ be an $\S$-filtered $k$-vector space 
	with an	exhaustive filtration $\F$, and let $\{v_i |i \in I\}$ be a subset of $V$.
	If $\{\gr v_i\}_{i \in I}$ is a homogeneous basis of $\gr_\F V$, then for
	every $s \in \S$ the set $\{v_i|i \in I\} \cap F_sV$ is a basis of $F_sV$.
\end{Lemma}
\begin{proof}
  First we set some notation. For each $s \in \S$ let $I_{s} = \{i \in I| v_i \in
  F_{s}V\}$, $I_{<s} = \cup_{s'<s} I_{s'}$ and $I_s^\circ = I_s \setminus I_{<s}$. We
  point out that the hypothesis implies that the set $\{\gr v_i| i
  \in I^\circ_s\}$ is a basis of the homogeneous component $(\gr_\F V)_s$. We will
  prove that $\{v_i|i \in I_s\}$ is a basis of $F_{s}V$ by induction on the totally
  ordered set\footnote{Induction on ordered sets is one of \emph{those}
  pieces of mathematical folklore. The Wikipedia entry for \emph{well-founded relations}
  includes a proof that it is valid in the present case.} $\S$. The case $s = 0$ is clear 
  since $\gr V_0 = F_0 V$.

  Let $s \in \S$ and suppose the result holds for all $s'<s$ and that there are 
  scalars $\lambda_i$ for every $i \in I_s$ such that
  \begin{align*}
    0 = \sum_{i \in I_s} \lambda_i v_i = \sum_{i \in I_{<s}} \lambda_i v_i + \sum_{i \in
    I_s^\circ} \lambda_i v_i.
  \end{align*}
  Reducing this equality modulo $F_{<s}V$ we get
  \begin{align*}
    0 = \sum_{i \in I_s^\circ} \lambda_i \gr v_i
  \end{align*}
  which implies that $\lambda_i = 0$ for $i \in I_s^\circ$. On the other hand, there are
  at most finitely many $i \in I_{<s}$ such that $\lambda_i \neq 0$, so
  there exists $s' < s$ such that $\lambda_i \neq 0$ implies $i \in I_{s'}$. By the
  inductive hypothesis the set $\{v_i| i \in I_{s'}\}$ is linearly independent, so
  $\lambda_i = 0$ for all $i \in I_s$, and $\{v_i|i \in I_s\}$ is linearly independent.

  Let $v \in F_sV$. If $v \in F_{s'}V$ for some $s'<s$ then the inductive hypothesis
  guarantees that it is in the $k$-vector space generated by $\{v_i|i \in I_{s'}\}$. If
  not, then $\gr v  \in (\gr_\F V)_s$, and there are scalars $\lambda_i$ for $i \in
  I_s^\circ$ such that 
  \begin{align*}
    \gr v = \sum_{i \in I_s^\circ} \lambda_i \gr v_i
  \end{align*}
  or equivalently, such that $v - \sum_{i \in I_s^\circ} \lambda_i v_i \in F_{s'}V$ for 
  some $s' < s$. By the previous observation, this difference lies in the $k$-vector space
  generated by $\{v_i|i \in I_{s'}\}$, so $v$ is in the $k$-vector space generated by
  $\{v_i|i  \in I_s\}$. This completes the proof.  
\end{proof}

Given a $2$-cocycle $\alpha: \S \times \S \to k^\times$, the \emph{$\alpha$-twisted
semigroup algebra} of $\S$, denoted by $k^\alpha[\S]$, is the algebra with underlying
vector space generated by the set $\{x^s|s \in \S\}$ and product defined over
generators by $x^s x^{s'} = \alpha(s,s') x^{s+s'}$ and extended bilinearly. The fact that
$\alpha$ is a 2-cocycle implies that this product is associative. These algebras are
studied by the authors in \cite{RZ2}, were it is proved that for any completely ordered
semigroup $\S$ and any 2-cocycle $\alpha$, the algebra $k^\alpha[\S]$ is integral.

We now turn to the problem of giving a presentation of $k^\alpha[\S]$.
Let $\{s_1, \ldots, s_r\}$ be a minimal set of generators of $\S$. There is an obvious
surjective semigroup morphism $\pi: \NN_0^r \rightarrow \S$ induced by the assignation $e_i
\mapsto s_i$ for $1 \leq i \leq r$, where $e_i$ denotes the $i$-th element of the
canonical basis of $\NN_0^r$. Since $\S$ is commutative and finitely generated, it is also
finitely presented, i.e. the semigroup 
\begin{align*}
  L = L(\S) = \{(p,p') \in \NN_0^r \times \NN_0^r| \pi(p) = \pi(p')\}
\end{align*}
is also finitely generated. Let $\{(p_1,p'_1), \ldots, (p_{r'},p'_{r'})\}$ be a minimal
set of generators of $L$.

Let $T$ be the free algebra with generators $X_1, \ldots, X_r$. There is a surjective
algebra morphism $\phi: T \to k^\alpha[\S]$ sending $X_i$ to $x^{s_i}$. If we set the
degree of $X_i$ as $s_i$ then the free algebra $T$ becomes an $\S$-graded algebra, and
$\phi$ is an $\S$-graded algebra morphism.

Given $p = (p^1, \ldots, p^r) \in \NN_0^r$, we write $X^p$ for the monomial $X_1^{p^1}
X_2^{p^2} \ldots X_n^{p^n} \in T$. For every $1 \leq i<j\leq r$ the element
\begin{align*}
C_{i,j}: =\alpha(s_j,s_i) X_i X_j - \alpha(s_i,s_j)X_j X_i
\end{align*}
is in the kernel of $\phi$. Also, for any $s \in \S$ and $p \in \pi^{-1}(s)$ there exists 
$c_p \in k^\times$ such that $\phi(X^p) = \prod_i (x^{s_i})^{p_i} = c_p x^s$. For each $1
\leq l \leq r'$ write $c_l = c_{p_l}$ and $c'_l = c_{p'_l}$ and put
\begin{align*}
S_l:= c'_l X^{p_l} - c_l X^{p_l'}.
\end{align*}
These elements also belong to the kernel of $\phi$. 

\begin{Lemma}
\label{semigroup-presentation}
Keep the notation from the previous paragraph. Let $I \subset T$ be the ideal generated by
the elements $C_{i,j}$ and $S_l$, with $1 \leq i < j \leq r$ and $1 \leq l \leq r'$. The
twisted semigroup algebra $k^\alpha[\S]$ is isomorphic to $A = T / I$.
\end{Lemma}
\begin{proof}
We make a slight abuse of notation and for all $1 \leq i \leq r$ and all $p \in \NN_0^r$
we denote by $X_i$ and $X^p$ the class of $X_i$ and $X^p$ in $T/I$ respectively. Denote by
$\tilde \phi: A \to k^\alpha[\S]$ the morphism induced by $\phi$. Since the generators of
the ideal $I$ are $\S$-homogeneous, $A$ is an $\S$-graded algebra and $\tilde \phi$ is a
homogeneous morphism of $\S$-graded algebras. 

The algebra $A$ is a quotient of a quantum affine space, so $\{X^p|p \in \NN_0^r\}$ is a
set of homogeneous generators of $A$ as a $k$-vector space. Furthermore $\deg X^p =
\pi(p)$, so for every $s \in \S$ the set $\{X^p|\pi(p) = s\}$ generates the homogeneous 
component $A_s$.

Given $(p,p') \in L$ there exist $n_1, \ldots, n_r \in \NN_0$ such that $(p,p')
= \sum_l n_l (p_l,p_l')$. Since $A$ is a quotient of an affine space and $X^{p_l} =
\frac{c_l}{c'_l} X^{p_l'}$, there exist nonzero scalars $d, d', d''$ such that
\begin{align*}
X^p = d \prod_l (X^{p_l})^{n_l} = d \prod_l \left(\frac{c_l}{c'_l} X^{p_l'}\right)^{n_l}
= d'\prod_l X^{n_l p_l'} = d'' X^{p'}.
\end{align*}
This implies that any two monomials of degree $s \in \S$ are nonzero multiples of each
other, so $\dim_k A_s = 1$. Since $\tilde \phi$ is an epimorphism of $\S$-graded
$k$-vector spaces, its homogeneous component $\tilde \phi_s: A_s \to k^\alpha[\S]_s$ is
also surjective. This is a linear transformation between one dimensional $k$-vector
spaces, so it is an isomorphism, which implies that $\tilde \phi$ is an isomorphism.
\end{proof}

We now define the class of algebras on which we will focus.
\begin{Definition}
Let $A$ be a $k$-algebra, $\S$ an affine semigroup and $\alpha: \S \times \S \to k^\times$ 
a function. We say that $A$ has an $(\S,\alpha)$-basis if:
\begin{enumerate}
 \item There is an injective map $\S \to A$, $s \in \S \mapsto b_s \in
A$, with $b_0 = 1$. We call $b_s$ the \emph{standard $s$-element} of $A$.

 \item The set $\{b_s|s \in \S\}$ is a $k$-basis of $A$, called the \emph{standard basis}.

 \item For all $s,s',s'' \in \S$ with $s'' < s+ s'$ there exists $d_{s,s'}^{s''} \in k$
 such that
\begin{align*}
b_s b_{s'} = \alpha(s,s') b_{s+s'} + \sum_{s'' < s+s'} d_{s,s'}^{s''} b_{s''}
\end{align*}
\end{enumerate}
\end{Definition}
From now on we fix an algebra $A$ with an $(\S,\alpha)$-basis. For every $1 \leq i \leq r$ 
denote $b_i = b_{s_i}$, and for every $p = (p^1, \ldots, p^r) \in \NN_0^r$ denote $b^p =
b_1^{p^1} b_2^{p^2}\ldots b_r^{p^r}$.

Let $s, s', s'' \in \S$. By comparing the two associators $(b_s b_{s'})b_{s''}$ and $ b_s 
(b_{s'}b_{s''})$, we see that for every $t < s + s' + s''$ there exist $d_t, d'_t \in k$
such that
\begin{align*}
 \alpha(s,s')&\alpha(s + s',s'') b_{s+s'+s''} + \sum_{t < s+s'+s''} d_t b_t \\
  &=  \alpha(s,s'+s'')\alpha(s',s'') b_{s+s'+s''} + \sum_{t< s+s'+s''} d'_t b_t.
\end{align*}
Since the standard elements are linearly independent, $\alpha$ must be a
$2$-cocyle. 

Next we prove that $A$ can be given an $\S$-filtration such that its associated graded
algebra is isomorphic to the twisted semigroup algebra $k^\alpha[\S]$.
\begin{Lemma}
\label{varia-dominated} 
For each $s \in \S$ set $F_sA = \langle b_{s'}|s'\leq s \rangle$.
The set $\F= \{F_sA\}_{s \in \S}$ is an exhaustive $\S$-filtration of $A$, and its
associated graded algebra $\gr_\F A$ is isomorphic to $k^\alpha[\S]$ as an $\S$-graded
$k$-algebra.
\end{Lemma}
\begin{proof}
By hypothesis, the product of two elements $b_{t}$ and $b_{t'}$ with $t \leq s$ and $t' \leq
s'$ is a linear combination of standard elements $b_{s''}$ with $s'' \leq t + t' \leq s +
s'$ and hence lies in $F_{s+s'}A$. This proves that $\F$ is an $\S$-filtration on $A$, and
since the standard elements form a $k$-basis of $A$, it is exhaustive.

Let $B = \gr_\F A$. This is an $\S$-graded $k$-algebra, and $B_s = F_sA /F_{ < s}A$ for
each $s \in \S$. Since $F_sA = \langle b_s \rangle \oplus F_{<s}A$, the component $B_s$ is 
generated by $\gr b_s$, so $\{\gr b_s| s \in \S\}$ generates $B$ over $k$. Furthermore,
\begin{align*}
b_s b_{s'} \equiv \alpha(s,s') b_{s+s'} \mod F_{<s+s'}A
\end{align*}
so $\gr b_s \gr b_{s'} = \alpha(s,s') \gr b_{s+s'}$. Thus the vector space morphism
$k^\alpha[\S] \to A$ induced by the map $x^s \mapsto \gr b_s$ is an isomorphism of
$\S$-graded algebras.
\end{proof}

For the rest of this section $\F = \{F_sA\}$ denotes the filtration defined in Lemma
\ref{varia-dominated} 
\begin{Lemma}
\label{monomial-basis}
  For every $p = (p^1, \ldots, p^r) \in \NN_0^r$ the element $b^p$ belongs to $F_{\pi(p)}A
  \setminus F_{<\pi(p)}A$. If for every $s \in \S$ we choose an element $p_s \in
  \pi^{-1}(s)$ then the set $\{b^{p_t}|t \leq s\}$ is a basis of $F_{s}A$.
\end{Lemma}
\begin{proof}
  Let $B = \gr_\F A \cong k^\alpha[\S]$. We prove the first statement by induction on $n =
  p^1 + \ldots + p^r$, with the case $n = 0$ being obvious. Let $l = \max \{p^i \neq 0
  \mid 1 \leq i \leq r \}$ and let $q = p - e_l$. By
  the inductive hypothesis $b^q \in F_{\pi(q)}A \setminus F_{< \pi(q)}A$, which implies
  that $\gr b^q \in B_{\pi(q)}$. Since $B$ is integral, $(\gr b^q) (\gr b_l)
  \in B_{\pi(p)} \setminus 0$, or equivalently $b^p = b^q b_l \in F_{\pi(p)}A \setminus
  F_{<\pi(p)}A$. 
  
  Since the homogeneous components of $B$ are one dimensional, $\gr b^p$ generates
  $B_{\pi(p)}$ and so $\{\gr b^{p_s}|s \in \S\}$ is a homogeneous basis of $B$. By Lemma
  \ref{s-filtered-basis} the set $\{b^{p_t}|t \leq s\}$ is a basis of $F_sA$.
\end{proof}

For every $1 \leq l \leq r'$ let $c_l$ and $c'_l$ be as in the preamble to Lemma
\ref{semigroup-presentation}. The fact that the map $\phi: k^\alpha[\S] \to \gr_\F A$ from
the aforementioned lemma is an isomorphism implies 
\begin{align*}
\alpha(s_j,s_i) b_i b_j - \alpha(s_i,s_j) b_j b_i &\equiv 0 \mod F_{<s_i + s_j}A &
\mbox{for } 1 &\leq i < j \leq r,\\
c'_l b^{p_l} - c_l b^{p'_l} &\equiv 0 \mod F_{<\pi(p_l)}A &\mbox{for } 1 &\leq l \leq r'.
\end{align*}
Now choose for every $s \in \S$ an element $p_s \in \pi^{-1}(s)$. Lemma
\ref{monomial-basis} guarantees that for all $1 \leq i < j \leq r$ and $s < s_i + s_j$
there exist scalars $d^s_{i,j}$ such that 
\begin{align*}
\alpha(s_j, s_i)b_i b_j - \alpha(s_i,s_j)b_j b_i &= \sum_{s < s_i + s_j} d_{i,j}^{s}
b^{p_s},
\end{align*}
and for all $1 \leq l \leq r'$ and $t < \pi(p_l)$ there exist scalars $c^t_l$ such that
\begin{align*}
c'_l b^{p_l} - c_l b^{p_l'} &= \sum_{t < \pi(p_l)} c_l^{t} b^{p_t}.
\end{align*}

\begin{Theorem}
\label{deformation}
Let $\S$ be a positive affine semigroup and let $A$ be a $k$-algebra with an
$(\S,\alpha)$-basis. There exists an $\NN_0$-filtration on $A$ such that its associated
graded algebra is isomorphic to $k^\alpha[\S]$.
\end{Theorem}
\begin{proof}
Keeping the notation from the previous paragraph, let $\Sigma \subset \S$ be the set
formed by:
\begin{itemize}
	\item the elements $s_i + s_j$, for $1 \leq i < j \leq r$,
	\item the elements of the form $\pi(p_l)$ for $1 \leq l \leq r'$,
	\item the elements $s \in \S$ such that $d_{i,j}^s \neq 0$ for some $1 \leq i < j
		\leq r$,
	\item the elements $t \in \S$ such that $c_l^t \neq 0$ for some $1 \leq l \leq
		r'$.
\end{itemize}
For any $s = (s^1, \ldots, s^N) \in \NN_0^N$ let $||s||_\infty$ denote the
maximum of the $s^i$'s. We fix
$$K = \max \{||s||_{\infty}, s \in \Sigma\} + 1,$$ 
which makes sense because $\Sigma$ is a finite set. Let $\gamma: \NN_0^N \to \NN_0$ be the
semigroup morphism defined by $e_i \mapsto K^{N-i}$ for $1 \leq i \leq N$. The image of
any element $s \in \Sigma$ is the integer with $K$-adic expansion $s$. In particular if $s, 
s' \in \Sigma$ then $s' < s$ if and only if $\gamma(s') < \gamma(s)$.

Given $n \in \NN_0$ let 
\begin{align*}
G_nA &= \sum_{\{s \in \S| \gamma(s) \leq n\}} F_sA.
\end{align*}
Lemma \ref{varia-dominated} together with the fact that $\gamma$ is a semigroup morphism 
imply that $\mathcal G = \{G_nA\}_{n \in \NN_0}$ is an exhaustive $\NN_0$-filtration on 
$A$. Let $B = \gr_\G A$, and for every $p = (p^1, \ldots, p^r) \in \NN_0^r$ let $(\gr b)^p
= (\gr b_1)^{p^1} \ldots (\gr b_r)^{p^r}$. 

Since $\gamma$ is strictly increasing over $\Sigma$, 
\begin{align*}
\label{pre-eq}
\tag{$\dagger$}
  \alpha(s_j,s_i) b_i b_j - \alpha(s_i,s_j) b_j b_i &\equiv 0 \mod G_{\gamma(s_i + s_j)-1}A &
 \mbox{for } 1 \leq i < j \leq r\\
 c'_l b^{p_l} - c_l b^{p_l'} &\equiv 0 \mod G_{\gamma(\pi(p_l))-1}A & \mbox{for } 1 \leq l
 \leq r'.
\end{align*}

Also $G_{\gamma(s)}A = F_sA$ for every $s \in \Sigma$. In particular $G_{\gamma(s)-1}A
\subset F_{<s}A$, and by Lemma \ref{monomial-basis} the following hold:
\begin{align*}
b_i b_j, b_j b_i &\in F_{s_i + s_j}A \setminus F_{<s_i + s_j}A \subset G_{\gamma(s_i +
s_j)}A \setminus G_{\gamma(s_i + s_j)-1}A &\mbox{for every } 1 &\leq i < j \leq r \\
b^{p_l}, b^{p'_l} & \in F_{\pi(p_l)}A \setminus F_{<\pi(p_l)}A \subset G_{\gamma(\pi(p_l))}A
\setminus G_{\gamma(\pi(p_l)) -1}A & \mbox{for every } 1 &\leq l \leq r'.\\
\end{align*}
From this and the congruences in (\ref{pre-eq}) the following equalities hold in $B$:
\begin{align*}
  \alpha(s_j,s_i) \gr b_i \gr b_j - \alpha(s_i,s_j) \gr b_j \gr b_i &= 0 \\
 c'_l (\gr b)^{p_l} - c_l (\gr b)^{p'_l} &= 0.
\end{align*}

By Lemma \ref{semigroup-presentation} there is a surjective $k$-algebra morphism $\psi: 
k^\alpha[\S] \to B$ with $\psi(x^{s_i}) = \gr b_{i}$. We now put an $\NN_0$-grading on
$k^\alpha[\S]$, setting $\deg x^s = \gamma(s)$. With this grading $\psi$ is an
$\NN_0$-graded algebra morphism. Since $\S \subset \NN_0^N$, we know that $\dim_k 
k^\alpha[\S]_n = \#\{s \in \S|\gamma(s) = n\}$ is finite for every $n \in \NN_0$. On the
other hand Lemma \ref{monomial-basis} implies that $G_nA = \langle b^{p_s}|\gamma(s) \leq
n \rangle$ for every $n \in \NN_0$, so $B_n = \langle \gr b^{p_s}| \gamma(s) = n \rangle$. 
Since $\psi_n: k^\alpha[\S]_n \to B_n$ is a surjective morphism between $k$-vector spaces
of the same dimension, it is an isomophism, and the same holds for $\psi$.
\end{proof}

We recall that $\S \subset \NN_0^N$ is said to be \emph{normal} if whenever
there exist $n \in \NN$ and $s \in \NN_0^N$ such that $n s \in \S$, then $s \in \S$. The
previous Theorem has the following consequence:
\begin{Corollary}
\label{transfer-regularity}
	Let $\S$ be a positive affine semigroup and let $\alpha: \S \times \S \to
	k^\times$ be a $2$-cocycle. Suppose $A$ is a connected $\NN_0$-graded
	$k$-algebra with an $(\S,\alpha)$-basis, and that the standard elements of this
	basis are homogeneous. Then $A$ has proprety $\chi$ and finite local dimension as
	an algebra and it has a dualizing complex. Furthermore, if $\S$ is normal, then
	$A$ is AS Cohen Macaulay.
\end{Corollary}
\begin{proof}
	Since for every $n \in \NN_0$, $G_n A$ as defined in Theorem \ref{deformation} is
	generated by homogeneous elements, it is clear that it is a graded vector space,
	so $A$ is a GF-algebra and by the same theorem its associated graded ring is
	isomorphic to $k^\alpha[\S]$. 
	
	On the other hand, as proved in \cite{RZ2}, twisted semigroup algebras have
	property $\chi$ and finite local dimension as algebras , and also have balanced 
	dualizing complexes for any connected $\NN_0$-grading on them, so by items 1. and 
	2. of Theorem \ref{AS-transfer}, $A$ has these properties too.

	Finally, in \cite{RZ2} it is proven that if $\S$ is a normal affine semigroup then
	$k^\alpha[\S]$ is AS Cohen Macaulay, so by item 3. of Theorem \ref{AS-transfer},
	so is $A$.
\end{proof}

As another application of Theorem \ref{deformation}, we prove that algebras with an
$(\S,\alpha)$-basis are finitely presented. We carry over the notation for the elements
$\{p_s \in \NN_0^r| s \in \S|\}$ and the constants $d_{i,j}^s$ and $c_l^t$ defined just
before Theorem \ref{deformation}.
\begin{Corollary}
Consider in $T = k\langle X_i | i=1, \ldots, r\rangle$ the elements
\begin{align*}
\tilde C_{i,j}&=  \alpha(s_j, s_i)X_i X_j - \alpha(s_i,s_j)X_j X_i - \sum_{s < s_i +
s_j} d_{i,j}^{s} X^{p_s} &(1 \leq i,j \leq r)\\
\tilde S_l&= c'_l X^{p_l} - c_l X^{p_l'} -
\sum_{s < \pi(\p)} c_l^{t} X^{p_s} &(1 \leq l \leq r')
\end{align*}
and let $\I$ be the ideal generated by $\tilde C_{i,j}, \tilde S_l$, with $1 \leq i < j \leq
r$, $1 \leq l \leq r'$. The natural map $\rho :T/\I \to A$ is an isomorphism
\end{Corollary}
\begin{proof}
	By abuse of notation we write $X_i$ for the image of the generator $X_i$ of $T$ in
	the quotient $B = T/\I$.

	For each $n \in \NN_0$ let $G_nB = \rho^{-1}(G_nA)$. The set $\tilde \G =
	\{G_nB\}_{n \in \NN_0}$ is an $\NN_0$-filtration of $B$. By construction $\rho$ is
	a filtered strict morphism and it induces a morphism $\gr \rho: \gr_{\tilde \G} B
	\to \gr_\G A \cong k^\alpha[\S]$ that sends $\gr X_i$ to $x^{s_i}$.
	
	On the other hand, since for each $s \in \S$ the element $\rho(X^{p_s})$ is in
	$G_{\gamma(s)}A$,
	\begin{align*}
 		\alpha(s_i,s_j)X_j X_i &\equiv \alpha(s_j, s_i)X_i X_j \mod
		G_{\gamma(s_i + s_j)-1}B
		&\mbox{for } 1 &\leq i<j \leq r \mbox{ and}\\
		c'_l X^{p_l} &\equiv c_l X^{p_l'} \mod G_{\gamma(\pi(p_l))-1}B & \mbox{for
		} 1 &\leq l \leq r'
	\end{align*}
	so by Lemma \ref{semigroup-presentation} the assignation $x^{s_i} \in
	k^\alpha[\S]  \mapsto \gr X_i \in \gr_{\tilde \G} B$ defines a morphism of
	$k$-algebras $\tau: k^\alpha[\S] \to \gr_{ \tilde \G} B$, which is clearly an
	inverse to $\gr \rho$, so $\gr \rho$ is an isomorphism. Finally from item 1. of
	Lemma \ref{filtered-complex} we deduce that $\rho$ is also an isomorphism.
\end{proof}

\subsection{Quantum Flag varieties and Schubert cells}
In this subsection we adapt Caldero's arguments from \cite{C} to show that Schubert cells
of arbitrary quantum flag varieties have $(\S,\alpha)$-bases, and that in each case $\S$
is a normal affine semigroup.

We review the definitions which are relevant to our purpose. Let $\g$ be a semisimple Lie
algebra over $k$ and let $G$ be its Lie group. Write $P$ for the weight lattice of $\g$, 
$\{p_1, \ldots, p_n\}$ for its
fundamental weights, $P^+ = \sum_i \NN_0 p_i$ for the set of dominant weights, and
$\alpha_1, \ldots,\alpha_n$ for the positive roots of $P$. Let $W$ be the Weyl group of
$\g$, and $s_i \in W$ the reflection corresponding to the $i$-th root. Given an element $w
\in W$ we denote its length by $\ell(w)$; let $N$ be the length of the longest word of
$W$. A \emph{decomposition} of $w \in W$ is a word on the generators $s_i$ that equals $w$
in $W$. A decomposition of the longest word of $W$ is said to be \emph{adapted} to $w$ if
it is of the form $s_{i_1} \ldots s_{i_N}$ with $s_{i_1} \ldots s_{i_{\ell(w)}} = w$. For
every element $w \in W$ there is a decomposition of the longest word of $W$ adapted to
$w$.

Fix $q \in k^\times$. Let $U_q(\g)$ be the quantum enveloping algebra of $\g$ as in
\cite{J}*{Definition 4.3}. The algebra $U_q(\g)$ is generated by elements $\{E_i, F_i,
K_i^{\pm 1}|i = 1, \ldots, n \}$ where $n$ is the rank of $P$. Denote by $U_q(\mathfrak b)$,
resp. $U_q(\mathfrak n)$, the subalgebra of $U_q(\g)$ generated by the elements $E_i, K_i$,
resp. $E_i$. For each $\lambda \in P^+$ there is an irreducible highest-weight
representation of $U_q(\g)$ which we denote by $V_q(\lambda)$. Each $V_q(\lambda)$
decomposes as the direct sum of weight spaces $\bigoplus_{\mu \in \Lambda}
V_q(\lambda)_\mu$, where $\Lambda$ is a finite subset of $P^+$. We fix a highest-weight 
vector $v_\lambda \in V_q(\lambda)_\lambda$. For proofs of these facts the reader is
refered to \cite{J}*{chapters 4,5}.

Let $I$ be a subset of the set of fundamental weights and set $\mathcal J(I) = \sum_{p_i
\notin I} \NN_0 p_i$. Denote by $W_I \subset W$ the subgroup generated by reflections
$s_{\alpha_i}$ with $p_i \in I$, and for each class in $W/W_I$ pick a representative of
smallest length. We call $W^I$ the set of these representatives. For each $w \in W^I$ and
$\lambda \in \mathcal J(I)$ the vector space $V_q(\lambda)_{w\lambda}$ has dimension $1$.
The \emph{Demazure module} $V_q(\lambda)_w$ is the $U_q(\mathfrak b)$-submodule of
$V_q(\lambda)$ generated by a vector of  weight $w\lambda$ in $V_q(\lambda)$.

Given a $k$-vector space $V$ we denote its dual space by $V^*$. Since $U_q(\g)$ is a Hopf
algebra, its dual is an algebra with convolution product induced by the coproduct of
$U_q(\g)$. There is a map $V_q(\lambda)^* \ot V_q(\lambda) \to U_q(\g)^*$ defined by
sending $\xi \ot v \in V_q(\lambda)^* \ot V_q(\lambda)$ to the linear functional
$c^\lambda_{\xi,v}$, defined by assigning to each $u \in U_q(\g)$ the scalar
$c^\lambda_{\xi,v}(u) = \xi(uv)$. Functionals of type $c^\lambda_{\xi,v}$ are called
\emph{matrix coefficients}.


We now review the definitions of quantum flag varieties. They were first defined by
Soibelman in \cite{S} and by Lakshmibai and Reshetkin in \cite{LR}.
If $U$ is the maximal borel subgroup of $G$ then $G/U$ is the full flag variety of $G$.
Let $C^+(\lambda)$ be the vector space of matrix coefficients of the form
$c^\lambda_{\xi,v_{\lambda}}$ in $U_q(\g)^*$ and set 
$$\O_q[G/U] := \bigoplus_{\lambda \in P^+} C^+(\lambda) \subset U_q(\g)^*.$$
This is a subalgebra of $U_q(\g)$, called the \emph{quantum full flag variety} of $G$. The 
above decomposition as a direct sum gives $\O_q[G/U]$ the structure of a $P^+$-graded
algebra.

To every subset $I$ of the set of fundamental weights corresponds a parabolic subgroup
$P_I$, and the variety $G/P_I$ is the corresponding generalized flag variety. To this data 
we associate the $P^+$-graded subalgebra of $\O_q[G/U]$
$$\O_q[G/P_I]:= \bigoplus_{\lambda \in \mathcal J(I)} C^+(\lambda)$$
called \emph{quantum flag variety associated to $I$}. The case $I = \emptyset$
corresponds to the full flag variety.

Given vector spaces $V_2 \subset V_1$, we denote by $V_2^\perp \subset V_1^*$ the set of
linear functionals over $V_1$ which are zero over $V_2$. For every $w \in W^I$ the vector
space 
$$J_w^I = \bigoplus_{\lambda \in \mathcal J(I)} k-\mathsf{span}\{c_{\xi,v_\lambda}^\lambda \in C^+(\lambda)|\xi  \in V_q(\lambda)_w^\perp \} \subset \O_q[G/P_I]$$ 
is an ideal of $\O_q[G/P_I]$ called the \emph{Schubert ideal} associated to $w$. The
quotient algebra $\O_q[G/P_I]_w = \O_q[G/P_I]/J^I_w$ is called the \emph{quantum Schubert
variety} associated to $w$.

We assume from now on that $\mathsf{char}\ k = 0$ and that $q$ is not a root of unity.
The subalgebra $U_q(\n) \subset U_q(\g)$ has a so called \emph{canonical basis} $\B$,
discovered by Lusztig and Kashiwara. For each $\lambda \in P^+$ there is a subset
$\B_\lambda \subset \B$ such that $\# \B_\lambda = \dim V_q(\lambda)$ and $\B_\lambda
v_\lambda \subset V_q(\lambda)$ is a basis of $V_q(\lambda)$. Denote by $\B_\lambda^*$ the
basis of $V_q(\lambda)^*$ dual to $B_\lambda v_\lambda$, and by $\xi_b$ the element of
$\B_\lambda^*$ defined by the formula $\xi_b(b'v_\lambda) = \delta_{bb'}$ for all $b' \in
\B_\lambda$. The reader is refered to \cite{K}*{sections 8 and 12} for proofs of these facts
and further references. We recall the following result:

\begin{Theorem}\cite{C}*{Theorem 1.8}
\label{Demazure-basis}
	For every $w \in W^I$ there is a set $\B_w \subset \B$ such that $(\B_w \cap
	\B_\lambda)v_\lambda$ is a basis of $V_q(\lambda)_w$. Furthermore the set
	$\{\xi_b| b \in \B_\lambda \setminus \B_w \}$ is a basis of
	$V_q(\lambda)_w^\perp$, and the restrictions of the functionals $\{\xi_b|b \in B_w \cap
	B_\lambda\}$ to the Demazure submodule $V_q(\lambda)_w$ form a basis of $V_q(\lambda)_w^*$.
\end{Theorem}

Littelman proved in \cite{Lit}*{Proposition 1.5, a)} that for every decomposition
$w_0$ of the longest word of $W$ there is a parametrizaton of the canonical basis $\B$ by a
set $\S_{w_0} \subset \NN_0^N$, where $N$ is the length of the longest word of $W$. 
For each $s \in \S_{w_0}$ let $b_s$ denote the corresponding element in the canonical basis.

Let $w \in W^I$, and let $w_0$ be a decomposition of the longest word of $W$ adapted to
$w$. Following Caldero we set
\begin{align*}
	\tilde \S_{w_0} &:= \{(s,\lambda)| s \in \mathcal \S_{w_0}, b_s \in \B_\lambda\} 
		\subset \NN_0^N \times P^+ \cong \NN_0^{N+n},\\
	\tilde \S_{w_0}^w &:= \{(s,\lambda) \in \tilde \S_{w_0} |b_s \in \B_\lambda \cap
		\B_w\},  \\
	\tilde \S_{w_0,I}^w &:= \{(s,\lambda) \in \tilde S_{w_0}^w| \lambda \in \mathcal
		J(I)\} = \tilde S_{w_0}^w \cap \mathcal J(I).
\end{align*}
\begin{Lemma}
\label{are-semigroups}
	The sets $\tilde \S_{w_0}, \tilde \S_{w_0}^w$ and $\tilde \S_{w_0, I}^w$ are
	normal affine semigroups.
\end{Lemma}
\begin{proof}
	For $\tilde \S_{w_0}$ and $\tilde \S_{w_0}^w$ see \cite{C}*{Theorem 2.2 and
	Theorem 2.4} respectively. Now, $\tilde \S_{w_0,I}^w$ is by definition $\tilde
	S_{w_0}^w \cap \mathcal J(I)$ and $\mathcal J(I) \subset P^+$ is a normal affine
	subsemigroup. Since the intersection of normal semigroups is normal, we are done.
\end{proof}

The following Theorem is a reformulation of results of \cite{C}, adapted to our needs:
\begin{Theorem}
\label{full-flag-basis}
Given $(s,\lambda) \in \tilde S_{w_0}$ set $b_s^\lambda := c^\lambda_{\xi_{b_s},v_\lambda} \in
C^+(\lambda)$. The set $\B_{w_0} := \{b_s^\lambda|(s,\lambda)\in \S_{w_0}\}$ is a basis of
$\O_q[G/U]$. Furthermore, given $(s,\lambda)$ and $(s',\lambda')$ in $\tilde S_{w_0}$, for every
$s < s+s'$ there exist $\alpha((s,\lambda), (s',\lambda')) \in k^\times$ and $d_{s,s'}^{s''}
\in k$ such that
		\begin{align}
		\label{product-formula}
		b_{s}^\lambda b_{s'}^{\lambda'} = \alpha((s,\lambda),(s',\lambda'))
		b_{s+s'}^{\lambda + \lambda'} + \sum_{s'' < s+s'} d_{s,s'}^{s''}
		b_{s''}^{\lambda + \lambda'}.
		\end{align}
\end{Theorem}
\begin{proof}
	If we fix a dominant weight $\lambda$, then $\{\xi_{b_s} |(s,\lambda) \in \tilde
	S_{w_0}\} = \B_\lambda^*$, and  hence $\{b_s^{\lambda} =
	c_{\xi_{b_s},v_\lambda}^\lambda| (s,\lambda) \in \tilde S_{w_0}\}$ is a basis of
	$C^+(\lambda)$. The union of these bases is a basis of
	$\O_q[G/U] = \bigoplus_{\lambda \in P^+}C^+(\lambda)$ and it is equal to $\B_{w_0}$. 
	Formula (1) is \cite{C}*{formula 2.1.1}. The fact
	that $\alpha((s,\lambda),(s',\lambda'))$ is nonzero, in fact a power of $q$, is a
	consequence of this formula and \cite{C}*{Theorem 2.3}.
\end{proof}
We now use Theorem \ref{full-flag-basis} to prove the result announced in the introduction:
\begin{Corollary}
Given $I$ a subset of the set of fundamental weights, $w \in W^I$ and $w_0$ a
decomposition of the longest word of $W$ adapted to $w$, the quantum Schubert variety
$\O_q[G/P_I]_w$ has an $(\tilde S_{w_0,I}^w, \alpha)$-basis $\B_{w_0,I}^w =
\{\overline{b_s^\lambda}|(s,\lambda) \in \S_{w_0,I}^w\}$, where $\overline b_s^\lambda$
denotes the class of $b_s^\lambda$ in the quotient. In particular, quantum Schubert
varieties are AS Cohen Macaulay and have dualizing complexes.
\end{Corollary}
\begin{proof}
	By Theorem \ref{Demazure-basis}, for every $\lambda \in P$ the vector space
	$C^+(\lambda)/(J^I_w \cap C^+(\lambda))$ is generated by the set
	$\{b^\lambda_s|(s,\lambda) \in \tilde S_{w_0,I}^w\}$. This proves that
	$B_{w_0,I}^w$ is a basis of $\O_q[G/P_I]_w$. The fact that it has the desired
	multiplicative property follows from formula \ref{product-formula} of Theorem
	\ref{full-flag-basis}. The last statement is a direct application of Corollary
	\ref{transfer-regularity}.
\end{proof}



\begin{bibdiv}
\begin{biblist}
\bib{AZ}{article}{
   author={Artin, M.},
   author={Zhang, J. J.},
   title={Noncommutative projective schemes},
   journal={Adv. Math.},
   volume={109},
   date={1994},
   number={2},
   pages={228--287},
}

\bib{B}{article}{
   author={Bj{\"o}rk, Jan-Erik},
   title={The Auslander condition on Noetherian rings},
   conference={title={ Ann\'ee},
	       address={Paris},
	       date={1987/1988},
	      },
   book={
       series={Lecture Notes in Math.},
       volume={1404},
       publisher={Springer},
       place={Berlin},
     },
   date={1989},
   pages={137--173},
}

\bib{BH}{book}{
   author={Bruns, Winfried},
   author={Herzog, J{\"u}rgen},
   title={Cohen-Macaulay rings},
   series={Cambridge Studies in Advanced Mathematics},
   volume={39},
   publisher={Cambridge University Press},
   place={Cambridge},
   date={1993},
   pages={xii+403},
}


\bib{C}{article}{
   author={Caldero, Philippe},
   title={Toric degenerations of Schubert varieties},
   journal={Transform. Groups},
   volume={7},
   date={2002},
   number={1},
   pages={51--60},
}

\bib{J}{book}{
   author={Jantzen, Jens Carsten},
   title={Lectures on quantum groups},
   series={Graduate Studies in Mathematics},
   volume={6},
   publisher={American Mathematical Society},
   place={Providence, RI},
   date={1996},
   pages={viii+266},
}

\bib{JZ}{article}{
   author={J{\o}rgensen, Peter},
   author={Zhang, James J.},
   title={Gourmet's guide to Gorensteinness},
   journal={Adv. Math.},
   volume={151},
   date={2000},
   number={2},
   pages={313--345},
}

\bib{K}{article}{
   author={Kashiwara, Masaki},
   title={On crystal bases},
   conference={
      title={Representations of groups},
      address={Banff, AB},
      date={1994},
	      },
   book={
      series={CMS Conf. Proc.},
      volume={16},
      publisher={Amer. Math. Soc.},
      place={Providence, RI},
              },
   date={1995},
   pages={155--197},
}

\bib{Lit}{article}{
   author={Littelmann, P.},
   title={Cones, crystals, and patterns},
   journal={Transform. Groups},
   volume={3},
   date={1998},
   number={2},
   pages={145--179},
}

\bib{LR}{article}{
   author={Lakshmibai, V.},
   author={Reshetikhin, N.},
   title={Quantum flag and Schubert schemes},
   conference={
      title={ physics},
      address={Amherst, MA},
      date={1990},
      },
   book={
      series={Contemp. Math.},
      volume={134},
      publisher={Amer. Math. Soc.},
      place={Providence, RI},
      },
   date={1992},
   pages={145--181},
}

\bib{RZ2}{article}{
  author={Rigal, L.},
  author={Zadunaisky, P.},
  title={Quantum Toric varieties (?)},
}

\bib{VO}{book}{
   author={N{\u{a}}st{\u{a}}sescu, Constantin},
   author={Van Oystaeyen, F.},
   title={Graded and filtered rings and modules},
   series={Lecture Notes in Mathematics},
   volume={758},
   publisher={Springer},
   place={Berlin},
   date={1979},
   pages={x+148},
}

\bib{NV}{book}{
   author={N{\u{a}}st{\u{a}}sescu, Constantin},
   author={Van Oystaeyen, Freddy},
   title={Methods of graded rings},
   series={Lecture Notes in Mathematics},
   volume={1836},
   publisher={Springer-Verlag},
   place={Berlin},
   date={2004},
   pages={xiv+304},
}

\bib{S}{article}{
   author={So{\u\i}bel{\cprime}man, Ya. S.},
   title={On the quantum flag manifold},
   language={Russian},
   journal={Funktsional. Anal. i Prilozhen.},
   volume={26},
   date={1992},
   number={3},
   pages={90--92},
   translation={
       journal={Funct. Anal. Appl.},
       volume={26},
       date={1992},
       number={3},
       pages={225--227},
 	},
}


\bib{vdB}{article}{
   author={van den Bergh, Michel},
   title={Existence theorems for dualizing complexes over non-commutative
   graded and filtered rings},
   journal={J. Algebra},
   volume={195},
   date={1997},
   number={2},
   pages={662--679},
}

\bib{W}{book}{
   author={Weibel, Charles A.},
   title={An introduction to homological algebra},
   series={Cambridge Studies in Advanced Mathematics},
   volume={38},
   publisher={Cambridge University Press},
   place={Cambridge},
   date={1994},
   pages={xiv+450},
}

\bib{Ye}{article}{
   author={Yekutieli, Amnon},
   title={Dualizing complexes over noncommutative graded algebras},
   journal={J. Algebra},
   volume={153},
   date={1992},
   number={1},
   pages={41--84},
}
\end{biblist}
\end{bibdiv}



\end{document}

Fix $n \in \NN$ and let $\phi: \ZZ^n \to \ZZ$ be a group morphism such that $\phi(\NN_0^n)
\subset \NN_0$. For every $\NN_0^n$ graded $k$-algebra $B$ we define $\phi_!(B)$ to be the 
$\NN_0$-graded algebra with the same underlying algebra as $B$, and homogeneous components
given by
\begin{align*}
  \phi_!(B)_m &= \bigoplus_{\{\xi \in \NN_0^n | \phi(\xi) = m\}} B_\xi & \mbox{for any } m
  \in \NN_0.
\end{align*}
We write ${}_!B$ instead of $\phi_!(B)$ to lighten up notation.
We say that $B$ is \emph{$\phi$-connected} if ${}_!B$ is a connected $\NN_0$-graded
algebra. This is equivalent to the fact that $B$ is a connected $\NN_0^n$-graded algebra
and that $\supp B \cap \ker \phi = \{(0, \ldots, 0)\}$. 

If $M$ is an object in $\Gr B$ then we define $\phi_!(M)$ to be the object in $\Gr {}_!B$
with the same underlying $B$-module structure as $M$ and homogeneous components given by
\begin{align*}
  \phi_!(M)_m &= \bigoplus_{\{\xi \in \ZZ^n | \phi(\xi) = m\}} M_\xi & \mbox{for any } m 
  \in \ZZ.
\end{align*}
We also write ${}_!M$ instead of $\phi_!(M)$.
If $N$ is another object in $\Gr B$ and $f \in \GrHom_B(N,M)$ is a homogeneous morphism
of degree $\xi \in \ZZ^n$, then $f: {}_!N \to {}_!M$ is homogeneous of degree $\phi(\xi)$,
so there is a functor $\phi_!: \Gr B \to \Gr {}_!B$ that sends an object $M$ to ${}_!M$ 
and an homogeneous $B$-linear morphism to itself. Since $\phi_!$ leaves the underlying 
$B$-linear structures intact and only changes the gradings, it is an additive and exact
functor. 

\begin{Lemma}
\label{phi-functor}
 Let $\phi: \ZZ^n \to \ZZ$ be a group morphism such that $\phi(\NN_0^n) \subset \NN_0$, and 
 let $B$ be a noetherian $\phi$-connected $\NN_0^n$-graded algebra. Given an object $M$ in
 $\Gr B$, the following hold:
 \begin{enumerate}
   \item For every object $N$ in $\Gr B$ and every $i \geq 0$ there exists a natural morphism 
   $\GrExt^i_B(N,M) \to \GrExt^i_{{}_! B}({}_!N,{}_!M)$. If $N$ is finitely generated, 
   this is an isomorphism.

   \item The torsion submodule $\GG_\m({}_!M)$ can be given the structure of a
   $\ZZ^n$-graded $B$-submodule of $M$. Thus the torsion functor $\GG_\m: \Gr {}_! B \to \Gr {}_!B$ 
   induces a functor $\GG_\m: \Gr B \to \Gr B$ such that for every $i \geq 0$ the following 
   diagrams of functors commute:
   \begin{align*}
  \xymatrix{
	\Gr B \ar[r]^-{\GG_\m} \ar[d]_-{\phi_!}& \Gr B\ar[d]^-{\phi_!} & &
	\Gr B \ar[r]^-{\R^i \GG_\m} \ar[d]_-{\phi_!}& \Gr B\ar[d]^-{\phi_!} \\
	\Gr {}_!B \ar[r]^-{\GG_{\m}} & \Gr {}_!B & &
	\Gr {}_!B \ar[r]^-{H^i_{\m}} & \Gr {}_!B
  }	
   \end{align*}

   \item The graded injective dimension of $B$ is equal to the graded injective 
   dimension of ${}_!B$. 
   
   \item If $M$ is finitely generated, then its graded projective dimension is equal to
   the projective dimension of ${}_!M$. The global dimension of ${}_!B$ is equal to the 
   graded projective dimension of the trivial graded $B$-module $k$.

  \item The algebra ${}_!B$ has finite local dimension as an algebra if and only if the
   set $$\{\ldim {}_! M | M \mbox{ is an object of $\grmod B$}\} \subset \NN_0$$ is bounded.

   \item If ${}_!B$ has finite local dimension, then ${}_!B$ has property $\chi$ as an
   algebra if and only if every graded ${}_!B$-module of the form ${}_!M$ for some object
   $M$ in $\Gr B$ has property $\chi$.
 \end{enumerate}
\end{Lemma}
\begin{proof}
All of these items should be proved in \cite{RZ2}.
\end{proof}



The following remark is the following remark
\begin{Remark}
\label{multigraded-regularity}. 
Suppose $B$ is a noetherian connected $\NN_0^n$-graded algebra.

\begin{itemize}
\item
Let $\phi: \ZZ^n \to \ZZ$ be the group morphism that sends each element $\xi = (\xi^1, 
\ldots, \xi^n) \in \ZZ^n$ to $\xi^1 + \ldots + \xi^n$. Then $B$ is $\phi$-connected, in
particular there is always a group morphism $\phi$ such that $B$ is $\phi$-connected.

\item
By item 2. of the previous lemma,
the local cohomology modules of ${}_!B$ and ${}_!B^\opp$ are independent of the chosen
morphism $\phi$, so it makes sense to say that $B$ is AS Cohen Macaulay if an only if
$\phi_!(B)$ is AS Cohen Macaulay for some, or any, group morphism $\phi$ such that $B$ is
$\phi$-connected.

\item
By items 1. and 3. of the lemma, it makes sense to say that $B$ is AS Gorenstein if and
only if ${}_!B$ is AS Gorenstein for some (or any) morphism $\phi$ such that $B$ is
$\phi$-connected. 

\item
Being AS regular is also independent of $\phi$, since the global dimension of $B$ and
${}_!B$ coincide (after all they are the same algebra with different gradings). Item
4. of the lemma shows that AS regularity can be detected simply by looking at
$\ZZ^n$-graded $B$-modules.

\item
By item 1. of the lemma, given an object $M$ in $\Gr B$, for every $i \geq 0$ there is a 
vector space isomorphism $\GrExt_B^i(k,M) \cong \GrExt_{{}_!B}^i(k,M)$, so the graded 
${}_!B$-module 
${}_!M$ has property $\chi$ if and only if $\GrExt^i_B(k,M)$ is finite dimensional for
every $i \geq 0$. In that case we simply say that $M$ has property $\chi$. Also we may set
$\depth M = \depth {}_!M$.

\item
Finally by Proposition \ref{vdb-criterion} and items 5. and 6. of the lemma, ${}_!B$ has a 
balanced dualizing complex if and only if all $\ZZ^n$-graded modules have property $\chi$
and there is a global bound for their local dimensions.
\end{itemize}
\end{Remark}





