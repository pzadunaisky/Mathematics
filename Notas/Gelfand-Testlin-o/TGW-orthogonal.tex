%%%%%%%%%%%%%%%%%%%%%% Generalities %%%%%%%%%%%%%%%%%%5
\documentclass[11pt,fleqn]{amsart}
\usepackage[paper=a4paper]
 {geometry}

\pagestyle{plain}
\pagenumbering{arabic}
%%%%%%%%%%%%%%%%%%%%%%%%%%%%%%%%
\usepackage[small]{titlesec}
\usepackage{hyperref}

\usepackage{amsthm,thmtools}
\usepackage{showlabels}
\linespread{1.05}
\setlength{\parskip}{1.2ex}

\usepackage[utf8]{inputenc}
\usepackage[english]{babel}
\usepackage{enumerate}
\usepackage[osf,noBBpl]{mathpazo}
\usepackage[alphabetic,initials]{amsrefs}
\usepackage{amsfonts,amssymb,amsmath}
\usepackage{mathtools}
\usepackage{graphicx}
\usepackage[poly,arrow,curve,matrix]{xy}
\usepackage{wrapfig}
\usepackage{xcolor}
\usepackage{helvet}
\usepackage{stmaryrd}
\usepackage{tikz}
\usepackage{mathdots}

%%%%%%%%%%%%%%%%%%%%%%%%%%% Theorems et al%%%%%%%%%%%%%%%%%%%%%%%%%%
\declaretheoremstyle[
%headformat=swapnumber, 
bodyfont=\itshape,]{mystyle}
\declaretheorem[name=Lemma, style=mystyle, numberwithin=section]{Lemma}
\declaretheorem[name=Proposition, style=mystyle, sibling=Lemma]{Proposition}
\declaretheorem[name=Theorem, style=mystyle, sibling=Lemma]{Theorem}
\declaretheorem[name=Corollary, style=mystyle, sibling=Lemma]{Corollary}
\declaretheorem[name=Definition, style=mystyle, sibling=Lemma]{Definition}
%\declaretheorem[name=Example(s), style=mystyle]{Example}
%\declaretheorem[name=Remark, style=mystyle]{Remark}

% unnumbered versions
\declaretheoremstyle[numbered=no, 
bodyfont=\itshape]{mystyle-empty}
\declaretheorem[name=Lemma, style=mystyle-empty]{Lemma*}
\declaretheorem[name=Proposition, style=mystyle-empty]{Proposition*}
\declaretheorem[name=Theorem, style=mystyle-empty]{Theorem*}
\declaretheorem[name=Corollary, style=mystyle-empty]{Corollary*}
\declaretheorem[name=Definition, style=mystyle-empty]{Definition*}
\declaretheorem[name=Example(s), style=mystyle-empty]{Example*}
\declaretheorem[name=Remark, style=mystyle-empty]{Remark*}

%%%%%%%%%%%%%%%%%%%%%%%%%%% Paragraphs %%%%%%%%%%%%%%%%%%%%%%%%%%%%%
\renewcommand\thesection{\arabic{section}}
\titleformat{\section}
 {\normalfont\bfseries\large}
 {\thesection. \space}{0em}{}

\renewcommand\thesubsection{\arabic{subsection}}
\titleformat{\subsection}
 {\normalfont\bfseries}
 {\thesection.\thesubsection \space}{0em}{}

\renewcommand\proofname{Proof}

\makeatletter
\renewenvironment{proof}[1][\textit{\proofname}]{\par
 \pushQED{\qed}%
 \normalfont \topsep.75\paraskip\relax
 \trivlist
 \item[\hskip\labelsep
 \itshape
 #1\@addpunct{.}]\ignorespaces
}{%
 \popQED\endtrivlist\@endpefalse
}
\makeatother

\newskip\paraskip
\paraskip=0.75ex plus .2ex minus .2ex

\newcounter{para}[section]
\setcounter{para}{0}
\renewcommand\thepara{\thesection.\arabic{para}}
\def\paragraph{%
 \noindent
 \refstepcounter{para}%
 \textbf{\thepara.}\hspace{1ex}%
}

\newcommand\about[1]{%
 {\bfseries#1.}%
}

\newcommand\pref[1]{\textbf{\ref{#1}}}

\renewcommand\theHpara{\theHsection.\arabic{para}}
%%%%%%%%%%%%%%%%%%%%%%%%%%% The usual stuff%%%%%%%%%%%%%%%%%%%%%%%%%
\newcommand\NN{\mathbb N}
\newcommand\CC{\mathbb C}
\newcommand\QQ{\mathbb Q}
\newcommand\RR{\mathbb R}
\newcommand\ZZ{\mathbb Z}
\newcommand\II{\mathbb I}
\renewcommand\k{\Bbbk}

\newcommand\maps{\longmapsto}
\newcommand\ot{\otimes}
\renewcommand\to{\longrightarrow}
\renewcommand\phi{\varphi}
\newcommand\Id{\mathsf{Id}}
\newcommand\im{\mathsf{im}}
\newcommand\coker{\mathsf{coker}}
%%%%%%%%%%%%%%%%%%%%%%%%% Specific notation %%%%%%%%%%%%%%%%%%%%%%%%%
\newcommand\A{\mathcal A}

\newcommand\g{\mathfrak g}
\newcommand\h{\mathfrak h}
\newcommand\gl{\mathfrak{gl}}
\newcommand\so{\mathfrak{so}}
\newcommand\ii{\mathtt{i}}

\newcommand\vectspan[1]{\left\langle #1 \right\rangle}
\newcommand\interval[1]{\llbracket #1 \rrbracket}
\newcommand\floor[1]{\left\lfloor#1\right\rfloor}

\DeclareMathOperator\Frac{Frac}
\DeclareMathOperator\Specm{Specm}
\DeclareMathOperator\End{End}

\renewcommand\labelitemi{--}
%%%%%%%%%%%%%%%%%%%%%%%%%%%%%%%%%%%%%% TITLES %%%%%%%%%%%%%%%%%%%%%%%%%%%%%%
\title{Twisted generalized Weyl algebras and Gelfand-Tsetlin theory for 	
	orthogonal Lie algebras}
\author{[TGW-orthogonal.tex]}

\begin{document}
\maketitle

\noindent\textbf{Objectives:}
\begin{enumerate}
\item Introduce the rational skew twisted algebra $\A$.
\item Prove that $U(\so(\CC, n))$ embedds in $\A$.
\end{enumerate}

Our base field is the complex numbers, so all vector spaces, tensor products,
etc. are over $\CC$ unless explicitly stated.

\section{Preeliminaries on $\so(\CC, n)$}
\paragraph
\about{Notation}
\label{notation}
Given $n, m, k \in \ZZ$ we set $\interval{n,m} = \{l \in \ZZ \mid n \leq l 
\leq m \}$, $\interval{m,n}_k = \{(k,l) \in \ZZ^2 \mid l \in \interval{n,m}\}$
and $\interval{n} = \interval{1,n}$. Given two elements $x,y$ of a set $Z$, 
the symbol $\delta_{x,y}$ will denote the usual Kronecker delta.

\paragraph
\about{The orthogonal Lie algebra}
\label{orhtogonal-lie-algebra}
Fix $n \in \NN$. We denote by $\gl_n$ the general linear Lie algebra of $n$
by $n$ complex matrices. This algebra is generated as a vector space by the
elementary matrices $\{E_{i,j} \mid i,j \in \interval{n}\}$, and their product 
is given by $[E_{i,j}, E_{k,l}] = \delta_{j,k} E_{i,l} - \delta_{l,i} 
E_{k,j}$. 

As usual we denote by $\so_n$ the Lie subalgebra of $\gl_n$ formed by matrices 
$X$ such that $X^t = -X$. The elementary skew symmetric matrices are $F_{i,j} 
= E_{i,j} - E_{j,i}$, and the set $\{F_{i,j} \mid 1 \leq i < j \leq n\}$ is a
basis of $\so_n$ (notice that $F_{j,i} = - F_{i,j}$). The Lie product among 
these matrices is given by
\begin{align*}
[F_{i,j}, F_{k,l}]
	&= \delta_{j,k} F_{i,l} + \delta_{i,l} F_{j,k} 
		- \delta_{i,k} F_{j,l} - \delta_{j,l} F_{i,k}.
\end{align*}
The bracket of $\so_n$ can be easily remembered as follows: for any four 
different integers $i,j,k,l$ we have $[F_{i,j}, F_{j,k}] = F_{i,k}$ and 
$[F_{i,j}, F_{k,l}] = 0$. 

\paragraph
\label{psi-n}
We denote by $\Psi_n$ be the Lie algebra generated by $\{A_k \mid k \in 
\interval{n-1}\}$ modulo the relations
\begin{align*}
[A_{k}, A_{l}]
	&= 0 & \mbox{ if } |k-l| > 1; \\
[[A_{k}, A_{l}], A_{l}] 
	&= - A_{k} & \mbox{ if } |k-l| = 1; \\
[[[A_{k}, A_{k+1}], A_{k+2}], A_{k+1}] 
	&= 0 & \mbox{ for all } k \in \interval{n-2}.
\end{align*}
If we replace $A_k$ by $F_{k,k+1}$ in these formulas then we obtain equalities
that hold in $\so_n$. In other words there is a Lie algebra morphism $c_n:
\Psi_n \to \so_n$ given by $A_k \mapsto F_{k,k+1}$. We will prove that this 
map is an isomorphism. 

For each $k \in \interval{n-1}$ set $A_{k,k+1} = A_k$ and for each $j \in
\interval{k+1,n-1}$ set $A_{k,j+1} = [A_{k,j} A_{j,j+1}]$. By definition 
$c_n(A_{i,j}) = F_{i,j}$ for any $1 \leq i < j \leq n$. If we set $A_{i,i} = 0$
and $A_{j,i} = - A_{i,j}$ then $c_n(A_{i,j}) = E_{i,j}$ for any $i,j \in 
\interval n$. With this notation we can rewrite the defining relations of
$\Psi_n$ as
\begin{align}
[A_{i,i+1}, A_{k,k+1}]
	&= 0 & \mbox{ if } |i-k| > 1; \\
[A_{i,i+2}, A_{i+1,i+2}] 
	&= - A_{i,i+1} & \mbox{ for all } i \in \interval{n-2} \\
	[A_{i,i+1}, A_{i,i+2}] 
	&= - A_{i+1,i+2}; & \mbox{ for all } i \in \interval{n-2}  \\
[A_{i,i+3}, A_{i+1,i+2}] 
	&= 0 & \mbox{ for all } i \in \interval{n-3}.
\end{align}
\begin{Lemma*}
The following equalities hold in $\Psi_n$ for all $1 \leq i < j < k < l \leq 
n$.
\begin{align*}
[A_{i,j}, A_{k,l}]
	&= 0; 
&[A_{i,j}, A_{j,k}]
	&= A_{i,k}; \\
[A_{i,l}, A_{j,k}] 
	&= 0.
&[A_{i,k}, A_{j,k}] 
	&= - A_{i,j}; \\
[A_{i,j}, A_{i,k}] 
	&= - A_{j,k}; 
&[A_{i,k}, A_{j,l}] 
	&=0
\end{align*}
\end{Lemma*}
\begin{proof}
The element $A_{i,j}$ belongs to the Lie subalgebra of $\Psi_n$ generated by
$A_i, A_{i+1}, \ldots, A_{j-1}$, while $A_{k,l}$ lies in the subalgebra 
generated by $A_k, A_{k+1}, \ldots, A_{l-1}$. Since the generators of the first
subalgebra commute with those of the second subalgebra, it follows that 
$[A_{i,j}, A_{k,l}] = 0$.

Let us prove that $[A_{i,j}, A_{j,k}] = A_{i,k}$ by induction on $k$, starting
at $k = i+2$. In that case $j = i+1$ and this holds by definition. Now 
assuming the result is true for some $k$, we have
\begin{align*}
[A_{i,j}, A_{j,k+1}]
	&= [A_{i,j}, [A_{j,k}, A_{k,k+1}]]
	= [[A_{i,j}, A_{j,k}], A_{k,k+1}] + [A_{j,k},[A_{i,j}, A_{k,k+1}]].
\end{align*}
Applying the inductive hypothesis to the first term and the first equality to
the second term this equals $[A_{i,k}, A_{k,k+1}] + 0 = A_{i,k+1}$, so we are 
done.

We now prove that $[A_{i,l}, A_{j,k}] = 0$. First consider the case $i = j-1, 
k = j+1, l = j+2$, which is the last defining relation of $\Psi_n$. Fix 
momentarily $k = j+1, l = j+2$ and suppose that the equality holds for some 
$i \in \interval{2, j-1}$. Then
\begin{align*}
[A_{i-1,j+2}, A_{j,j+1}]
	&=[[A_{i-1,i}, A_{i,j+2}], A_{j,j-1}] \\
	&=[[A_{i-1,i}, A_{j,j-1}], A_{i,j+2}] + [A_{i-1,i}, [A_{i,j+2},A_{j,j-1}]].
\end{align*}
The hypothesis implies that the second term is zero, and since $i < j$ the 
previous equality implies that the first term is zero. Knowing that the result
holds for all $i$, one can proceed simmilarly by induction on $l$. Once 
established that the result holds for arbitrary $i,l$, a similar induction on
$k-j$ proves the result in general.

Let us now prove that $[A_{i,k}, A_{k-1,k}] = - A_{i,k-1}$ by descending 
induction on $i$, starting at $i = k-2$, in which case this is just the second 
defining relation of $\Psi_n$. Using the previous two equalities and the
inductive hypothesis we get
\begin{align*}
[A_{i-1,k}, A_{k-1,k}]
	&= [[A_{i-1,i}, A_{i,k}], A_{k-1, k}]\\
	&= [[A_{i-1,i}, A_{k-1,k}], A_{i,k}] + [A_{i-1,i}, [A_{i,k}, A_{k-1,k}]] \\
	&= 0 - [A_{i-1,i}, A_{i,k-1}] = A_{i-1, k-1}.
\end{align*}
To complete the proof of the third equality, we do finite descending induction
on $j$. We have just proved the case $j = k-1$, and if the equality holds for
$j$ and $i < j-1$ then
\begin{align*}
[A_{i,k}, A_{j-1, k}]
	&= [A_{i,k}, [A_{j-1,j}, A_{j,k}]] \\
	&= [[A_{i,k}, A_{j-1,j}], A_{j,k}] + [A_{j-1,j},[A_{i,k}, A_{j,k}]] \\
	&= 0 - [A_{j-1,j}, A_{i,j}] = [A_{i,j}, A_{j-1,j}] = - A_{i,j-1}.
\end{align*}
A similar argument shows that $[A_{i,j}, A_{i,k}] = - A_{j,k}$.

Finally we prove that $[A_{i,k}, A_{j,l}] = 0$. We begin with the base case
\begin{align*}
[A_{j-1,j+1},A_{j,j+2}]
	&= [[A_{j-1,j}, A_{j,j+1}], A_{j,j+2}] \\
	&= [[A_{j-1,j}, A_{j,j+2}], A_{j,j+1}] + [A_{j-1,j},[A_{j,j+1}, A_{j,j+2}]]
		\\
	&= [A_{j-1,j+2}, A_{j,j+1}] + [A_{j-1,j}, -A_{j+1,j+2}] = 0.
\end{align*}
As before, we proceed by descending induction on $i$, by ascending induction on
$l$, and finally by induction on $k-j$.
\end{proof}

This has as an immediate consequence the following.
\begin{Proposition*}
The map $c_n: \Psi_n \to \so_n$ is an isomorphism.
\end{Proposition*}
\begin{proof}
It follows from the lemma that the vector space spanned by $\{A_{i,j} \mid 1 
\leq i < j \leq n-1\}$ is a Lie subalgebra of $\Psi_n$, and since it contains 
all $A_i$'s it is equal to $\Psi_n$. Since $c_n(A_{i,j}) = F_{i,j}$ the 
morphism $c_n$ maps a spanning set of $\Psi_n$ onto a basis of $\so_n$, and 
hence is an isomorphism.
\end{proof}


\section{The twisted generalized Weyl algebra $\A$}
Recall that we have fixed $n \in \NN$. 

\paragraph
\label{polynomial-algebra}
\about{Polynomial algebra}
For each $k \in \NN$ set $p(k) = \floor{\frac{k+1}{2}}$. Set $\mu = (p(1), 
p(2), \ldots, p(n)) = (1,1,2,2,3, \ldots) \in \NN^n$ and $\Sigma = 
\{(k,i) \mid 1 \leq k \leq n-1, 1 \leq i \leq p(k)\}$.

We put $\Lambda_k = \CC[\lambda_{k,i} \mid i \in \interval{p(k)}]$ and $\Lambda
= \CC[\lambda_{k,i} \mid 1 \leq k \leq n, 1 \leq i \leq p(k)]$. Also we denote 
by $L$ the fraction field of $\Lambda$.

The group $S_{p(k)}$ acts on $\Lambda_k$ by permuting the variables. We define
a ring automorphism of $\Lambda_k$ called $\epsilon_k$ as follows:
\begin{align*}
\lambda_{k,i} \stackrel{\epsilon_k}\longmapsto \lambda_{k,i}^* &= \begin{cases}
1 - \lambda_{k,i} & \mbox{ if } k \equiv 0 \mod 2 \\
- \lambda_{k,i} & \mbox{ if } k \equiv 1 \mod 2
\end{cases}
\end{align*}
We set $\epsilon = \prod_{i=1}^n \epsilon_k $, and for each $f \in L$ we write 
$f^*$ instead of $\epsilon(f)$. Notice that if we set $\rho_{k,i}$ to be 
$\lambda_{k,i} + \frac12$ for $k$ even and $\lambda_{k,i}$ for $k$ odd then 
$\rho_{k,i}^* = - \rho_{k,i}$ for all $(k,i)$. We set $G = \langle \epsilon 
\rangle \times \prod_{k=1}^{n} S_{p(k)}$, so $G$ acts on $\Lambda$ in the 
obvious way.

\paragraph
\label{the-algebra-a}
\about{The algebra $\A$}
Let $\ZZ^\mu = \ZZ^{p(1)} \times \cdots \times \ZZ^{p(n)}$, and write $e_{k,i}$
for the $i$-th element in the canonical basis of $\ZZ^{p(k)}$. The abelian
group $\ZZ^\mu$ acts on $\Lambda$ by $e_{k,i} \cdot \lambda_{l,j} = 
\lambda_{l,j} + \delta_{k,l}\delta_{i,j}$. Given $f \in \Lambda$ and $z \in 
\ZZ^\mu$ we will usually write $f^z$ or $f(\lambda+z)$ for $z \cdot f$. This 
extends to an action of $\ZZ^\mu$ on $L$, and we denote by $\A$ the smash 
product $L \# \ZZ^\mu$. This a $\CC$-algebra whose underlying space is the 
$L$-vector space with basis $\{t_z \mid z \in \ZZ^\mu\}$ and product $f t_z 
\cdot g t_{z'} = f g^z t_{z+z'}$. There is an injective map $\A \to 
\End_\CC(L)$ mapping $f t_z$ to the linear endomorphism given by $g \mapsto f 
g(\lambda + z)$.

For each $(k,i) \in \Sigma$ we set $t_{k,i} = t_{e_{k,i}}$, and set
\begin{align*}
p^X_{k,i}
	&= \begin{cases}
		\displaystyle
		\prod_{j=1}^{p(k+1)} (\lambda_{k,i}^* + \lambda_{k+1,j})
			(\lambda_{k,i}^* + \lambda_{k+1,j}^*)
			& \mbox{ if } k \equiv 0 \mod 2; \\
		\displaystyle
		\prod_{j=1}^{p(k+1)} (\lambda_{k,i}^* + \lambda_{k+1,j})
			\prod_{j=1}^{p(k-1)}(\lambda_{k,i}^* + \lambda_{k-1,j}^*)
			& \mbox{ if } k \equiv 1 \mod 2.
	\end{cases} \\
p^Y_{k,i}
	&= \begin{cases}
		\displaystyle
		\prod_{j=1}^{p(k+1)} (\lambda_{k,i} + \lambda_{k-1,j})
			(\lambda_{k,i} + \lambda_{k-1,j}^*)
			& \mbox{ if } k \equiv 0 \mod 2; \\
		\displaystyle
		\prod_{j=1}^{p(k+1)} (\lambda_{k,i} + \lambda_{k+1,j})
			\prod_{j=1}^{p(k-1)}(\lambda_{k,i} + \lambda_{k-1,j}^*)
			& \mbox{ if } k \equiv 1 \mod 2.
	\end{cases} \\
q_{k,i}
	&= 
		\displaystyle
	\prod_{j\neq i} (\lambda_{k,i} - \lambda_{k,j})
		(\lambda_{k,i} - \lambda_{k,j}^*).
\end{align*}

We set
\begin{align*}
X_{k,i}
	&= \frac{p_{k,i}^X}{q_{k,i}};
	&Y_{k,i}
	&= \frac{p_{k,i}^Y}{h_{k,i} q_{k,i}};
\end{align*}
where $h_{k,i} = \lambda_{k,i}^2 (\lambda_{k,i}^2 -1/4)$ if $k$ is even and
$1$ if $k$ is odd.
\textcolor{red}{These elements satisfy Natalia's commutation relations}

For each $(k,i), (m,j) \in \Sigma$ set
\begin{align*}
\mu_{(m,j),(k,i)}
	= &\begin{cases}
		1 
			& \mbox{ if } (k,i) = (m,j); \\
		\frac{\lambda_{k,j}-\lambda_{k,i}+1}{\lambda_{k,j}-\lambda_{k,i}-1}
			& \mbox{ if } k = m, i \neq j; \\
		-\frac{\lambda_{m,j}^*+\lambda_{k,i}}{\lambda_{m,j}+\lambda_{k,i}^*}
			& \mbox{ if } |k-m| = 1; \\
		1 
			& \mbox{ otherwise}.
	\end{cases}
\end{align*}
Then $X_{k,i} X_{m,j} = \mu_{(k,i);(m,j)} X_{m,j} X_{k,i}$.

\end{document}
\paragraph
We set
\begin{align*}
H_{k,i}
	&= \prod_{j\neq i} (\lambda_{k,i} - \lambda_{k,j})(\lambda_{k,i} - \lambda_{k,j}^*)
	= \begin{cases}
		\displaystyle 
			\prod_{j\neq i} (\lambda_{k,i}^2 - \lambda_{k,j}^2 - \lambda_{k,i} + \lambda_{k,j} )
			& \mbox{ if } k \equiv 0 \mod 2 \\
		\displaystyle \prod_{j\neq i} (\lambda_{k,i}^2 - \lambda_{k,j}^2) 
			& \mbox{ if } k \equiv 1 \mod 2
	\end{cases} \\
R_{k,i}
	&= \prod_{j = 1}^{p(k+1)} (\lambda_{k,i} + \lambda_{k+1,j})(\lambda_{k,i} + \lambda_{k+1,j}^*)
		\prod_{j = 1}^{p(k-1)} (\lambda_{k,i} + \lambda_{k-1,j})(\lambda_{k,i} + \lambda_{k-1,j}^*) \\
h_{k,i}
	&= \lambda_{k,i}^2 (4 \lambda_{k,i}^2 - 1) \\
t_{k,i}
	&= - \frac{R_{k,i}}{h_{k,i} H_{k,i} H_{k,i}^{e_{k,i}}} \\
t_{k,i}
	&= -\frac{\displaystyle \prod_{j = 1}^{p(k+1)}
		(\lambda_{k,i}+\lambda_{k+1,j})(\lambda_{k,i} + \lambda_{k+1,j}^*)
		\displaystyle \prod_{j = 1}^{p(k-1)} (\lambda_{k,i} + \lambda_{k-1,j})
			(\lambda_{k,i} + \lambda_{k-1,j}^*)}
	{\lambda_{k,i}^2 (4 \lambda_{k,i}^2 - 1) \displaystyle \prod_{j\neq i} 
		(\lambda_{k,i} - \lambda_{k,j})(\lambda_{k,i} - \lambda_{k,j}^*)}
\end{align*}

\paragraph
We set
\begin{align*}
q_{(k,i),(m,j)}
	&= \begin{cases}
		1 
			& \mbox{ if } k = m, i = j \\
		\frac{\lambda_{k,i} - \lambda_{k,j} + 1}{\lambda_{k,i} - \lambda_{k,j} - 1}
			& \mbox{ if } k = m, i \neq j \\
		- \frac{\lambda_{k,i}^* + \lambda_{m,j}}{\lambda_{k,i} + \lambda_{m,j}^*}
			& \mbox{ if } |k - m| = 1 \\
		1 
			& \mbox{ if } |k - m|>1 
	\end{cases}
\end{align*}

\paragraph
We set
\begin{align*}
r_{(k,i),(m,j)}
	&= \begin{cases}
		1 
			& \mbox{ if } k = m, i = j \\
		\frac{\lambda_{k,i} - \lambda_{k,j}^* + 1}{\lambda_{k,i} - \lambda_{k,j}^* - 1}
			& \mbox{ if } k = m, i \neq j \\
		- \frac{\lambda_{k,i}^* + \lambda_{m,j}^*}{\lambda_{k,i} + \lambda_{m,j}}
			& \mbox{ if } |k - m| = 1 \\
		\frac{t_{k,i}^*}{t_{k,i}} 
			& \mbox{ if } |k - m|>1 \\
	\end{cases}
\end{align*}

\paragraph
We set
\begin{align*}
C_{k}
	&= \begin{cases}
		0 & \mbox{ if } k = 2q + 1 \\
		(-1)^q \ii \frac{\prod_{r=1}^q \lambda_{k-1, r} \prod_{r=1}^{q+1} 
		\lambda_{k+1,r}}
			{\prod_{r=1}^q \lambda_{k,r} \lambda_{k,r}^*} 
			& \mbox{ if } k = 2q
	\end{cases}
\end{align*}

\begin{Definition}
The algebra $\A$ is the $\CC$-algebra generated by $L$ and the variables 
$\{X_{k,i}, Y_{k,i} \mid (k,i) \in \Sigma\}$ subject to the following relations
\begin{align*}
X_{k,i} f 
	&= f(\lambda+e_{k,i}) X_{k,i}
&Y_{k,i} f 
	&= f(\lambda-e_{k,i}) Y_{k,i}
	&\mbox{ for all } f \in L, (k,i) \in \Sigma \\
X_{k,i} X_{m,j} 
	&= q_{(k,i),(m,j)} X_{m,j} X_{k,i} 
&Y_{m,j} Y_{k,i}
	&= q_{(k,i),(m,j)}^* Y_{k,i} Y_{m,j}
	& \mbox{ for all } (k,i), (m,j) \in \Sigma \\
X_{k,i} Y_{k,i} 
	&= t_{k,i}
& Y_{k,i} X_{k,i}
	&= t_{k,i}^*
	& \mbox{ for all } (k,i) \in \Sigma \\
X_{k,i} Y_{m,j}
	&= r_{(k,i),(m,j)} Y_{m,j} X_{k,i}
	&&	& \mbox{ for all } (k,i) \neq (m,j) \in \Sigma 
\end{align*}
\end{Definition}

\begin{Proposition}
The group $G$ acts on $\A$ by ring automorphisms. For each $k \in \interval n$ 
and each $\sigma \in S_{p(k)}$ we have 
\begin{align*}
\sigma \cdot (X_{m,i}, Y_{m,i})
	&= \begin{cases}
		(X_{k,\sigma(i)}, Y_{k, \sigma(i)})
			&\mbox{ if } m = k; \\
		(X_{m,i}, Y_{m,i})
			&\mbox{otherwise.}
	\end{cases} \\
\epsilon_k \cdot (X_{m,i}, Y_{m,i})
	&= \begin{cases}
		(-Y_{m,i}, -X_{k,i})
			&\mbox{ if } m = k; \\
		(X_{m,i}, Y_{m,i})
			&\mbox{otherwise.}
	\end{cases}
\end{align*}
Furthermore the action of $G$ restricted to $L \subset \A$ coincides with the
action described in \ref{polynomial-algebra}.
\end{Proposition}

\begin{Theorem}
Set
\begin{align*}
A_k 
	&= C_k + \sum_{i=1}^{p(k)} X_{k,i} - Y_{k,i} \in \A
\end{align*}
Then each $A_k \in \A^G$, and the Lie algebra generated by these elements is 
isomorphic to $\Psi_n$.
\end{Theorem}



\end{document}