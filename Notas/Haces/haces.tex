%%%%%%%%%%%%%%%%%%%%%% Generalities %%%%%%%%%%%%%%%%%%5
\documentclass[11pt,fleqn]{article}
\usepackage[paper=a4paper]
  {geometry}

\pagestyle{plain}
\pagenumbering{arabic}
%%%%%%%%%%%%%%%%%%%%%%%%%%%%%%%%
\usepackage{notas}

%%%%%%%%%%%%%%%%%%%%%%%%%%% The usual stuff%%%%%%%%%%%%%%%%%%%%%%%%%
\newcommand\NN{\mathbb N}
\newcommand\CC{\mathbb C}
\newcommand\QQ{\mathbb Q}
\newcommand\RR{\mathbb R}
\newcommand\ZZ{\mathbb Z}

\newcommand\maps{\longmapsto}
\newcommand\ot{\otimes}
\renewcommand\to{\longrightarrow}
\renewcommand\phi{\varphi}
\newcommand\id{\mathsf{Id}}
\newcommand\im{\mathsf{im}}
\newcommand\coker{\mathsf{coker}}
%%%%%%%%%%%%%%%%%%%%%%%%% Specific notation %%%%%%%%%%%%%%%%%%%%%%%%%
\newcommand\U{\mathcal U}
\newcommand\C{\mathcal C}
\newcommand\F{\mathcal F}
\newcommand\G{\mathcal G}
\newcommand\GG{\Gamma}
\renewcommand\O{\mathcal O}
\newcommand\E{\mathcal E}

\DeclareMathOperator\Sh{\mathfrak{Sh}}
\DeclareMathOperator\PreSh{\mathfrak{PreSh}}
\DeclareMathOperator\Et{\mathfrak{Et}}
%%%%%%%%%%%%%%%%%%%%%%%%%%%%%%%%%%%%%% TITLES %%%%%%%%%%%%%%%%%%%%%%%%%%%%%%
\title{Notas sobre teoría de haces}
\date{[haces.tex]}
\author{Pablo Zadunaisky}

\begin{document}
\maketitle
El objetivo de estas notas es dar las definiciones necesarias para comprender resultados sobre teoría de haces. Dos resultados a lso que apuntamos son el teorema de de Rham, que afirma que la cohomología de de Rham de una variedad diferenciable coincide con su cohomología simplicial a coeficientes en $\RR$, y la sucesión espectral de Leray.

Doy por sentados resultados de álgebra homológica, donde siempre uso \cite{Wei}
y \cite{Toh} como referencia, de topología donde cito a \cite{Mun}, y de 
teoría de variedades diferenciables, donde cito a \cite{War}.

Dos aclaraciones: la primera es que dado que estas notas tienen un objetivo muy concreto,
no trabajo en total generalidad; en particular los haces considerados toman valores en
categorías de grupos abelianos enriquecidas, mayormente módulos sobre un anillo. Otro
detalle es que todos los anillos considerados tienen unidad, pero por cuestiones técnicas
\emph{no} exijo que $1 \neq 0$. En particular considero al anillo $0$ como el objeto final
de la categoría de anillos.

\section{Definiciones generales}
\label{definiciones-generales}
A lo largo de toda esta sección $X$ es un espacio topológico Haussdorf. 
Nuestros principales ejemplos van a ser variedades diferenciables, el lector 
deberá asegurarse de entender las demostraciones al menos en estos casos.

\paragraph
\label{categoria-abiertos}
Asociamos a $X$ su categoría de abiertos $\U(X)$, cuyos objetos son los abiertos de $X$, y
ponemos una flecha $U \to V$ si $V \subset U$.

\begin{Definition}
\label{prehaz}
Un \newterm{prehaz} $\F$ de grupos abelianos (o de anillos, módulos, o de objetos de una
categoría cualquiera) sobre el espacio topológico $X$ es una asignación que:
\begin{itemize}
\item[-] a cada abierto $U \subset X$ le asigna un objeto $\F(U)$ en la categoría de
	llegada;
\item[-] a todo par de abiertos $U \subset V$ le asigna una función $\rho_{V,U}: \F(V) \to
	\F(U)$, llamada \newterm{función de restricción},
\end{itemize}
sujetas a las siguientes condiciones:
\begin{enumerate}
\item $\F(\emptyset)$ es objeto final de la categoría de llegada ($\emptyset$ en
	conjuntos, $0$ en la categoría de módulos, etc.);
\item $\rho_{U,U} = \id_{\F(U)}$;
\item dados tres abiertos $U \subset V \subset W$ el siguiente diagrama conmuta
\begin{align*}
\xymatrix{
	\F(W) \ar[dr]_-{\rho_{W,V}} \ar[rr]^-{\rho_{W,U}}& {} & \F(U) \\
	{} & \F(V) \ar[ur]_-{\rho_{V,U}}
}
\end{align*}
\end{enumerate}
En otras palabras, un prehaz a valores en una categoria $\C$ es un funtor de la categoría
$\U(X)$ en $\C$ que respeta objetos finales.
\end{Definition}

\begin{Example}
\label{ejemplos-de-prehaces}
Algunos ejemplos de  prehaces:
\begin{enumerate}
\item Un ejemplo idiota: a cada $U \in \U(X)$ le asignamos el $\RR$-espacio vectorial
	$\RR$, y las funciones de restricción son todas identidades.

\item Tenemos el \newterm{prehaz de funciones continuas} $\C$, que a cada entorno $U$ le
	asigna las funciones continuas $U \to \RR$. Las funciones de restricción  $\rho: \C(U)
	\to \C(V)$ están dadas por la restricción. Este es un prehaz de $\RR$-espacios
	vectoriales, y aún más de $\RR$-álgebras.

\item Si $M$ es una variedad diferenciable, podemos considerar $\Omega^p$, el prehaz de
	$p$-formas diferenciables, con funciones de restricción dadas por la restricción de
	formas diferenciables.
\end{enumerate}
\end{Example}

\paragraph
\label{notacion-secciones}
Introducimos un poco de notación. Dado un abierto $U$ de $X$ y un haz $\F$, los elementos de $\F(U)$ suelen llamarse \newterm{secciones} de $\F$ sobre $U$. Dados abiertos $V \subset U$ y una sección $s \in \F(U)$, a veces notamos por $s|_V$ al elemento $\rho_{U,V}(s) \in \F(V)$.

\paragraph
\label{haz}
La idea de \emph{prehaz} intenta capturar la idea de información global que se
restringe a espacios más pequeños: si tenemos información (funciones a valores reales,
$p$-formas, etc.) sobre un abierto $U$ entonces restringiendo podemos obtener información
acerca de los abiertos contenidos en $U$. Una propiedad importante que tienen ciertos
prehaces es que permiten reconstruir la información global, es decir de todo $X$, a partir
de información local, es decir a partir de la información que tenemos sobre un conjunto de
abiertos que cubra a $X$. Motivamos esta idea con un ejemplo sencillo.

Supongamos que tenemos un espacio $X = U \cup V$, con $U$ y $V$ abiertos. Una función
continua sobre $X$ determina por restricción funciones continuas sobre $U$ y $V$. Al mismo
tiempo, podemos recuperar la función a partir de sus valores sobre $U$ y $V$, y como $U$ y
$V$ son abiertos, toda propiedad local queda determinada si conocemos nuestra función
sobre estos abiertos. Supongamos ahora que tenemos dos funciones, $f_U,f_V$ definidas en
los dos abiertos respectivos. ¿En qué caso son la restricción de una función definida
globalmente sobre X? Claramente la condición necsearia y suficiente es que ambas deben
coincidir en $U \cap V$, o en otras palabras $f_U|_{U \cap V} = f_V|_{U \cap V}$.

Tomemos un abierto $U$ y un cubrimiento $\mathcal V$. Diremos que una colección de
elementos $\{s_V \in \F(V) \mid V \in \mathcal V\}$ es \newterm{compatible} si ``sus
restricciones coinciden'', es decir que dados $V, V' \in \mathcal V$ se tiene $s_V|_{V
\cap V'} = s_{V'}|_{V \cap V'}$. Decimos que un prehaz $\F$ tiene la propiedad de
levantado si dado cualquier abierto $U$, y cualquier cubrimiento $\mathcal V$ de $U$, para
toda familia compatible $\{s_V \in \F(V) \mid V \in \mathcal V\}$ existe una sección $s
\in \F(U)$ tal que $s|_V = s_V$.

\begin{Definition*}
Un \newterm{haz} es un prehaz tal que toda familia compatible de secciones tiene un único
levantado.
\end{Definition*}

\begin{Example}
\label{ejemplos-de-haces}
Algunos ejemplos y no ejemplos de haces:
\begin{enumerate}
\item Tanto el prehaz de funciones continuas como el de $p$-formas diferenciables son haces. 

\item El prehaz idiota de \ref{ejemplos-de-prehaces} es un haz solo si todo abierto de $X$
	es conexo. En cambio, si $U = V \cup V'$ con la unión disjunta, no existe ningún
	elemento en $\F(U) = \RR$ que se restrinja a la familia trivialmente coherente $1 \in
	\F(V), -1 \in \F(V')$.

\item Consideremos el siguiente prehaz sobre el conjunto $\{a,b\}$ con la topología
	disceta: $\F(\{a\}) = \RR, \F(\{b\}) = \RR, \F(\{a,b\}) = \RR \times \RR \times \RR$, y
	los morfismos de restricción a $\{a\}$, resp. $\{b\}$ están dados por la proyección a
	la primer, resp. tercer, coordenada. Entonces $\F$ tiene la propiedad de levantado,
	pero no es un haz porque el levantado no es único.

\item Sea $\pi: Y \to X$ una función sobreyectiva (cualquiera, continua, diferenciable,
	holomorfa...) con la siguiente propiedad: cada fibra $Y_x = \pi^{-1}(x)$ es un grupo
	abeliano. Para cada $U \subset X$ notamos por $\GG(\pi, U)$ al grupo abeliano de
	secciones de $\pi$ en $U$, es decir funciones (arbitrarias, continuas, diferenciables,
	holomorfas...) $s: U \to Y$ tales que $\pi \circ s = \id_U$, con la operación definida
	punto a punto. Entonces $\GG(\pi, -)$ es una haz de grupos abelianos sobre $X$.
\end{enumerate}
\end{Example}

\paragraph
\about{Fibras}
\label{fibras}
Dado $p \in X$, el conjunto de abiertos a los que pertenece $p$ forman un conjunto
ordenado filtrante $\U(p)$, con lo cual tiene sentido calcular la \newterm{fibra} de $\F$
en el punto $p$, definido como $\F_p = \varinjlim_{\U(p)} \F(U)$. Dado que un morfismo de
prehaces es compatible con las restricciones, la propiedad universal del límite directo
garantiza que todo morfismo de prehaces $\phi: \F \to \G$ induce un morfismo $\phi_p: \F_p
\to \G_p$ en las fibras. Si esto no queda claro, medite sobre el siguiente diagrama:
\begin{align*}
\xymatrix{
  \F(U) \ar[dd] \ar@/^6pt/[dr] \ar[rr]^-{\phi(U)}
    &{} 
    &\G(U) \ar[dd] \ar@/^6pt/[dr] 
    &{} \\
  {}
    & \varinjlim \F(U) \ar@{-->}[rr] 
    & {} 
    & \varinjlim \G(U)\\
  \F(V) \ar@/_6pt/[ur] \ar[rr]^-{\phi(V)}
    &{} 
    &\G(V) \ar@/_6pt/[ur] 
    &{} 
}
\end{align*}
Notar que esta construcción tiene sentido sobre grupos abelianos, o más en general para
haces con valores en una categoría $\C$ con límites directos.

\paragraph
\about{Gérmenes}
\label{germenes}
Hay una construcción concreta de $\F_p$. Podemos considerarla como el conjunto de los 
pares $\{(s, U) \mid s \in \F(U)\}$ dividido por la relación de equivalencia $(s,U) \sim 
(t,V)$ si y solo si existe $W \subset U \cap V$ tal que $s|_W = t|_W$. Es un ejercicio 
instructivo comprobar que este conjunto es efectivamente un objeto en la categoría 
pertinente y que tiene la propiedad universal deseada. Notamos por $s_p$ o por $[(s,U)]$ 
a la imagen de $(s,U)$ en $F_p$, también llamada \newterm{germen} de $s$. 

\begin{Lemma*}
Sea $\F$ un haz, $V \subset U$ abiertos y $s \in \F(U)$ una sección. Entonces $s|_V = 0$ si y solo si para todo $p \in V$ se tiene $s_p = 0$.
\end{Lemma*}
\begin{proof}
Es claro que $s|_V = 0$ implica que $s_p = 0$ para todo $p \in V$. Por otro lado, si $s_p
= 0$ entonces existe un entorno de $p$, digamos $V_p$, tal que $s|_{V_p} = 0$. Como los
$V_p$ forman un cubrimiento de $V$, la propiedad de levantado único implica que $s|_V =
0$.
\end{proof}

\section{Las categorías $\PreSh(X)$ y $\Sh(X)$}
\paragraph
\about{Morfismos de haces}
\label{morfismos-de-haces}
Dados $\F, \G$ prehaces sobre $X$, un morfismo de prehaces es una colección de morfismos
$\phi(U): \F(U) \to \G(U)$ tales que conmuta el diagrama
\begin{align*}
\xymatrix{
	\F(U) \ar[d]_-{\rho^F_{U,V}} \ar[r]^-{\phi(U)}& \G(U) \ar[d]^-{\rho^\G_{U,V}} \\
	\F(V) \ar[r]^-{\phi(V)} & \G(V)
}
\end{align*}
o en otras palabras una transformación natural entre los funtores $\F$ y $\G$. Un morfismo
de haces es un morfismo de prehaces entre dos haces.

Con estas definiciones tenemos categorías de prehaces y de haces sobre $X$ con valores en
$\C$, que se suelen notar $\PreSh(X, \C)$ y $\Sh(X, \C)$, respectivamente.
\begin{Example*}
De morfismos de haces:
\begin{itemize}
\item $i: \C^\infty \to \C$, con $i(U): \C^\infty(U) \to \C(U)$ dada por la inclusión;
\item $d: \C^\infty \to \Omega^1$, con $d(U): \C^\infty(U) \to \Omega^1(U)$ dada por la
	diferencial exterior;
\item Si $\pi: Y \to X$ y $\pi': Z \to X$ son dos funciones sobreyectivas como en
	\ref{ejemplos-de-haces} y $f: Y \to Z$ es una función de la clase correspondiente que
	hace conmutar el diagrama
\begin{align*}
\xymatrix{
  Y \ar[dr] \ar[rr]^-f &{} & Z \ar[dl] \\
  {} & X \ar@{..>}@/^12pt/[ul]^-s {}
}
\end{align*}
entonces $f_*(U): \GG(\pi, U) \to \GG(\pi',U)$ dada por $s \mapsto f \circ s$ es un
morfismo de haces.
\end{itemize}
\end{Example*}

\begin{Proposition}
\label{isomorfismo-de-haces}
Un morfismo de haces $\phi: \F \to \G$ es un isomorfismo si y solo si para todo punto $p
\in X$ el morfismo $\phi_p$ es un isomorfismo.
\end{Proposition}
\begin{proof}
Supongamos que $\phi$ tiene una inversa $\psi$. Entonces la funtorialidad de los límites
directos implica que $\psi_p = \phi_p^{-1}$.

Para demostrar la implicación inversa veremos que si $\phi_p$ es un isomorfismo entonces
cada $\phi(U)$ es un isomorfismo; en ese caso podemos definir $\phi^{-1}(U) =
\phi(U)^{-1}$ que resulta automáticamente ser un morfismo de haces. Para ver que $\phi(U)$
es inyectiva, observamos que si $s \in \ker \phi(U)$ entonces $s_p \in \ker \phi_p$ y por
lo tanto $s_p = 0$ para todo $p \in X$, y por el lema \ref{germenes} $s = 0$.

Resta ver que $\phi(U)$ es sobreyectiva. Sea entonces $t \in \G(U)$, la hipótesis dice 
que para todo $p \in X$ existe $s^p \in \F_p$ tal que $\phi_p(s^p) = t_p$, es decir que 
para todo $p \in X$ existen un abierto $V_p$ y una sección $s^p \in \F(V_p)$ tal que 
$\phi (V_p)(s^p) = t|_{V_p}$. Nos gustaría que $s^p = s_p$ para algún $s \in \F(U)$, y
\note{Atención con el abuso de notación: estamos llamando $s^p$ a una sección y también 
a su gérmen en el punto $p$}
para demostrar esto alcanza con ver que dados dos puntos $p, p'$, las secciones $s^p$ y 
$s^{p'}$ coinciden en $V_p \cap V_{p'}$. Para aligerar notación ponemos $s = s^p, s' = 
s^{p'}, V = V_p$ y $V' = V_{p'}$. Dado $x \in V \cap V'$, tanto $s_x$ y $s'_x$ van a 
parar a $t_x$ por $\phi_x$, con lo cual $s_x = s'_x$, o en otras palabras $s|_W = s'|_W$ 
para algún entorno $W$ de $x$, y al ser $x$ arbitrario esto vale para todo punto de 
$V \cap V'$. Luego $s|_{V \cap V'}$ y $s'|_{V \cap V'}$ coinciden en un entorno de todo 
punto, y la propiedad  de levantado único de los haces dice $s|_{V \cap V'} = s'|_{V \cap
V'}$. Como $p, p'$ eran puntos arbitrarios, tenemos que la familia $\{s^p|_{V_p}\}$ se 
levanta a una única sección $s \in \F(U)$. Además, sabemos que para todo punto $p$ se 
tiene $\phi_p(s_p) = \phi(s^p) = t_p$, con lo cual $\phi(s)$ coincide con $t$ en algún 
entorno de $p$. Una nueva aplicación de la propiedad del levantado único nos dice que 
$\phi(s) = t$.
\end{proof}

\paragraph
\about{Núcleos y conúcleos}
\label{nucleos-y-conucleos}
Dado un morfismo de prehaces $\phi: \F \to \G$, se definen el prehaz $\ker \phi$ dado 
por $(\ker \phi)(U)= \ker \phi(U)$, con las funciones de restricción inducidas por las 
de $\F$. Análogamente se definen los prehaces $\im \phi$ y $\coker \phi$ junto con sus 
funciones de restricción inducidas. 

\begin{Proposition*}
Si $\phi: \F \to \G$ es un morfismo de haces entonces $\ker \phi$ es un haz. Se tiene que $\ker \phi$ es el haz identicamente nulo si y solo si $\phi_p$ es inyectiva para todo $p \in X$.
\end{Proposition*}
\begin{proof}
Sea $U$ un abierto, $\U$ un cubrimiento de $U$, y supongamos que tenemos $\{s_V \in \ker 
\phi(V) : V \in \U\}$ una familia compatible de secciones. Dado que $\F$ es un haz, 
existe una única sección $s \in \F(U)$ tal que $s|_V = s_V$. Además se tiene que $\phi(s)
|_V = \phi(s_V) = 0$, con lo cual $\phi(s) = 0$ por la unicidad de los levantados en 
$\G$. Por otro lado, $s \in \ker \phi(U)$ si y solo si $\phi(s) = 0$, y esto es 
equivalente a que $\phi_p(s_p) = \phi(s)_p = 0$ para todo $p \in U$.
\end{proof}

\begin{Remark}
Sea $X = \CC$ con los haces $\F$ de funciones continuas a valores complejos con la 
operación suma, $\G$ de funciones holomorfas nunca nulas con el producto, y $\exp: \F \to
\G$ la exponencial. Dada una función nunca nula $g \in \G(U)$, tomemos un punto $p \in U$
y elijamos $U' \subset U$ de forma que $g(U') \subset V'$, donde $V'$ es un entorno 
simplemente conexo de $p$. Elegimos una rama del logaritmo complejo definida sobre $V'$ 
y tomamos $f = \log \circ g \in \F(U')$, de forma que $\exp(f) = g$, es decir que $\exp(
f_p) = \exp([(f,U')]) = [(g,U')] = g_p$ y la función $\exp$ es sobreyectiva en los 
stalks. Sin embargo si tomamos $U = \CC^\times$ y $g(z) = z$ no existe ninguna función 
$f$ tal que $\exp(f) = g$, justamente porque $U$ no es simplemente conexo. Así $\exp$ es
sobreyectiva en los stalks pero no es cierto que $\exp(U)$ sea sobreyectiva en cualquier
abierto.

Esto muestra que la proposición análoga a \ref{nucleos-y-conucleos} para los prehaces
$\im$ y $\coker$ es falsa, lo cual atenta contra la expectativa de que $\Sh(X)$ sea una
categoría abeliana. Para solucionar este problema tenemos que tomar un pequeño desvío.
\end{Remark}

\paragraph
\about{Haz asociado a un prehaz}
\label{haz-asociado-a-un-prehaz}
El ejemplo básico de un haz es el de haz de secciones de una proyección. Está claro que la
colecciones de secciones de una función tienen la propeidad de levantado único. Ahora
vamos a ver que a todo prehaz se le puede asociar un haz de funciones.

A cada prehaz $\F$ le vamos a asociar un espacio topológico $E(\F)$. Como conjunto 
$E(\F)$ está formado por la unión disjunta de las fibras $\bigsqcup_{x \in X} \F_x$. Hay 
una función obvia $\pi: E(\F) \to X$, dada por $\pi(s_x) = x$, y muchas funciones en el 
sentido inverso definidas localmente: para cada abierto $U \subset X$, y asociada a cada 
sección $s \in \F(U)$, temenos una función $E(s): x \in U \mapsto s_x \in \bigsqcup_{x \
in U} \F_x$. 
\note{Notar que $\pi$ es continua porque su composición con todas las secciones $s$ lo es.}
Le damos a $E(\F)$ la topología final respecto de todas las secciones, y llamamos $\F^+$ 
al haz de secciones continuas de la función $\pi: E(\F) \to X$; como $\F_x$ es un grupo
abeliano, este será un haz de grupos abelianos como el descrito en
\ref{ejemplos-de-haces}.

El espacio topológico $E(\F)$ es llamado el \newterm{espacio étale} de $\F$. A cada sección $s \in \F(U)$ le asociamos el conjunto $U_s = E(s)^{-1}(U) = \{s_x \mid x \in 
U\}$ de los gérmenes de $s$ en $E(\F)$; por definición es abierto, y la colección de 
todos los conjuntos $U_s$ con $U$ abierto de $X$ y $s \in \F(U)$ forman una base de la 
topología de $E(\F)$. 

Si $\phi: \F \to \G$ es un morfismo de haces entonces como vimos en \ref{germenes} induce
una función en los gérmenes, y por lo tanto entre los espacios étale $E(\phi): E(\F) \to
E(\G)$ dada por $E(\phi)(s_x) = \phi(s)_x$. Esta función es continua, para verlo alcanza
con probar que para toda $s \in \F(U)$ la función $E(\phi) \circ s$ es continua. Ahora, de
la definición se deduce que $E(\phi) \circ E(s) = E(\phi(s))$, que es continua también por
definición. Así $\phi$ induce una función $\phi^+: \F^+ \to \G^+$ entre los haces
asociados, que es verificablemente funtorial.

\paragraph
\about{La categoría de espacios étale}
\label{espacios-etale}
Podemos resumir la discusión del apartado \ref{haz-asociado-a-un-prehaz} de la siguiente
forma. Un \newterm{espacio étale} sobre $X$ es una función $p: Y \to X$ de espacios
topológicos que es un homeomorfismo local, y tal que toda fibra $U_x = p^{-1}(x)$ es un
grupo abeliano. En otras palabras, todo punto $y \in Y$ tiene un entorno $U$ tal que la
restricción $p: U \to p(U)$ es un homemorfismo. Si $p: Y \to X$ y $p': Z \to X$ son
espacios étale sobre $X$ entonces un morfismo entre ellos es una función $f: Y \to Z$ que
hace conmutar el diagrama
\begin{align*}
\xymatrix{
  Y \ar[rr]^-f \ar[dr]_-p & {} & Z \ar[dl]^-{p'} \\
  {} & X & {}
}
\end{align*}
y tal que $f_x: Y_x \to Z_x$ es un morfismo de grupos abelianos.
Tenemos entonces una categoría de espacios étale sobre $X$, que notamos $\Et(X)$.
\note{Estoy inventando en este párrafo... ¿alguien conoce una notación estándar?}

\begin{Lemma*}
Sea $\pi: Y \to X$ un espacio étale y sea $\F$ el haz de secciones continuas de $\pi$. Para cada $x \in X$ se tiene un isomorfismo de grupos abelianos $\F_x \cong p^{-1}(x)$.
\end{Lemma*}
\begin{proof}
Basta ver que $p^{-1}(x)$ tiene la propiedad universal de $\F_x$. En primer lugar, 
existen morfismos $\F(U) \to p^{-1}(x)$, dados por $s \in \F(U) \mapsto s(x) \in 
p^{-1(x)}$. Como la operación en $\F(U)$ está dada punto a punto, este es un morfismo de 
grupos abelianos, y por definición es compatible con la restricción de secciones.

Ahora supongamos que tenemos un grupo abeliano $A$ y una familia de morfismos $f(U): 
\F(U) \to A$ para cada entorno $U$ de $x$, compatible con las restricciones. Sea $y \in 
p^{-1}(x)$ y sea $V$ un entorno de $y$ tal que $p|_V: V \to p(V)$ es un homeomorfismo, 
cuya inversa notamos por $s$. Por definición $s \in \F(V)$ y por lo tanto tiene sentido 
tomar $f(V)(s)$. Además, cualquier otra sección $s': p(V') \to V'$ construida de la 
misma manera cumple que $s|_{p(V) \cap p(V')} = s|_{p(V) \cap p(V')}$, porque ambas son 
inversas de $p|_{V \cap V'}$, y por lo tanto $f(V)(s) = f(V \cap V')(s) = f(V \cap V')
(s') = f(V')(s')$. Entonces $s \in \F(V)$, y podemos tomar $f: p^{-1}(x) \to A$ como 
$f(y) = f(V)(s)$. Eso define una función $f: p^{-1}(x) \to A$, con lo cual $p^{-1}(x)$
tiene la propiedad universal del limite, en particular es isomorfa a $\F_x$.
\end{proof}

\paragraph
\about{Cuestiones de funtorialidad}
\label{cuestiones-de-funtorialidad}
En el aparatdo anterior definimos un funtor $E: \PreSh(X) \to \Et(X)$, y como vimos en 
\ref{ejemplos-de-haces} existe un funtor $\GG: \Et(X) \to \Sh(X)$ que a cada espacio 
étale le asigna el haz de sus secciones continuas. Por definición, $\F^+ = \GG \circ E(
\F)$. Agregamos dos funtores a nuestra lista: primero el funtor olvido $\O: \Sh(X) \to 
\PreSh(X)$ que envía un haz en sí mismo pensado como prehaz, y finalmente el funtor $\E: 
\Sh(X) \to \Et(X)$ dado por $\E \circ \O$, es decir la construcción del espacio étale 
restringida a la categoría de haces. 
\note{La palabra ``restringida'' no es muy feliz, pero bueh}

Tenemos un diagrama de funtores
\begin{align*}
\xymatrix{
  \PreSh(X) \ar@/^10pt/[rr]^-E \ar@/^10pt/[dr]^-{-^+} & {} & \Et(X) \ar@/^10pt/[dl]^-{\GG} \\
  {} & \Sh(X) \ar@/^10pt/[ul]^-\O \ar@/^10pt/[ur]^-\E& {}.
}
\end{align*}
\begin{Proposition*}
Los funtores $(-^+, \O)$ forman un par adjunto.
\end{Proposition*}
\begin{proof}
Para demostrar el enunciado, vamos a construir la unidad y counidad de la adjunción. 

Primero, para cada prehaz $\F$ construimos una transformación $\theta_\F: \F \to \F^+$.
Para cada abierto $U \subset X$ definimos $\theta(U): \F(U) \to \F^+(U)$ por la 
asignación $s \mapsto E(s)$; esto es un morfismo de prehaces. Si $\phi: \F \to \G$ es un morfismo de prehaces, entonces el diargama
\begin{align*}
\xymatrix{
  \F \ar[r]^-\phi \ar[d]_-{\theta(\F)} & \G \ar[d]^-{\theta(\G)}\\
  \F^+ \ar[r]^-{\phi^+} & \G^+
}
\end{align*}
conmuta. Por lo tanto $\theta$ define una transformación natural $\theta: \id \Rightarrow
\O \circ -^+$.

Ahora, si $\theta(\F)$ es un isomorfismo entonces $\F$ es isomorfo como prehaz a un haz, 
y por lo tanto un haz. Por otro lado $\theta_x: \F_x \to \F^+_x$ es la función inversa 
del isomorfismo construido en el Lema \ref{espacios-etale}, y por lo tanto si $\F$ era 
un haz entonces $\theta(\F): \F \to \O(\F)^+$ es un isomorfismo por 
\ref{isomorfismo-de-haces}. Poniendo $\rho(\F) = \theta(\F)^{-1}$ para cada haz $\F$ obtenemos una transformación natural $\rho: -^+ \circ \O \Rightarrow \id$. En ese caso es claro que para cada prehaz $\F$ y cada haz $\G$ las composiciones
\begin{align*}
\xymatrix{
  \F^+ \ar[r]^-{\theta(\F)^+} & \O(\F^+)^+ \ar[r]^-{\O(\rho(\F^+))} & \F^+ \\
  \mbox{y} \\
  \O(\G) \ar[r]^-{\O(\rho(\G))} & \O(\O(G)^+) \ar[r]^-{\theta(\O(G))} & \O(\G)
}
\end{align*}
son identidades, en el primer caso porque lo son al nivel de las fibras, y en el segundo 
por inspección directa.
\end{proof}

Resaltamos que de la construcción anterior se deduce el siguiente resultado.
\begin{Corollary*}
Para cada prehaz $\F$ hay un morfismo de prehaces $\theta(\F): \F \to \F^+$, que es un
isomorfismo si y solo si $\F$ es un haz. Más aún, si $\phi: \F \to \G$ es un morfismo de
prehaces y $\G$ es un haz, entonces $\phi$ se factoriza por $\theta(\F)$.
\end{Corollary*}
\begin{proof}
La primera afirmación fue probada en la demostración anterior. Por otro 
lado, dado un morfismo de prehaces $\phi$ como en el enunciado, sabemos que $\theta(\F) 
\circ \phi^+ = \phi \circ \theta(\G)$ (ver el primer diagrama de la demostración 
anterior), y como $\theta(\G)$ es un isomorfismo tenemos que $\phi = (\theta(\G)^{-1} 
\circ \phi^+) \circ \theta(\F)$.
\end{proof}

\paragraph
\about{La categoría $\Sh(X)$}
\label{categoria-sh}
Hemos construido para cada morfismo de haces $\phi: \F \to \G$ un núcleo $\ker \phi$ dado
por la asignación $U \mapsto \ker \phi(U)$, una imagen que es el haz asociado al prehaz
$U \mapsto \phi(U)$, y un conúcleo que es el haz asociado a $U \mapsto \coker \phi(U)$.
En los tres casos, tenemos que la fibra en cada punto $x \in X$ es el objeto
correspondiente al morfismo $\phi_x$. Con esto se puede probar que si $\phi$ es mono,
entonces es el núcleo de su conúcleo, y viceversa. Por otro lado, dados dos haces $\F, \G$
su suma directa está dada por $U \mapsto \F(U) \oplus \F(V)$, con lo cual $\Sh$ es una
categoría abeliana.

Sea $\{\F_i\}_{i \in I}$ un sistema codirigido de haces sobre $X$. Entonces la asignación
$U \mapsto \varinjlim_I \F_i(U)$ define un prehaz, que tiene la propiedad universal de los
colímites en la categoría $\PreSh(X)$; de esto se deduce que el haz asociado tiene la
propiedad universal del colímite en la categoría $\Sh(X)$. Como normalmente trabajamos
sobre la categoría de haces, notamos a este haz simplemente como $\varinjlim_I \F_i$. En
cambio el prehaz de los límites resulta ser un haz, con lo cual $(\varprojlim_I \F_i)$
existe y puede calcularse tomando límite sobre cada abierto.

Veamos ahora que la categoría de haces tiene suficientes objetos inyectivos. Dado $\F$,
consideremos para cada $x \in X$ la fibra $\F_x$, que es un grupo abeliano inyectivo y por
lo tanto existe un morfismo de grupos $i_x: \F_x \to \mathcal I_x$ con $\mathcal I_x$ un
grupo abeliano inyectivo. Consideramos $\prod_{x \in X} I_x \to X$ la función obvia, y
llamamos $\mathcal I$ al haz de secciones (sin ninguna hipótesis de continuidad) de esta
función. Tenemos una función $\F \to \mathcal I$ que envía cada $f \in \F(U)$ a la sección
$x \mapsto i_x(f_x)$, que es un morfismo inyectivo de haces porque cada $i_x$ es
inyectiva. 

\begin{bibdiv}
\begin{biblist}
\bib{Mun}{book}{
	author={Munkres},
	title={Topology},
}

\bib{Toh}{article}{
   author={Grothendieck, Alexander},
   title={Sur quelques points d'alg\`ebre homologique},
   language={French},
   journal={T\^ohoku Math. J. (2)},
   volume={9},
   date={1957},
   pages={119--221},
}

\bib{War}{book}{
	author={Frank W. Warner},
    title={Foundations of differentiable manifolds and Lie groups. Reprint.},
    year={1983},
    series={Graduate Texts in Mathematics},
    publisher={Springer-Verlag},
    pages={272},
}

\bib{Wei}{book}{
   author={Weibel, Charles A.},
   title={An introduction to homological algebra},
   series={Cambridge Studies in Advanced Mathematics},
   volume={38},
   publisher={Cambridge University Press},
   place={Cambridge},
   date={1994},
   pages={xiv+450},
}
\end{biblist}
\end{bibdiv}

\end{document}
