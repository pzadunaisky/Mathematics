%%%%%%%%%%%%%%%%%%%%%% Generalities %%%%%%%%%%%%%%%%%%5
\documentclass[11pt,fleqn]{article}
\usepackage[paper=a4paper]
  {geometry}

\pagestyle{plain}
\pagenumbering{arabic}
%%%%%%%%%%%%%%%%%%%%%%%%%%%%%%%%
\usepackage{notas}

%%%%%%%%%%%%%%%%%%%%%%%%%%% The usual stuff%%%%%%%%%%%%%%%%%%%%%%%%%
\newcommand\NN{\mathbb N}
\newcommand\CC{\mathbb C}
\newcommand\QQ{\mathbb Q}
\newcommand\RR{\mathbb R}
\newcommand\ZZ{\mathbb Z}
\renewcommand\P{\BbbP}

\newcommand\A{\mathcal A}
\newcommand\B{\mathcal B}
\newcommand\Z{\mathsf Z}

\newcommand\maps{\longmapsto}
\newcommand\ot{\otimes}
\renewcommand\to{\longrightarrow}
\renewcommand\phi{\varphi}
\newcommand\id{\mathsf{Id}}
\newcommand\im{\mathsf{im}}
\newcommand\coPer{\mathsf{coPer}}
%%%%%%%%%%%%%%%%%%%%%%%%% Specific notation %%%%%%%%%%%%%%%%%%%%%%%%%
\newcommand\g{\mathfraP g}
\renewcommand\sl{\mathfraP{sl}}

\DeclareMathOperator\rep{\mathsf{rep}}
\DeclareMathOperator\Rep{\mathsf{Rep}}

\DeclareMathOperator\Hom{\mathsf{Hom}}
\DeclareMathOperator\End{\mathsf{End}}


%%%%%%%%%%%%%%%%%%%%%%%%%%%%%%%%%%%%%% TITLES %%%%%%%%%%%%%%%%%%%%%%%%%%%%%%
\title{Cohomología de Koszul}
\date{[Koszul-cohomology.tex]}
\author{21/04/16}
\begin{document}
\maketitle
\paragraph 
Sea $A$ un álgebra cuadrática $A = T(V)/(R)$, con $R \subset V \ot V$. El álgebra es 
graduada y notamos su componente de grado $t$ por $A^t$

\paragraph
El álgebra $A^e$ es graduada y conexa, por lo tanto $A$ tiene una resolución minimal 
como $A^e$-módulo. Notamos esta resolución por $P_\bullet(A) = (A \ot P_\bullet \ot A, 
d^P)$. Como $A$ es graduada también lo es $P_p$ para cada $p \geq 0$ y $P_p^q = 0$ si 
$q < p$.

\paragraph
Cada módulo del complejo es trigraduado con $P_\bullet(A)^{(r,s,t)} = A^r \ot 
P_\bullet^s \ot A^t$. Llamamos \emph{peso} a $(r,s,t)$ y \emph{grado total} a $r+s+t$.
El diferencial $d^P$ respeta el grado total,
\[
d^P(1 \ot P^n_p \ot 1) \subset \bigoplus_{i+j+k = n} A^i \ot P^j_{p-1} \ot A^k
\]
con lo cual $j \leq n$. Por bilinealidad de $d^P$ obtenemos que $d^P(A \ot P_p^n \ot A)$
está contenido en $\bigoplus_{j \leq n} A \ot P_p^j \ot A$.

\paragraph
Sea $F_q P_\bullet(A) = \bigoplus_{n \leq q} A \ot P_\bullet^n \ot A$. Esta es
una filtración de $A \ot P_\bullet \ot A$ por espacios trigraduados y $d^P$ es filtrado
para esta graduación. El complejo graduado asociado es un complejo de $A$-bimódulos 
$\tilde P_\bullet(A)$ con $\tilde P(A)_p = A \ot P_p \ot A$ y diferencial $\tilde d^P$ 
que respeta la graduación por peso. Explícitamente, dado $x \in A^r \ot P_p^s \ot A^t$ 
tenemos $\tilde d^P(x) = d^P(x)^{(r,s,t)}$.

\paragraph
Dado $p \geq 0$ sea $W_p$ como en \emph{Koszul calculus}.
En este caso $W_p = P_p^{p}$ y $d^P(A \ot W_p \ot A) \subset A \ot W_{p-1} \ot A$.
Notamos $d^W$ a la restricción de $d^P$ a $A \ot W_\bullet \ot A$ y así tenemos un 
complejo $K_\bullet(A) = (A\ot W_\bullet \ot A, d^W)$. Este complejo también es 
trigraduado, y $K_q(A)^{(r,s,t)} = 0$ si $q \neq s$. Además
$K_s(A)^{(r,s,t)} = P_s(A)^{(r,s,t)}$ y la inclusón $K_\bullet(A) \to P_\bullet(A)$ es un
morfismo de complejos.


\end{document}
