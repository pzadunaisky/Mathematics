

%%%%%%%%%%%%%%%%%%%%%% Generalities %%%%%%%%%%%%%%%%%%5
\documentclass[11pt,fleqn]{article}
\usepackage[paper=a4paper]
  {geometry}

\pagestyle{plain}
\pagenumbering{arabic}
%%%%%%%%%%%%%%%%%%%%%%%%%%%%%%%%
% esto tiene que estar, en este orden...
\usepackage[small]{titlesec}
\usepackage{paragraphs}
\usepackage{hyperref}
\usepackage{amsthm,thmtools}

%\linespread{1.2}
\setlength{\parskip}{1.2ex}

\usepackage[utf8,latin1]{inputenc}
\usepackage[spanish,english]{babel}
\usepackage{enumerate}
\usepackage{mathpazo}
%\usepackage{euler}
\usepackage[alphabetic,initials]{amsrefs}
\usepackage{amsfonts,amssymb,amsmath}
\usepackage{dsfont}
\usepackage{mathtools}
\usepackage{graphicx}
\usepackage[poly,arrow,curve,matrix]{xy}
\usepackage{stmaryrd}
\usepackage{showlabels}
\usepackage{tikz}
%\swapnumbers

% numbered versions
\declaretheoremstyle[headformat=swapnumber, spaceabove=\paraskip,
bodyfont=\itshape]{mystyle}
\declaretheorem[name=Lemma, sibling=para, style=mystyle]{Lemma}
\declaretheorem[name=Proposition, sibling=para, style=mystyle]{Proposition}
\declaretheorem[name=Theorem, sibling=para, style=mystyle]{Theorem}
\declaretheorem[name=Corolllary, sibling=para, style=mystyle]{Corollary}
\declaretheorem[name=Definition, sibling=para, style=mystyle]{Definition}
%\declaretheorem[name=Example, sibling=para, style=mystyle]{Example}


% unnumbered versions
\declaretheoremstyle[numbered=no, spaceabove=\paraskip,
bodyfont=\itshape]{mystyle-empty}
\declaretheorem[name=Lemma, style=mystyle-empty]{Lemma*}
\declaretheorem[name=Proposition, style=mystyle-empty]{Proposition*}
\declaretheorem[name=Theorem, style=mystyle-empty]{Theorem*}
\declaretheorem[name=Corollary, style=mystyle-empty]{Corollary*}
\declaretheorem[name=Definition, style=mystyle-empty]{Definition*}
%\declaretheorem[name=Example, style=mystyle-empty]{Example*}

% plain style
\declaretheoremstyle[
        headformat={{\bfseries\NUMBER.}{\itshape\NAME}\NOTE\ignorespaces},
        spaceabove=\paraskip, 
        headpunct={.},
        headfont=\itshape,
        bodyfont=\normalfont
        ]{mystyle-plain}
\declaretheorem[sibling=para, style=mystyle-plain]{Example}
\declaretheorem[sibling=para, style=mystyle-plain]{Remark}

% proofs, just as in amsthm but with adapted spacing
\makeatletter
\renewenvironment{proof}[1][\proofname]{\par
  \pushQED{\qed}%
  \normalfont \topsep.75\paraskip\relax
  \trivlist
  \item[\hskip\labelsep
        \itshape
    #1\@addpunct{.}]\ignorespaces
}{%
  \popQED\endtrivlist\@endpefalse
}
\makeatother

%%%%%%%%%%%%%%%%%%%%%%%%%%% The usual stuff%%%%%%%%%%%%%%%%%%%%%%%%%
\newcommand\NN{\mathbb N}
\newcommand\CC{\mathbb C}
\newcommand\QQ{\mathbb Q}
\newcommand\RR{\mathbb R}
\newcommand\ZZ{\mathbb Z}
\newcommand\KK{\mathbb K}

\newcommand\maps{\longmapsto}
\newcommand\ot{\otimes}
\renewcommand\to{\longrightarrow}
\renewcommand\phi{\varphi}
\newcommand\op{\mathsf{op}}
\newcommand\blabla[1]{\noindent\framebox[1.05\width]{\texttt{Note: #1}}}
\renewcommand\ll{\llbracket}
\newcommand\rr{\rrbracket}
%%%%%%%%%%%%%%%%%%%%%%%%% Specific notation %%%%%%%%%%%%%%%%%%%%%%%%%
\renewcommand\k{\Bbbk}

\DeclareMathOperator\Mod{\mathsf{Mod}}
\DeclareMathOperator\Hom{\mathsf{Hom}}
\DeclareMathOperator\Ext{\mathsf{Ext}}
\DeclareMathOperator\Tor{\mathsf{Tor}}

\DeclareMathOperator\HOM{\underline{\mathsf{Hom}}}
\DeclareMathOperator\EXT{\underline{\mathsf{Ext}}}
\DeclareMathOperator\TOR{\underline{\mathsf{Tor}}}

\newcommand\join{\vee}
\newcommand\meet{\wedge}
\renewcommand\q{\mathbf{q}}

%%%%%%%%%%%%%%%%%%%%%%%%%%%%%%%%%%%%%% TITLES %%%%%%%%%%%%%%%%%%%%%%%%%%%%%%
\title{Hochschild cohomology for Koszul algebras}
\date{[Hochschild-Koszul-complexes.tex]}
\author{Pablo Zadunaisky}
\begin{document}
\maketitle

In this note we follow \cite{VdB}
and give a general construction for complexes that calculate Hochschild cohomology of
Koszul algebras. We hope to apply this construction to study quantum graded ASLs.

Throughout this document $\k$ is a field and all vector spaces and tensor products are 
over $\k$ unless otherwise noted.

\paragraph
Let $A$ be a finitely generated Koszul algebra with quadratic dual $A^!$. Set $V = A_1$,
and let $R \subset V \ot V$ the space of quadratic relations of $A$, so $A \cong T(V)/R$
and $A^! \cong T(V^*)/R^\perp$.

We consider $A \ot A^!$ as an algebra with the usual product. Notice that $A_1 \ot A_1^! 
= V \ot V^* \cong \hom_\k(V,V)$, and hence there is a distinguished element $e \in A_1 
\ot A^!_1$ corresponding to the identity of $V$. If $x_1, \ldots, x_n$ is a basis of $V$ 
and $\zeta^1, \ldots, \zeta^n$ is the corresponding dual basis of $V^*$ then $e = 
\sum_i e_i \ot \zeta^i$. Analogously we set $f = \sum_i \zeta^i \ot x_i \in A^! \ot A$. 
Then $e^2 = 0$ and $f^2 = 0$.

Let $M$ be a graded $A$-bimodule. Then $M \ot A^!$ is naturally an $A \ot A^!$-bimodule.
Denote the graded dual of $A^!$ by $(A^!)^\#$. Then $M \ot (A^!)^\#$ is also an
$A \ot A^!$-bimodule. For each $n \in \NN$ write
\begin{align*}
C^n(M) &= M \ot A^!_n \\
C_n(M) &= M \ot (A^!)^\#_{-n} =  M \ot (A^!_n)^*
\end{align*}
Define $d^n: C^n(M) \to C^{n+1}(M)$ and $\delta_n: C_n(M) \to C_{n-1}(M)$ by the formulae
\begin{align*}
	d(x) &= e \cdot (x) + (-1)^n(x) \cdot e \\
	\delta(y) &= e \cdot (y) + (-1)^n (y) \cdot e
\end{align*}
In other words, $d$ and $\delta$ are the super-commutators with the element $e$.

Our aim is to prove the following results

\begin{Theorem*}[A]
The homology of the complex $(C^\bullet(M), d)$ is naturally isomorphic to 
$HH^\bullet(A,M)$, while the homology of $(C_\bullet(M), \delta)$ is naturally 
isomorphic to $HH_\bullet(A,M)$
\end{Theorem*}

Taking $M = A$ we see that $C_\bullet(A)$ is an $C^\bullet(A)$-bimodule in a natural way.
\begin{Theorem*}[B]
The action of $C^\bullet(A)$ on $C_\bullet(A)$ induces the natural action of 
$HH^\bullet(A)$ on $HH_\bullet(A)$.
\end{Theorem*}


\section{Graded vector spaces and modules}
Let $G$ be a commutative group, such that there exists a non-zero homomorphism $\phi: G 
\to \ZZ$; equivalently $G \cong G' \times \ZZ$. We fix such an isomorphism and write 
$g = (g', n)$. We work with $G$-graded vector spaces, and assume that $\k$ is
concentrated in degree $0 = (0_{G'}, 0)$.

\paragraph
Let $M = \bigoplus_{g \in G} M_g$ be a $G$-graded vector space. Given an element $m \in 
M_g$ with $g \in G$, we call $g$ the \emph{total degree} of $m$, while the \emph{degree} 
of $m$ is $\phi(g)$; clearly this gives $M$ the structure of a $\ZZ$-graded vector space.

\paragraph
Given $g \in G$, we denote by $M[g]$ the $G$-graded vector space with the same 
underlying vector space as $M$ and with homogeneous component of degree $h \in G$ given
by $M[g]_h = M_{h+g}$. In case $n \in \ZZ$, we will write $M[n]$ instead of $M[(0_{G'}, 
n)]$.

\paragraph
Let $N$ be a second $G$-graded module. A linear transformation $f: N \to M$ is 
homogeneous of total degree $g \in G$ if $f(N_h) \subset M_{g+h}$ for all $h \in G$. By 
definition a linear transformation $f: N \to M$ is homogeneous of degree $g$ if and only 
if the linear transformation $f: N \to M[g]$ is homogeneous of total degree $0$. We 
write $\Hom_k^G(N,M)$ for the family of all morphisms of total degree $0$, and set
\begin{align*}
	\HOM_k^G(N,M) = \bigoplus_{n \in \ZZ} \Hom_k^G(N,M[n]),
\end{align*}
which is a $\ZZ$-graded vector space in an obvious way.

\paragraph
Let $A$ be a $G$-graded algebra. We say that $A$ is connected if it is connected as a 
$\ZZ$-graded algebra.

\paragraph
Let $P = (P_\bullet, d^P_\bullet)$ and $Q = (Q_\bullet, d^Q_\bullet)$ be complexes. Then 
we set $P \ot Q = ((P \ot Q)_\bullet), D_\bullet)$, where
\[
	(P \ot Q)_n = \bigoplus_{i+j = n} P_i \ot Q_j \\
	D_n = \bigoplus_{i + j = n} d^P_i + (-1)^i d^Q_j
\]

\section{The complexes}
\paragraph
\label{Algebras-setup}
Let $A$ be a finitely generated quadratic algebra with quadratic dual $A^!$. 
Set $V = A_1$, and let $R \subset V \ot V$ the space of quadratic relations of $A$, so 
$A \cong T(V)/R$ and $A^! \cong T(V^*)/R^\perp$. With our conventions, $A^!$ is a 
$\ZZ_{\leq 0}$-graded algebra, with $A^!_{-1} = V^*$.

We consider $A \ot A^!$ as an algebra with the usual product. Notice that $A_1 \ot A_1^! 
= V \ot V^* \cong \hom_\k(V,V)$, and hence there is a distinguished element $e \in A_1 
\ot A^!_1$ corresponding to the identity of $V$. If $x_1, \ldots, x_n$ is a basis of $V$ 
and $\zeta^1, \ldots, \zeta^n$ is the corresponding dual basis of $V^*$ then $e = 
\sum_i e_i \ot \zeta^i$. Analogously we set $f = \sum_i \zeta^i \ot x_i \in A^! \ot A$.

\paragraph
\label{e^2=0}
Fix $I \subset \ll n \rr^2$ such that for every $(k,l) \notin I$ there exist
$(\lambda^{i,j}_{k,l})_{(i,j) \in I} \subset \k$ with
\[
	x_k x_l - \sum_{(i,j) \in I} \lambda_{k,l}^{i,j} x_i x_j \in R.
\]
Notice that $R$ is generated by all such elements, so $\dim R = n^2 - |I|$ and $\dim
R^\perp = |I|$. Direct computation shows that for each $(i,j) \in I$ the element
\[
	\zeta^i \zeta^j + \sum_{(k,l) \notin I} \lambda^{i,j}_{k,l} \zeta^k \zeta^l
\]
is in $R^\perp$, and so $R^\perp$ is generated by these elements. Hence
\begin{align*}
	e^2 
		&= \sum_{i,j \in \ll n \rr^2} x_i x_j \ot \zeta^i \zeta^j = \sum_{(i,j) \in I}
			x_i x_j \ot \zeta^i \zeta^j + \sum_{(k,l) \notin I}  \sum_{(i,j) \in I}
			\lambda^{i,j}_{k,l} x_i x_j \ot \zeta^k \zeta^l \\
		&= \sum_{(i,j) \in I} x_i x_j \ot \left(\zeta^i \zeta^j + \sum_{(k,l) \notin I}
		\lambda^{i,j}_{k,l} \zeta^k \zeta^l\right) = 0.
\end{align*}
An analogous argument shows that $f^2 = 0$.

\paragraph
\label{graded-dual-module}
Let $C = (A^!)^\#$ be the graded dual of $A^!$, i.e. 
\[
	C_n = (A^!_{-n})^* = \hom_\k(A^!_{-n}, \k).
\] 
This is a graded $A^!$-bimodule, where for every $\xi, \zeta, \phi \in A^!$
and every $\hat \psi \in C_n$, 
\[
	\xi \cdot \hat \psi \cdot \zeta (\phi) = \hat\psi(\zeta \phi \xi).
\] 

\paragraph
\label{graded-dual-coalgebra}
The $A^!$-module $C$ is also an $\NN_0$-graded coalgebra, whose comultiplication is 
induced by the product of $A^!$. Let $n,i,j \in \NN_0$ with $n = i+j$. The 
multiplication map $\mu_{i,j}: A^!_{-i} \ot A^!_{-j} \to A^!_{-n}$ induces a dual map 
$\rho^{i,j}: C_n \to C_i \ot C_j$. We use Sweedler's notation and write 
$\rho^{i,j}(\hat\phi) = \hat \phi^i_1 \ot  \hat \phi^j_2$. 

\blabla{We are careful not to introduce signs!}


\paragraph
\label{K-complexes}
We now define the complexes $K^L_\bullet(A) = (A \ot C_\bullet, d^L)$ and $K_\bullet^R(A)
= (C_\bullet \ot A, d^R)$, where $d_L$ is multiplication by $e$ on the right and $d_R$ 
given by multiplication with $f$ on the left. These are complexes since $e^2 = f^2 = 0$, 

Set $K(A) = A \ot C \ot A$, which is naturally an $A \ot A^! \ot A$-module. We see this 
as a complex with $K_\bullet(A) = A \ot C_\bullet \ot A$, and with differential $D^K$ 
given by $x \in K_n(A) \mapsto x \cdot e \ot 1 + (-1)^n 1 \ot f \cdot x$. Notice that 
$K^0(A) = A \ot A$, $K^1 = A \ot V \ot A$ and $D_1(1 \ot x\ot 1)= x \ot 1 - 1 \ot x$, so 
we can augment this complex with the multiplication $\mu: A \ot A \to A$. We denote by 
$\tilde K^\bullet (A)$ the augmented complex.

\paragraph
\label{iso-complex}
Let $M$ be a graded $A$-bimodule. For every $n \in \NN_0$ set $T^n(M) = M \ot A^!_{-n}$, 
and let $d^n: T^n(M) \to T^{n+1}(M)$ be the map $d^n(x) = e \cdot x + (-1)^{n+1} x \cdot 
e$. Since $e^2 = 0$, we obtain a cochain complex $T(M) = (T^\bullet(M), d^\bullet)$.
If we apply the functor $\HOM_{A^e}(-, M)$ to the complex $K(A)$, we obtain
a cochain complex $(\HOM_{A^e}(K_\bullet(A), M), \delta^\bullet)$. 

\begin{Lemma*}
The complexes $(T^\bullet(M), d^\bullet)$ and $(\HOM_{A^e}(K_\bullet(A), M), 
\delta^\bullet)$ are isomorphic.
\end{Lemma*}
\begin{proof}
By the adjunction between graded tensor products and $\HOM$-spaces, there exists
an isomorphism
\begin{align*}
\HOM_{A^e}(A^e \ot C, M) &\cong \HOM_\k(C,M),
\end{align*}
and for each $n \geq 0$ there exist isomorphisms
\begin{align*}
\HOM_{A^e}(K_n(A), M) = \HOM_{A^e}(A^e \ot C_n, M) & \cong \HOM_\k(C_n, M).
\end{align*}
We see $\HOM_\k(C_n, M)$ as a subspace of $\HOM_\k(C, M)$, extending functions by $0$.

Now $\HOM_\k(C,M)$ is a $A \ot A^!$-bimodule in a natural way, and direct inspection 
shows that 
\begin{align*}
\bigoplus_{n \geq 0} \HOM_\k(C_n,M) \subset \HOM_\k(C,M)
\end{align*}
is an $A \ot A^!$-sub-bimodule. Furthermore, if $\overline \delta^n: \HOM_\k(C_n, M) \to 
\HOM_\k(C_{n+1}, M)$ is the map induced by $\delta^n$, then
\begin{align*}
	\overline\delta^n(x) = e \cdot x - (-1)^n x \cdot e.
\end{align*}

Since $M$ is an $A$-bimodule, $M \ot A^!$ is also an $A \ot A^!$-bimodule, and there is 
a natural $A \ot A^!$-linear map
\begin{align*}
	E: M \ot A^! \to \HOM_\k(C,M)
\end{align*}
given by assigning to each elementary tensor $m \ot \phi \in M \ot A^!_{-n}$ the 
$\k$-linear map $\hat \psi \in C_n \mapsto \hat\psi(\phi)m \in M$. Since $C_n = 
(A_{-n}^!)^*$ is finite dimensional, the image of $T^n(M) = M \ot A^!_{-n}$ through this 
map is precisely $\HOM_\k(C_n, M)$. Let $E^n: T^n(M) \to \HOM_\k(C_n, M)$ be the 
isomorphism induced by $E$. The $A \ot A^!$-linearity of $E$ implies that the following 
diagram commutes
\begin{align*}
\xymatrix{
	T^n(M) 
		\ar[d]_-{d^n} \ar[r]^-{E^n} 
	& \HOM_\k(C_n, M) 
		\ar[d]_-{\overline \delta^n} \ar[r]^-{\cong}
	& \HOM_{A^e}(K_n(A), M) 
		\ar[d]^-{\delta^n} \\
	T^{n+1}(M) 
		\ar[r]^-{E^{n+1}} 
	& \HOM_\k(C_{n+1}, M) 
		\ar[r]^-{\cong}
	& \HOM_{A^e}(K_{n+1}(A), M) 
}
\end{align*}
The horizontal arrows are the desired isomorphism.
\end{proof}

\begin{Theorem}
Assume $A$ is a quadratic Koszul algebra. Then the cohomology of the complex $T(M)$ is
naturally isomorphic to the Hochschild cohomology of $M$.
\end{Theorem}

\paragraph
\label{diagonal-morphism-definition}
We now look for a diagonal morphism $\Delta: K_\bullet(A) \to K_\bullet(A) \ot_A 
K_\bullet(A)$. We will obtain it using the coalgebra structure of $C = (A^!)^\#$, 
dual to the algebra structure of $A^!$. Set $S_\bullet(A) = K_\bullet(A) \ot_A 
K_\bullet(A)$, and denote its differential by $D^{S}$. For each 
$n \in \NN_0$
\begin{align*}
	S_n(A) 
		&= \bigoplus_{i+j = n} A \ot C_i \ot A \ot_A A \ot C_j \ot A \\
		&\cong \bigoplus_{i+j = n} A \ot C_i \ot A \ot C_j \ot A
\end{align*}
Put $S_{i,j}(A) = A \ot C_i \ot A \ot C_j \ot A$, and let $\Delta^{i,j}: K_n(A) \to 
S_{i,j}(A)$ be the map $\Delta^{i,j}(1|\hat\phi|1) = 1|\hat\phi_1^i|1|\hat\phi^j_2|1$, 
and $\Delta_n: K_n(A) \to S_n(A)$ be the map
\[
	\Delta_n(1|\hat\phi|1) 
		= \sum_{i+j=n} \Delta^{i,j}(1|\hat\phi|1) 
		= \sum_{i+j=n} 1|\hat\phi_1^i|1|\hat\phi^j_2|1
\]

\begin{Lemma*}
Fix $i,j \in \NN_0$ and set $n = i + j$. Then for every $x \ot \gamma \in A \ot A^!_1$ 
the following diagrams commute.
\begin{align*}
\xymatrix{
	 K_{n+1}(A)
		\ar[r]^-{\Delta^{i+1,j}} 
		\ar[d]_-{- \cdot x | \gamma |1}
	& S_{i+1,j}(A) 
		\ar[d]^-{- \cdot x |\gamma| 1 | 1 | 1} 
	&& K_{n+1}(A)
		\ar[r]^-{\Delta^{i,j+1}} 
		\ar[d]_-{1|\gamma|x \cdot -}
	& S_{i, j+1}(A)
		\ar[d]^-{ 1 | 1 |1| \gamma|x \cdot -}\\
	K_n(A)
		\ar[r]^-{\Delta^{i,j}} 
	& S_{i,j}(A)
	&& K_n(A)
		\ar[r]^-{\Delta^{i,j}} 
	& S_{i,j}(A)
}
\end{align*}

\begin{align*}
\xymatrix{
	K_{n+1}(A) 
		\ar[rr]^-{\Delta^{i+1,j}\oplus \Delta^{i,j+1}} \ar[d]_-0 
	&& S_{i+1,j}(A) \oplus S_{i,j+1}(A)
		\ar[d]^-{[1|\gamma|x|1|1 \cdot -] \oplus -[- \cdot 1|1|x|\gamma|1]} \\
	K_{n}(A)
		\ar[rr]^-{\Delta^{i,j}} 
	&& S_{i,j}(A)
}
\end{align*}
\end{Lemma*}
\begin{proof}
To prove that the first diagram commutes we begin with the trivial observation that the 
following diagram commutes
\begin{align*}
\xymatrix{
	A^!_{n+1} 
	& A_{i+1}^! \ot A_j^! 
		\ar[l]_-{\mu_{i+1,j}} \\
	A_n^!
		\ar[u]^-{\gamma \cdot -}
	& A^!_i \ot A_j^!
		\ar[u]_-{\gamma \ot 1 \cdot -}
		\ar[l]_-{\mu_{i,j}} 
}
\end{align*}
where $\mu$ is the product of $A^!$. Taking duals we see that the diagram
\begin{align*}
\xymatrix{
	 C_{n+1} 
		\ar[r]^-{\rho^{i+1,j}} 
		\ar[d]_-{- \cdot \gamma}
	& C_{i+1} \ot C_j 
		\ar[d]^-{- \cdot \gamma \ot 1}\\
	C_n
		\ar[r]^-{\rho^{i,j}} 
	& C_i \ot C_j
}
\end{align*}
also commutes; notice that by taking duals, left actions turn into right actions. With 
this the commutativity of the first diagram follows from the defintions. The 
commutativity of the second diagram is proved in a similar fashion.

Finally, since the diagram 
\begin{align*}
\xymatrix{
	A^!_{n+1} 
	& (A^!_{i+1} \ot A^!_j) \oplus (A^!_i \ot A^!_{j+1})
		\ar[l]_-{\mu_{i+1,j} \oplus \mu_{i,j+1}} \\
	A^!_{n} 
		\ar[u]^-0
	& A^!_i \ot A^!_j 
		\ar[u]_-{[- \cdot \gamma \ot 1] \oplus [- 1 \ot \gamma \cdot -]} 
		\ar[l]_-{\mu_{i,j}}
}
\end{align*}
commutes, the dual diagram
\begin{align*}
\xymatrix{
	C_{n+1} 
		\ar[r]^-{[\rho^{i+,j}] \oplus [\rho^{i,j+1}]}
		\ar[d]_-0
	& (C_{i+1} \ot C_j) \oplus (C_i \ot C_{j+1})
		\ar[d]^-{[\gamma \ot 1 \cdot -] \oplus [- \cdot (- 1 \ot \gamma)]}\\
	C_n 
		\ar[r]^-{\rho^{i,j}}
	& C_i \ot C_j
}
\end{align*}
also commutes, and the commutativity of the last diagram follows by definition.
\end{proof}

\begin{Proposition*}
The map $\Delta_\bullet: K_\bullet(A) \to S_\bullet(A)$ is a morphism of complexes.
\end{Proposition*}
\begin{proof}
Fix $i,j \in \NN_0$ and let $n = i+j$. Let $D_{i,j}: S_{i+1,j}(A) \oplus S_{i,j+1}(A) \to
S_{i,j}(A)$ be the restriction and correstriction of the differential of $S_\bullet(A)$. 
In order to prove that $\Delta_\bullet$ commutes with the differentials, it is enough to
see that the diagram
\begin{align*}
\xymatrix{
	K_{n+1}(A) 
		\ar[rr]^-{\Delta^{i+1,j}\oplus \Delta^{i,j+1}} \ar[d]_-{D^K_n}
	&& S_{i+1,j}(A) \oplus S_{i,j+1}(A)
		\ar[d]^-{D_{i,j}} \\
	K_{n}(A)
		\ar[rr]^-{\Delta^{i,j}} 
	&& S_{i,j}(A)
}
\end{align*}
commutes for all $i,j$.

Set
\begin{align*}
	e_1 &= \sum_i x_i| \zeta_i|1|1|1 
	& f_1 &= \sum_i 1| \zeta_i |x_i|1|1 \\
	e_2 &= \sum_i 1|1|x_i| \zeta_i|1 
	& f_2 &= \sum_i 1|1|1| \zeta_i |x_i.
\end{align*}
Then for every $x \in S_{i+1,j}(A)$ and $y \in S_{i,j+1}(A)$
\begin{align*}
D_{i,j}(x) &= x \cdot e_1 + (-1)^{i+1} f_1 \cdot x \\
D_{i,j}(y) &= (-1)^i y \cdot e_2 + (-1)^{i+1+j} f_2 \cdot y
\end{align*}

Let $a \in K_{n+1}(A)$. The commutativity of the third diagram implies that
\begin{align*}
f_1 \cdot \Delta^{i+1,j}(a) = \Delta^{i,j+1}(a) \cdot e_2,
\end{align*}
from which we get
\begin{align*}
D_{i,j}(\Delta^{i+1,j} \oplus \Delta^{i,j+1}(a)) 
	&= \Delta^{i+1,j}(a) \cdot e_1 
		+ (-1)^{n+1} f_2 \cdot \Delta^{i,j+1}(a).
\end{align*}
The commutativity of the first two diagrams of the lemma imply
\begin{align*}
\Delta^{i+1,j}&(a) \cdot e_1 
		+ (-1)^{n+1} f_2 \cdot \Delta^{i,j+1}(a) \\
	&= \Delta^{i,j}(a \cdot e \ot 1 + (-1)^{n+1} 1 \ot f \cdot a)\\
	&= \Delta^{i,j} \circ D_{n+1}^K(a).
\end{align*}
This shows that $\Delta_\bullet$ is a morphism of complexes.
\end{proof}

\paragraph
\label{Koszul-bimodule-complex-A}
By \cite{PP}*{Corollary 3.2} the algebra $A$ is Koszul if and only if $K^L_\bullet(A)$ 
is a minimal-free resolution of $\k$ as a left $A$-module, and a similar result holds for
$K^R_\bullet(A)$.

\begin{Proposition*}
	The complex $K^\bullet(A)$ is the minimal projective resolution of $A$ as $A^e$-module.
\end{Proposition*}
\begin{proof}
\blabla{Taken from \cite{VdB}*{Proposition 3.1}.}

Since $A$ is connected, so is $A^e$, and hence every graded $A$-module $M$ with bounded
left grading has a minimal projective resolution, consisting of free $A^e$-modules; a
projective resolution $P^\bullet \to M$ is minimal if and only if $\k \ot_{A^e} 
P^\bullet$ has zero differentials.

Consider the complex $\tilde K^\bullet(A) \ot_A \k$. This is a complex of left 
$A$-modules with $A \ot (A^!_n)^*$ in $n$-th homological degree, and since the action
of $A_{\geq 1}$ on $\k$ is trivial, its differential is given by right multiplication
by $e_L$. Hence $\tilde K^\bullet(A) \ot_A \k \cong K_L^\bullet(A)$ and since $A$ is 
Koszul this complex is exact. The graded version of Nakayama's Lemma implies that $\k$ 
is a faithfully flat graded $A$-module, so $\tilde K^\bullet(A)$ is exact, i.e. 
$K^\bullet(A)$ is a projective resolution of $A$ as $A^e$-module. Furthermore 
$\k \ot_{A^e} K(A) \cong \k \ot_A K_L(A)$ has zero differentials since $K_L(A)$ is a 
minimal resolution of $\k$ as $A$-module, so $K(A)$ is minimal.
\end{proof}

\newpage
\section{Twisted lattice algebras}

\paragraph\textbf{General definitions.}
\label{poset-defintions}
A \emph{lattice} is a poset $L$ such that any two elements $x,y \in L$ have an infimum 
$x \meet y$, also called their \emph{meet}, and a supremum $x \meet y$, also called 
their \emph{join}. Clearly any finite subset $T \subset L$ has a supremum, which we 
denote either by $\sup T$ or $\bigvee T$. An element $z \in L$ is called \emph{join 
irreducible} if it is not the supremum of any finite subset; in particular, the infimum 
of $L$ is the supremum of the empty set and hence is \emph{not} join-irreducible. 

A lattice is said to be \emph{distributive} if the $\join$ distributes over $\meet$, 
that is 
\[
	x \join (y \meet z) = (x \join y) \meet (x \join z)
\]
for all $x,y,z \in L$. 

Let $P$ be any poset. A \emph{poset ideal} of $P$ is a subset $I \subset P$ such that
if $x, y \in P$ with $x \in I$ and $y \leq x$ then $y \in I$. We denote by $J(P)$ the set
of poset ideals of $P$; this is a distributive lattice with union as join and 
intersection as meet. 

\paragraph
\label{Birkhoff}
Let $L$ be a finite distributive lattice, and let $P$ be the subset of join irreducible 
elements of $L$. Birkkhoff's representation theorem states that the map
\begin{align*}
T \in J(P) \mapsto \sup T \in L
\end{align*}
is a lattice isomorphism. Furthermore the cardinality of $P$ equals the rank of $L$, 
which is also the length of any maximal chain in $L$.

\paragraph
\label{lattice-semigroup}
Let $L$ be a distributive lattice of rank $n$. 
If we extend the order in $P$ to a total order and write $P = \{p_1, p_2, \ldots, p_n\}$,
we can define a map
\begin{align*}
\psi: J(P) & \to\NN_0^{n + 1} \\
T &\mapsto e_0 + \sum_{\{i \mid p_i\}} e_i,
\end{align*}
where $e_i$ is the $i$-the element of the canonical basis of $\NN_0^{n+1}$. This is also
a lattice isomorphism, with the added property that $\psi(T) + \psi(S) = \psi(T \cup S) 
+ \psi(T \cap S)$. Identifying $L$ with $J(P)$ through Birkhoff's isomorphism we obtain 
a funcion $\iota: L \to \NN_0^{n+1}$ such that $\iota(x) + \iota(y) = \iota(x \join y) 
\iota(x \meet y)$.

We associate to $S$ a commutative semigroup $S(L)$
\begin{align*}
	S(L) &= \langle x \in L \mid x \cdot y = (x \meet y) \cdot (x \join y)\rangle.
\end{align*}
Each element of $S(L)$ can be written in a unique way as the product of an ascending
sequence of elements of $L$ \cite{RZ15}*{}; any such product is called a \emph{standard
monomial}.

Clearly $\iota$ extends to a function $i: S(L) \to \NN_0^{n+1}$, which turns out to be
injective. Setting $S_{i,j} = \{(a_0, a_1, \ldots, a_n) \in \NN_0^{n+1} \mid a_i 
\geq a_j\}$, the image of $i$ can be written as
\[
	i(S(L)) = \bigcap_{i = 1}^n S_{0,i} \cap \bigcap_{p_i < p_j} S_{i,j}.
\]
The semigroup $i(S(L))$ is a full, normal and 
positive subsemigroup of $\ZZ^{n+1}$. By definition this is the semigroup generated by
The set $\{i(x) \mid x \in L \}$, and in fact this set is the Hilbert basis of the 
affine semigroup $i(S(L))$. For the rest of this section we fix an embedding $i: S(L)
\to \ZZ^{n+1}$ and identify the semigroup with its image.

\paragraph
The semigroup $S$ has two families of automorphisms. If $f: L \to L$ is a poset 
automorphism, then by the universal property of $S(L)$ it extends to an automorphism of 
$S(L)$. Furthermore since $S(L) \subset \ZZ^{n+1}$ is a full subsemigroup, $f$ extends 
to a group automorphism of $\ZZ^{n+1}$. A simple way to see this is that $f$ must fix 
the subposet $P$ of join-irreducible elements, and hence defines a permutation of $P$ so 
there exists $\sigma \in S_n$ such that $f(p_i) = p_{\sigma(i)}$. Thus the automorphism 
induced by $f$ on $\ZZ^{n+1}$, and by restriction on $S(L)$, is multiplication by 
$\begin{pmatrix} 1 & 0 \\ 0 & [\sigma] \end{pmatrix}$, where $[\sigma]$ is the matrix 
corresponding to the permutation. This takes into account the fact that we made an 
arbitrary choice when we extended the order of $P$.

On the other hand it is clear that $S(L) = S(L^{\op})$. Since join-irreducibles 
in $L$ are meet-irreducibles in $L^\op$, the embedding corres


\paragraph
\label{twisted-lattice-algebra}
Let $\alpha: S(L) \times S(L) \to \k^\times$ be a $2$-cocycle. The \emph{
$\alpha$-twisted lattice algebra} $\k_\alpha[L]$ is the quotient of the free algebra on 
$L$ by the relations
\begin{align*}
xy &= \alpha(x,y)(x\join y)(x \meet y) & x,y \in L
\end{align*}
By defintion $\k[L] = \k[S(L)]$, the classical semigroup algebra, and the 
$\alpha$-twisted lattice algebra is the usual $2$-cocycle twist of the semigroup algebra 
$\k[S(L)]$. Two twisted semigroup algebras are isomorphic if and only if their 
respective cocycles are cohomologous, so the cohomology group $H^2(S, \k^\times)$ 
parametrizes the isomorphism classes of twisted lattice algebras over $L$.

The map $i: S(L) \to \ZZ^{n+1}$ induces a group isomorphism $i^*: H^2(S(L), \k^\times) 
\to H^2(\ZZ^{n+1}, \k^\times) \cong \k^{\binom{n}{2}}$. Thus to each $2$-cocycle $\alpha$
over $S(L)$ corresponds a family of parameters $i^*(\alpha) = \q = \{q_{i,j} \mid 1 
\leq i < j \leq n \}$. It also induces an injective algebra morphism
\begin{align*}
	\k_\alpha[L] & \to \k_\q[X_0^{\pm 1}, \ldots, X_n^{\pm 1}].
\end{align*}
For this reason we will usually write $A(L; \q)$ instead of $\k_\alpha[L]$. We will also 
write $T(L; \q)$ for $\k_\q[X_0^{\pm 1}, \ldots, X_n^{\pm 1}]$, and refer to it as the 
\emph{quantum torus associated to $A(L; \q)$}.






\newpage
\begin{tikzpicture}
\node (f) at (3,4) {$f$};
\node (e) at (2,3) {$e$};
\node (d) at (1,2) {$d$};
\node (c) at (3,2) {$c$};
\node (b) at (2,1) {$b$};
\node (a) at (1,0) {$a$};
\draw (a) -- (b) -- (c) -- (e) -- (d) -- (b) (e) -- (f);

\node (6) at (7,4) {$\{f\}$};
\node (5) at (6,3) {$\{c,d\}$};
\node (4) at (5,2) {$\{d\}$};
\node (3) at (7,2) {$\{c\}$};
\node (2) at (6,1) {$\{b\}$};
\node (1) at (5,0) {$\emptyset$};
\draw (1) -- (2) -- (3) -- (5) -- (4) -- (2) (5) -- (6);

\node (61) at (11,4) {$(1,1,1,1,1)$};
\node (51) at (10,3) {$(1,1,1,1,0)$};
\node (41) at (9,2) {$(1,1,0,1,0)$};
\node (31) at (11,2) {$(1,1,1,0,0)$};
\node (21) at (10,1) {$(1,1,0,0,0)$};
\node (11) at (9,0) {$(1,0,0,0,0)$};
\draw (11) -- (21) -- (31) -- (51) -- (41) -- (21) (51) -- (61);
\end{tikzpicture}

\begin{tikzpicture}
\node (1) at (1,0) {$a$};
\node (2) at (0,1) {$b$};
\node (3) at (1,1) {$c$};
\node (4) at (2,1) {$d$};
\node (5) at (0,2) {$e$};
\node (6) at (1,2) {$f$};
\node (7) at (2,2) {$g$};
\node (8) at (1,3) {$h$};
\draw (6) -- (4) -- (1) -- (2) -- (6) (3) -- (7) -- (8) -- (5) -- (3) (1) -- (3)
	(2) -- (5) (6) -- (8) (4) -- (7);

\node (11) at (5,0) {$\emptyset$};
\node (21) at (4,1) {$\{b\}$};
\node (31) at (5,1) {$\{c\}$};
\node (41) at (6,1) {$\{d\}$};
\node (51) at (4,2) {$\{b,c\}$};
\node (61) at (5,2) {$\{b,d\}$};
\node (71) at (6,2) {$\{c,d\}$};
\node (81) at (5,3) {$\{b,c,d\}$};
\draw (61) -- (41) -- (11) -- (21) -- (61) (31) -- (71) -- (81) -- (51) -- (31) 
	(11) -- (31) (21) -- (51) (61) -- (81) (41) -- (71);


\node (12) at (11,-1) {$(1,0,0,0)$};
\node (22) at (9,1) {$(1,1,0,0)$};
\node (32) at (11,1) {$(1,0,1,0)$};
\node (42) at (13,1) {$(1,0,0,1)$};
\node (52) at (9,3) {$(1,1,1,0)$};
\node (62) at (11,3) {$(1,1,0,1)$};
\node (72) at (13,3) {$(1,0,1,1)$};
\node (82) at (11,5) {$(1,1,1,1)$};
\draw (62) -- (42) -- (12) -- (22) -- (62) (32) -- (72) -- (82) -- (52) -- (32) 
	(12) -- (32) (22) -- (52) (62) -- (82) (42) -- (72);
\end{tikzpicture}

\newpage
\begin{bibdiv}
\begin{biblist}
\bib{PP}{book}{
	author={Polishchuk, Alexander},
	author={Positselski, Leonid},
	title={Quadratic algebras},
	series={University Lecture Series},
	volume={37},
	publisher={American Mathematical Society, Providence, RI},
	date={2005},
	pages={xii+159},
}

\bib{VdB}{article}{
   author={Van den Bergh, Michel},
   title={Noncommutative homology of some three-dimensional quantum spaces},
   booktitle={Proceedings of Conference on Algebraic Geometry and Ring
	   Theory in honor of Michael Artin, Part III (Antwerp, 1992)},
   journal={$K$-Theory},
   volume={8},
   date={1994},
   number={3},
   pages={213--230},
}
\end{biblist}
\end{bibdiv}
\end{document}
