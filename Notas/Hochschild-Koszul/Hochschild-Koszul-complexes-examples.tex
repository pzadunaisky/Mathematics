

%%%%%%%%%%%%%%%%%%%%%% Generalities %%%%%%%%%%%%%%%%%%5
\documentclass[11pt,fleqn]{article}
\usepackage[paper=a4paper]
  {geometry}

\pagestyle{plain}
\pagenumbering{arabic}
%%%%%%%%%%%%%%%%%%%%%%%%%%%%%%%%
% esto tiene que estar, en este orden...
\usepackage[small]{titlesec}
\usepackage{paragraphs}
\usepackage{hyperref}
\usepackage{amsthm,thmtools}

%\linespread{1.2}
\setlength{\parskip}{1.2ex}

\usepackage[utf8,latin1]{inputenc}
\usepackage[spanish,english]{babel}
\usepackage{enumerate}
\usepackage[noBBpl]{mathpazo}
%\usepackage{euler}
\usepackage{array}
\usepackage[alphabetic,initials]{amsrefs}
\usepackage{amsfonts,amssymb,amsmath}
\usepackage{dsfont}
\usepackage{mathtools}
\usepackage{graphicx}
\usepackage[poly,arrow,curve,matrix]{xy}
\usepackage{stmaryrd}
\usepackage{showlabels}
\usepackage{tikz}

%\swapnumbers

% numbered versions
\declaretheoremstyle[headformat=swapnumber, spaceabove=\paraskip,
bodyfont=\itshape]{mystyle}
\declaretheorem[name=Lemma, sibling=para, style=mystyle]{Lemma}
\declaretheorem[name=Proposition, sibling=para, style=mystyle]{Proposition}
\declaretheorem[name=Theorem, sibling=para, style=mystyle]{Theorem}
\declaretheorem[name=Corolllary, sibling=para, style=mystyle]{Corollary}
\declaretheorem[name=Definition, sibling=para, style=mystyle]{Definition}
%\declaretheorem[name=Remark, sibling=para, style=mystyle]{Remark}


% unnumbered versions
\declaretheoremstyle[numbered=no, spaceabove=\paraskip,
bodyfont=\itshape]{mystyle-empty}
\declaretheorem[name=Lemma, style=mystyle-empty]{Lemma*}
\declaretheorem[name=Proposition, style=mystyle-empty]{Proposition*}
\declaretheorem[name=Theorem, style=mystyle-empty]{Theorem*}
\declaretheorem[name=Corollary, style=mystyle-empty]{Corollary*}
\declaretheorem[name=Definition, style=mystyle-empty]{Definition*}
\declaretheorem[name=Remark, style=mystyle-empty]{Remark*}

% plain style
\declaretheoremstyle[
        headformat={{\bfseries\NUMBER.}{\itshape\NAME}\NOTE\ignorespaces},
        spaceabove=\paraskip, 
        headpunct={.},
        headfont=\itshape,
        bodyfont=\normalfont
        ]{mystyle-plain}
\declaretheorem[sibling=para, style=mystyle-plain]{Example}
\declaretheorem[sibling=para, style=mystyle-plain]{Remark}

% proofs, just as in amsthm but with adapted spacing
\makeatletter
\renewenvironment{proof}[1][\proofname]{\par
  \pushQED{\qed}%
  \normalfont \topsep.75\paraskip\relax
  \trivlist
  \item[\hskip\labelsep
        \itshape
    #1\@addpunct{.}]\ignorespaces
}{%
  \popQED\endtrivlist\@endpefalse
}
\makeatother

%%%%%%%%%%%%%%%%%%%%%%%%%%% The usual stuff%%%%%%%%%%%%%%%%%%%%%%%%%
\newcommand\NN{\mathbb N}
\newcommand\CC{\mathbb C}
\newcommand\QQ{\mathbb Q}
\newcommand\RR{\mathbb R}
\newcommand\ZZ{\mathbb Z}
\newcommand\KK{\mathbb K}
\newcommand\V{\mathcal V}

\newcommand\maps{\longmapsto}
\newcommand\ot{\otimes}
\renewcommand\to{\longrightarrow}
\renewcommand\phi{\varphi}
\renewcommand\q{\mathbf{q}}
\newcommand\blabla[1]{\noindent\framebox[1.05\width]{\texttt{Note: #1}}}
\renewcommand\ll{\llbracket}
\newcommand\rr{\rrbracket}
%%%%%%%%%%%%%%%%%%%%%%%%% Specific notation %%%%%%%%%%%%%%%%%%%%%%%%%
\renewcommand\k{\Bbbk}
\newcommand\orho{\overline\rho}

\DeclareMathOperator\Mod{\mathsf{Mod}}
\DeclareMathOperator\Hom{\mathsf{Hom}}
\DeclareMathOperator\Ext{\mathsf{Ext}}
\DeclareMathOperator\Tor{\mathsf{Tor}}
\DeclareMathOperator\hh{\mathsf{ht}}
\DeclareMathOperator\wt{\mathsf{wt}}
\DeclareMathOperator\supp{\mathsf{supp}}

\DeclareMathOperator\HOM{\underline{\mathsf{Hom}}}
\DeclareMathOperator\EXT{\underline{\mathsf{Ext}}}
\DeclareMathOperator\TOR{\underline{\mathsf{Tor}}}


%%%%%%%%%%%%%%%%%%%%%%%%%%%%%%%%%%%%%% TITLES %%%%%%%%%%%%%%%%%%%%%%%%%%%%%%
\title{Hochschild cohomology for Koszul algebras: examples}
\date{[Hochschild-Koszul-complexes-examples.tex]}
\author{Pablo Zadunaisky}
\begin{document}
\maketitle
Fix a field $\k$ and $q \in \k^\times$.
\section{The diamond algebra and its Koszul dual}
\paragraph
\label{diamond-poset}
Let $L$ be the poset with Hasse diagram

\begin{tikzpicture}
\node (d) at (2,2) {$d$};
\node (c) at (3,1) {$c$};
\node (b) at (1,1) {$b$};
\node (a) at (2,0) {$a$};
\draw (a) -- (b) -- (d) -- (c) -- (a);
\end{tikzpicture}

A sequence $(x_1, x_2, \ldots, x_n)$ with $x_i \in L$ is called a \emph{chain} 
if $x_i \leq x_{i+1}$ for all $1 \leq i < n$, and an \emph{antichain} if $x_i \nleq 
x_{i+1}$ for all $1 \leq i < n$

\paragraph
\label{V}
Let $V$ be the vector space generated by $L$, and let $V^*$ be the dual vector space, 
with basis $L^* = \{\alpha, \beta, \gamma, \delta \}$.

\paragraph
\label{diamond-algebra}
Let $A = A(q)$ be the algebra generated by $L$ with relations
\begin{align*}
da &= q^{-2}ad 
  & bc &= cb = q^{-1}ad 
  & yx = q^{-1}xy \mbox{ for all } x<y, (x,y) \neq (a,d).
\end{align*}
This algebra has a PBW basis $\{a^ib^jd^k, a^ic^jd^k \mid i,j,k \in \NN_0\}$ whose 
elements are in bijective correspondence with chains over $L$. It is a Koszul algebra 
with Hilbert series 
\[
	h_A(t) = \frac{1-t^2}{(1-t)^4} = \frac{1+t}{(1-t)^3} = \sum_{n \geq 0} (n+1)^2 t^n.
\]


\paragraph
\label{grading-and-bases}
Set $\wt a = (1,0,0), \wt b = (1,1,0), \wt c = (1,0,1), \wt d = (1,1,1)$, and
extend this function to the PBW-basis in an obvious way. This defines a fine $\NN_0^3$-
grading on $A$, and we refer to the $\NN_0^3$-degree of an element of the PBW-basis as 
its \emph{weight}. The usual $\NN_0$-degree of a homogeneous element $x \in A$ is 
recovered by taking the first coordinate of $\wt x$.
The Hilbert multiseries corresponding to this grading is
\[
	h_A(x,y,z) = \sum_{i \geq j,k \geq 0} x^i y^j z^j =
		\frac{1-x^2yz}{(1-x)(1-xy)(1-xz)(1-xyz)}
\]

\paragraph
\label{toric-embedding}
Let $T = \k_\q[x^{\pm 1}, y^{\pm 1}, z^{\pm 1}]$, with $yz = q^{-1}xy, zx = q^{-1}xzw,
zy=yz$. Then there is an injective map of $\NN^3$-graded algebras $A \to T$ defined by
\begin{align*}
	a &\mapsto x; & b&\mapsto xy; & c&\mapsto xz; & d&\mapsto xyz.
\end{align*}

\paragraph
\label{support-A}
The support of $A$ is the positive normal affine semigroup 
\[
	S = \{(w_1, w_2, w_3) \in \NN_0^3 \mid w_1 \geq \max \{w_2, w_3\} \}. 
\]
For each $w \in S$ there exist unique $i,j,k \in \NN_0$ such that $w = (i+j+k, j+k, k)$ 
if $w_2 \geq w_3$, or $w = (i+j+k, k, j+k)$ if $w_3 \geq w_2$, namely $k = \min\{w_2, 
w_3\}, j = w_2 + w_3 -k, i = w_1 - \max\{w_2, w_3\}$. Thus
\begin{align*}
A_{w} &= 
  \begin{cases}
    \langle a^i b^j d^k \rangle&\mbox{ if } w_3 \leq w_2 \leq w_1 \\
    \langle a^i c^j d^k \rangle&\mbox{ if } w_2 \leq w_3 \leq w_1 \\
    0 & \mbox{ otherwise}.
  \end{cases}
\end{align*}

\paragraph
\label{diamond-algebra-dual}
Let $L^* = \{\alpha, \beta, \gamma, \delta\}$ be the basis of $V^*$ dual to $L$.
The Koszul dual $A^! = A(q)^!$ is the algebra generated by $L^*$ with relations
\begin{align*}
x^2 &= 0 &\mbox{ for } &x \in L^*; \\
yx &= -q^{-1}xy  &\mbox{ for } &(y,x) \in \{(\gamma, \delta); 
  (\beta, \delta); (\alpha, \beta); (\alpha, \gamma)\};\\
\alpha \delta &= -q^{-2} \delta \alpha - q^{-1} \beta \gamma - q^{-1} \gamma \beta
\end{align*}
Since $A^!$ is the quadratic dual of $A$ it is Koszul, and its Hilbert series is
\[
  h_{A^!}(t) 
    = 1/h_A(-t) 
    = \frac{(1+t)^3}{1-t} = 1 + 4t + 7t^2 + 8 t^3 + 8t^4 + 8t^5 + \cdots.
\]

\paragraph
\label{dual-weight}
The algebra $A^!$ also has an $-\NN^3$-grading by $\wt$, with $\wt \alpha = - \wt a =
-(1,0,0)$, etc. The usual grading is recovered taking \emph{minus} the first coordinate 
of the weight. 

It follows that the Hilbert multiseries of $A^!$ can be written as
\begin{align*}
	h_{A^!} &= 1+ u(1+v)(1+w) + u^2 (v(1+w)^2 + w(1+v^2) -vw) \\
			  	&+2u^3\frac{(1+v)(1+w)}{1-u^2} + u^4\frac{v(1+w)^2 + w(1+v)^2}{1-u^2} 
\end{align*}

Let 
\begin{align*}
	\V_1 &= \{(1,0,0); (1,1,0); (1,0,1);(1,1,1)\} \\
	\V_2 &= \{(2,1,0); (2,0,1); (2,1,1); (2,2,1); (2,1,2)\}
\end{align*}
Then for $k \geq 0$
\begin{align*}
\supp A^!_0 &= \{(0,0,0)\} \\
\supp A^!_{2k+1} &= -k(2,1,1) - \mathcal V_1 \\
\supp A^!_{2k+2} &= -k(2,1,1) - \mathcal V_2.
\end{align*}

\paragraph
\label{dual-PBW-basis}
The algebra $A^!$ has a PBW-basis, indexed by \emph{anti-chains} in $L$. The following 
is a list of bases $B^!_n$ for each homogeneous component $A^!_n$ with $n \geq 0$, along 
with their weights

\begin{align*}
B^!_0 
  &= \{1\} \mbox{ in weight } (0,0,0) \\
B^!_1 
  &= \begin{cases} 
    \delta  & \mbox{ in weight} -(1,1,1) \\
    \gamma  & \mbox{ in weight} -(1,0,1) \\
    \beta  & \mbox{ in weight} -(1,1,0) \\
    \alpha  & \mbox{ in weight} -(1,0,0)
    \end{cases}  \\
B^!_{2n+2}  
  &= \begin{cases}
  \delta(\gamma \beta)^n \gamma & \mbox{ in weight} -(2n+2,n+1,n+2) \\
  \delta(\beta\gamma)^n\beta & \mbox{ in weight} -(2n+2,n+2,n+1) \\
  \delta(\gamma \beta)^n\alpha, \delta(\beta\gamma)^n\alpha, 
  \beta(\gamma \beta)^n\gamma, \gamma(\beta\gamma)^n\gamma,
    & \mbox{ in weight} -(2n+2,n+1,n+1) \\
  \gamma(\beta\gamma)^n \alpha & \mbox{ in weight} -(2n+2,n,n+1) \\
  \beta(\gamma \beta)^n\alpha, & \mbox{ in weight} -(2n+2,n+1,n)     
    \end{cases}\\
B^!_{2n+3} 
  &= \begin{cases} 
  \delta(\gamma\beta)^n\gamma\beta, \delta(\beta\gamma)^n\beta\gamma
    & \mbox{ in weight} -(2n+3,n+2,n+2) \\
  \delta(\gamma\beta)^n\gamma\alpha, \gamma(\beta\gamma)^n\beta\alpha
    & \mbox{ in weight} -(2n+3,n+1,n+2) \\
  \beta(\gamma\beta)^n\gamma\beta, \delta(\beta\gamma)^n\beta\alpha
    & \mbox{ in weight} -(2n+3,n+2,n+1) \\
  \beta(\gamma\beta)^n\gamma\alpha, \gamma(\beta\gamma)^n\beta\alpha
    & \mbox{ in weight} -(2n+3,n+1,n+1) \\
  \end{cases}\\
\end{align*}

\paragraph
\label{A-shriek-basis}
Although $A^!$ has a PBW-basis, we will construct a diferent one that is better suited 
for the calculation of the Hochschild cohomology of $A$.

Using the PBW-basis from \ref{diamond-algebra-dual} it is easy to check that the element
$\gamma + \beta \in A^!_1$ is a regular normal element. Thus multiplication by $\gamma +
\beta$ is an injective map $A^!_n \to A^!_{n+1}$, and hence an isomorphism for 
$n \geq 3$. We set $\rho = (\gamma + \beta)^2 = \gamma\beta + \beta\gamma$. Then 
$\rho^n =
(\gamma\beta)^n + (\beta\gamma)^n$ for each $n \geq 0$, and the following commutation
relations hold
\begin{align*}
	\rho^n \delta&= q^{-2n} \delta\rho^n,
		& \gamma \rho^n &= \rho^n \gamma, 
		&\beta \rho^n &= \rho^n \beta,
		& \alpha \rho^n &= q^{-2n}\rho^n \alpha.
\end{align*}

We also set $\orho^n = (\gamma\beta)^n - (\beta\gamma)^n$; notice that this is \emph{not}
the $n$-th power of $\orho^1 = \gamma\beta-\beta\gamma$, but rather $\orho^{n+1} = 
\orho^n
\rho$. The following commuation relations hold.
\begin{align*}
	\orho^n \delta&= q^{-2n} \delta\orho^n,
		& \gamma \orho^n &= -\orho^n \gamma, 
		&\beta \orho^n &= -\orho^n \beta,
		& \alpha \orho^n &= q^{-2n}\orho^n \alpha.
\end{align*}

With this notation we obtain bases for $A^!_n$:
\begin{align*}
C^!_0 
  &= \{1\} \mbox{ in weight } (0,0,0) \\
C^!_1 
  &= \begin{cases} 
    \delta  & \mbox{ in weight} -(1,1,1) \\
    \gamma  & \mbox{ in weight} -(1,0,1) \\
    \beta  & \mbox{ in weight} -(1,1,0) \\
    \alpha  & \mbox{ in weight} -(1,0,0)
    \end{cases}  \\
C^!_{2k+2}  
  &= \begin{cases}
  \delta \rho^k \gamma & \mbox{ in weight} -(2k+2,k+1,k+2) \\
  \delta \rho^k\beta & \mbox{ in weight} -(2k+2,k+2,k+1) \\
	q^{-2k}\delta\rho^k\alpha, q^{-2k}\delta\orho^k\alpha, 
	\rho^{k+1}, \orho^{k+1},
    & \mbox{ in weight} -(2k+2,k+1,k+1) \\
	q^{-2k}\gamma\rho^k \alpha & \mbox{ in weight} -(2k+2,k,k+1) \\
	q^{-2k}\beta\rho^k\alpha, & \mbox{ in weight} -(2k+2,k+1,k)     
    \end{cases}\\
C^!_{2k+3} 
  &= \begin{cases} 
	\delta\rho^{k+1}, \delta\orho^{k+1}
    & \mbox{ in weight} -(2k+3,k+2,k+2) \\
	q^{-2k}\delta\rho^k\gamma\alpha, q^{-2k-2}\rho^{k+1}\alpha
    & \mbox{ in weight} -(2k+3,k+1,k+2) \\
	\rho^{k+1}\beta, q^{-2k}\delta\rho^k\beta\alpha
    & \mbox{ in weight} -(2k+3,k+2,k+1) \\
  q^{-2k-2}\rho^{k+1}\alpha, q^{-2k-2}\orho^{k+1}\alpha
    & \mbox{ in weight} -(2k+3,k+1,k+1) \\
  \end{cases}\\
\end{align*}
We denote by $\rho_n: A_n^! \to A^!_{n+2}$ the linear map given by right multiplication 
by $\rho$. Using these bases the matrix associated to $\rho_n$ is the identity for $n 
\geq 3$.



\section{The Hochschild Cohomology of $A$}

\paragraph
Let $w \in \ZZ^3$ and let $C^n(q; w) = (A(q) \ot A^!(q)_n)_w$; we usually ommit the $q$ to
lighten up notation. Setting $D^n_w$ as $D^n_w(x) = e \cdot x - (-1)^n x \cdot e$ we
obtain a complex $C(w) = (C^\bullet(w), D^\bullet_w)$. By \ref{}
\begin{align*}
	HH^n(A(q))_w \cong H^n(C^\bullet(q;w)).
\end{align*}

\paragraph
\label{toric-embedding-complex}
Let $T$ be the quantum torus from \ref{toric-embedding}. Tensoring the complex $K^\bullet
(A)$ with $T$ on the left we obtain a complex $K^\bullet(T) = (T \ot A^!_\bullet, 
d_\bullet)$, and the cohomology of this complex is the Hochschild cohomology of $T$ 
\cite{???}. There is an obvious morphism of $\ZZ^3$-graded dg-algebras $A \ot A^! \to T 
\ot A^!$ which induces a morphism of complexes $K^\bullet(A) \to K^\bullet(T)$. For each
$w \in \ZZ^3$ we set $K(w) = K^\bullet(T)_w$. The complex $C^\bullet(w)$ is isomorphic to
a subcomplex of $K(w)$, and from now on we identify $C^\bullet(w)$ with this subcomplex.

\paragraph
\label{minimal-k}
By definition
\begin{align*}
C^n(w) &= \bigoplus_{s \in \supp A^!_n} A_{w-s} \ot (A^!_n)_s;\\
K^n(w) &= \bigoplus_{s \in \supp A^!_n} T_{w-s} \ot (A^!_n)_s.
\end{align*}
Thus $C^n(w) = K^n(w)$ if and only if $w-s \in S$ for all $s \in \supp A^!_n$. 

Recall that we have set
\begin{align*}
  \V_1
    &= \{(1,0,0); (1,1,0); (1,0,1); (1,1,1);\} \\
  \V_2
  	&= \{(2,1,0); (2,0,1); (2,1,1); (2,2,1); (2,1,2)\},
\end{align*}
and $\V = \V_1 \cup \V_2$. By \ref{A-shriek-basis} for $k \geq 0$
\begin{align*}
	C^0(w) 
		&= K^0(w) 
		&\mbox{ if and only if } w &\in S.\\
	C^{2k+1}(w)
		&= K^{2k+1}(w) 
		&\mbox{ if and only if } w + (2k,k,k) + \mathcal V_1 &\subset S\\
	C^{2k+2}(w)
		&= K^{2k+2}(w) 
		&\mbox{ if and only if } w + (2k,k,k) + \mathcal V_2 &\subset S 
\end{align*}
Since $(2,1,1) \in S$ this shows that $C^n(w) = K^n(w)$ implies $C^m(w) = K^m(w)$ for all
$m \geq n$. The following lemma implies that there exists $n$ such that $C^n(w) = K^n(w)$,
so the equality holds for $n \gg 0$.
\begin{Lemma*}
Let $w \in \ZZ^3$. Then there exists $k(w) \in \NN$ such that $w + k(2,1,1) + v \in S$ for
all $k \geq k(w)$ and all $v \in \V_1 \cup \V_2$.
\end{Lemma*}
\begin{proof}
Assume $w_2 \geq w_3$. Then
\begin{align*}
w + k(2,1,1) 
  &= (w_1 + 2k, w_2 + k, w_3 + k) \\
  &= (w_1 - w_2 + k)(1,0,0) + (w_2-w_3)(1,1,0) + (w_3+k)(1,1,1)
\end{align*}
which lies in $S$ if and only if each coefficient in the linear combination is a positive
integer. Since $w_2 \geq w_3$, we may take $k(w) = \max\{w_2 - w_1, -w_3, 0\}$. For the 
case $w_2 < w_3$ replace $(1,1,0)$ with $(1,0,1)$, and take $k(w) = \max\{w_3 - w_1, 
-w_1, 0\}$.
\end{proof}

\paragraph
\label{HH-in-S}
Assume $w \in S$. Then we can obviously take $k(w)$ in the previous lemma, and hence the 
map $C(w) \to K(w)$ is an isomorphism. From this we obtain 
that
\begin{align*}
  HH^\bullet(A)_w \cong HH^\bullet(T)_w.
\end{align*}
Now $HH^0(T) = Z(T) = \k[(yz^{-1})^{\pm 1}]$, while $HH^1(T)$ is the $HH^0(T)$-free 
module generated by eulerian derivations. Finally $HH^\bullet(T)$ is the exterrior 
$HH^0(T)$-algebra generated by $HH^1(T)$; this is an algebra with support contained in 
the line $(0,k,-k)$ that only crosses $S$ at $(0,0,0)$. Thus we obtain the following 
result.
\begin{Proposition*}
Let $w \in S$. If $w \neq (0,0,0)$ then $HH^n(A)_w = 0$ for all $n \geq 0$, while
\begin{align*}
HH^1(A) 
  &= \langle [a|\alpha]; [a|\alpha + b|\beta]; 
    [a|\alpha + b|\beta + c|\gamma + d|\delta\rangle]
\end{align*}
and $HH^n(A)_{(0,0,0)} = \bigwedge^n HH^1(A)_{(0,0,0)}$
\end{Proposition*}

\paragraph
We will now study the complexes $C(w)$ for $w \notin S$. It turns out that this complex
depends on the position of $w$ relative to $S$. We now make some considerations that will
simplify the analysis.

Given $w = (w_1, w_2, w_3) \in \ZZ^3$ we set 
\begin{align*}
	\Theta(w) 
		&=w^* = (w_1, w_2, w_3)\\
	P(w) 
		&= \overline w
		= (w_1, w_1 - w_2, w_1 - w_3).
\end{align*}
Both $\Theta$ and $P$ are involutions and preserve $S$, so $w \in S$ if and only if $w^*
\in S$ if and only if $\overline w \in S$. Clearly $\Theta$ is the reflection through the
hyperplane $w_2 = w_3$; also $P$ sends the half-space $w_1 \geq w_2 + w_3$ to $w_1 \leq
w_2 + w_3$.

\paragraph
\label{flip-automorphism}
There is an automorphism $f: A \to A$ induced by the assignation.
\begin{align*}
	f(a) &= a & f(b) &= c & f(c) &= b & f(d) &= d
\end{align*}
If we denote by $f^!$ the transpose automorphism, we get
\begin{align*}
	f^!(\alpha) &= \alpha 
		& f^!(\beta) &= \gamma 
		& f^!(\gamma) &= \beta 
		& f^!(\delta) &= \delta
\end{align*}
Then for $w \in \ZZ^3$ we see that $f(A_w) = A_{w^*}$ and $f^!(A^!_w) = A^!_{w^*}$.
Setting $F = f \ot f^!$ we obtain an algebra automorphism of $A \ot A^!$ such that 
$F(A \ot A^!)_w = (A \ot A^!)_{w^*}$. Furthermore it is clear that $F(e) = e$, and since
$F$ is an algebra automorphism it induces an isomorphism of complexes $F: C(w) \to
C(w^*)$. Thus we may restrict our attention to $C(w)$ with $w_2 \geq w_3$.

\paragraph
\label{inverse-automorphism}
There is also an algebra isomorphism $g: A(q) \to A(q^{-1})$ given by 
\begin{align*}
  g(a) &= d & g(b) &= c & g(c) &= b & g(d) &= a,
\end{align*}
with transpose automorphism $g^!: A^!(q) \to A^!(q^{-1})$ given by 
\begin{align*}
  g^!(\alpha) &= \delta 
    & g^!(\beta) &= \gamma 
    & g^!(\gamma) &= \beta 
    & g^!(\delta) &= \alpha
\end{align*}
Let $G = g \ot g^!$, then $G$ is an algebra isomorphism and
\[
	G((A(q) \ot A^!(q))_{w} = (A(q^{-1}) \ot A^!(q^{-1}))_{w^*},
\]
so $C(q;w) \cong C(q^{-1}; \overline w)$, and we can restrict our atention to the case
where $w_1 \leq w_2 + w_3$.

\paragraph
\label{W-equations}
Set 
\[
	W = \{w \in \ZZ^3 \mid w_1 \leq w_2 + w_3 \mbox{ and } w_3 \leq w_2\} \cap S^c.
\]
By the previous pargraphs, we only need to study the cohomology of $C(w)$ with $w \in W$.
Before we proceed we make the following remark; assume $w \in W$, and that $w_1 \geq 
w_2$. Since $w_2 \geq w_3$ we get $w_1 \geq \max\{w_2, w_3\}$, and furthermore $0 \leq 
w_1 - w_2 \leq w_3$, so $w \in \NN_0^3$ and hence $w \in S$, which contradicts the 
definition of $W$. It follows that
\begin{align*}
  W 
    &= \{w \in \ZZ^3 \mid w_1 \leq w_2 + w_3; w_3 \leq w_2; w_1 < w_2\}
\end{align*}
If $w \in \ZZ^3$  then
\begin{align*}
   w = (w_2 + w_3 - w_1)(1,1,1) + (w_2 - w_3)(1,1,0) + (w_2 - w_1)(-2,-1,-1),
\end{align*}
which implies
\begin{align*}
  W
    &= (-2,-1,-1) + \NN_0(1,1,1) + \NN_0(1,1,0) + \NN_0(-2,-1,-1).
\end{align*}

\paragraph
\label{finite-cohomology-C}
We now begin out analysis of the cohomology of $C(w)$. Our first result states that
this cohomology is always finite.

\begin{Lemma*}
Let $w \in W$. 
\begin{enumerate}	
  \item $C^n(w) \to K^n(T)_w$ is an isomorphism for $n \geq 2(w_2 - w_1) + 1$.
	\item $C^n(w) = 0$ for $n \leq 2(w_2 - w_1)-3$.	
  \item $H^n(C(w)) = HH^n(A)_w = 0$ for $n \notin 2(w_2 - w_1) + [-2,1]$
\end{enumerate}
\end{Lemma*}
\begin{proof}
	As seen in \ref{minimal-k} we have $w + (2k,k,k) \in S$ if and only if $k \geq k(w) = 
	\max \{w_2 - w_1, -w_3, 0\}$, and since $w \in W$ we get $k(w) = w_2 - w_1$. Since
	$\mathcal V \subset S$ we get that $w + (2k,k,k) + \V \subset S$ and by \ref{minimal-k}
  $C^n(w) = K^n(w)$ for $n \geq 0 + 1$.

  Now assume $C^{n}(w) \neq 0$ for $n = 2(w_2 - w_1) - 3 = 2(w_2 - w_1 - 2) + 1$. Once 
  again by \ref{minimal-k} this implies that there exists $v \in \V_1$ such that $w + 
  (w_2 - w_1 - 2)(2,1,1) + v \in S$. Now $v + \overline v = (2,1,1)$ for each $v \in 
  \V_1$, which implies $w + (w_2 - w_1 - 1)(2,1,1) \in S$ and this contradicts the fact 
  that $k(w) = w_2 - w_1$.

  Clearly the previous item implies $HH^n(A)_w = 0$ for $n < 2(w_2 - w_1) - 2$. On the 
  other hand, $HH^n(A)_w = HH^n(T)_w$ for $n > 2(w_2 - w_1) + 1$. Since $w_2 - w_1 
  \geq 1$ we get $2(w_2 - w_1) + 1 \geq 3$, and $HH^n(T) = 0$ for $n > 3$.
\end{proof}


\paragraph
\label{subcomplex-tilde}
Define a subcomplex $\tilde C(w) \subset C(w)$ by setting
\begin{align*}
\tilde C^n(w)
  &= \begin{cases}
    C^n(w) & \mbox{ for } n \leq 2(w_2 - w_1) \\
    \ker D^{2(w_2-w_1)+1}(w) & \mbox{ for } n = 2(w_2 - w_1) + 1 \\
    0 &\mbox{ for } n \geq 2(w_2 - w_1) +2
  \end{cases}
\end{align*}
It follows from Lemma \ref{finite-cohomology-C} that $H^n(C(w)) = H^n(\tilde C(w))$.



\paragraph
\label{tune-table}
Let 
\begin{align*}
  W_{<,<} 
    &= \{w \in W \mid w_1 < w_2 + w_3; w_3 < w_2\} \\
    &= (0,1,0) + \NN_0(1,1,1) + \NN_0(1,1,0) + \NN_0(-2,-1,-1);\\
  W_{<,=} 
    &= \{w \in W \mid w_1 < w_2 + w_3; w_3 = w_2\} \\
    &= (-1,0,0) + \NN_0(1,1,1) + \NN_0(-2,-1,-1);\\
  W_{=,<} 
    &=  \{w \in W \mid w_1 = w_2 + w_3; w_3 < w_2\} \\
    &= (-1,0,-1) + \NN_0(1,1,0) + \NN_0(-2,-1,-1);\\
  W_{=,=} 
    &= \{w \in W \mid w_1 = w_2 + w_3; w_3 = w_2\} \\
    &= \NN(-2,-1,-1).\\
\end{align*}
Clearly $W$ is the union of these four sets. The following is a table recording the 
first $n$ such that $w + (w_2 - w_1 + n)(2,1,1) + v \in S$ for each $v \in \V$. It turns 
out that the set of these $n$ deppends on which of the previous four sets $w$ lies. 

\begin{tabular}{|>{$}c<{$}||>{$}c<{$}|>{$}c<{$}|>{$}c<{$}|>{$}c<{$}|}
\hline
v & <,< & <,= & =,< & =,=\\
\hline
(1,0,0) & -1 & -1 & +1 & +1\\
(1,1,0) & +1 & +1 & +1 & +1\\
(1,0,1) & -1 & +1 & -1 & +1\\
(1,1,1) & +1 & +1 & +1 & +1 \\
(2,1,0) & 0  & 0  & 0  & +2\\
(2,0,1) & -2 & 0  & +2 & +2\\
(2,1,1) & 0  & 0  & 0  & 0  \\
(2,2,1) & +2 & +2 & +2 & +2\\
(2,1,2) & 0  & +2 & 0  & +2\\
\hline
\min: 2(w_2-w_1) & -2 & -1 & -1 & 0\\
\hline
\end{tabular}
\bigskip


\begin{Lemma}
If $w \in W_{<,<}$ then $C^n(w)$ is exact.
\end{Lemma}
\begin{proof}
First notice that $(0,t,-t) \notin W_{<,<}$, so $K^n(w)$ is an exact complex.

\textbf{If $w_2 - w_1 = 1$}. Then
\begin{align*}
\tilde C^{0}(w) 
  &= 0 \\
\tilde C^{1}(w) 
  &= \langle [w+(1,0,0)] | \alpha; [w+(1,0,1)] | \gamma \rangle \\
\tilde C^{2}(w)
  &= \langle  [w+(2,1,2)] |\delta\gamma;  [w+(2,1,1)]|\delta\alpha; [w+(2,1,1)]|\rho; \\
    &\qquad [w+(2,1,1)]|\orho; [w+(2,0,1)]|\gamma\alpha; [w+(2,1,0)]|\beta\alpha\rangle; 
    \\
\tilde C^3(w) 
  &= \ker D^3(w).
\end{align*}
As seen in \ref{torus-cohomology} the dimension of the image of $D^n$ is $4$ for $n 
\geq 2$, and hence $\dim \tilde C^3(w) = 4$; furthermore $\tilde D^2(\tilde C^2(w)) = 
D^2(K^2(w))$ so $H^3(\tilde C(w)) = 0$ and $\dim \ker D^2(w) = 2$. Now the kernel of 
$\tilde D^1(w)$ is 
\[
  \ker D^1 \cap C^1(w) = \langle D^0([w]|1)\rangle \cap C^1(w),
\]
and since the coefficient of $[w+(1,1,0)]|\beta$ in $ D^0([w]|1)$ is $(q^{-w_1} - 
q^{-w_2-w_3})$, which is not $0$ since $w_1 < w_2 + w_3$, we see that $\tilde D^1$ is 
injective, which implies its image is exactly the kernel of $\tilde D^2$. From this it 
follows that the rest of the cohomology spaces of $\tilde C(w)$ are zero.

\textbf{If $w_2 - w_1 \geq 2$}. Then 
\begin{align*}
\tilde C(w)
  = \cdots 0 &\to 
   K^{2k-2}_w(2,0,1) \to
    K^{2k-1}_w(1,0,0) \oplus K^{2k-1}_w(1,0,1) \to \\
    & \to \bigoplus_{v \in \V_2 \setminus (2,2,1)} K^{2k}_w(v)
      \to \ker D^{2k+1} \to 0 \to \cdots
\end{align*}
Let
\begin{align*}
\hat C(w)
  = \cdots 0 &\to 
   K^{2}_w(2,0,1) \to
    K^{3}_w(1,0,0) \oplus K^{3}_w(1,0,1) \to \\
    & \to \bigoplus_{v \in \V_2 \setminus (2,2,1)} K^{4}_w(v)
      \to \ker D^{5} \to 0 \to \cdots
\end{align*}
Then $(ad|\rho)\cdot : \hat C(w) \to \tilde C(w)$ is an isomorphism of complexes, so it 
is enough to check that $\hat C(w)$ is acyclic. By the same reasoning as above $\dim \hat
C(w) = (\ldots, 0, 1,4,7,4, 0 \ldots)$ and furthermore $\hat D^{4}$ is surjective so 
$HH^5(\tilde C(w)) = 0$ and $\dim \ker \hat D^4 = 3$. On the other hand 
$D^2(w)([w+(2,0,1)]|\gamma \alpha) \neq 0$, so $H^2(\hat C^2(w)) = 0$. Finally we can 
use the bases given in \ref{dual-PBW-basis} and get
\begin{align*}
[\hat D^3]
  =
\left(
\begin{array}{cccc}
 -q^{-w_2-w_3-4} & q^{-2 (w_1+3)}-q^{-(w_2+w_3+6)} & 0 & 0 \\
 -q^{-w_2-w_3-4} & 0 & q^{-2 (w_1+3)}-q^{-(w_2+w_3+6)} & 0 \\
 -q^{-w_1-4} & 0 & 0 & q^{-2 (w_1+3)}-q^{-(w_2+w_3+6)} \\
 0 & q^{-w_2-w_3-3} & -q^{-w_2-w_3-3} & 0 \\
 -q^{-1} & q^{-w_2-w_3-3}-q^{-3} & -q^{-w_2-w_3-3} & q^{-w_1-3} \\
 0 & q^{-w_1-3} & 0 & -q^{-w_2-w_3-3} \\
 0 & 0 & q^{-w_1-3} & -q^{-w_2-w_3-3} \\
\end{array}
\right)
\end{align*}
If $w_1 \neq -3$ then the minor $[567|123]$ is not zero. If $w_1 = - 3$ then $w_2 + w_3 
> -3$, so the minor $[123|234]$ is non-zero. In either case, the rank of $\hat D^3$ is 
$3$ and hence $\hat C(w)$ is acyclic.
\end{proof}



\newpage

\newpage

The tune to $K^n(w)$

$w+(2,1,1)k+$

\begin{tikzpicture}[every node/.style = {fill=white, minimum size=0.9em}]
	\draw (0,0) -- (7,0) (0,0.5) -- (7,0.5) (0,1) -- (7,1) 
				(0,1.5) -- (7,1.5) (0,2) -- (7,2);
	
	\node (one) at (0.5,1) {$1$};
	
	\node (a) at (2,0.25) {$1$};
	\node (b) at (2,0.75) {$1$};
	\node (c) at (2,1.25) {$1$};
	\node (d) at (2,1.75) {$1$};
	
	\node (ba) at (3.5, 2) {$1$};
	\node (ca) at (3.5, 1.5) {$1$};
	\node (da) at (3.5, 1) {$3$};
	\node (db) at (3.5, 0.5) {$1$};
	\node (dc) at (3.5, 0) {$1$};

	\node (cba) at (5,0.25) {$2$};
	\node (cbc) at (5,0.75) {$2$};
	\node (bcb) at (5,1.25) {$2$};
	\node (dbc) at (5,1.75) {$2$};
	
	\node (ba) at (6.5, 2) {$1$};
	\node (ca) at (6.5, 1.5) {$1$};
	\node (da) at (6.5, 1) {$4$};
	\node (db) at (6.5, 0.5) {$1$};
	\node (dc) at (6.5, 0) {$1$};

	\node () at (0.0, -0.75) {$n:$};
	\node () at (0.5, -0.75) {$0$};
	\node () at (2.0, -0.75) {$1$};
	\node () at (3.5, -0.75) {$2$};
	\node () at (5.0, -0.75) {$2k+1$};
	\node () at (6.5, -0.75) {$2k+2$};

	\node () at (0.0, -1.75) {$k$:};
	\node () at (0.5, -1.75) {$-1$};
	\node () at (2.0, -1.75) {$0$};
	\node () at (3.5, -1.75) {$0$};
	\node () at (5.0, -1.75) {$k$};
	\node () at (6.5, -1.75) {$k$};

	
	\node () at (9,0) {$(2,1,0)$};
	\node () at (9,0.5) {$(2,0,1)$};
	\node () at (9,1) {$(2,1,1)$};
	\node () at (9,1.5) {$(2,2,1)$};
	\node () at (9,2) {$(2,1,2)$};

	\node () at (-2,0.25) {$(1,0,0)$};
	\node () at (-2,0.75) {$(1,1,0)$};
	\node () at (-2,1.25) {$(1,0,1)$};
	\node () at (-2,1.75) {$(1,1,1)$};
\end{tikzpicture}

The tune to $w$ with $w_1 < w_2 + w_3$ and $w_3 < w_2$, with $n_0 = 2(w_2 - w_1) -2$.

\tikzstyle{inter} = [draw, circle, fill=white, minimum size=0.9em]

\begin{tikzpicture}
	\draw (0,0) -- (7,0) (0,0.5) -- (7,0.5) (0,1) -- (7,1) 
				(0,1.5) -- (7,1.5) (0,2) -- (7,2);
	
	\node[inter] (one) at (0.5,0.5) {};
	
	\node[inter] (a) at (2,0.25) {};
	\node[inter] (c) at (2,1.25) {};
	
	\node[inter] (ba) at (3.5, 2) {};
	\node[inter] (da) at (3.5, 1) {};
	\node[inter] (db) at (3.5, 0.5) {};
	\node[inter] (dc) at (3.5, 0) {};

	\node[inter] (cba) at (5,0.25) {};
	\node[inter] (cbc) at (5,0.75) {};
	\node[inter] (bcb) at (5,1.25) {};
	\node[inter] (dbc) at (5,1.75) {};
	
	\node[inter] (ba) at (6.5, 2) {};
	\node[inter] (ca) at (6.5, 1.5) {};
	\node[inter] (da) at (6.5, 1) {};
	\node[inter] (db) at (6.5, 0.5) {};
	\node[inter] (dc) at (6.5, 0) {};

	\node () at (0.0, -0.75) {$n:$};
	\node () at (0.5, -0.75) {$n_0$};
	\node () at (2.0, -0.75) {$+ 1$};
	\node () at (3.5, -0.75) {$+ 2$};
	\node () at (5.0, -0.75) {$+ 2k+1$};
	\node () at (6.5, -0.75) {$+ 2k+2$};
\end{tikzpicture}

The tune to $C^n(w)$ for $w_1 < w_2 + w_3$ and $w_2 = w_3$ , with $n_0 = 2(w_2 - w_1)
-2$.

\begin{tikzpicture}[every node/.style = {fill=white, minimum size=0.9em}]
  \draw (0,0) -- (7,0) (0,0.5) -- (7,0.5) (0,1) -- (7,1) 
        (0,1.5) -- (7,1.5) (0,2) -- (7,2);
  
  %\node[inter] (one) at (0.5,1) {};
  
  \node[inter] (a) at (2,0.25) {};
  %\node[inter] (b) at (2,0.75) {};
  %\node[inter] (c) at (2,1.25) {};
  %\node[inter] (d) at (2,1.75) {};
  
  %\node[inter] (ba) at (3.5, 2) {};
  %\node[inter] (ca) at (3.5, 1.5) {};
  \node[inter] (da) at (3.5, 1) {};
  \node[inter] (db) at (3.5, 0.5) {};
  \node[inter] (dc) at (3.5, 0) {};

  \node[inter] (cba) at (5,0.25) {};
  \node[inter] (cbc) at (5,0.75) {};
  \node[inter] (bcb) at (5,1.25) {};
  \node[inter] (dbc) at (5,1.75) {};
  
  \node[inter] (ba) at (6.5, 2) {};
  \node[inter] (ca) at (6.5, 1.5) {};
  \node[inter] (da) at (6.5, 1) {};
  \node[inter] (db) at (6.5, 0.5) {};
  \node[inter] (dc) at (6.5, 0) {};

  \node () at (0.0, -0.75) {$n:$};
  \node () at (0.5, -0.75) {$n_0$};
  \node () at (2.0, -0.75) {$+1$};
  \node () at (3.5, -0.75) {$+2$};
  \node () at (5.0, -0.75) {$+2k+1$};
  \node () at (6.5, -0.75) {$+2k+2$};
\end{tikzpicture}

The tune to $C^n(w)$ for $w_1 = w_2 + w_3$ and $w_2 < w_3$ , with $n_0 = 2(w_2 - w_1)
-2$.

\begin{tikzpicture}[every node/.style = {fill=white, minimum size=0.9em}]
  \draw (0,0) -- (7,0) (0,0.5) -- (7,0.5) (0,1) -- (7,1) 
        (0,1.5) -- (7,1.5) (0,2) -- (7,2);
  
  %\node[inter] (one) at (0.5,1) {};
  
  %\node[inter] (a) at (2,0.25) {};
  %\node[inter] (b) at (2,0.75) {};
  \node[inter] (c) at (2,1.25) {};
  %\node[inter] (d) at (2,1.75) {};
  
  \node[inter] (ba) at (3.5, 2) {};
  %\node[inter] (ca) at (3.5, 1.5) {};
  \node[inter] (da) at (3.5, 1) {};
  %\node[inter] (db) at (3.5, 0.5) {};
  \node[inter] (dc) at (3.5, 0) {};

  \node[inter] (cba) at (5,0.25) {};
  \node[inter] (cbc) at (5,0.75) {};
  \node[inter] (bcb) at (5,1.25) {};
  \node[inter] (dbc) at (5,1.75) {};
  
  \node[inter] (ba) at (6.5, 2) {};
  \node[inter] (ca) at (6.5, 1.5) {};
  \node[inter] (da) at (6.5, 1) {};
  \node[inter] (db) at (6.5, 0.5) {};
  \node[inter] (dc) at (6.5, 0) {};

  \node () at (0.0, -0.75) {$n:$};
  \node () at (0.5, -0.75) {$n_0$};
  \node () at (2.0, -0.75) {$+1$};
  \node () at (3.5, -0.75) {$+2$};
  \node () at (5.0, -0.75) {$+2k+1$};
  \node () at (6.5, -0.75) {$+2k+2$};
\end{tikzpicture}

The tune to $C^n(w)$ for $w_1 = w_2 + w_3$ and $w_2 = w_3$, with $n_0 = 2(w_2-w_1)$.

\begin{tikzpicture}[every node/.style = {fill=white, minimum size=0.9em}]
	\draw (0,0) -- (7,0) (0,0.5) -- (7,0.5) (0,1) -- (7,1) 
				(0,1.5) -- (7,1.5) (0,2) -- (7,2);
	
	\node[inter] (one) at (0.5,1) {};
	
	\node[inter] (a) at (2,0.25) {};
	\node[inter] (b) at (2,0.75) {};
	\node[inter] (c) at (2,1.25) {};
	\node[inter] (d) at (2,1.75) {};
	
	\node[inter] (ba) at (3.5, 2) {};
	\node[inter] (ca) at (3.5, 1.5) {};
	\node[inter] (da) at (3.5, 1) {};
	\node[inter] (db) at (3.5, 0.5) {};
	\node[inter] (dc) at (3.5, 0) {};

	\node[inter] (cba) at (5,0.25) {};
	\node[inter] (cbc) at (5,0.75) {};
	\node[inter] (bcb) at (5,1.25) {};
	\node[inter] (dbc) at (5,1.75) {};
	
	\node[inter] (ba) at (6.5, 2) {};
	\node[inter] (ca) at (6.5, 1.5) {};
	\node[inter] (da) at (6.5, 1) {};
	\node[inter] (db) at (6.5, 0.5) {};
	\node[inter] (dc) at (6.5, 0) {};

	\node () at (0.0, -0.75) {$n:$};
	\node () at (0.5, -0.75) {$n_0$};
	\node () at (2.0, -0.75) {$+1$};
	\node () at (3.5, -0.75) {$+2$};
	\node () at (5.0, -0.75) {$+2k+1$};
	\node () at (6.5, -0.75) {$+2k+2$};
\end{tikzpicture}


\newpage
\section{The complex $T \ot A^!$}

\paragraph
Fix $w \in \ZZ^3$. For every $k \geq 0$ we set 
\begin{align*}
	K^{2k+1}_w(v) &= K^{2k+2}_w(v) = T_{w + k(2,1,1) + v} \ot A^!_{-k(2,1,1)-v}.
\end{align*}

With this notation
\begin{align*}
	K^0_w &= T_w &
	K^{2k+1}_w &= \bigoplus_{v \in \V_1} K^{2k+1}_w(v) &
	K^{2k+2}_w &= \bigoplus_{v \in \V_2} K^{2k+2}_w(v)
\end{align*}

\paragraph
Let $\tau \in A^!$ be a homogenenous nonzero element of weight $v$. Then setting $(i,j,k)
= v +w$ we denote write $[\tau]_w = x^i y^j z^k \ot \tau$. With this notation we obtain
bases
\begin{align*}
	K_w^0(0,0,0) &= \left \langle [1]_w \right \rangle \\
	K_w^{2k+1}(1,1,1)
		&=\left\langle 
			q^{-2k}[\delta \rho^k]_{w};
			q^{-2k}[\delta \orho^k]_{w} 
		\right\rangle\\
	K_w^{2k+1}(1,0,1)
		&= \left\langle  
			q^{-2k} [\delta \gamma \rho^{k-1}\alpha]_{w};  
			[\gamma \rho^k]_{w}
		\right\rangle\\
	K_w^{2k+1}(1,1,0)
		&=\left\langle 
			q^{-2k}[\delta \rho^{k-1}\beta\alpha]_{w}; 
			[\rho^k \beta]_{w}
		\right\rangle\\
	K_w^{2k+1}(1,0,0)
		&=\left\langle 
			q^{-2k} [\rho^k\alpha]_{w},q^{-2k}[\orho^k\alpha]_{w}
		\right\rangle\\
	K_w^{2k+2}(2,1,2)
		&=\left\langle
			[\delta\gamma\rho^k]_{w}
		\right\rangle\\
	K_w^{2k+2}(2,2,1)
		&=\left\langle
			[\delta\beta\rho^k]_{w}
		\right\rangle\\
	K_w^{2k+2}(2,1,1)
		&=\left\langle
			q^{-2k}[\delta\rho^k\alpha]_{w}; 
			q^{-2k}[\delta\orho^k\alpha]_{w};
			[\rho^{k+1}]_{w}; [\orho^{k+1}]_{w}
		\right\rangle\\
	K_w^{2k+2}(2,0,1)
		&=\left\langle
			q^{-2k}[\rho^k\gamma\alpha]_{w}
		\right\rangle\\
	K_w^{2k+2}(2,1,0)
		&=\left\langle
			q^{-2k}[\rho^k\beta\alpha]_{w}
		\right\rangle
\end{align*}
(for $k = 0$ we consider $\rho^{-1} = 0$ and delete all zero elements from the basis).




\begin{Lemma}
\begin{enumerate}
	\item The element $ad|\rho \in T \ot A^!$ is regular and of weight $0$.
	\item $[e,ad|\rho] = 0$.
	\item $K_w$ is exact for $w \notin \langle (0,1,-1)\rangle$.
\end{enumerate}
\end{Lemma}

\newpage

\section{Affine semigroup algebras}


\end{document}



























\paragraph
For each $w \in S$ we have a basis of $A_w$, given by the unique element in the 
PBW-basis of $A$ of weight $w$, which we will denote $[w]$. Set $w^k = w + (2k,k,k)$, 
then $C^0(q;w) = [w]|1$, and setting $\orho^{-1} = 0, \orho^{0} = 1$ we obtain bases

\begin{tikzpicture}
\node (d) at (3,4) 
  {
  $[w^k+(1,1,1)]\left| 
    \begin{matrix}
      \delta\rho^k \\ 
      \delta\orho^k 
    \end{matrix} \right.$
  };
\node (c) at (6,2) 
  {
    $[w^k+(1,0,1)]\left| 
      \begin{matrix}
        \gamma\rho^k \\ 
        q^{-k+2}\delta\gamma\rho^{k-1} \alpha
      \end{matrix} \right.$
  };
\node (b) at (0,2)
  {
    $[w^k+(1,1,0)]\left| 
      \begin{matrix}
        \rho^k\beta \\ 
        q^{-2k+2}\delta\rho^{k-1}\beta\alpha 
      \end{matrix} \right.$
  };
\node (a) at (3,0)
  {
    $[w^k+(1,0,0)]\left| 
      \begin{matrix}
        q^{-2k}\rho^k\alpha \\ 
        q^{-2k}\orho^k\alpha 
      \end{matrix} \right.$
  };
\node (t) at (3,-1) {Basis for $C^{2k+1}(q; w)$};
\draw (a) -- (b) -- (d) -- (c) -- (a);
\end{tikzpicture}

\begin{tikzpicture}
\node (tl) at (0,6)
  {
    $[w^k+(2,2,1)]| \delta \rho^k \beta$
  };
\node (tr) at (6,6)
{
    $[w^k+(2,1,2)]| \delta \rho^k \gamma$
  };
\node (c) at (3,3)
{
    $[w^k+(2,1,1)]\left| 
      \begin{matrix}
        q^{-2k}\delta \rho^k \alpha \\
        q^{-2k}\delta \orho^k \alpha \\
        \rho^{k+1}\\
        \orho^{k+1}
      \end{matrix} \right.$
  };
\node (bl) at (0,0)
{
    $[w^k+(2,1,0)]| q^{-2k}\beta \rho^k \alpha$
  };
\node (br) at (6,0)
{
    $[w^k+(2,0,1)]| q^{-2k}\gamma \rho^k \alpha$
  };
\node (t) at (3,-1) {Basis for $C^{2k+2}(q; w)$};
\draw (br) -- (c) -- (tr) (tl) -- (c) -- (bl);
\end{tikzpicture}


