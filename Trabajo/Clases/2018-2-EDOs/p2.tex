\documentclass[11pt,a4paper,pdftex]{amsart}
\usepackage[psamsfonts]{amssymb}
\usepackage{amsmath,amsfonts,latexsym}
\usepackage[dvips]{graphicx}
\usepackage[utf8x]{inputenc}
\usepackage[spanish]{babel}
\usepackage{epsfig}
\usepackage{amscd}
\usepackage{verbatim}
\usepackage{multicol}

\newtheorem{teo}{Teorema}[section]%establece un contador para
%el entorno 'teo' que aparecer� con el nombre 'Teorema' y que
%volver� a empezar cuando cambien de 'chapter'.

\newtheorem{coro}[teo]{Corolario}%vincula las numeraciones de 'coro'
%con las de 'teo'.

\newtheorem{lema}{Lema}[section]
%\newtheorem{lema}[lema]{Lema}


\newtheorem{definition}{Definici\'on}[section]

\newtheorem{conc}{Conclusión}[section]

\newtheorem{prop}{Proposición}[section]
%\newtheorem{prop}[prop]{Propiedad}

\newtheorem{obs}{Observación}[section]

\renewcommand{\thesection}{{}}

%\newtheorem{obs}{Observaci�n} %numera las 'Observaciones' de corrido
%sin volver a resetear.

\newtheorem{ax}{Axioma}[section]

%\newtheorem{defini}{Definici�n}[section]

\newtheorem{ej}{Ejercicio}%[section] %numera os 'Ejercicios' reseteando
%en cada 'cap�tulo'.


\numberwithin{equation}{section}%numera las formulas reseteando
%cada vez que cambia de 'capítulo'.

\newcommand{\bej}[1]{\begin{ej}\rm{#1}}
\newcommand{\eej}{\end{ej}\vspace{-0.2cm}}

%---------------------------------------
\newcommand{\be}{\begin{enumerate}}
\newcommand{\ee}{\end{enumerate}}
\newcommand{\bit}{\begin{itemize}}
\newcommand{\eit}{\end{itemize}}
\newcommand{\bc}{\begin{center}}
\newcommand{\ec}{\end{center}}
\newcommand{\ba}{\begin{array}}
\newcommand{\ea}{\end{array}}
\newcommand{\bq}{\begin{quotation}}
\newcommand{\eq}{\end{quotation}}
\newcommand{\mc}[1]{\mathcal{#1}}
\newcommand{\mb}[1]{\;\mbox{#1}\;}
\newcommand{\su}[1]{\underline{#1}}
\newcommand{\so}[1]{\overline{#1}}
\newcommand{\ang}[1]{\widehat{#1}}
\newcommand{\arc}[1]{\wideparen{#1}}
\newcommand{\Cc}{QQ\;}
\renewcommand{\bf}{\textbf}
\newcommand{\Comb}[2]{\left(\!\!\!\ba{c}#1\\[1ex]#2 \ea
\!\!\!\right)}
\newcommand{\di}{\displaystyle}
%-----------------------------------------
%conjuntos
\newcommand{\W}{\mathbb W}
\newcommand{\K}{\mathbb K}
\newcommand{\N}{\mathbb N}
\newcommand{\C}{\mathcal C}
\newcommand{\Su}{\mathcal S}
\newcommand{\Z}{\mathbb Z}
\newcommand{\Q}{\mathbb Q}
\newcommand{\R}{\mathbb R}
\newcommand{\F}{\mathbb F}
\newcommand{\A}{\mathbb A}
\newcommand{\V}{\mathbb V}
\newcommand{\I}{\mathbb I}
\newcommand{\0}{\mathbb O}
%------------------------------------------
%operaciones
\newcommand{\8}{\infty}
\newcommand{\ie}{\langle}
\newcommand{\de}{\rangle}
\newcommand{\pe}[2]{\ie {#1} \, ; \, {#2} \de }
\newcommand{\f}[1]{\overrightarrow{#1}}
\newcommand{\DT}[2]{\frac{d{#1}}{d{#2}}}
\newcommand{\ds}{\displaystyle}
\newcommand{\dg}{\Delta}
\newcommand{\g}{\nabla}
\newcommand{\D}{\mbox{div}}
\newcommand{\Dp}[1]{\partial_{#1}}
\newcommand{\DP}[2]{\frac{\partial{#1}}{\partial{#2}}}
\newcommand{\sen}[1]{\mbox{sen}\;{#1}}
\newcommand{\Lp}[1]{L^{#1}}
\newcommand{\To}{\longrightarrow}
%-------------------------------------
%griegos
\newcommand{\vi}{\varphi}
\newcommand{\om}{\omega}
\newcommand{\Om}{\Omega}
\newcommand{\ve}{\varepsilon}
%-------------------------
%comandos particulares
\newcommand{\id}[1]{\text{id}_{#1}}
\newcommand{\mD}{\mc{D}}
\newcommand{\mC}{\mc{S}}
\newcommand{\mS}{\mc{SS}}
\newcommand{\mH}{\mc{H}}
\newcommand{\HD}{\mc{H}_\mD}
%%%%%%%%%%%%%%%%%%%%%%%%%%%%%%%%%%%%%%%%%%%%%%%%%%%%%%%%%%%%%%%%%
%Un comando para insertar dibujos
\newcommand{\putfig}[4]{\bigskip \bigskip
             \begin{figure}[ht]
             \epsfxsize=#1cm\hfil{\epsfbox{#2}}
             \caption{#3}
         \label{#4}
             \end{figure}\bigskip}
%La sintaxis
%\putfig{ancho}{.eps}{titulo}{label}
%%%%%%%%%%%%%%%%%%%%%%%%%%%%%%%%%%%%%%%%%%%%%%%%%%%%%%%
%Un comando para insertar dos dibujos
\newcommand{\putfigg}[4]{\bigskip \bigskip
             \begin{figure}[ht]
             \epsfxsize=5cm\hfil{\epsfbox{#1}}
             \epsfxsize=5cm\hfil{\epsfbox{#2}}
             \caption{#3}
         \label{#4}
             \end{figure}\bigskip}
%La sintaxis
%\putfigg{.eps}{.eps}{titulo1}{label1}
%%%%%%%%%%%%%%%%%%%%%%%%%%%%%%
%un dibujo en jpg
\newcommand{\putjpg}[3]{\bigskip
\begin{center}
 \begin{figure}[ht]
  \hspace{5cm}\includegraphics[scale=1.7]{#1.jpg}
  \caption{#2}
  \label{#3}
  \bigskip
  \bigskip
 \end{figure}
\end{center}
}
%la sintaxis \putjpg{.jpg}{Titulo}{label}
%----------------------------------------------------
%diseñoo de la página

\vfuzz7pt % Don't report over-full v-boxes if over-edge is 7pt small
\hfuzz7pt % Don't report over-full h-boxes if over-edge is 7pt small

\pagestyle{myheadings}

\renewcommand{\labelenumi}{({\it \alph{enumi}})}
\renewcommand{\labelenumii}{\arabic{enumii})}

\flushbottom \topmargin-0.5cm \textwidth17cm \textheight24.5cm \hoffset=-2cm
\voffset=-0.5cm


\begin{document}



%----------------------------------------------------------
%Encabezado:

\centerline{{\small Universidad de Buenos Aires - Facultad de Ciencias Exactas y Naturales - Depto. de Matemática}}

 \vskip 0.2cm
 \hrulefill
 \vskip 0.2cm

 \centerline{{\bf{\Large{\sc Análisis II - Análisis Matemático II - Matemática 3}}}}
 \vskip 0.2cm
 \centerline{\ttfamily Primer Cuatrimestre 2018}
 \hrulefill

 \medskip
 \centerline{\bf {Práctica 6: Ecuaciones de segundo orden y sistemas de de primer orden.}}
 \medskip

\setcounter{equation}{0}



\bej  Encontrar un sistema fundamental de soluciones reales de las siguientes
ecuaciones:
\[
\begin{array}{ll}
\mbox{i)}&y''-8y'+16y=0\\
\mbox{ii)}&y''-2y'+10y=0\\
\mbox{iii)}&y''-y'-2y=0
\end{array}
\]

 En cada uno de los casos anteriores encontrar una soluci\'on exacta
de la ecuaci\'on no homog\'enea correspondiente con t\'ermino
independiente $x, e^x,1 \mbox{ y } e^{-x}$.

\eej

\bej Sean $(a_{1},b_{1})$ y $(a_{2},b_{2})$ dos puntos del plano tales que
$\frac{a_{1}-a_{2}}{\pi}$ no es un n\'umero entero.
\begin{enumerate}
\item Probar que existe exactamente una soluci\'on de la ecuaci\'on
diferencial $y''+y=0$ cuya gr\'afica pasa por esos puntos.

\item ?`Se cumple en alg\'un caso la parte (a) si $a_{1}-a_{2}$ es un m\'ultiplo
entero de $\pi$?

\item Generalizar el resultado de (a) para la ecuaci\'on $y''+k^{2}y=0$. Discutir
tambi\'en el caso $k=0$.
\end{enumerate}
\eej

\bej Hallar todas las soluciones de $y''-y'-2y=0$ y de $y''-y'-2y= e^{-x}$
que verifiquen:
\[
\begin{array}{llllll}
\mbox{i)} &y(0)=0,\ &y'(0)=1\qquad\quad&\mbox{ii)}&y(0)=1,\ \ \ y'(0)=0\\
\mbox{iii)}&y(0)=0,&y'(0)=0&\mbox{iv)}&\lim_{x \rightarrow +\infty} y(x)=0&\\
\mbox{v)}&y(0)=1&&\mbox{vi)}&y'(0)=1&
\end{array}
\]
\eej

\bej En el interior de la Tierra, la fuerza de gravedad es
proporcional a la distancia al centro. Si se perfora un orificio
que atraviese la Tierra pasando por el centro, y se deja caer
una piedra en el orificio, ?`con qu\'e velocidad llegar\'a al centro?.
\eej

\bej La ecuaci\'on $x^{2}y''+pxy'+qy=0$ ($p,q$ constantes) se denomina
ecuaci\'on de Euler.

\begin{enumerate}
\item Demuestre que el cambio de variables $x= \mbox{e}^{t}$ transforma la
ecuaci\'on en una con coeficientes constantes.

\item Aplique (a) para resolver en $\R_{>0}$ las ecuaciones:
\[
\begin{array}{ll}
\mbox{i)}&x^{2}y''+2xy'-6y=0\\
\mbox{ii)}&x^{2}y''-xy'+y=2x
\end{array}
\]
\end{enumerate}
\eej

\bej Vibraciones en sistemas mec\'anicos.



Una carreta de masa $M$ est\'a sujeta a una pared por medio de un resorte,
que no ejerce fuerza cuando la carreta est\'a en la posici\'on de
equilibrio $x=0$. Si la carreta se desplaza a una distancia $x$, el
resorte ejerce una fuerza de restauraci\'on igual a $-\kappa x$, donde
$\kappa $ es una constante positiva que mide la rigidez del resorte.
Por la segunda ley del movimiento de Newton, se tiene que:
\begin{equation}
 M \frac{d^{2}x}{dt^{2}} = -\kappa x \ \mbox{ o bien} \ x''+a^{2}x=0,
\ \ a=\sqrt{\kappa /M}
\label{resortelibre}
\end{equation}

\begin{enumerate}
\item Si la carreta se lleva a la posición $x=x_{0}$ y se libera sin velocidad
inicial en el instante $t=0$, hallar la función $x(t)$. Verificar
que se trata de una función peri\'odica. Calcular su período $\tau$, y
su frecuencia $f= 1/\tau$ (la cantidad de ciclos por unidad de tiempo).
Verificar que la frecuencia de vibraci\'on aumenta al aumentar la rigidez
del resorte, o al reducir la masa de la carreta (como dice el sentido
com\'un) y que la amplitud de esta oscilaci\'on es $x_{0}$.

\medskip

Si se produce una amortiguaci\'on que se opone al movimiento, y de magnitud
proporcional a la velocidad ($=-c \frac{dx}{dt}$) debida al rozamiento,
la ecuaci\'on (\ref{resortelibre}) que describe el movimiento de la carreta
en funci\'on del tiempo se convierte en:
\[ M \frac{d^{2}x}{dt^{2}} + c \frac{dx}{dt} + \kappa x =0\]
o bien:
\begin{eqnarray}
\frac{d^{2}x}{dt^{2}}+2b\frac{dx}{dt}+a^{2}x=0,  \label{resortefrenado}\\
b=\frac{c}{2M},\ \ \ a=\sqrt{\frac{\kappa}{M}}. \nonumber
\end{eqnarray}

\item Si $b>a$ (la fuerza de fricci\'on debida al rozaminto es grande en
comparaci\'on con la rigidez del resorte), encontrar la soluci\'on de
(\ref{resortefrenado}) que verifique como antes $x(0)=x_{0},\ x'(0)=0 $.
Probar que no hay ninguna vibraci\'on y que la carreta vuelve simplemente
a su posici\'on de equilibrio. Se dice que el movimiento est\'a
sobreamortiguado.


\item Si $b=a$, ver que tampoco hay vibraci\'on y que el comportamiento es similar
al del caso anterior. Se dice que el movimiento es críticamente amortiguado.

\item Si ahora $b<a$ (caso subamortiguado), probar que la soluci\'on de
(\ref{resortefrenado}) con las condiciones iniciales $x(0)=x_{0},\ x'(0)=0 $
es:
\begin{equation}
 x(t)= x_{0} \frac{\sqrt{\alpha ^{2} + b^{2}}}{\alpha} \mbox{e}^{-bt}
\cos (\alpha t - \theta)
\end{equation}
donde $\alpha = \sqrt{a^{2}-b^{2}}$, y $\tan\theta= b/\alpha$.

Esta funci\'on oscila con una amplitud que se reduce exponencialmente.
Su gr\'afica cruza la posici\'on de equilibrio $x=0$ a intervalos regulares,
aunque no es peri\'odica. Hacer un dibujo. Probar que el tiempo requerido
para volver a la posici\'on de equilibrio es:
\[ T=\frac{2 \pi}{\sqrt{\frac{\kappa}{M}-\frac{c^{2}}{4 M^{2}}}} \]
y su ``frecuencia'' est\'a dada por $f=1/T$, llamada {\it frecuencia natural del
sistema}. Notar que esta frecuencia disminuye al disminuir la constante
de amortiguaci\'on $c$.


\medskip

Hasta ahora hemos considerado vibraciones libres, porque s\'olo act\'uan
fuerzas internas al sistema. Si una fuerza $F(t)$ act\'ua sobre la carreta,
la ecuaci\'on ser\'a:
\begin{equation}
M \frac{d^{2}x}{dt^{2}} + c \frac{dx}{dt} + \kappa x =F(t)
\label{resorteforzado}
\end{equation}

\item Si esta fuerza es peri\'odica de la forma $F(t)=F_{0} \cos(\omega t)$, con
$F_{0}, \omega$ constantes, hallar $x(t)$. Al valor $\omega /2\pi$ se lo
llama frecuencia impresa \typeout{que feo esto de impresa} al sistema.

 Si $\tan(\phi) = \frac{\omega c}{\kappa - \omega^{2}M}$, probar que la
soluci\'on general de (\ref{resorteforzado}), con  $F(t)=F_{0} \cos(\omega t)$
puede escribirse:
\begin{equation}
x(t)= \mbox{e}^{-bt} (C_{1} \cos (\alpha t) + C_{2} \mbox{sen} (\alpha t))
+ \frac{F_{0}}{\sqrt{(\kappa-\omega^{2}M)^{2}+\omega^{2} c^{2}}} \cos( \omega
t - \phi)
\end{equation}
El primer término tiende a cero para $t \to +\infty$, luego es
``transitorio'', es decir, a medida que pasa el tiempo, la soluci\'on se
parece m\'as y m\'as al segundo sumando. Notar que la frecuencia de esta
funci\'on es la frecuencia impresa al sistema, y que la amplitud es el
coeficiente
$\displaystyle \frac{F_{0}}{\sqrt{(\kappa-\omega^{2}M)^{2}+\omega^{2} c^{2}}}$.
 ?`Qu\'e pasa cuando la frecuencia impresa se
acerca a la frecuencia natural del sistema? (Este fen\'omeno se conoce
con el nombre de resonancia).

\item Si $b<a$ (caso subamortiguado) hallar la frecuencia impresa $\omega$ que
provoca amplitud m\'axima. ?`Siempre existe este valor? Este valor de
frecuencia impresa  (cuando existe) se denomina frecuencia de resonancia.
Demostrar que la frecuencia de resonancia es siempre menor que la frecuencia
natural.
\end{enumerate}
\eej

\bej Hallar la soluci\'on general de las siguientes ecuaciones, empleando la
soluci\'on dada:
\[
\begin{array}{llll}
\mbox{i)}&xy''+2y'+xy=0, & I=\R_{>0}, & y_{1}(x)=\frac{\mbox{sen}x}{x}.\\
\mbox{ii)}&xy''-y'-4x^{3}y=0, & I=\R_{>0}, & y_{1}(x)=\exp (x^{2}).\\
\mbox{iii)}&xy''-y'-4x^{3}y=0, & I=\R_{<0}, & y_{1}(x)=\exp (x^{2}).\\
\mbox{iv)}&(1-x^{2})y''-2xy'+2y=0, & I=(-\infty,-1), (-1,1), (1, \infty),
& y_{1}(x)=x.
\end{array}
\]
El último ítem es un caso especial de la ecuaci\'on $(1-x^{2})y''
-2xy'+p(p+1)y=0$ (ecuaci\'on de Legendre), correspondiente al caso $p=1$,
 en los intevalos en que la
ecuaci\'on es normal.
\eej

\bej Sabiendo que $y_{1}(x)=e^{x^2}$ es
soluci\'on de la ecuaci\'on homog\'enea asociada, hallar todas las
soluciones de $xy''-y'-4x^3y=x^3$.

\eej

\bej Probar que las funciones
\[
\begin{array}{cc}
\begin{array}{ll}
\phi _{1}(t)= &\left\{ \begin{array}{ll}
t^{2}&t \le 0\\
0&t\geq 0
\end{array}
\right.
\end{array}
&
\begin{array}{ll}
\quad \text{ y }\quad \phi _{2}(t)= &\left\{ \begin{array}{ll}
0&t \le 0\\
t^{2}&t\geq 0
\end{array}
\right.
\end{array}
\end{array}
\]
son linealmente independientes en $\R$ pero que $W(\phi_{1}, \phi_{2})(0)=0$.
¿Existe algún sistema lineal normal de orden 2 definido en algún intervalo
$(-\epsilon, \epsilon)$ que admita a $\{ \phi_{1}, \phi_{2}\}$ como base de
soluciones?
\eej


\bej Hallar la soluci\'on general de los siguientes sistemas
$$
\hskip-.2cm\text{(a)}\: \left\{\ba{l} x_1'=-x_2 \\x_2'=2x_1+3x_2 \ea \right.\qquad\qquad\qquad
\hskip.3cm\text{(b)}\: \left\{\ba{l} x_1'=-8x_1-5x_2\\x_2'=10x_1+7x_2\ea\right.
$$
$$
\hskip1cm\text{(c)}\: \left\{\ba{l}  x_1'=-4x_1+3x_2 \\x_2'=-2x_1+x_2\ea\right.\qquad\hskip1.5cm
\text{(d)}\:\left\{\ba{l} x_1'=-x_1+3x_2-3x_3\\   x_2'=-2x_1+x_2\\ x_3'=-2x_1+3x_2-2x_3
\ea\right.
$$

En cada caso, hallar el conjunto de datos iniciales tales que la
solución correspondiente tienda a 0 cuando $t$ tienda a
$+\infty$. Ídem con $t$ tendiendo a $-\infty$.

\eej

\bej Dos tanques, conectados mediante tubos, contienen cada uno 24 litros de una soluci\'on salina. Al tanque I entra agua pura a raz\'on de 6 litros por minuto y del tanque II sale, al exterior, el agua que contiene a raz\'on de 6 litros por minuto. Adem\'as el l\'iquido se bombea del tanque I al tanque II a
una velocidad de 8 litros por minuto y del tanque II al tanque I a una
velocidad de 2 litros por minuto. Se supone que los tanques se agitan de igual forma constantemente de manera tal que la mezcla sea homog\'enea. Si en un principio hay $x_0$ kg de sal en el tanque I e $y_0$ Kg de sal en el tanque II, determinar la cantidad de sal en cada tanque a tiempo $t>0$. Cu\'al es el l\'imite, cuando $t\to +\infty$, de las respectivas concentraciones de sal en cada tanque.?
\eej

%\bej Inicialmente el tanque I contiene 100 litros de agua salada a una
%concentraci\'on de 1 kg por litro y el tanque II tiene 100 litros
%de agua pura. El l\ii quido se bombea del tanque I al tanque II a
%una velocidad de 1 litro por minuto, y del tanque II al I a una
%velocidad de 2 litros por minuto. Los tanques se agitan
%constantemente. \textquestiondown Cu\'al es la concentraci\'on en el tanque I
%despu\'es de 10 minutos?
%\eej

\bej Hallar la soluci\'on general de los siguientes sistemas

$$\text{(a)}\; \left\{\ba{l}x_1'=x_1-x_2\\ x_2'=x_1+x_2  \ea \right. \qquad
\ \ \text{(b)}\; \left\{\ba{l}x_1'=2x_1-x_2\\ x_2'=4x_1+2x_2  \ea \right. \qquad
$$
$$\ \ \ \text{(c)}\; \left\{\ba{l}x_1'=2x_1+x_2   \\ x_2'=2x_2    \ea \right. \qquad
\text{(d)}\; \left\{\ba{l}x_1'=-5x_1+9x_2\\ x_2'=-4x_1+7x_2  \ea \right. \qquad
$$

\eej

\bej Hallar la soluci\'on general de los siguientes sistemas
$$
\text{(a)}\: \left\{\ba{l}x_1'=-x_2+2\\ x_2'=2x_1+3x_2+t \ea\right.\qquad
\text{(b)} \:\left\{\ba{l} x_1'=2x_1-x_2+e^{2t}\\ x_2'=4x_1+2x_2+4 \ea\right.
$$
\eej

\end{document}
