\documentclass[11pt,a4paper,pdftex]{amsart}
\usepackage[psamsfonts]{amssymb}
\usepackage{amsmath,amsfonts,latexsym}
\usepackage[dvips]{graphicx}
\usepackage[utf8x]{inputenc}
\usepackage[spanish]{babel}
\usepackage{epsfig}
\usepackage{amscd}
\usepackage{verbatim}
\usepackage{multicol}

\newtheorem{teo}{Teorema}[section]%establece un contador para
%el entorno 'teo' que aparecer� con el nombre 'Teorema' y que
%volver� a empezar cuando cambien de 'chapter'.

\newtheorem{coro}[teo]{Corolario}%vincula las numeraciones de 'coro'
%con las de 'teo'.

\newtheorem{lema}{Lema}[section]
%\newtheorem{lema}[lema]{Lema}


\newtheorem{definition}{Definici\'on}[section]

\newtheorem{conc}{Conclusión}[section]

\newtheorem{prop}{Proposición}[section]
%\newtheorem{prop}[prop]{Propiedad}

\newtheorem{obs}{Observación}[section]

\renewcommand{\thesection}{{}}

%\newtheorem{obs}{Observaci�n} %numera las 'Observaciones' de corrido
%sin volver a resetear.

\newtheorem{ax}{Axioma}[section]

%\newtheorem{defini}{Definici�n}[section]

\newtheorem{ej}{Ejercicio}%[section] %numera os 'Ejercicios' reseteando
%en cada 'cap�tulo'.


\numberwithin{equation}{section}%numera las formulas reseteando
%cada vez que cambia de 'capítulo'.

\newcommand{\bej}[1]{\begin{ej}\rm{#1}}
\newcommand{\eej}{\end{ej}\vspace{-0.2cm}}

%---------------------------------------
\newcommand{\be}{\begin{enumerate}}
\newcommand{\ee}{\end{enumerate}}
\newcommand{\bit}{\begin{itemize}}
\newcommand{\eit}{\end{itemize}}
\newcommand{\bc}{\begin{center}}
\newcommand{\ec}{\end{center}}
\newcommand{\ba}{\begin{array}}
\newcommand{\ea}{\end{array}}
\newcommand{\bq}{\begin{quotation}}
\newcommand{\eq}{\end{quotation}}
\newcommand{\mc}[1]{\mathcal{#1}}
\newcommand{\mb}[1]{\;\mbox{#1}\;}
\newcommand{\su}[1]{\underline{#1}}
\newcommand{\so}[1]{\overline{#1}}
\newcommand{\ang}[1]{\widehat{#1}}
\newcommand{\arc}[1]{\wideparen{#1}}
\newcommand{\Cc}{QQ\;}
\renewcommand{\bf}{\textbf}
\newcommand{\Comb}[2]{\left(\!\!\!\ba{c}#1\\[1ex]#2 \ea
\!\!\!\right)}
\newcommand{\di}{\displaystyle}
%-----------------------------------------
%conjuntos
\newcommand{\W}{\mathbb W}
\newcommand{\K}{\mathbb K}
\newcommand{\N}{\mathbb N}
\newcommand{\C}{\mathcal C}
\newcommand{\Su}{\mathcal S}
\newcommand{\Z}{\mathbb Z}
\newcommand{\Q}{\mathbb Q}
\newcommand{\R}{\mathbb R}
\newcommand{\F}{\mathbb F}
\newcommand{\A}{\mathbb A}
\newcommand{\V}{\mathbb V}
\newcommand{\I}{\mathbb I}
\newcommand{\0}{\mathbb O}
%------------------------------------------
%operaciones
\newcommand{\8}{\infty}
\newcommand{\ie}{\langle}
\newcommand{\de}{\rangle}
\newcommand{\pe}[2]{\ie {#1} \, ; \, {#2} \de }
\newcommand{\f}[1]{\overrightarrow{#1}}
\newcommand{\DT}[2]{\frac{d{#1}}{d{#2}}}
\newcommand{\ds}{\displaystyle}
\newcommand{\dg}{\Delta}
\newcommand{\g}{\nabla}
\newcommand{\D}{\mbox{div}}
\newcommand{\Dp}[1]{\partial_{#1}}
\newcommand{\DP}[2]{\frac{\partial{#1}}{\partial{#2}}}
\newcommand{\sen}[1]{\mbox{sen}\;{#1}}
\newcommand{\Lp}[1]{L^{#1}}
\newcommand{\To}{\longrightarrow}
%-------------------------------------
%griegos
\newcommand{\vi}{\varphi}
\newcommand{\om}{\omega}
\newcommand{\Om}{\Omega}
\newcommand{\ve}{\varepsilon}
%-------------------------
%comandos particulares
\newcommand{\id}[1]{\text{id}_{#1}}
\newcommand{\mD}{\mc{D}}
\newcommand{\mC}{\mc{S}}
\newcommand{\mS}{\mc{SS}}
\newcommand{\mH}{\mc{H}}
\newcommand{\HD}{\mc{H}_\mD}
%%%%%%%%%%%%%%%%%%%%%%%%%%%%%%%%%%%%%%%%%%%%%%%%%%%%%%%%%%%%%%%%%
%Un comando para insertar dibujos
\newcommand{\putfig}[4]{\bigskip \bigskip
             \begin{figure}[ht]
             \epsfxsize=#1cm\hfil{\epsfbox{#2}}
             \caption{#3}
         \label{#4}
             \end{figure}\bigskip}
%La sintaxis
%\putfig{ancho}{.eps}{titulo}{label}
%%%%%%%%%%%%%%%%%%%%%%%%%%%%%%%%%%%%%%%%%%%%%%%%%%%%%%%
%Un comando para insertar dos dibujos
\newcommand{\putfigg}[4]{\bigskip \bigskip
             \begin{figure}[ht]
             \epsfxsize=5cm\hfil{\epsfbox{#1}}
             \epsfxsize=5cm\hfil{\epsfbox{#2}}
             \caption{#3}
         \label{#4}
             \end{figure}\bigskip}
%La sintaxis
%\putfigg{.eps}{.eps}{titulo1}{label1}
%%%%%%%%%%%%%%%%%%%%%%%%%%%%%%
%un dibujo en jpg
\newcommand{\putjpg}[3]{\bigskip
\begin{center}
 \begin{figure}[ht]
  \hspace{5cm}\includegraphics[scale=1.7]{#1.jpg}
  \caption{#2}
  \label{#3}
  \bigskip
  \bigskip
 \end{figure}
\end{center}
}
%la sintaxis \putjpg{.jpg}{Titulo}{label}
%----------------------------------------------------
%diseñoo de la página

\vfuzz7pt % Don't report over-full v-boxes if over-edge is 7pt small
\hfuzz7pt % Don't report over-full h-boxes if over-edge is 7pt small

\pagestyle{myheadings}

\renewcommand{\labelenumi}{({\it \alph{enumi}})}
\renewcommand{\labelenumii}{\arabic{enumii})}

\flushbottom \topmargin-0.5cm \textwidth17cm \textheight24.5cm \hoffset=-2cm
\voffset=-0.5cm


\begin{document}



%----------------------------------------------------------
%Encabezado:

\centerline{{\small Universidad CAECE}}

 \vskip 0.2cm
 \hrulefill
 \vskip 0.2cm

 \centerline{{\bf{\Large{\sc Ecuaciones Diferenciales Ordinarias}}}}
 \vskip 0.2cm
 \centerline{\ttfamily Segundo Cuatrimestre 2018}
 \hrulefill

 \medskip
 \centerline{\bf {Práctica 1: Ecuaciones Diferenciales de Primer Orden.}}
 \medskip


\bej Para cada una de las ecuaciones diferenciales que siguen, encontrar la 
solución general y la soluci\'on  particular que satisfaga la condici\'on 
dada. En todos los casos hallar el intervalo maximal de existencia de las 
soluciones y decir si cada solución es única en ese intervalo. 
\[
\begin{array}{llll}
\mbox{a)}&x'-2tx=t,\ \, \quad x(1)=0,\qquad
\ \, \mbox{b)}&x'=\di\frac{1+x^2}{1+t^2},\quad x(1)=0,\\
\\
\mbox{c)}&x'=\di
\frac{1+x}{1+t}, \qquad x(0)=1,\qquad
\,\mbox{d)}&x'=\di\frac{1+x}{1-t^2}, \quad x(0)=1, \\
\\
\mbox{e)} &x'-x^{1/3}=0,\ \quad x(0)=0, \qquad
\mbox{f)}&x'=\di\frac{1+x}{1+t}, \quad x(0)=-1.
\end{array}
\]

\eej

\bigskip
\bej Si $y=y(t)$ denota el n\'umero de habitantes de una poblaci\'on en 
funci\'on del tiempo, se denomina tasa de cre\-ci\-miento de la poblaci\'on a 
la funci\'on $y'/y$.

\begin{enumerate}
\item[(a)] Dibujar el gr\'afico de $y(t)$ para poblaciones con tasa de
crecimiento cons\-tante, positiva y negativa. ¿Cu\'ales son las poblaciones 
con tasa de crecimiento nula?

\item[(b)] Una población tiene tasa de crecimiento constante. El 1 de
enero de 2002 tenía 1000 individuos, y cuatro meses despu\'es
tenía 1020. Estimar el número de individuos que tendr\'a el
1 de enero del año 2022, usando los resultados anteriores.

\item[(e)] Caracterizar las poblaciones con tasa de crecimiento $at + b$ para
$a,b \in \R$.

\item[(f)] Caracterizar las poblaciones cuya tasa de crecimiento es
igual a $r-cy$, donde $r$ y $c$ son constantes positivas. Hallar posiciones
de equilibrio y analizar la evolución de la población a largo plazo.
\end{enumerate}
\eej

% \bigskip


% \bej Si un cultivo de bacterias crece con un coeficiente de
% variación proporcional a la cantidad existente y se sabe
% además que la población se duplica en 1 hora ¿Cuánto
% habrá aumentado en 2 horas?.
% \eej

\bigskip

\bej Verifique que las siguientes ecuaciones son homog\'eneas de
grado cero y
resuelva:
\[
\mbox{(a)}\ tx'=x+2t \exp(-x/t)\qquad \mbox{(b)}\ txx'=2x^2-t^{2}\qquad
\mbox{(c)}\ x'=\di\frac{x+t}{t},\ \ x(1)=0
\]
\eej

\bigskip

\bej Demuestre que la sustituci\'on $y=at+bx+c$ cambia
$x'=f(at+bx+c)$ en una ecuaci\'on con variables separables y
aplique este m\'etodo para resolver las ecuaciones siguientes:
$$ \mbox{(a)}\ x'=(x+t)^{2} \qquad\qquad
 \mbox{(b)}\ x'=\mbox{sen}^{2}(t-x+1) $$
\eej


\bej Supongamos que $ae \ne bd$. Hallar constantes $h, k$ de modo que las 
sustituciones $t=s-h$, $x=y -k$ reduzcan la ecuaci\'on
\[ \frac{dx}{dt}= F \left( \frac{at+bx+c}{dt+ex+f} \right) \]
a una ecuaci\'on homog\'enea. Usando esta idea resuelva las siguientes 
ecuaciones.
 \[
 \begin{aligned}&\qquad\qquad\mbox{a) } x'=\frac{2x-t+4}{x+t-1}
 \qquad \qquad\mbox{b) }x'=\frac{x+t+4}{t-x-6}
 &\qquad\qquad \mbox{c) }x'=\frac{x+t+4}{x+t-6},\quad{x(0)=2}.
 \end{aligned}
 \]
\eej

\bigskip

% \bej Resuelva las siguientes ecuaciones:
% \[
% \begin{array}{ll}
% \mbox{a)} \ (y-x^{3})dx+(x+y^{3})dy =0
% &\hskip-1.2cm\mbox{b)} \ \cos x \cos^{2} y\, dx - 2 \sen x \,\sen y
% \cos y\, dy=0 \\
% \ \\
% \mbox{c)} \ (3x^{2}-y^{2})\,dy -2xy\, dx =0
% &\hskip-1.2cm\mbox{d)} \ x\, dy= (x^{5}+x^{3}y^{2} +y)\,dx   \\
% \ \\
% \text{(e)}\:2(x+y)\sen y\,dx+\big(2(x+y)\sen y+\cos y\big)\,dy=0
% &\hskip-1.2cm\mbox{f)} \ 3y \,dx+x\, dy=0
% \\
% \ \\
% \text{(g)}\:  \big(1-y(x+y)\text{tan}\,(xy)\big)\,dx+\big(1-x(x+y)\text{tan}\,(xy)\big)\,dy=0.
% \end{array}
% \]
% \eej

% \bigskip

\bej Considere la ecuación lineal de primer orden
$x'+p(t)\,x=q(t)$.
\begin{enumerate}
\item Busque una función $\mu(t)$ tal que
$$
\mu(t)\big(x'(t)+p(t)\,x(t)\big)=\big(\mu(t)\,x(t)\big)'.
$$

\item Multiplique la ecuación por $\mu$ y halle su solución 
general. La función $\mu$ se denomina {\it factor integrante}.
\end{enumerate}

\eej

\bigskip

\bej 
Algunas aplicaciones a la geometría.
\begin{enumerate}
\item Hallar la ecuaci\'on de una curva tal que la pendiente de la
recta tangente en un punto cualquiera es la mitad de la
pendiente de la recta que une el punto con el origen.

\item Hallar la ecuaci\'on de las curvas con la siguiente propiedad: si $P$ es 
un punto de la curva, $L$ la recta tangente a la curva en $P$, entonces la 
única recta que pasa por $P$ y es perpendicular a $L'$ también pasa por el 
origen.

\item Tenemos una curva con la siguiente propiedad: si $(x,y)$ es un punto de 
la curva y $L$ es la tangente entonces la pendiente de $L$ es $a x$, donde $a$
es una constante. Probar que la curva está contenida en una parábola. 
\end{enumerate}
\eej


% \bigskip

% \bej Hallar la ecuación de una curva del primer cuadrante tal que para cada punto
% $(x_0,y_0)$ de
% la misma, la ordenada al origen de la recta tangente a la curva en $(x_0,y_0)$ sea
% $2(x_0+y_0)$.
% \eej

%  \ \ \newpage
 \bej\ \ \newline

\begin{enumerate}
\item Hallar las soluciones de:
\[
\begin{array}{ll}
\mbox{i)}&y'+y= \sen(x), \\
\mbox{ii)}&y'+y=3 \cos(2x).
\end{array}
\]

\item Halle las soluciones de $ y'+y=\sen(x) +3 \cos(2x)  $
cuya gr\'afica pase por el origen (Piense, y no haga cuentas de m\'as).
\end{enumerate}
\eej
\bigskip

% \bej Sea la ecuaci\'on no homog\'enea $y'+a(x)y=b(x)$ donde
% $a,b:\mathbb{R}\to \R$ son continuas con período $p>0$ y $b
% \not \equiv 0$:
% \begin{enumerate}
% \item Pruebe que una soluci\'on $\Phi$ de esta ecuaci\'on verifica:
% \[ \Phi (x+p) = \Phi (x),\, \forall x\in\R\
% \Leftrightarrow \ \ \Phi(0)= \Phi (p). \]
% \item Encuentre las soluciones de período $2 \pi$ para las ecuaciones:
% \[ y'+3y= \cos(x), \qquad  y'+\cos (x) y = \mbox{sen} (2x). \]
% \end{enumerate}
% \eej
% \bigskip

\bej Suponga que el ritmo al que se enfría un cuerpo caliente
es proporcional a la diferencia de temperatura entre \'el y el
ambiente que lo rodea (ley de enfriamiento de Newton). Un cuerpo
se calienta 110 $^{\circ}$C y se expone al aire libre a una
temperatura de 10 $\mbox{}^{\circ }$C. Al cabo de una hora su
temperatura es de 60 $^{\circ}$C. ?`Cu\'anto tiempo adicional
debe transcurrir para que se enfríe a 30 $^{\circ}$C?
\eej
\bigskip

\bej Se sabe que el Carbono 14 tiene una semivida de 5600 años, es decir que 
su cantidad se reduce a la mitad por desintegración radioactiva en ese lapso 
de tiempo. Un sedimento se formó a partir de material con una concentración 
del 40\% de Carbono 14, pero hoy en día solo tiene una concentración del 2\%. 
Suponiendo que la tasa de cambio del Carbono 14 $\dot x/x$ es constante a lo 
largo del tiempo calcular cuánto tiempo pasó desde que se depositaron los 
sedimentos.

Observación: La suposición sobre la tasa de cambio no es realista, pero las 
estimaciones obtenidas con este método pueden corregirse para datar restos 
arqueológicos o paleontológicos con alta precisión.
\eej

% \bigskip

% \bej Si la resistencia del aire que act\'ua sobre un cuerpo de
% masa $m$ en caída libre ejerce una fuerza retardadora sobre
% el mismo proporcional a la velocidad ($=-k
% v$), la ecuaci\'on diferencial del movimiento es:
% \[ \frac{d^{2}y}{dt^{2}}=g -c \frac{dy}{dt}, \mbox{ o bien }
% \frac{dv}{dt}= g - cv \]
% donde $c=k/m$. Supongamos $v=0$ en el instante $t=0$, y $c>0$. Encontrar
% $\lim _{t \rightarrow \infty} v(t)$ (llamada velocidad terminal).

% Si la fuerza retardadora es proporcional al cuadrado de la
% velocidad, la ecuaci\'on se convierte en:
% \[ \frac{dv}{dt}=g-c v^{2}. \]
% Si $v(0)=0$, encuentre la velocidad terminal en este caso.
% \eej
% \bigskip

% \bej La ecuación $y'+P(x) y =Q(x) y^{n}$, que se conoce como la
% ecuación de Bernoulli, es lineal cuando $n=0,1$. Demuestre que
% se puede reducir a una ecuación lineal para cualquier valor de
% $n \ne 1$ por el cambio de variable $z=y^{1-n}$, y aplique este
% método para resolver las ecuaciones siguientes:
% \[
% \begin{array}{ll}
% \mbox{(a)}&xy'+y=x^{4}y^{3},\\
% \mbox{(b)}&xy^{2}y'+y^{3}=x \cos x,\\
% \mbox{(c)}&x y'-3y=x^{4}.
% \end{array}
% \]
% \eej

\end{document}
