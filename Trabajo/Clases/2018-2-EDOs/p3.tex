\documentclass[11pt,a4paper,pdftex]{amsart}
\usepackage[psamsfonts]{amssymb}
\usepackage{amsmath,amsfonts,latexsym}
\usepackage[dvips]{graphicx}
\usepackage[utf8x]{inputenc}
\usepackage[spanish]{babel}
\usepackage{epsfig}
\usepackage{amscd}
\usepackage{verbatim}
\usepackage{multicol}

\newtheorem{teo}{Teorema}[section]%establece un contador para
%el entorno 'teo' que aparecer� con el nombre 'Teorema' y que
%volver� a empezar cuando cambien de 'chapter'.

\newtheorem{coro}[teo]{Corolario}%vincula las numeraciones de 'coro'
%con las de 'teo'.

\newtheorem{lema}{Lema}[section]
%\newtheorem{lema}[lema]{Lema}


\newtheorem{definition}{Definici\'on}[section]

\newtheorem{conc}{Conclusión}[section]

\newtheorem{prop}{Proposición}[section]
%\newtheorem{prop}[prop]{Propiedad}

\newtheorem{obs}{Observación}[section]

\renewcommand{\thesection}{{}}

%\newtheorem{obs}{Observaci�n} %numera las 'Observaciones' de corrido
%sin volver a resetear.

\newtheorem{ax}{Axioma}[section]

%\newtheorem{defini}{Definici�n}[section]

\newtheorem{ej}{Ejercicio}%[section] %numera os 'Ejercicios' reseteando
%en cada 'cap�tulo'.


\numberwithin{equation}{section}%numera las formulas reseteando
%cada vez que cambia de 'capítulo'.

\newcommand{\bej}[1]{\begin{ej}\rm{#1}}
\newcommand{\eej}{\end{ej}\vspace{-0.2cm}}

%---------------------------------------
\newcommand{\be}{\begin{enumerate}}
\newcommand{\ee}{\end{enumerate}}
\newcommand{\bit}{\begin{itemize}}
\newcommand{\eit}{\end{itemize}}
\newcommand{\bc}{\begin{center}}
\newcommand{\ec}{\end{center}}
\newcommand{\ba}{\begin{array}}
\newcommand{\ea}{\end{array}}
\newcommand{\bq}{\begin{quotation}}
\newcommand{\eq}{\end{quotation}}
\newcommand{\mc}[1]{\mathcal{#1}}
\newcommand{\mb}[1]{\;\mbox{#1}\;}
\newcommand{\su}[1]{\underline{#1}}
\newcommand{\so}[1]{\overline{#1}}
\newcommand{\ang}[1]{\widehat{#1}}
\newcommand{\arc}[1]{\wideparen{#1}}
\newcommand{\Cc}{QQ\;}
\renewcommand{\bf}{\textbf}
\newcommand{\Comb}[2]{\left(\!\!\!\ba{c}#1\\[1ex]#2 \ea
\!\!\!\right)}
\newcommand{\di}{\displaystyle}
%-----------------------------------------
%conjuntos
\newcommand{\W}{\mathbb W}
\newcommand{\K}{\mathbb K}
\newcommand{\N}{\mathbb N}
\newcommand{\C}{\mathcal C}
\newcommand{\Su}{\mathcal S}
\newcommand{\Z}{\mathbb Z}
\newcommand{\Q}{\mathbb Q}
\newcommand{\R}{\mathbb R}
\newcommand{\F}{\mathbb F}
\newcommand{\A}{\mathbb A}
\newcommand{\V}{\mathbb V}
\newcommand{\I}{\mathbb I}
\newcommand{\0}{\mathbb O}
%------------------------------------------
%operaciones
\newcommand{\8}{\infty}
\newcommand{\ie}{\langle}
\newcommand{\de}{\rangle}
\newcommand{\pe}[2]{\ie {#1} \, ; \, {#2} \de }
\newcommand{\f}[1]{\overrightarrow{#1}}
\newcommand{\DT}[2]{\frac{d{#1}}{d{#2}}}
\newcommand{\ds}{\displaystyle}
\newcommand{\dg}{\Delta}
\newcommand{\g}{\nabla}
\newcommand{\D}{\mbox{div}}
\newcommand{\Dp}[1]{\partial_{#1}}
\newcommand{\DP}[2]{\frac{\partial{#1}}{\partial{#2}}}
\newcommand{\sen}[1]{\mbox{sen}\;{#1}}
\newcommand{\Lp}[1]{L^{#1}}
\newcommand{\To}{\longrightarrow}
%-------------------------------------
%griegos
\newcommand{\vi}{\varphi}
\newcommand{\om}{\omega}
\newcommand{\Om}{\Omega}
\newcommand{\ve}{\varepsilon}
%-------------------------
%comandos particulares
\newcommand{\id}[1]{\text{id}_{#1}}
\newcommand{\mD}{\mc{D}}
\newcommand{\mC}{\mc{S}}
\newcommand{\mS}{\mc{SS}}
\newcommand{\mH}{\mc{H}}
\newcommand{\HD}{\mc{H}_\mD}
%%%%%%%%%%%%%%%%%%%%%%%%%%%%%%%%%%%%%%%%%%%%%%%%%%%%%%%%%%%%%%%%%
%Un comando para insertar dibujos
\newcommand{\putfig}[4]{\bigskip \bigskip
             \begin{figure}[ht]
             \epsfxsize=#1cm\hfil{\epsfbox{#2}}
             \caption{#3}
         \label{#4}
             \end{figure}\bigskip}
%La sintaxis
%\putfig{ancho}{.eps}{titulo}{label}
%%%%%%%%%%%%%%%%%%%%%%%%%%%%%%%%%%%%%%%%%%%%%%%%%%%%%%%
%Un comando para insertar dos dibujos
\newcommand{\putfigg}[4]{\bigskip \bigskip
             \begin{figure}[ht]
             \epsfxsize=5cm\hfil{\epsfbox{#1}}
             \epsfxsize=5cm\hfil{\epsfbox{#2}}
             \caption{#3}
         \label{#4}
             \end{figure}\bigskip}
%La sintaxis
%\putfigg{.eps}{.eps}{titulo1}{label1}
%%%%%%%%%%%%%%%%%%%%%%%%%%%%%%
%un dibujo en jpg
\newcommand{\putjpg}[3]{\bigskip
\begin{center}
 \begin{figure}[ht]
  \hspace{5cm}\includegraphics[scale=1.7]{#1.jpg}
  \caption{#2}
  \label{#3}
  \bigskip
  \bigskip
 \end{figure}
\end{center}
}
%la sintaxis \putjpg{.jpg}{Titulo}{label}
%----------------------------------------------------
%diseñoo de la página

\vfuzz7pt % Don't report over-full v-boxes if over-edge is 7pt small
\hfuzz7pt % Don't report over-full h-boxes if over-edge is 7pt small

\pagestyle{myheadings}

\renewcommand{\labelenumi}{({\it \alph{enumi}})}
\renewcommand{\labelenumii}{\arabic{enumii})}

\flushbottom \topmargin-0.5cm \textwidth17cm \textheight24.5cm \hoffset=-2cm
\voffset=-0.5cm


\begin{document}



%----------------------------------------------------------
%Encabezado:

\centerline{{\small Universidad de Buenos Aires - Facultad de Ciencias Exactas y Naturales - Depto. de Matemática}}

 \vskip 0.2cm
 \hrulefill
 \vskip 0.2cm

 \centerline{{\bf{\Large{\sc Análisis II - Análisis Matemático II - Matemática 3}}}}
 \vskip 0.2cm
 \centerline{\ttfamily Primer Cuatrimestre 2018}
 \hrulefill

 \medskip
 \centerline{\bf {Práctica 7: Diagramas de fase.}}
 \medskip

\setcounter{equation}{0}

\bej
{\emph {Din\'amica unidimensional}.} En cada una de las siguientes ecuaciones de la forma $\dot{x}=f(x),$ realizar el gráfico de $f(x),$ hallar los puntos de equilibrio y realizar un bosquejo de la din\'amica en el eje $x.$ A partir de esto analizar la estabilidad de los puntos de equilibrio.
$$
\text{(a)}\; \dot{x}=1-x^2, \hspace{1cm} \text{(b)}\; \dot{x}=x^3-x, \hspace{1cm} \text{(c)}\; \dot{x}=\sen(x).
$$
\eej

\bej Dibujar los campos vectoriales siguientes y tratar de deducir cuáles son las correspondientes
líneas de flujo:
\[%
\begin{array}
[c]{clcccl}%
\text{(a)} & F(x,y)=(x,y), &  &  & \text{(b)} & F(x,y)=(-y,x),\\
&  &  &  &  & \\
\text{(c)} & F(x,y)=(y,0), &  &  & \text{(d)} & F(x,y)=(-x+2y,-2x-y).
\end{array}
\]%
\eej

%\bej Esbozar el plano de fases de la ecuaci\'on $x'=Ax$, sin resolver
%para,
%$$\mbox{(a)}\ A = \left(\begin{array}{rr}
%            3 & 0 \\
%            0 & 3
%        \end{array}\right)\quad
%\mbox{(b)}\ A = \left(\begin{array}{rr}
%            1/2 & -2 \\
%            2 & 0
%        \end{array}\right)\quad
%\mbox{(c)}\ A = \left(\begin{array}{rr}
%            0 & 0 \\
%            -2 & 0
%        \end{array}\right).$$
%\eej

\bej Considerar el sistema a un par\'ametro,
$$\begin{aligned}
\dot{x}=2x, &\\
\dot{y}=\lambda y, &\ \ \ \ \ \mbox{ con } \lambda \in \mathbb{R}.
\end{aligned}$$
Determinar todas las soluciones y hacer un bosquejo del diagrama
de fases para $\lambda=-1,0,1,2$.
\eej

\bej Sea $A$ una matiz diagonal de $2\times 2$. Encontrar condiciones
sobre $A$ que garanticen que:
$$\lim_{t\to +\infty} x(t)=0$$ para todas las soluciones de $\dot{x}=Ax$.
\eej

\bej Sea $A$ una matriz de $2\times 2$.
\begin{enumerate}
\item[(a)] ?`Cu\'al es la relaci\'on entre los campos $x\rightarrow Ax$
y $x\rightarrow (-A)x$?
\item[(b)] ?`Cu\'al es la relaci\'on geom\'etrica entre la soluci\'on
de $\dot{x}=Ax$ y $\dot{x}=-Ax$?
\end{enumerate}
\eej

\bej Determinar todas las soluciones del sistema
$$\left(\begin{aligned}\dot{x}\\ \dot{y}\end{aligned}\right)=
\left(\begin{array}{rr}5 & 3\\-6 & -4\end{array}\right)
\left(\begin{aligned}x\\ y\end{aligned}\right)$$ y hacer un
bosquejo del diagrama de fases.
\eej

\bej Hallar las soluciones y realizar un bosquejo del diagrama de fases para los
sistemas (b) y (d) del Ejercicio 2.
\eej

\bej Realizar un {gráfico aproximado} de las líneas de flujo de los siguientes campos vectoriales:
\[%
\text{(a)} \ F(x,y)=(x^{2},y^{2}) \qquad \text{(b)} \ F(x,y)=(x,x^{2}) \qquad
 \text{(c)} \ F\left(  x,y\right)
=\left(  1,x+y\right)
\]
\eej

\bej Para los siguientes  sistemas, hallar los equilibrios y analizar su estabilidad.
$$\text{(a)}\; \left\{\ba{l}\dot{x}=\sen x+\cos y\\ \dot{y}=xy \ea \right. \qquad
\text{(b)}\; \left\{\ba{l}\dot{x}=(x+1)e^y-1\\ \dot{y}=x+ y \ea \right.
\qquad \text(c)\left\{\ba{l}\dot x=x^2-y-1\\ \dot y = xy\ea\right.$$
\eej

\bej Para las siguientes  ecuaciones, hallar los equilibrios y analizar su estabilidad:
$$\text{(d)}\quad\ddot x+c\dot x-x^3=1,
\quad\quad \qquad\text{(e)}\quad\ddot x+x^3-x=0,
\quad       \quad \qquad\text{(f)}\quad\ddot x-x+\cos x=0.    $$
\eej

\bej Para cada uno de los siguientes sistemas no lineales hallar los puntos de equilibrio y { esbozar} el diagrama de fases
alrededor de cada uno de ellos:

$$\text{(a)}\; \left\{\ba{l}\dot{x}=x e^y\\ \dot{y}=-1+y+\sen(x) \ea \right. \qquad
\text{(b)}\; \left\{\ba{l}\dot{x}=e^{x-y}-1\\ \dot{y}=x y-1 \ea \right.
\qquad \text(c)\left\{\ba{l}\dot x=x(y-1)-4\\ \dot y = x^2-(y-1)^2\ea\right.$$
\eej

\bigskip

\bigskip



\bf{Modelos de crecimiento poblacional.}



\bej  Dada una población $x(t)$ se denomina
\bf{tasa de crecimiento} a la razón $\frac{\dot x}x$.

Como se vió en la Práctica 5, un modelo que considera que hay una población límite
$K$ es el modelo logístico en el que la razón de crecimiento es de la forma
$r\big(1-\frac xK\big)$.

Cuando hay dos poblaciones que conviven, digamos
$x$ e $y$, las razones de crecimiento de éstas dependen de ambas poblaciones. El modelo más sencillo que considera una población
de depredadores $y$ y sus presas $x$ es el de Lotka--Volterra:
$$
\left\{\begin{aligned}
&\dot x=x(a-by)\\
&\dot y=y(-c+dx)
\end{aligned}\right.
$$
con $a,\,b,\,c,\,d>0$.

Si a este modelo se le agrega el hecho de que cada población tiene por sí misma
un límite para su supervivencia se obtiene el modelo
$$
\left\{\begin{aligned}
&\dot x=x(a-\delta x-by)\\
&\dot y=y(-c+dx-\gamma y)
\end{aligned}
\right.
$$
con todas las constantes positivas.

\medskip

Considerar los siguientes sistemas correspondientes a poblaciones de depredador--presa
con crecimiento limitado:
$$
\text{(a)}\; \left\{\ba{l}\dot{x}=x(2-x-y)\\ \dot{y}=y(-1+x-y) \ea \right. \qquad
\text{(b)}\:\left\{\ba{l}\dot{x}=x(2-3x-y)\\ \dot{y}=y(-1+x-y) \ea \right.
$$

Para cada uno de estos sistemas, encontrar los puntos de equilibrio y esbozar
 el diagrama de fases alrededor de cada uno de ellos. Observar que el comportamiento
 depende fuertemente de los valores de los parámetros.
\eej

\bigskip

\bf {Din\'amica en un campo de fuerzas conservativo.}

\medskip

Dado $V(x)\in C^2(\mathbb{R})$ consideremos la ecuaci\'on diferencial
$$\ddot{x}=-V'(x),$$
que toma la siguiente forma de sistema en el plano de fases
\[%
\left\{
\begin{aligned}
&\dot{x} =  y,\\
&\dot{y}  =  -V'(x).
\end{aligned}
\right.
\]%

En mecánica $V(x)$ es el potencial y $-V^\prime(x)$ es la fuerza.

\bej Demostrar que la energía $$H(x,y)=\frac12 y^2+V(x)$$ es una constante del movimiento.
El primer t\'ermino es el de la energ\'{\i}a cin\'etica y el segundo, la potencial.

\eej

\bej Considerar $V(x)=\frac 12 k x^2.$ Esbozar el diagrama de fases considerando diferentes niveles de la energ\'{\i}a $H(x,y).$ Verificar que todas las trayectorias son acotadas. Utilizar la cantidad conservada obtenida en el punto anterior para obtener la posición máxima en funcion del dato inicial.
Este potencial es el del oscilador armónico, es decir un resorte sin la acci\'on de fuerzas externas.

\eej

\bej

$V(x)=\frac12 kx^2-mgx$ es el potencial que da origen a la fuerza a la que est\'a
sometida una masa $m$ que cuelga de un resorte con costante $k$ (llamamos $x$ a
la posici\'on aunque el movimiento se desarrolla en sentido vertical para
no confundir con la variable $y$ del plano de fases que representa la velocidad).
Compare este caso con  el del ejercicio anterior.

\eej

\bej
Considerar el potencial $V(x)=\frac{a}{x^2}-\frac{1}{x},$ con $x>0,\ a>0.$
Esbozar el diagrama de fases considerando diferentes niveles de la energía.
Describir cualitativamente el movimiento para valores iniciales tales que la
energ\'{\i}a sea positiva, negativa o nula. Se trata del movimiento radial
(es decir, $x$ representa la distancia al origen) de una partícula sometida a un campo gravitatorio, la cantidad $\frac{1}{x^2}$ suele interpretarse como un {\em potencial centrífugo}, y $V(x)$ es el {\em potencial eficaz}.

\eej

\bej Considerar el potencial del punto anterior. Fijado $x_0$  obtener el valor mínimo de $|y_0|$ para el cual la trayectoria cuyo nivel de energía es $H(x_0, y_0)$ es no acotada. (La cantidad $y_0$ es la {\em velocidad de escape}.)

\eej

%\bej Considerar $V(x)=1-\cos x$ (el potencial asociado al p�ndulo). Esbozar el
%diagrama de fases. Verificar que todas las trayectorias est�n acotadas en el eje $y.$ Para cada $x_0\in [-\pi, \pi]$ obtener la {\em velocidad de escape}.
%
%\eej
%
%\bej Se lanza hacia arriba un cuerpo de masa $m$ con una
%velocidad inicial $v_0$. Sabiendo que la fuerza gravitatoria
%($F=-mg$) es conservativa.
%\begin{enumerate}
%\item Hallar la energ\'ia potencial y cin\'etica del cuerpo cuando
%alcanza una altura $h$. \item Graficar el potencial y encontrar el
%valor de $h$ donde el movimeiento cambia de direcci\'on.
%\end{enumerate}
%
%\eej

\bej  Hacer el bosquejo del diagrama de fases de los siguientes
potenciales:
\begin{center}
\includegraphics[width=5cm]{potencial1.eps}
\qquad
\includegraphics[width=5cm]{potencial2.eps}
\end{center}

\eej

\vspace{4.cm}

\bf{Dinámica de un campo gradiente.}

\bej Sea $V(x,y)=ax^2+by^2+x^2y,$ con $a,b \in \mathbb{R}$ no nulos. Consideremos
el sistema $\dot{X}=-\nabla V(X),$ donde $X=(x,y)$.

\begin{enumerate}

\item[(a)] Verificar que el origen es un punto de equilibrio y clasificar su estabilidad en función de los parámetros $a$ y $b.$

\item[(b)] Si $a$ y $b$ son no nulos, hallar los restantes puntos de equilibrio y analizar su estabilidad.

\item[(c)] Para una función $V(x)$ general, observar que los equilibrios son los puntos
críticos de $V$. \textquestiondown De qué depende su estabilidad?
\end{enumerate}

\eej



\bigskip

\bf {Campo central de fuerzas}

\bej Considerar el movimiento de una part\'icula en un campo de
fuerzas central, esto es, se considera la ecuaci\'on
$$m\ddot{x}=-\nabla V(x) \ \ x\in \mathbb{R}^3/\{0\}$$ donde
$V(x)=V_0(|x|), \, \mbox{ con } V_0 \in C^2((0,\infty),\mathbb{R}).$

\begin{enumerate}
\item[(a)] Probar que se conserva el momento angular $M$, relativo al
origen, donde $M=x\times m\dot{x}$ (producto vectorial).
\item[(b)] Mostrar que todas las \'orbitas son planares (en el plano
perpendicular a $M$).% \item Probar la ley de Kepler, que dice que
%el radiovector barre \'areas iguales en tiempos iguales.\\
%(Sug: Usar la formula de Green $\int_{\bigtriangleup} dx dy=\frac{1}{2}\int_{\partial \bigtriangleup} x dy-y dx $ para
%calcular el \'area del tri\'angulo curvo).
\end{enumerate}

\eej

\end{document}
