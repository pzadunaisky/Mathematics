%%%%%%%%%%%%%%%%%%%%%% Generalities %%%%%%%%%%%%%%%%%%5
\documentclass[11pt,fleqn]{article}
\usepackage[paper=a4paper]
  {geometry}

\pagestyle{plain}
\pagenumbering{arabic}
%%%%%%%%%%%%%%%%%%%%%%%%%%%%%%%%
\usepackage{notas}
\usepackage{tikz}
%%%%%%%%%%%%%%%%%%%%%%%%%%% The usual stuff%%%%%%%%%%%%%%%%%%%%%%%%%
\newcommand\NN{\mathbb N}
\newcommand\CC{\mathbb C}
\newcommand\QQ{\mathbb Q}
\newcommand\RR{\mathbb R}
\newcommand\ZZ{\mathbb Z}
\renewcommand\k{\Bbbk}

\newcommand\F{\mathcal F}
\newcommand\V{\mathcal V}
\newcommand\D{\mathcal D}
\renewcommand\H{\mathcal H}

\newcommand\maps{\longmapsto}
\newcommand\ot{\otimes}
\renewcommand\to{\longrightarrow}
\renewcommand\phi{\varphi}
\newcommand\Id{\mathsf{Id}}
\newcommand\im{\mathsf{im}}
\newcommand\coker{\mathsf{coker}}
%%%%%%%%%%%%%%%%%%%%%%%%% Specific notation %%%%%%%%%%%%%%%%%%%%%%%%%
\newcommand\g{\mathfrak g}
\newcommand\gl{\mathfrak{gl}}
\newcommand\gen{\mathsf{gen}}
\newcommand\std{\mathsf{std}}
\newcommand\crit{\mathsf{crit}}
\newcommand\ceq{\stackrel{.}{=}}

\newcommand\vectspan[1]{\left\langle #1 \right\rangle}

\renewcommand\ll{\llbracket}
\newcommand\rr{\rrbracket}

\DeclareMathOperator\End{End}
\DeclareMathOperator\Frac{\mathsf{Frac}}
\DeclareMathOperator\ev{ev}
\DeclareMathOperator\sg{sg}
%%%%%%%%%%%%%%%%%%%%%%%%%%%%%%%%%%%%%% TITLES %%%%%%%%%%%%%%%%%%%%%%%%%%%%%%
\title{Singular Gelfand Tsetlin modules over $\gl$ à la Zadunaisky}
\date{[singular-GT.tex]}
\author{Pablo Zadunaisky}
\begin{document}
\maketitle

The objective of these notes is to understand the construction of Singular
Gelfand-Tsetlin modules due to Futorny, Grantcharov and Ramirez found in 
\cite{FGR-singular-gt} and generalize it to build Gelfand-Tsetlin modules associated to arbitrary tableaux. 

\section{GT Tableaux and GT algebras}
References for all results can be found in \cite{FGR-singular-gt}*{section 3}.

\paragraph
For each $m \in \NN$ set $\ll m \rr = \{1, 2, \ldots, m\}$.

Fix $n \in \NN_{\geq 2}$, and set $N = \frac{n(n+1)}{2}$. For each $m \in \NN$ 
set $U_m = U(\gl(m, \CC))$ and $Z_m \subset U_m$ its center, and set $U = 
U_n$. We get a chain of inclusions $U_1 \subset U_2 \subset \cdots \subset U_n$
induced by the maps sending standard generators $E_{i,j} \in \gl(k,\CC)$ to 
the corresponding $E_{i,j} \in \gl(k+1, \CC)$. 

\paragraph
\label{HC-morphism}
For each $m \in \NN$ the algebra $Z_m$ is a polynomial algebra on the 
generators
\[
	c_{m,k} = \sum_{(i_1, \ldots, i_k) \in [m]^k} E_{i_1,i_2} E_{i_2,i_3} 
		\cdots E_{i_k, i_1} \qquad \qquad 1 \leq k \leq m.
\]
Let $\Lambda_m = \CC[\lambda_{m,k} \mid 1 \leq k \leq m]$ be a polynomial
algebra in $m$ variables, and set
\[
	\gamma_{m,k} = \sum_{i = 1}^m (\lambda_{m,i}+m-1)^k 
	\prod_{i \neq j} \left( 1 - \frac{1}{\lambda_{m,i} - \lambda_{m,j}}\right).
\]
Although it is not obvious, the $\gamma_{m,k}$ are algebraically independent 
polynomials and they are invariant by the obvious action of the symmetric 
group $S_m$ on $\Lambda$; in fact $\Lambda_m^{S_m} = \CC[\gamma_{m,k} \mid 
1 \leq k \leq m]$, and there is an embedding $Z_m \to \Lambda$ given by 
$c_{m,k} \mapsto \gamma_{m,k}$. 

\paragraph
\label{GT-algebra}
We write $\Gamma = \CC[c_{m,k} \mid 1 \leq k \leq m \leq n] = 
\subset U$, which is the algebra generated by $\bigcup_{k=1}^n Z_k$. The 
$c_{m,k}$ are algebraically independent and hence this is isomorphic to a 
polynomial algebra in $N$ generators.

Let $\Lambda = \CC[\lambda_{i,j} \mid 1 \leq i \leq j \leq n]$ be a polynomial
algebra in $N$ variables. The group $S_m$ acts on $\Lambda$ 
permuting the variables $\lambda_{m,k}$ with $1 \leq k \leq m$ and fixing the 
rest. This induces an action of the group $G = S_n \times S_{n-1} \times 
\cdots \times S_1$ on $\Lambda$. Composing with the embedding from 
\ref{HC-morphism} we get that $\Gamma$ is isomorphic to $\Lambda^G$.

\paragraph
\label{GT-tableaux}
Recall we have fixed $n \in \NN$ and set $N = \frac{n(n+1)}{2}$. A 
\newterm{Gelfand-Tsetlin 
tableau} of size $n$ is a vector $v \in \CC^{N}$, whose coordinates are 
indexed by $\{(i,j) \mid 1 \leq i \leq j \leq n\}$. We can represent this 
graphically as [INSERT IMAGE]. For ease of reference we will sometimes write 
\[
v = (v_{n,1}, v_{n,2}, \ldots v_{n,n} 
	\mid v_{n-1,1}, \ldots, v_{n-1,n-1} 
	\mid \cdots 
	\mid v_{1,1}).
\]

\paragraph
\label{various-tableau}
We say that a GT-tableau $v$ is 
\begin{itemize}
\item \newterm{integral} if $v \in \ZZ^{N}$. 

\item \newterm{standard} if for all $1 \leq i < j \leq k < n$ the difference 
$v_{k,i}-v_{k-1,i} \in \ZZ_{\geq 0}$ and $v_{k-1,i}-v_{k,i+1} \in \ZZ_{>0}$; 
we denote the set of standard tableaux by $\CC^N_\std$.

\item \newterm{generic} if for all $1 \leq i < j \leq k 
< n$ the difference $v_{k,i}-v_{k,j}$ is not in $\ZZ$; we denote the set of 
generic tableaux by $\CC^N_\gen$. 

\item \newterm{singular} if it is non-generic tableau, i.e. there is a pair of 
entries in the same line differing by an integer. We refer to any pair of such 
entries as a \newterm{singular pair}. If there is exactly one singular pair 
then the tableau is called \newterm{1-singular}. Finally if two 
entries in the same row are equal the tableau is called \newterm{critical}, 
and we refer ti them as a \newterm{critical pair}.
\end{itemize}

\paragraph
Set $\ZZ^N_0 = \{v \in \ZZ^N \mid v_{n,i} = 0 \mbox{ for } 1 \leq i \leq n\}$.
For any ring $A$ we denote by $V_A$ the free $A$-module generated by all 
tableaux of the form $T(v)$ with $v \in \ZZ_0^N$. The group $G = S_n \times 
S_{n-1} \times \cdots \times S_1$ acts on $V_A$ in an obvious fashion.


\newpage

\section{The construction of a GT-module associated to a fully critical 
tableau}

\subsection{Polynomials over symmetric polynomials}
In this section we fix $m \in \ll n \rr$, and write $x_i = \lambda_{m,i}$
for all $i \in \ll m \rr$, so $\Lambda_m = \CC[x_i \mid i \in \ll m \rr]$. 
Set $\Gamma_m = \Lambda^{S_m}$. By classical results $\Gamma_m$ is isomorphic 
to a polynomial algebra in $m$ generators and its Hilbert series is given by 
$\prod_{i = 1}^m \frac{1}{1-t^i}$, and $\Lambda_m$ is a free 
$\Gamma_m$-module, see \cite{Chev-reflection-groups}.

Set $M = \{\lambda_{k,i} - \lambda_{k,j} - z \mid 1 \leq i, j \leq k < n, z 
\in \NN\}$ and $M' = \{\lambda_{k,i} - \lambda_{k,j} - z \mid 1 \leq i, j 
\leq k < n, z \in \NN^*\}$. Put $A = \Lambda[M^{-1}]$ and $B = 
\Lambda[M'^{-1}]$, so $B \subset A \subset \Lambda$. We set $F = 
\Frac(\Lambda)$ and $E = \Frac(\Gamma)$. The action of $S_m$
extends to $F$, and $E = F^{S_m}$ so $E \subset F$ is a Galois extension with
Galois group $S_m$. 


\paragraph
Let $J$ be the ideal generated by symmetric polynomials of positive degree
in $\Lambda_m$. Since $J$ is a symmetric ideal, for each $l \in \NN$ the 
space of polynomials of degree $l$ decomposes as $K_l \oplus J_l$. Setting 
$K = \bigoplus_l K_l$ we get $\Lambda_m / J \cong K$ as a representation of 
$S_m$. By \cite{Chev-reflection-groups} $K$ is a graded version of the standard
representation of $S_m$, and its Hilbert series is given by $\prod_{j = 1}^m
\frac{1-t^i}{1-t}$.

Let $\Delta = \prod_{1 \leq i < j \leq m} (x_i - x_j)$. According to 
\cite{Man-symm-book}*{2.5.3} there is a symmetric bilinear form on $\Lambda_m$
defined by
\begin{align*}
(f,g)
	&=\ev_0 \left(
		\frac{1}{\Delta}\sum_{\sigma \in S_m} \sg(\sigma) f^\sigma g^\sigma
	\right)
\end{align*}
where $\ev_0(p) =p(0)$. The kernel of this bilinear form is exactly $J$,
so in particular its restriction to $K$ is non-degenerate. For $f,g \in K$
the inner product 



There is a bilinear form $\phi: S/J \times S/J \to \CC$ given by $\phi(f,g)
= \overline{\sum_{\sigma \in S_m} \sigma(fg/\Delta)}$. Equivalently, 
$\phi(f,g)$ is obtained by antisymmetrizing the product $fg$ and dividing by 
$\Delta$. This is well-defined since all antisymmetric polynomials of degree
less than $2m!$ are scalar multiples of $\Delta$. This is a non-degenerate 
bilinear form, and hence every basis has a dual basis (this is \emph{not} a
scalar product, since $\phi(f,f) = 0$ for any $f$).

\paragraph
Let $f_1, \ldots, f_{m!}$ be a homogeneous basis of $R$ as $S$-module, and let
$f_i^*$ denote the $i$-th element of the dual basis. For each $v \in \ZZ^N_0$ 
set 
\begin{align*}
D_iT(v) 
	&= \sum_{\sigma \in S_m} \sigma \left( \frac{f_i}{\Delta} T(v)\right).
\end{align*}	

\begin{Lemma*}
\begin{itemize}
\item For all $\sigma \in S_m$ we have $\langle D_iT(v) \mid 1 \leq i \leq 
m! \rangle_\CC = \langle D_iT(\sigma\cdot v) \mid 1 \leq i \leq m! 
\rangle_\CC$.

\item $T(\sigma(v)) \in \langle D_iT(v) \mid 1 \leq i \leq m! \rangle$.

\item $\dim \langle D_iT(v) \mid 1 \leq i \leq m!\rangle_\CC = \# S_m \cdot v$.
\end{itemize}

\end{Lemma*}

\begin{Definition}
Set $L_B = \langle D_iT(v) \mid 1 \leq i \leq m!, v \in \ZZ^N_0 \rangle_\CC$
\end{Definition}

\begin{Lemma*}
For all $v \in \ZZ^N_0$ we have $U T(v) \in L_B$.
\end{Lemma*}
\begin{proof}
We claim that for all $v \in \ZZ^N_0$. 
\begin{align*}
\sum_{r=1}^t \frac{p_{t,r}^\pm(\lambda^v)}{q_{t,r}(\lambda^v)} \in B.
\end{align*}
Clearly this holds if $r \neq k$ or if all the entries in $v$ are diferent. If
$v$ has entries $v_{k,i} = v_{k,j}$ let $\tau = (i,j)$, so $\tau(v) = v$, and
\begin{align*}
\tau \cdot \sum_{r=1}^k \frac{p_{k,r}^\pm(\lambda^v)}{q_{k,r}(\lambda^v)}
	&= \sum_{r=1}^k \frac{p_{k,\tau(r)}^\pm(\lambda^{\tau(v)})}
	{q_{k,\tau(r)}(\lambda^{\tau(v)})},
\end{align*}
so this is a rational function which is symmetric on $x =x_{k,i}, y =x_{k,j}$, 
which implies that it can be written as $f/g (x-y)^{2l}$ for $f,g$ coprime 
with $(x-y)$ and $l \in \ZZ$. Since $(x-y)$ appears once in 
$q_{k,i}(\lambda^v)$ and $q_{k,j}(\lambda^v)$ and never in the other terms, it 
is clear that the sum has a pole of order at most one in $(x-y)$, so $l \geq 
0$. Since $i,j$ were arbitrary, we see that no polynomial of the form $x_{k,i} 
- x_{k,j}$ appears in the denominator
\end{proof}


\begin{Proposition}
$U L_B \subset L_B$
\end{Proposition}


\newpage
For each $(k,i,j) \in \Sigma$ let $\alpha^k_{i,j} = \lambda_{k,i} - 
\lambda_{k,j}$, and let $\mathcal H$ be the infinite hyperplane arrangement
$\bigcup_{(k,i,j) \in \Sigma, z \in \ZZ} \{\alpha_{i,j}^k = z\}$. Clearly
$\CC^N_\gen = \CC^N \setminus \H$, and $A$ is the set of regular functions
defined in the complement of $\H$. Now let $\H^* = \bigcup_{(k,i,j) \in 
\Sigma, z \in \ZZ\setminus \{0\}} \{\alpha_{i,j}^k = z\}$, which is clearly
the set of all singular non-fully critical points in $\CC^N$. Set $\CC^N_\crit
= \CC^N \setminus \H^*$ and let $B$ be the algebra of regular functions over
$\CC^N_\crit$. Once again, we will look for $L_B \subset V_A$ a $U$-submodule
stable by the action of $U$.


We will now consider matrices with coefficients in $R$ with rows indexed by 
$S_m$ and columns indexed by $[[m!]]$; the reader is free to choose some order 
on $S_m$ but this is not strictly necessary. Whenever we say ``matrix'' in 
this section we will mean such a matrix, unless it is explicitly stated. The 
symmetric group $S_m$ acts on these matrices by acting on their coefficients 
Given a matrix $M$, we will write $M^\sigma$ for its $\sigma$-row, $M_j$ for 
its $j$-th column, and $M^\sigma_j$ for the element in row $\sigma$ and column 
$j$. We will say that a matrix has equivariant rows if $\tau M^\sigma = 
M^{\tau \sigma}$ for all $\tau, \sigma \in S_m$.

\begin{Lemma}
\label{L:equivariant-rows}
Let $\Delta = \prod_{i < j} (x_i - x_j)$. If a matrix $M$ has equivariant rows 
then $\Delta^{m!/2} \mid \det M$.
\end{Lemma}
\begin{proof}
We first prove and intermediate result. Let $X$ be a set with an action by 
$\langle (12)\rangle$ without fixed points of cardinality $s$. Let $T$ be a 
square matrix with coefficients in $R$ and rows indexed by $X$ and columns by
$[[s]]$, such that $(12)T^x = T^{(12)x}$. Then $(x_1 - x_2)^{|X|/2} \mid \det 
T$. We will do this by induction on $s = |X|$ starting with $s = 2$. In that 
case $M$ is of the form
$\begin{pmatrix}
f & g \\ f^{(12)} & g^{(12)} 
\end{pmatrix}$, and $\det M = fg^{(12)} - f^{(12)}g = fg^{(12)} - 
(fg^{(12)})^{(12)}$ which is divisible by $x_1 - x_2$. 

Now assume the result holds for matrices of size less than $s$.
The hypothesis that the action of $(12)$ has no fixed points implies that 
$X$ can be decomposed as $Y \sqcup (12)Y$, with $|Y| = s/2$. Fix $y \in Y$,
and for each $i,j \in [[s]]$ write $T(i,j)$ for the martix $T$ with rows
$y, (12)y$ and columns $i,j$ erased. Then $T(i,j)$ is again a matrix of the
desired type, and hence $(x_1 - x_2)^{s/2-1} \mid \det T(i,j)$, and
\begin{align*}
\det T
	&= \sum_{i<j} (T^y_i T^{(12)y}_j - T^y_jT^{(12)y}_i) \det T(i,j),
\end{align*}
and since $x_1 - x_2 \mid T^y_i T^{(12)y}_j - T^y_jT^{(12)y}_i = T^y_i 
T^{(12)y}_j - (T^y_iT^{(12)y}_j)^{(12)}$, the result follows. 

In particular for any matrix $M$ with equivariant rows, taking $X = S_m$ we 
obtain that $(x_1 - x_2)^{m!/2} \mid \det M$. Now fix $1 \leq i<j \leq m$ and 
select $C \subset S_m$ such that $S_m = C \sqcup (ij) C$. By definition $(ij) 
\cdot \det M = \det (ij) \cdot M$, and since $M$ has equivariant rows this 
ammounts to interchanging $M^\tau$ and $M^{\sigma \tau}$. We are making $m!/2$ 
such interchanges, so $(ij) \det M = (-1)^{m!/2} \det M$. This implies that
$\sigma \det M = (\sg \sigma)^{m!/2} \det M$, so in particular $x_i - x_j$
divides $\det M$ as many times as $x_1 - x_2$, and hence $\Delta^{m!/2} \mid
\det M$.
\end{proof}



Now let $f_1, \ldots, f_{m!}$ be a basis of $R$ as $S$-module, and let $M = 
M(f_1, \ldots, f_{m!})$ be the matrix $M^\sigma_j = f_j^\sigma$. Notice that 
$\det M$ is well defined up to a sign. The \emph{discriminant} of the basis is 
defined as $(\det M)^2$.

\begin{Lemma*}
Let $f_1, \ldots, f_{m!}$ be any basis of $R$ as $S$-module and let $M = 
M(f_1, \ldots, f_{m!})$. Then $\sigma \cdot \det M = (\sg \sigma)^{m!/2}
\det M$, so the discriminant is an invariant polynomial.
\end{Lemma*}
\begin{proof}
It is enough to check that the formula holds for $\sigma = (ij)$ with $1 \leq 
i < j \leq m$. Choose representatives $\tau_1, \ldots, \tau_{m!/2} \in S_m$ of
the coclasses in $S_m/\langle \sigma \rangle$. For each $1 \leq t \leq m!/2$ 
set $F_t = (\tau_i f_j)_{1 \leq j \leq m!}$ and $F_{t + m!/2} = \sigma F_t$,
and let $N \in M_{m!}(R)$ be the matrix whose $t$-th row is $F_t$ for all 
$t$, so $\det M = \pm \det N$. Now applying $\sigma$ to $N$ results in 
exchanging rows $F_t$ and $F_{t + m!/2}$, so the sign of the determinant 
changes exactly $m!/2$ times. 
\end{proof}  

We now introduce some notation. For $p,q \in K = Frac(R)$, write $p \ceq q$
if there exists $c \in \CC^\times$ such that $p = c q$.
\begin{Proposition}
Let $f_1, \ldots, f_{m!}$ be a good basis of $R$ as $S$-module. Then $\det 
M(f_1, \ldots, f_{m!}) \ceq \Delta^{m!/2}$.
\end{Proposition}
\begin{proof}
Put $M = M(f_1, \ldots, f_{m!})$. Since $\sigma M^\tau = M^{\sigma \tau}$
then $\Delta^{m!/2} \mid \det M$ by Lemma \ref{L:equivariant-rows}. Also the
degree of $\det M$ equals the sum of the degrees of the $f_i$ which is 
$\frac{m!}{2} \binom{m}{2}$ by Lemma \ref{L:sum-of-degrees}, and clearly this 
is also the degree of $\Delta^{m!/2}$. Finally, since $\det M$ is not zero it
follows that $\det M \ceq \Delta^{m!/2}$. 
\end{proof}


Let $B \subset K$ be the subalgebra
\[
B 
	= \{f/g \in A \mid \mbox{ $g$ is not divisible by }
		\lambda_{l,i} - \lambda_{l,j} \mbox{ for any } 1 \leq i < j \leq l\}
\]
Recall that we call a tableau fully critical if every singular pair is 
critical. Notice that any fully critical tableau $v$ defines a one dimensional 
$B$-module $\CC_v$. The main objective of this section is to find a 
$B$-lattice of $V_K$ which is also $U$-stable.

\paragraph
\label{S-l-actions}
\about{$S_l$-actions}
Fix $l < n$. The group $S_l$ acts on $\ZZ_0^N$, by permuting the entries in
the $l$-th row of the tableaux, and hence it naturally acts on the complex 
vector space $V_\CC$; we will write $\sigma \cdot T(v)$ for $T(\sigma(v))$
for each $\sigma \in S_l$. 

The symmetric group also acts on $\Lambda = \CC[\lambda]$ by permuting the 
generators $\lambda_{l,i}$ and fixing the others; by extension, it also acts 
on $K = \CC(\lambda)$; we will denote the action of $\sigma \in S_l$ over $f 
\in \Lambda$ by $\sigma \cdot f$ or ${}^\sigma f$. With this definition the 
algebra $B$ is stable by the action of $S_l$. Also, since $V_K = K \ot V_\CC$, 
the symmetric group acts on $V_K$ by a diagonal action, i.e. for each $\sigma 
\in S_l, f \in K$ and $v \in \ZZ_0^N$ we have $\sigma \cdot (f \ot T(v)) = 
 {}^\sigma f \ot T(\sigma(v))$. 

\paragraph
\label{L:action-p-q}
Given $\sigma \in S_l$ we set 
\begin{align*}
\sigma(k,i)
	&= \begin{cases}
		(k,i) & k \neq l \\
		(l,\sigma(i)) & k = l.
	\end{cases}
\end{align*}
With this notation $\sigma \cdot \delta^{k,i} = \delta^{\sigma(k,i)}$ and 
$\sigma \cdot \lambda_{k,j} = \lambda_{\sigma(k,j)}$. Also we get that
$\sigma \cdot p(v) = p(\sigma^{-1} \cdot v)$ for each $\sigma \in S_l, p \in K$
and $v \in \ZZ_0^N$ whenever the evaluation makes sense. The next lemma follows
from this.

\begin{Lemma*}
For each $1 \leq k \leq j \leq n$ and each $\sigma \in S_l$ the following 
formulas hold.
\begin{align*}
\sigma \cdot p_{k,j}^\pm 
	&= p_{\sigma(k,j)} 
	&\sigma \cdot q_{k,j} = q_{\sigma(k,j)}.
\end{align*}
Also $\sigma \cdot (p(\lambda^v)) = (\sigma \cdot p)(\lambda^{\sigma(v)})$
for each $p \in K$ and each $v \in \CC^N$.
\end{Lemma*}

\begin{Proposition}
\label{P:action-S-l-equivariant}
The operators $E_{k,k+1}, E_{k,k}, E_{k+1,k}$ are $S_l$-equivariant.
In particular the subspace of invariant elements $V_K^{S_l}$ is stable
by the action of $U$.
\end{Proposition}
\begin{proof}
Fix $\sigma \in S_l$. We have
\begin{align*}
E_{k,k+1}(\sigma\cdot T(v))
	&= -\sum_{j=1}^k 
		\frac{p_{k,j}^+(\lambda^{\sigma(v)})}{q_{k,j}(\lambda^{\sigma(v)})}
			T(\sigma(v) + \delta^{k,j}).
\end{align*}
Taking $(k,i) = \sigma^{-1}(k,j)$ the last sum equals
\begin{align*}
-\sum_{i=1}^k 
	\frac{p_{\sigma(k,i)}^+(\lambda^{\sigma(v)})}{q_{\sigma(k,i)}
		(\lambda^{\sigma(v)})}	T(\sigma(v + \delta^{k,i}))
	&= -\sum_{i=1}^k \sigma \cdot \left( 
		\frac{p_{k,i}^+(\lambda^{v})}{q_{k,i}(\lambda^{v})} T(v+\delta^{k,i})
		\right) = \sigma (E_{k,k+1} T(v)).
\end{align*}
The proofs that $E_{k+1,k}$ is equivariant is similar. In the case of $E_{k,k}$
the result follows immediately from the definitions.
\end{proof}

\paragraph
\label{L:determinant}
Recall that $\Lambda_l = \CC[\lambda_{l,i} \mid 1 \leq i \leq l]$. By classical
results, $\Lambda_l^{S_l} \subset \Lambda_l$ is a polynomial ring generated by 
the elementary symmetric polynomials, and there is an $S_l$-equivariant 
isomorphism of $\Lambda_l^{S_l}$-modules $\Lambda_l \cong \Lambda_l^{S_l} \ot 
\CC[S_l]$. In particular $\Lambda_l$ is a free $\Lambda_l^{S_l}$-module of 
rank $l!$. 

\begin{Lemma*}
Let $\{f_i \mid 1 \leq i \leq l!\}$ be a basis of $\Lambda_l$ as 
$\Lambda_l^{S_l}$-module. Enumerate the elements of $S_l$ as $\sigma_1, 
\sigma_2, \ldots, \sigma_{l!}$, with $s_1 = \Id$, and let 
$X = ({}^{\sigma_i} f_j) \in M_{l!}(\Lambda_l)$. Then $\det X$ is a nonzero 
polynomial. Furthermore, we have $\sigma \cdot \det X = 
\begin{cases} \sg(\sigma) \det X & l \leq 3 \\
\det X & l > 3 \end{cases}$.
\end{Lemma*}
\begin{proof}
Let $F$ be the fraction field of $\Lambda_l$, and $E$ the fraction field of 
$\Lambda_l^{S_l}$. Then $E = F^{S_l}$, and the set $\{f_1, \ldots, f_{l!}\}$
is $E$-linearly independent, and hence an $E$-basis of $F$. Thus the square of
the determinant we are trying to find is the discriminant of this basis, and
since the extension $E \subset F$ is Galois, we may apply 
\cite{Neu-NT-book}*{(2.8) Proposition} to conclude that it is nonzero.

Set $p = \det X$, and let $\tau = (i,j) \in S_l$. Since the enumeration of the
elements of $S_l$ is arbitrary (this only changes the sign of the determinant 
of $X$), we may and do assume that $\sigma_i = \tau \sigma_{l!/2+i}$ for all
$i \in [l!/2]$, so $\tau \cdot p = \det ({}^{\tau\sigma_i} f_j) = (-1)^{l!/2} 
\det X$, since we are interchanging $l!/2$ pairs of rows. Thus if $l > 3$ the
sign is positive, while for $l = 2,3$ it is negative. Hence the vector space
generated by $p$ is a $1$-dimensional representation of $S_l$: if $l =2,3$ it
is the sign representation, and if $l > 3$ it is the trivial representation.  
\end{proof}

\paragraph
Let $v \in \ZZ^l$ and let $\lambda \vdash l$. We say that $v$ has 
\newterm{shape} $\lambda$ if the $i$-th most common value of its entries
appears with frequency $\lambda_i$. For example, if $l = 10$ and
$v = (a,a,a,a,a,b,b,b,c,d)$ then $v$ is of shape $(5,3,1,1)$. Clearly the
shape of $v$ is invariant by the action of $S_l$. We will say that $v$
of shape $\lambda$ is in \newterm{good shape} if all its coordinates from
$\lambda_i + 1$ to $\lambda_{i+1}$ are equal. In our example $v$ is in good
shape, and if we exchange the last two coordinates it remains in good shape,
but this no longer holds if we exchange the first and the last coordinates,
or if we put all the $b$'s in front of all the $a$'s. Clearly for any $v \in 
\ZZ^l$ there exists $\sigma \in S_l$ such that $\sigma \cdot v$ is in good
shape, but this is not unique. Say that $v$ is \newterm{fit} if it is in 
good shape, and $v <_{lex} \sigma \cdot v$ whenever $\sigma \cdot v$ is also 
in good shape; then for each $v \in \ZZ^l$ there is a unique $\sigma \in S_l$
such that $\sigma \cdot v$ is fit.

\paragraph
Given $v \in \ZZ_0^N$, its \newterm{$l$-shape} is the shape of its $l$-th row.
We can now define the $B$-lattice we are looking for.
\begin{Definition*}
For each $i = 1, \ldots, l!$ set $D_iT(v) = \sum_{\sigma \in S_l} \sigma 
\left( \frac{f_i}{\Delta} T(v) \right)$. Let $Z_l \subset \ZZ_0^N$ be the set 
of all elements whose $l$-shape is fit, and let 
\[
\mathcal B 
	= \left\{D_iT(v) \mid v \in Z_l, 1 \leq i \leq l!\right\}.
\] 
We denote by $L_B$ the $B$-module generated by $\mathcal B$
\end{Definition*}

\begin{Lemma}
\label{L:full-lattice}
Let $v \in \ZZ^N_0$. Then 
\begin{align*}
\vectspan{T(\sigma \cdot v) \mid \sigma \in S_l}_B
	&\subset \vectspan{
		\sum_{\sigma \in S_l} \sigma \left( \frac{f_i}{\Delta}
		T(v)\right) \mid 1 \leq i \leq l!
		}_B
\end{align*}
In particular, $T(v) \in L_B$.
\end{Lemma}
\begin{proof}
The determinant of the matrix $M = [\sigma_i\cdot(f_j/\Delta)]$ is equal to 
$\frac{(-1)^{l!/2}}{\Delta} \det X$, where $X$ is the matrix from Lemma 
\ref{L:determinant}, and hence nonzero. Thus there exists $H = (h_{s,t})
\in M_{l!}(K)$ such that $H M = \Id$, and in particular 
\begin{align*}
\sum_{j} h_{s,j} D_jT(v)
	&= \sum_{\sigma \in S_l} \sum_j 
		h_{s,j} \left( \sigma \cdot \frac{f_j}{\Delta} \right) T(\sigma(v))
	= T(\sigma_s(v)). 
\end{align*}
\end{proof}







\newpage
\begin{bibdiv}
\begin{biblist}
\bibselect{biblio}
\end{biblist}
\end{bibdiv}
\end{document}