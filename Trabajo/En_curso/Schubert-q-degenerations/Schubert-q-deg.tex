%%%%%%%%%%%%%%%%%%%%%% Generalities %%%%%%%%%%%%%%%%%%%
\documentclass[11pt,fleqn]{article}

\usepackage[paper=a4paper]
  {geometry}

\pagestyle{plain}

\usepackage{paragraphs}
\usepackage{hyperref}
\usepackage{amsthm,thmtools}

\usepackage[utf8]{inputenc}
\usepackage[spanish,english]{babel}
\usepackage{enumitem}
\usepackage[osf,noBBpl]{mathpazo}
\usepackage[alphabetic,initials]{amsrefs}
\usepackage{amsfonts,amssymb,amsmath}
\usepackage{mathtools}
\usepackage{graphicx}
\usepackage[poly,arrow,curve,matrix]{xy}
\usepackage{wrapfig}
\usepackage{xcolor}
\usepackage{helvet}
\usepackage{stmaryrd}
%\usepackage{showlabels}


%%%%%%%%%%%%Theorems, for paragraphs package%%%%%%%%%%%%%%%%%%%%%%%%%%
% numbered versions
\declaretheoremstyle[headformat=swapnumber, spaceabove=\paraskip,
bodyfont=\itshape]{mystyle}
\declaretheoremstyle[headformat=swapnumber, spaceabove=\paraskip,
bodyfont=\normalfont]{mystyle-plain}
\declaretheorem[name=Lemma, sibling=para, style=mystyle]{Lemma}
\declaretheorem[name=Proposition, sibling=para, style=mystyle]{Proposition}
\declaretheorem[name=Theorem, sibling=para, style=mystyle]{Theorem}
\declaretheorem[name=Corollary, sibling=para, style=mystyle]{Corollary}
\declaretheorem[name=Definition, sibling=para, style=mystyle]{Definition}

% unnumbered versions
\declaretheoremstyle[numbered=no, spaceabove=\paraskip,
bodyfont=\itshape]{mystyle-empty}
\declaretheoremstyle[numbered=no, spaceabove=\paraskip,
bodyfont=\itshape]{mystyle-empty-plain}
\declaretheorem[name=Lemma, style=mystyle-empty]{Lemma*}
\declaretheorem[name=Proposition, style=mystyle-empty]{Proposition*}
\declaretheorem[name=Theorem, style=mystyle-empty]{Theorem*}
\declaretheorem[name=Corollary, style=mystyle-empty]{Corollary*}
\declaretheorem[name=Definition, style=mystyle-empty]{Definition*}
\declaretheorem[name=Remark, style=mystyle-empty]{Remark*}

% plain style
\declaretheoremstyle[
        headformat={{\bfseries\NUMBER.}{\itshape\NAME}\NOTE\ignorespaces},
        spaceabove=\paraskip, 
        headpunct={.},
        headfont=\itshape,
        bodyfont=\normalfont
        ]{mystyle-plain}
\declaretheorem[sibling=para, style=mystyle-plain]{Example}
\declaretheorem[sibling=para, style=mystyle-plain]{Remark}

% proofs, just as in amsthm but with adapted spacing

\makeatletter
\renewenvironment{proof}[1][\textit{Proof}]{\par
  \pushQED{\qed}%
  \normalfont \topsep.75\paraskip\relax
  \trivlist
  \item[\hskip\labelsep
        \itshape
    #1\@addpunct{.}]\ignorespaces
}{%
  \popQED\endtrivlist\@endpefalse
}
\makeatother

\newcommand\note[1]{\marginpar{{
\begin{flushleft}
\tiny#1
\end{flushleft}
}}}

\renewcommand\labelitemi{-}
%%%%%%%%%%%%%%%%%%%%%%%%%%% The usual stuff%%%%%%%%%%%%%%%%%%%%%%%%%
\newcommand\NN{\mathbb N}
\newcommand\CC{\mathbb C}
\newcommand\QQ{\mathbb Q}
\newcommand\RR{\mathbb R}
\newcommand\ZZ{\mathbb Z}

\newcommand\maps{\longmapsto}
\newcommand\ot{\otimes}
\renewcommand\to{\longrightarrow}
\renewcommand\phi{\varphi}
\newcommand\qbinom[2]{\genfrac{[}{]}{0pt}{0}{#1}{#2}}
\newcommand\vspan[1]{\left\langle #1 \right\rangle}

%%%%%%%%%%%%%%%%%%%%%%%%% Specific notation %%%%%%%%%%%%%%%%%%%%%%%%%
\newcommand\A{\mathcal A}
\newcommand\B{\mathcal B}
\newcommand\C{\mathcal C}
\newcommand\D{\mathcal D}
\newcommand\I{\mathcal I}
\renewcommand\S{\mathcal S}
\newcommand\K{\mathcal K}
\newcommand\F{\mathcal F}
\newcommand\G{\mathcal G}
\newcommand\GG{\Gamma}
\renewcommand\L{\mathcal L}
\newcommand\M{\mathcal M}
\renewcommand\O{\mathcal O}
\newcommand\R{\mathcal R}
\newcommand\tphi{\tilde \phi}
\renewcommand\b{\mathbf b}
\renewcommand\k{\Bbbk}
\renewcommand\q{\mathbf{q}}
\newcommand\m{\mathfrak m}
\newcommand\n{\mathfrak n}
\newcommand\g{\mathfrak g}
\newcommand{\Rho}{\mathrm{P}}
\newcommand\opp{\circ}
\newcommand\norm{\mathsf{norm}}
\newcommand\join{\vee}
\newcommand\meet{\wedge}
\newcommand\DC{\mathsf{DC}}
\renewcommand\AA{\mathbb A}
\newcommand\lex{\mathsf{lex}}

\newcommand\flagbasis{FB}
\newcommand\schubertbasis{SB}


\DeclareMathOperator\Sgrp{\mathsf{Sgrp}}
\DeclareMathOperator\Grp{\mathsf{Grp}}

\DeclareMathOperator\Mod{\mathsf{Mod}}
\DeclareMathOperator\mmod{\mathsf{mod}}
\DeclareMathOperator\Hom{\mathsf{Hom}}
\DeclareMathOperator\Ext{\mathsf{Ext}}
\DeclareMathOperator\Tor{\mathsf{Tor}}
\DeclareMathOperator\GrHom{\underline{\mathsf{Hom}}}
\DeclareMathOperator\GrExt{\underline{\mathsf{Ext}}}
\DeclareMathOperator\GrTor{\underline{\mathsf{Tor}}}


\DeclareMathOperator\gr{\mathsf{gr}}
\DeclareMathOperator\im{Im}
\DeclareMathOperator\id{injdim}
\DeclareMathOperator\pd{pdim}
\DeclareMathOperator\ldim{ldim}
\DeclareMathOperator\height{\mathsf{ht}}
\DeclareMathOperator\st{\mathsf{st}}
\DeclareMathOperator\depth{depth}
\DeclareMathOperator\lcd{lcd}
\DeclareMathOperator\Spec{Spec}
\DeclareMathOperator\supp{supp}
\DeclareMathOperator\Id{Id}
\DeclareMathOperator\rank{rk}
\DeclareMathOperator\rk{rk}
\DeclareMathOperator\irr{irr}
\DeclareMathOperator\GKdim{\mathsf{GKdim}}
\DeclareMathOperator\relint{\mathsf{relint}}
\DeclareMathOperator\coker{\mathsf{coker}}
\DeclareMathOperator\tr{\mathsf{tr}}
\DeclareMathOperator\SL{\mathsf{SL}}
\DeclareMathOperator\ev{\mathsf{ev}}
\DeclareMathOperator\wt{\mathsf{wt}}


%%%%%%%%%%%%%%%%%%%%%%%%%%%%%%%%%%%%%% TITLES %%%%%%%%%%%%%%%%%%%%%%%%%%%%%%
\title{
Quantum toric degeneration of quantum Schubert varieties
\footnote{file:[lrpz3-paris-1.tex]}
}
\date{31/05/2018}
\author{L. Rigal, P. Zadunaisky}

\begin{document}
\maketitle

\begin{abstract}
We show that certain homological regularity properties of graded connected 
algebras, such as being AS-Gorenstein or AS-Cohen-Macaulay, can be tested by 
passing to associated graded rings. In the spirit of noncommutative algebraic 
geometry, this can be seen as an analogue of the classical result that, in a 
flat family of varieties over the affine line, regularity properties of one 
fiber extend to all fibers. 
We then show that quantized coordinate rings of flag varieties and Schubert 
varieties can filtered so that the associated graded rings are twisted 
semigroup rings in the sense of \cite{RZ}. This is a noncommutative version
of the result due to Caldero \cite{C} stating that flag varieties degenerate
into toric varieties, and implies that quantized coordinate rings of flag
varieties are AS-Cohen-Macaulay.
\end{abstract}


%\newpage 
\section{Introduction}
Let $\k$ be a field, and let $A$ be a noetherian commutative algebra. If we put
an ascending filtration on $A$ then we can build the Rees ring of the 
filtration, which is a free $\k[t]$-algebra $\mathcal R$ such that $A \cong 
\mathcal R / (t - \lambda)\mathcal R$ for all $\lambda \in \k^\times$, while
$\mathcal R / t \mathcal R$ is isomorphic to the associated graded ring. In 
geometric terms, the variety associated to $\mathcal R$ is a flat family over
the affine line, whose generic fiber is isomorphic to $\Spec R$ and whose
fiber over $0$ is isomorphic to $\Spec \gr R$. Now a standard result says that
fibers in a flat family have similar properties: if one fiber is regular, or 
Gorenstein, or Cohen-Macaulay, or has any other of a long list of properties, 
then the other fibers also have this property.

Of course, the idea of studying a ring by imposing a filtration and passing to 
the associated graded ring is a basic tool in an algebraist's toolbox, and 
applies well beyond the geometric context we just described. In 
particular the hypotheses of noetherianity and commutativity are not necessary
for the filtered-to-graded methods to work. However, in the spirit of 
noncommutative algebraic geometry, we should look at the case where $A$ is
noetherian, $\NN$-graded and connected (i.e. $A_0 = \k$) with an eye on the
geometric case. In this case, although there are no varieties associated to
our algebras, we have suitable analogues of being regular, or Gorenstein, or 
Cohen-Macaulay, defined in purely homological terms. Hence it makes sense to 
ask whether the algebra $A$ has any of these properties, and whether they are 
``stable by flat deformation'', i.e. if the fact that $\gr A$ has any of these 
properties implies that $A$ also has it. 

The main result of this article is that this is the case under a mild 
homological hypothesis over $\gr A$ known as property $\chi$, which was 
identified as central in noncommutative geometry already in the 
foundational paper \cite{AZ}. Notice that, in our context, $A$ is already 
graded (in fact as is usual in noncommutative geometry, in most examples it 
will be multigraded), so we must ask for the filtration to be by graded 
subspaces, and hence $\gr A$ comes with two gradings. This requires some 
careful bookkeepping, and property $\chi$ guarantees that all the relevant 
homological invariants have finite dimensional homogeneous components with 
respect to the multigrading.

As an example, we show that quantized coordinate rings of flag varieties and 
their Schubert subvarieties can be filtered so that the assocaited graded 
rings are semigroup rings, twisted by a $2$-cocycle. These latter algebras can 
be seen as noncommutative analogues of affine toric varieties, and where 
studied from this point of view in \cite{RZ2}. In particular, we know that
they are all AS-Cohen-Macaulay, and in view of our results the same holds for
quantum flag and Schubert varieties. We were originally inspired in our 
research by the article \cite{C}, where Caldero shows that classical flag and 
Schubert varieties have toric degenerations. Caldero proves this by finding a 
filtration in the quantized case, using as his base field a trascendental 
extenssion of $\CC$, and then specializing to the classical algebras. We 
include a modified version of his argument to show that such a degeneration is 
possible over an arbitrary transcendental extension of $\QQ$, which requires 
some technical results on canonical and monomial bases of quantized enveloping 
algebras. 

Toric degenerations of varieties (i.e. flat families with generic fiber 
isomorphic to the original variety, and special fiber a toric variety) 
are a staple in algebraic geometry, see for example the recent survey 
\cite{FFL}. As mentioned there, geometric properties of toric varieties can be 
translated into combinatorial properties of associated lattices, cones and 
polytopes. In the affine case this information is mostly codified by the 
associated affine semigroup ring, and as shown in \cite{RZ2} the $2$-cocycle 
twists of these rings hold essentially the same combinatorial information. 
Thus quantum affine toric varieties are valuable test models in noncommutative 
geometry, and hence obvious targets for noncommutative degenerations. We hope 
that, just as in the commutative case, quantum toric degenerations will become 
a valuable tool for the noncommutative geometer.
\\

The article is organized as follows: section \ref{graded-connected-algebras} 
contains general material on connected $\NN^r$-graded, their homological 
regularity properties, and how they behave with respect to a change of 
grading. In the last subsection we prove our main transfer result: if a 
connected $\NN^r$-graded has a filtration by finite dimensional graded 
subspaces and the associated graded ring has property $\chi$, then the 
original algebra inherits the regularity properties of the associated graded 
ring, see Theorem \ref{transfer}. 

In section \ref{qatv} we review the basic theory of affine semigroups and 
the regularity properties of quantum affine toric varieties from \cite{RZ2}. 
We then give an axiomatic description of algebras with a quantum toric 
degeneration for a fixed affine semigroup.

Section \ref{q-flag-schubert} contains a quick review of the definitions of 
quantum flag varieties and their Schubert subvarieties. We prove that these
algebras have a quantum toric degeneration and apply results from the previous
section to show that they are always AS-Cohen Macaulay in Theorem 
\ref{Schubert-deg}. 

We finish with a short section \ref{richardson}, where we revisit the results 
of our previous article \cite{RZ} in the context of this article.

%\newpage 
\section{Degeneration of graded connected algebras}
\label{graded-connected-algebras}
Let $r$ be a positive integer. Throughout this section $A$ denotes a 
noetherian connected $\NN^{r}$-graded $\k$-algebra. Here connected means that 
the homogeneous component of $A$ of degree $(0, \ldots, 0)$ is isomorphic to 
$\k$ as a ring, so the ideal generated by all homogeneous elements of non-zero 
degree is the unique maximal graded ideal of $A$; we denote this ideal by 
$\m$. Clearly $A/\m \cong \k$ as vector spaces, and whenever we consider $\k$ 
as $A$-bimodule, it will be with this structure.

\subsection*{Graded modules}
%\label{graded-connected-algebras}

\paragraph
\label{grmod-generalities}
We denote by $\ZZ^{r}\Mod A$ the category of $\ZZ^{r}$-graded $A$-modules with 
homogeneous morphisms of degree $0$. We review some general properties of this 
category; the reader is referred to \cite{NV}*{chapter 2} for proofs and 
details.

The category $\ZZ^r\Mod A$ has enough projectives and injectives, so we may 
speak of the graded projective and injective dimensions of an object $M$, 
which we denote by $\pd_A^{\ZZ^r} M$ and  $\id^{\ZZ^r}_A M$, respectively. We 
denote by $\ZZ^{r}\mmod A$ the subcategory of finitely generated 
$\ZZ^{r}$-graded $A$-modules. Since $A$ is noetherian $\ZZ^r \mmod A$ is an 
abelian category with enough projectives. 

For every object $M$ of $\ZZ^{r}\Mod A$ and every $\xi \in \ZZ^{r}$ we denote 
by $M_\xi$ the homogeneous component of $M$ of degree $\xi$. Also, we denote 
by $M[\xi]$ the object of $\ZZ^{r}\Mod A$ with the same underlying $A$-module 
as $M$ and with homogeneous components $M[\xi]_{\zeta} = M_{\zeta + \xi}$ for 
all $\zeta \in \ZZ^{r}$. If $f: N \to M$ is a morphism in $\ZZ^{r}\Mod A$ then 
the same function defines a morphism $f[\xi]:N[\xi] \to M[\xi]$. In this way 
we get an endofunctor $[\xi]: \ZZ^{r}\Mod A \to \ZZ^{r}\Mod A$, called the 
$\xi$-suspension functor; it is an autoequivalence, with inverse $[-\xi]$.

Given $N, M$ objects of $\ZZ^{r}\Mod A$ we set 
\begin{align*} 
  \GrHom_A(N, M) = \bigoplus_{\xi \in \ZZ^r} \Hom_{\ZZ^r\Mod A}(N,M[\xi]).  
\end{align*} 
This is a $\ZZ^{r}$-graded vector space, with its component of degree $\xi \in 
\ZZ^{r}$ equal to the space of homogeneous $A$-linear maps of degree $\xi$ 
from $N$ to $M$. For every $i \geq 0$ we denote by $\GrExt_A^i$ the $i$-th 
right derived functor of $\GrHom_A$. We point out that 
\[
  \GrHom_A(N[\xi], M) \cong \GrHom_A(N,M[-\xi]) \cong \GrHom_A(N,M)[-\xi],
\]
as $\ZZ^r$-graded vector spaces, and that these isomorphisms induce analogous 
ones for the coresponding right derived functors.

\paragraph
\label{ext-change-of-grading}
Given $\xi = (\xi_1, \ldots, \xi_r) \in \ZZ^r$ we set $|\xi| = \xi_1 + \cdots +
\xi_r$. Given a $\ZZ^{r}$-graded vector space $V$, we denote by $|V|$ the 
$\ZZ$-graded vector space whose $n$-th homogeneous component is 
\begin{align*}
  |V|_n = \bigoplus_{|\xi| = n} V_\xi.
\end{align*}
In particular $|A|$ is a connected $\NN$-graded algebra. Also, if $M$ is a 
$\ZZ^r$-graded $A$-module then $|M|$ is a $\ZZ$-graded $|A|$-module, and this 
assignation is functorial. Since $A$ is noetherian, \cite{RZ2}*{Proposition 
1.3.7} implies that for every $i \geq 0$ and any pair of $\ZZ^r$-graded 
modules $N, M$, with $N$ finitely generated, there is an isomorphism of 
$\ZZ$-graded modules
\[
  |\GrExt^i_A(N,M)| \cong \GrExt^i_{|A|}(|N|,|M|),
\]
natural in both variables. 

%\newpage 
\subsection*{Homological regularity properties}
In this subsection we discuss some homological properties a connected 
$\NN^r$-graded algebra may posses. Most of the material found in this section 
is standard for connected $\NN$-graded algebras.

\paragraph
\label{chi}
Let $M$ be a $\ZZ^r$-graded $A$-module. We say that $\chi(M)$ holds if for 
each $i \geq 0$ the graded vector space $\GrExt_A^i(\k, M)$ is finite 
dimensional, and say that the algebra $A$ has property $\chi$ if $\chi(M)$ 
holds for every finitely generated $\ZZ^r$-graded $A$-module $M$. Property 
$\chi$ was originally introduced in \cite{AZ}*{section 3} and plays a 
fundamental role in noncommutative algebraic geometry.

\paragraph
\label{torsion-functor}
Associated to $A$ and $\m$ there is a \emph{torsion functor} 
\begin{align*} 
  \GG_\m: \ZZ^{r}\Mod A &\to \ZZ^{r}\Mod A \\ 
  M &\longmapsto \{x \in M \mid \m^n x = 0 \mbox{ for } n \gg 0\}, 
\end{align*} 
which acts on morphisms by restriction and correstriction. The torsion functor 
is left exact, and for each $i \geq 0$ its $i$-th right derived functor is 
denoted by $H_\m^i$ and called the \emph{$i$-th local cohomology functor} of 
$A$. 

There exists a natural isomorphism
\begin{align*} 
\GG_\m \cong \varinjlim_n \GrHom_A(A/\m^n, -) 
\end{align*}
which by standard homological algebra extends to natural isomorphisms
\begin{align*} 
H^i_\m \cong \varinjlim_n \GrExt^i_A(A/\m^n, -) 
\end{align*} 
for all $i \geq 1$. The proof of this fact is completely analogous to the 
one found in \cite{BS}*{Theorem 1.3.8} for commutative ungraded algebras.

We denote by $A^\opp$ the opposite algebra of $A$, which is also
a connected $\NN^r$-graded algebra, and by $\m^\opp$ its maximal graded ideal. 
We write $\GG_{\m^\opp}$ and $H^i_{\m^\opp}$ for the corresponding torsion and 
local cohomology functors, respectively.

\paragraph
\label{depth-and-ldim}
Given an object $M$ of $\ZZ^{r}\Mod A$, its \emph{depth} and \emph{local 
dimension} are defined as
\begin{align*} 
  \depth_\m M &= \inf \{i \in \NN \mid \GrExt^i_A(\k, M) \neq 0\}, \\ 
  \ldim_\m M &= \sup \{i \in \NN \mid H^i_\m(M) \neq 0\},
\end{align*} 
respectively. The \emph{local cohomological dimension} of $A$, denoted by 
$\lcd_\m A$, is the supremum of the $\ldim_\m M$ with $M$ finitely generated.

\paragraph
\label{AS-reg}
The following definition is taken from \cite{RZ2}*{Definition 2.1.1}. It is an 
$\NN^r$-graded analogue of the definition of the AS-Cohen-Macaulay, 
AS-Gorenstein and AS-regular properties for connected $\NN$-graded algebras 
found in the literature, see for example the introduction to \cite{JZ}.
\begin{Definition*}
Let $A$ be a connected noetherian $\NN^{r}$-graded algebra.  
\begin{enumerate} 
\item $A$ is called \emph{AS-Cohen-Macaulay} if there exists $n \in \NN$ such 
that $H^i_{\m}(A) = 0$ and $H_{\m^\opp}^i(A) = 0$ for all $i \neq n$.

\item $A$ is called \emph{left AS-Gorenstein} if it has finite graded 
injective dimension $n$ and there exists $\ell \in \ZZ^{r}$, called the 
\emph{Gorenstein shift} of $A$, such that
\[ 
\GrExt_{A}^i(\k,A) \cong 
  \begin{cases} 
    \k[\ell] & \mbox{for } i = n, 
    \\ 0 & \mbox{for } i \neq n,
  \end{cases} 
\] 
as $\ZZ^{r}$-graded $A^\opp$-modules. We say $A$ is \emph{right} AS-Gorenstein 
if $A^\opp$ is left AS-Gorenstein. Finally $A$ is \emph{AS-Gorenstein} if $A$ 
and $A^\opp$ are left AS-Gorenstein, with the same injective dimensions and 
Gorenstein shifts.

\item $A$ is called \emph{AS-regular} if it is AS-Gorenstein, and its left and 
right graded global dimensions are finite and equal.
\end{enumerate} 
\end{Definition*} 

\paragraph
\label{invariance-by-grading}
The properties discussed in paragraphs \ref{chi} to \ref{AS-reg} are defined 
in terms of the category of $\ZZ^r$-graded $A$-modules. In 
\ref{ext-change-of-grading} we defined the algebra $|A|$, which is equal to $A$
as algebra but is endowed with a connected $\NN$-grading induced by the 
grading on $A$ and the group morphism $|\cdot|$. The maximal graded ideals of 
$A$ and $|A|$ coincide as vector spaces, so we may ask whether the fact that 
$A$ has property $\chi$, or finite local dimension, or the AS-Cohen-Macaulay 
property, etc., implies that $|A|$ has the corresponding property.

Let us say that a property $P$ does not depend on the grading of $A$ if the 
following holds: for every connected $\NN^t$-graded algebra $B$ with maximal 
ideal $\n$, which is isomorphic to $A$ as algebra through an isomorphism that 
sends $\n$ to $\m$, $A$ has property $P$ if and only if $B$ has property $P$. 
As shown in \cite{RZ2}*{Corollary 1.3.9}, the local dimension of $A$ does not 
depend on the grading of $A$. An analogous result is proved in 
\cite{RZ2}*{Remark 2.1.7} for the properties defined in \ref{AS-reg}. However, 
the situation is more delicate for property $\chi$. The following lemma shows 
that property $\chi$ is independent of the grading of $A$ under the hypothesis 
that $A$ has finite local dimension; we do not know whether this hypothesis 
can be eliminated.

\begin{Lemma*}
Suppose $\lcd_\m A < \infty$. Then the algebra $A$ has property $\chi$ if and 
only if $\chi(A)$ holds.
\end{Lemma*}
\begin{proof}
If $A$ has property $\chi$ then clearly $\chi(A)$ holds. To prove the opposite
implication, assume $\chi(A)$ holds. Recall from \ref{ext-change-of-grading} 
that for every $\ZZ^r$-graded $A$-module $M$ and for every $i \geq 0$ there 
exists a graded vector space isomorphism
\[
  |\GrExt^i_A(\k, M)|
    \cong \GrExt_{|A|}^i(|\k|, |M|),
\]
so $\chi(M)$ holds if and only if $\chi(|M|)$ holds; in view of this, the 
hypothesis implies $\chi(|A|)$ holds. 

Since $\ldim_{|\m|} |A| = \ldim_\m A < \infty$, we may apply 
\cite{RZ}*{Proposition 2.2.6} and conclude that $|A|$ has property $\chi$. 
From this it follows that $\chi(|M|)$ holds for every $\ZZ^r$-graded 
$A$-module $M$, and hence so does $\chi(M)$.
\end{proof}

\paragraph
\label{chi-and-local-cohomology}
We finish this subsection with a technical result on the relation between 
property $\chi$ and local cohomology. 

\begin{Lemma*}
For every $n \in \NN$, let $A_{\geq n}$ the ideal generated by all homogeneous 
elements of degree $\xi$ with $|\xi| \geq n$. Let $M$ be a finitely generated 
$\ZZ^r$-graded $A$-module such that $\chi(M)$ holds. Then for every $i \geq 0$ 
and every $t \in \ZZ$ there exists $n_0 \in \ZZ$ such that 
\begin{align*}
  \GrExt^i_A(A/A_{\geq n}, M)_\xi &\cong H^i_\m(M)_\xi
\end{align*}
for all $n \geq n_0$ and all $\xi \in \ZZ^r$ such that $|\xi| \geq t$.
\end{Lemma*}
\begin{proof}
Since $\m$ is finitely generated, say by elements $x_1, \ldots, x_r$ with 
degrees $\xi_1, \ldots, \xi_r$ such that $|\xi_i| \geq 1$, clearly $\m^n 
\subset A_{\geq n}$. Setting $l = \max \{|\xi_i| : 1 \leq i \leq r\}$ we 
obtain $A_{\geq ln} \subset \m^n$. Knowing this, the proof of 
\cite{BS}*{Proposition 3.1.1} easily adapts to show that for every $i \geq 0$ 
there exist natural isomorphisms
\begin{align*}
\varinjlim_n \GrExt^i_A(A/A_{\geq n}, -) 
  \cong \varinjlim_n \GrExt^i_A(A/\m^n, -) 
  \cong H^i_\m.
\end{align*} 
The statement of the lemma will follow if we show that for all $\xi$ as in the
statement, the homogeneous component of degree $\xi$ of the natural map
\begin{align*}
  \pi^n: \GrExt^i_A(A/A_{\geq n}, M) \to H^i_\m(M)
\end{align*}
is an isomorphism for $n \gg 0$.

Fixing $t$ as in the statement, \cite{AZ}*{Proposition 3.5 (1)} implies that
the natural map $\pi^n_d: \GrExt^i_{|A|}(|A|/|A_{\geq n}|, |M|)_d \to 
H^i_{|\m|}(|M|)_d$ is an isomorphism for all $d \geq t$ if $n$ is large 
enough. Now by \cite{RZ2}*{Propositions 1.3.7 and 1.3.8} there exist 
isomorphisms
\begin{align*}
\GrExt^i_{|A|}(|A|/|A_{\geq n}|, |M|)_d 
&\cong \bigoplus_{|\xi| = d}\GrExt^i_{A}(A/A_{\geq n}, M)_\xi;\\
H^i_{|\m|}(|M|)_d 
  &\cong \bigoplus_{|\xi| = d} H^i_\m(M)_\xi.
\end{align*}
Since the assignation $M \mapsto |M|$ is functorial, we also get that 
$\pi^n_d = \bigoplus_{|\xi| = d} \pi^n_\xi$. Thus for all $\xi$ such
that $|\xi| \geq t$, the map $\pi^n_\xi$ is an isomorphism if $n$ is large 
enough.
\end{proof}


%\newpage 
\subsection*{Transfer of regularity properties by degeneration}
\label{transfer-of-regularity}
In this subsection we prove that if $A$ has a filtration compatible with its 
grading, and the associated graded algebra has property $\chi$, then the 
regularity properties discussed in the previous subsection transfer from 
$\gr A$ to $A$. All undefined terms regarding filtrations can be found in 
\cite{VO}*{chapter I}.

\paragraph
\label{regular-filtration}
Recall that $A$ denotes a noetherian $\NN^r$-graded algebra. The general setup 
for the subsection is as follows: we assume that $A$ has a \emph{connected} 
filtration, that is an exhaustive filtration $\F = \{F_nA\}_{n \geq 
0}$, with $\k = F_0A \subset F_1A \subset \cdots \subset F_nA \subset \cdots 
\bigcup_{n \geq 0} F_nA = A$, such that each layer $F_nA$ is a finite 
dimensional graded vector space, and $F_nA \cdot F_m A \subset F_{n+m}A$ for 
all $n, m \in \NN$. For each $\xi \in \NN^r$ the homogeneous component $A_\xi$ 
has an induced filtration $\{F_nA_\xi\}_{n \geq 0}$, where $F_nA_\xi = F_nA 
\cap A_\xi$. Since $A_\xi$ is finite dimensional this filtration is finite, so 
the associated graded ring $\gr A$ is a connected and locally finite 
$\NN^{r+1}$-graded algebra. 

Given any $\ZZ^r$-graded $A$-module $M$ with a filtration whose layers are 
$\ZZ^r$-graded subspaces, we can construct the $\ZZ^{r+1}$-graded 
$\gr A$-module $\gr M$. If $M$ is any $\ZZ^r$-graded $A$-module then it can be 
endowed with such a filtration as follows: fix a graded subspace $N \subset M$ 
that generates $M$ over $A$, and for each $n \geq 0$ set $F_nM = (F_nA) N$. 
Any such filtration is called \emph{standard}, and is an exhaustive and 
discrete filtration by graded subspaces. If $M$ is finitely generated and $N$ 
is finite dimensional then the layers of this filtration are also finite 
dimensional. 

\paragraph
\label{P:ext-ss} 
The main tool used to transfer homological information from $\gr A$ to $A$ is 
a spectral sequence that we associate to any pair of $\ZZ^r$-graded 
$A$-modules $N, M$, which converges to $\GrExt^i_A(N,M)$ and whose first page 
consists of the homogeneous components of $\GrExt^i_{\gr A}(\gr N, \gr M)$. 
The proof is straightforward, but relies on several graded analogues of 
classical constructions for filtered rings. These constructions can be found 
in \cite{VO}*{Chapter I} and \cite{MR}*{Section 7.6}, and the proofs found in 
the references easily adapt to the graded context, so we use them without 
further comment.

In order to keep track of the extra component in the grading when passing to 
associated graded objects, we make a slight abuse of notation: given a 
$\ZZ^{r+1}$-graded vector space $V$, we denote by $V_{(\xi, p)}$ its 
homogeneous component of degree $(\xi_1, \ldots, \xi_r, p)$.

\begin{Proposition*} 
Let $A$ be an $\NN^r$-graded algebra with a connected filtration, and 
assume $\gr A$ is noetherian. Let $M, N$ be filtered $\ZZ^r$-graded 
$A$-modules, with $N$ finitely generated, and suppose that the filtration on 
$N$ is standard and the filtration on $M$ is discrete. Then
for every $\xi \in \ZZ^r$ there exists a convergent spectral sequence
\begin{align*} 
  E(N,M)_\xi: E_{p,q}^1 = \GrExt_{\gr A}^{-p-q}(\gr N, \gr M)_{(\xi,p)}
    &\Rightarrow \GrExt_A^{-p-q}(N,M)_\xi 
    &p,q \in \ZZ, 
\end{align*} 
such that the filtration of the vector spaces on the right hand side is finite.
\end{Proposition*}
\begin{proof}
By the $\ZZ^r$-graded version of \cite{MR}*{Theorem 6.17}, there exists a 
projective resolution $P^\bullet \to N$ by filtered projective $\ZZ^r$-graded 
$A$-modules with filtered differentials, such that the associated graded 
complex $\gr P^\bullet \to \gr N$ is a $\ZZ^{r+1}$-graded projective 
resolution of the $\gr A$-module $\gr N$. Using the filtration for the $\GrHom$
spaces defined in \cite{VO}*{section I.2}, the complex $\GrHom_A(P^\bullet, M)$
is a graded complex with a filtration by graded subcomplexes, whose 
differentials are filtered maps.

If we fix $\xi \in \ZZ^r$, the homogeneous component $\GrHom_A(P^\bullet, 
M)_\xi$ is a complex of filtered finite dimensional vector spaces. By 
\cite{W}*{5.5.1.2} there exists a spectral sequence with page one equal to
\begin{align*} 
  E_{p,q}^1 
  &= H_{p+q}\left(\frac{F_p \GrHom_A(P^\bullet,M)_\xi}
      {F_{p-1}  \GrHom_A(P^\bullet,M)_\xi}\right) 
    & p,q \in \ZZ,
\end{align*} 
that converges to   
\begin{align*} 
  H_{p+q}(\GrHom_{A}(P^\bullet,M)_\xi) \cong \GrExt^{-p-q}_A(N,M)_\xi.
\end{align*}
This last space is finite dimensional, and hence the filtration on it is 
finite. Thus we only need to prove that for each $p,q \in \ZZ$ there exists an 
isomorphism
\begin{align*} 
  H_{p+q}
  \left(
    \frac{F_p \GrHom_A(P^\bullet,M)_\xi}{F_{p-1} \GrHom_A(P^\bullet,M)_\xi}
  \right) 
    &\cong \GrExt_{\gr A}^{-p-q}(\gr N, \gr M)_{(\xi,p)}.
\end{align*} 
By \cite{VO}*{Lemma 6.4}, there exists an isomorphism of complexes 
$\phi(P^\bullet, M): \gr (\GrHom_A(P^\bullet, M)) \to \GrHom_{\gr A}(\gr 
P^\bullet, \gr M)$, which is defined explicitly in the reference. Direct 
inspection shows that the map $\phi(P^\bullet, M)$ is homogeneous, so looking 
at its component of degree $\xi$ we obtain an isomorphism
\begin{align*}
\frac{F_p \GrHom_A(P^\bullet,M)_\xi}{F_{p-1} \GrHom_A(P^\bullet,M)}_\xi 
  \cong \GrHom_{\gr A}(\gr P^\bullet, \gr M)_{(\xi,p)}.
\end{align*}
Since $\gr P^\bullet$ is a $\ZZ^{r+1}$-graded projective resolution of $\gr 
N$, we obtain the desired isomorphism by applying $H_{p+q}$ to both sides of 
the isomorphism.
\end{proof}

The following Corollary is an immediate consequence of the previous 
Proposition.
\begin{Corollary*}
Let $A$ be an $\NN^r$-graded algebra with a connected filtration, and 
assume $\gr A$ is noetherian. Let $M, N$ be filtered $\ZZ^r$-graded 
$A$-modules, with $N$ finitely generated, and suppose that the filtration on 
$N$ is standard and the filtration on $M$ is discrete. Then the following hold.
\begin{enumerate}[label=(\alph*)]
\item \label{I:dims} For each $i \geq 0$ and each $\xi \in \ZZ^{r}$
\[
 \dim_\k \GrExt^i_A(N,M)_\xi 
   \leq \sum_{p=-\infty}^\infty 
    \dim_\k \GrExt^i_{\gr A}(\gr N, \gr M)_{(\xi, p)}.
\]
  
\item \label{I:pd-id} $\pd_A^{\ZZ^r} N \leq \pd_{\gr A}^{\ZZ^{r+1}} \gr N$ 
and $\id_A^{\ZZ^r} M \leq \id_{\gr A}^{\ZZ^{r+1}} \gr M$. 

\item \label{I:chi} If $\chi(\gr M)$ holds then $\chi(M)$ holds.
\end{enumerate}
\end{Corollary*}

\paragraph
\label{ldim} 
We now prove a result that relates the local cohomology of a $\ZZ^r$-graded 
$A$-module $M$ with that of its associated graded module $\gr M$. We will do 
this by combining Proposition \ref{P:ext-ss} with Lemma 
\ref{chi-and-local-cohomology} and the formalism of the change of grading 
functors introduced in \cite{RZ}*{Section 1.3}. We recall the
relevant details. Given a group morphism $\phi: \ZZ^r \to \ZZ^t$, with $t \in 
\NN$, there exists a functor $\phi_!: \ZZ^r\Mod \k \to \ZZ^t \Mod \k$ that 
sends a $\ZZ^r$-graded vector space $V$ to the $\ZZ^t$-graded vector space 
$\phi_!(V)$, whose homogeneous component of degree $\zeta \in \ZZ^t$ is
\[
  V_\zeta = \bigoplus_{\phi(\xi) = \zeta} V_\xi.
\]
Notice that $\phi_!$ does not change the underlying vector space of its 
argument, only its grading. If $R$ is a $\ZZ^r$-graded algebra then $\phi_!(R)$
is a $\ZZ^t$-graded algebra, and if $M$ is a $\ZZ^r$-graded $R$-module then 
$\phi_!(M)$ is a $\ZZ^t$-graded $\phi_!(R)$-module with the same underlying 
$R$-module structure as $M$.

\begin{Corollary*}
Let $A$ be an $\NN^r$-graded algebra with a connected filtration, and 
assume $\gr A$ is noetherian. Let $M$ be a filtered $\ZZ^r$-graded $A$-module 
with a discrete filtration, and assume $\chi(\gr M)$ holds. Then for each 
$i \geq 0$ and each $\xi \in \ZZ^r$ 
\[
  \dim_\k H^i_\m(M)_\xi 
    \leq \sum_{p=- \infty}^\infty \dim_\k H^i_{\gr \m}(\gr M)_{(\xi, p)}.
\]
\end{Corollary*}
\begin{proof}
Let $\pi: \ZZ^{r+1} \to \ZZ^r$ be the projection to the first $r$-coordinates, 
and set $B = \pi_!(\gr A)$, so for every $\xi \in \ZZ^r$
\begin{align*}
B_\xi = \bigoplus_{p \in \ZZ} \gr A_{(\xi, p)} = \gr (A_\xi).
\end{align*}
Thus $B$ is a connected $\NN^r$-graded algebra. Set $\n = \pi_!(\gr \m)$, 
which is the maximal graded ideal of $B$, and set $\tilde M = \pi_!(\gr M)$.

By \cite{RZ2}*{Proposition 1.3.7} $\chi(\gr M)$ implies
$\chi(\tilde M)$ and by item \ref{I:chi} of Corollary \ref{P:ext-ss} it 
also implies $\chi(M)$, so we may apply Lemma \ref{chi-and-local-cohomology} 
to $M$ and $\tilde M$, and deduce that for given $\xi \in \ZZ^r$ and $i \geq 0$
the natural maps 
\begin{align*}
  \GrExt^i_A(A/A_{\geq n}, M)_\xi &\to H^i_\m(M)_\xi;\\
  \GrExt^i_B(B/B_{\geq n}, \tilde M)_\xi &\to H^i_\n( \tilde M)_\xi;
\end{align*}
are isomorphisms for $n \gg 0$.

Combining this with \cite{RZ2}*{Propositions 1.3.7 and 1.3.8}, we obtain a 
chain of isomorphisms
\begin{align*}
\bigoplus_{p \in \ZZ} 
  \GrExt^i_{\gr A}&\left(\frac{\gr A}{\gr (A_{\geq n})}, 
    \gr M \right)_{(\xi, p)} \\
  &\cong 
    \GrExt^i_B \left(B/B_{\geq n}, \tilde M \right)_\xi 
  \cong 
    H^i_\n(\tilde M)_\xi 
  \cong 
    \bigoplus_{p \in \ZZ} H^i_{\gr \m}(\gr M)_{(\xi, p)} 
\end{align*}
By definition $\gr(A/A_{\geq n}) \cong \gr A / \gr (A_{\geq n})$, so applying 
item  \ref{I:dims} of Corollary \ref{P:ext-ss} and taking $n \gg 0$ we obtain
\begin{align*}
\dim_\k H^i_\m(M)_\xi 
  &= \dim_\k \GrExt^i_A(A/A_{\geq n}, M)_\xi \\
  &\leq \sum_{p = -\infty}^\infty \dim_\k
  \GrExt^i_{\gr A}(\gr A / \gr (A_{\geq n}), M)_{(\xi, p)} \\
  &= \sum_{p = -\infty}^\infty \dim_\k H^i_{\gr \m}(\gr M)_{(\xi, p)}.
\end{align*}
\end{proof}

\paragraph
\label{transfer}
We are now ready to prove the main result of this section. 
\begin{Theorem*}
Suppose $A$ is a connected $\NN^r$-graded algebra endowed with a connected 
filtration, and that $\gr A$ is noetherian and has property $\chi$. Then the 
following hold.
\begin{enumerate}[label=(\alph*)]
\item $A$ has property $\chi$.
\item $\lcd_\m A \leq \lcd_{\gr \m} \gr A$.
\item If $\gr A$ is AS-Cohen-Macaulay, AS-Gorenstein or AS-Regular, so is $A$.
\end{enumerate}
\end{Theorem*}
\begin{proof}
The hypothesis that $\gr A$ is noetherian implies that $A$ is noetherian 
\cite{MR}*{1.6.9}. If $M$ is any finitely generated $\ZZ^r$-graded $A$-module, 
we may filter it using the procedure described in \ref{regular-filtration}. 
Since $\chi(\gr M)$ holds by hypothesis, item \ref{I:chi} of Corollary 
\ref{P:ext-ss} implies $\chi(M)$ holds, which proves that $A$ has property 
$\chi$. 

Item 2 follows from Corollary \ref{ldim}, as does the fact that if $\gr A$ is
AS-Cohen-Macaulay so is $A$. If $\gr A$ is AS-Gorenstein of injective 
dimension $n$ and Gorenstein shift $(\xi, p) \in \ZZ^{r+1}$, the algebra $A$ 
has injective dimension at most $n$ by item \ref{I:pd-id} of Corollary 
\ref{P:ext-ss}. Also, for each $\zeta \in \ZZ^r$ the spectral sequence 
$E(\k, A)_\zeta$ of Proposition \ref{P:ext-ss} degenerates at page $1$: it is 
zero if $\zeta \neq -\xi$, while for $\zeta = - \xi$ there is a single, one 
dimensional non-zero entry in the diagonal $-p-q = n$. Hence we obtain vector 
space isomorphisms
\begin{align*}
  \GrExt^i_A(\k, A)_\zeta 
    &\cong  \begin{cases}
      \k & \mbox{ if  $\zeta = -\xi$ and $i = n$;} \\
      0 & \mbox{otherwise.}
    \end{cases}
\end{align*}
Thus $\GrExt^i_A(\k,A) \cong \k[\xi]$ as $\ZZ^r$-graded vector spaces.
Any such isomorphism is also $A^\opp$-linear, so $A$ is left AS-Gorenstein of 
injective dimension $n$ and Gorenstein shift $\xi$. The same proof applies to 
show that $A$ is right AS-Gorenstein with the same injective dimension and 
Gorenstein shift, so $A$ is AS-Gorenstein. 

Finally, assume that $\gr A$ is AS-regular. Then $A$ is AS-Gorenstein, and item
\ref{I:pd-id} of Corollary \ref{P:ext-ss} implies that it has finite
left and right global dimensions. These dimensions are equal to the left and 
right projective dimensions of $\k$ as $A$-module \cite{RZ2}*{Lemma 2.1.5}, 
which in turn equal the left and right injective dimensions of $A$ and hence 
coincide, so $A$ is AS-regular.
\end{proof}


%\newpage 
\section{Quantum affine toric degenerations}
\label{qatv}
In this section we recall the homological properties of quantum affine toric 
varieties proved in \cite{RZ2}, and give necessary and sufficient conditions 
for a connected $\NN^r$-graded algebra $A$ to have a connected filtration such 
that the associated graded ring is a quantum positive affine toric variety. 

\subsection*{Quantum affine toric varieties}

\paragraph
\label{affine-semigroups}
Recall that an affine semigroup is a finitely generated monoid isomorphic to a 
subsemigroup of $\ZZ^r$ for some $r \in \NN$. An embedding of $S$ is an 
injective semigroup morphism $i: S \hookrightarrow \ZZ^r$ for some $r \in 
\NN$. By definition, every affine semigroup $S$ is commutative and 
cancellative, so it has a group of fractions, which we denote by $G(S)$, and 
the natural map from $S$ to $G(S)$ is injective. The group $G(S)$ is a 
finitely generated and torsion-free commutative group, so there exists a 
natural number $r$ such that $G(S) \cong \ZZ^r$ as groups; we refer to $r$ as 
the \emph{rank} of $S$ and denote it $\rk S$. We say that $i$ is a \emph{full 
embedding} if the image of $S$ generates $\ZZ^r$ as a group, in which case 
$r = \rank S$. Fixing an isomorphism $G(S) \cong \ZZ^{\rank(S)}$ we obtain a 
full embedding of $S$ in an obvious way. 

\paragraph
\label{interior}
Let $S$ be an affine semigroup. Fixing an embedding $i: S \hookrightarrow 
\ZZ^r$ we identify $S$ with its image and see $\ZZ^r$ as a subgroup of $\RR^r$ 
in the obvious way. The \emph{rational cone} generated by $S$ in $\RR^r$ is 
the set 
\[
  \RR_+S 
    = \left\{\sum_{i=1}^n r_i s_i \mid r_i \in \RR_{\geq 0}, 
              s_i \in S, n \in \NN\right\}.
\]
The \emph{real interior} of $S$ is defined as $\relint S = S \cap 
\RR_+S^\circ$, where $\RR_+S^\circ$ is the topological interior of $\RR_+ S$ 
as a subset of the vector space $\RR S$ endowed with the subspace topology 
induced from $\RR^r$. The relative interior is intrinsic to $S$ and does not 
depend on the chosen embedding \cite{BG}*{Remark 2.6}.

An affine semigroup $S$ of rank $r$ is called \emph{normal} if it verifies the 
following property: given $z \in G(S)$, if there exists $m \in \NN^*$ such 
that $mz \in S$, then $z \in S$. If we identify $S$ with a subset of $\ZZ^r$ 
through a full embedding and consider the real cone $\RR_+ S$ of $S$ inside 
$\RR^r$ in the obvious way, then Gordan's lemma \cite{BH}*{Proposition 6.1.2} 
states that $S$ is normal if and only if $S = \ZZ^r \cap \RR_+S$.

\paragraph
\label{hilbert-basis}
Let $S$ be an affine semigroup. An element $s \in S$ is called 
\emph{irreducible} if whenever $s = x + y$ with $x,y \in S$, the elements $x$ 
and $y$ are invertible. On the other hand $S$ is called \emph{positive} if its 
only invertible element is $0$. If $S$ is positive then the set of its 
irreducible elements is finite and generates $S$; for this reason it is called 
the \emph{Hilbert basis} of $S$. The fact that $S$ is positive also implies 
that there exists a full embedding $S \hookrightarrow \ZZ^r$ such that the 
image of $S$ is contained in $\NN^r$. It follows that a semigroup is positive 
if and only if there exists $r \geq 0$ such that $S$ is isomorphic to a 
finitely generated subsemigroup of $\NN^r$. Proofs of these results can be 
found in \cite{BG}*{pp. 54--56}. 

\paragraph
\label{presentation}
Let $S$ be a positive affine semigroup and let $\{s_1, \ldots, s_n\}$ be its 
Hilbert Basis. We denote by $\pi: \NN^n \to S$ the semigroup morphism defined 
by the assignation $e_i \mapsto s_i$ for each $1 \leq i \leq n$, where $e_i$ 
is the $i$-th element in the canonical basis of $\NN^n$. This map determines 
an equivalence relation in $\NN^n$ where $p \sim p'$ if and only if $\pi(p) = 
\pi(p')$; we denote this relation by $L(\pi)$. Clearly $L(\pi)$ is compatible 
with the additive structure of $\NN^n$, and hence the quotient $\NN^n / 
L(\pi)$ is a commutative monoid with the operation induced by addition in 
$\NN^n$, and there is an isomorphism of monoids $\NN^n/L(\pi) \cong S$. 

In general, an equivalence relation on $\NN^n$ closed under addition is called 
a \emph{congruence}. If $\rho$ is a subset of $\NN^n \times \NN^n$ then the 
congruence generated by $\rho$ is the smallest congruence containing the set 
$\rho$. Redei's theorem \cite{RGS}*{Theorem 5.12} states that every congruence 
in $\NN^n$ is finitely generated, i.e. there exists a finite set $\rho 
\subset \NN^n \times \NN^n$ that generates it; in particular there exists a 
finite set $P =  \{(p_1, p'_1), \ldots, (p_m,p'_m) \} \subset L(\pi)$ which 
generates $L(\pi)$. A \emph{presentation} of $S$ will be for us a pair 
$(\pi, P)$, where $\pi: \NN^n \to S$ is the map described above and $P$ is a 
finite generating set of the congruence $L(\pi)$. By the previous discussion 
every positive affine semigroup has a presentation.

\paragraph
\label{twisted-semigroup-algebras}
Let $S$ be a commutative semigroup.
A $2$-cocycle over $S$ is a function $\alpha: S \times S \to \k^\times$ such 
that $\alpha (s,s') \alpha(s+s',s'') = \alpha(s,s'+s'')\alpha(s',s'')$ for all 
$s,s',s'' \in S$. Given a $2$-cocycle $\alpha$ over $S$, the 
\emph{$\alpha$-twisted semigroup algebra} $\k^\alpha[S]$ is the associative 
$\k$-algebra whose underlying vector space has basis $\{X^s \mid s \in S\}$ 
and whose product over these generators is given by $X^s X^t = \alpha(s,t)
X^{s+t}$. This is a noncommutative deformation of the classical semigroup 
algebra $\k[S]$. 

\begin{Definition*}
Let $A$ be a connected $\NN^r$-graded algebra. We say that $A$ is a 
\emph{quantum positive affine toric variety} if there exist a positive affine 
semigroup $S$ and a $2$-cocycle $\alpha$ over $S$ such that $A$ is isomorphic 
to the twisted semigroup algebra $\k^\alpha[S]$, and for each $s \in S 
\setminus \{0\}$ the element $X^s$ is homogeneous of nonzero degree with 
respect to the $\NN^r$-grading of $\k^\alpha[S]$ induced by this isomorphism. 
In that case we refer to $S$ as the underlying semigroup of $A$.
\end{Definition*}

Let $r \geq 1$ and let $\psi: S \to \NN^r$ be a monoid morphism such that 
$\psi^{-1}(0) = \{0\}$. The twisted semigroup algebra $\k^\alpha[S]$ can be
endowed with a connected $\NN^r$-grading setting $\deg X^s = \psi(s)$ for each
$s \in S$. Conversely, any connected grading such that the elements of the 
form $X^s$ are homogeneous arises in this manner. In particular, if $A$ is a 
quantum positive toric variety with underlying semigroup $S$ then there is a 
corresponding monoid morphism $\psi: S \to \NN^r$, to which we will refer as 
the \emph{grading} morphism. 

\paragraph
\label{properties-of-qatv}
Quantum positive affine toric varieties were studied from the point of view of 
noncommutative geometry in \cite{RZ2}*{section 3}. The following is a summary 
of the results proved there.
\begin{Proposition*}
Let $S$ be a positive affine semigroup and let $A$ be a positive quantum toric 
variety with underlying semigroup $S$. Then the following hold.
\begin{enumerate}[label=(\alph*)]
\item $A$ is noetherian and integral.

\item $A$ has property $\chi$ and finite local dimension equal to the rank of 
$S$.

\item Suppose $S$ is normal. Then $A$ is AS-Cohen-Macaulay and a maximal order 
in its division ring of fractions. Furthermore, $A$ is AS-Gorenstein if and 
only if there exists $s \in S$ such that $\relint S = s + S$.
\end{enumerate}
\end{Proposition*}
\begin{proof}
This is proved in \cite{RZ2}*{section 3.2} in the case where the grading 
morphism $\psi$ is given by a full embedding of $S$ in $\NN^r$ for some $r 
\geq 0$. The proposition follows from the fact that all the properties 
mentioned in it are independent of the grading, see paragraph 
\ref{invariance-by-grading}.
\end{proof}

%\newpage 
\subsection*{Algebras with a quantum toric degeneration}
%\label{algebras-with-qatd}
Classically, a toric degeneration of an algebraic variety $V$ is a flat 
deformation of $V$ into a toric variety $T$. Since varieties inherit many good 
properties from their flat deformations, and positive affine toric varieties 
are well studied, toric degeneration are a standard method to study algebraic 
varieties. With this in mind we introduce a noncommutative analogue of toric 
degeneration.  

\begin{Definition}
Let $A$ be a connected $\NN^r$-graded algebra. We will say that 
$A$ has a \emph{quantum positive affine toric degeneration} if it has a 
connected filtration such that its associated graded ring is a quantum 
positive affine toric variety. We refer to the underlying semigroup of this 
quantum positive affine toric variety as the semigroup associated to the 
degeneration.
\end{Definition}
For the sake of brevity we will write ``quantum toric degeneration'', ommiting 
the adjectives ``positive affine''. In view of Theorem \ref{transfer} and 
Proposition \ref{twisted-semigroup-algebras}, an algebra with a quantum toric 
degeneration is noetherian, integral, has property $\chi$ and finite local 
dimension. Furthermore, we can determine whether $A$ is AS-Cohen-Macualay or 
AS-Gorenstein by studying the semigroup associated to the degeneration.

\paragraph
\label{s-phi-type}
A standard technique for proving toric degeneration of a variety is to find 
the structure of a Hodge algebra in the coordinate ring of the variety. We now 
introduce a noncommutative notion, inspired in the definition of Hodge 
algebras and its descendants such as classical and quantum algebras with a 
straightening law, which will play a similar role for the rest of this article.

\begin{Definition*}
Let $A$ be a noetherian connected $\NN^r$-graded algebra. Let $S$ be a 
positive affine semigroup and let $(\pi, P)$ be a presentation of $S$ as 
defined in \ref{presentation}, with $P = \{(p_i,p'_i) \mid 1 \leq i \leq m\}$. 
Let $\phi:~S~\to~\NN$ be a semigroup morphism such that $\phi^{-1}(0) = 
\{0\}$, and set $\tphi = \phi \circ \pi$.  

We say that the algebra $A$ is of $(S, \phi)$-type with respect to $(\pi, P)$
if the following hold. 
\begin{enumerate}
\item \label{generation}
$A$ is generated as algebra by a finite set of homogeneous elements $\{b_1, 
\ldots, b_n\}$ of the same cardinality as the Hilbert basis of $S$. We set
$\b^\xi = \displaystyle \prod_{i = 1}^n b_i^{\xi_i}$ for each $\xi \in \NN^n$.

\item \label{commutation}
For each $1 \leq i < j \leq n$ and each $\xi \in \NN^n$ such that $\tphi(\xi) 
< \phi(s_i + s_j)$, there exist $c_{i,j} \in \k^\times$ and $c^{i,j}_\xi \in 
\k$ such that
\begin{align*}
  b_j b_i = c_{i,j} b_i b_j + \sum_{\tphi(\xi) < \phi(s_i + s_j)} c^{i,j}_\xi
    \b^\xi.
\end{align*}

\item \label{straightening}
For each $1 \leq i \leq m$ and each $\xi \in \NN^n$ such that $\tphi(\xi) <
\tphi(p_i)$ there exist $d_i \in \k^\times$ and $d^i_\xi \in \k$ such that
    \begin{align*}
      \b^{p'_i} = d_i \b^{p_i} + \sum_{\tphi(\xi) < \tphi(p_i)} d^i_\xi \b^\xi.
    \end{align*}
\end{enumerate}
We say that $A$ is of $(S,\phi)$-type if there exists a presentation of $S$
such that $A$ is of $(S,\phi)$-type with respect to it.
\end{Definition*}

\begin{Remark*}
Since $\tphi(e_i) = \phi(s_i) > 0$ for all $i$, given $l \in \NN$ there exist 
finitely many $\xi \in \NN^n$ such that $\tphi(\xi) \leq l$ and so the sums on 
the right hand side of the formulas displayed in \ref{commutation} and 
\ref{straightening} are finite.
\end{Remark*}

\paragraph
\label{type-filtration}
If $A$ is an algebra of $(S,\phi)$-type we write $F_lA = \langle \b^\xi \mid 
\tphi(\xi) \leq l \rangle$ for each $l \in \NN$. The fact that $\tphi(e_i) > 0$
for all $i$ implies that this is a finite dimensional vector space, and that 
$F_0A = \k$. The following lemma shows that $\F = \{F_lA\}_{l \geq 0}$ is a 
connected filtration on $A$.

\begin{Lemma*}
Let $S$ be a positive affine semigroup. Let $A$ be a noetherian connected 
$\NN^r$-graded algebra, and assume that it is of $(S, \phi)$-type with respect 
to a presentation $(\pi, P)$. Then the following hold.
\begin{enumerate}
\item \label{filtration}
Given $\xi, \nu \in \NN^n$ there exists $c_{\xi, \nu} \in \k^\times$ such that 
$\b^\xi \b^\nu \equiv c_{\xi, \nu}\b^{\xi + \nu} \mod F_{\tphi(\xi + \nu) - 1}
A$. In particular $\F$ is a filtration on $A$.

\item \label{straight}
Given $(\xi, \nu) \in L(\pi)$ there exists $d_{\xi, \nu} \in \k^\times$ such 
that $\b^\xi \equiv d_{\xi, \nu} \b^\nu \mod F_{\tphi(\xi) - 1} A$.
\end{enumerate}
\end{Lemma*}
\begin{proof}
To prove item \ref{filtration} we proceed by induction on $\tphi(\xi + \nu)$, 
with the $0$-th step being obvious since $F_0A = \k$ is a subalgebra of $A$. 
Suppose that the result holds for all $l < \tphi(\xi + \nu)$ and let $i$ be 
the least integer such that $\xi_i \neq 0$, so $\b^\xi \b^\nu = b_i (\b^{\xi - 
e_i} \b ^\nu)$. Using the inductive hypothesis we obtain
\begin{align*}
b_i (\b^{\xi - e_i} \b^\nu) 
&= b_i \bigg( c_{\xi- e_i, \nu} \b^{\xi - e_i + \nu} 
  + \sum_{\tphi(\rho) < \tphi(\xi - e_i + \nu)} c_\rho \b^\rho \bigg)\\
&= c_{\xi- e_i, \nu} b_i \b^{\xi - e_i + \nu} 
  + \sum_{\tphi(\rho) < \tphi(\xi - e_i + \nu)} c_\rho b_i \b^\rho
\end{align*}
where the $c_\rho \in \k$ and $c_{\xi - e_i,\nu} \in \k^\times$. The 
inductive  hypothesis also implies each product $b_i \b^\rho$ lies in 
$F_{\tphi(\xi + \nu) -1} A$, so
\[
  b_i \b^{\xi - e_i} \b^\nu 
    \equiv c_{\xi- e_i, \nu} b_i \b^{\xi - e_i + \nu} 
    \mod F_{\tphi(\xi + \nu)-1}A.
\]
Now let $j$ be the least integer such that $\nu_j \neq 0$. If $i \leq j$ then 
$b_i \b^{\xi - e_i + \nu} = \b^{\xi + \nu}$ and we are finished; otherwise, 
using item 2 of Definition \ref{s-phi-type} and a similar argument as before, 
we obtain
\begin{align*}
  b_i \b^{\xi - e_i + \nu} 
    = (b_i b_j)(\b^{\xi - e_i + \nu - e_j}) 
    \equiv c_{j,i} (b_j b_i)(\b^{\xi - e_i + \nu - e_j}) 
    \mod F_{\tphi(\xi + \nu) - 1}A
\end{align*}
The same reasoning applied to the product $b_j(b_i\b^{\xi - e_i + \nu - e_j})$ 
shows that
\begin{align*}
  b_j (b_i \b^{\xi - e_i + \nu - e_j}) 
    &\equiv c_{e_i, \xi - e_i + \nu - e_j} b_j(\b^{\xi  + \nu - e_j}) \mod
  F_{\tphi(\xi + \nu) - 1}A.
\end{align*}
Since $j < i$, the definition of $j$ implies $b_j(\b^{\xi  + \nu - e_j}) = 
\b^{\xi + \nu}$, so the proof of item \ref{filtration} is complete.

We now prove item \ref{straight}. Set
\[
  T = \{(\xi, \nu) \in L(\pi) \mid \b^\xi \equiv c \b^{\nu} \mod
    F_{\tphi(\xi) -1}
    \mbox{ for some } c \in \k^\times\}.
\]
We will show that $L(\pi) \subset T$, which clearly implies the desired result.
By definition $T$ is an equivalence relation, and item \ref{filtration} 
implies it is a congruence on $\NN^n$. By item \ref{straightening} of 
Definition \ref{s-phi-type}, every pair $(p_i, p'_i) \in P$ lies in $T$. Since 
$L(\pi)$ is the smallest congruence containing $P$, we deduce that $L(\pi) 
\subset T$.
\end{proof}

\paragraph
\label{P:equivalent-qatd}
Let $S$ be a positive affine semigroup and let $(\pi, P)$ be a presentation of 
$S$. A \emph{section} of $\pi$ is a function $\tau: S \to \NN^n$ such that 
$\pi \circ \tau = \Id_S$, that is $s = \sum_i \tau(s)_i s_i$ for every $s \in 
S$. If $A$ is an algebra of $(S,\phi)$-type then Lemma \ref{type-filtration} 
implies that for any section $\tau$ of $\pi$ the set $\{\b^{\tau(s)} \mid s 
\in S\}$ spans $A$. We now show that an algebra has a quantum toric 
degeneration if and only if it is of $(S,\phi)$-type for an adequate morphism 
$\phi$ and the spanning set determined by any section is linearly independent.

\begin{Proposition*}
Let $S$ be a positive affine semigroup, and let $A$ be a noetherian connected 
$\NN^r$-graded algebra. The following statements are equivalent.
\begin{enumerate}
\item \label{has-qat-degeneration}
The algebra $A$ has a quantum toric degeneration with associated semigroup $S$.

\item \label{every-section}
For every presentation $(\pi, P)$ of $S$ there exists a semigroup morphism 
$\phi: S \to \NN$ such that $A$ is of $(S, \phi)$-type with respect to 
$(\pi, P)$, and for every section $\tau: S \to \NN^n$ of $\pi$ the set 
$\{\b^{\tau(s)} \mid s \in S\}$ is linearly independent.

\item \label{some-section}
There exist a presentation $(\pi, P)$, a semigroup morphism $\phi: S \to \NN$ 
and a section $\tau: S \to \NN^n$ of $\pi$ such that $A$ is of 
$(S, \phi)$-type with respect to $(\pi, P)$ and the set $\{\b^{\tau(s)} \mid s 
\in S\}$ is linearly independent.
\end{enumerate}
\end{Proposition*}
\begin{proof}
We first show that \ref{has-qat-degeneration} implies \ref{every-section}. By 
hypohtesis there exists a filtration by graded subspaces $\F = \{F_lA\}_{l 
\geq 0}$ such that $\gr_\F A \cong \k^\alpha[S]$ as $\NN^{r+1}$-graded 
algebras for some $2$-cocycle $\alpha: S \times S \to \k^\times$, with the 
grading on the twisted semigroup algebra given by a semigroup morphism 
$\psi: S \to \NN^{r+1}$ such that $\psi^{-1}(0) = \{0\}$. We identify 
$\gr_\F A$ with $\k^\alpha[S]$ through this isomorphism to simplify notation. 

Fix a presentation $(\pi, P)$ of $S$. For each $1 \leq i \leq n$ we choose 
homogeneous elements $b_i \in A$ such that $\gr b_i = X^{s_i}$. By definition 
of the product of an associated graded ring, for each $\xi \in \NN^n$ the 
element $\prod_{i = 1}^n (\gr b_i)^{\xi_i}$ equals either $\gr 
\left(\prod_{i = 1}^n b_i^{\xi_i}\right)$ or zero. Since $\k^\alpha[S]$ is an 
integral ring the last possibility cannot occur, so $\gr (\b^\xi)$ equals a 
nonzero multiple of $X^{\pi(\xi)}$. Thus if $\tau: S \to \NN^n$ is a section 
of $\pi$ then for each $s \in S$ there exists a nonzero constant $c_s$ such 
that $\gr b^{\tau(s)} \equiv c_s X^s$, and so the set $\{\gr \b^{\tau(s)} 
\mid s \in S\}$ is a basis of $\k^\alpha[S]$, which implies that 
$\{\b^{\tau(s)} \mid s \in S\}$ is a basis of $A$. This also proves that $A$ 
satisfies item \ref{generation} of Definition \ref{s-phi-type}.

Let $\phi: S \to \NN$ be the additive map given by $s \mapsto \psi(s)_{r+1}$; 
equivalently $\phi(s)$ is the minimal $l$ such that $\b^\xi \in F_l A$ for all 
$\xi \in \pi^{-1}(s)$. In particular $\phi(s_i) > 0$ since $F_0A = \k$. Also 
$F_lA = \vspan{\b^\xi \mid \tphi(\xi) \leq l}$, and since for each $\xi \in 
\NN^n$ there exists a nonzero constant $c_\xi$ such that $\gr b^\xi \equiv 
c_\xi \gr b^{\tau(\pi(\xi))}$ we actually have $F_lA = \vspan{\b^{\tau(s)} 
\mid \phi(s) \leq l}$. Finally, for each $1 \leq i,j \leq n$ and each $1 
\leq k \leq m$ there exist $c_{i,j}, d_k \in \k^\times$ such that
\begin{align*}
  \gr (b_j b_i) &= c_{i,j} \gr (b_i b_j),\\
  \gr \b^{p'_k} &= d_k \gr \b^{p_k}
\end{align*}
hold in $\gr_\F A$, which implies that items \ref{commutation} and 
\ref{straightening} of Definition \ref{s-phi-type} hold in $A$ for the morphism
$\phi$ we have just defined. Thus $A$ is of $(S,\phi)$-type, and we have 
proved \ref{has-qat-degeneration} implies \ref{every-section}.

We said in \ref{presentation} that every positive affine semigroup has a 
presentation so clearly \ref{every-section} implies \ref{some-section}. Let us 
see that \ref{some-section} implies \ref{has-qat-degeneration}. Define the 
filtration $\F = \{F_l A\}_{l \geq 0}$ as in \ref{type-filtration}. By item 2 
of Lemma \ref{type-filtration}, the set $\{\b^{\tau(s)} \mid \phi(s) \leq l\}$ 
generates $F_lA$ for each $l \in \NN$, and since by hypothesis it is linearly 
independent, it is a basis of $F_lA$. Hence $\gr_\F A$ is generated by $\{\gr 
\b^{\tau(s)} \mid s \in S\}$. Once again by Lemma \ref{type-filtration} for 
each $s,s' \in S$ there exist $\beta(s,s'), \alpha(s,s') \in \k^\times$ such 
that
\begin{align*}
(\gr \b^{\tau(s)})(\gr \b^{\tau(s')}) 
  = \beta(s,s') \gr \b^{\tau(s) + \tau(s')} 
  = \alpha(s,s') \gr \b^{\tau(s+s')}.
\end{align*}
Associativity of the product of $\gr_\F A$ implies that $\alpha: S \times S \to
\k^\times$ is a $2$-coycle, so we may consider the $\k$-linear map 
$\k^\alpha[S] \to \gr_\F A$ induced by the assignation $X^s \mapsto \gr 
\b^{\tau(s)}$, which is a multiplicative map. Since $S$ is positive we must 
have $\tau(0) = 0$, and hence $\alpha(s,0) = \alpha(0,s) = 1$ for all $s \in S$
which implies that our multiplicative map is unitaty and hence an isomorphism 
of $\k$-algebras. Furthermore, the elements $\gr \b^{\tau(s)}$ are
homogeneous, so this algebra is indeed a quantum positive affine toric variety.
\end{proof}

\paragraph
\label{lex-degeneration}
We now introduce a second notion related to quantum toric degenerations. 
Recall that a commutative semigroup $S$ is said to be well-ordered if there 
exists a well-order $<$ on $S$ compatible with the additive structure, i.e. 
such that for all $s,s',s'' \in S$ the inequality $s<s'$ implies $s+s'' < 
s'+s''$. 

\begin{Definition*}
Let $S$ be a commutative semigroup, and let $<$ be a well-order on $S$ 
compatible with the semigroup structure. Let $A$ be a connected $\NN^r$-graded 
algebra. An \emph{$(S,<)$-basis} for $A$ is an ordered basis $\{b_s \mid s 
\in S\}$ consisting of homogeneous elements, such that for all $s,s',s'' \in 
S$ with $s'' < s+s'$ there exist $c_{s,s'} \in \k^\times$ and $c_{s,s'}^{s''} 
\in \k$ such that
\[
  b_s b_{s'} = c_{s,s'} b_{s+s'} + \sum_{s'' < s+s'} c_{s,s'}^{s''} b_{s''}.
\]
\end{Definition*}

\paragraph
\label{S-order-bases}
Let $S$ be a commutative semigroup. Assume $<$ is a well-order on $S$ 
compatible with its additive structure, and let $A$ be an algebra. An 
$(S,<)$-filtration on $A$ is a collection of vector spaces $\F = \{F_sA \mid 
s \in S\}$, such that $F_sA \cdot F_{s'} A \subset F_{s+s'}A$ for all $s,s' 
\in S$, and such that $F_sA \subset F_{s'}A$ whenever $s < s'$. The standard 
notions related to $\NN$-filtrations translate easily to the context of 
$(S,<)$-filtrations. In particular, one can define the associated graded 
algebra $\gr_\F A$ by setting $F_{<s}A = \sum_{t<s} F_tA$ and taking
\[
  \gr_\F A = \bigoplus_{s \in S} \frac{F_sA}{F_{<s}A}.
\] 
As usual, for each element $a \in A$ we may define $\gr a$ as the image of $a$ 
in the quotient $F_sA / F_{<s}A$ where $s$ is the first element of $S$ such 
that $a \in F_sA$; notice that this element exists because $<$ is a 
well-order. The product can then be defined as in the $\NN$-filtered case, 
namely if $a,b \in A$ and $s,t$ are minimal elements such that $a \in F_sA, b 
\in F_tA$, then $(\gr a)(\gr b)$ equals the image of $ab$ in $F_{s+t}A / 
F_{<s+t}A$, which equals $\gr(ab)$ if $t+s$ is minimal with respect to the 
property that $ab \in F_{t+s}A$ and zero otherwise.

We now describe a mechanism to obtain well-orders on positive affine 
semigroups. Since $S$ is a positive affine semigroup it can be embedded in 
$\NN^r$ for some $r \geq 0$ through a monoid morphism $\iota: S \to \NN^r$. 
Now $\NN^r$ is a well-ordered semigroup with the lexicographic order, which is 
compatible with its additive structure, so we may pull-back the lexicographic 
order thourgh $\iota$ and thus obtain a well-order $<^\iota$ over $S$, which 
is also compatible with its additive structure. Notice that in this case $0$ 
is always the minimal element of $S$.

\begin{Lemma*}
Let $S$ be a positive affine semigroup, let $\iota: S \to \NN^r$ be an 
embedding, and let $< = <^\iota$ be the pull-back of the lexicographic order 
on $S$. Let $A$ be a noetherian $\NN^r$-graded connected algebra, and assume 
it has an $(S,<)$-basis $\B = \{b_s \mid s \in S\}$. Set $F_sA = \langle b_t 
\mid t \leq s \in S \rangle$ and $\F = \{F_sA \mid s \in S\}$. Also let 
$\{s_1, \ldots, s_n\}$ be the Hilbert basis of $S$ and set $b_i = b_{s_i}$ for 
all $1 \leq i \leq n$. The following hold.
\begin{enumerate}
\item The family $\F$ is an exhaustive $(S,<)$-filtration. Furthermore,
each quotient $F_sA / F_{<s}A$ is of dimension $1$ and $F_0A = \langle 1 
\rangle$.

\item There exists a $2$-cocycle $\alpha$ over $S$ such that $\gr_\F A$ is 
isomorphic as $S$-graded algebra to $\k^\alpha[S]$.

\item The algebra $A$ is generated by the set $\{b_i \mid 1 \leq i \leq n\}$.
\end{enumerate} 
\end{Lemma*}
\begin{proof}
The fact that $\F$ is an exhaustive $(S,<)$-filtration is an immediate 
consequence of the defintion of an $(S,<)$-basis. Also $F_{<s}A = \langle b_t 
\mid t < s\rangle$, so $F_sA / F_{<s}A$ is generated by the image of $b_s$ in 
the quotient. Finally, writing $1$ as a linear combination of the $b_s$
and using a leading term argument, it is easy to see that $1 \in F_0A$ and 
hence it must generate it. Notice that this implies that $b_0$ is a scalar, 
and without loss of generality we may assume that $b_0 = 1$.

Set $\alpha(s,s') = c_{s,s'}$ for each $s,s' \in S$. By item $1$, the set 
$\{\gr b_s \mid s \in S\}$ is a basis of $\gr_\F A$, and by definition of the 
product on the associated graded ring $(\gr b_s) (\gr b_{s'}) = c_{s,s'} \gr 
b_{s+s'}$. Associativity of the product in $\gr_\F A$ implies then that 
$\alpha$ is a $2$-cocycle over $S$, and furthermore the map $\gr_\F A \to 
\k^\alpha[S]$ sending $\gr b_s$ to $X^s$ is a multiplicative $S$-graded 
vector-space isomorphism. Since we are assuming that $b_0 = 1$, it follows 
that $\alpha(1,s) = \alpha(s,1) = 1$ for all $s \in S$ and hence our 
isomorphism preserves the unit, and is thus a ring isomorphism.
This proves item 2.

Finally, in order to prove that the $b_i$'s generate $A$ it is enough to show 
that each $b_s$ is in the algebra generated by these elements. Suppose this is 
not the case. Then, since $S$ is well-ordered by $<$, there exists a minimal 
$s$ such that $b_s$ is not in the algebra generated by the $b_i$'s. Take $\xi 
\in \NN^r$ such that $\pi(\xi) = s$. By the definition of the product of the 
associated graded ring, $\prod_i(\gr b_i)^{\xi_i}$ equals either $\gr (\b^\xi)$
or zero, and since $\gr_\F A \cong \k^\alpha[S]$ is integral the second 
possibility can not occur. Thus $\gr(\b^\xi)$ is a nonzero element of $\gr_\F 
A$ of degree $\pi(\xi)$, so item 1 of this lemma implies that $\gr b_s = 
c\gr(\b^\xi)$ for some $c \in \k^\times$, and hence $b_s  = c\b^\xi + 
\sum_{t<s} c_t b_t$. By the minimality of $s$ all the $b_t$'s appearing in the 
sum on the right hand side of the equation lie in the algebra generated by the 
$b_i$'s, and clearly so does $\b^\xi$, a contradiction. 
\end{proof}

\begin{Proposition}
\label{S-ordered-basis-degeneration}
Let $S$ be a positive affine semigroup, and let $A$ be a noetherian connected 
$\NN^r$-graded algebra. The algebra $A$ has a quantum affine toric 
degeneration with underlying semigroup $S$ if and only if there exists an 
embedding $\iota: S \to \NN^t$ such that $A$ has an ordered 
$(S, <^\iota)$-basis.
\end{Proposition}
\begin{proof}
Suppose $A$ has a quantum affine toric degeneration with underlying semigroup 
$S$. Then by Proposition \ref{P:equivalent-qatd} there exists a semigroup 
morphism $\phi: S\to \NN$ such that $A$ is of $(S, \phi)$-type, and we may 
choose any section $\tau: S \to \NN^n$ to obtain a basis $\mathcal B = 
\{\b^{\tau(s)} \mid s \in S\}$. Let $\rho: S \to \NN^{t}$ be an embedding and 
let $\iota: S \to \NN^{t+1}$ be defined as $\iota(s) = (\phi(s), \rho(s))$, 
which is an embedding of $S$ since $\rho$ is an embedding. Write $<$ for 
$<^\iota$, and notice that $\phi(s) < \phi(s')$ implies $s<s'$. By Lemma 
\ref{type-filtration}, for all $s,s' \in S$ and all $s''$ such that 
$\phi(s'') < \phi(s+s')$ there exist $c_{s,s'} \in \k^\times$ and 
$c_{s,s'}^{s''} \in \k$ such that
\[
  \b^{\tau(s)}\b^{\tau(s')} 
    = c_{s,s'}\b^{\tau(s+s')} 
      + \sum_{\phi(s'') < \phi(s+s')} c_{s,s'}^{s''} \b^{\tau(s'')},
\]
which implies $\mathcal B$ is an ordered $S$-basis with respect to $<$.

Now assume $A$ has an ordered $S$-basis with respect to some total order $<$ 
induced by an embedding $\iota: S \to \NN^t$. Since $<$ is the pull-back of 
the lexicographic order through an embedding, we might as well assume $S 
\subset \NN^t$ and that $<$ is the lexicographic order. By the previous lemma, 
we already know that the $b_i$'s generate $A$, so all that is left to do is to 
prove the existence of an additive map $\phi:S \to \NN$ and that of the 
desired relations.

Recall that using the $(S,<)$-filtration $\F$ defined in the previous lemma, 
we proved that $\gr_\F A \cong \k^\alpha[S]$ for some $2$-cocycle $S$. This 
implies that for all $1 \leq i < j \leq n$ and all $\xi \in \NN^n$ such that 
$\pi(\xi) < s_i +s_j$ there exist $c_{i,j} \in \k^\times$ and $c^{i,j}_\xi 
\in \k$ such that
\begin{align*}
b_j b_i 
  = c_{i,j} b_i b_j 
    + \sum_{\pi(\xi) < \pi(s_i + s_j)} c^{i,j}_\xi \b^\xi,
\end{align*}
and for each $1 \leq i \leq m$ and each $\xi \in \NN^n$ such that $\pi(\xi) < 
p_i$ there exist $d_i \in \k^\times$ and $d^i_\xi \in \k$ such that
\begin{align*}
  \b^{p'_i} = d_i \b^{p_i} + \sum_{\pi(\xi) < p} d^i_\xi \b^\xi.
\end{align*}
Let $C \subset S$ be the set consisting of the following elements:
\begin{itemize}
\item all $s_i + s_j$ with $1 \leq i<j\leq n$;
\item all $\pi(p_i)$ with $1 \leq i \leq m$;
\item all $\pi(\xi)$ such that $c^{i,j}_\xi \neq 0$ for some $1 \leq i < j 
  \leq n$;
\item and all $\pi(\xi)$ such that $d^i_\xi \neq 0$ for some $1 \leq i \leq m$.
\end{itemize}  
The set $C$ is finite and hence is contained in a cube $[0,N+1]^t$ for $N$ 
large enough, so the morphism $\phi: \NN^t \to \NN$ mapping $e_i$ to $(N+1)^i$ 
sends each $c \in C$ to the natural number having $N+1$-adic expansion $c$. In 
particular $\phi$ respects the restriction of the lexicographic order to $C$, 
and thus $A$ is of $(S,\phi)$-type. 

Let $\tau: S \to \NN^n$ be any section of $\pi$. The algebra $\gr_\F A$ has a 
natural $S$-grading, and for each $s \in S$ the element $\gr \b^{\tau(s)}$ is 
of degree $s$. As we have already observed, this is a non-zero element so the 
set $\{\gr \b^{\tau(s)} \mid s \in S\}$ is a basis of $\gr_\F A$, which 
implies that $\{\b^{\tau(s)} \mid s \in S\}$ is a basis of $A$. Thus by 
Proposition \ref{P:equivalent-qatd} $A$ has a quantum affine toric 
degeneration with underlying semigroup $S$.
\end{proof}

\begin{Remark*}
The trick of turning the $S$-filtration into an $\NN$-filtration using 
$N+1$-adic expansions is due to Caldero \cite{C}*{Lemma 3.2}. A similar though 
less general version of this idea appears in \cite{GL} and \cite{RZ}.
\end{Remark*}

\paragraph
We finish this section with an easy consequence of Lemma 
\ref{type-filtration}. It will not be used in the sequel, but we include it 
for completeness.
\begin{Proposition*}
Let $S$ be a positive affine semigroup and $A$ a noetherian connected $\NN^r$-
graded algebra. If $A$ is of $(S,\phi)$-type for some monoid morphism $\phi: 
S \to \NN$, and there exists some section $\tau$ of $\pi$ such that the set 
$\{\b^{\tau(s)} \mid s \in S\}$ is linearly independent, then the relations 
given in items \ref{commutation} and \ref{straightening} of Definition 
\ref{s-phi-type} give a presentation of $A$.
\end{Proposition*}
\begin{proof}
Since $A$ is of $(S,\phi)$-type, it is generated as algebra by homogeneous 
elements $b_1, \ldots, b_n$, and there exist constants $c_{i,j}, c^{i,j}_\xi, 
d_i, d^i_\xi$ such that $A$ complies with Definition \ref{s-phi-type}. 
Furthermore, the relations described in items \ref{commutation} and 
\ref{straightening} of this definition are homogeneous.

Let $B$ be the free algebra generated by $X_1, \ldots, X_n$ and let $I$ be the 
ideal of $B$ generated by the elements
\begin{align*}
  X_j X_i &- c_{i,j} X_i X_j - \sum_{\tphi(\xi) < \phi(s_i + s_j)} c^{i,j}_\xi
  X^\xi, 
    & \mbox{for } 1 \leq i < j \leq n;\\
  X^{p'_i} &- d_i X^{p_i} - \sum_{\tphi(\xi) < \tphi(p_i)} d^i_\xi X^\xi, 
    &\mbox{for } 1 \leq i \leq m;
\end{align*}
where $X^\xi = X_1^{\xi_1} X_2^{\xi_2} \cdots X_n^{\xi_n}$ for each $\xi \in 
\NN^n$. We put an $\NN^r$ grading on $B$ by setting $\deg X_i = \deg b_i$, and 
this induces an $\NN^r$ grading on $B/I$. Since $B$ is a free algebra the 
assignation $X_i \mapsto b_i$ induces a morphism of graded algebras $B \to A$, 
which factors through $B/I$. We thus obtain a morphism of $\NN^r$-graded 
algebras $f: B/I \to A$.

We denote by $Y_i$ the image of $X_i$ in $B/I$. Cleary $B/I$ is an $(S,\phi)$-
algebra, and the algebra map $f: B/I \to A$ sends $Y_i$ to $b_i$ for all $i$. 
Since $f(Y^{\tau(s)}) = \b^{\tau(s)}$, the set $\{Y^{\tau(s)} \mid s \in S\}$ 
is linearly independent and hence a basis of $B/I$. Thus $f$ maps a basis onto 
a basis, so it is an isomorphism.
\end{proof}


%\newpage 
\section{Quantum affine toric degeneration of quantum Schubert varieties}
\label{q-flag-schubert}
We apply the results in the previous section to study Schubert varieties 
of quantum flag varieties. We recall the definitions of quantum flag and 
Schubert varieties with some detail in order to establish notation. We then
adapt an argument due to P. Caldero to show that these algebras have 
$(S,<)$-bases for adequate semigroups $S$. The main ingredient in the 
construction of these bases is the canonical or global basis of $U_q^-(\g)$ 
discovered independently by Lusztig and Kashiwara. The semigroup arises out
of the string parametrization of this basis.

\subsection*{Quantum flag and Schubert varieties}

\paragraph
\label{lie-algebra-notation}
Let $\g$ be a complex semisimple Lie algebra. We denote by $\Phi$ the root
system of $\g$ with respect to a fixed Cartan subalgebra, and by $\ZZ\Phi$ its 
root lattice. We also fix a basis $\Pi \subset \Phi$ of positive roots, and 
write $P = \{\varpi_\alpha \mid \alpha \in \Pi\}$ for the set of 
corresponding fundamental weights. We denote by $\Lambda$ the weight lattice 
$\sum_{\alpha \in \Pi} \ZZ \varpi_\alpha$ and by $\Lambda^+$ the set of 
dominant weights $\sum_{\alpha \in \Pi} \NN \varpi_\alpha$. 

Let $W$ be the Weyl group of $\g$, and $s_\alpha \in W$ the reflection 
corresponding to $\alpha \in \Pi$. Given an element $w \in W$ we denote 
its length by $\ell(w)$, and set $N$ to be the length of $w_0$, the longest 
element of $W$. A \emph{decomposition} of $w \in W$ is a word on the 
generators $s_\alpha$ that equals $w$ in $W$. The decomposition is 
\emph{reduced} if it is of minimal length, i.e. its length equals $\ell(w)$. 
We denote by $(-,-)$ the standard $W$-invariant pairing between $\ZZ\Phi$ and 
$\Lambda$, and write $\langle \lambda, \alpha\rangle = 
\frac{2(\lambda,\alpha)}{(\alpha,\alpha)}$ for all $\lambda \in \Lambda$ and
$\alpha \in \Phi$, so if $\lambda = \sum_{\alpha \in \Pi} r_\alpha 
\varpi_\alpha$ then $\langle \lambda, \alpha \rangle = r_\alpha$.


\paragraph
\label{quantized-enveloping-algebra}
Fix $q \in \k^\times$. Let $U_q(\g)$ be the quantum enveloping algebra of 
$\g$; this is an algebra generated by elements $F_\alpha, E_\alpha, K_\alpha$
for $\alpha \in \Pi$, with the relations given in \cite{Jan}*{Definition 4.3}. 
We denote by $U_q^+(\g), U_q^-(\g)$ the subalgebras generated by the 
$F_\alpha$'s and the $E_\alpha$'s, which are respectively called the positive 
and negative parts of $U_q(\g)$ \cite{Jan}*{4.4}. As shown in 
\cite{Jan}*{Proposition 4.11}, these are all Hopf algebras.

If $q$ is not a root of unity then by \cite{Jan}*{chapter 5} for each $\lambda 
\in \Lambda^+$ there is an irreducible highest-weight representation of 
$U_q(\g)$ of type $\mathbf{1}$, which we denote by $V_q(\lambda)$. Each 
$V_q(\lambda)$ decomposes as the direct sum of weight spaces $\bigoplus_{\mu 
\in \Lambda} V_q(\lambda)_\mu$; the dimensions of the weight spaces are the 
same as the corresponding representation over $\g$, so the Weyl character 
formula holds for these representations, see \cite{Jan}*{5.15}. 

\paragraph
Let $G$ be the connected and simply connected Lie group corresponding to $\g$. 
Since $U_q(\g)$ is a Hopf algebra, its dual $U_q(\g)^*$ is an algebra 
with convolution product induced by the coproduct of $U_q(\g)$. There is a map 
$V_q(\lambda)^* \ot V_q(\lambda) \to U_q(\g)^*$ defined by
sending $\phi \ot v \in V_q(\lambda)^* \ot V_q(\lambda)$ to the linear 
functional $c^\lambda_{\phi, v}$, which assigns to each $u \in U_q(\g)$ the 
scalar $c^\lambda_{\phi, v}(u) = \phi(uv)$. Functionals of the form 
$c^\lambda_{\phi, v}$ are called \emph{matrix coefficients}. The $\k$-linear 
span of the matrix coefficients is a subalgebra of $U_q(\g)^*$ denoted by 
$\O_q[G]$, called the quantized algebra of coordinate functions over the group 
$G$ \cite{Jan}*{7.11}.

\paragraph
\label{q-full-flag-varieties}
Quantum analogiues of flag varieties and their Schubert subvarieties were 
introduced by Soibelman in \cite{S} and by Lakshmibai and Reshetikhin in 
\cite{qLR}; we review their definition. 
We assume that $q$ is not a root of unity. Fix a maximal Borel subgroup $B$ of
$G$. The \emph{full flag variety} associated to $G$ is $G/B$. Let 
$C^+_q(\lambda)$ be the vector space of matrix coefficients of the form 
$c^\lambda_{\phi,v_{\lambda}}$ in $U_q(\g)^*$, where $v_\lambda$ is a highest 
weight vector in $V_q(\lambda)$, and set 
\[ 
  \O_q[G/B] = \bigoplus_{\lambda \in \Lambda^+} C^+_q(\lambda) \subset \O_q[G] 
    \subset U_q(\g)^*.  
\] 
This is called the \emph{quantum full flag variety} of $G$. The product of two 
matrix coefficients in $\O_q[G/B]$ is again in $\O_q[G/B]$, and its 
decomposition as a direct sum gives $\O_q[G/B]$ the structure of a 
$\Lambda^+$-graded algebra. 

Let $I$ be a subset of the set of fundamental weights and set $\mathcal J(I) = 
\sum_{\varpi \notin I} \NN \varpi$. Denote by $W_I \subset W$ the subgroup 
generated by the reflections $s_\alpha$ with $\varpi_\alpha \in I$, and for 
each class in $W/W_I$ pick a representative of minimal length. We denote by 
$W^I$ the set of these representatives. Since the Weyl character formula 
holds, for each $w \in W$ and each $\lambda \in \mathcal J(I)$ the vector 
space $V_q(\lambda)_{w\lambda}$ has dimension $1$. The \emph{Demazure module} 
$V_q(\lambda)_{w}$ is the $U_q^+(\g)$-submodule of $V_q(\lambda)$ generated by 
a vector of weight $w\lambda$ in $V_q(\lambda)$.

The set $I$ determines a Lie subalgebra $\mathfrak p \subset \g$, and a 
parabolic subgroup $P_I \subset G$. The variety $G/P_I$ is the corresponding 
generalized flag variety. To these data we associate the $\Lambda^+$-graded 
subalgebra of $\O_q[G/B]$ 
\[ 
  \O_q[G/P_I] = \bigoplus_{\lambda \in \mathcal J(I)} C_q^+(\lambda) 
\] 
called the \emph{quantum partial flag variety associated to $I$}. 

Given vector spaces $V_2 \subset V_1$, we denote by $V_2^\perp$ the set of 
linear functionals over $V_1$ which are zero on $V_2$. For every $w \in W^I$ 
the vector space 
\[
  J_w^I = \bigoplus_{\lambda \in \mathcal J(I)} \vspan{c_{\phi,v_\lambda}^\lambda
  \in C_q^+(\lambda) \mid \phi  \in V_q(\lambda)_w^\perp } \subset \O_q[G/P_I] 
\] 
is an ideal of $\O_q[G/P_I]$ called the \emph{Schubert ideal} associated to 
$w$. The quotient algebra $\O_q[G/P_I]_w = \O_q[G/P_I]/J^I_w$ is called the 
\emph{quantum Schubert variety} associated to $w$.

%\newpage 
\subsection*{Degeneration of quantum Schubert varieties}
%\label{qtad-flag}
In this subsection we assume that $\k$ is of characteristic zero and that 
$q \in \k^\times$ is trascendental over $\QQ$. Our aim is to show that under 
these conditions quantum Schubert varieties have quantum affine toric 
degenerations. In order to do so we work for a moment over the field $\QQ(v)$, 
where $v$ is an indeterminate over $\QQ$, and consider the $\QQ(v)$-algebra 
$U = U_v(\g)$. The general case will follow by extension of scalars. 

We now review Caldero's argument on the existence of an $(S,<_{\lex})$-basis 
of $\O_q[G/B]$, and its natural extension to arbitrary partial flag and 
Schubert varieties. Since he was interested in classical flag varieties, 
Caldero works with a large base field $\CC(q)$ that allows him to specialize 
at $q=1$ and still get algebraic varieties over the complex numbers. We give a
simplified version of his argument which works over $\QQ(v)$.

\paragraph
Fix as our base field $\QQ(v)$, and set $U=U_v(\G)$. We denote by 
$U^+$ and $U^-$ the algebras $U^+_v(\g)$ and $U^-_v(\g)$, respectively. 

Let $\A = \ZZ[v, v^{-1}] \subset \QQ(v)$. For each $n \in \NN$ we write
$[n] = \frac{v^n - v^{-n}}{v-v^{-1}}$, and $[n]! = [1][2] \cdots [n]$. For 
each $\alpha \in \Pi$ we denote by $F_\alpha^{(n)} = \frac{1}{[n]!} 
F_\alpha^n$ and $E_\alpha^{(n)} = \frac{1}{[n]!} E_\alpha^n$. 
The algebra $U$ has an $\A$-form which we denote by $U_\A$; it is the 
$\A$-subalgebra of $U$ generated by the elements of the form $F_\alpha^{(n)}, 
E_\alpha^{(n)}, K_\alpha^{\pm 1}$ for all $\alpha \in \Pi$ and all $n \geq 0$ 
\cite{Jan}*{11.1}. This is a subring of $U$. The algebra $U^+$, resp. $U^-$,
also has an $\A$-form which we denote by $U_\A^+$, resp. and $U_\A^-$. It is 
generated by all the $E_\alpha^{(n)}$, resp. $F^{(n)}_\alpha$, with $n \geq 0$.
These $\A$-forms are compatible with the weight decomposition of $U$. 
By construction $U \cong \QQ(v) \ot_\A U_\A$, and analogous results hold for 
$U_\A^+$ and $U_\A^-$. 

The highest weight modules $V_v(\lambda)$ have corresponding $\A$-forms which 
we denote by $V_\A(\lambda)$; these are obtained by fixing a highest weight 
vector $v_\lambda$ and looking at the $\A$-lattice $U^-_\A v_\lambda \subset 
V_v(\lambda)$. These $\A$-forms are compatible with the weight decompositions 
of the original objects, see \cite{Jan}*{chapters 9,10}. Now let $\lambda$ be 
a dominant weight and fix $w \in W$. The $\A$-form of the Demazure module 
$V_q(\lambda)_{w}$ is defined as $V_\A(\lambda)_{w} = U_\A^+ 
V_\A(\lambda)_{w\lambda}$.

\paragraph
\label{T:A-crystal-basis} 
The algebra $U_\A^-$ has a homogeneous $\A$-basis, called the \emph{canonical}
or \emph{global} basis of $U^-$, discovered independently by Lusztig and 
Kashiwara. Its contruction is the subject of \cite{Jan}*{chapters 9 - 11}, and 
we will use the notation from this source to recapitulate some relevant facts. 

Set $A_0 \subset \QQ(v)$ to be the ring of rational functions without a pole 
at $0$. For each $\alpha \in \Pi$ define the operators $\tilde F_\alpha: 
U^- \to U^-$ as in \cite{Jan}*{10.2}, and let $\L(\infty)$ be the 
$A_0$-lattice generated by all elements of the form $\tilde F_{\alpha_1} 
\tilde F_{\alpha_2} \cdots \tilde F_{\alpha_r}(1)$; by definition these are 
weight elements, so setting $\L(\infty)_{-\nu} = \L(\infty) \cap U^-_{-\nu}$ 
for each $\nu \in \ZZ \Phi$ with $\nu \geq 0$, we get $\L(\infty) = 
\bigoplus_{\nu \geq 0} \L(\infty)_{-\nu}$, and furthermore each 
$\L(\infty)_{-\nu}$ is a finitely generated $A_0$-module that generates 
$U^-_\nu$ over $\QQ(v)$. 

Set
\begin{align*}
\B(\infty)_{-\nu} 
  &=  \left\{\tilde F_{\alpha_1} \tilde F_{\alpha_2} \cdots 
    \tilde F_{\alpha_r}(1) + q\L(\infty) \mid \sum_{i} \alpha_{i} = \nu\right\}
    \subset \L(\infty)_{-\nu}/q\L(\infty)_{-\nu},
\end{align*}
and set $\B(\infty) = \bigsqcup_{\nu \geq 0} \B(\infty)_{-\nu}$. Although it is
not obvious, $\B(\infty)_{-\nu}$ is a basis of $\L(\infty)_{-\nu} / 
q \L(\infty)_{-\nu}$ \cite{Jan}*{Proposition 10.11}; this is the \emph{crystal}
basis of $U^-_{-\nu}$ at $q = 0$. It turns out that each 
$b \in \B(\infty)_{-\nu}$ has a unique lift $G(b) \in \L(\infty)_{-\nu} \cap 
U_\A^-$, which is invariant under the action of certain automorphisms 
of $U$ \cite{Jan}*{Theorem 11.10 a)}. The set $G(\B)$ of all $G(b)$ with $b 
\in \B(\infty)$ is the global basis of $U_\A^-$.

Let $w \in W$, and let $w = s_{\alpha_1}\cdots s_{\alpha_r}$ be a reduced 
decomposition of $w$. Set $\B_w(\infty)$ as the set of elements in $\B(\infty)$
of the form $\tilde F_{\alpha_1}^{k_1} \tilde F_{\alpha_2}^{k_2} \cdots \tilde 
F_{\alpha_r}^{k_r}(1) + q\L(\infty)$ with $k_j \geq 0$. This set does not 
depend on the decomposition of $w$ \cite{K1}*{Proposition 3.2.5}.

\begin{Theorem*}
Let $\nu \in \ZZ \Phi$ with $\nu \geq 0$.
\begin{enumerate}[label=(\alph*)]
\item The set $\{G(b) \mid b \in \B(\infty)_{-\nu}\}$ is an $\A$-basis of 
$(U_\A^-)_{- \nu}$.

\item Let $\lambda$ be a dominant weight and let $v_\lambda$ be a highest 
weight vector of $V_\A(\lambda)$. The set $\{G(b)v_\lambda \mid b \in 
\B(\infty)_{-\nu}\} \setminus \{0\}$ is an $\A$-basis of 
$V_\A(\lambda)_{\lambda - \nu}$. Furthermore, if $G(b)v_\lambda = 
G(b')v_\lambda \neq 0$ then $b = b'$.

\item Let $w \in W$. The set $\{G(b)v_\lambda \mid b \in \B_w(\infty)\} 
\setminus \{0\}$ is an $\A$-basis of $V_\A(\lambda)_{w}$. 
\end{enumerate}
\end{Theorem*}
\begin{proof}
The first two items are part of \cite{Jan}*{Theorem 11.10}. The third is 
\cite{K1}*{Proposition 3.2.5 (vi)}.
\end{proof}

\paragraph
\label{q-var-bases}
We now use the crystal basis to produce bases for quantum Schubert varieties.
For each dominant weight $\lambda$ we fix a highest weight vector $v_\lambda 
\in V_v(\lambda)$ and write $\B(\lambda) = \{b \in \B(\infty) \mid G(b)
v_\lambda \neq 0\}$. By the previous theorem the set $\{G(b) v_\lambda \mid b 
\in \B(\lambda)\}$ is a basis of $V_v(\lambda)$, so we can take its dual 
basis. Given $b \in \B(\lambda)$, we denote by $b_\lambda^*$ the unique 
element of $V_v(\lambda)^*$ such that $b^*_\lambda(G(b') v_\lambda) = 
\delta_{b,b'}$ for all $b' \in \B(\lambda)$. Thus to each element $b \in 
\B(\lambda)$ we can associate the matrix coefficient $b_\lambda = 
c^{\lambda}_{b^*_\lambda,v_\lambda} \in C^+_v(\lambda)$, and the set 
$\{b_\lambda \mid b \in \B(\lambda)\}$ is a basis of $C^+_v(\lambda)$. Since 
the quantum flag variety is the direct sum of all these spaces with $\lambda$ 
running over all dominant weights, we obtain a basis of $\O_v[G/B]$ as defined
in \ref{q-full-flag-varieties} by taking
\[
  \flagbasis = \{b_\lambda \mid b \in \B(\lambda), \lambda \in \Lambda^+\}.
\]

If $I$ is a subset of the fundamental weights, then we obtain a basis of the 
partial flag variety $\O_v[G/P_I]$ by taking
\[
  \flagbasis_I = \{b_\lambda \mid b \in \B(\lambda), \lambda \in \mathcal 
    J(I)\}.
\]
Finally, the third item of the previous theorem implies that the ideal $J^I_w$ 
defined in \ref{q-full-flag-varieties} is spanned over $\QQ(v)$ by all 
elements of the form $b_\lambda$ with $b \in \B(\lambda) \setminus 
\B_w(\infty)$. Thus we can obtain a basis for the quantum Schubert variety 
$\O_v[G/P_I]_w$ by taking the image of
\[
  \schubertbasis_I(w) 
    = \{b_\lambda \mid b \in \B(\lambda) \cap \B_w(\infty), \lambda \in 
    \mathcal J(I) \}.
\]
in the quotient.

\paragraph
\label{Littelman-parametrizations}
The operators $\tilde E_\alpha, \tilde F_\alpha$ induce operators $\tilde 
E_\alpha, \tilde F_\alpha: \B(\infty) \to \B(\infty) \cup \{0\}$ 
\cite{Jan}*{10.12}. The operators $\tilde E_\alpha$ are locally nilpotent, so 
it makes sense to set $e_{\alpha}(b) = \max \{k \in \NN \mid \tilde 
E_\alpha^k(b) \neq 0\}$ for each $b \in \B$. We write $\overline E_\alpha(b)
= \tilde E_{\alpha}^{e_\alpha(b)}(b)$.

In \cite{Lit}, P. Littelman proves that to each reduced decomposition of 
$w \in W$ we can associate a parametrization of $B_w(\infty)$ as 
follows. If $\tilde w = s_{\alpha_1} \cdots s_{\alpha_r}$ is the chosen 
decomposition then we define $\Lambda_{\tilde w}: \B \to \NN^r$ by the formula
\[
  \Lambda_{\tilde w}(b) 
    = (e_{\alpha_1}(b), e_{\alpha_2}(\overline E_{\alpha_1}(b)), \ldots,
    e_{\alpha_r}(\overline E_{\alpha_{r-1}} \cdots \overline E_{\alpha_1}(b))).
\]
Now according to \cite{Lit}*{Proposition 1.5, Theorem 1.7 and Corollary 1}, 
the set $S_{\tilde w} = \Lambda_{\tilde w}(\B_w(\infty))$ is the set of 
integral points of a convex polyhedral cone, and hence by Gordan's lemma a 
normal affine semigroup. 

A decomposition of $w_0$, the longest word of $W$, is said to be \emph{adapted}
to $w$ if it is of the form $s_{\alpha_1} \cdots s_{\alpha_N}$ with 
$s_{\alpha_1} \cdots s_{\alpha_{\ell(w)}} = w$. For every element $w \in W$ 
there exists a decomposition of the longest word of $W$ adapted to $w$, or in 
other words the longest word of $W$ is the maximum for the weak right Bruhat 
order on $W$, see \cite{BB}*{Proposition 3.1.2}. 
\begin{Lemma*}
Set $\Pi = \{\alpha_1, \ldots, \alpha_n\}$ and set $\varpi_i = 
\varpi_{\alpha_i}$. Fix $w \in W$ and fix $\tilde w_0$ a decomposition of 
$w_0$ adapted to $w$. The map
\begin{align*}
\Gamma_{\tilde w_0}:
   FB &\to \NN^{N} \times \NN^n \\
   b_\lambda & \longmapsto \Lambda_{\tilde w_0}(b) \times 
   (\langle \lambda, \alpha_1 \rangle, \ldots, \langle \lambda, \alpha_n 
   \rangle).
\end{align*}
is injective, and the sets $\tilde S_{\tilde w_0} = 
\GG_{\tilde w_0}(\flagbasis), \tilde S_{\tilde w_0}^I = 
\GG_{\tilde w_0}(\flagbasis_I)$ and $\tilde S_{\tilde w_0, w}^I = 
\GG_{\tilde w_0}(\schubertbasis_I(w))$ are normal affine semigroups.
\end{Lemma*}
\begin{proof}
Recall that $\lambda = \sum_{\alpha \in \Pi} \langle \lambda, \alpha \rangle 
\varpi_\alpha$, so if $\Gamma_{\tilde w_0}(b_\lambda) = 
\Gamma_{\tilde w_0}(b'_{\lambda'})$ then $\lambda = \lambda'$ and 
$\Lambda_{\tilde w_0}(b) = \Lambda_{\tilde w_0}(b')$, which implies 
$b_\lambda = b'_{\lambda'}$. Thus $\Gamma_{\tilde w_0}$ is injective .

According to \cite{Lit}*{Proposition 1.5} $
\tilde S_{\tilde w_0}$  is the set of all $a \times (r_1, \ldots, r_n) \in 
S_{\tilde w_0} \times \NN^n$ such that
\begin{align*}
a_l 
  &\leq \left\langle \lambda - \sum_{j=1}^{N-l} a_{N-j+1} \alpha_{i_{N-j+1}}, 
  \alpha_{i_l} \right\rangle
& (1 \leq l \leq N)
\end{align*}
where $\lambda = \sum_i r_i \varpi_i$. Thus $\tilde S_{\tilde w_0}$ is the set 
of points of $S_{\tilde w_0} \times \NN^n$ that comply with these 
inequalities, and hence it is also a normal affine semigroup. 

Furthermore, $\tilde S_{\tilde w_0}^I = \Gamma_{\tilde w_0}(\flagbasis_I)$ is 
the intersection of $\tilde S_{\tilde w_0}$ with the hyperplanes defined by 
the equations $x_{N+i} = 0$ for all $i$ such that $\varpi_i \in I$, and hence 
is also a normal affine semigroup. Finally, the fact that the decomposition 
$\tilde w_0$ is adapted to $w$ implies that 
$\tilde S_{\tilde w_0, w}^I = \Gamma_{\tilde w_0}(\schubertbasis_I(w))$ is the 
intersection of $\tilde S_{\tilde w_0}^I$ with the hyperplanes defined by 
$x_i = 0$ for all $\ell(w) < i \leq N$, and hence it is also a normal affine 
semigroup. 
\end{proof}

\paragraph
\label{coproduct-coefficients}
We have just shown that the bases of quantum flag varieties and quantum 
Schubert varieties defined in \ref{q-var-bases} are parametrized by normal 
semigroups. All that is left to check is that they have the multiplicative 
property of $(S,<_{\lex})$-bases, where $<_{\lex}$ is the lexicographic order 
of $\NN^N \times \NN^n$.

Let $\lambda, \lambda' \in \Lambda^+$ and $b \in \B(\lambda), b' \in
\B(\lambda')$. Recall that $b_\lambda^*$ denotes the element in the dual basis 
of $V_v(\lambda)^*$ as defined in \ref{q-var-bases}. The product $b_\lambda 
b'_{\lambda'}$ is by definition the matrix coefficient corresponding to the 
functional $b_\lambda^* \ot b_{\lambda'}'^*$ and the vector $v_\lambda \ot 
v_{\lambda'}$ over $V(\lambda) \ot V(\lambda')$. Now the $U$-module generated 
by $v_\lambda \ot v_{\lambda'}$ is isomorphic to $V_v(\lambda + \lambda')$, so 
$b_\lambda^* \ot b_{\lambda'}'^*$ naturally induces an element in $C^+(\lambda 
+ \lambda')$, and the product $b_\lambda b'_{\lambda'}$ is a linear 
combination of matrix coefficients in $C^+_v(\lambda + \lambda')$
\[
  b_\lambda b'_{\lambda'} 
    = \sum_{b'' \in \B(\lambda + \lambda')} c^{b''}_{b,b'} b''_{\lambda + 
    \lambda'}
\]
with $c^{b''}_{b,b'} \in \QQ(v)$. By definition the scalar $c_{b,b'}^{b''}$ is 
the value $b_\lambda b'_{\lambda'}(G(b''))$; in particular it depends on 
$\lambda$ and $\lambda'$. On the other hand, by the definition of the product 
of matrix coefficients
\[
  b_\lambda b'_{\lambda'}(G(b''))
    = b_\lambda^* \ot b_{\lambda'}'^*(\Delta(G(b'')) \cdot 
      v_\lambda \ot v_{\lambda'}),
\]
so we need to study the corproduct $\Delta(G(b''))$.

It follows easily from \cite{Jan}*{4.9 (4)} that $\Delta(F_\alpha^{(r)}) = 
\sum_{i+j=r}^r v_\alpha^{ji} F^{(i)}_\alpha \ot F^{(j)}_\alpha K_\alpha^{-i}$,
where $v_\alpha = v^{(\alpha,\alpha)/2}$. 
Since the canonical basis is contained in $U_\A^-$, it follows that 
$\Delta(G(b''))$ is a sum of terms of the form $p (G(b_{(1)}) \ot 
G(b_{(2)}) K_\mu)$ with $p \in \A, b_{(1)}, b_{(2)} \in \B(\infty)$, and $\mu 
= \wt(b_{(1)})$, the weight of $b_{(1)}$. Among all these terms there is one 
of the form $d_{b,b'}^{b''} (G(b) \ot G(b') K_{\wt(b)})$ with $d_{b,b'}^{b''} 
\in \A$, and from the reasoning above we get that $c_{b,b'}^{b''} 
= v^{(\wt(b), \lambda')} d_{b,b'}^{b''}$. Notice that unlike before, the 
element $d_{b,b'}^{b''}$ is independent of $\lambda$ and $\lambda'$.

The problem of showing that $\flagbasis$ is indeed an $(\tilde S_{\tilde w_0}, 
<_{\lex})$-basis thus reduces to the follwing: show that if $d_{b,b'}^{b''} 
\neq 0$ then $\Lambda_{\tilde w_0}(b'') \leq_{\lex} \Lambda_{\tilde w_0}(b) + 
\Lambda_{\tilde w_0}(b')$. Caldero shows that this is indeed the case in 
\cite{C}*{Theorem 2.3}. We give a shorter proof in the following paragraphs.

\paragraph
\label{coproduct-coefficients-convex}
We first fix some notation. Given a decomposition of the longest word of $W$
$\tilde w_0 = s_{\alpha_1} \cdots s_{\alpha_N}$, for each $a \in \NN^N$ we 
write $F^{(a)} = F_{\alpha_1}^{(a_1)} \cdots F_{\alpha_N}^{(a_N)}$ and 
$\tilde F^{a} = \tilde F_{\alpha_1}^{a_1} \cdots \tilde F_{\alpha_N}^{a_N}$. 
If $a$ lies in the image of $\Lambda_{\tilde w_0}$ then we write $b^a = 
\Lambda_{\tilde w_0}^{-1}(a)$.

\begin{Lemma*}
Fix $\tilde w_0 = s_{\alpha_1} \cdots s_{\alpha_N}$ a decomposition of the 
longest word of $W$. Let $a \in \NN^N$, and let $\nu \in \ZZ\Phi, \nu \geq 0$ 
be such that $F^{(a)} \in U^-_{-\nu}$.
\begin{enumerate}[label=(\alph*)]
\item 
\label{pre}
Let $\alpha \in \Pi$ and $r > 0$. Let $b \in \B(\infty)_{-\nu}$. Then
for each $b' \in \B(\infty)_{-\nu - r\alpha}$ with $e_\alpha(b') > r + 
e_\alpha(b)$ there exists $x_{b'} \in \A$ such that
\begin{align*}
F^{(r)}_\alpha G(b) 
  &= \qbinom{r+e_\alpha(b)}{r} G(\tilde F_{\alpha}^r b) 
  	+ \sum_{e_{\alpha}(b') > r + e_\alpha(b)} x_{b'} G(b').
\end{align*}

\item 
\label{in-image}
Suppose $a$ lies in the image of $\Lambda_{\tilde w_0}$. Then for each $a' \in 
\Lambda_{\tilde w_0} (\B(\infty)_{-\nu})$ with $a' >_{\lex} a$ there exist 
$x_{a,a'}, y_{a,a'} \in \A$ such that
\begin{align*}
F^{(a)}
  &= G(b^a) + \sum_{a' >_{\lex} a} x_{a,a'} G(b^{a'}), &
G(b^a) 
  &= F^{(a)} + \sum_{a' >_{\lex} a} y_{a,a'} F^{(a')}.
\end{align*}

\item 
\label{not-in-image}
Suppose $a$ does not lie in the image of $\Lambda_{\tilde w_0}$. Let $b = 
\tilde F^a 1 + q\L(\infty) \in \B(\infty)_{-\nu}$ and let $s = 
\Lambda_{\tilde w_0}(b)$. Then $s >_{\lex} a$, and for each $a'\in 
\Lambda_{\tilde w_0} (\B(\infty)_{-\nu})$ with $a' >_{\lex} a$ there exists 
$z_{a,a'} \in \A$ such that
\begin{align*}
	F^{(a)} &= \sum_{a' >_\lex a} z_{a,a'} G(b^{a'}).
\end{align*}
Furthermore $z_{a,s} \neq 0$.
\end{enumerate}
\end{Lemma*}
\begin{proof}
Let us prove item \ref{pre}. Put $e = e_\alpha(b)$, and for each $n \geq 0$
set $M_{n} = \tilde F_\alpha^{n}U^-_{-\nu+n\alpha}$. As shown in the proof
of \cite{Jan}*{Lemma 11.12, p. 249}, 
\begin{align}
\label{1}
G(b) 
	\equiv \tilde F_\alpha^e(G(\tilde E_\alpha^e b))
	\mod M_{e+1}.
\end{align}
This plus \cite{Jan}*{Lemma 11.3} and the definition of the divided powers 
imply that
\begin{align*}
F_\alpha^{(r)} G(b) 
	&\equiv F_\alpha^{(r)} \tilde F_\alpha^{e} G(\tilde E_\alpha^e b) 
	\equiv F_\alpha^{(r)} F_\alpha^{(e)} G(\tilde E_\alpha^e b) \\
	&\equiv \qbinom{r+e}{r} F_\alpha^{(r+e)} G(\tilde E_\alpha^e b) 
	\equiv \qbinom{r+e}{r} \tilde F_\alpha^{r+e} G(\tilde E_\alpha^e b) 
	\mod M_{e+r+1}.
\end{align*}
Finally
\begin{align*}
\tilde F_\alpha^{r+e} G(\tilde E_\alpha^e b)
	&= \tilde F_\alpha^{r+e} G(\tilde E_\alpha^{r+e} \tilde F_\alpha^r b)
	\equiv G(\tilde F_\alpha^r b) \mod M_{r+e+1}
\end{align*}
where the last congruence follows from \ref{1}. Thus $F_\alpha^{(r)}G(b) 
\equiv \qbinom{r+e}{r} G(\tilde F_\alpha^rb) \mod M_{r+e+1}$. By 
\cite{Jan}*{Lemma 11.12 (b)} the $G(b')$ with $e_\alpha(b') > e+r$ form
an $\A$-basis of $M_{r+e+1}$, so item \ref{pre} follows.

Item \ref{in-image} is a simple consequence of \cite{Lit}*{Proposition 10.3},
which states that given a dominant weight $\lambda$ and a highest weight 
$v_\lambda \in V_\A(\lambda)$, then if $G(b^a) v_{\lambda} \neq 0$ there exist 
$x_{a,a'}$ as in the statement such that
\begin{align*}
F^{(a)}v_\lambda
  &= G(b^a)v_\lambda + \sum_{a' >_{\lex} a} x_{a,a'} G(b^{a'})v_\lambda.
\end{align*}
As shown in the proof of \cite{Jan}*{Theorem 10.10}, there exists a dominant
weight $\lambda$ such that the map $U^-_{-\nu} \to V_q(\lambda)_{\lambda - 
\nu}$ given by $G(b) \mapsto G(b)v_\lambda$ is an isomorphism, so the first 
formula is proved. The second follows from the first since there is only a 
finite number of elements $a' >_{\lex} a$ such that $b^{a'} \in 
\B(\infty)_{-\nu}$. 

Let us prove the last item. Suppose that $s <_\lex a$. Then by definition 
$s_1 = a_1, \ldots, s_j = a_j, s_{j+1} < a_{j+1}$ for some $1 \leq j \leq N$. 
This implies that 
\begin{align*}
	\tilde E_{\alpha_{j+1}}^{s_{j+1}+1} 
		\tilde E_{\alpha_j}^{s_j} \cdots \tilde E_{\alpha_1}^{s_1}(b) 
		= \tilde F_{\alpha_{j+1}}^{a_j-s_j-1} \cdots \tilde 
			F_{\alpha_N}^{a_N}1 + q\L(\infty) \neq 0,
\end{align*}
which contradicts the definition of $\Lambda_{\tilde w_0}(b)$. Thus $s 
\geq_{\lex} a$, and the hypothesis implies that the inequality is strict.

Now let $j = \min \{i \mid a_i \neq 0\}$, so $F^{(a)} = F_{\alpha_j}^{(a_j)} 
\cdots F_{\alpha_N}^{(a_N)}$. Item \ref{pre} and a simple induction shows that
\begin{align*}
F^{(a)} 
  &= z_{a,s} G(b) 
  	+ \sum_{e_{\alpha_j}(b') > a_j} z_{b,b'} G(b')
\end{align*}
with $z_{a,s} \neq 0$. Notice that the condition $e_{\alpha_j}(b') > a_j$ 
automatically implies that $\Lambda_{\tilde w_0}(b') >_{\lex} a$, so we are 
finished.
\end{proof}

\begin{Proposition*}
Fix $\tilde w_0 = s_{i_1} \cdots s_{i_N}$ a decomposition of the longest word 
of $W$. Let $b,b',b'' \in \B(\infty)$. If $d_{b,b'}^{b''} \neq 0$ then 
$\Lambda_{\tilde w_0}(b'') \leq_{\lex} \Lambda_{\tilde w_0}(b) + 
\Lambda_{\tilde w_0}(b')$, and if equality holds then $d_{b,b'}^{b''}$
is a power of $v$.
\end{Proposition*}
\begin{proof}
We have already observed that $\Delta(F_\alpha^{(r)}) = \sum_{i+j = r} 
v_\alpha^{ij} F_\alpha^{(i)} \ot F_\alpha^{(j)} K^{-i}_\alpha$, and it follows 
that for each $a \in \NN^N$ we get $\Delta(F^{(a)}) = 
\sum_{t+u = a} v^{z(t,u)} F^{(t)} \ot F^{(u)}K_{\wt(F^{(t)})}$, with $z(t,u) 
\in \ZZ$. By item \ref{in-image} of the previous lemma 
\begin{align*}
\Delta(G(b^a))
  &= \Delta(F^{(a)}) + \sum_{a' >_{\lex} a} y_{a,a'} \Delta(F^{(a')}) \\
  &= \sum_{t+u = a} v^{z(t,u)} F^{(t)} \ot F^{(u)}K_{\wt(F^{(t)})} 
  + \sum_{t+u >_\lex a} y_{a,t+u} v^{z(t,u)} F^{(t)}
    \ot F^{(u)}K_{\wt(F^{(t)})},
\end{align*}
and using item \ref{not-in-image} of the lemma we get that the element in the 
last display equals
\begin{align*}
\sum_{t+u = a} v^{z(t,u)} G(b^t) \ot G(b^u) K_{\wt(b^t)}
    + \sum_{t+u >_{\lex} a} y'_{t,u} G(b^t) \ot G(b^u) K_{\wt(b^t)},
\end{align*}
where $t,u$ run over the image of $\Lambda_{\tilde w_0}$, and $z(t,u) \in \ZZ$
and $y'_{t,u} \in \A$ for each such pair $(t,u)$. The result follows by taking 
$b^a = b''$.
\end{proof}

\begin{Theorem}
\label{Schubert-deg}
Let $\k$ be a field of characteristic $0$ and let $q \in \k^\times$ be 
trascendental over $\QQ$. Let $I \subset P$ be a set of fundamental weights, 
and let $w \in W$. Let $\tilde w_0$ be a reduced decomposition of $w_0$ 
adapted to $w$.

The quantum Schubert variety $\O_q[G/P_I]_w$ degenerates to a quantum affine 
toric variety with associated semigroup $\tilde S^I_{\tilde w_0, \tilde w}$. 
In particular any quantum Schubert variety has property $\chi$, finite local 
dimension, the AS-Cohen-Macaulay property, and is a maximal order in its 
skew-field of fractions.
\end{Theorem}
\begin{proof}
We put on $\k$ the right $\A$-module structure given by $1_\k \cdot v = q$.
Thus $\O_q[G/P_I]_w \cong \k \ot_\A \O_v[G/P_I]_w$. 

For each dominant weight $\lambda$ and each $a \in \Lambda_{\tilde w_0}
(\B(\lambda))$, we denote by $b^a_\lambda$ the element $1 \ot_\A b_\lambda
\in \k \ot_\A \O_v[G/P_I]_w$ where $b = \GG_{\tilde w_0}^{-1}(a)$.
With this notation, it follows from \ref{coproduct-coefficients-convex} that 
for each pair of dominant weights $\lambda, \lambda'$ and each $a \in 
\Lambda_{\tilde w_0}(\B(\lambda)), a' \in \Lambda_{\tilde w_0}(\B(\lambda')), 
a'' \in \Lambda_{\tilde w_0}(\B(\lambda + \lambda'))$ such that $a'' 
\geq_{\lex} a + a'$, there exists $x_{a''} \in \A$ such that 
\begin{align*}
b^a_{\lambda} b^{a'}_{\lambda'} = \sum_{a'' \geq a+a'} x_{a''} 
  b^{a''}_{\lambda + \lambda'},
\end{align*}
with $x_{a+a'}$ a power of $v$. This implies that the basis 
$\{1 \ot b \mid b \in \schubertbasis_I(w)\}$ is a $(\tilde S^I_{\tilde w_0, 
\tilde w}, \leq_{\lex}~)$ basis. Thus by Proposition 
\ref{S-ordered-basis-degeneration} we get the degeneration result.

Since $\tilde S^w_{w_0, I}$ is a normal semigroup, we know by Proposition 
\ref{properties-of-qatv} that the associated graded ring of the quantum 
Schubert variety has property $\chi$, finite local dimension and the 
AS-Cohen-Macaulay property, which $\O_q[G/P_I]_w$ inherits by Theorem 
\ref{transfer}. 
Also by Proposition \ref{properties-of-qatv}, a quantum affine toric variety 
whose underlying semigroup is normal is a maximal order in its ring of 
fractions, and it follows from \cite{Mau}*{Chapitre IV, Proposition 2.1 and 
Chapitre V, Corollaire 2.6} that $\O_q[G/P_I]_w$ is also a maximal order. 
\end{proof}


%\newpage 
\section{Quantum affine toric degeneration of an algebras with a straightening 
law over a distributive lattice}
\label{richardson}

We finish this article by relating our previous work \cite{RZ} with the material found 
here. In that article we proved that Richardson varieties in the quantum grassmanian as 
defined in \cite{RZ}*{Definition 5.2.1} are symmetric quantum graded algebras with a 
straightening law over a poset \cite{RZ}*{Definition 3.1}, and that they satisfy a 
technical ``Condition (C)'' \cite{RZ}*{Definition 4.1}. In \cite{RZ}*{Theorem 4.9} we 
proved that, in the language of the present article, all symmetric quantum graded ASL's 
satisfying condition (C) have quantum toric degenerations. We now prove a similar result
under conditions milder than (C). In order to do this, we place 
ourselves in the context of \cite{RZ2}*{section 3}, the reader is referred to it for 
definitions and unexplained notation.

\paragraph
For the rest of this subsection $\L$ denotes a finite distributive lattice. Given $x,y 
\in \L$ we denote their meet by $x \meet y$ and their join by $x \join y$, as usual. 
Recall that the rank of $\L$ is defined as the length of any maximal chain in $\L$, which
we denote by $r$. We also denote by $J(\L)$ the set of join irreducible elements of 
$\L$, so $r = |J(\L)|$. We see $J(\L)$ as a poset with the order induced by $\L$, and
extend this to a total order, so $J(\L) = \{p_1, \ldots, p_n\}$ and $p_i < p_j$ implies 
$i < j$. Birkhoff's representation theorem states that $\L$ is isomorphic to the lattice
of poset ideals of $J(\L)$, through the assignation $l \in \L \mapsto \{p \in J(I) \mid 
p \leq l\}$.

\paragraph 
As shown in \cite{RZ2}*{section 3.3}, associated to $\L$ there is a semigroup $S(\L)$, 
defined as follows: $S(\L)$ is the abelian semigroup generated by $\L$ modulo the 
relations $\{x + y \equiv (x \meet y) + (x \join y) \mid x,y \in \L\}$.
The map $\iota: \L \to \NN^{r}$ defined as
\[
  \iota(l) = \sum_{\{i : p_i \leq l\}} e_i,
\]
for each $l \in L$ is injective and a lattice morphism. 
By \cite{RZ2}*{Proposition 3.3.3}, $\iota$ induces an injective semigroup morphism 
$\iota: S(\L) \to \NN^{r}$ whose image is a full normal subsemigroup, so $S(\L)$ is a
normal affine semigroup. 

Set $n = |\L|$; write $\L = \{x_1, \ldots, x_n\}$, and set $s_i = \iota(x_i)$ for each 
$1 \leq i \leq n$. Let $\pi: \NN^n \to S(\L)$ be the semigroup morphism induced 
by the assignation $e_i \mapsto s_i$ for all $i$. It follows from 
\cite{RZ2}*{Lemma 3.3.2} that every element of $S(\L)$ can be 
written in a unique way as $s = \sum_{i=1}^n m_i s_i$ with the condition that the set 
$\{x_i \mid m_i \neq 0\}$ is a chain of $\L$. Setting $t(s) = (m_1, \ldots, m_n)$ we 
obtain a section $t: S(\L) \to \NN^n$ of $\pi$. We refer to $t$ as the \emph{standard 
section} of $\pi$. 

\paragraph
For every pair of elements $x,y \in \L$ we define
\begin{align*} 
  R(x,y) &= \{(z,z') \in \L \times \L \mid z < x, y  <z'\}.  
\end{align*} 
We write $R(x,y)^*$ for $R(x,y) \setminus (x \meet y, x \join y)$.

\paragraph
We continue to denote by $A$ an $\NN^r$-graded noetherian algebra.
Suppose that there is an injective function $[-]: \L \to A$. A \emph{standard monomial} 
in $A$ is a product of the form $[x_0] [x_1] \ldots [x_n]$ with $x_{i} \leq x_{i+1}$ for 
all $1 \leq i < n$. 
\begin{Definition*}
\label{D:symm-q-gr-asl} 
Let $\L$ be a finite distributive lattice, and let $[-]: \L \to A$ be an injective 
function. We say that $A$ is a 
\emph{symmetric quantum graded algebra with a straightening law over $\L$}, or symmetric 
quantum ASL for short, if the following hold.

\begin{enumerate} 
  \item The set $\{[x] \mid x \in \L\}$ consists of homogeneous elements of nonzero
    degree that generate $A$ as an algebra.

  \item The set of standard monomials is linearly independent.

  \item For every $x,y \in \L$ there exist $\{c^{z,z'}_{x,y}\}_{(z,z') \in R(x,y)^*}
  \subset \k$ and $c_{x,y} \in \k^\times$, such that 
  \begin{align*} 
    [y][x] = c_{x,y} [x][y] + \sum_{(z,z') \in R(x,y)^*} c^{z,z'}_{x,y} [z][z'].  
  \end{align*}

  \item For every pair of incomparable elements $x,y \in \L$ there exist
  $\{d^{z,z'}_{x,y}\}_{(z,z') \in R(x,y)^*} \subset \k$ and $d_{x,y} \in \k^\times$, such
  that
  \begin{align*} 
    [x][y] = d_{x,y} [x \meet y][x \join y] +
    \sum_{(z,z') \in R(x,y)^*} d^{z,z'}_{x,y} [z][z'].  \end{align*}
  \end{enumerate} 
\end{Definition*}
\begin{Remark*}
The word \emph{symmetric} is there to distinguish this class from the more general class
of quantum graded ASL's over a poset, defined in \cite{LR1}*{Definition 1.1.1}. Also
notice that the definition of a symmetric quantum ASL over a general poset $P$ given in
\cite{RZ}*{Definition 3.1} is slightly more general, as it does not require that 
$d_{x,y} \neq 0$. This simple requirement is what takes the place of the much more 
technical condition $(C)$.
\end{Remark*}

\begin{Theorem} 
\label{T:ASL-are-dominated} 
If $A$ is a symmetric quantum graded ASL over $\L$ then it has a quantum affine toric 
degeneration with associated semigroup $S(\L)$.
\end{Theorem} 
\begin{proof} 
Let $C = \{\iota(x) + \iota(y) \mid x,y \in \L\}$. This is a finite set, and so by 
reasoning as in the proof of Proposition \ref{lex-degeneration} there is a morphism 
$\phi: \NN^{\rk \L} \to \NN$ such that $\xi <_{\lex} \nu$ implies $\phi(\xi) < 
\phi(\nu)$ for all $\xi, \nu \in C$. 

Let $x,y \in \L$ and let $(z,z') \in R(x,y)^*$. Using Birkhoff's representation theorem, 
it follows from the defintion of $\iota$ that $\iota(z') - \iota(y) <_{\lex} \iota(x) - 
\iota(z)$, or equivalently $\iota(z) + \iota(z') <_{\lex} \iota(x) + \iota(y)$. Thus by 
definition $A$ is of $(S,\phi)$-type with generators $b_i = [x_i]$. Also, by 
\cite{LR1}*{Proposition 1.14}, standard monomials form a basis of $A$. The set of 
standard monomials is the generating set associated to the standard section $t$ as 
defined in \ref{P:equivalent-qatd}, so the proposition in that paragraph implies that
$A$ has a quantum toric degeneration with associated semigroup $S(\L)$.
\end{proof}

%\newpage
\begin{bibdiv} 
\begin{biblist} 
\bib{AZ}{article}{ 
  author={Artin, M.}, 
  author={Zhang, J. J.}, 
  title={Noncommutative projective schemes}, 
  journal={Adv. Math.}, 
  volume={109}, 
  date={1994}, 
  number={2},
  pages={228--287}, 
}

    
\bib{BS}{book}{ 
  author={Brodmann, M. P.}, 
  author={Sharp, R. Y.}, 
  title={Local cohomology: an algebraic introduction with geometric applications}, 
  series={Cambridge Studies in Advanced Mathematics}, 
  volume={60}, 
  publisher={Cambridge University Press},
  place={Cambridge}, 
  date={1998}, 
  pages={xvi+416}, 
}

\bib{BH}{book}{ 
  author={Bruns, Winfried}, 
  author={Herzog, J{\"u}rgen},
  title={Cohen-Macaulay rings}, 
  series={Cambridge Studies in Advanced Mathematics},
  volume={39}, 
  publisher={Cambridge University Press}, 
  place={Cambridge}, 
  date={1993},
  pages={xii+403}, 
}

\bib{BG}{book}{
   author={Bruns, Winfried},
   author={Gubeladze, Joseph},
   title={Polytopes, rings, and $K$-theory},
   series={Springer Monographs in Mathematics},
   publisher={Springer, Dordrecht},
   date={2009},
   pages={xiv+461},
}

\bib{BB}{book}{
   author={Bj{\"o}rner, Anders},
   author={Brenti, Francesco},
   title={Combinatorics of Coxeter groups},
   series={Graduate Texts in Mathematics},
   volume={231},
   publisher={Springer},
   place={New York},
   date={2005},
   pages={xiv+363},
}

\bib{C}{article}{ 
  author={Caldero, Philippe}, 
  title={Toric degenerations of Schubert varieties}, 
  journal={Transform. Groups}, 
  volume={7}, 
  date={2002}, 
  number={1},
  pages={51--60}, 
}


\bib{FFL}{article}{
   author={Fang, Xin},
   author={Fourier, Ghislain},
   author={Littelmann, Peter},
   title={On toric degenerations of flag varieties},
   conference={
      title={Representation theory---current trends and perspectives},
   },
   book={
      series={EMS Ser. Congr. Rep.},
      publisher={Eur. Math. Soc., Z\"urich},
   },
   date={2017},
   pages={187--232},
}

\bib{RGS}{book}{
   author={Rosales, J. C.},
   author={Garc{\'{\i}}a-S{\'a}nchez, P. A.},
   title={Finitely generated commutative monoids},
   publisher={Nova Science Publishers, Inc., Commack, NY},
   date={1999},
   pages={xiv+185},
}
    
\bib{GL}{article}{
   author={Gonciulea, N.},
   author={Lakshmibai, V.},
   title={Degenerations of flag and Schubert varieties to toric varieties},
   journal={Transform. Groups},
   volume={1},
   date={1996},
   number={3},
   pages={215--248},
}

\bib{Jan}{book}{ 
  author={Jantzen, Jens Carsten}, 
  title={Lectures on quantum groups},
  series={Graduate Studies in Mathematics}, 
  volume={6}, 
  publisher={American Mathematical Society}, 
  place={Providence, RI}, 
  date={1996}, 
  pages={viii+266}, 
}

\bib{JZ}{article}{ 
  author={J{\o}rgensen, Peter}, 
  author={Zhang, James J.},
  title={Gourmet's guide to Gorensteinness}, 
  journal={Adv. Math.}, 
  volume={151},
  date={2000}, 
  number={2}, 
  pages={313--345}, 
}

\bib{K1}{article}{ 
  author={Kashiwara, Masaki}, 
  title={The crystal base and Littelmann's refined Demazure character formula}, 
  journal={Duke Math. J.}, 
  volume={71}, 
  date={1993},
  number={3}, 
  pages={839--858}, 
}

\bib{K}{article}{ 
  author={Kashiwara, Masaki}, 
  title={On crystal bases}, 
  conference={
    title={Representations of groups}, 
    address={Banff, AB}, date={1994}, }, 
  book={
    series={CMS Conf. Proc.}, 
    volume={16}, publisher={Amer. Math. Soc.},
    place={Providence, RI}, 
    }, 
  date={1995}, 
  pages={155--197}, 
}


\bib{qLR}{article}{ 
  author={Lakshmibai, V.}, 
  author={Reshetikhin, N.}, 
  title={Quantum flag and Schubert schemes}, 
  conference={ 
    title={ physics}, 
    address={Amherst, MA},
    date={1990}, 
  }, 
  book={ 
    series={Contemp. Math.}, 
    volume={134}, 
    publisher={Amer. Math. Soc.}, 
    place={Providence, RI}, 
    }, 
  date={1992}, 
  pages={145--181}, 
}

\bib{LR1}{article}{ 
  author={Lenagan, T. H.}, 
  author={Rigal, L.}, 
  title={Quantum graded algebras with a straightening law and the AS-Cohen-Macaulay   
    property for quantum determinantal rings and quantum Grassmannians}, 
  journal={J. Algebra}, 
  volume={301},
  date={2006}, 
  number={2}, 
  pages={670--702}, 
}

\bib{Lit}{article}{ 
  author={Littelmann, P.}, 
  title={Cones, crystals, and patterns},
  journal={Transform. Groups}, 
  volume={3}, 
  date={1998}, 
  number={2}, 
  pages={145--179}, 
}

\bib{Lu}{book}{ 
  author={Lusztig, George}, 
  title={Introduction to quantum groups},
  series={Modern Birkh\"auser Classics}, 
  note={Reprint of the 1994 edition},
  publisher={Birkh\"auser/Springer, New York}, 
  date={2010}, 
  pages={xiv+346}, 
}

\bib{Mau}{book}{
   author={Maury, Guy},
   author={Raynaud, Jacques},
   title={Ordres maximaux au sens de K. Asano},
   language={French},
   series={Lecture Notes in Mathematics},
   volume={808},
   publisher={Springer},
   place={Berlin},
   date={1980},
   pages={viii+192},
}

\bib{MR}{book}{ 
  author={McConnell, J. C.}, 
  author={Robson, J. C.}, 
  title={Noncommutative Noetherian rings}, 
  series={Graduate Studies in Mathematics}, 
  volume={30},
  edition={Revised edition}, 
  note={With the cooperation of L. W. Small},
  publisher={American Mathematical Society}, 
  place={Providence, RI}, 
  date={2001},
  pages={xx+636}, 
}

\bib{MS}{book}{ 
  author={Miller, Ezra}, 
  author={Sturmfels, Bernd}, 
  title={Combinatorial commutative algebra}, 
  series={Graduate Texts in Mathematics}, 
  volume={227},
  publisher={Springer-Verlag}, 
  place={New York}, 
  date={2005}, 
  pages={xiv+417}, 
}


\bib{VO}{book}{ 
  author={N{\u{a}}st{\u{a}}sescu, Constantin}, 
  author={Van Oystaeyen, F.},
  title={Graded and filtered rings and modules}, 
  series={Lecture Notes in Mathematics},
  volume={758}, 
  publisher={Springer}, 
  place={Berlin}, 
  date={1979}, 
  pages={x+148}, 
}



\bib{NV}{book}{ 
  author={N{\u{a}}st{\u{a}}sescu, Constantin}, 
  author={Van Oystaeyen, Freddy}, 
  title={Methods of graded rings}, 
  series={Lecture Notes in Mathematics},
  volume={1836}, 
  publisher={Springer-Verlag}, 
  place={Berlin}, 
  date={2004}, 
  pages={xiv+304},
}



\bib{RZ}{article}{
  author={Rigal, L.}, 
  author={Zadunaisky, P.}, 
  title={Quantum analogues of Richardson varieties in the grassmannian and their toric
  degeneration},
  journal={J. Algebra}, 
  volume={372}, 
  date={2012}, 
  pages={293--317}, 
}

\bib{RZ2}{article}{
  author={Rigal, L.}, 
  author={Zadunaisky, P.}, 
  title={Twisted Semigroup Algebras},
  journal={Alg. Rep. Theory},
  year={2015},
  doi={10.1007/s10468-015-9525-z},
}

\bib{S}{article}{ 
  author={So{\u\i}bel{\cprime}man, Ya. S.}, 
  title={On the quantum flag manifold}, 
  language={Russian}, 
  journal={Funktsional. Anal. i Prilozhen.},
  volume={26}, 
  date={1992}, 
  number={3}, 
  pages={90--92}, 
  translation={
    journal={Funct. Anal. Appl.}, 
    volume={26}, 
    date={1992}, 
    number={3},
    pages={225--227}, 
  }, 
}

\bib{St}{book}{
   author={Sturmfels, Bernd},
   title={Gr\"obner bases and convex polytopes},
   series={University Lecture Series},
   volume={8},
   publisher={American Mathematical Society, Providence, RI},
   date={1996},
   pages={xii+162},
}

\bib{W}{book}{ 
  author={Weibel, Charles A.}, 
  title={An introduction to homological algebra}, 
  series={Cambridge Studies in Advanced Mathematics}, 
  volume={38},
  publisher={Cambridge University Press}, 
  place={Cambridge}, 
  date={1994}, 
  pages={xiv+450}, 
}
\end{biblist} 
\end{bibdiv}

\end{document}

\newpage
\section*{To do}
\begin{enumerate}
\item Clarify hypothesis of Props, thms, etc. in sections 2,4.

\item $\NN$ includes $0$, while $\NN^*$ does not. Make sure everything makes sense!
\end{enumerate}

\newpage
\section{How to complete the details of Caldero's argument with
Janzen's definition}



\begin{Lemma}
Let $b,b',b'' \in \B(\infty)$. Let $i \in \{1, \ldots, n\}$ and let 
$e = e_i(b)$.
\begin{enumerate}
\item For each $r \geq 0$ we have $\tilde F_i^r G(b) \equiv G(\tilde F_i^r b)
\mod \tilde F^{r+1} U_\A^-$  

\item For each $r \geq 0$ we have $F_i^{(r)} G(b) \equiv \qbinom{r+e}{e} 
G(\tilde F_i^r b) \mod \tilde F^{r+e+1} U_\A^-$.

\item If $d_{b,b'}^{b''} \neq 0$ then $e_i(b'') \leq e_i(b) + e_i(b')$.
\end{enumerate}
\end{Lemma}
\begin{proof}
The statements are obvious if $r = 0$ so take $r > 0$.
As stated in the proof of \cite{Jan}*{Lemma 11.12}, page 249, $G^\alpha(b') 
\equiv \tilde F_i^r  G(\tilde E_i^r b') \mod \tilde F_i^{r+1} U^-_\A$ whenever
$e_i(b') > r$; notice that $G^\alpha(b') = G(b')$ by \cite{Jan}*{11.15}. 
Taking $b' = \tilde F_i^r b$, we get $G(\tilde F_i^r b) \equiv \tilde F_i^r 
G(b) \mod \tilde F_i^{r+1}U_\A^-$, which is item 1. 

Now by \cite{Jan}*{Lemma 11.3} we get
\begin{align*}
G(b) 
  \equiv \tilde F_i^e G(\tilde E_i^e b) 
  \equiv F_i^{(e)} G(\tilde E_i^e b) 
  \mod \sum_{s > 0} F_i^{(r+s)} U_\A^-,
\end{align*}
and hence multplying by $F^{(r)}$ we get
\begin{align*}
F^{(r)} G(b) 
  \equiv \qbinom{r+e}{e} F_i^{(r+e)} G(\tilde E_i^e b) 
  \mod \sum_{s > 0} F_i^{(r+s+e)} U_\A^-.
\end{align*}
By the same lemma, $\sum_{s > 0} F_i^{(r+s+e)} U_\A^- = \tilde F_i^{r+e+1} 
U_\A^-$, so item 2 is proved. 

For each $r \geq 0$ we have
\begin{align*}
\Delta(F_i^{(r)}G(b))
  &= \Delta(F^{(r)}) \Delta(G(b))
  &= \sum_{s+t = r} \sum_{c,c' \in \B(\infty)} 
    d_{c,c'}^b F_i^{(s)} G(c) \ot F_i^{(t)} G(c') K_{wt(c)}
\end{align*}
By item 2, each element of the form $F_i^{(s)} G(c) \ot F_i^{(t)} G(c')$
lies in $\tilde F_i^s U_\A^- \ot \tilde F_i^t U_\A^-$, and by 
\cite{Jan}*{Lemma 11.12(b)} the elements of the form $G(c) \ot G(c')$ with 
$e_i(c) \geq s, e_i(c') \geq t$ are a basis of this space. Now $G(b'') = 
F_i^{(e'')} G(\tilde E_i^{e''} b'') + \sum_{s>0} F_i^{(e''+s)} u_s$ and each 
$u_s$ is an $\A$-linear combination of elements in the canonical basis, so 
applying $\Delta$ we obtain an $\A$ linear combination of elements of the form 
$G(c)\ot G(c')K_{wt(c)}$ with $e_i(c) + e_i(c') \geq e''$. This completes the 
proof of item 3.
\end{proof}

\paragraph
As stated above, the main point is to prove that
the coefficient $d_{b,b'}^{b''}$ is zero if $\Lambda_{\tilde w_0}(b) + 
\Lambda_{\tilde w_0}(b') <_{\lex} \Lambda_{\tilde w_0}(b'')$. Let $(-,-): U^- 
\times U^+ \to \QQ(v)$ the pairing defined in \cite{Jan}*{Proposition 6.12}. 
Fix $\nu \in \ZZ\Phi$ with $\nu \geq 0$. By \cite{Jan}*{Proposition 6.18} this 
pairing is non-degenerate when restricted to $U^-_{-\nu} \times U^+_{\nu}$, 
and hence $U^+_{\nu}$ has basis $\{G(b)^* \mid b \in \B(\infty)_{-\nu}\}$ 
uniquely defined by the property that $(G(b'),G(b)^*) = \delta_{b,b'}$ for all
$b' \in \B(\infty)_{-\nu}$. Now it follows from \cite{Jan}*{Proposition 6.12 
(1) and 6.13 (1)} that $d_{b,b'}^{b''}$ is the coefficient of $G(b'')^*$
in the product $G(b')^*G(b)^*$, and this is what we prove.

In order to prove this result about the dual of the canonical basis, we 
need to understand the string parametrization in terms of it.
Since multiplication by $F_i$ sends $U^-_{-\nu}$ to $U^-_{-\nu - \alpha_i}$, 
we obtain a transpose morphism $L_i: U^+_{\nu + \alpha_i} \to U^+_{\nu}$, which
is a $q$-derivation, since it is a multiple of the map $r'_{\alpha_i}$ defined 
in \cite{Jan}*{6.14}; we write $L_i^{(n)} = \frac{1}{[n]!} L_i^n$. 

\begin{Lemma*}
Fix $b \in \B(\infty)_{\nu}$ and set $\Delta(G(b)) = 
\sum_{b',b'' \in \B(\infty)} d_{b',b''}^{b} G(b') \ot G(b'')K_{wt(G(b'))}$. 
The following hold.
\begin{enumerate}

\item For all $r \geq 0$,
\[
  F_i^{(r)}G(b) 
    \equiv G(\tilde F_i^r b) \mod 
    \sum_{s \geq 0} F_i^{(r+s+1)}(U^-_\A)_{-\nu+(r+s+1)\alpha_i}.
\]

\item We have $e_i(b) = \max \{k \mid L_i^{(k)}G(b)^* \neq 0\}$ and
$L_i^{(e_i(b))}(G(b)^*) = G(\overline E_i(b))^*$.

\item Put $a_i = \max \{k \mid L_i^{(k)}(G(b)^*)\}$ and fix $\tilde w_0
=s_{i_1}, \cdots s_{i_N}$ a decomposition of the longest word in the Weyl 
group. Then
\[
  \Lambda_{\tilde w_0}(b)
    = (a_{i_N}(G(b)^*), a_{i_{N-1}}(\overline L_N(G(b)^*)), \ldots, 
    a_{i_i}(\overline L_{i_2} \overline L_{i_3} \cdots \overline L_{i_N}
    (G(b)^*)))
\]

\item $\overline L_{i_1} \overline L_{i_2} \overline L_{i_3} \cdots \overline 
L_{i_N} (G(b)^*)) = 1$.
\end{enumerate}
\end{Lemma*}
\begin{proof}
By \cite{Jan}*{11.12, p. 249}, the $G(b)$ have the property that
\[
  G(\tilde F_i^r b) \equiv \tilde F_i^r G(b) \mod 
    \tilde F_i^{r+1}(U_\A)_{-\nu + (r+1) \alpha_i}.
\]
The first item follows from \cite{Jan}*{Lemma 11.3 (1) and (2)}.

Put $e = e_i(b)$. Clearly $L_i^{(t)}(G(b)) = 0$ for $t \gg e$. Now assume this 
holds for some $t+1$ with $t \geq e$. By definition we get
\[
  (G(b'), L^{(t)}_i G(b)^*) = (F_i^{(t)}G(b'), G(b)^*)
\]
By the previous item, $F_i^{(t)}G(b') = G(\tilde F_i^t b') + F_i^{(t+1)}u$ 
for some $u \in U^-_\A$, so assuming $t \geq e$
\[
  (G(b'), L^{(t)}_i G(b)^*) = (G(\tilde F_i^t b'), G(b)^*) = 
  \delta_{\tilde F_i^t(b'), b}.
\]
Thus $(-, L_i^{(t)}(G(b)^*)) = (-,G(\overline E_i(b)))$, which implies the 
desired equality. Item three follows immediately from this and the definition
of $\Lambda_{\tilde w_0}$.

\textcolor{red}{
In view of the previous items, the last one follows from the fact that 
$\overline E_{i_N} \cdots \overline E_{i_1} (b) = 1$. Now Caldero claims that 
this follows from \cite{Lit}*{section 1}. Littelmann says in the intro to 
that article that this is efectively what he does, and claims it follows from 
Kashiwara's work. And so on, and so on. The only point where this is relevant
is to prove that the coefficient of $b_{(s+s',\lambda+\lambda')}$ in the next 
formula is $v^c$.
}
\end{proof}



\end{document}





Our objective in the present article is to adapt to the quantum setting a technique that 
has been a staple of algebraic geometry during the last thirty years, namely that of 
toric degenerations, and apply this technique to the study of quantum analogues of 
Schubert varieties in arbitrary flag varieties. This statement is made in the spirit of 
noncommutative algebraic geometry, in the sense that we see quantum varieties as defined 
implicitly by a noncommutative deformation of the homogeneous coordinate ring of a 
classical variety.

By degeneration we refer to the process of deforming an algebraic variety $V$ over a 
field $\k$ into a new variety $T$. In the simpler examples, this is done by finding a 
flat regular map $p: W \to \AA^1(\k)$ such that a generic fiber is 
isomorphic to $V$ but a special fiber, usually $p^{-1}(0)$, is isomorphic to $T$. If we 
have such a flat deformation of $V$ into $T$, then we can show that $V$ has some nice 
property by showing first that $T$ has the desired property, and then using flatness to 
show that this proprety extends to all fibers. A \emph{toric} degeneration is one in
which the target $T$ of the degeneration is a toric variety. The underlying combinatorial
structure of toric varieties is well documented, see for example \cite{MS}*{section II}, 
and has been aplied to study many of their homological and geometrical properties with 
great success, which makes them an ideal target for a degeneration.

The degeneration process can be carried out by purely algebraic means. In this case, the 
transfer of properties from $T$ to $V$ can be understood in algebraic terms as the 
classic idea of studying a ring $A$ by filtering it and passing to an associated graded 
ring. Of course this idea is not new in noncommutative algebra. Our contribution is to 
show that degeneration techniques are well suited to study the regularity properties of
homological nature and geometrical flavor usually found in noncommutative algebraic 
geometry, such as being AS-Cohen-Macaulay, having finite local dimension, etc. Once
we have proved this, we introduce the notion of an algebra with a \emph{quantum affine 
toric degeneration}, i.e. a graded algebra with a filtration by graded subspaces whose
associated graded algebra is a quantum affine toric variety. These quantum varieties 
were studied in our previous article \cite{RZ2}, where we showed that they share most 
of the combinatorial and homological properties of their classical counterparts, and 
hence are well-suited to be the target of a noncommutative degeneration. As an example, 
we show that Schubert subvarieties of quantum flag varieties have quantum affine toric 
degenerations and thus establish that they are AS-Cohen-Macaulay and maximal orders
in their skew-fields of fractions.

Since we were interested in applying these techniques to quantum flag and Schubert 
varieties, the works of Gonciulea and Lakshmibai \cite{GL} and Caldero \cite{C} served 
as guiding examples. In particular the last article uses techniques from quantum groups 
to prove that Schubert subvarieties of flag varieties have toric degenerations, and 
was the main inspiration to the present article.