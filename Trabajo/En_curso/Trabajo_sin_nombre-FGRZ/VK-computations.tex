%%%%%%%%%%%%%%%%%%%%%% Generalities %%%%%%%%%%%%%%%%%%5
\documentclass[11pt,fleqn]{article}
\usepackage[paper=a4paper]
  {geometry}

\pagestyle{plain}
\pagenumbering{arabic}
%%%%%%%%%%%%%%%%%%%%%%%%%%%%%%%%
\usepackage{notas}
\usepackage{tikz}
\usepackage{mathdots}

%%%%%%%%%%%%%%%%%%%%%%%%%%% The usual stuff%%%%%%%%%%%%%%%%%%%%%%%%%
\newcommand\NN{\mathbb N}
\newcommand\CC{\mathbb C}
\newcommand\QQ{\mathbb Q}
\newcommand\RR{\mathbb R}
\newcommand\ZZ{\mathbb Z}
\renewcommand\k{\Bbbk}

\newcommand\B{\mathcal B}
\newcommand\F{\mathcal F}
\newcommand\V{\mathcal V}
\newcommand\D{\mathcal D}
\newcommand\DD{\mathfrak D}
\renewcommand\H{\mathcal H}
\renewcommand\O{\mathcal O}
\newcommand\I{\mathcal I}
\newcommand\J{\mathcal J}

\newcommand\maps{\longmapsto}
\newcommand\ot{\otimes}
\renewcommand\to{\longrightarrow}
\renewcommand\phi{\varphi}
\newcommand\Id{\mathsf{Id}}
\newcommand\im{\mathsf{im}}
\newcommand\coker{\mathsf{coker}}
%%%%%%%%%%%%%%%%%%%%%%%%% Specific notation %%%%%%%%%%%%%%%%%%%%%%%%%
\newcommand\g{\mathfrak g}
\newcommand\p{\mathfrak p}
\newcommand\m{\mathfrak m}
\newcommand\gl{\mathfrak{gl}}
\renewcommand\sl{\mathfrak{sl}}
\newcommand\nn{\mathfrak{n}}


\newcommand\gen{\mathsf{gen}}
\newcommand\std{\mathsf{std}}
\newcommand\sh{\mathsf{sh}}
\newcommand\OTheta{\overline \Theta}
\newcommand\vv{\overline{v}}

\newcommand\vectspan[1]{\left\langle #1 \right\rangle}
\newcommand\interval[1]{\llbracket #1 \rrbracket}
\newcommand\Shuffle{\mathsf{Shuffle}}

\DeclareMathOperator\Frac{Frac}
\DeclareMathOperator\Specm{Specm}

\DeclareMathOperator\sym{sym}
\DeclareMathOperator\asym{asym}
\DeclareMathOperator\sg{sg}
\DeclareMathOperator\st{\mathsf{st}}
\DeclareMathOperator\n{norm}

\newcommand\bigmodule{big GT module}

%%%%%%%%%%%%%%%%%%%%%%%%%%%%%%%%%%%%%% TITLES %%%%%%%%%%%%%%%%%%%%%%%%%%%%%%
\title{Computations on the Big Module}
%\author{[gamma-structure.tex]}
\date{}

\begin{document}
\maketitle
%\vspace{-2cm}

\section{Notation}
Throughout this document we fix $n \in \NN$ and $N = \frac{n(n+1)}{2}$. We
also set $\mu = (1, 2, \ldots, n)$ and $\Sigma = \{(k,i) \mid 1 \leq i \leq k 
\leq n\}$.

\paragraph
Given $a,b,k \in \NN$ we set $\interval{a,b} = \{i \in \NN \mid a \leq i \leq 
b\}$ and $\interval{b} = \interval{1,b}$; also set $\interval{a,b}_k = \{
(k,i) \mid i \in \interval{a,b}\}$. If $I = \interval{a,b}$ we define $a(I)
= a$ and $b(I) = b$ and also set $|I| = b(I) - a(I) + 1$. 

An interval of $\Sigma$ is a set $\interval{a,b}_k \subset \Sigma$, so
$1 \leq a \leq b \leq k$. An composition of $\Sigma$ is a disjoint 
family $\I$ of intervals of $\Sigma$ whose union is $\Sigma$. Given a 
composition $\I$ of $\Sigma$ and $k \in \interval n$, we denote by $\I_k$ the
set of all intervals $\interval{a,b}_k \in \I$. Given two 
interval partitions $\I, \J$ of $\Sigma$, we say that $\J$ refines $\I$ and 
write $\J < \I$ if for each $J \in \J$ there exists $I \in \I$ such that $J 
\subset I$. 

A composition of $\mu$ is an $n$-tuple $\eta = (\eta^{(1)}, \ldots, 
\eta^{(n)})$ where $\eta^{(k)}$ is a composition of $k$; since $\mu$ is fixed
we can identify $\eta$ with the concatenation of its parts. To each
composition $\I$ of $\Sigma$ we assign a composition $\eta(\I)$ of $\mu$, 
setting $\eta^{(k)}_j$ to be the cardinality of the $j$-th interval in $\I_k$. 
This assignation is bijective, and for each composition $\eta$ of $\mu$ we 
denote by $\I(\eta)$ the unique composition of $\Sigma$ corresponding to 
$\eta$. We say that a composition $\epsilon$ refines $\eta$, and write 
$\epsilon < \eta$, whenever $\I(\epsilon) < \I(\eta)$. Compositions of $\mu$ 
form a finite poset with maximum $\mu$, so whenever we write $\eta < \mu$ we 
will mean that $\eta$ is a composition of $\mu$.

\paragraph
For each finite set $X$ we denote by $S(X)$ the set of all bijections from
$X$ to $X$. Set $S_\mu = S(\interval{1}_1) \times S(\interval{2}_2) \times
\cdots \times S(\interval{n}_n)$, which is a subset of $S(\Sigma)$. 
Every element $\sigma \in S_\mu$ can be written as $\sigma^{(1)} \sigma^{(2)}
\cdots \sigma^{(n)}$ with $\sigma^{(k)} \in S(\interval{k}_k)$. Obviously
$S(\interval{k}_k)$ is isomorphic to the symmetric group in $k$ elements,
and given a permutation $\sigma \in S_k$ we will denote the corresponding
element of $S(\interval{k}_k)$ by $\sigma^{(k)}$. If $I 
\subset \Sigma$ is an interval then we see $S(I) \subset S_\mu$ in the obvious 
way. For each composition $\I$ of $\Sigma$ we define $S_\I = \prod_{I \in \I} 
S(I) \subset S_\mu$, and for each $\eta < \mu$ we set $S_\eta = S_{\I(\eta)}$.
Since it is a product of symmetric groups, $S_\mu$ is a Coxeter group
with a well defined notion of length, and each $S_\eta$ is a parabolic
subgroup of $S_\mu$. 

\paragraph
Set $\CC^\mu = \CC^{1} \times \CC^{2} \times \cdots \times \CC^{n}$. A point 
in $\CC^\mu$ is thus an $n$-tuple $(v_1, \ldots, v_n)$ with each $v_k$ a 
vector of $k$ entries; we refer to the $v_k$'s as the \emph{rows} of $v$
and denote by $v_{k,i}$ the $i$-th entry of $v_k$. Also given an interval
$\interval{a,b}_k \subset \Sigma$ we write $v(\interval{a,b}_k) = (v_{k,a},
\ldots, v_{k,b})$. Given $\eta < \mu$ and $\I = \I(\eta)$ the $\eta$-blocks
of $v$ are the elements of the set $\{v(I) \mid I \in \I\}$.

The group $S_\mu$ acts on $\CC^\mu$ by permuting the entries, i.e. 
$\sigma(v)_{k,i} = v_{\sigma(k,i)} = v_{k,\sigma^{(k)}(i)}$. We set 
$\Lambda = \CC[x_{k,i} \mid (k,i) \in \Sigma]$, the ring of polynomial 
functions on $\CC^\mu$, and by $K = \Frac(\Lambda)$ its fraction field. The 
action of $S_\mu$ on $\CC^\mu$ induces an action on $\Lambda$ which is given 
by $\sigma(x_{k,i}) = x_{\sigma(k,i)} = x_{k,\sigma^{(k)}(i)}$, and extends to 
the fraction field. If $\eta < \mu$ then $S_\eta$ acts on $\CC^\mu$ by 
restriction, which induces $S_\eta$ actions on $\Lambda$ and $K$. Notice that
the action of $S_\eta$ only permutes elements in the same $\eta$-block.

\paragraph
We denote by $\ZZ^\mu_0 \subset \CC^\mu$ the set of points with integral 
entries and $z_{n,i} = 0$ for all $i \in \interval{n}$. Fix $\eta < \mu$ and 
let $\I = \I(\eta)$ be the corresponding composition of $\Sigma$. 
We say that $z \in \ZZ^\mu_0$ is in \emph{$\eta$-normal form}, or just normal 
form if $\eta$ is clear from the context, if each $\eta$-block of $z$ is a 
weakly descending sequence. For each $z \in \ZZ^\mu_0$ there is exactly one 
element in the orbit $S_\eta z$ which is in $\eta$-normal form, and we denote 
it by $\n_\eta(z)$. Thus the set $\n(\eta) = \{\n_\eta(z) \mid 
\ZZ^\mu_0\}$ is a complete set of representatives of the equivalence classes 
in $\ZZ^\mu_0 /S_\eta$. 

Given $z \in \n(\eta)$ we denote by $\I^\eta(z)$ the maximal refinement of 
$\I(\eta)$ such that for any $I \in \I^\eta(z)$ all the entries in the block 
$z(I)$  are equal, and we denote by $\epsilon_\eta(z)$ the corresponding 
refinement of $\eta$. This is equivalent to stating that $S_{\epsilon_\eta(z)}$
is the stabilizer of $z$ in $S_\eta$. We will say that $z$ is $\eta$-critical, 
resp. fully $\eta$-critical, if $\epsilon(z)$ is nontrivial, resp. equal to 
$S_\eta$.  

\section{The algebra $B_\eta$}
Throughout this section we fix $\eta$ a composition of $\mu$ and write
$\I = \I(\eta)$. All notions such as being in normal form, critical,  
fully critical, etc. are relative to this fixed composition unless stated
otherwise.

\paragraph
Recall that $K$ is the field of rational functions over $\CC^\mu$, and that
$S_\mu$ acts on it by permuting the variables. As usual we denote by $K \# 
S_\mu$ the smash product algebra. 

Let $I \subset \Sigma(\mu)$ be an interval. We set 
\begin{align*}
\sym_I 	
	&= \frac{1}{|I|!}\sum_{\sigma \in S_I} \sigma;
&\asym_I 	
	&= \frac{1}{|I|!}\sum_{\sigma \in S_I} \sg(\sigma)\sigma.
\end{align*} 
Since we are seeing $S_I$ as a subrgroup of $S_\mu$, these are idempotent 
elements in $K \# S_\mu$. We also set
\begin{align*}
\sym_\eta 	
	&= \prod_{I \in \I(\eta)} \sym_I;
&\asym_\eta	
	&= \prod_{I \in \I(\eta)} \asym_I.
\end{align*} 
For each $\sigma \in S_\mu$ there is a divided difference operator 
$\partial_\sigma$. We set $D_\sigma^\eta = \sym_\eta \cdot \partial_\sigma$.
By definition these operators are $\Lambda^{S_\eta}$-linear.

Let $V$ be a $K$-vector space with an $S_\mu$-equivariant action. We say that 
an additive subgroup $W \subset V$ is a $\partial_\eta$-module if it is closed
by the action of $S_\eta$ and $\partial_\sigma$ for each $\sigma \in S_\eta$.
In particular it is closed by the action of $D_\sigma^\eta$ for each $\sigma
\in S_\eta$. For example $\Lambda \subset K$ is a $\partial_\mu$-module.

\paragraph
For each interval $I = \interval{a,b}_k \subset \Sigma$ we define $D(I) = 
\{x_{k,i} - x_{k,j} \mid a \leq i < j \leq b\}$ and $\Delta_I = 
\prod_{p \in D(I)} p$. Accordingly $D(\eta) = \sqcup_{I \in \I(\eta)} D(I)$
and $\Delta_\eta = \prod_{I \in \I(\eta)} \Delta_I$. By a classical result
$\asym_I(\Lambda) = \Lambda^{S_\eta} \Delta_I$ and $\asym_\eta(\Lambda) = 
\Lambda^{S_\eta} \Delta_\eta$.

We denote by $w(I)$ the longest word of $S_I$, and by $w_\eta$ the longest 
word in $S_\eta$. If $W \subset V$ is a $\partial_\eta$ submodule and $I \in
\I(\eta)$ then the following hold for each $f \in W$.
\begin{align*}
\frac{1}{|I|!}\partial_{w(I)}(f) 
	&= \frac{1}{\Delta_I} \asym_I(f) 
	= \sym_I \left(\frac{1}{\Delta_I} f \right) \\
\frac{1}{\prod_{I \in \mathcal I(\eta)}|I|!}\partial_{w_\eta}(f) 
	&= \frac{1}{\Delta_\eta} \asym_\eta(f) 
	= \sym_\eta \left(\frac{1}{\Delta_\eta} f \right) \\
D^\eta_\sigma(f) 
	&= \sym_\eta \left(
		\frac{\partial_{\sigma^{-1}} \Delta_I}{\Delta_I} f 
	\right)
\end{align*}

\paragraph
\about{The algebras $B_\eta, A_\eta$} 
\label{L:schubert-basis}
Set 
\begin{align*}
C = \{x_{k,i} - x_{k,j} - z \mid 1 \leq i < j \leq k < n, z \in \ZZ 
\}\subset \Lambda
\end{align*} 
and set $C_\eta = C \setminus D(\eta)$. We define $B_\eta = C_\eta^{-1} 
\Lambda$. It follows from the definitions that $B_\eta \subset K$ 
is a $\partial_\eta$ submodule. Now set $\p_\eta$ to be the ideal generated by 
$D(\eta)$. This is a prime ideal and $\Lambda / \p_\eta \cong \CC[Y_1, 
\ldots, Y_r]$, where $r = |\I(\eta)|$. The ideal $\p_\eta$ extends to a prime 
ideal $B_\eta \p_\eta \subset B_\eta$, and we set $A_\eta = B_\eta / B_\eta 
\p_\eta$. We denote by $\pi_\eta: B_\eta \to A_\eta$ the obvious projection 
map.

Set $\eta! = \prod_{I \in \I(\eta)} |I|!$ and $s^\eta_\sigma = \frac{1}{\eta!}
\partial_{\sigma^{-1} w_\eta} \Delta_\eta$. By definition $\deg s^\eta_\sigma
= \ell(\sigma)$, and if $\ell(\sigma) > 0$ then $s_\sigma^\eta \in \p_\eta$.
The set $\{s^\eta_\sigma \mid \sigma \in S_\eta\}$ is a basis of $B_\eta$ as a 
$B_\eta^{S_\eta}$-module, and for each $f \in B_\eta$ we write $f_{(\sigma)}$
for its coordinate in this basis, so $f = \sum_{\sigma \in S_\eta} f_{(\sigma)}
s_\sigma^\eta$.

\begin{Lemma*}
For each $f \in B_\eta$ and each $\sigma \in S_\eta$ we have $f_{\sigma}
\equiv D_{\sigma}^\eta(f) \mod (B_\eta\p_\eta)^{S_\eta}$.
\end{Lemma*}
\begin{proof}
Applying $D^\eta_\sigma$ to $f$ and using that $\D_\sigma^\eta$ is 
$\Lambda^{S_\eta}$-linear we get
\begin{align*}
D^\eta_\sigma(f) 
	&= \sum_{\tau \in S_\eta} f_{(\tau)} D^\eta_\sigma(s_\tau^\eta)
	= \sum_{\tau \in S_\eta} 
		f_{(\tau)} \sym_\eta (\partial_\sigma s_{\tau}^\eta)
\end{align*}
If $\ell(\sigma) \geq \ell(t)$ then $\partial_\sigma s_\tau^\eta = 0$ unless
$\tau = \sigma$, in which case we get $1$; on the other hand, if 
$\ell(\sigma) < \ell(\tau)$ then $\partial_\sigma s_\tau^\eta \in \p_\eta$,
and since this ideal is stable by the action of $S_\eta$ we get
\begin{align*}
D_\sigma^\eta(f) 
	&= f_{(\sigma)} + \sum_{\ell(\tau) > \ell(\sigma)} 
		f_{(\tau)} \sym_\eta (\partial_\sigma s^\eta_\tau) 
		\equiv f_{(\sigma)} \mod (B_\eta\p_\eta)^{S_\eta}
\end{align*}
\end{proof}
It follows that the coefficient of $s_\sigma$ in the product $s_\rho s_\tau$
is $D_\sigma^\eta(s_\rho s_\tau)$. We set $c^\sigma_{\rho, \tau} = 
\pi_\eta(D_\sigma^\eta(s_\rho s_\tau))$; notice that this is zero unless
$\ell(\sigma) = \ell(\rho) + \ell(\tau)$.

In \cite{PS-chains-bruhat} Postnikov and Stanley introduce a family of
differential operators $\{\DD_\sigma^\eta \mid \sigma \in S_\eta\}$ with the
property that $\DD_\sigma(f)(0) = f_{(\sigma)}(0))$.

\section{The module $V_K$}
Set 
\begin{align*}
p_{k,i}^\pm 
	&= \prod_{j = 1}^{k \pm 1} (x_{k,i} - x_{k \pm 1, j}); 
&q_{k,i} 
	&= \prod_{j \neq i} (x_{k,i} - x_{k,j}); \\
r_{k} 
	&= x_{k,1} + \cdots + x_{k,k} - (x_{k-1,1} + \cdots + x_{k-1,k-1}) + k -1.
\end{align*}
The definitions imply that $\sigma p_{k,i}^\pm = p_{\sigma(k,i)}^\pm$ and 
$\sigma q_{k,i} = q_{\sigma(k,i)}$, while $\sigma r_k = r_k$ for all $\sigma 
\in S_\mu$.

\paragraph
\about{The action of $\ZZ^\mu_0$ on $K$}
For each $(k,i) \in \Sigma(\mu)$ set $\delta^{k,i} \in \CC^\mu$ to be the point
whose $(k,i)$-th entry is equal to $1$, and whose other entries equal zero. 
The set $\{\delta^{k,i} \mid 1 \leq i \leq k < n\}$ generates the abelian group
$\ZZ^\mu_0$. This group acts on $\Lambda$ by setting $\delta^{k,i} \cdot 
x_{l,j} = x_{l,j} + \delta_{k,l} \delta_{i,j}$ and the action extends to 
the fraction field $K$. Given $z \in \ZZ^\mu_0$ and $f \in K$ we sometimes
write $f(x+z)$ for $z \cdot f$. Notice that the actions of $\ZZ^\mu_0$ and 
$S_\mu$ are related by the formula $\sigma f(x + z) = (\sigma f)(x + 
\sigma(z))$.


\paragraph
Set $V_\CC$ to be the $\CC$-vector-space freely generated by the set $\{T(z)
\mid z \in \ZZ^\mu_0\}$, and for each $\CC$-algebra $R$ set $V_R = R \ot_\CC
V_\CC$; we refer to $T(z)$ as the tableaux corresponding to $z$, and to $z$
as the point correspoding to $T(z)$. The group $S_\mu$ acts on the set of all 
tableaux by acting on the correspoding points. Since $V_K = K \ot_\CC V_\CC$
is a tensor product of $S_\mu$-modules, it is again an $S_\mu$-module. 
Furthermore $V_K$ is a $\gl(n,\CC)$-module with action given by
\begin{align*}
E_{k,k+1} T(z) 
	&= -\sum_{j=1}^k \frac{p^+_{k,i}(x+z)}{q_{k,i}(x+z)}T(z + \delta^{k,i})
	& (1 \leq k \leq n-1)\\
E_{k+1,k} T(z) 
	&= \sum_{j=1}^k \frac{p^-_{k,i}(x+z)}{q_{k,i}(x+z)}T(z - \delta^{k,i})
	& (1 \leq k \leq n-1) \\
E_{k,k} T(z)
	&= r_k(x+z) T(z)
	& (1 \leq k \leq n).
\end{align*}
A direct computation shows that the action of $\gl(n,\CC)$ is 
$S_\mu$-equivariant. Furthermore for each $c \in \Gamma$ there exists a 
symmetric polynomial $\gamma \in \Lambda^{S_\mu}$, given by Zhelobenko's
isomorphism, such that $c T(z) = \gamma(x+z) T(z)$.

\paragraph
\about{Operators on $V_K$}
\label{L:pre-formulas}
Since $V_K$ is a $K$-vector space with an equivariant $S_\mu$ action we can
apply divided difference operators to it. We will now rewrite the operators
$E_{k,k+1}, F_{k+1,k}$ in a way better suited to the study of singular modules.
Let $I = \interval{a,b}_k \subset \Sigma$ be an interval. We set 
\begin{align*}
e_{k,I}^+
	&= \frac{\prod_{j = 1}^{k+1} (x_{k,a(I)} - x_{k+1,j})}
		{\prod_{(k,j) \notin I} x_{k,a(I)} - x_{k,j}} 
&
e_{k,I}^-
	&= \frac{\prod_{j = 1}^{k-1} (x_{k,b(I)} - x_{k+1,j})}
		{\prod_{(k,j) \notin I} x_{k,b(I)} - x_{k,j}}.
\end{align*}
Notice that if $I \in \I(\eta)$ then $e_{k,I}^+, e_{k,I}^- \in B_\eta$.
We also set $\alpha(I) = (b,b-1, \cdots, a)^{(k)}, \beta(I) = 
(a, a(I)+1, \cdots, b)^{(k)} \in S_I$.

\begin{Lemma*}
Let $\eta < \mu$ and suppose $z$ is in normal form. Then for each $1 
\leq k < n$ we have
\begin{align*}
E_{k,k+1} T(z)
	&= - \sum_{I \in \I_k^\eta(z)} \partial_{\alpha(I)} 
		\left(
			e_{k,I}^+(x+z) T(z + \delta^{k,a(I)})
		\right) \\
E_{k+1,k} T(z)
	&= \sum_{I \in \I_k^\eta(z)} \partial_{\beta(I)}
		\left(
			e_{k,I}^-(x+z) T(z - \delta^{k,b(I)})
		\right).
\end{align*}
\end{Lemma*}
This is analogous to \cite{EMV-orthogonal}*{Proposition 6}.


\begin{proof}
First notice that if $I \in \I_k^\eta(z)$ then every element of $S_I$ leaves 
$z$ fixed, so $T(z) = \sym_I T(z)$, which implies that $\sym_I E_{k,k+1} T(z)
= E_{k,k+1} T(z)$. By definition
\begin{align*}
\sym_I e_{k,j}^+(x+z) T(z+\delta^{k,j}) 
&= \begin{cases}
e_{k,j}^+(x+z) T(z+\delta^{k,j}) & j \notin I; \\
\sym_I(e_{k,a(I)}^+(x+z) T(z + \delta^{k, a(I)})) & j \in I.
\end{cases}
\end{align*}
Thus 
\begin{align}
\label{action}
\tag{1}
E_{k,k+1} T(z)
	&= \left( 
		\prod_{I \in \I_k^\eta(z)} \sym_I 
	\right) (E_{k,k+1} T(z))\\
	&= \sum_{I \in \I_k^\eta(z)} 
		|I| \sym_I(e_{k,a(I)}^+(x+z) T(z+ \delta^{k, a(I)}))
\end{align}
Fix an interval $I = \interval{a,b}_k$. We rewrite the term 
$\sym_I(e_{k,a}^+(x+z) T(z+ \delta^{k, a}))$ as a divided difference
\begin{align*}
\frac{1}{|I|!}
	\partial_{w(I)}(\Delta_I e_{k,a}^+(x+z) T(z+ \delta^{k, a}))
\end{align*}
Notice that $\Delta_I e_{k,a}^+(x+z) = \Delta_{I \setminus \{(a,k)\}} 
e_{k,I}^+(x+z)$, so we can further rewrite this as
\begin{align*}
\frac{1}{|I|!}
	&(\partial_{\alpha(I)} \circ \partial_{w(I \setminus \{a(I)\})})
	(\Delta_{I\setminus \{a(I)\}} e_{k,I}^+(x+z) T(z+ \delta^{k, a(I)})) \\
&= \frac{1}{|I|}
	\partial_{\alpha(I)} (\sym_{I \setminus \{a(I)\}}
	(e_{k,I}^+(x+z) T(z+ \delta^{k, a(I)})))\\
&= \frac{1}{|I|} \partial_{\alpha(I)} 
	(e_{k,I}^+(x+z) T(z+ \delta^{k, a(I)})),
\end{align*}
where the last equality follows from the fact that $e_{k,I}^+(x+z) 
T(z+ \delta^{k, a(I)})$ is invariant by any permutation in $S_I$ that leaves
$(k,a(I))$ fixed. Replacing this in \ref{action} we obtain the formula for
$E_{k,k+1}T(z)$. The proof of the other formula is analogous.
\end{proof}

\paragraph
\label{L:function-tableaux}
For each $z \in \ZZ^\mu_0$ write $D_\sigma^\eta(z)$
for $D_\sigma^\eta(T(z))$. As shown in \cite{RZ-singular-characters} the set 
\begin{align*}
\B(\eta) =
\left\{D_\sigma^\eta(z) \mid z \in \n_\eta(\ZZ^\mu_0), \sigma \in 
\Shuffle_{\epsilon(z)}^\eta\right\}
\end{align*}
is a $K$-basis of $V_K$. Furthermore there exist polynomials $\{
s_{\sigma}^* \in \Lambda \mid \sigma \in S_\eta\}$ such that
\begin{align*}
T(\tau(z))
	&= \sum_{\sigma \in S_\eta} \tau(s_\sigma^*) D_\sigma^\eta(z).
\end{align*}
The polynomials $s_\sigma^*$ are determined implicitly by the condition
$D_\tau^\eta(s_\sigma^*) = \delta_{\sigma, \tau}$, so in view of Lemma
\ref{L:schubert-basis} $s_\sigma^* \equiv s_\sigma \mod 
\p_\eta^{S_\eta}$. We denote by $L_\eta$ the $B_\eta$-lattice generated
by $\B(\eta)$. Set $V_\eta = L_\eta/\p_\eta L_\eta$. We will prove that it is 
a $U(\gl(n,\CC))$-module and will give explicit equations for the action of
$\gl(n,\CC)$ on it. 
\begin{Lemma*}
Let $f \in B_\eta$ and $z \in \n_\eta(\ZZ^\mu_0)$. Then for each $\sigma \in
S_\eta$
\begin{align*}
D_\sigma^\eta(f T(z))
	&= \sum_{\tau \in S_\eta} 
		D_\sigma^\eta(f s_\tau^*) D_\tau^\eta(z)
	\equiv \sum_{\tau \in S_\eta} 
		\left(
			\sum_{\rho \in S_\eta} c_{\rho, \tau}^\sigma
				D_\rho^\eta(f)
		\right) D_\tau^\eta(z)
	\mod \p_\eta L_\eta.
\end{align*}
\end{Lemma*}
\begin{proof}
The equality follows by replacing $T(z)$ with $\sum_{\tau} s_\tau 
D_\tau^\eta(z)$. To obtain the congruence replace $f$ with $\sum_\rho 
f_{(\rho)} s_\rho$ and apply Lemma \ref{L:schubert-basis}; instead of
the coefficients $c^\sigma_{\rho, \tau}$ we get $D_\sigma^\eta(s_\rho 
s_\tau^*)$, but since $s_\tau^* \equiv s_\tau \mod \p_\eta^{S_\eta}$
this equals $c^\sigma_{\rho, \tau}$ modulo $\p_\eta$.
\end{proof}

\begin{Definition*}
Given $\sigma, \tau \in S_\eta$ we define $D_{\sigma,\tau}^\eta: B_\eta \to
B_\eta$ as $D_{\sigma,\tau}^\eta(f) = \sum_{\rho \in S_\eta} 
c_{\rho, \tau}^\sigma D_\rho^\eta(f)$. 
\end{Definition*} 

Notice that Stanley and Postnikov have analogous operators $\DD_{\sigma,\tau}
= \sum_{\rho} c_{\rho, \tau}^\sigma \DD_\rho$. Since $\pi \circ D_\sigma^\eta 
= \pi \circ \DD_\sigma$ we also have that $\pi \circ D_{\sigma,\tau}^\eta 
= \pi \circ \DD_{\sigma, \tau}$.

\begin{Proposition}
\label{P:module-congruence}
The following congruences hold in $L_\eta$ modulo $\p_\eta L_\eta$
\begin{align*}
E_{k,k+1} D_\sigma^\eta(z)
	&\equiv - \sum_{I \in \I_k^\eta(z)}
	\sum_{\tau \in S_\eta}
	\DD^\eta_{\sigma\alpha(I),\tau}
		\left(
			e_{k,I}^+
		\right)(x+z) D_\tau^\eta(z+\delta^{k,a(I)}); \\
E_{k+1,k} D_\sigma^\eta(z)
	&\equiv \sum_{I \in \I_k^\eta(z)} 
		\sum_{\tau \in S_\eta} 
		\DD_{\sigma\beta(I),\tau}^\eta \left(e_{k,I}^- \right)
		(x+z) D_\tau^\eta(z - \delta^{k,b(I)});\\
E_{k,k} D_\sigma^\eta(z)
	&\equiv r_k(x+z) D_\sigma^\eta(z) +
		\sum_{\ell(\tau) < \ell(\sigma)} 
		\DD_{\sigma, \tau}^\eta(r_k)(x+z) 
			D_\tau^\eta(z).
\end{align*}
Furthermore, for each $c \in \Gamma$ we have
\begin{align*}
c D_\sigma^\eta(z)
	&\equiv \gamma_c(x+z) D_\sigma^\eta(z) +
		\sum_{\ell(\tau) < \ell(\sigma)} 
			\DD_{\sigma, \tau}^\eta(\gamma_c)(x+z) 
			D_\tau^\eta(z).
\end{align*}
\end{Proposition}
\begin{proof}
We work first in $V_K$. If $c \in \Gamma$ then $c T(z) = \gamma(x+z) T(z)$, so
applying Lemma \ref{L:function-tableaux} and passing to the quotient we obtain
\begin{align*}
c D_\sigma^\eta(z)
	&= \overline \gamma(x+z) D_\sigma^\eta(z) +
		\sum_{\ell(\tau) < \ell(\sigma)} 
			\pi(D_{\sigma, \tau}^\eta(\gamma(x+z))) 
			D_\tau^\eta(z).
\end{align*}
Now $\pi(D_{\sigma, \tau}^\eta(\gamma(x+z))) = \pi(\DD_{\sigma,\tau}^\eta(
\gamma(x+z))$. Since the $\DD_{\sigma,\tau}$ are differential operators and
$x+z$ is an affine change of variables we see that $\pi(\DD_{\sigma,\tau}^\eta
(\gamma(x+z)) = \overline{\DD_{\sigma,\tau}^\eta(\gamma)}(x+z)$, and this 
proves the last equality. Since $E_{k,k} \in \Gamma$ the third formula is just 
a particular case of this.

From Lemma \ref{L:pre-formulas} we get
\begin{align*}
E_{k,k+1} D_\sigma^\eta T(z)
	&= - \sum_{I \in \I_k^\eta(z)} D_\sigma^\eta (\partial_{\alpha(I)} 
		\left(
			e_{k,I}^+(x+z) T(z + \delta^{k,a(I)})
		\right))
\end{align*}
Now $\sigma \in \Shuffle_{\epsilon(z)}^\eta$ and $\alpha(I) \in 
S_{\epsilon(z)}$, so by \cite{BB-coxeter-book}*{Proposition 2.4.4}
$\ell(\sigma \alpha(I)) = \ell(\sigma) + \ell(\alpha(I))$ and
$D_\sigma^\eta \partial_{\alpha(I)} = D^\eta_{\sigma \alpha(I)}$. Thus
\begin{align*}
E_{k,k+1} D_\sigma^\eta T(z)
	&= - \sum_{I \in \I_k^\eta(z)} D_{\sigma\alpha(I)}^\eta  
		\left(
			e_{k,I}^+(x+z) T(z + \delta^{k,a(I)})
		\right).
\end{align*}
By definition $e_{k,I}^+(x+z) \in B_\eta$ for each $I \in \I_\eta(z)$, so
we can apply again Lemma \ref{L:function-tableaux} and pass to the quotient
to obtain a formula involving operators of the form $D_{\sigma\alpha(I), 
\tau}^\eta$, and using a similar argument as before we can replace them with
differential operators $\DD_{\sigma\alpha(I), \tau}^\eta$ and obtain the 
first formula in the statement. A simmilar proof works for the second one.
\end{proof}

\section{The GT module associated to an arbitrary character}

\paragraph
To each $v \in \CC^\mu$ we associate a composition $\eta = \eta(v)$ of $\mu$ 
as follows. For any $k \in \interval{n-1}$ form a graph with vertices 
$\interval{k}$, and put an edge between $i$ and $j$ if and only if $v_{k,i} - 
v_{k,j} \in \ZZ$; the resulting graph is the disjoint union of complete 
graphs, and we set $\eta^{(k)}$ to be the cardinalities of each connected 
component arranged in descending order. Finally we set $\eta(v) = (\eta^{(1)}, 
\ldots, \eta^{(n-1)}, 1^n)$, where $1^n$ denotes the composition of $n$ 
consisting of $n$ ones. Thus if $v$ is generic then $\eta^{(k)}(v) = 1^k$, and 
if it is singular then $\eta(v)$ will have at least one part larger than $1$. 
Notice that $\eta(v) = \eta(\sigma(v) + z)$ for any $\sigma \in S_\eta$ and 
$z \in \ZZ^\mu_0$, and that by replacing $v$ with a suitable element $\vv = 
\sigma(v) + z$ we can characterize $S_\eta$ as the stabilizer of $\vv$; we 
call any such element a \emph{fully critical} representative of $v$. By 
definition of $B_\eta$, evaluation at $\vv$ defines a $1$-dimensional 
representation of $B_\eta$ which we will denote by $\CC_{\vv}$. Also notice 
that if $\vv'$ is another fully critical representative of $v$ then translation
by $\vv' - \vv$ is a well defined automorphism of $B_\eta$ which in turn 
induces an isomorphism $\CC_{\vv} \cong \CC_{\vv'}$

\begin{Definition*}
Let $v \in \CC^\mu$, let $\eta = \eta(v)$ and let $\vv$ be a fully critical 
representative of $v$. We set $V(T(v)) = \CC_{\vv} \ot_{B_\eta} L_\eta$, and 
for each $\sigma \in \eta$ and each $z \in \ZZ^\mu_0$ we write $D_\sigma(\vv 
+ z) = 1_{\vv} \ot_{B_\eta} D_\sigma^\eta(z)$.
\end{Definition*}

By the last remark before the definition, $V(T(v))$ is well defined up to
isomorphism. We set some notation for the rest of the document. Let $v \in 
\CC^\mu$. Given $\sigma, \tau \in S_\mu$ we write $\DD_\sigma^v = ev_v \circ 
\DD_\sigma^\eta$ and $\DD_{\sigma, \tau}^v = ev_{v} \circ \DD_{\sigma, \tau}$.
Since $ev_{\vv}(\p_{\eta(v)}) = 0$ by definition, we get the following 
from Proposition \ref{P:module-congruence}.

\begin{Theorem}
\label{T:big-module}
Let $v \in \CC^\mu$ and let $\vv$ be a fully critical representative of
$v$. The action of $\gl(n,\CC)$ on $V(T(v))$ is given by the following
formulas
\begin{align*}
E_{k,k+1} D_\sigma(\vv + z)
	&= - \sum_{I \in \I_k^\eta(z)}
	\sum_{\tau \in S_\eta}
	\DD^{\vv +z}_{\sigma\alpha(I),\tau}
		\left(
			e_{k,I}^+
		\right) D_\tau (\vv + z+\delta^{k,a(I)}); \\
E_{k+1,k} D_\sigma(\vv + z)
	&= \sum_{I \in \I_k^\eta(z)} 
		\sum_{\tau \in S_\eta} 
		\DD_{\sigma\beta(I),\tau}^{\vv+z} \left(e_{k,I}^- \right)
		D_\tau(\vv + z - \delta^{k,b(I)});\\
E_{k,k} D_\sigma(\vv + z)
	&= r_k(\vv+z) D_\sigma(\vv+z) +
		\sum_{\ell(\tau) < \ell(\sigma)} 
		\DD_{\sigma, \tau}^{\vv + z}(r_k) 
			D_\tau(\vv + z).
\end{align*}
Furthermore, for each $c \in \Gamma$ we have
\begin{align*}
c D_\sigma(\vv + z)
	&= \gamma_c(\vv+z) D_\sigma(\vv + z) +
		\sum_{\ell(\tau) < \ell(\sigma)} 
			\DD_{\sigma, \tau}^{\vv+z}(\gamma_c) 
			D_\tau (\vv+ z).
\end{align*}
\end{Theorem}

\paragraph
\label{L:invariants-dual-basis}
Let $\eta < \mu$. By \cite{PS-chains-bruhat}*{Corollary 6.1} the ideal $I_\eta
= \bigcap_{\sigma \in S_\eta} \ker \DD_\sigma^0 \subset \Lambda$ is exactly 
the ideal generated by $S_\eta$-symmetric polynomials with nonzero constant 
term, and the induced functionals $\{\DD_\sigma^0: \Lambda / I \to \CC \mid 
\sigma \in S_\eta\}$ form a homogeneous basis of the graded dual of 
$\Lambda/I$, with $\deg \DD_\sigma^0 = - \ell(\sigma)$. 
For each $v \in \CC^\mu$ we set $\m_{v} \subset \Gamma$ to be the ideal of 
elements $c \in \Gamma$ such that $\gamma_c(v) = 0$.

\begin{Lemma*}
Let $\epsilon < \eta$ and suppose $v \in \CC^\mu$ is such that its stabilizer
in $S_\eta$ is $S_\epsilon$. 
\begin{enumerate}[(a)]
\item The set $\{\DD_\rho^0 \mid \rho \in \Shuffle_\epsilon^\eta\}$ is a 
	homogeneous basis of the graded dual of the invariant space
	$(\Lambda/I)^{S_\epsilon}$.
\item The map $c\in \Gamma \mapsto \phi_v(c)
	= \overline{\gamma_c(x+v)} \in (\Lambda/I)^{S_\epsilon}$ is surjective.
\item For each $\sigma \in \Shuffle_{\epsilon}^\eta$ there exists $c_\sigma \in
\m_v$ such that $\DD_\sigma^0(c_\sigma^{\ell(\sigma)}) = 1$.
\end{enumerate}
\end{Lemma*}
\begin{proof}
Since $I_\eta$ is $S_\epsilon$-invariant $(\Lambda/I_\eta)^{S_\epsilon}$ is the
image of $\Lambda^{S_\epsilon}$ in the quotient. Now for $p \in 
\Lambda^{S_\epsilon}$ and $\sigma$ not an $\epsilon$-shuffle we have 
$\partial_\sigma p = 0$ so $\DD_\sigma^0(p) = (\partial_\sigma p)(0) = 0$.
Hence $\displaystyle (\Lambda/I_\eta)^{S_\epsilon} \subset \bigcap_{\sigma 
\notin \Shuffle_\epsilon^\eta} \ker \DD_\sigma^0$. Since both spaces have 
dimension $\eta!/\epsilon!$ the inclusion must be an equality, and the dual 
space of $(\Lambda/I_\eta)^{S_\epsilon}$ is generated by the restrictions of
the linear functionals $\{\DD_\sigma^0 \mid \sigma \in \Shuffle_\epsilon^\eta
\}$. This proves the first item. The second item is classical according to
Soergel; it probably goes back at least to Weyl in The Classical Groups, part 
3 of Chapter II.

Recall that by definition $\DD^0_\sigma$ is a positive multiple of the sum of 
differential operators $m_C$, where $C = (e < \tau_1 < \cdots < 
\tau_{\ell(\sigma)} = \sigma)$ is a saturated chain in the Bruhat order and 
$m_C$ is the product of all $\partial_{a_i} - \partial_{b_i}$ where 
$\tau_{i+1} = \tau_i (a_i,b_i)$, followed by evaluation at zero. If $f(v) = 0$ 
then $m_C(f^{\ell(\sigma)})$ is the product of all $\partial_{a_i}(f) - 
\partial_{b_i}(f)$ evaluated at $0$. By the previous two items we can choose
$f \in (\Lambda/I)^{S_\epsilon}$ such that $D_{(i,i+1)}^0(f) = 1$ for all 
$(i,i+1) \in \Shuffle_\epsilon^\eta$, so $m_C(f) \geq 0$ and hence 
$\DD_\sigma^0(f^j) \geq 0$. Now by the chain property of shuffles 
\cite{BB-coxeter-book}*{Theorem 2.5.5} there is at least one chain $C$
from $e$ to $\sigma$ consisting of $\epsilon$-shuffles; we claim that in this
case $m_C(f^j) \neq 0$. Indeed, if $\tau_{i+1} = \tau_i (a_i,b_i)$ then
$(a_i, b_i) v \neq v$, since otherwise $\tau_{i+1}$ would not be of minimal
length in $\tau_{i+1} S_\epsilon$, and hence there exists an $\epsilon$-shuffle
$(j,j+1)$ such that $a_i \leq j < b_i$, which in turn implies that 
\begin{align*}
	(\partial_{a_i} - \partial_{b_i})(f)(0)
		&= \sum_{r = a_i}^{b_i-1} (\partial_{r} - \partial_{r+1})(f)(0)
		> 0,
\end{align*}
so $m_C(f^j)(0) > 0$ and hence $\DD_\sigma^0(f^j) > 0$. Multiplying $f$ by
a suitable constant we get $\DD_\sigma^0(f^j) = 1$. Now take $c \in \Gamma$
such that $\phi_v(c) = \overline f$. Since $f(v) = 0$ we see that $c \in 
\m_v$, so we are done.
\end{proof}


\begin{Proposition}
Let $\vv \in \CC^\mu$ be $\eta$-critical, let $z \in \ZZ^\mu_0$ be in 
$\eta$-normal form. Set $\epsilon = \epsilon(z)$ and $v = \vv + z$. 
For generic elements $c \in \Gamma$ we have $(c-\gamma_c(v))^{\ell(\sigma)} 
D_\sigma(v) \neq 0$. In particular the Jordan form of the endomorphism of 
$V(T(\vv))[\m_{v}]$ induced by $c$ has a unique largest block of size 
$\ell(w_\epsilon)$, where $w_\epsilon$ is the $\epsilon$-shuffle associated to 
the longest word $w_\eta$. 
\end{Proposition}
\begin{proof}
Let $f \in (\Lambda/I)^{S_\epsilon}$. The proof of the previous lemma shows 
that $\DD_\sigma^0(f^{\ell(\sigma)} - f(v))$ is a nontrivial polynomial in 
$(\partial_i - \partial_{i+1})(f)(0)$ where $i$ runs over all $(i,i+1)$ which 
are $\epsilon$-shuffles. This implies that the set of all $f$ such that 
$\DD_\sigma^0(f^{\ell(\sigma)}) \neq 0$ is a nonempty Zarisky open set of 
$(\Lambda/I)^{S_\epsilon}$, which we denote $V_\sigma$. If $c \in \Gamma$ 
is such that the class of $\gamma_c - \gamma_c(v)$ lies in $V_\sigma$
then $(c- \gamma_c(v))^{\ell(\sigma)} D_\sigma(v) = \DD_\sigma^0(c - 
\gamma_c(v)^{\ell(\sigma)}) D_e(v) \neq 0$.

Now take $c \in \Gamma$ such that $\gamma_c - \gamma_c(v) \in V_{w_\epsilon}$.
Then $(c-\gamma_c)^j D_{w_\epsilon}(v)$ is nonzero if and only if $j \leq 
\ell(w_\epsilon)$, which shows that the Jordan form of $c$ over 
$V(T(\vv))[\m_v]$ must have a block of size $\ell(w_\epsilon)$. Since 
$w_\epsilon$ is the unique longest element in $\Shuffle_\epsilon^\eta$, we
get $(c-\gamma_c)^{\ell(w_\epsilon)} D_\sigma(v) = 0$ for any $\sigma < 
w_\epsilon$ so there can be no other Jordan block of size $\ell(w_\epsilon)$
in the Jordan form of $c$.
\end{proof}

\paragraph
\about{Gelfand-Tsetlin components}
\label{L:most-derived-tableaux}
Fix $w = v + z$ and $\epsilon = \epsilon(w) \subset \S_\eta$. Denote by 
$\omega \in S_\eta^\epsilon$ the $\epsilon$-shuffle corresponding to
the longest word of $S_\eta$.

\begin{Lemma*}
Let $x = \sum_{\sigma \in S_\eta^\epsilon} a_\sigma D_\sigma(w)$. Then 
$\Gamma x = V(T(v))[\chi_w]$ if and only if $a_{\omega} \neq 0$.
\end{Lemma*}  
\begin{proof}
If $a_\omega = 0$ then $\Gamma x$ is contained in the $\CC$-span of derived
tableaux $D_\sigma(w)$ with $\sigma < \omega$, so the condition $a_\omega \neq
0$ is clearly necessary. Now assume $a_\omega \neq 0$. For each $\sigma \in 
S_\eta^\epsilon$ let $c_\sigma \in \Gamma$ be such that $\DD_{\omega, \tau}
(c_\sigma) = \delta_{\sigma, \tau}$. Then $c_\sigma x = a_\omega D_\sigma(w)$
plus a linear combination of tableaux $D_\tau(w)$ with $\tau < \sigma$. Thus
each $D_\sigma(w) \in \Gamma x$.
\end{proof}

\paragraph
\label{L:domination}
Fix a fully critical seed $\overline v \in \CC^\mu$ with $\eta = 
\eta(\overline v)$. We say that the intervals $\interval{a,b}_{k+1}, 
\interval{c,d}_{k} \in \I$ are related if $\overline v_{k+1, a} - \overline 
v_{k,c} \in \ZZ$. Notice that this implies $\overline v_{k+1, i} - v_{k,j} 
\in \ZZ$ for all $i \in \interval{a,b}$ and all $j \in \interval{c,d}$; also 
notice that an interval at level $k$ is related to at most one interval at 
level $k+1$. A family of intervals is a sequence $\{I_k, I_{k+1}, \cdots, 
I_{k+r}\}$ such that $I_j$ is related to $I_{j+1}$. We say two intervals are
related if they belong to the same maximal family of intervals; clearly each
interval belongs to a unique maximal family. We say two entries are related
if they lie in related intervals. Clearly if $(k,i)$ and $(l,j)$ are related 
then $v_{k,i} - v_{l,j} \in \ZZ$.

\begin{Definition*}
Let $w, w' \in \overline v + \n(\eta)$. We say that $w$ \emph{dominates}
$w'$, and denote this by $w \searrow w'$ if $w_{k,i} - w'_{k,i} \in 
\ZZ_{\geq 0}$ for all $(k,i) \in \Sigma$ 
and the following condition holds: for each $\interval{a,b}_k \in \I(\eta)$,
if there exists $t \in \interval{a,b}$ such that $w_{k,t} > w_{k,t+1}$ then
$w_{k,i-1} > w_{k,i}$ for all $i \leq t$.
\end{Definition*}

We refer to the last condition in the definition as monotonicity on intervals.
Visually this can be represented as follows. Setting $z = w - \overline v$ 
and $z' = w' - \overline v$ we consider the graphs of $z$ and $z'$ over a 
fixed interval $I \in \I$, in red and blue respectively. In these graphs
$w$ dominates $w'$ in the first and third case, but not in the second.

\begin{tikzpicture}
\draw (0,-0.2) -- (0,3);
\draw (-0.2,0) -- (4,0);

\draw[red] (0,2.5) -- (1, 2.5) -- (1,2) -- (2,2) -- (2,1.5) 
	-- (3,1.5) -- (3,0.5) -- (4,0.5);
\draw[blue] (0,1.9) -- (1,1.9) -- (1,0.9) 
	-- (3,0.9) -- (3,0.4) -- (4,0.4);
\end{tikzpicture}
\begin{tikzpicture}
\draw (0,-0.2) -- (0,3);
\draw (-0.2,0) -- (4,0);

\draw[red] (0,2.5) -- (2,2.5) -- (2,1.5) 
	-- (3,1.5) -- (3,0.5) -- (4,0.5);
\draw[blue] (0,1.9) -- (1,1.9) -- (1,0.9) 
	-- (3,0.9) -- (3,0.4) -- (4,0.4);
\end{tikzpicture} 
\begin{tikzpicture}
\draw (0,-0.2) -- (0,3);
\draw (-0.2,0) -- (4,0);

\draw[red] (0,2.5) -- (1, 2.5) -- (1,2) -- (2,2) -- (2,1.5) 
	-- (3,1.5) -- (3,0.5) -- (4,0.5);
\draw[blue] (0,2.4) -- (1,2.4) -- (1,0.9) 
	-- (3,0.9) -- (3,0.4) -- (4,0.4);
\end{tikzpicture}

With this aide it is easy to see that domination is an order relation. Also, 
if $v, v' \in \n(\eta)$ with $v_{k,i} \geq v'_{k,i}$ and $v_{k,i}$ is 
noncritical then $v \searrow v'$.

\begin{Lemma*}
Let $v,v' \in \overline v + \ZZ^\mu_0$, and assume that $v'$ is in normal
$\eta$-form. 
\begin{enumerate}
\item 
\label{i:normal}
If $v \searrow v'$ then $v$ is in $\eta$-normal form and every 
$\epsilon(v')$-shuffle is an $\epsilon(v)$-shuffle.

\item
\label{i:common}
If $v, v'$ are in $\eta$-normal form then there exists a noncritical $v''
\in \overline v + \n(\eta)$ such that $v'' \searrow v, v'$ and 
$\Omega^+(v) \cap \Omega^+(v') \subset \Omega^+(v'') \subset \Omega^+(v) \cup 
\Omega^+(v')$. 
\end{enumerate}
\end{Lemma*}
\begin{proof}
It is clear from the definition that if $v'$ is in normal $\eta$-form then
so is $v$, and that $\I_\eta(v) < \I_\eta(v')$. Now $\sigma \in S_\eta$
is an $\epsilon(v')$-shuffle if and only if it is monotonous on each interval
of $\I_\eta(v')$, which implies that it is monotonous in each interval of 
$\I_\eta(v)$ and hence it is an $\epsilon(v)$-shuffle. This proves item 
\ref{i:normal}.

Set $\tilde v$ with $\tilde v_{k,i} = \max\{v_{k,i}, v'_{k,i}\}$ for all $(k,i)
\in \Sigma$. It is easy to check that $\Omega^+(\tilde v)$ satisfies the
inclusions in the statement, so we only need to prove that we can find a 
noncritical $v''$ such that $v''_{k,i} \geq \tilde v_{k,i}$ for all $(k,i)
\in \Sigma$ and $\Omega^+(v'') = \Omega^+(\tilde v)$. Given $(k,i), (l,j)$
two related entries we write $(k,i) \rightarrow_1 (l,j)$ if one of the 
following conditions holds:
\begin{itemize}
\item $k = l$ and $j = i+1$;
\item $k = l+1$ and $\tilde v_{k,i} \geq \tilde v_{l,j}$, or equivalently 
$(k,i,j) \in \Omega^+(\tilde v)$;
\item $k = l-1$ and $\tilde v_{k,i} > \tilde v_{l,j}$ or equivalently 
$(k+1,j,i) \notin \Omega^+(\tilde v)$.
\end{itemize}
By definition, if we have a sequence $v_{k_1,i_1} \rightarrow_1 v_{k_2, i_2}
\rightarrow_1 \cdots \rightarrow_1 v_{k_r, i_r} \rightarrow_1 v_{k_1, i_1}$
then $v_{k_1, i_1} = v_{k_2, i_2} = \cdots = v_{k_r,i_r}$, which implies that
$k_1 \geq k_2 \geq \cdots \geq k_r$, and so $k_1 = k_2 = \cdots = k_r$ and
finally $i_1 = \cdots = i_r$, which is impossible. Thus the system of 
inequalities $\{v''_{k,i} > v''_{l,j} \mid(k,i) \rightarrow_1 
(l,j)\}$ has a solution. By construction $\Omega^+(v'') = \Omega^+(\tilde v)$
and $v''$ is noncritical, so we are done.
\end{proof}


\begin{Lemma}
\label{L:sequence}
Let $v,v' \in \overline v + \ZZ^\mu_0$ and suppose $v \searrow v'$. 
There exists a sequence $v = v^{(0)} \searrow v^{(1)} \searrow \cdots \searrow 
v^{(r)} = v'$ with the following properties.
\begin{enumerate}[(a)]
\item 
\label{i:ik}
For each $s \in \interval{r}$ there exists $(k_s, i_s) \in \Sigma$ such 
that $v^{(s)} = v^{(s-1)} - \delta^{k_s,i_s}$.

\item 
\label{i:omega}
$\Omega^+(v) \cap \Omega^+(v') \subset \Omega^+(v^{(s)}) \subset
\Omega^+(v) \cup \Omega^+(v')$ for each $s \in \interval{r}$.

\item 
\label{i:interval}
Set $I \in \I(v^{(s)})_{k_s}$ to be the unique interval containing $(k_s,
i_s)$. Then $a(I) = i_s$ and $\omega^{(s)} = \omega^{(s+1)} \alpha(I)$. 
\end{enumerate}

\end{Lemma}
\begin{proof}
If $v = v'$ then the statements are trivial, so we can assume that $v \neq v'$,
or equivalently that the set $\Sigma' = \{(r,s) \mid v_{r,s} - v'_{r,s} > 0\}$
is not empty. Let $\Sigma'' \subset \Sigma'$ be the subset consisting of those
$(r,s)$ with the following property: if $(t,l)$ is in $\Sigma$ and related to 
$(r,s)$, then $v_{r,s} - v_{t,l} \leq 0$. Such elements always exist, since 
they are the minimal elements in the intersection of a maximal family of 
related entries with $\Sigma'$. Finally choose $(k,i) \in \Sigma''$ with
$k$ minimal and $i$ maximal for that $k$, and set $v^{(1)} = v - 
\delta^{k,i}$. We claim that the lemma holds for $s = 1$.

First let us check that $v^{(1)} \searrow v'$. Since $v_{k,i} > v'_{k,i}$ by 
the choice of $v_{k,i}$, it is clear that $v^{(1)}_{r,s} \geq v'_{r,s}$ for all
$(r,s) \in \Sigma$. Since we have only changed the $(k,i)$-th entry of $v$, we
only need to check the condition for the interval containing $(k,i)$, say
$\interval{a,b}_k$. If $i < j \leq b$ then $v_{k,i} \geq v_{k,j}$ since $v$
is in $\eta$-normal form, and the choice of $(k,i)$ implies that $v_{k,j} = 
v'_{k,j}$. Thus if there exists $t \in \interval{a,b}$ such that $v^{(1)}_{k,t}
> v'_{k,t}$ then $t \in \interval{a,i}$, and since $v_{k,i} > v'_{k,i}$
we have that $v_{k,a} > v_{k,a+1} > \cdots > v_{k,i-1} > v_{k,i} -1$, which
is precisely the monotonicity condition for $v^{(1)}$ over the interval $I$.
Thus $v^{(1)} \searrow v'$ and since $v'$ is in $\eta$-normal form so is 
$v^{(1)}$. Thus $v^{(1)}$ satsfies item \ref{i:ik}.

Let us check item \ref{i:omega} for $v^{(1)}$. Once again, since $v^{(1)}$ 
differs from $v$ only in the $(k,i)$-th entry, we only have to check the
inclusions for elements of the form $(k+1,j,i)$ and $(k,i,j)$. For simplicity 
we denote $X = \Omega^+(v) \cap \Omega^+(v')$ and $Y = \Omega^+(v) \cup 
\Omega^+(v')$
\begin{itemize}
\item Suppose $(k+1,j,i) \in X$. Then $v_{k+1,j} - v_{k,i} \geq 0$ and hence 
$v^{(1)}_{k+1,j} - v^{(1)}_{k,i} = v_{k+1,j} - v_{k,i} + 1 \geq 0$, so 
$(k+1,i,j) \in \Omega^+(v^{(1)})$.

\item Suppose $(k,i,j) \in X$. Then $v_{k,i} - v_{k-1,j} \geq 0$, and the 
choice of $(k,i)$ implies that $v_{k-1,j} = v'_{k-1,j}$. Since $v_{k,i} - 
v_{k,j} \geq v'_{k,i} - v'_{k-1,j} \geq 0$, we have $(k,i,j) \in 
\Omega^+(v^{(1)})$.

\item Suppose $(k+1,j,i) \in \Omega^+(v^{(1)})$. Then $v_{k+1,j} - v_{k,i} + 1
\geq 0$. If $v_{k+1,j} - v_{k,i} \geq 0$ then $(k+1, j,i) \in \Omega^+(v) 
\subset Y$ and we are done. Otherwise $v_{k+1,j} - v_{k,i} = - 1$, and again
the choice of $(k,i)$ implies that $v'_{k+1,j} = v_{k+1,j}$. Thus $v'_{k+1,j}
- v'_{k,j} \geq v_{k+1,j} - v_{k,i} + 1 \geq 0$ and $(k+1,i,j) \in \Omega^+(v')
\subset Y$.

\item Suppose $(k,i,j) \in \Omega^+(v^{(1)})$. Then $v_{k,i} - 1 - v_{k-1,j}
\geq 0$ and hence $(k,i,j) \in \Omega^+(v) \subset Y$.
\end{itemize}
This completes the proof for item \ref{i:omega}.

Recall that $\interval{a,b}_k$ is the unique interval in $\I_\eta(v)$ 
containing $(k,i)$, and that we have already observed that the choice of 
$(k,i)$ along with the fact that $v$ dominates $v'$ implies that 
$v_{k,i-1} > v_{k,i} > v_{k,i+1}$ and $v_{k,t} = v'_{k,t}$ for all $i < t 
\leq b$, and this implies that $a(I) = i$. Now if $v_{k,i} - 1 > v_{k,i}$
then $I = \{(k,i)\}$ so $\alpha(I)$ is the identity and $\I_\eta(v) = 
\I_\eta(v')$, which implies that $\omega(v) = \omega(v')$. Thus we only
need to check that item \ref{i:interval} holds when $v_{k,i} - 1 = v_{k,i+1}$.
Let $I' = I \setminus \{(k,i)\}$. Then $\I(v) = \I(v') \setminus 
\{I', \{(k,i)\}\} \cup \{I\}$, and so
\begin{align*}
\omega(v) 
	&= \omega_\eta \prod_{J \in \I_\eta(v)} \omega_J 
	= \omega_\eta \left(\prod_{J \neq I'} \omega_J\right) \omega_{I'}
\end{align*} 
and it is easy to check that $\omega_{I'} = \omega_I \alpha(I)$. This shows 
that $v^{(1)}$ has all the desired properties. Now let $N = \sum_{(r,s) \in 
\Sigma} v_{r,s} - v'_{r,s}$ and set $v^{(s)} = (v^{(s-1)})^{(1)}$ for all $s 
\in \interval N$. An inductive argument shows that this sequence has the 
desired properties, and that $v^{(N)} = v'$. This completes the proof.
\end{proof}

\begin{Proposition}
\label{P:omega}
Let $v,v' \in \overline v + \n(\eta)$. If $\Omega^+(v) \subset \Omega^+(v')$ 
then $V(T(\overline v))[\chi_{v'}] \subset U D_{\omega(v)}(v)$. In particular
if $v$ is fully critical and $\Omega^+(v) = \emptyset$ then $V(T(\overline v))$
is cyclic generated by $D_e(v)$.
\end{Proposition}
\begin{proof}
By item \ref{i:common} of Lemma \ref{L:domination} there exists $v'' \searrow 
v, v'$ with $\Omega^+(v) = \Omega^+(v') \subset \Omega^+(v'')$. Thus it is 
enough to prove the statement for $v \searrow v'$ and $v \nearrow v'$, and by 
Lemma \ref{L:sequence} we may restrict to the case $v' = v \pm \delta^{k,i}$ 
for some $(k,i) \in \Sigma$. 

If $v' = v - \delta^{(k,i)}$ then $\omega(v')$ is an $\epsilon(v)$-shuffle, 
and hence $D_{\omega(v')}(v) \neq 0$. Furthermore, the coefficient of 
$D_{\omega(v')}(v')$ in $E_{k+1,k} D_{\omega(v)}(v)$ is
\begin{align*}
\DD_{\omega(v'), \omega(v')}^v(e_{k,I}^+)
	&= e_{k,I}^+(v) 
		= \frac{\prod_{j=1}^{k-1} v_{k,i} - v_{k-1,j}}
			{\prod_{I \in \I_k(v)} v_{k,i} - v_{k, a(I)}}.
\end{align*}
This is zero if and only if $v_{k,i} = v_{k-1, j}$ for some $j \in \interval 
{k-1}$ which would imply $(k,i,j) \in \Omega^+(v) \subset \Omega^+(v')$,
and hence $-1 = v_{k,i} - v_{k-1,j} -1 = v'_{k,i} - v'_{k-1,j}  \geq 0$, a
contradiction. Thus the projection of $E_{k+1,k}$ to the component 
$V(T(\overline v))[\chi_v]$ is a linear combination of derived tableaux of
character $\chi_{v'}$ in which $D_{\omega(v')}(v')$ appears with nonzero
coefficient. By lemma \ref{L:most-derived-tableaux} this linear combination
generates $V(T(\overline v))[\chi_{v'}]$.

If $v' = v + \delta^{k,i}$ then the coefficient of $D_{\omega(v)}(v)$ in
$E_{k,k+1} D_{\omega(v')}(v')$ is
\begin{align*}
\DD_{\omega(v')\alpha(I), \omega(v)}^{v'}(e_{k,I}^+)
	&= \DD_{\omega(v),\omega(v)}^{v'}(e_{k,I}^-)
		= \frac{\prod_{j=1}^{k+1} v_{k,i} - v_{k+1,j}}
			{\prod_{I \in \I_k(v)} v_{k,i} - v_{k, a(I)}}.
\end{align*}
As before this is not zero, since otherwise $v_{k+1,j} = v_{k,i}$ for some
$j \in \interval{k+1}$, which in turn implies that $-1 = v_{k+1,i} - v_{k,i}
-1 = v'_{k+1,j} - v'_{k,i} \geq 0$, a contradiction. Hence just as before
$V(T(\overline v))[\chi_{v'}]$ is contained in the module generated by 
$D_{\omega(v)}(v)$.
\end{proof}
The following are some immediate consequences of the Proposition.
\begin{Corollary*}
Let $v \in \overline v + \n(\eta)$ be fully critical.
\begin{enumerate}[(a)]
\item If $\Omega^+(v) = \emptyset$ then $D_e(v)$ generates $V(T(\overline v))$ 
over $U$.

\item There is a unique minimal submodule $N_\Omega \subset V(T(\overline v))$.
If $\Omega^+(v) = \Omega(v)$ then $N_\Omega = U D_e(v)$.
\end{enumerate}
\end{Corollary*}
\begin{proof}
The first item is an obvious consequence of the Proposition. For the second, 
let $N \subset V(T(\overline v))$ be any submodule and let $x \in N$ be a 
nonzero element. Since $N$ is a Gelfand-Tsetlin submodule of $V(T(\overline 
v))$ we can assume that $x \in V(T(\overline v))[\chi_{v'}]$ for some $v' \in 
\overline v + \n(\eta)$, and hence $D_e(v') \in N$. Thus by Proposition 
\ref{P:omega} $D_e(v) \in N$ and hence $U D_e(v) \subset N$, so $N_\Omega = U
D_e(v)$ is a minimal element in the lattice of submodules of $V(T(\overline 
v))$.
\end{proof}

\section{A very special Verma module}
In this section we focus on the minimal module associated to the tableau
$\overline v$ whose entries are all equal to a fixed scalar $a \in \CC$.
We fix $V = V(T(\vv))$, and given $v \in \vv + \n(\ZZ)$ write $V[v]$ instead of
$V[\chi_v]$.

\paragraph
Clearly $V[\chi_{\overline v}] = \CC D_e(\overline v)$. 
Furthermore $\Omega^+(\vv) = \Omega(\vv)$ and hence $N = U D_e(\vv)$ is the
unique minimal module of $V(T(\vv))$. On the other hand $D_e(\overline v)$ is 
a highest weight vector of weight $\lambda = (a, a+1, a+2, \cdots)$ and hence 
there is a map $M(\lambda) \to N$ from the Verma module of highest weight 
$\lambda$ to $N$. 

\begin{Lemma*}
The map $p: M(\lambda) \to N$ is an isomorphism.
\end{Lemma*}
\begin{proof}
Let us see $M(\lambda)$ as an $\sl(n,\CC)$ module. Then with respect to the
usual Cartan algebra its weight is $(-1, -1, \ldots, -1)$ which is the half sum
of the roots. Thus $M(\lambda)$ is simple as $\sl(n,\CC)$ module and hence
as $\gl(n,\CC)$ module. Since both $M(\lambda)$ and $N$ are simple and $p$ is
nonzero, it must be an isomorphism.
\end{proof}

\paragraph
Let $\nn_-$ be the Lie algebra formed by the strictly lower triangular matrices
in $\gl(n,\CC)$. The previous lemma implies that the map $U(\nn_-) \to N$ 
given by $u \mapsto u D_e(\vv)$ is an isomorphism. Using the PBW theorem we 
obtain a weight basis of $N$. We will now show that $N$ does not have a basis 
formed by derived tableaux. Given $i > j$ we denote by $\delta[i,j] = 
\delta^{i-1,i-1} + \delta^{i-2, i-2} + \cdots \delta^{j,j}$.

\begin{Lemma}
\begin{enumerate}
\item For each $i > j $ we have $E_{i,j} D_e(\vv) \subset V[v - 
\delta[i,j]]$.

\item There is no other tableau in $N$ of the same weight as $v - 
\delta[i,j]$.
\end{enumerate}
\end{Lemma}
\begin{proof}
\textcolor{blue}{DO!}
\end{proof}
It follows from the lemma that the set \[\{E_{4,3} E_{3,2} E_{2,1} D_e(\vv),
E_{4,3} E_{2,1} E_{3,2} D_e(\vv), E_{3,2} E_{4,3} E_{2,1} D_e(\vv), E_{2,1} 
E_{3,2} E_{4,3} D_e(\vv)\}\] is a basis of $V[\vv-\delta^{1,1}-\delta^{2,2} 
- \delta^{3,3}]$, which thus has dimension $4$.

On the other hand $E_{2,1} D_e(\vv) = D_e(\vv - \delta^{1,1})$, and 
\begin{align*}
E_{3,2} D_e(\vv - \delta^{1,1})
	&= - D_e(\vv - \delta[3,1]) + D_{(12)^2}(\vv - \delta[3,1]).
\end{align*}
Since $N[\vv - \delta[3,1]] \neq 0$ it must contain the element $D_e(\vv- 
\delta[3,1])$. On the other hand it is easy to check that this space has 
dimension $2$, and hence $N[\vv - \delta[3,1]] = \langle D_e(\vv-\delta[3,1]), 
D_{(12)^2}(\vv - \delta[3,1])\rangle$. Next we compute
\begin{align*}
E_{4,3} D_e(\vv - \delta[3,1])
	&= D_{(23)^3}(\vv - \delta[4,1]) - D_e(\vv - \delta[4,1]) \\
E_{4,3} D_{(12)^2}(\vv - \delta[3,1])
	&= - 2 D_{(12)^2}(\vv - \delta[4,1]) - D_{(123)^3}(\vv - \delta[4,1]) \\
		& \qquad + 	D_{(23)^3(12)^2}(\vv - \delta[4,1]).
\end{align*}
Now this expressions involve five different derived tableaux of $T(\vv - 
\delta[4,1])$. Since the corresponding space has dimension $4$ it follows that
it cannot have a basis of derived tableaux. Further computations show that
\begin{align*}
V[\vv - \delta[4,1]]
	&= \langle D_e(\vv - \delta[4,1]), D_{(12)^2}(\vv - \delta[4,1]),
	D_{(23)^3}(\vv - \delta[4,1]),  \\
	& \qquad D_{(123)^3}(\vv - \delta[4,1]) - 
	D_{(12)^2(23)^3}(\vv - \delta[4,1])
	\rangle.
\end{align*}


\begin{bibdiv}
\begin{biblist}
\bibselect{biblio}
\end{biblist}
\end{bibdiv}

\end{document}