%%%%%%%%%%%%%%%%%%%%%% Generalities %%%%%%%%%%%%%%%%%%5
\documentclass[11pt,fleqn]{amsart}
%\usepackage[paper=a4paper]
 %{geometry}

\pagestyle{plain}
\pagenumbering{arabic}
%%%%%%%%%%%%%%%%%%%%%%%%%%%%%%%%
\usepackage[small]{titlesec}
\usepackage{hyperref}
\usepackage{amsthm,thmtools}
%\usepackage{showlabels}
\linespread{1.05}
%\setlength{\parskip}{1.2ex}

\usepackage[utf8]{inputenc}
\usepackage[english]{babel}
\usepackage{enumerate}
\usepackage[osf,noBBpl]{mathpazo}
\usepackage[alphabetic,initials]{amsrefs}
\usepackage{amsfonts,amssymb,amsmath}
\usepackage{mathtools}
\usepackage{graphicx}
\usepackage[poly,arrow,curve,matrix]{xy}
%\usepackage{wrapfig}
%\usepackage{xcolor}
\usepackage{helvet}
\usepackage{stmaryrd}
\usepackage{tikz}
\usepackage{mathdots}
%\usepackage[normalem]{ulem}

\renewcommand\thesection{\arabic{section}}
\titleformat{\section}
 {\normalfont\bfseries\large}
 {\thesection. \space}{0em}{}

\renewcommand\thesubsection{\arabic{subsection}}
\titleformat{\subsection}
 {\normalfont\bfseries}
 {\thesection.\thesubsection \space}{0em}{}

\renewcommand\proofname{Proof}

\makeatletter
\renewenvironment{proof}[1][\textit{\proofname}]{\par
 \pushQED{\qed}%
 \normalfont \topsep.75\paraskip\relax
 \trivlist
 \item[\hskip\labelsep
 \itshape
 #1\@addpunct{.}]\ignorespaces
}{%
 \popQED\endtrivlist\@endpefalse
}
\makeatother
%%%%%%%%%%%%%%%%%%%%%%%%%%% Theorems et al%%%%%%%%%%%%%%%%%%%%%%%%%%
\declaretheoremstyle[
%headformat=swapnumber, 
bodyfont=\itshape,]{mystyle}
\declaretheorem[name=Lemma, style=mystyle, numberwithin=section]{Lemma}
\declaretheorem[name=Proposition, style=mystyle, sibling=Lemma]{Proposition}
\declaretheorem[name=Theorem, style=mystyle, sibling=Lemma]{Theorem}
\declaretheorem[name=Corollary, style=mystyle, sibling=Lemma]{Corollary}
\declaretheorem[name=Definition, style=mystyle, sibling=Lemma]{Definition}
\declaretheorem[name=Example, style=mystyle, sibling=Lemma]{Example}
\declaretheorem[name=Remark, style=mystyle, sibling=Lemma]{Remark}

% unnumbered versions
\declaretheoremstyle[numbered=no, 
bodyfont=\itshape]{mystyle-empty}
\declaretheorem[name=Lemma, style=mystyle-empty]{Lemma*}
\declaretheorem[name=Proposition, style=mystyle-empty]{Proposition*}
\declaretheorem[name=Theorem, style=mystyle-empty]{Theorem*}
\declaretheorem[name=Corollary, style=mystyle-empty]{Corollary*}
\declaretheorem[name=Definition, style=mystyle-empty]{Definition*}
\declaretheorem[name=Example, style=mystyle-empty]{Example*}
\declaretheorem[name=Remark, style=mystyle-empty]{Remark*}

%%%%%%%%%%%%%%%%%%%%%%%%%%% Paragraphs %%%%%%%%%%%%%%%%%%%%%%%%%%%%%
\newskip\paraskip
\paraskip=0.75ex plus .2ex minus .2ex

\newcounter{para}[section]
\setcounter{para}{0}
\renewcommand\thepara{\thesection.\arabic{para}}
\def\paragraph{%
 \noindent
 \refstepcounter{para}%
 \textbf{\thepara.}\hspace{1ex}%
}

\newcommand\about[1]{%
 {\bfseries#1.}%
}

\newcommand\pref[1]{\textbf{\ref{#1}}}

\renewcommand\theHpara{\theHsection.\arabic{para}}
%%%%%%%%%%%%%%%%%%%%%%%%%%% The usual stuff%%%%%%%%%%%%%%%%%%%%%%%%%
\newcommand\NN{\mathbb N}
\newcommand\CC{\mathbb C}
\newcommand\QQ{\mathbb Q}
\newcommand\RR{\mathbb R}
\newcommand\ZZ{\mathbb Z}
\renewcommand\k{\Bbbk}

\newcommand\maps{\longmapsto}
\newcommand\ot{\otimes}
\renewcommand\to{\longrightarrow}
\renewcommand\phi{\varphi}
\newcommand\id{\mathsf{Id}}
\newcommand\im{\mathsf{im}}
\newcommand\coker{\mathsf{coker}}
%%%%%%%%%%%%%%%%%%%%%%%%% Specific notation %%%%%%%%%%%%%%%%%%%%%%%%%
%\newcommand\II{\mathcal I}
%\newcommand\II'{\mathcal J}

\newcommand\g{\mathfrak g}
\newcommand\p{\mathfrak p}
\newcommand\m{\mathfrak m}
\newcommand\gl{\mathfrak{gl}}
\newcommand\vv{\overline{v}}
\newcommand\II{\mathbb I}
\newcommand\RI{\overline{\mathbb I}}

\newcommand\vectspan[1]{\left\langle #1 \right\rangle}
\newcommand\interval[1]{\llbracket #1 \rrbracket}
\newcommand\abs[1]{|#1|}

\DeclareMathOperator\Frac{Frac}
\DeclareMathOperator\Specm{Specm}
\DeclareMathOperator\End{End}
\DeclareMathOperator\Hom{Hom}

\DeclareMathOperator\sym{sym}
\DeclareMathOperator\asym{asym}
\DeclareMathOperator\sg{sg}
\DeclareMathOperator\ev{\mathsf{ev}}
\DeclareMathOperator\supp{supp}

\newcommand\DD{\mathbb D}
\newcommand\D{\mathfrak D}
\newcommand\bigmodule{big GT module~}

\renewcommand\labelitemi{--}

%%%%%%%%%%%%%%%%%%%%%%%%%%%%%%%%%%%%%% TITLES %%%%%%%%%%%%%%%%%%%%%%%%%%%%%%
\title{On the structure of the Big GT-module associated to a character}
%\author{[structure.tex]}
\date{}


\begin{document}
\maketitle
%\vspace{-2cm}
The \bigmodule is the one built in \cite{RZ18}.

\section{Combinatorial preliminaries}

\paragraph
\about{Notation}
\label{notation}
Given $a,b,k \in \NN$ we set $\interval{a,b} = \{i \in \NN \mid a \leq i \leq 
b\}$ and $\interval{b} = \interval{1,b}$. Also we denote by $S_k$ the 
symmetric group in $k$-elements, and given $\pi = (\pi_1, \ldots, \pi_r)$
with $\sum_i \pi_i = k$ we denote by $S_\pi$ the product of symmetric groups
$S_{\pi_1} \times S_{\pi_2} \times \cdots \times S_{\pi_r}$, seen in the 
natural way as subgroup of $S_k$.

Fix $n \in \NN$ and let $\mu = (1, 2, \ldots, n)$. Given $\sigma \in S_\mu$ we 
write $\sigma_{(k)}$ for its projection to $S_k$. Alternatively, given $\sigma 
\in S_k$ we write $\sigma_{(k)}$ for the unique element in $S_\mu$ whose 
projection to $S_k$ is $\sigma$ and whose projection to all other $S_i$ is the
identity. Thus we can write $\sigma = \sigma_{(1)} \sigma_{(2)} \cdots 
\sigma_{(n)}$.

The group $S_\mu$ is a Coxeter group with generating set $\{(i,i+1)_{(k)}
\mid k \in \interval{n}, i \in \interval{k-1}\}$. The usual notions of length,
Bruhat order, parabolic subgroups, etc. will be considered with respect to 
this generating set. In particular the length of $\sigma \in 
S_\mu$ is equal to $\ell_1(\sigma_{(1)}) + \ell_2(\sigma_{(2)}) + \cdots + 
\ell_n(\sigma_{(n)})$, where $\ell_k$ is the usual notion of length in $S_k$.
Also $\sigma < \tau$ in the strong or weak Bruhat order if and only if 
$\sigma_{(k)} < \tau_{(k)}$ for all $k$.

Set $\Sigma = \{(k,i) \mid 1 \leq i \leq k \leq n\}$. Setting $\sigma \cdot 
(k,i) = (k, \sigma_{(k)}(i))$ we obtain an action of $S_\mu$ over $\Sigma$.
The subset $\Sigma' = \{(k,i) \mid 1 \leq i \leq k \leq n-1\}$ is clearly 
invariant by this action. 

Let $\CC^\mu = \CC \times \CC^2 \times \cdots \times \CC^n$, so $v \in \CC^\mu$
is an $n$-tuple $(v_1, v_2, \ldots, v_n)$ with $v_k = (v_{k,1}, v_{k,2},
\ldots, v_{k,k}) \in \CC^k$. For each $(k,i) \in \Sigma$ we denote by 
$\delta^{k,i} \in \CC^\mu$ the unique element in $\CC^\mu$ such that 
$\delta^{k,i}_{l,j} = \delta_{k,l} \delta_{i,j}$, and refer to the set 
$\{\delta^{k,i}\mid (k,i) \in \Sigma\}$ as the canonical basis of $\CC^\mu$. 
The group $S_\mu$ acts on $\CC^\mu$ by linear operators whose action on the 
canonical basis is given by $\sigma \cdot \delta^{k,i} = \delta^{\sigma \cdot 
(k,i)} = \delta^{k \sigma_{(k)}, (i)}$. We denote by $\ZZ^\mu_0$ the abelian 
group generated by $\{\delta^{k,i} \mid 1 \leq i \leq k \leq n-1\}$, which is 
clearly stable by the action of $S_\mu$.

Given $v \in \CC^\mu$ we construct a graph $\Omega(v)$ as follows. The set of 
vertices of $\Omega(v)$ is $\{[k,i] \mid (k,i) \in \Sigma\}$, and we have an 
edge between $[k,i]$ and $[l,j]$ if and only if $v_{k,i} - v_{l,j} \in \ZZ$ 
and $\abs{k-l} \leq 1$. We will write $[k,i] - [l,j]$ as an abbreviation of 
``there is an edge between $[k,i]$ and $[l,j]$ in $\Omega(v)$'' (we use square 
brackets to denote vertices so there is less chance of confusing the vertex 
with a minus). We say two vertices are connected if they lie in the same
connected component of $\Omega(v)$.
\begin{Definition}
We say that $v \in \CC^\mu$ is in \emph{normal form} if $[k,a] - [k,b]$
and $a < b$ implies that $v_{k,i} - v_{k,j} \in \NN_0$ for all $i,j \in 
\interval{a,b}$ with $i < j$. 
\end{Definition}
Clearly for each $v \in \CC^\mu$ there exists at least one element in its 
$S_\mu$-orbit which is in normal form. 

\paragraph
\label{intervals}
\about{Intervals and partitions}
Given $a \leq b \leq k$ we set $\interval{a,b}_k = \{(k,i) \mid i \in 
\interval{a,b}\} \subset \Sigma$. Such a set will be called an \emph{interval}
of $\Sigma$, and given an interval $I = \interval{a,b}_k$ we write $a(I) = a,
b(I) = b, k(I) = k$. A partition of $\Sigma$ is a family of nonempty subsets 
of $\Sigma$, which we cal \emph{blocks}, whose disjoint union is $\Sigma$.
An \emph{interval partition} is a partition $\II$ whose blocks are intervals. 
We write $\II(k)$ for the set of all intervals $I \in \II$ with $k(I) = k$. 
We denote by $S(\II)$ the subgroup of $S_\mu$ stabilizing the blocks of $\II$,
which is a parabolic subgroup of $S_\mu$.

Let $v \in \CC^\mu$. Given an interval $I = \interval{a,b}_k$ we will write 
$v(I)$ for $(v_{k,a}, v_{k,a+1}, \cdots, v_{k,b})$. Given an interval 
partition $\II$ of $\Sigma$ we refer to the tuples $v(I)$ with $I \in \II$
as the $\II$-blocks of $v$.

We associate with $v$ a partition of $\Sigma$, where two elements $(k,l),(i,j)$
lie in the same block if and only if $k = l < n$ and $v_{k,i} - v_{l,j} \in 
\ZZ$; we set the block of an element $[n,i]$ to be $\{[n,i]\}$. We denote this 
partition by $\II(v)$, and set $\II(v,k)$ for the set of blocks involving only 
elements of the form $(k,i)$. 

Suppose $v$ is in normal form and let $\II(v,k) = \{I_1, I_2, \ldots, I_r\}$, 
with $a(I_i) = b(I_{i-1}) + 1$. We set $\pi(v,k) = (\abs{I_1},\abs{I_2}, 
\ldots, \abs{I_r})$, so $S_{\pi(v,k)}$ is a parabolic subgroup of $S_k$; 
observe that by definition $S_{\pi(v,n)}$ is the trivial subgroup of $S_n$. 
We denote by $\pi(v)$ the concatenation of $\pi(v,1), \pi(v,2), \ldots,
\pi(v,n)$, and set $S_{\pi(v)} = S_{\pi(v,1)} \times S_{\pi(v,2)} \cdots 
\times S_{\pi(v,n)} \subset S_\mu$, which is also a parabolic subgroup.
\begin{Definition}
We say that $\vv \in \CC^\mu$ is a \emph{seed} if it is in normal form and 
$(k,i) - (l,j)$ implies $v_{k,i} = v_{l,j}$.
\end{Definition}

\paragraph
\label{descending-z}
\about{Descending $\II$-sequences}
Recall that we denote by $\ZZ^\mu_0$ the set of all $z \in \CC^\mu$ with
$z_{k,i} \in \ZZ$ for all $(k,i) \in \Sigma'$ and $z_{n,i} = 0$ for all $n$.
\begin{Definition}
Let $\II$ be an interval partition of $\Sigma$. We denote by $\DD(\II)$ the 
set of all $z \in \ZZ^\mu_0$ such that $z(I)$ is a decreasing sequence for all 
$I \in \II$. If $\vv$ is a seed we write $\DD(\vv)$ for $\DD(\II(\vv))$.
\end{Definition}
We will only be interested in the case where $\II$ arises as the partition 
associated to a seed, so for the rest of this section we fix a seed $\vv$ and
set $\II = \II(v)$ and $\pi = \pi(v)$.

Let $z \in \DD(\vv)$. The stabilizer of $z$ in $S_\pi$ is again a parabolic 
subgroup of $S_\pi$, which we denote as usual by $(S_\pi)_z$. Thus each 
coclass in $S_\pi/(S_\pi)_z$ has a unique minimal length element, usually 
called a shuffle. We denote by $S_\pi^z$ the set of these minimal length 
representatives, and refer to them as \emph{$z$-shuffles}. Given $\sigma \in
S_\pi$ we denote by $\sigma^z$ the unique $z$-shuffle in the coclass 
$\sigma S_\pi$.

We denote by $\RI(z)$ the interval partition of $\Sigma$ where two elements 
$(k,i), (l,j)$ lie in the same block if they lie in the same block of $\II$ 
and $z_{k,i} = z_{l,j}$. Notice that $(S_\pi)_z$ is exactly $S(\RI(z))$. Let 
us say that $\sigma \in S_\pi$ is increasing over an interval 
$\interval{a,b}_k \subset \Sigma$ if $\sigma_{(k)}(i) < \sigma_{(k)}(j)$ 
whenever $a \leq i < j \leq b$. The permutation $\sigma$ is a $z$-shuffle if 
and only if it is increasing over every interval in $\RI(z)$, so $\sigma^z$
is the unique permutation in $S_\pi$ increasing over all intervals in $\RI(z)$
and such that $\sigma^z(z) = \sigma(z)$.

Given an interval $I = \interval{a,b}_k$ we denote by $\omega(I)$ the 
permutation $i \mapsto  b+a-i$. This is the longest element in the symmetric
group of the interval $I$. We also write $\alpha(I)$ for the permutation $(b b-
1 \cdots a)_{(k)}$ and $\beta(I)$ for its inverse, namely $(a a+1 \cdots 
b)_{(k)}$. It follows that if $\II$ is an interval partition then the longest 
element in $S(\II)$ is $\prod_{I \in \II} \omega(I)$.
\begin{Lemma}
\label{L:omega-delta}
Let $z\in \DD(\vv)$ and let $\omega_0$ be the longest element in $S_\pi$.
\begin{enumerate}[(a)]
\item 
\label{i:omega-z}
We have $\omega_0^z = \omega_0 \prod_{I \in \RI(z)} \omega(I)$.

\item 
\label{i:D-delta}
We have $z + \delta^{k,i} \in \DD(\vv)$ if and only if $i = a(I)$ for some $I 
	\in \RI(z,k)$. Analogously, $z - \delta^{k,i} \in \DD(\vv)$ if and only 
	if $i = b(I)$ for some $I \in \RI(z)$.

\item 
\label{i:omega-delta}
Let $I \in \RI(z,k)$. If $z_{k,a(I)-1} \neq z_{k,a(I)} + 1$ then 
	$\omega_0^{z + \delta^{k,a(I)}} = \omega_0^z \alpha(I)$. Analogously,
	if $z_{k,b(I)+1} \neq z_{k,b(I)} - 1$ then $\omega_0^{z - \delta^{k,b(I)}} 
	= \omega_0^z \beta(I)$.
\end{enumerate}
\end{Lemma}
\begin{proof}
By definition $\omega_0$ reverses the order of each interval $J$ in $\II$. In
particular it is decreasing over each interval $I$ contained in any $J$. Since 
$\omega_0(I)$ is decreasing over the corresponding interval $I$, it follows 
that $\omega_0 \prod_{I \in \RI(z)} \omega(I)$ is increasing over every 
interval $I \in \RI(z)$, so it is a $z$-shuffle. Furthermore it clearly lies 
in the coclass $\omega_0 (S_\pi)_z$. This proves item \ref{i:omega-z}.
Item \ref{i:D-delta} follows immediately from the definitions. 

Suppose we are in the first case of item \ref{i:omega-delta} and set $y = z + 
\delta^{k,i}$. Notice first that $\omega(\interval{a(I)+1,b(I)}_k) = 
\omega(I) \alpha(I)$. The hypothesis implies that $\RI(z,l) = \RI(y,l)$ for 
$l \neq k$, while $\RI(y,k) = \RI(z,k) \setminus I \cup \{\{(a(I),k)\},
\interval{a(I)+1,b(I)}_k\}$, so by item \ref{i:omega-z}
\begin{align*}
\omega_0^y 
	&= \prod_{J \in \RI(y)} \omega(J)
	= \left(\prod_{J \in \RI(z), J \neq I} \omega(J)\right) 
		\omega(\interval{a(I)+1,b(I)}_k) \\
	&= \left(\prod_{J \in \RI(z)} \omega(J)\right)\alpha(I)
	= \omega_0^z \alpha(I).
\end{align*}
The second case of this item is proved similarly.
\end{proof}

\section{Big Gelfand-Tsetlin modules and their submodules}
Throughout this section we fix a seed $\vv$ and set $\II = \II(\vv), \pi = 
\pi(\vv)$.

\paragraph
\about{Big Gelfand-Tsetlin modules}
\label{big-gt-modules}
Given $I = \interval{a,b}_k$ we set
\begin{align*}
e_I
	&= \frac{\displaystyle \prod_{j = 1}^{k+1} x_{k,a} - x_{k+1,j}}
		{\displaystyle \prod_{j \notin I} x_{k,a} - x_{k,j}}
&f_I
	&= \frac{\displaystyle \prod_{j = 1}^{k-1} x_{k,b} - x_{k-1,j}}
		{\displaystyle \prod_{j \notin I} x_{k,b} - x_{k,j}}
\end{align*}
Notice that if $I \in \RI(z)$ then $e_I(\vv + z)$ and $f_I(\vv + z)$ are
well defined. We also set 
\begin{align*}
h_k = x_{k,1} + \cdots + x_{k,k} - (x_{k-1, 1} + \cdots + x_{k-1,k-1}) + k-1.
\end{align*}

The following theorem describes the action of $\gl(n,\CC)$ on the big 
Gelfand-Tsetlin modules introduced in \cite{RZ18}. This description is taken
from \cite{FGRZ18}*{Lemma 8.4}.
\begin{Theorem}
\label{T:gt-big-module}
Let $\vv$ be a seed. There exists a Gelfand-Tsetlin module, denoted by 
$V(T(\vv))$, whose support equals $\vv + \DD(\vv)$. Its component of weight 
$\vv + z$ has basis $\{D_\sigma(\vv+z) \mid \sigma \in S_\pi^z\}$, and
the action of the canonical generators of $\gl(n,\CC)$ on $V(T(\vv))$ is given 
by
\begin{align*}
E_{k,k+1} D_\sigma(\vv + z)
	&= - \sum_{I \in \RI(z,k)} 
		\sum_{\tau < \sigma \alpha(I)} 
			\D_{\tau,\sigma\alpha(I)}^{\vv + z}(e_{I})
			D_{\tau}(\vv + z + \delta^{k,a(I)}); \\
E_{k+1,k} D_\sigma(\vv + z)
	&= \sum_{I \in \RI(z,k)} 
		\sum_{\tau < \sigma \beta(I)} 
			\D_{\tau,\sigma\beta(I)}^{\vv + z}(f_{I})
			D_{\tau}(\vv + z - \delta^{k,b(I)}); \\
E_{k,k} D_{\sigma}(\vv + z) 
	&= h_k(\vv + z) D_{\sigma}(\vv+z). 
\end{align*}
\end{Theorem}

Notice that through the map $D_\sigma (\vv + z) \mapsto \vv + \sigma(z)$ the
support of $V(T(\vv))$ can be identified with $\vv + \ZZ^\mu_0$. This 
identification has the advantage of taking multiplicities into account.

The following proposition describes the weight components of $V(T(\vv))$. 
Proofs for these results can be found in \cite{FGRZ18}*{Proposition 6.4 and
Lemma 6.5}.
\begin{Proposition}
\label{P:gt-weight-spaces}
Let $z \in \DD(\vv)$ and set $T = \sum_\sigma a_\sigma D_\sigma(\vv + z)$.
\begin{enumerate}[(a)]
\item 
\label{i:action}
If $c \in \Gamma$ then
\begin{align*}
c D_\sigma(\vv + z)
	&= \gamma(c)(\vv+z) +
		\sum_{\tau < \sigma} \D_{\tau,\sigma}^{\vv + z}(\gamma(c)) 
			D_\tau(\vv+z).
\end{align*}

\item 
\label{i:generator}
The element $T$ generates the component of Gelfand-Tsetlin weight $\vv+z$
if and only if $a_{\omega_0^z} \neq 0$.

\item 
\label{i:eigenvalue}
If $(c - \gamma(c))^r T = 0$ for all $c \in \Gamma$ then $a_\sigma = 0$
whenever $\ell(\sigma) \geq r$. In particular the space of eigenvectors with
eigenvalue $\chi_{\vv + z}$ is generated by $D_e(\vv + z)$.
\end{enumerate}
\end{Proposition}

\paragraph
\about{Gelfand-Tsetlin chambers}
\label{gt-chambers}
Recall that $\Omega(\vv)$ is a graph with vertex set $\Sigma$ such that
$[k,i] - [l,j]$ if and only if $\vv_{k,i} - \vv_{l,j} \in \ZZ$ and $|k-l| 
\leq 1$. Given $z \in \DD(\vv)$ we define an directed graph $\Omega(\vv,z)$. 
For this we introduce the obvious notation $[i,j] \rightarrow [l,j]$ as an
abbreviation of ``there is an oriented vertex with tail $[i,j]$ and head
$[l,j]$''. The digraph $\Omega(\vv,z)$ has $\Omega(\vv)$ as its underlying
graph, and its orientation is given by the following rules.
\begin{itemize}
\item If $[k,i] - [k,j]$ and $i < j$ then $[k,i] \rightarrow [k,j]$; we refer
to these \emph{horizontal edges}.

\item If $[k,i] - [k-1,j]$ and $z_{k,i} \geq z_{k-1,j}$ then  $[k,i] 
\rightarrow [k-1,j]$; we refer to these as \emph{descending edges}.

\item If $[k,i] - [k-1,j]$ and $z_{k,i} < z_{k-1,j}$ then  $[k-1,j] 
\rightarrow [k,i]$; we refer to these as \emph{ascending edges}.
\end{itemize}
Thus an arrow points from the entry corresponding to the larger number to the 
one corresponding to the smaller one, and in case the two numbers are equal it 
goes from left to right (if the two entries lie in the same row) or from top 
to bottom (if they are in consecutive rows). It follows that $\Omega(\vv,z)$ 
has no loops, so each of its connected components has at least one source (a 
vertex that is not the head of any edge) and at least one sink (a vertex that
is not the source of any edge).
\begin{Definition}
\label{D:gt-chamber}
The \emph{Gelfand-Tsetlin chamber} of $z \in \DD(\vv)$ is the set of all 
$y \in \DD(\vv)$ such that $\Omega(\vv,z) = \Omega(\vv,y)$. 
\end{Definition}
Fix $z \in \DD(\vv)$ and let $\Omega = \Omega(\vv,z)$. The graph $\Omega$ will
be necessary for computations, but it is quite difficult to write down since it
has many edges. Let us say that a directed edge $[k,i] \rightarrow [l,j]$ in 
$\Omega$ is \emph{superfluous} if there exists a sequence of directed edges 
$[k,i] = [k_0, i_0] \rightarrow [k_1, i_1]\rightarrow \cdots \rightarrow 
[k_r, i_r] = [l,j]$ with $r > 1$. If an edge is not superfluous then we will 
say it is an \emph{essential} edge. Notice that by definition any edge of the 
form $[n,i] \rightarrow [n,i+1]$ is essential. The \emph{reduced graph of $z$} 
$\Omega_0 = \Omega_0(\vv,z)$ is obtained by deleting all superfluous edges. 

Since $\Omega$ is a directed graph without loops, the same holds for the 
reduced graph. A few experiments will convince the reader that the graph 
$\Omega_0$ is always planar. We can recover $\Omega$ from $\Omega_0$ by adding 
a directed edge $[k,i] \rightarrow [l,j]$ whenever there is a path from 
$[k,i]$ to $[l,j]$ in $\Omega_0$. Thus $z,y \in \DD(\vv)$ lie in the same 
GT-chamber if and only if their reduced graphs are equal. 

Recall that $z\in \DD(\vv)$ is called non-critical if $z_{k,i} \neq z_{k,j}$
whenever $[k,i] - [k,j]$ and $k<n$.
\begin{Lemma}
\label{L:canonical-element}
Every Gelfand-Tsetlin chamber contains a non-critical element.
\end{Lemma}
\begin{proof}
We put a weight on each edge of $\Omega$: edges $[k,i] \rightarrow [k-1,j]$ 
and $[n,i] \rightarrow [n,j]$ are assigned weight zero, while edges 
$[k-1,i] \rightarrow [k,j]$ and $[k,i] \rightarrow [k,j]$ with $k<n$ are 
assigned weight $1$. The weight of a path in $\Omega$ is the sum of the 
weights of the edges that form the path. The \emph{depth} of a vertex $[k,i]$, 
denoted $d([k,i])$, is the maximum of the weights of all paths from $[k,i]$ to 
a sink of $\Omega$. Notice that if there is a path from $[n,i]$ to $[n,j]$ 
then it must consist of arrows of weight $0$, and hence both vertices have the 
same depth. 

Set
\begin{align*}
y_{k,i}
	&= 
	\begin{cases}
	d([k,i]) - d([n,j]) 
		&\mbox{if $[k,i]$ and $[n,j]$ are connected;} \\
	d([k,i])
		&\mbox{otherwise.}
	\end{cases}
\end{align*}
By definition $y_{n,i} = 0$ for all $i \in \interval n$. Let $1 \leq i < j 
\leq k < n$ be such that $[k,i] \rightarrow [k,j]$ in $\Omega$. By definition 
this implies $d([k,i]) > d([k,j])$, which in turn implies $y_{k,i} > y_{k,j}$, 
so $y \in \DD(\vv)$ and it is clearly non-critical. A similar argument shows 
that if $[k,i] \rightarrow [k-1,j]$ in $\Omega$ then $y_{k,i} \geq y_{k-1,j}$, 
while if $[k-1,j] \rightarrow [k,i]$ then $y_{k,i} < y_{k-1,j}$. Thus $y$ lies 
in the same GT-chamber as $z$.
\end{proof}
We refer to the non-critical element found in the proof of the lemma as the
\emph{canonical element} of the GT-chamber.
Notice that if $[k,i] \to [l,j]$ in $\Omega(\vv,z)$ is superfluous then there 
is a path connecting the same vertices which is formed by essential arrows, 
and that the weigth of that path is larger than or equal to that of the edge.
Thus the canonical element of the chamber can be found by repeating the 
procedure of the proof with the reduced graph of $z$.

\paragraph
\about{The internal structure of the big module}
We now begin with a series of lemmas about the internal structure of 
$V(T(\vv))$.
\begin{Lemma}
Suppose $y,z \in \DD(\vv)$ lie in the same Gelfand-Tsetlin chamber. Then the
derived tableaux $D_{\omega_0^y}(\vv + y)$ and $D_{\omega_0^{z}}(\vv + z)$
generate the same module.
\end{Lemma}
\begin{proof}
We will prove the result taking $y$ to be the canonical element in the GT 
chamber. Since $z$ is arbitrary, the result follows from this particular case.

Set $\Omega = \Omega(\vv,z)$.
We proceed by induction on $|y-z| = \sum_{(k,i) \in \Sigma} |y_{k,i} - 
z_{k,i}|$. The base case is $y = z$ which is trivial. If $y \neq z$ then there 
exists $(l,j) \in \Sigma$ such that $y_{l,j} \neq z_{l,j}$. Suppose first
that $z_{l,j} < y_{l,j}$. This implies that the digraph $\Omega_<$ obtained by 
removing from $\Omega$ those vertices $[m,r]$ such that $z_{l,j} \geq 
y_{l,j}$ is nontrivial. 

Let $(k,i)$ be a source of $\Omega_<$, let $I \in \II(\vv,z)$ be such that 
$(k,i) \in I$, and set $u = z = \delta^{k,i}$. Suppose $[k,i-1] - [k,i]$ in 
$\Omega(\vv)$. Since the edge $[k,i-1] \rightarrow [k,i]$ is not present in 
$\Omega_>$, it follows that $y_{k,i-1} \leq z_{k,i-1}$, and since $y$ is 
non-critical we see obtain
\begin{align*}
z_{k,i} < y_{k,i} < y_{k,i-1} \leq z_{k,i-1},
\end{align*}
which implies that $u \in \DD(\vv)$ and that $\{(k,i)\} \in \II(\vv, u)$. On 
the other hand, if $[k,i-1]$ and $[k,i]$ are not connected by an edge in 
$\Omega(\vv)$ the same statement holds trivially. We also point out for
later use that item \ref{i:omega-delta} of Lemma \ref{L:omega-delta} 
implies that $\omega_0^u = \omega_0^z \alpha(I)$.

We claim that $u$ lies in the same chamber as $y$ and $z$. Since we have only
increased the $(k,i)$ entry of $z$ by one, it is enough to check that if 
$[k \pm 1,j] \rightarrow [k,i]$ in $\Omega(\vv,y)$, then the same edge appears
in $\Omega(\vv,u)$. Now in either case we have $y_{k \pm 1, j} \geq 
z_{k \pm 1, j}$, so
\begin{align*}
u_{k+1,j} 
	&= z_{k+1,j} \geq y_{k+1,j} \geq y_{k,i} \geq z_{k,i} + 1 = u_{k,i}, \\ 
u_{k-1,j} &= z_{k-1,j} \geq y_{k-1,j} > y_{k,i} \geq z_{k,i} + 1 = u_{k,i},
\end{align*}
so $u$ lies in the same GT chamber as $y$ and $z$. 

Notice that the first inequality in the previous display implies that if 
$[k,i] - [k+1,j]$ in $\Omega(\vv)$ then $z_{k,i} \neq z_{k,+1,j}$, and hence 
$e_I(\vv+z) \neq 0$. Thus if we take the component of $E_{k,k+1} 
D_{\omega_0^z}(\vv + z)$ of Gelfand-Tsetlin weight $\chi_{v+u}$ we obtain
\begin{align*}
e_{I}(\vv + z) D_{\omega_0^z \alpha(I)}(\vv + u) + T
\end{align*}
where $T$ is a linear combination of derived tableaux of order less than 
$\ell(\omega_0^z \alpha(I))$. Now as observed above $\omega_0^z \alpha(I) = 
\omega_0^u$, so by item \ref{i:generator} of Proposition 
\ref{P:gt-weight-spaces} $V(T(\vv))[\vv + u]$ is contained in the submodule
generated by $D_{\omega_0^z}(\vv + z)$. 

Now let $J \in \II(\vv, u)$. Then either $J \in \II(\vv, z)$, or $J = 
\interval{i+1,b(I)}_k$, or $J = \{(k,i)\}$. In each case it is easy to see 
that $\omega_0^z$ is increasing over $J$, and so it is a $u$-shuffle. In 
particular $D_{\omega_0^z}(\vv + u)$ is a nonzero element of $V(T(\vv))[\vv + 
u]$ and the projection of $E_{k+1,k} D_{\omega_0^z}(\vv + u)$ to the component
$V(T(\vv))[\vv + z]$ is
\begin{align*}
f_{\{(k,i)\}}(\vv + u) D_{\omega_0^z \alpha(I)}(\vv + u) + T'
\end{align*}
where $T'$ is a linear combination of derived tableaux of order less than 
$\ell(\omega_0^z)$. Now if $[k,i] \rightarrow [k-1,j]$ in $\Omega(\vv,u) = 
\Omega(\vv,z)$, then $u_{k,i} - 1 = z_{k,i} \geq z_{k-1,j} = u_{k-1,j}$ and 
the coefficient $f_{\{(k,i)\}}(\vv + u)$ is not zero. Again by item 
\ref{i:generator} of Proposition \ref{P:gt-weight-spaces} we see that 
$V(T(\vv))[\vv + z]$ is contained in the submodule generated by 
$D_{\omega_0^u}(\vv + u)$. In other words, $D_{\omega_0^z}(\vv + z)$ and 
$D_{\omega_0^u}(\vv + u)$ generate the same submodule of $V(T(\vv))$. Since
$|u-y| = |z-y| - 1$, by the inductive hypothesis either of these tableaux 
generate the same module as $D_{\omega_0}(\vv + z)$.
\end{proof}

\begin{bibdiv}
\begin{biblist}
\bibselect{references}
\end{biblist}
\end{bibdiv}

\end{document}

