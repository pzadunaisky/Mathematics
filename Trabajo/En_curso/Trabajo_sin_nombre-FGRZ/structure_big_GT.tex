%%%%%%%%%%%%%%%%%%%%%% Generalities %%%%%%%%%%%%%%%%%%5
\documentclass[11pt,fleqn]{amsart}
\usepackage[paper=a4paper]
 {geometry}

\pagestyle{plain}
\pagenumbering{arabic}
%%%%%%%%%%%%%%%%%%%%%%%%%%%%%%%%
\usepackage[small]{titlesec}
\usepackage{hyperref}
\usepackage{amsthm,thmtools}
%\usepackage{showlabels}
\linespread{1.05}
%\setlength{\parskip}{1.2ex}

\usepackage[utf8]{inputenc}
\usepackage[english]{babel}
\usepackage{enumerate}
\usepackage[osf,noBBpl]{mathpazo}
\usepackage[alphabetic,initials]{amsrefs}
\usepackage{amsfonts,amssymb,amsmath}
\usepackage{mathtools}
\usepackage{graphicx}
\usepackage[poly,arrow,curve,matrix]{xy}
%\usepackage{wrapfig}
%\usepackage{xcolor}
\usepackage{helvet}
\usepackage{stmaryrd}
%\usepackage{tikz}
%\usepackage{mathdots}
%\usepackage[normalem]{ulem}

\renewcommand\thesection{\arabic{section}}
\titleformat{\section}
 {\normalfont\bfseries\large}
 {\thesection. \space}{0em}{}

\renewcommand\thesubsection{\arabic{subsection}}
\titleformat{\subsection}
 {\normalfont\bfseries}
 {\thesection.\thesubsection \space}{0em}{}

\renewcommand\proofname{Proof}

\makeatletter
\renewenvironment{proof}[1][\textit{\proofname}]{\par
 \pushQED{\qed}%
 \normalfont \topsep.75\paraskip\relax
 \trivlist
 \item[\hskip\labelsep
 \itshape
 #1\@addpunct{.}]\ignorespaces
}{%
 \popQED\endtrivlist\@endpefalse
}
\makeatother
%%%%%%%%%%%%%%%%%%%%%%%%%%% Theorems et al%%%%%%%%%%%%%%%%%%%%%%%%%%
\declaretheoremstyle[
%headformat=swapnumber, 
bodyfont=\itshape,]{mystyle}
\declaretheorem[name=Lemma, style=mystyle, numberwithin=section]{Lemma}
\declaretheorem[name=Proposition, style=mystyle, sibling=Lemma]{Proposition}
\declaretheorem[name=Theorem, style=mystyle, sibling=Lemma]{Theorem}
\declaretheorem[name=Corollary, style=mystyle, sibling=Lemma]{Corollary}
\declaretheorem[name=Definition, style=mystyle, sibling=Lemma]{Definition}
\declaretheorem[name=Example, style=mystyle, sibling=Lemma]{Example}
\declaretheorem[name=Remark, style=mystyle, sibling=Lemma]{Remark}

% unnumbered versions
\declaretheoremstyle[numbered=no, 
bodyfont=\itshape]{mystyle-empty}
\declaretheorem[name=Lemma, style=mystyle-empty]{Lemma*}
\declaretheorem[name=Proposition, style=mystyle-empty]{Proposition*}
\declaretheorem[name=Theorem, style=mystyle-empty]{Theorem*}
\declaretheorem[name=Corollary, style=mystyle-empty]{Corollary*}
\declaretheorem[name=Definition, style=mystyle-empty]{Definition*}
\declaretheorem[name=Example, style=mystyle-empty]{Example*}
\declaretheorem[name=Remark, style=mystyle-empty]{Remark*}

%%%%%%%%%%%%%%%%%%%%%%%%%%% Paragraphs %%%%%%%%%%%%%%%%%%%%%%%%%%%%%
\newskip\paraskip
\paraskip=0.75ex plus .2ex minus .2ex

\newcounter{para}[section]
\setcounter{para}{0}
\renewcommand\thepara{\thesection.\arabic{para}}
\def\paragraph{%
 \noindent
 \refstepcounter{para}%
 \textbf{\thepara.}\hspace{1ex}%
}

\newcommand\about[1]{%
 {\bfseries#1.}%
}

\newcommand\pref[1]{\textbf{\ref{#1}}}

\renewcommand\theHpara{\theHsection.\arabic{para}}
%%%%%%%%%%%%%%%%%%%%%%%%%%% The usual stuff%%%%%%%%%%%%%%%%%%%%%%%%%
\newcommand\NN{\mathbb N}
\newcommand\CC{\mathbb C}
\newcommand\QQ{\mathbb Q}
\newcommand\RR{\mathbb R}
\newcommand\ZZ{\mathbb Z}
\newcommand\PiI{\mathbb I}
\renewcommand\k{\Bbbk}

\newcommand\maps{\longmapsto}
\newcommand\ot{\otimes}
\renewcommand\to{\longrightarrow}
\renewcommand\phi{\varphi}
\newcommand\Pid{\mathsf{Id}}
\newcommand\im{\mathsf{im}}
\newcommand\coker{\mathsf{coker}}
%%%%%%%%%%%%%%%%%%%%%%%%% Specific notation %%%%%%%%%%%%%%%%%%%%%%%%%
%\newcommand\Pi{\mathcal I}
%\newcommand\Pi'{\mathcal J}
\newcommand\D[3]{{}^{#1} \mathfrak D_{#2}^{#3}}
\newcommand\DD[3]{{}^{#1} \mathcal D_{#2}^{#3}}
\newcommand\Z{\mathsf Z}

\newcommand\g{\mathfrak g}
\newcommand\p{\mathfrak p}
\newcommand\m{\mathfrak m}
\newcommand\gl{\mathfrak{gl}}
\newcommand\gen{\mathsf{gen}}
\newcommand\std{\mathsf{std}}
\newcommand\sh{\mathsf{sh}}
\renewcommand\SS{\mathfrak S}
\newcommand\vv{\overline{v}}

\newcommand\vectspan[1]{\left\langle #1 \right\rangle}
\newcommand\interval[1]{\llbracket #1 \rrbracket}

\DeclareMathOperator\Frac{Frac}
\DeclareMathOperator\Specm{Specm}
\DeclareMathOperator\End{End}
\DeclareMathOperator\Hom{Hom}

\DeclareMathOperator\sym{sym}
\DeclareMathOperator\asym{asym}
\DeclareMathOperator\sg{sg}
\DeclareMathOperator\ev{\mathsf{ev}}
\DeclareMathOperator\ann{ann}
\DeclareMathOperator\supp{supp}


\newcommand\bigmodule{big GT module~}

%%%%%%%%%%%%%%%%%%%%%%%%%%%%%%%%%%%%%% TITLES %%%%%%%%%%%%%%%%%%%%%%%%%%%%%%
\title{On the structure of the Big GT-module associated to a character}
%\author{[structure.tex]}
\date{}


\begin{document}
\maketitle
%\vspace{-2cm}
The \bigmodule is the one built in \cite{RZ18}.

\section{Notation}
Fix $n \in \NN$ and let $\mu = (n, n-1, \ldots, 1)$. 
Given $a,b,k \in \NN$ we set 
$\interval{a,b} = \{i \in \NN \mid a \leq i \leq b\}$ and $\interval{b} = 
\interval{1,b}$; also set $\interval{a,b}_k = \{(k,i) \mid i \in 
\interval{a,b}\}$. If $I = \interval{a,b}$ we define $a(I) = a$ and $b(I)=b$, 
so $|I| = b(I) - a(I) + 1$. Finaly we set $\Sigma = \{\interval{k}_k \mid
1 \leq k \leq n\} = \{(k,i) \mid 1 \leq i \leq k \leq n\}$.

\paragraph
\about{Partitions and compositions}
\label{partitions-compositions}
An interval of $\Sigma$ is a set $\interval{a,b}_k \subset \Sigma$, so
$1 \leq a \leq b \leq k$. An composition of $\Sigma$ is a disjoint 
family $\Pi$ of intervals of $\Sigma$ whose union is $\Sigma$. Given a 
composition $\Pi$ of $\Sigma$ and $k \in \interval n$, we denote by $\Pi_k$ the
set of all intervals $\interval{a,b}_k \in \Pi$. Given two 
interval partitions $\Pi, \Pi'$ of $\Sigma$, we say that $\Pi'$ refines $\Pi$ 
and write $\Pi' < \Pi$ if for each $J \in \Pi'$ there exists $I \in \Pi$ such 
that $J \subset I$. 

A composition of $\mu$ is an $n$-tuple $\pi = (\pi^{(1)}, \ldots, 
\pi^{(n)})$ where $\pi^{(k)}$ is a composition of $k$; since $\mu$ is fixed
we can identify $\pi$ with the concatenation of its parts. To each
composition $\Pi$ of $\Sigma$ we assign a composition $\pi(\Pi)$ of $\mu$, 
setting $\pi^{(k)}_j$ to be the cardinality of the $j$-th interval in $\Pi_k$. 
This assignation is bijective, and for each composition $\pi$ of $\mu$ we 
denote by $\Pi(\pi)$ the unique composition of $\Sigma$ corresponding to 
$\pi$. We say that a composition $\epsilon$ refines $\pi$, and write 
$\epsilon < \pi$, whenever $\Pi(\epsilon) < \Pi(\pi)$. Compositions of $\mu$ 
form a finite poset with maximum $\mu$, so whenever we write $\pi < \mu$ we 
will mean that $\pi$ is a composition of $\mu$.

For each finite set $X$ we denote by $S(X)$ the set of all bijections from
$X$ to $X$. Set $S_\mu = S(\interval{1}_1) \times S(\interval{2}_2) \times
\cdots \times S(\interval{n}_n)$, which is a subset of $S(\Sigma)$. 
Every element $\sigma \in S_\mu$ can be written as $\sigma_{(1)} \sigma_{(2)}
\cdots \sigma_{(n)}$ with $\sigma_{(k)} \in S(\interval{k}_k)$. Obviously
$S(\interval{k}_k)$ is isomorphic to the symmetric group in $k$ elements,
and given a permutation $\sigma \in S_k$ we will denote the corresponding
element of $S(\interval{k}_k)$ by $\sigma_{(k)}$. If $I 
\subset \Sigma$ is an interval then we see $S(I) \subset S_\mu$ in the obvious 
way. For each composition $\Pi$ of $\Sigma$ we define $S_\Pi = \prod_{I \in \Pi} 
S(I) \subset S_\mu$, and for each $\pi < \mu$ we set $S_\pi = S_{\Pi(\pi)}$.
Since it is a product of symmetric groups, $S_\mu$ is a Coxeter group
with generating set $\{(i,i+1)_{(k)} \mid 1 \leq i \leq k-1, 1 \leq k \leq 
n\}$, and each $S_\pi$ is a parabolic subgroup of $S_\mu$. We define the 
length of an element always with respect to this generating set.

\paragraph
\about{$\mu$-points}
\label{mu-points}
Set $\CC^\mu = \CC^{n} \times \CC^{n-1} \times \cdots \times \CC^{1}$. A point 
in $\CC^\mu$ is thus an $n$-tuple $(v_n, \ldots, v_1)$ with each $v_k$ a 
vector with $k$ entries. We will refer to the $v_k$'s as the \emph{rows} of 
$v$ and denote by $v_{k,i}$ the $i$-th entry of $v_k$. Also given an interval
$\interval{a,b}_k \subset \Sigma$ we write $v(\interval{a,b}_k) = (v_{k,a},
\ldots, v_{k,b})$. Given $\pi < \mu$ and $\Pi = \Pi(\pi)$ the $\pi$-blocks
of $v$ are the elements of the set $\{v(I) \mid I \in \Pi\}$.

The group $S_\mu$ acts on $\CC^\mu$ by permuting the entries, i.e. 
$\sigma(v)_{k,i} = v_{\sigma(k,i)} = v_{k,\sigma_{(k)}(i)}$. We set 
$\Lambda = \CC[x_{k,i} \mid (k,i) \in \Sigma]$, the ring of polynomial 
functions on $\CC^\mu$, and by $K = \Frac(\Lambda)$ its fraction field. The 
action of $S_\mu$ on $\CC^\mu$ induces an action on $\Lambda$ which is given 
by $\sigma(x_{k,i}) = x_{\sigma(k,i)} = x_{k,\sigma_{(k)}(i)}$, and extends to 
the fraction field. If $\pi < \mu$ then $S_\pi$ acts on $\CC^\mu$ by 
restriction, which induces $S_\pi$ actions on $\Lambda$ and $K$. Notice that
the action of $S_\pi$ only permutes elements in the same $\pi$-block.

We denote by $\ZZ^\mu_0 \subset \CC^\mu$ the set of points with integral 
entries and $z_{n,i} = 0$ for all $i \in \interval{n}$. Fix $\pi < \mu$ and 
let $\Pi = \Pi(\pi)$ be the corresponding composition of $\Sigma$. We say that 
$z \in \ZZ^\mu_0$ is \emph{$\pi$-standard}, or just standard if $\pi$ is 
clear from the context, if for each $\interval{a,b}_k \in \Pi(\pi)$ we have 
$z_{k,a} \geq z_{k,a+1} \geq \cdots \geq z_{k,b}$; in other words, each 
$\pi$-block of $z$ is a weakly decreasing sequence. For each $z \in \ZZ^\mu_0$
there is exactly one element in the orbit $S_\pi z$ which is $\pi$-standard, 
and we denote it by $\std_\pi(z)$. Thus the set $\std(\pi) = \{\std_\pi(z) 
\mid \ZZ^\mu_0\}$ is a complete set of representatives of the equivalence 
classes in $\ZZ^\mu_0 /S_\pi$. 

If $z \in \std(\pi)$ then there existes a refinement $\epsilon(z) < \pi$ 
such that $S_{\epsilon(z)}$ is the stabilizer of $z$ in $S_\pi$. The 
corresponding partition is denoted by $E(z)$. 


\section{The modules}
Let $v \in \CC^\mu$. We will say that $(k,i), (l,j) \in \Sigma$ are $v$-related
if $v_{k,i} - v_{l,j} \in \ZZ$ and $|k-l| \leq 1$. If we restrict to a single 
row $k \in \interval n$ then $v$-relation is an equivalence relation; we 
set $\Pi(v)_k$ to be the partition of $\interval{k}_k$ given by this 
equivalence relation, and $\Pi(v) = \bigsqcup_k \Pi(v)_k$. Notice that $\Pi(v)$
is a partition of $\Sigma$, but not necessarily a composition since its 
elements need not be intervals. Indeed, $\Pi(v)$ is a composition if and only 
if whenever $v_{k,i} - v_{k,j} \in \ZZ$ then $v_{k,i'} - v_{k,j'} \in \ZZ$
for all $i',j' \in \interval{i,j}$.

\begin{Definition}
A \emph{seed} is a point $\vv \in \CC^\mu$ such that $\Pi(\vv)$ is an interval 
partition, and whenever $(k,i), (l,j)$ are $\vv$-related then $\vv_{k,i} = 
\vv_{l,j}$.
\end{Definition}
Notice that given $v \in \CC^\mu$ there is always a seed $\vv \in (S_\mu \# 
\ZZ^\mu_0) \cdot v$ (not unique), so $V(T(v)) \cong V(T(\vv))$. Hence we can
always assume that our big module is the big module associated to a seed. 
For the rest of this section we will fix a seed $\vv$, and write $\Pi$ for 
$\Pi(v)$ and $\pi$ for the partition of $\mu$ associated to $\Pi$.



\begin{bibdiv}
\begin{biblist}
\bibselect{references}
\end{biblist}
\end{bibdiv}

\end{document}