\documentclass[11pt,fleqn]{article}

\usepackage[paper=a4paper]
  {geometry}

\pagestyle{plain}

\usepackage{paragraphs}
\usepackage{hyperref}
\usepackage{amsthm,thmtools}

\usepackage[utf8]{inputenc}
\usepackage[spanish,english]{babel}
\usepackage{enumitem}
\usepackage[osf,noBBpl]{mathpazo}
\usepackage[alphabetic,initials]{amsrefs}
\usepackage{amsfonts,amssymb,amsmath}
\usepackage{mathtools}
\usepackage{graphicx}
\usepackage[poly,arrow,curve,matrix]{xy}
\usepackage{wrapfig}
\usepackage{xcolor}
\usepackage{helvet}
\usepackage{stmaryrd}
%\usepackage{showlabels}

%%%%%%%%%%%%Theorems, for paragraphs package%%%%%%%%%%%%%%%%%%%%%%%%%%
% numbered versions
\declaretheoremstyle[headformat=swapnumber, spaceabove=\paraskip,
bodyfont=\itshape]{mystyle}
\declaretheoremstyle[headformat=swapnumber, spaceabove=\paraskip,
bodyfont=\normalfont]{mystyle-plain}
\declaretheorem[name=Lemma, sibling=para, style=mystyle]{Lemma}
\declaretheorem[name=Proposition, sibling=para, style=mystyle]{Proposition}
\declaretheorem[name=Theorem, sibling=para, style=mystyle]{Theorem}
\declaretheorem[name=Corollary, sibling=para, style=mystyle]{Corollary}
\declaretheorem[name=Definition, sibling=para, style=mystyle]{Definition}

% unnumbered versions
\declaretheoremstyle[numbered=no, spaceabove=\paraskip,
bodyfont=\itshape]{mystyle-empty}
\declaretheoremstyle[numbered=no, spaceabove=\paraskip,
bodyfont=\itshape]{mystyle-empty-plain}
\declaretheorem[name=Lemma, style=mystyle-empty]{Lemma*}
\declaretheorem[name=Proposition, style=mystyle-empty]{Proposition*}
\declaretheorem[name=Theorem, style=mystyle-empty]{Theorem*}
\declaretheorem[name=Corollary, style=mystyle-empty]{Corollary*}
\declaretheorem[name=Definition, style=mystyle-empty]{Definition*}
\declaretheorem[name=Remark, style=mystyle-empty]{Remark*}

% plain style
\declaretheoremstyle[
        headformat={{\bfseries\NUMBER.}{\itshape\NAME}\NOTE\ignorespaces},
        spaceabove=\paraskip, 
        headpunct={.},
        headfont=\itshape,
        bodyfont=\normalfont
        ]{mystyle-plain}
\declaretheorem[sibling=para, style=mystyle-plain]{Example}
\declaretheorem[sibling=para, style=mystyle-plain]{Remark}

% proofs, just as in amsthm but with adapted spacing

\makeatletter
\renewenvironment{proof}[1][\textit{Proof}]{\par
  \pushQED{\qed}%
  \normalfont \topsep.75\paraskip\relax
  \trivlist
  \item[\hskip\labelsep
        \itshape
    #1\@addpunct{.}]\ignorespaces
}{%
  \popQED\endtrivlist\@endpefalse
}
\makeatother

\newcommand\note[1]{\marginpar{{
\begin{flushleft}
\tiny#1
\end{flushleft}
}}}

\renewcommand\labelitemi{-}
\setenumerate[0]{label=(\alph*)}
%%%%%%%%%%%%%%%%%%%%%%%%%%% The usual stuff%%%%%%%%%%%%%%%%%%%%%%%%%
\newcommand\NN{\mathbb N}
\newcommand\CC{\mathbb C}
\newcommand\QQ{\mathbb Q}
\newcommand\RR{\mathbb R}
\newcommand\ZZ{\mathbb Z}

\newcommand\maps{\longmapsto}
\newcommand\ot{\otimes}
\newcommand\sq{\square}
\renewcommand\to{\longrightarrow}
\renewcommand\phi{\varphi}
\renewcommand\k{\Bbbk}
\newcommand\vspan[1]{\left\langle #1 \right\rangle}
\renewcommand\O{\mathcal O}
\newcommand\R{\mathcal R}
\newcommand\op{\mathsf{op}}
\newcommand\K{\mathcal K}
\newcommand\D{\mathcal D}
\newcommand\A{\mathcal A}
\renewcommand\L{\mathcal L}
\newcommand\F{\mathcal F}

\newcommand\range[1]{\left\llbracket#1\right\rrbracket}

\DeclareMathOperator\ISyb{\mathsf{ISyb}}
\DeclareMathOperator\YB{\mathsf{YB}}
\DeclareMathOperator\End{\mathsf{End}}
\DeclareMathOperator\Hom{\mathsf{Hom}}
\DeclareMathOperator\Mat{\mathsf{Mat}}
\DeclareMathOperator\EXT{\underline{\mathsf{Ext}}}
\DeclareMathOperator\TOR{\underline{\mathsf{Tor}}}

\DeclareMathOperator\im{Im}


%%%%%%%%%%%%%%%%%%%%%%%%%%%%%%%%%%%%%% TITLES %%%%%%%%%%%%%%%%%%%%%%%%%%%%%%
\title{
Algebras associated to set-theoretical solutions of YBE
}

\author{
}
\date{}

\begin{document}
\maketitle

\section{Generalities}

Given $n,m \in \ZZ$ we set $\range{n,m} = \{ k \in \ZZ \mid n \leq k \leq 
m\}$, and if $n \in \NN$ we will also write $\range{n} = \range{1,n}$. We work
over $\CC$, so all vector spaces, tensor products, etc. are over this field.

We denote by $\ISyb$ the category of all non-degenerate and involutive set 
theoretical solutions to the Yang-Baxter equation, and by $\YB$ the category
of all finite dimensional solutions over the complex numbers.

\paragraph
Let $n \in \NN$. Given $i,j \in \range{n}$ we denote by $e_i \in \CC^n$ the
$i$-th element of the canonical basis, and by $E^i_j \in \mathsf{Mat}_n(\CC)$
the matrix having a $1$ as its $(i,j)$-th entry, all other entries equal to 
zero. We identify $\CC^n \ot \CC^n$ with $\Mat_n(\CC)$ through the linear
correspondence $e_i \ot e_j \mapsto E^i_j$.

Define the map $i: \Mat_n(\CC) \otimes \Mat_n(\CC) \to \End(\Mat_n(\CC))$ where
for each $A, B, C \in \Mat_n(\CC)$ we have $i(A\ot B)(C) = A C^t B$. If $F 
\in \End(\Mat_n(\CC))$ and $F(E^i_j) = \sum_{k,l} \alpha_{i,k}^{j,l} E^k_l$ 
then setting $\overline F = \sum_{i,j,k,l} \alpha^{j,l}_{i,k} E^k_j \ot E^i_l$ 
we obtain $i(\overline F) =F$. Thus $i$ is an epimorphism, and since the 
dimensions of domain and codomain are the same it is an isomorphism, with 
inverse $F \mapsto \overline F$.

Denote by $\tau \in \End(\Mat_n(\CC))$ the transposition map. Then we have
\begin{align*}
\overline{F \circ \tau}
  &= \sum_{i,j,k,l} \alpha^{i,l}_{j,k} E^k_j \ot E^i_l
  = \sum_{i,j,k,l} \alpha^{j,l}_{i,k} E^k_i \ot E^j_l; \\
\overline{\tau \circ F}
  &= \sum_{i,j,k,l} \alpha^{j,l}_{i,k} E^l_j \ot E^i_k
  = \sum_{i,j,k,l} \alpha^{j,k}_{i,l} E^k_j \ot E^i_l.
\end{align*}

Denote by $\langle-, -\rangle$ the only inner product on $\CC^n \to \CC^n$ 
such that $\{e_i \ot e_j \mid i,j \in \range{n}\}$ forms an orthogonal basis. 
This is the pullback of the usual inner product on $\Mat_n(\CC)$ for which the
basis $\{E^i_j \mid i,j \in \range{n}\}$ is an orthogonal basis. We denote by 
$F^{t_1}$ the only map such that $\langle F^{t_1}(E^k_j), E^i_l \rangle = 
\langle F(E^i_j), E^k_l \rangle$. In other words $F^{t_1}(E^k_j) = \sum_{i,l}
\alpha^{j,l}_{i,k} E^i_l$. We define $F^{t_2}$ analogously, so $F^{t_2}(E^i_l)
= \sum_{k,j} \alpha_{i,k}^{j,l} E^k_j$. Thus we have
\begin{align*}
\overline{F^{t_1}}
  &= \sum_{i,j,k,l} \alpha_{k,i}^{j,l} E^k_j \ot E^i_l 
  = \sum_{i,j,k,l} \alpha_{i,k}^{j,l} E^i_j \ot E^k_l \\
\overline{F^{t_2}}
  &= \sum_{i,j,k,l} \alpha_{i,k}^{l,j} E^k_j \ot E^i_l
  = \sum_{i,j,k,l} \alpha_{i,k}^{j,l} E_i^k \ot E^i_j.
\end{align*}

\paragraph
Let $n \in \NN$ and let $(\range{n}, S)$ be an symmetric set, and for each 
$i,j \in \range{n}$ put $S(i,j) = (g_i(j), f_j(i))$. Then $S$ induces a 
classical solution to the YBE by setting $S(e_i \ot e_j) = e_{g_i(j)} \ot 
e_{f_j(i)}$. Using the notation from the previous paragraph
\begin{align*}
\overline S 
  &= \sum_{i,j} E_j^{g_i(j)} \ot E^i_{f_j(i)};
& \overline{ \tau \circ S }
  &= \sum_{i,j} E_j^{f_j(i)} \ot E^i_{g_i(j)};\\  
\overline{S^{t_1}}
  &= \sum_{i,j} E_j^i \ot E^{g_i(j)}_{f_j(i)}; 
&\overline{S^{t_2}}
  &= \sum_{i,j} E^{g_i(j)}_{f_j(i)} \ot E^i_j.
\end{align*}


\end{document}