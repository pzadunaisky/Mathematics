%%%%%%%%%%%%%%%%%%%%%% Generalities %%%%%%%%%%%%%%%%%%5
\documentclass[11pt,fleqn]{article}
\usepackage[paper=a4paper]
  {geometry}

\pagestyle{plain}
\pagenumbering{arabic}
%%%%%%%%%%%%%%%%%%%%%%%%%%%%%%%%
\usepackage[small]{titlesec}
%\usepackage{paragraphs}

\usepackage{hyperref}

\usepackage{amsthm,thmtools}
\usepackage{showlabels}
%\linespread{1.2}
\setlength{\parskip}{1.2ex}

\usepackage[utf8]{inputenc}
\usepackage[english]{babel}
\usepackage{enumerate}
\usepackage[osf,noBBpl]{mathpazo}
%\usepackage[alphabetic,initials]{amsrefs}
\usepackage{biblatex}
\usepackage{amsfonts,amssymb,amsmath}
\usepackage{mathtools}
\usepackage{graphicx}
\usepackage[poly,arrow,curve,matrix]{xy}
\usepackage{wrapfig}
\usepackage{xcolor}
\usepackage{helvet}
\usepackage{stmaryrd}
\usepackage{tikz}
\usepackage{mathdots}


\renewcommand\thesection{\arabic{section}}
\titleformat{\section}
  {\normalfont\bfseries\large}
  {\thesection. \space}{0em}{}

\renewcommand\thesubsection{\arabic{subsection}}
\titleformat{\subsection}
  {\normalfont\bfseries}
  {\thesection.\thesubsection \space}{0em}{}

\renewcommand\proofname{Proof}

\makeatletter
\renewenvironment{proof}[1][\textit{\proofname}]{\par
  \pushQED{\qed}%
  \normalfont \topsep.75\paraskip\relax
  \trivlist
  \item[\hskip\labelsep
        \itshape
    #1\@addpunct{.}]\ignorespaces
}{%
  \popQED\endtrivlist\@endpefalse
}
\makeatother
%%%%%%%%%%%%%%%%%%%%%%%%%%% Theorems et al%%%%%%%%%%%%%%%%%%%%%%%%%%
\declaretheoremstyle[
%headformat=swapnumber, 
bodyfont=\itshape,]{mystyle}
\declaretheorem[name=Lemma, style=mystyle, numberwithin=section]{Lemma}
\declaretheorem[name=Proposition, style=mystyle, sibling=Lemma]{Proposition}
\declaretheorem[name=Theorem, style=mystyle, sibling=Lemma]{Theorem}
\declaretheorem[name=Corollary, style=mystyle, sibling=Lemma]{Corollary}
\declaretheorem[name=Definition, style=mystyle, sibling=Lemma]{Definition}
%\declaretheorem[name=Example(s), style=mystyle]{Example}
%\declaretheorem[name=Remark, style=mystyle]{Remark}

% unnumbered versions
\declaretheoremstyle[numbered=no, 
bodyfont=\itshape]{mystyle-empty}
\declaretheorem[name=Lemma, style=mystyle-empty]{Lemma*}
\declaretheorem[name=Proposition, style=mystyle-empty]{Proposition*}
\declaretheorem[name=Theorem, style=mystyle-empty]{Theorem*}
\declaretheorem[name=Corollary, style=mystyle-empty]{Corollary*}
\declaretheorem[name=Definition, style=mystyle-empty]{Definition*}
\declaretheorem[name=Example(s), style=mystyle-empty]{Example*}
\declaretheorem[name=Remark, style=mystyle-empty]{Remark*}

%%%%%%%%%%%%%%%%%%%%%%%%%%% Paragraphs %%%%%%%%%%%%%%%%%%%%%%%%%%%%%
\newskip\paraskip
\paraskip=0.75ex plus .2ex minus .2ex

\newcounter{para}[section]
\setcounter{para}{0}
\renewcommand\thepara{\thesection.\arabic{para}}
\def\paragraph{%
  \noindent
  \refstepcounter{para}%
  \textbf{\thepara.}\hspace{1ex}%
}

\newcommand\about[1]{%
  {\bfseries#1.}%
}

\newcommand\pref[1]{\textbf{\ref{#1}}}

\renewcommand\theHpara{\theHsection.\arabic{para}}
%%%%%%%%%%%%%%%%%%%%%%%%%%% The usual stuff%%%%%%%%%%%%%%%%%%%%%%%%%
\newcommand\NN{\mathbb N}
\newcommand\CC{\mathbb C}
\newcommand\QQ{\mathbb Q}
\newcommand\RR{\mathbb R}
\newcommand\ZZ{\mathbb Z}
\newcommand\II{\mathbb I}
\renewcommand\k{\Bbbk}

\newcommand\maps{\longmapsto}
\newcommand\ot{\otimes}
\renewcommand\to{\longrightarrow}
\renewcommand\phi{\varphi}
\newcommand\Id{\mathsf{Id}}
\newcommand\im{\mathsf{im}}
\newcommand\coker{\mathsf{coker}}
%%%%%%%%%%%%%%%%%%%%%%%%% Specific notation %%%%%%%%%%%%%%%%%%%%%%%%%
\newcommand\A{\mathcal A}
\newcommand\B{\mathcal B}
\newcommand\D[3]{{}^{#1} \mathfrak D_{#2}^{#3}}
\newcommand\DD[3]{{}^{#1} \mathcal D_{#2}^{#3}}
\newcommand\Z{\mathsf Z}

\newcommand\g{\mathfrak g}
\newcommand\p{\mathfrak p}
\newcommand\m{\mathfrak m}
\newcommand\gl{\mathfrak{gl}}
\newcommand\gen{\mathsf{gen}}
\newcommand\std{\mathsf{std}}
\newcommand\sh{\mathsf{sh}}
\renewcommand\SS{\mathfrak S}
\newcommand\vv{\overline{v}}

\newcommand\vectspan[1]{\left\langle #1 \right\rangle}
\newcommand\interval[1]{\llbracket #1 \rrbracket}

\DeclareMathOperator\Frac{Frac}
\DeclareMathOperator\Specm{Specm}
\DeclareMathOperator\End{End}

\DeclareMathOperator\sym{sym}
\DeclareMathOperator\asym{asym}
\DeclareMathOperator\sg{sg}
\DeclareMathOperator\ev{\mathsf{ev}}
\DeclareMathOperator\ann{ann}
\DeclareMathOperator\supp{supp}

\newcommand\bigmodule{big GT module}
\newcommand\module[1]{V(T(#1))}


\begin{document}
%%%%%%%%%%%%%%%%%%%%%%%%%%%%%%%%%%%%%% TITLES %%%%%%%%%%%%%%%%%%%%%%%%%%%%%%
\title{Gelfand-Tsetlin Theory for Rational Galois Algebras}
%\author{[gamma-structure.tex]}
\date{}



\author[V.Futorny]{Vyacheslav Futorny}
\address{Instituto de Matem\'atica e Estat\'istica, Universidade de S\~ao
Paulo,  S\~ao Paulo SP, Brasil} \email{futorny@ime.usp.br,}
\author[D.Gratcharov]{Dimitar Grantcharov}
\address{\noindent
University of Texas at Arlington,  Arlington, TX 76019, USA} \email{grandim@uta.edu}
%\author[A.Molev]{Alexander Molev}
%\address{\noindent
%University of Sydney,  Sydney, Australia} \email{alexm@maths.usyd.edu.au }
\author[L.E.Ramirez]{Luis Enrique Ramirez}
\address{Universidade Federal do ABC,  Santo Andr\'e-SP, Brasil} \email{luis.enrique@ufabc.edu.br,}
\author[P.Zadunaisky]{Pablo Zadunaisky}
\address{Instituto de Matem\'atica e Estat\'istica, Universidade de S\~ao
Paulo,  S\~ao Paulo SP, Brasil} \email{pzadun@ime.usp.br}


\begin{abstract}
?????????
\end{abstract}
\begin{document}
\maketitle
%\vspace{-2cm}





\section{Introduction}
A class of Galois algebras (rings and orders) was introduced in \cite{FO1-galois-orders} to combine the successful representation theories of generalized Weyl algebras   \cite{ba}, \cite{bavo} and the universal enveloping algebra of $\gl_n$ \cite{O}. The structure of these algebras intrinsically determined by a certain commutative subalgebra $\Gamma$.  
 The class of  Galois algebras   is a
noncommutative generalization of a classical notion of $\Gamma$-order
in skew group rings. 
Classical example of a Galois order is the universal enveloping algebra of
$\gl_n$ over the Gelfand-Tsetlin subalgebra \cite{dfo:gz},
\cite{dfo:hc}. Other well known examples  include:
generalized Weyl algebras over integral domains  such as the $n$-th Weyl algebra, the quantum plane, the $q$-deformed Heisenberg algebra, quantized
Weyl algebras, the Witten-Wo\-ro\-no\-wicz algebra 
\cite{ba},   finite
$W$-algebras of type $A$ \cite{FMO}) and, more generally, \emph{principal Galois orders} \cite{Hart}. Representations of Galois algebras were studies in \cite{FO2}. In particular, Gelfand-Tsetlin
modules over a Galois algebra $U$ were analysed. These modules have a generalzied $\Gamma$-eigenspace decomposition as $\Gamma$-modules.  Maximal ideals of $\Gamma$ "almost" parametrize 
simple Gelfand-Tsetlin
modules over Galois orders, as given such maximal ideal there exists only finitely many non-isomorphic simple Gelfand-Tsetlin
modules with this maximal ideal in their support. Nevertheless, this "finiteness" is the main obstacle for the complete classification of simple Gelfand-Tsetlin
modules. Recently breakthrough results in the $\gl_n$ case were obtained in  \cite{FGR1}, \cite{FGR2}, \cite{FGR3},  \cite{Z}, \cite{V1}, \cite{V2}, \cite{RZ}.   

Attempts to extend the developed   Gelfand-Tsetlin theory in the $\gl_n$ case to larger classes of Galois algebras were made in \cite{M}, \cite{EMV} and \cite{Hart}.
In particular, a class of \emph{Rational Galois algebras} was introduced in \cite{Hart} providing a framework for the study of Gelfand-Tsetlin representations for a large class of Galois algebras, which includes such algebras as the universal enveloping algebra of $\gl_n$, its quantum version, restricted Yangians of $\gl_n$ and, more generally, finite $W$-algebras of type $A$ among the others.




Some classes of simpe Gelfand-Tsetlin
modules for $\gl_n$ and for its quantization were constructed in \cite{FRZ1}, \cite{FRZ2}. The classification is known only in the generic \cite{FGR1} and $1$-singular \cite{FGR4} cases. On the other hand, explicit construction of all  simple Gelfand-Tsetlin
modules  remains at large unsolved.




The purpose of the current paper is  a deeper study of  Gelfand-Tsetlin representations for Rational Galois algebras. Let $V$ be a complex vector space, $G$  a reflection group acting on $V$, and hence on its 
symmetric algebra $\Lambda$ and its field of rational functions $L$. We view elements of $\Lambda$ as polynomial unctions on $V^*$. 
We fix
a Rational Galois algebra $U \subset (L*V)^G$ and set $ 
\Gamma=S(V)^G$. Any element $v\in V$ defines a character of $\Gamma$ via the evaluation. 

We say that a finitely generated $U$-module $M$ is a  {\em Gelfand-Tsetlin module} (with respect to $\Gamma$)
 if 
 \begin{equation*} M=\underset{{\bf m} \in \Specm {\Gamma}}{\bigoplus}M({\bf m}),
\end{equation*}
where
 \begin{equation*} M({\bf m}) \ = \ \{ x\in M\ | \exists k, \ {\bf m}^k x
=0\}.
\end{equation*}

The
{support}{} of a Gelfand-Tsetlin module $M$ is a subset of  $\Specm {\Gamma}$ consisting of those ${\bf m}$ fsuch that
$M({\bf m})\ne 0$.


The main result of \cite{Hart} is the \emph{fundamental} Gelfand-Tsetlin $U$-module structure on $\Gamma^*$ for any Rational Galois algebra which generalizes \cite{EMV}. 
Our main result
(Theorem \ref{thm-module-structure}) describes an explicit basis of the cyclic Gelfand-Tsetlin $U$-module generated by an arbitrary character of $\Gamma$. This  provides a uniform method of constructing representations of rational Galois algebras.  Our second main result is Theorem \ref{thm-Jordan} which gives bounds for the size of Jordan matrices of the generators of $\Gamma$. Finally, Theorem \ref{thm-simplicity} establishes the simplicity criterion of the fundamental  Gelfand-Tsetlin $U$-module for \emph{integral rational  Galois algebras} in type $A$ providing, in particular, new simple modules for any finite $W$-algebra of type $A$.


\noindent{\bf Acknowledgements.}  V.F. is
supported in part by  CNPq grant (301320/2013-6) and by 
Fapesp grant (2014/09310-5). D.G is supported in part by Simons Collaboration Grant 358245.  P.Z.  is
supported by Fapesp fellowship (2016-25984-1). 



\section{Preliminaries}
\paragraph
\about{Root systems and reflection groups}
Let $V$ be a complex vector space with a fixed inner product which we denote 
by $(-,-)$. As usual we identify $V$ with its dual and for each $\alpha \in
V^*$ we denote by $v_\alpha$ the unique element of $V$ such that $\alpha(v') = 
(v', v_\alpha)$ for all $v' \in V$. Given $\alpha \in V^*$ we denote by 
$s_\alpha$ the orthogonal reflection through the hyperplane $\ker \alpha$, and 
by $s_\alpha^*$ the corresponding endomorphism of $V^*$. In this article a
finite root system over $V$ will be a finite set $\Phi \subset V^*$ such that 
for each $\alpha \in \Phi$ we have 
\begin{itemize}
\item[(R1)] $\Phi \cap \RR \alpha = \{\pm \alpha\}$ and
\item[(R2)] $s_\alpha^*(\Phi) \subset \Phi$.
\end{itemize}

In classical references such as \cite{Hump-coxeter-book} and 
\cite{Hiller-coxeter-book} root systems are defined as subsets of a euclidian
vector space $V_\RR$, which a more natural definition in general. However,
since we work mostly over complex vector spaces endowed with the action of a 
reflection group we use the definition above, which is clearly equivalent. We 
now review the basic features of the theory of root systems.

The Weyl group associated to $\Phi$ is the group $W(\Phi)$ generated by 
$\{s_\alpha \mid \alpha \in \Phi\}$. We do not assume the root system to be 
reduced or cristallographic, nor that $\Phi$ generates $V_\RR^*$, so $W(\Phi)$ 
is a finite reflection group which may be decomposable and its action on $V$ 
may have a nontrivial stabilizer. Any reflection group $G \subset 
\mathsf{GL}(V)$ is the Weyl group of some root system $\Phi \subset V^*$.

Just as in the case of root systems for Lie algebras we can choose a subset 
$\Phi^+ \subset \Phi$ such that $\Phi = \Phi^+ \cup - \Phi^+$ and this 
determines a set of simple roots, or basis, $\Sigma \subset \Phi^+$ such that 
each $\alpha \in \Phi^+$ is of the form $\sum_{\alpha \in \Sigma} c_\alpha 
\alpha$ with $c_\alpha \in \RR_{\geq 0}$. If we fix a set of positive roots 
$\Phi^+$ with its corresponding set of simple roots $\Sigma$ then the set $S$ 
of reflections corresponding to simple roots is a minimal generating set of the
reflection group $W = W(\Phi)$, and hence $(W,S)$ is a finite Coxeter system 
in the sense of \cite{Hump-coxeter-book}*{1.9}. Each $s \in W$ of order two is 
of the form $s_\alpha$ for some $\alpha \in \Phi^+$ 
\cite{Hump-coxeter-book}*{Proposition 2.14}, and given $s \in W$ of order 
two we denote by $\alpha_s$ the corresponding positive root.

Once we fix a set of simple roots $\Sigma$, or equivalently a minimal 
generating set $S \subset W$, we obtain a notion of length of an element 
$\sigma \in W$, denoted by $\ell(\sigma)$ and defined as the least natural 
number $\ell$ such that $\sigma$ can be written as a composition of $\ell$ 
reflections in $S$. The group $W$ acts faithfully and transitively on the set 
of roots $\Phi$ and $\ell(\sigma)$ is also equal to the number $|\sigma(\Phi^+)
\cap -\Phi^+|$, so $W$ has a unique longest longest element whose length 
equals $|\Phi|$. We will denote this element by $\omega_0(W)$, or simply 
$\omega_0$ if the group is clear from the context.

For the rest of this section we fix a root system $\Phi$ with simple roots 
$\Sigma$ and denote by $(W,S)$ be the corresponding Coxeter system. 

\paragraph
\about{Subsystems, subgroups and stabilizers} 
\cite{Hump-coxeter-book}*{1.10}
Given $\Omega \subset \Sigma$ we denote by 
$\Phi(\Omega)$ the root subsystem generated by $\Omega$. We will call such 
subsystems \emph{standard}. If $\Psi \subset \Phi$ is an arbitrary subsystem 
then we can choose a basis $\Omega \subset \Psi$, and that basis can be 
extended to a basis $\overline \Omega$ of $\Phi$. By 
\cite{Hump-coxeter-book}*{1.4 Theorem} $W$ acts transitively on the set of all
bases of $\Phi$, so for some $\sigma \in W$ we have $\sigma(\overline \Omega) =
\Sigma$ and hence $\sigma(\Psi)$ is standard. 

Let $\theta \subset S$ and denote by $W_\theta$ the subgroup of $W$ generated 
by $\theta$. Then $(W_\theta, \theta)$ is also a Coxeter system and it 
determines a standard root system $\Phi_\theta \subset \Phi$ with simple roots 
$\Sigma_\theta = \{\alpha_s \mid s \in \theta\}$. We will refer to subgroups 
of the form $W_\theta$ as \emph{standard parabolic} subgroups. A parabolic 
subgroup is any subgroup of $W$ that is conjugate to a standard parabolic 
subgroup.

If $\sigma \in W_\theta$ then we can compute its length as an element of $W$ 
with respect to the generating set $S$ or as an element of $W_\theta$ with 
respect to the generating set $\theta$; however both lengths turn out to be 
equal, so we denote both by $\ell(\sigma)$. Since $W_\theta$ is also a Coxeter 
group it has a unique element of maximal length which we will denote by 
$\omega_0(\theta)$. The set $W^\theta = \{\sigma \in W \mid \ell(\sigma s) > 
\ell(\sigma) \mbox{ for all } s \in \theta\}$ is a set of representatives of 
the quotient $W/W_\theta$, and for each $\sigma \in W$ there exist unique 
elements $\sigma^\theta \in W^\theta$ and $\sigma_\theta \in W_\theta$ such 
that $\sigma = \sigma^\theta\sigma_\theta$ with $\ell(\sigma) = 
\ell(\sigma^\theta) + \ell(\sigma_\theta)$. The element $\sigma^\theta$ is the 
element of minimal length in the coclass $\sigma W_\theta$. It follows that 
$(\omega_0)_\theta = \omega_0(\theta)$ and therefore $\omega_0^\theta = 
\omega_0 \omega_0(\theta)^{-1}$.

Given $v \in V$ we denote by $\Phi_0(v)$ the set of all roots in $\Phi$ such 
that $\alpha(v) = 0$, which is clearly a root subsystem of $\Phi$. We will say 
that $v$ is $\Sigma$-standard, or just standard when $\Sigma$ is fixed or 
clear from the context, if $\Phi_0(v)$ is a $\Sigma$-standard subsystem of 
$\Phi$. It is immediate that $v$ is standard if and only if $W_v$ is a 
standard parabolic subgroup. Since $W_{\sigma(v)} = \sigma W_v \sigma^{-1}$ 
and $\Phi_0(\sigma(v)) = \sigma \Phi_0(v)$ for all $\sigma \in W$  it follows 
that for every $v \in V$there exists $\sigma \in W$ such that $\sigma(v)$ is 
standard and hence $W_{\sigma(v)}$ is a standard parabolic subgroup. If $v$ is 
standard then we denote by $W^v$ the set of minimal length representatives of 
the left coclasses $W/W_v$.





\section{Galois orders}

\paragraph
\about{Galois orders}
Now we recall a notion of a Galois ring (order) that was introduced in 
\cite{FO-galois-orders}. 
Let $R$ be a ring,  $\mathcal M$ a monoid acting on $R$ by ring automorphisms and  $G$  a finite subgroup of automorphisms of $R$ acting on $\mathcal M$ by conjugation. Then the action of $G$ extends  naturally to the action on the skew monoid ring $R* \mathcal M$.
Let $\Gamma$ be an integral domain, $K$  the field of fractions of $\Gamma$, $L$  a finite Galois extension of $K$ with 
the Galois group $G$,   $K=L^G$.
The monoid
 $\mathcal M$ is assumed to be \emph{$K$-separating}: $m_1|_K=m_2|_K\Rightarrow m_1=m_2$ for $m_1,m_2\in\mathcal M$.
 
 
\begin{Definition}
\begin{itemize}
\item[(i)] 
 A \emph{Galois ring over $\Gamma$} is
a  finitely generated $\Gamma$-subring $U\subset (L\ast\mathcal M)^G$ such that $UK=KU=\mathcal{K}$.
\item[(ii)] 
A Galois ring $U$ over  $\Gamma$.
  is
 \emph{right (respectively left)  Galois order}, if for any finite dimensional right (respectively left) $K$-subspace
$W\subset U[S^{-1}]$ (respectively $W\subset [S^{-1}]U$), $W\cap U$
is a finitely generated right (respectively left) $\Gamma$-module. A
Galois ring is \emph{Galois order}{} if it is both right
and left  Galois order.
\end{itemize}
\end{Definition}






We will always assume that all Galois rings are $\k$-algebras. In this case we say that a Galois ring  is a Galois algebra over $\Gamma$.
Representation theory of Galois orders  was developed in \cite{FO2}. If $U\subset (L*\mathcal M)^{G}$ be a Galois ring over an integral domain $\Gamma$ and $\bf m$ a maximal ideal of $\Gamma $. Denote by $\Phi(\bf m)$ the number of non-isomorphic simple Gelfand-Tsetlin modules $M$ for which $M(\bf m)\neq 0$. We will call this number the fiber of $\bf m$. It is absolutely not obvious why the fiber of $\bf m$ should be nontrivial anf finite. 
Sufficient
conditions for  the fiber $\Phi(\bf m)$ to be nontrivial and finite were established in \cite{FO2}. 

Consider  the integral closure $\bar{\Gamma}$
of $\Gamma$ in $L$. It is a standard fact that if $\Gamma$ is finitely generated as a $\k$-algebra then any character of $\Gamma$ has finitely many extensions to characters of 
$\bar{\Gamma}$. 

Let $\bar{\bf m}$  be any lifting of $\bf m$ to the integral closure
of $\Gamma$ in $L$, and $\mathcal M_{\bf m}$ the stabilizer of $\bar{\bf m}$ in $\mathcal M$. Note that the group $\mathcal M_{\bf m}$ is defined uniquely up to $G$-conjugation.
Thus the cardinality of $\mathcal M_{\bf m}$ does not depend on the choice of the lifting. We denote it by $|\bf m|$.



%If $a\in L*\mathcal M$ and  $a=\sum_{m\in \mathcal M}{} a_{m} m$, where
% $a_{m}\in L$. Set 
%$$d(a)=\{m\in\mathcal M|a_{m}\ne0\}.$$

%For a finite $G$-invariant subset $S\subset \mathcal M$ and a $\Ga$-subbimodule $B\subset  (L*\mathcal M)^{G}$ set
%\begin{equation*}
%\label{equation-definition-of-support-subsets}
%B(S)=\{a\in B\mid d(a) \subset S\}.
%\end{equation*}
%Denote by $r(\bm,\bn)$ the minimal number of generators of
%$U(S(\bm,\bn))$ as a right $\Ga$-module.


\begin{Theorem}\label{theorem-extension} [\cite{FO2}, Theorem A, , Theorem 8]
Let $\Gamma$ be a commutative domain
which is finitely generated as a $\k$-algebra,  $U\subset (L*\mathcal M)^{G}$ a right Galois order over $\Gamma$, $\bf m\in
\Specm \Gamma$. Suppose  $|\bf m|$ is finite.
\begin{itemize}
\item[(i)]\label{fiber-nontrivial} The fiber $\Phi(\bf m)$ is non-empty.
\item[(ii)]\label{fiber-finite} If $U$ is a
  Galois order  over
$\Gamma$, then the fiber $\Phi(\bf m)$ is finite and bounded..
\item[(iii)] Let $U$ be a Galois order over $\Gamma$, where $\Gamma$ is a
normal noetherian $\k$-algebra, and   $M$ is simple Gelfand-Tsetlin $U$-module with respect to $\Gamma$.  Then $M(\bf m)$ is finite
    dimensional and  bounded.
    \end{itemize}
\end{Theorem}






\paragraph
\label{L:translation}
Recall that $V$ is a complex vector space, and let $G \subset \mathsf{GL}(V)$ 
be a pseudo reflection group. 
Let $\Lambda = S(V)$ be the algebra of polynomial functions on $V$, and let
$L = \Frac(\Lambda)$ be the field of rational functions on $V$. As usual the
action of $G$ on $V$ induces actions on $\Lambda$ and $L$, and we denote by
$\Gamma$ the algebra of $G$-invariant elements of $\Lambda$ and set $K = L^G$.

Let $L \hookrightarrow \End_\CC(L)$ be the algebra map that sends any rational 
function $f \in L$ to the $\CC$-linear map $m_f: f' \in L \mapsto ff' \in L$.
This is an algebra map, and although $\End_\CC(L)$ is not a $L$-algebra it is
at least an $L$-vector space, with $f \cdot \phi = m_f \circ \phi$ for all
$\phi \in \End_\CC(L)$. Also $G$ acts on $\End_\CC(L)$ by conjugation and 
$\sigma \cdot m_f = \sigma \circ m_f \circ \sigma^{-1} = m_{\sigma(f)}$, so 
the map $f \mapsto m_f$ is $G$-equivariant. We will often simply write $f$ for 
the operator $m_f$.

Given $v \in V$ we define a map $a_v: V \to V$ given by $a_v(v') = v'+v$. This
in turn induces an endomorphism of $\Lambda$, which we denote by $t_v$, given  
bt $t_v(f) = f \circ a_{v}$; we sometimes write $f(x+v)$ for $t_v(f)$. Of
course each map $t_v$ can be extended to a $\CC$-linear operator in $L$ and
$t_v \circ t_{v'} = t_{v+v'}$, so we obtain a group morphism $v \in V \mapsto
t_v \in \End_\CC(L)$. The definitions imply that $\sigma \cdot t_v = 
t_{\sigma(v)}$ for each $\sigma \in G$ so this map is $G$-equivariant.


\begin{Lemma}
Let $G$, $V$ and $L$ be as above. Let $A \subset L$ be a subalgebra and let 
$Z \subset V$ be an arbitrary subset. We denote by $A * Z$ the $A$-module 
generated by the monoid $\{t_z \mid z \in Z\}$ in $\End_\CC(L)$. 
\begin{enumerate}
\item 
\label{i:translation-li}
The set $\{t_z \mid z \in Z\}$ is linearly independent over $L$.

\item 
\label{i:translation-algebra}
The subspace $L*V$ is a $\CC$-algebra with product given by $(ft_v)(f't_{v'}) 
= f t_v(f') t_{v+v'}$ for each $f,f' \in L, v,v' \in V$.

\item
\label{i:translation-subalgebra}
If $t_z(A) \subset A$ for each $z \in Z$ then $A * Z$ is a free $A$-module 
with basis $\{t_z \mid z \in Z\}$ and a subalgebra of $L * V$. Furthermore if 
$A$ and $Z$ are stable by the action of $G$ then so is $A * Z$.
\end{enumerate}
\end{Lemma}
\begin{proof}
Put $T = \sum_{i=1}^N f_i t_{z_i}$ where $f_i \in L^\times$ and each $z_i 
\in Z$, and assume $T = 0$. Given $p \in \Lambda$ we obtain that $0 = T(p) = 
\sum_i f_i p(x+z)$, or equivalently $p \sum_i f_i = \sum_i [p(x+z_i)- p] f_i$. 
Let $v \in V$ be arbitary and choose a polynomial $p$ of positive degree such 
that $p(v) = p(v + z_j)$ for all $j \neq i$ but $p(v + z_i) = p(v) + 1$. Then 
$0 = p(v+z_i) f_i(v)$ so $f_i(v) = 0$. Since $v$ was arbitrary this implies 
that $f_i = 0$ and we have proved item \ref{i:translation-li}. Item 
\ref{i:translation-algebra} is immediate from the definitions and item 
\ref{i:translation-subalgebra} is an easy consequence. 
\end{proof}


\paragraph
\about{Rational Galois Algebras} 
Given a character $\chi: G \to \CC^\times$, the space of relative invariants 
$\Lambda^G_\chi = \{p \in \Lambda \mid g \cdot p = \chi(g)p\} \subset 
\Lambda$ is a $\Lambda^G$ submodule of $\Lambda$, and by a theorem of Stanley 
\cite{Hiller-coxeter-book}*{(4.4) Proposition} $\Lambda^G_\chi$ is a free 
$\Lambda^G$-module of rank $1$. The generator of $\Lambda^G_\chi$ can be
presented explictly: it is $d_\chi = \prod_{H \in \A(G)} (\alpha_H)^{a_H}$,
where $\A(G)$ is the set of hyperplanes fixed pointwise by some element of 
$G$, each $\alpha_H$ is a linear form such that $\ker \alpha_H = H$, and 
$a_H \in \NN_0$ is minimal with respect to the property that 
$\det[s_H^*]^{a_H} = \chi(s_H)$, for $s_H$ an arbitrary generator of the 
stabilizer of $H$ in $G$ (the number $a_H$ is independent of the choice of 
$s_H$). Notice that if $G$ is a Coxeter group as above then $a_H$ is either 
$1$ or $0$.

\begin{Definition}[\cite{Hart-rational-galois}*{Definition 4.3}]
A \emph{Rational Galois algebra} is a subalgebra $U \subset \End_\CC(L)$
generated by $\Gamma$ and a finite set of operators $\mathcal X \subset 
(L * V)^G$ such that for each $X \in \mathcal X$ there exists 
$\chi \in \hat G$ with $d_\chi X \in \Lambda * V$.
\end{Definition}

Given $X \in L*V$ we define its support as the set of all $v \in V$ such that
$t_v$ appears with nonzero coefficient in $X$; this is well defined since the 
set $\{t_v \mid v \in V\}$ is free over $L$. We denote the support of $X$
by $\supp X$. Given a Rational Galois Algebra $U \subset (L*V)$ we denote by 
$\Z(U)$ the additive monoid generated by $\{\supp X \mid X \in U\}$ in $V$.
Then \cite{Hart-rational-galois}*{Theorem 4.2} says that $U$ is a Galois order 
in $(L* \Z)^G$.





%\newpage
\section{Divided differences and Postnikov-Stanley operators}
We recall that $V$ is a complex vector space and we denote by $\Lambda$ the 
symmetric algebra of $V$ and by $L$ the fraction field of $\Lambda$. We fix
a finite root system $\Phi$ with basis $\Sigma$, and set $W = W(\Phi)$ to be
the corresponding reflection group with minimal generating set $S$. Thus $W$ 
acts on $\Lambda$ and $L$, and we set $\Gamma = \Lambda^W$ and $K = L^W$.



\paragraph
\about{Divided differences}
The smash product $L \# W$ is the $L$ vector space with basis $W$ endowed 
with the product defined by $f \sigma \cdot g \tau = f \sigma(g) \sigma \tau$.
There is an embedding $L \# W \hookrightarrow \End_\CC(L)$ given by sending
$f \sigma \in L \# W$ to the function $l \in L \mapsto f\sigma(l)$.
In particular $L \# W$ acts on $L$ and given $l\sigma \in L\# W$ and $f \in 
L$ we must be careful to distinguish between the result of applying the 
operator $l \sigma$ to $f$, which gives the result $l\sigma(f)$, and their 
product in the smash product $L \#W$, which is $l \sigma \cdot f = l\sigma(f) 
\sigma$. 

Let $s\in W$ be a reflection and set
\begin{align*}
\nabla_s 
	&= \frac{1}{\alpha_s}(1-s) \in L \# W.
\end{align*}
As is easily shown from this definition, for each $f \in L$ we have $\nabla_s 
\cdot f = \nabla_s(f) + s(f)\cdot \nabla_s$ so $\nabla_s$ is a twisted 
derivation of $L$. Notice that $\ker \nabla_s$ is exactly $L^{\vectspan s}$ 
and so $\nabla_s$ is $L^{\vectspan s}$-linear. Also it follows from the
definition that $\nabla_s(\Lambda) \subset \Lambda$.

Given $\sigma \in W$ we take a reduced decomposition $\sigma = s_1 \cdots 
s_{\ell}$ and set $\partial_\sigma = \nabla_{s_1} \circ \cdots \circ 
\nabla_{s_\ell}$; this element is called the \emph{divided difference} 
corresponding to $\sigma$ and does not depend on the chosen reduced 
decomposition \cite{Hiller-coxeter-book}*{Chapter IV (1.6)}. Notice though that
the definition of $\partial_\sigma$ does depend on the choice of a basis 
$\Sigma \subset \Phi$. 

By definition an $L\# W$-module is the same as an $L$ vector space endowed 
with a $W$-module structure such that the action of $L$ is $W$-equivariant. A 
simple induction on the length of $\sigma$ shows that the divided operator
$\partial_\sigma$ defines a $K$-linear map over any $L\# W$-module $Z$. 
In particular $L$ is such a module and $\Lambda \subset L$ is closed under the 
action of divided differences.

\paragraph
\about{Coinvariant spaces and Schubert polynomials}
The algebra $\Lambda$ has a natural $\NN$-grading with $\Lambda_1 = V^*$ and 
$\Gamma$ is a graded subalgebra. We denote by $I_W$ the ideal generated in 
$\Lambda$ by the elements of $\Gamma$ of positive degree. By the 
Chevalley-Shephard-Todd theorem $\Gamma$ is isomorphic to a polynomial 
algebra in $\dim_\CC V$ variables, $\Lambda$ is a free $\Gamma$-module of rank 
$|W|$, and a set $B \subset \Lambda$ is a basis of $\Lambda$ as a 
$\Gamma$-module if and only if its image in the quotient $\Lambda/I_W$ is a 
$\CC$-basis. Furthermore $\Lambda/I_W$ is naturally a graded $W$-module 
isomorphic to the regular representation of $W$ with Hilbert series 
$\sum_{\sigma \in W} t^{\ell(\sigma)}$. For proofs see 
\cite{Hiller-coxeter-book}*{Chapter II, section 3}.

We now recall the construction of a basis of $\Lambda/I_W$ by Schubert 
polynomials, due to Demazure \cite{Dem-schubert} and Bernstein, Gelfand and 
Gelfand \cite{BGG-cohomology} in the case where $W$ is a Weyl group, and
generalized to Coxeter groups by Hiller in 
\cite{Hiller-coxeter-book}*{chapter IV}. Set 
$\Delta(\Phi) = \prod_{\alpha \in \Phi^+} \alpha$, and for each $\sigma \in W$ 
set $\SS_\sigma^\Sigma = \partial_{\sigma^{-1}\omega_0} \Delta(\Phi)$. We will 
often write $\SS_\sigma$ instead of $\SS_\sigma^\Sigma$ when the basis is 
clear from the context. Notice that by definition $\deg \SS_\sigma 
= \ell(\sigma)$. The polynomials $\{\SS_\sigma \mid \sigma \in W\}$ are known 
as the \emph{Schubert polynomials} associated to the pair $(\Phi, \Sigma)$ and 
form a basis of $\Lambda$ as a $\Gamma$-module, so their images form a basis 
of $\Lambda/I_W$ as a complex vector space. Since $K = L^W$ we know that $L$ 
is a $K$-vector space of dimension $|W|$ and so $\{\SS_\sigma \mid \sigma \in 
W\}$ is also a basis of $L$ over $K$. Given $f \in L$ we will denote by 
$f_{(\sigma)}$ the coefficient of $\SS_\sigma$ in the expansion of $f$ in this 
basis, so $f = \sum_{\sigma \in G} f_{(\sigma)} \SS_\sigma$.

Since Schubert polynomials form a basis of $\Lambda / I_W$ there exist 
$c_{\sigma, \tau}^\rho \in \CC$ defined implicitly by the equation
\begin{align*}
\SS_\sigma \SS_\tau 
	&= \sum_{\rho \in W} c^\rho_{\sigma, \tau} \SS_\rho \mod I_W.
\end{align*}
The coefficients $c_{\sigma, \tau}^\rho$ are the \emph{generalized 
Littlewood-Richardson coefficients} attached to the basis $\Sigma$. It follows 
from the definition that $c^\rho_{\sigma, \tau} = 0$ unless $\ell(\sigma) + 
\ell(\tau) = \ell(\rho)$. If $\theta \subset S$ then the space of 
$W_\theta$-invariants $(\Lambda/I_W)^{W_\theta}$ is generated by the set 
$\{\SS_\sigma \mid \sigma \in W^\theta\}$ 
\cite{Hiller-coxeter-book}*{Chapter IV (4.4)}, so in particular if $\sigma, 
\tau \in W^\theta$ then $c^{\rho}_{\sigma, \tau} \neq 0$ implies that 
$\rho \in W^\theta$. 

\paragraph
\about{Postnikov-Stanley operators}
\label{ps-operators}
Given $\alpha \in V^*$ there is a unique $\CC$-linear derivation 
$\Theta(\alpha): \Lambda \to \Lambda$ such that $\Theta(\alpha)(\beta) = 
(\beta, \alpha)$ for each $\beta \in V^*$. This map extends uniquely to a 
morphism $\Theta: \Lambda \to \mathsf{Der}_\CC(\Lambda, \Lambda)$. Fixing an 
orthonormal basis $x_1, \ldots, x_n$ of $V^*$ so $S(V) \cong \CC[x_1, \ldots, 
x_n]$ we get that $\Theta(x_i) = \frac{\partial}{\partial x_i}$.

Let $(-,-)_\Theta: \Lambda \times \Lambda \to \CC$ be the bilinear form given 
by $(f,g) = \Theta(f)(g)(0)$. This is a nondegenerate bilinear form which 
identifies $\Lambda$ with its graded dual. For every graded ideal $I \subset 
\Lambda$ we write $\mathcal H_\theta = \{g \in \Lambda \mid (f,g)_\Theta = 0 
\mbox{ for all } f \in I\}$. Since the pairing $(-,-)_\Theta$ is nodegenerate 
the space $\mathcal H_\theta$ is naturally isomorphic to the graded dual 
$(\Lambda/I)^\circ$. We denote by $P_\sigma^\Sigma$ the unique element in 
$\mathcal H_{I_W}$ such that $(P_\sigma^\Sigma, \SS_{\tau}^\Sigma) = 
\delta_{\sigma, \tau}$ for all $\sigma, \tau \in W$; as before we usually 
write $P_\sigma$ instead of $P_\sigma^\Sigma$. It follows that the set 
$\{P_\sigma \mid \sigma \in W\}$ is a graded basis of $\mathcal H_{I_W}$, dual 
to the Demazure basis of $\Lambda/I_W$. Also for each $\theta \subset S$ the 
set $\{P_\sigma \mid \sigma \in W^\theta\}$ is a graded basis of the dual of 
$(\Lambda/I_W)^{W_\theta}$.

The polynomials $P_\sigma$ were described by Postnikov and Stanley in terms of 
the Bruhat order of $W$ in \cite{PS-chains-bruhat} when $W$ is a Weyl 
group. For each covering relation $\sigma \preceq \sigma s_\alpha$ with 
$\alpha \in \Phi^+$ we set $\alpha = m(\sigma,\sigma s_\alpha) \in V^* = 
S(V)_1$, and for a saturated chain $C = (\sigma_1, \sigma_2, \ldots, 
\sigma_r)$ we denote by $m_C$ the product $\prod_{i=1}^{r-1} 
m_C(\sigma_i,\sigma_{i+1})$. Set
\begin{align*}
P_{\sigma, \tau} &= \frac{1}{(\ell(\tau) - \ell(\sigma))!}\sum_C m_C
\end{align*}
where the sum is taken over all saturated chains from $\sigma$ to $\tau$. 
Now according to \cite{PS-chains-bruhat}*{Corollary 6.9} if $\sigma \leq \tau$ 
in the Bruhat order then $P_{\sigma,\tau} = \sum_{\rho \in W} 
c^{\tau}_{\sigma,\rho} P_{\rho}$. This inspires the following definition.
\begin{Definition}
For each $\sigma, \tau \in W$ with $\sigma \leq \tau$ in the Bruhat order of 
$W$ we set $\D{\Sigma}{\sigma}{} = \Theta(P_\sigma^\Sigma)$ and 
$\D{\Sigma}{\tau,\sigma}{} = \sum_{\rho \in W} c^\tau_{\sigma, \rho} 
\D{\Sigma}{\rho}{}$. Whenever $\Sigma$ is clear from the context we will ommit 
the superscript $\Sigma$.
\end{Definition}

\paragraph
\label{P:leibniz-rule}
Notice that although by defintion $\D{\Sigma}{\tau, \sigma}{}$ is a 
differential operator on $\Lambda$ it has a well defined extension to the 
fraction field $L$, which we will denote by the same symbol. We denote by 
$\D{}{\sigma}{0}$ and $\D{}{\tau, \sigma}{0}$ the linear functional of 
$\Lambda$ obtained by applying the corresponding differential operator followed
by evaluation at $0$. It follows from the definition that $\D{}{\sigma}{0}
(\gamma f) = \gamma(0) \D{}{\sigma}{0}(f)$ for all $f \in \Lambda$ and 
$\gamma \in \Gamma$. The following proposition shows that this functional 
extends to the algebra of rational functions without poles at $0$ and gives a 
generalized Leibniz rule to compute the result of applying this operator to 
the product of two such functions. 

\begin{Proposition}
Let $f \in L$ be regular at zero and let $\sigma \in W$. Then $f_{(\sigma)}$ 
is also regular at $0$ and $f_{(\sigma)}(0) = \D{}{\sigma}{}(f)(0) = 
(\partial_{\sigma} f)(0)$. Furthermore if $g \in L$ is also regular at $0$
then
\begin{align*}
\D{}{\sigma}{0}(fg)
	&= \sum_{\rho \leq \sigma} \D{}{\rho, \sigma}{0}(f) \D{}{\rho}{0}(g)
	= \sum_{\rho \leq \sigma} \D{}{\rho}{0}(f) \D{}{\rho, \sigma}{0}(g)
\end{align*}
\end{Proposition}
\begin{proof}
Let $T \subset \Gamma$ be the set of $W$-invariant rational functions with 
nonzero constant term. This is clearly a $W$-invariant set and hence 
$T^{-1} \Gamma$ is a subalgebra of $K = \Frac(\Gamma)$. Denoting by $A$ the 
subalgebra of $L$ consisting of rational functions regular at $0$, the product 
map $T^{-1} \Gamma \ot \Lambda \to A$ is an isomorphism, since any fraction 
$p/q \in L$ with $p,q \in \Lambda$ can be rewritten so that $q \in \Gamma$. 
Thus $A$ is a free $T^{-1}\Gamma$-module with basis $\{\SS_\sigma \mid \sigma 
\in W\}$ and $f_{(\sigma)} \in A$ for all $\sigma \in W$. 

As noted in the preamble for each $\gamma \in \Gamma$ we have 
$\D{}{\sigma}{0}(\gamma f) = \gamma(0) \D{}{\sigma}{0}(f)$, and it follows 
that the same holds if $\gamma \in A^G$. Thus
\begin{align*}
\D{}{\sigma}{}(f)(0)
	&= \sum_{\tau} f_{(\tau)}(0) \D{}{\sigma}{}(\SS_\tau)(0)
	= f_{(\sigma)}(0)
\end{align*}
as stated. Analogously $\partial_\sigma$ is a $K$-linear operator, and hence
\begin{align*}
(\partial_{\sigma} f)(0)
	&= \sum_{\tau} f_{(\tau)}(0) (\partial_{\sigma} 
		\partial_{\tau^{-1}\omega_0} \Delta(\Phi))(0)
\end{align*}
Now $\partial_{\sigma} \partial_{\tau^{-1}\omega_0} \Delta(\Phi)$ is 
zero unless $\ell(\tau \sigma) = \ell(\tau) + \ell(\sigma)$, in which case this
equals $\SS_{\tau\sigma^{-1}}$. Evaluation of this polynomial at $0$ is zero 
unless $\tau = \sigma$, in which case it evaluates to one, so 
$(\partial_{\sigma} f)(0) = f_{(\sigma)}(0) = \D{}{\sigma}{0}(f)$.

Finally
\begin{align*}
\D{}{\sigma}{0}(fg)
	&= \sum_{\tau, \rho} f_{(\tau)}(0) g_{(\rho)}(0) 
		\D{}{\sigma}{}(\SS_\tau \SS_\rho)(0) 
	= \sum_{\tau, \rho} c_{\tau, \rho}^\sigma \D{}{\tau}{0}(f) 
	\D{}{\rho}{0}(g)
\end{align*}
and the formulas in the last item follow from the fact that 
$c^\sigma_{\tau, \rho} = c^\sigma_{\rho, \tau}$.
\end{proof}

%\newpage

\section{The fundamental $\Gamma$-modules}
Throughout this section we fix a complex vector space $V$, and a root system
$\Phi$. We will also fix a root subsytem $\Psi \subset \Phi$ with basis 
$\Omega \subset \Psi$. We will denote by $G$ the Weyl group associated to 
$\Phi$ and by $W$ the one associated to $\Psi$. As above, we will denote by
$\Lambda$ the symmetric algebra of $V$ and by $L$ its fraction field, and set
$\Gamma = \Lambda^G$ and $K = K^G$. Since $W \subset G$ this group also acts
on all these spaces. All Schubert polynomials, Stanley-Postnikov operators,
normal elements, etc. are defined with respect to this fixed subsystem 
$\Psi$ and the basis $\Omega$ unless explicitly stated.

\begin{Lemma}
\label{L:translation}
Let $v \in V$ and let $\pi^W: \Lambda \to \Lambda/I_W$ be the natural 
projection. Then $\pi^W(t_v(\Gamma)) = (\Lambda/I_W)^{W_v}$.
\end{Lemma}
\begin{proof}
Recall that $K$ is the fixed field of $G$, and hence the fraction field of 
$\Gamma$. Since the extension $L^W \subset L$ is a Galois extension with 
Galois group $W$, the field $L^W t_v(K) \subset L$ must be the fixed field
of a subgroup $\tilde W \subset W$. If $\sigma \in W_v$ and $f \in K$ then
$\sigma \cdot t_v(f) = t_{\sigma(v)} (\sigma \cdot f) = t_v(f)$, so $W_v 
\subset \tilde W$. On the other hand if $\sigma \in \tilde W$ then $t_v(f)
= t_{\sigma(v)}(f)$ so $t_{\sigma(v) - v}(f) = f$ for all $f \in K$ and this
implies that $\sigma(v) = v$ so $\sigma \in W_v$. Thus $L^W t_v(K) = L^{W_v}$
which implies that $\Lambda^{W} t_v(\Gamma) = \Lambda^{W_v}$. Since all non
constant polynomials in $\Lambda^W$ are in the kernel of $\pi^W$ we see that
$\pi^W(\Lambda^W t_v(\Gamma)) = \pi^W(t_v(\Gamma))$, so this last space equals
$\pi^W(\Lambda^{W_v}) = (\Lambda/I_W)^{W_v}$.
\end{proof}

\paragraph
Let $\D{}{}{}(\Omega, v) = \vectspan{\D{\Omega}{\sigma}{v} \mid \sigma \in W}
\subset L^*$, where the star denotes the dual over the complex numbers,
and let $L(v)$ be the set of all rational functions regular at $v$. From this
point on we ommit the superscript $\Omega$.
The generalized Leibniz rule from Proposition \ref{P:leibniz-rule} implies
that $\D{}{}{}(\Omega, v)$ is a $\Lambda$-submodule of $L^*$, since for each 
$f \in L(v)$ and $g \in L$ we have
\begin{align*}
	(f \cdot \D{}{\sigma}{v})(g)
		&= \D{}{\sigma}{0} (t_v(f)t_v(g))
		= \sum_{\tau \leq \sigma} 
			\D{}{\tau,\sigma}{0}(t_v(f)) \D{}{\tau}{0}(t_v(g)) \\
		&= \sum_{\tau \leq \sigma} \D{}{\tau,\sigma}{v}(f) \D{}{\tau}{v}(g).
\end{align*} 
Now let $\DD{}{\sigma}{v}$ denote the restriction of $\D{}{\sigma}{v}$ to 
$\Gamma$ and let $\DD{}{}{}(\Omega, v) = \vectspan{\DD{}{\sigma}{v} \mid 
\sigma \in W} \subset \Gamma^*$. The same computation shows that this is a 
$\Gamma$-submodule of $\Gamma^*$. We record this result in the following 
theorem.
\begin{Theorem}
\label{T:gamma-module}
Let $v \in V$. The space $\mathcal D(\Omega, v)$ is a $\Gamma$ submodule of 
$\Gamma^*$ and for each $\gamma \in \Gamma$
\begin{align*}
\gamma \cdot \DD{}{\sigma}{v}
	&= \gamma(v) \DD{}{\sigma}{v} + 
		\sum_{\tau < \sigma} \D{}{\sigma, \tau}{v}(\gamma) 
			\DD{}{\tau}{v}.
\end{align*}
\end{Theorem}

\paragraph 
\about{The sutrcture of $\DD{}{}{}(\Omega, v)$ as $\Gamma$-module}
The modules $\DD{}{}{}(\Omega, v)$ will play an important role in our study
of the character modules of a Rational Galois Algebra. We thank David Speyer
for pointing out the following technical result in response to the 
MathOverflow question \cite{MathOver}, which greatly simplified
our presentation.
\begin{Lemma}
\label{L:speyer-lemma}
Let $v \in V$ be in normal form, let $A = (\Lambda / I_W)^{W_v}$ and let
$\omega_0^v \in W^v$ be the representative of the coset $\omega_0 W_v$.
The bilinear form $(a,b) \in A \times A \mapsto \D{}{\omega_0^v}{v}(ab) 
\in \CC$ is non-degenerate.
\end{Lemma}
\begin{proof}
Bt the Chevalley-Shephard-Todd theorem $\Lambda^{W_v}$ and $\Lambda^W$ are 
polynomial algebras, generated by algebraically independent sets $p_1, \ldots,
p_r$ and $q_1, \ldots, q_s$ respectively. Clearly $p_i \in \Lambda^{W_v}$ and 
$A = \CC[q_1, \ldots, q_s]/ J$, where $J$ is the ideal generated by the 
$p_i$'s. This implies that $A$ is a finite dimensional complete intersection,
and hence a graded Artinian self injective ring. 

Set $r = \ell(\omega_0^v)$. Then $A_n = 0$ for $n > r$, while $A_r$ is spanned
over $\CC$ by $\SS_{\omega_0^v}$ and the bilinear form in the statement is 
given by taking the coefficient of $\SS_{\omega_0^v}$ in the product $fg$.
By \cite{Lam-modules-book}*{(16.22) and (16.55)} $A$ is a symmetric algebra 
and there exists a nonsingular associative bilinear pairing $B: A \times A 
\to \CC$; associative in this case means that $B(a,bc) = B(ab,c)$ for every 
$a,b,c \in A$. We will prove that we can choose $B$ so that $B(a,b) = 
\D{}{\omega_0^v}{ab}$. Since $B$ is non-degenerate there exists $a' \in A$ 
such that $B(a',\SS_{\omega^v_0}) = 1$, and if $a'$ were of positive degree 
then $B(a',\SS_{\omega_0^v}) = B(1, a' \SS_{\omega^v_0}) = 0$ so $a' \in \CC$. 
Without loss of generality we may assume that $a' = 1$, which implies that 
$B(f,g) = B(1,fg) = \D{}{\omega_0^v}{}(fg)$.
\end{proof}

\begin{Proposition}
\label{P:gamma-module-structure}
Suppose $v \in V$ is $\Omega$-standard and let $W^v$ be the set of canonical 
representatives of left $W_v$-cosets.
\begin{enumerate}[(a)]
\item 
\label{i:gamma-basis}
The set $\{\DD{}{\sigma}{v} \mid \sigma \in W^v\}$ is a basis of 
$\mathcal D(\Omega, v)$, and $\DD{}{\sigma}{v} = 0$ for all $\sigma \notin 
W^v$.

\item 
\label{i:gamma-cyclic}
$\mathcal D(\Omega, v) = \Gamma \cdot \DD{}{\omega_0^v}{v}$

\item 
\label{i:gamma-comparision}
Let $v' \in V$. The space $\mathcal D(\Omega, v) \cap \mathcal 
D(\Omega, v')$ is not zero if and only if $v'$ is in the $G$-orbit of $v$. 
Furthermore if $v'$ is in the $W$-orbit of $v$ then they are equal.
\end{enumerate}
\end{Proposition}
\begin{proof}
Since $\DD{}{\sigma}{v}$ is a differential operator we have $\DD{}{\sigma}{v}
= \D{}{\sigma}{0} \circ t_v |_\Gamma$, so for every $\gamma \in \Gamma$
$\D{}{\sigma}{v}(\gamma) = \D{}{\sigma}{0}(t_v(\gamma))$. By construction of
the operators $\D{}{\sigma}{0}$ this value depends only on the image of 
$t_v(\gamma)$ modulo the ideal $I_W$, and as we saw in Lemma 
\ref{L:translation} $\pi^W(t_v(\Gamma))$ is exactly the space of $W_v$ 
invariants of $\Lambda/I_W$. The set of Schubert polynomials $\{\SS_\sigma 
\mid \sigma \in W^v\}$ forms a basis of this space so for each $\sigma \in W^v$
there exists $\gamma_\sigma \in \Gamma$ such that $t_v(\gamma_\sigma) \equiv 
\SS_\sigma \mod I_W$ and these elements span $\pi^W(t_v(\Gamma))$. Thus
$\DD{}{\sigma}{v}(\gamma_\tau) = \delta_{\sigma,\tau}$ for all $\sigma \in W$
and $\tau \in W^v$, and this implies item \ref{i:gamma-basis}.

By Lemma \ref{L:speyer-lemma} there also exist for each $\sigma \in W^v$ 
polynomials $\gamma_\sigma^*$ such that $\DD{}{\omega_v^0}{v}(\gamma_\sigma^*
\gamma_\tau) = \delta_{\sigma,\tau}$ for all $\tau \in W^v$. This implies that
$\gamma_\sigma^* \cdot \DD{}{\omega_0^v}{v} = \DD{}{\sigma}{v}$ and hence 
$\DD{}{}{} (\Omega, v)$ is a cyclic $\Gamma$-module generated by 
$\DD{}{\omega_0^v}{v}$.

It follows from the explicit formulas for the action of $\gamma \in \Gamma$ 
that each element in $\DD{}{}{}(\Omega, v)$ is a generalized eigenvector of 
$\gamma$ with eigenvalue $\gamma(v)$. Thus if $\DD{}{}{}(\Omega, v) \cap 
\DD{}{}{}(\Omega, v') \neq 0$ we must have $\gamma(v) = \gamma(v')$ for all
$\gamma \in \Gamma$ which implies that $v' \in G \cdot v$. Now if $v' = 
\tau(v)$ for some $\tau \in W$ then $\DD{}{\sigma}{\tau(v)} = \D{}{\sigma}{0}
\circ t_{\tau(v)}|\Gamma = \D{}{\sigma}{0} \circ \tau \circ t_v \circ \tau^{-1}
|_\Gamma = \D{}{\sigma}{0} \circ \tau \circ t_v$. Since $\D{}{\sigma}{0} 
\circ \tau$ lies in $\mathcal H_W$ there exist $c_\rho \in \CC$ for each $\rho
\in W$ such that $\D{}{\sigma}{0} \circ \tau = \sum_\rho c_\rho \D{}{\rho}{0}$,
and from this we obtain $\DD{}{\sigma}{\tau(v)} = \sum_\rho 
c_\rho \DD{}{\rho}{v}$, which proves item \ref{i:gamma-comparision}.
\end{proof}

\paragraph
\about{Jordan blocks}
Let $v \in V$ be standard, and for each $\gamma \in \Gamma$ let us denote
by $[\gamma]$ the matrix of the endomorphism of $\DD{}{}{}(\Omega, v)$ induced 
by $\gamma$ in the basis described in item \ref{i:gamma-basis} of 
\ref{P:gamma-module-structure}, which we order by decreasing length. By 
Theorem \ref{T:gamma-module} $[\gamma]$ is a lower triangular matrix and every
entry in the diagonal equals $\gamma(v)$, so its Jordan form consists of Jordan
blocks with this eigenvalue. We will give some extra information on the Jordan
form of $[\gamma]$ for generic elements of $\Gamma$, and for this we need the 
following lemma.

\begin{Lemma}
\label{L:differential-power}
For each $\sigma \in W$ and each $f \in \Lambda_1$ we have
$\D{}{\sigma}{0}(f^{\ell(\sigma)}) = \sum_{C(\sigma)} 
\prod_{i=1}^{\ell(\sigma)} \D{}{s_i}{0}(f)$, where $C(\sigma)$ is the set of 
reduced expressions $\sigma = s_1s_2 \cdots s_{\ell(\sigma)}$.
\end{Lemma}
\begin{proof}
We will prove the statement by induction on $r = \ell(\sigma)$, with the
case $r = 0$ true since $f(0) = 0$. Now writing $f^r = f f^{r-1}$ and using
Proposition \ref{P:leibniz-rule} and the fact that $\D{}{\tau}{0}(f) = 0$
if $\ell(\tau) \neq 1$ we obtain
\begin{align*}
	\D{}{\sigma}{0}(ff^{r-1}) 
		&= \sum_{\ell(\tau) = \ell(\sigma) - 1} 
			\D{}{\tau}{0}(f^{r-1})\D{}{\tau,\sigma}{0}(f) \\
		&= \sum_{\ell(\tau) = \ell(\sigma) - 1} 
				\left(\sum_{C(\tau)}\prod_{i=1}^{\ell(\tau)} 
					\D{}{s_i}{0}(f^{r-1}) \right)
				\D{}{\tau,\sigma}{0}(f)
\end{align*} 
Now $\D{}{\tau,\sigma}{0} = \D{}{s}{0}$ if $\sigma = \tau s$, and otherwise
it equals $0$, and the result follows.
\end{proof}

We now summarize some general information on the Jordan form of the 
endomorphisms of $\DD{}{}{}(\Omega, v)$ which define its $\Gamma$-module 
structure.
\begin{Theorem}\label{thm-Jordan}
Let $v \in V$ be standard and let $\gamma \in \Gamma$. Then the Jordan form
of the matrix $[\gamma]$ consists of Jordan blocks of size at most 
$\ell(\omega_0^v)+1$ and eigenvalue $\gamma(v)$. Furthermore there is at most
one block of this maximal size, and for a generic element of $\Gamma$ there 
is exactly one.  
\end{Theorem}
\begin{proof}
Set $r = \ell(\omega_0^v)$.
The formula for the action of $\Gamma$ given in Theorem \ref{T:gamma-module}
implies that $(\gamma - \gamma(v)) \DD{}{\sigma}{v}$ is a linear combination of
$\D{}{\tau}{v}$ with $\ell(\tau) < \ell(\sigma)$. It follows that 
\begin{align*}
(\gamma - \gamma(v))^{\ell(\sigma) + 1} \DD{}{\sigma}{v} &= 0
\end{align*}
so $\gamma(v)$ is the only possible eigenvalue of $\gamma$ acting on this 
space, and $\ker (\gamma - \gamma(v))^{r}$ is contained 
in the linear span of $\DD{}{\omega_0^v}{v}$. This proves that all Jordan
blocks are of size at most $r+1$, and that there is at most one
block of this size. We will now show that generically the Jordan form of 
$[\gamma]$ has one such block.

Denote by $N$ the subset of $\Gamma / \ann \DD{}{}{}(\Omega, v)$ consisting of
the coclasses of those $\gamma \in \Gamma$ whose Jordan form contains only 
blocks of size strictly smaller than $r + 1$. Equivalently, 
this is the set of coclasses of $\gamma$ such that $(\gamma - 
\gamma(v))^{r} \DD{}{}{}(\Omega, v) = 0$, which is a closed 
Zariski subset of $\Gamma / \ann \DD{}{}{}(\Omega, v)$. Now let $S_v \subset 
W_v$ be the set of all simple transpositions in $W_v$, let $\SS = 
\sum_{s \in S_v} \SS_s \in (\Lambda/I_W)^{W_v}$ and let $\gamma \in \Gamma$
be such that $\pi \circ t_v(\gamma) = \SS$, which must exist by Lemma 
\ref{L:translation}. Then $\gamma(v) = 0$ and
\begin{align*}
\gamma^{r} \cdot \DD{}{\omega_0^v}{}(v)
	&= \DD{}{e,\omega_0^v}{v}(\gamma^{r}) \DD{}{e}{v}
	= \D{}{\omega_0^v}{0}(\SS^{r}) \DD{}{e}{v}.
\end{align*}
Now by Lemma \ref{L:differential-power} 
\begin{align*}
\D{}{\omega_0^v}{0}(\SS^{r})
	&= \sum_{C} \prod_{i=1}^{r} \D{}{s_i}{0}(\SS)
	= \sum_{C} \prod_{i=1}^{r} \mathbb I_{S_v}(s_i)
\end{align*}
where the sum is over all reduced decompositions $s_1 \cdots s_{r}$ of 
$\omega_0^v$ and $\mathbb I_{S_v}$ is the indicator function of the set $S_v$,
which is $1$ over $S_v$ and $0$ in its complement. Thus the product 
$\prod_{i=1}^{r} \mathcal I_{S_v}(s_i)$ is zero unless each $s_i$ in the 
reduced decomposition lies in $S_v$. In view of 
\cite{Hump-coxeter-book}*{1.10 Proposition, item (b)} there is at least one
such reduced decomposition and hence $\D{}{\omega_0^v}{0}(\SS^{r}) \in 
\ZZ_{>0}$. This shows that $\gamma \notin N$ and hence $N$ is a closed Zariski
subset of $\Gamma/ \ann \DD{}{}{}(\Omega, v)$ and it is not the whole space, 
whence its complement is dense.
\end{proof}


%\newpage
\section{Action of a Rational Galois Algebra}
In this section $G$ is a reflection group acting on $V$, and hence on its 
symmetric algebra $\Lambda$ and its field of rational functions $L$. We fix
a Rational Galois algebra $U \subset (L*V)^G$ and denote by $\Z \subset V$ the 
additive monoid generated by $\supp U$. 

We assume that $\Phi \subset V^*$ is a root system with basis $\Sigma$ and 
$G = W(\Phi)$. We denote by $\Psi$ a standard subsystem with basis $\Omega
\subset \Sigma$ and $W = W(\Psi)$. All Schubert polynomials and 
Postnikov-Stanley differential operators appearing in this section are defined 
with respect to $\Omega$ unless explicitly stated.

\paragraph
Recall that for each $\sigma \in G$ we introduced a divded difference operator 
as an element of the smash product $L \# G$. Since $\End_\CC(L)$ is an 
$L\# G$-module given $X \in \End_\CC(L)$ and $\sigma \in G$ we obtain a new 
operator on $L$ by taking $\partial_\sigma(X)$. Notice that in general this is 
\emph{not} equal to the composition of $\partial_\sigma$ seen as an element 
of $\End_\CC(L)$ with $X$. The following lemma gathers some properties of 
these operators.
\label{L:dd-varia}
\begin{Lemma}
Let $X \in \End_\CC(X)$.
\begin{enumerate}[(a)]
\item 
\label{i:dd-on-operators}
For each $\sigma \in G$ we have $\partial_\sigma(X)|_K = \partial_\sigma 
\circ X |_K$.

\item
\label{i:diff-dd-composition}
Let $v \in V$ be $\Omega$-standard. If $\sigma \in W^v$ and $\tau \in W_v$ 
then 
\begin{align*}
\D{}{\sigma}{v} \circ 
\partial_\tau = 
\begin{cases} 
	\D{}{\sigma\tau}{v} 
		& \mbox{ if } \ell(\sigma\tau) = \ell(\sigma) +\ell(\tau);\\ 
	0 & \mbox{otherwise.}\end{cases}
\end{align*}

\item 
\label{i:symmetrizing}
Let $\tilde \Psi \subset \Psi$ be a standard subsystem, let $W_\theta \subset 
W$ be the corresponding parabolic subgroup, let $\omega_0^\theta$ be the 
longest word in $W^\theta$, and let $\Delta(\Psi)^\theta = \Delta(\Psi) / 
\Delta(\tilde \Psi)$. If $X \in \End_\CC(L)^{W_\theta}$ then 
$\sum_{\sigma \in W} \sigma \cdot X = |W_\theta| \partial_{\omega_0^\theta} 
(X \Delta(\Psi)^\theta)$.
\end{enumerate}
\end{Lemma}
\begin{proof}
We will prove item \ref{i:dd-on-operators} by induction on the length of 
$\sigma$. If $\sigma$ is the identity then there is nothing to prove, so let
us assume that $\sigma = s \tau$ with $\ell(\sigma) = 1 + \ell(\tau)$ and 
$s \in S$, and that the statement holds for $\tau$. Putting $X' = 
\partial_\tau(X)$ we get
\begin{align*}
\partial_\sigma(X) (f)
	&= \partial_s(X')(f)
	= \frac{1}{\alpha_s} (X'(f) - s\circ X' \circ s (f)) 
	= \frac{1}{\alpha_s} (X'(f) - s(X'(f)))\\
	&= \partial_s(X'(f))
	= \partial_s(\partial_\tau(X(f)))
	= \partial_\sigma(X(f))
\end{align*} 
which is what we wanted to prove.

Let us now prove item \ref{i:diff-dd-composition}. The fact that $\tau \in W_v$
implies that $t_v \circ \partial_\tau = \partial_\tau \circ t_v$. Now recall
from Proposition \ref{P:leibniz-rule} that $\D{}{\sigma}{0} = \ev_0 \circ 
\partial_\sigma$, so 
\begin{align*}
\D{}{\sigma}{v} \circ \partial_\tau 
	&= \D{}{\sigma}{0} \circ \partial_\tau \circ t_v \\
	&= \ev_0 \circ \partial_\sigma \circ \partial_\tau \circ t_v
	= \begin{cases}
		\ev_0 \circ \partial_{\sigma\tau} \circ t_v = \D{}{\sigma\tau}{v}
			& \mbox{ if } \ell(\sigma \tau) = \ell(\sigma) + \ell(\tau); \\
		0 & \mbox{otherwise}.
	\end{cases}
\end{align*}

Finally we prove item \ref{i:symmetrizing}.
The statement of \cite{Hiller-coxeter-book}*{Chapter IV (1.6)} implies that
$\partial_{\omega_0} = \frac{1}{\Delta(\Phi)} \sum_{\sigma \in G} 
(-1)^{\ell(\sigma)} \sigma$ as operators on $L$, and since the map from $L \# 
G$ to $\End_\CC(L)$ is injective this equality holds in $L \# G$. Using the 
fact that $\sigma \cdot \Delta(\Phi) = (-1)^{\ell(\sigma)} \Delta(\Phi)$ we 
deduce from this that $\sum_{\sigma \in G} \sigma \cdot X = \partial_{\omega_0}
(X \Delta(\Phi))$ for any $X \in \End_\CC(L)$. Of course an analogous equality 
holds if we replace $G$ with any subgroup and $\Phi$ with a corresponding root 
subsystem.

Let $\omega_0, \omega_1$ be the longest elements of $W$ and $W_\theta$ 
respectively. Then $\omega_0 \omega_1^{-1} \in \omega_0 W_\theta$ and its 
length is equal to $\ell(\omega_0) - \ell(\omega_1)$ which is clearly the 
smallest possible length of an element in the coset $\omega_0 G_\theta$. Thus 
$\omega_0^\theta = \omega_0 \omega_1^{-1}$ and
\begin{align*}
\sum_{\sigma \in W} \sigma \cdot X 
	&=\partial_{\omega_0}(X \Delta(\Psi))
	= \partial_{\omega^\theta_0} \partial_{\omega_1}(X \Delta(\tilde \Psi) 
	\Delta(\Psi)^\theta)
\end{align*}
Now both $\Delta(\Psi)^\theta$ and $X$ are $W_\theta$-invariant, so the last 
expression equals
\begin{align*}
\partial_{\omega^\theta_0}(
	X \Delta(\Psi)^\theta \partial_{\omega_1}(
		\Delta(\tilde \Psi)))
	= |W_\theta| \partial_{\omega^\theta_0}(X \Delta(\Psi)^\theta)
\end{align*}
and we are done.
\end{proof}

Recall that for each $z \in V$ there exists some $\Omega$-standard element in 
the orbit $W \cdot z$. Thus given $Z \subset V$ stable by the action of $W$ we 
can choose a set of $\Omega$-standard representatives of $Z/W$. The following
proposition shows how this fact can be used to find different ways to express
elements of $U$.

\begin{Proposition}
\label{P:form}
Let $X \in (L*V)^G$ and assume that there exists $\chi \in \hat G$ such that 
$d_\chi X \in \Lambda * V$. 
\begin{enumerate}[(a)]
\item 
\label{i:invariant-form}
For each $z \in \supp X$ there exists $f_z \in \Lambda^{G_z}$ such that
\begin{align*}
X
	&= \sum_{z \in \supp X} \frac{f_z}{d_{\chi}^z} t_z,
\end{align*}
where $d_{\chi}^z$ is the product of all $\alpha \in \Phi^+$ dividing $d_\chi$ 
such that $\alpha(z) \neq 0$. 

\item
\label{i:dd-form}
Let $Y$ be a set of $\Omega$-standard representatives of $\supp X / W$, and 
for each $z \in Y$ denote by $W^z$ the set of canonical coset representatives 
of $W/W_z$, by $\omega_0^z$ the longest element in $W^z$, and by 
$\Delta(\Psi)^z$ be the product of all roots in $\Psi^+$ with $\alpha(z) 
\neq 0$. Then  
\begin{align*}
X
	&= \sum_{z \in Y} \frac{1}{|W_z|}
		\partial_{\omega_0^{z}}\left(
			\frac{f_z \Delta(\Psi)^z}{d_{\chi}^z} t_z
		\right).
\end{align*}
\end{enumerate}
\begin{proof}
Fix $z \in \supp X$. Since $X$ is $G$-invariant we know that $\sigma \cdot X = 
X$ for any $\sigma \in G_z$. By item \ref{i:translation-li} of Lemma
\ref{L:translation} the coefficient of $t_z$ must be $G_z$ invariant, so 
writing this coefficient as $\frac{g}{d_\chi}$ we get
\begin{align*}
\frac{g}{d_\chi} 
	= \sigma \cdot \frac{g}{d_\chi} 
	=\frac{\sigma \cdot g}{\chi(g) d_\chi},
\end{align*}
so $\sigma \cdot g = \chi(\sigma) g$ for all $\sigma \in G_z$. 

Denote by $\chi'$ the restriction of $\chi$ to $G_z$. Now $G_z$ is the 
reflection group generated by the reflections fixing $z$ and it acts on 
$\Lambda$ by restriction, so by Stanley's theorem the space of relative 
invariants $\Lambda^{G_z}_{\chi'}$ is generated over $\Lambda^{G_z}$ by 
$d_{\chi'}$, and this polynomial is the product of all roots $\alpha \in 
\Phi^+$ dividing $d_\chi$ such that $\alpha(z) = 0$. Thus $g = f_z d_{\chi}$
for some $f_z \in \Lambda^{G_z}$, so $\frac{g}{d_{\chi}} = 
\frac{f_z}{d_\chi/d_{\chi'}} = \frac{f_z}{d_{\chi}^z}$ and item 
\ref{i:invariant-form} is proved. 

Since $X$ is $G$-invariant, it is clear that
\begin{align*}
X 
	&= \frac{1}{|W|} \sum_{\sigma \in W} \sigma \cdot X
	=\sum_{z \in Y} \frac{1}{|W|}\sum_{\sigma \in W} \sigma \cdot 
		\left( \frac{f_z}{d_\chi^z} t_z\right).
\end{align*}
As we mentioned above the coefficient of $t_z$ is $G_z$-invariant and hence it 
is $W_z$ invariant, so applying item \ref{i:symmetrizing} of Lemma 
\ref{L:dd-varia} to $W$ we obtain
\begin{align*}
\sum_{\sigma \in W} \sigma \cdot
		\left(\frac{f_z}{d_\chi^z} t_z \right)
		&= |W^z| \partial_{\omega_0^{z}}\left(
			\frac{f_z \Delta(\Psi)^z}{d_\chi^z} t_z 
		\right)
\end{align*}
and the result follows.
\end{proof}
\end{Proposition}

\paragraph
\about{A submodule associated to $v$}
To each $v \in V$ we associate a character $\ev_v: \Gamma \to \CC$ given by
evaluation at $v$. Since $\Gamma$ consists of $G$-symmetric polynomials
$\ev_v = \ev_{\sigma(v)}$ for any $\sigma \in G$, so we can assume that
$v$ is standard. Furthermore notice that $\ev_v = \DD{}{e}{v} \in \DD{}{}{}
(\Omega, v) \in \Gamma^*$, so to each $v$ we associate the module $U \cdot 
\ev_v \subset \Gamma^*$. This paragraph is devoted to the study of this module.

\begin{Definition}
Let $v \in V$ be standard. We define $\module{v}$ to be the space $\sum_{z 
\in \Z} \DD{}{}{}(\Omega, v+z)$.
\end{Definition}

Recall that $\Phi_0(v)$ is the set of all rots in $\Phi$ such that $\alpha(v)
= 0$.




\begin{Theorem}\label{thm-module-structure}
Let $v \in V$ be standard and assume that $\Phi_0(v+z) \subset \Psi$ for each 
$z \in \Z$. Then $\module{v}$ is a $U$-submodule of $\Gamma^*$.
\end{Theorem}




\begin{proof}
We need to prove the following: for each $z' \in \Z$, each $\sigma \in G$ and
each $X \in U$ we have $\DD{}{\sigma}{v+z'} \circ X \in \module{v}$. We will
prove this in several steps.

First, let $v'$ be a standard element in the $W$-orbit of $v + z'$. Since 
$\DD{}{}{}(\Omega, v') = \DD{}{}{}(\Omega, v+ z')$ by item 
\ref{i:gamma-comparision} of Proposition \ref{P:gamma-module-structure}
our objective is equivalent to showing that $\DD{}{\sigma}{v'} \circ X \in 
\module{v}$. Now let $\tilde W = W_{v'}$ and let $\tilde \Psi = \Psi_0(v')$ be 
the associated standard root subsystem. By item \ref{i:dd-form} of Proposition 
\ref{P:form} $X$ can be written as a sum of operators of the form 
$\partial_{\tilde \omega_0^z}(F_z t_z)$ for $z \in \Z$, where $\tilde 
\omega_0^z$ is the longest element of $\tilde W^z$ and $F_z = \frac{f_z 
\Delta(\tilde \Psi)^z}{d_\chi^z}$. Thus 
\begin{align*}
\DD{}{\sigma}{v'} \circ X
	&= \sum_{z \in Y} \frac{1}{|\tilde W_z|}
		\D{}{\sigma}{v'} \circ \partial_{\tilde \omega_0^{z}}(
			F_z t_z)|_\Gamma
\end{align*}
where $Y$ is a set of $\tilde \Omega$-standard representatives of $\supp X / 
\tilde W$, so it is enough to show that $\D{}{\sigma}{v'} \circ 
\partial_{\tilde \omega_0^z}(F_z t_z)|_\Gamma \in \module{v}$ for any $z \in 
Y$.

We claim that $F_z$ is regular at $v'$. Recall that $d_\chi^z$ is the product
of all roots $\alpha_s$ such that $\chi(s) = -1$ and $\alpha_s(z) \neq 0$. If
one of this factors is such that $\alpha_s(v') = 0$ then $\alpha_s \in 
\Phi_0(v') = \Phi_0(\tau(v+z')) = \tau(\Phi_0(v+z'))$ for some $\tau \in W$. 
Now since $\Phi_0(v+z') \subset \Psi$ by hypothesis, and since $\Psi$ is stable
by the action of $W$, it follows that $\Phi_0(v') \subset \Psi$, and hence 
$\alpha_s$ is also a factor of $\Delta(\tilde \Psi)^z$. Thus the term
$\Delta(\tilde \Psi)^z$ in the numerator cancels out all the linear terms in
the denominator which are zero at $v'$ and $F_z$ is regular at $v'$.

We make one further simplification. By items \ref{i:dd-on-operators} and
\ref{i:diff-dd-composition} of Lemma \ref{L:dd-varia} 
\begin{align*}
\D{}{\sigma}{v'} \circ 
	\partial_{\tilde \omega_0^z}(F_z t_z) |_\Gamma
	&= \D{}{\sigma}{0} \circ t_{v'} \circ \partial_{\tilde \omega_0^z} 
		\circ F_z t_z |_\Gamma
	= \D{}{\sigma}{0} \circ \partial_{\tilde \omega_0^z} \circ t_{v'}(F_z) 
		t_{v'+z} |_\Gamma \\
	&= \begin{cases}
		\D{}{\sigma \tilde \omega_0^z}{0} \circ t_{v'}(F_z) t_{v'+z}
			& \mbox{ if } \ell(\sigma \tilde \omega_0^z) = \ell(\sigma)
				+ \ell(\tilde \omega_0^z);\\
		0 & \mbox{otherwise.}
	\end{cases}
\end{align*}
Here we have used that $t_v$ and $\partial_{\tilde \omega_0^z}$ commute since
$\tilde \omega_0^z \in W_{v'}$. If the result is $0$ then we are done. On the 
other hand, since $F_z$ is regular at $v'$ then $t_{v'}(F_z)$ is regular at 
$0$, so writing $t_{v'}(F_z) = \sum_{\rho \in W} (t_{v'}(F_z))_{(\rho)} 
\SS_\rho$ and recalling from \ref{P:leibniz-rule} that 
$(t_{v'}(F_z))_{(\rho)}(0) = \D{}{\rho}{0}(t_{v'}(F_z)) = \D{}{\rho}{v'}(F_z)$
we obtain that
\begin{align*}
\D{}{\sigma}{v'} \circ \partial_{\tilde \omega_0^z} (F_z t_z)|_\Gamma
	&= \begin{cases}
		\sum_{\rho \in W} 
			\D{}{\rho}{v'}(F_z)
				(\D{}{\sigma \tilde \omega_0^z}{0} \circ 
					\SS_\rho t_{v'+z})|_\Gamma
			& \mbox{ if } \ell(\sigma \tilde \omega_0^z) = \ell(\sigma)
				+ \ell(\tilde \omega_0^z);\\
		0 & \mbox{otherwise.}
	\end{cases}
\end{align*}
Finally, let $\gamma \in \Gamma$. Then 
\begin{align*}
(\D{}{\sigma \tilde \omega_0^z}{0} \circ \SS_\rho t_{v'+z}) (\gamma)
	&= \D{}{\sigma \tilde \omega_0^z}{0}(\SS_\rho t_{v'+z}(\gamma)) \\
	&= \sum_{\nu \in W} t_{v'+z}(\gamma)_{(\nu)}(0)
		\D{}{\sigma \tilde \omega_0^z}{0} (\SS_{\rho} \SS_{\nu}) \\
	&= \sum_{\nu \in W}
		c^{\sigma \tilde \omega_0^z}_{\rho, \nu} \D{}{\nu}{v+z'}(\gamma).
\end{align*} 
Putting all this together gives us that
\begin{align*}
\DD{}{\sigma}{v'} \circ X 
	&= \sum_{\substack{z \in Y \\ \ell(\sigma \tilde \omega_0^z)
		= \ell(\sigma) + \ell(\tilde \omega_0^z)}} \frac{1}{|\tilde W_z|}
		\sum_{\rho, \nu \in W} 
			c^{\sigma \tilde \omega_0^z}_{\rho, \nu}
				\D{}{\rho}{v'}(F_z) \DD{}{\nu}{v'+z} \\
	&= \sum_{\substack{z \in Y \\ \ell(\sigma \tilde \omega_0^z)
		= \ell(\sigma) + \ell(\tilde \omega_0^z)}} \frac{1}{|\tilde W_z|}
		\sum_{\nu \in W} 
			\D{}{\nu, \sigma \tilde \omega_0^z}{v'}(F_z) \DD{}{\nu}{v'+z}
\end{align*}
Now $v' + z = \tau(v+z') + z = \tau(v + z' + \tau^{-1}(z))$ and hence 
$\DD{}{\nu}{v'+z} \in \module{v}$.
\end{proof}

\section{Integral Rational Galois algebras in type $A$}
Fix $r \in \NN$, let $\mu = (\mu_1, \ldots, \mu_r) \in \NN^r$ and set
$\CC^\mu = \CC^{\mu_1} \times \cdots \times \CC^{\mu_r}$. 

\paragraph
\about{Integral Rational Galois algebras}
Set $\II = \II(\mu) = \{(k,i) \mid 1 \leq k \leq r, 1 \leq i \leq \mu_k\}$. 
Given $v \in \CC^\mu$ and $(k,i) \in I$ we will denote by $v_k$ the projection 
of $v$ to the component $\CC^{\mu_k}$, and by $v_{k,i}$ the $i$-th coordinate 
of $v_k$. We will denote by $e_{k,i}$ the vector of $\CC^\mu$ with 
$(e_{k,i})_{l,j} = \delta_{k,l}\delta_{i,j}$, and refer to the set $\{e_{k,i}
\mid (k,i) \in \II\}$ as the \emph{canonical basis} of $\CC^\mu$. We denote 
by $\{x_{k,i} \mid (k,i) \in \II\}$ the dual basis to the canonical basis.
Thus $\CC[X_\mu] = \CC[x_{k,i} \mid (k,i) \in \II]$ is the algebra of 
polynomial functions over $\CC^\mu$, and we denote its fraction field by 
$\CC(X_\mu)$. For each $(k,i) \in \II$ we will denote by $t_{k,i}$ the 
automorphism $t_{e_{k,i}} \in \End_\CC(\CC(X_\mu))$.

For each $1 \leq j \leq r$ the symmetric group $S_{\mu_j}$ acts on 
$\CC^{\mu_j}$ by permuting the coordinates of a vector, and hence $S_\mu = 
S_{\mu_1} \times \cdots \times S_{\mu_r}$ acts on $\CC^\mu$. This is a 
reflection group corresponding to the root system $\Phi = \{x_{k,i} - x_{k,j} 
\mid (k,i),(k,j) \in \II\}$; we fix a basis for $\Phi$ setting $\Sigma = \{
x_{k,i} - x_{k,i+1} \mid 1 \leq k \leq r, 1 \leq i < \mu_k\}$.
Given $\sigma \in S_{\mu}$ we will denote by $\sigma[k]$ its projection to
$S_{\mu_k}$, so $\sigma = \sigma[1] \circ \sigma[2] \circ \cdots 
\sigma[r]$. Also, given $\tau \in S_{\mu_k}$ we will denote by $\tau^{(k)}$
the unique element of $S_\mu$ such that $\tau^{(k)}[k] = \tau$ and 
$\tau^{(k)}[l] = \Id_{S_{\mu_l}}$ for $l \neq k$. We denote by $\sg[k]: S_\mu
\to \CC$ the character given by $\sg[k](\sigma) = \sg(\sigma[k])$, and set
$\sym_k = \frac{1}{\mu_k!}\sum_{\sigma \in S_{\mu_k}} \sigma^{(k)} \in 
\CC[S_\mu]$.

The action of $S_\mu$ on $\CC^\mu$ induces actions on $\CC[X_\mu]$ and 
$\CC(X_\mu)$, so we may consider Galois orders in 
$(\CC(X_\mu) * \CC^\mu)^{S_\mu}$. 

\begin{Definition}
An \emph{integral Rational Galois order of type $A$} is a Galois order in 
$(\CC(X_\mu) * \CC^\mu)^{S_\mu}$ generated by $\CC[X_\mu]^{S_\mu}$ and a 
finite set $\mathcal X$ with the following property: for each $X \in \mathcal 
X$ there exists $1 \leq k \leq r$ such that $d_{\sg_k} X \in \CC[X_\mu] * 
\CC^\mu$ and $\supp X = \{e_{k,i} \mid 1 \leq i
\leq \mu_k\}$.
\end{Definition}
For the rest of this section $U$ denotes an integral Rational Galois order 
of type $A$. By item \ref{i:invariant-form} of Proposition \ref{P:form} every 
generator $X$ of $U$ is of the form
\begin{align*}
\sym_k \left( \frac{f_X}{\prod_{j=2}^{\mu_k} (x_{k,1} - x_{k,j})}
	t_{k,1}\right)
\end{align*}
where $f_X$ is a polynomial which is symmetric in all variables except 
$x_{k,1}$. Now let $\overline \mu = (\overline \mu_1, \ldots, \overline 
\mu_r)$, where $\overline \mu_k$ is equal to $\mu_k$ if $e_{k,1} \in \supp U$, 
and is $0$ otherwise. Then $\supp U = \ZZ^{\overline \mu} = 
\ZZ^{\overline \mu_1} \times \cdots \times \ZZ^{\overline \mu_r} \subset 
\CC^\mu$. We will set $\overline \Phi = \{x_{k,i} - x_{k,j} \mid (k,i), (k,j)
\in \II(\overline \mu)\}$.

\begin{Example*}
$W$ algebras, $p$-yangians, Maxorchuk's Orthogonal Gelfand-Tsetlin algebras.
\end{Example*}

\paragraph
\about{The module $\module{v}$}
Let $v \in \CC^\mu$, and let $\Psi(v) = \{\alpha \in \overline \Phi \mid 
\alpha(v) \in \ZZ\}$. Since $\Psi(\sigma(v)) = \sigma^{-1}(\Psi(v))$ there is 
always an element $v'$ in the $S_\mu$-orbit of $v$ such that $\Psi(v')$ is a 
standard subsystem of $\Phi$, and hence $\module{v} = \module{v'}$. 
Furthermore it follows from the definition that $\module{v} = \module{v' + z}$ 
for all $z \in \supp U = \ZZ^{\overline \mu}$, and it is easy to check that 
there exist elements $\vv \in v' + \ZZ^{\overline \mu}$ such that $\Psi(\vv) = 
\Phi_0(\vv)$.

\begin{Definition}
Let $\vv \in \CC^\mu$. We will say that $\vv$ is a \emph{seed} if it is 
$\Sigma$-standard and $\Psi(\vv) = \Phi_0(\vv)$. 
\end{Definition}
By definition $\vv$ is a seed if and only if the following holds: whenever
$\vv_{k,i} - \vv_{k,j} \in \ZZ$ for $1 \leq i < j \leq \overline \mu_k$ then
$\vv_{k,i} = \vv_{k,i+1} = \cdots = \vv_{k,j}$. As we have seen given any $v 
\in \CC^\mu$ there is always some seed $\vv$ such that $\module{v} = 
\module{\vv}$.

\begin{Proposition}
Let $\vv \in \CC^\mu$ be a seed, let $\Psi = \Psi(\vv)$, let $\Omega = \Psi
\cap \Sigma$. Let $\Z(\vv) = \{z \in \ZZ^{\overline \mu} \mid \alpha(z) \geq 0 
\mbox{ for all } \alpha \in \Omega\}$. Then
\begin{align*}
\module{\vv} = \bigoplus_{z \in \Z(\vv)} \DD{}{}{}(\Omega, \vv + z).
\end{align*}
In particular the set $\{\DD{}{\sigma}{\vv + z} \mid \sigma \in W(\Psi)^z\}$
is a basis of $\module{\vv}$.
\end{Proposition}
\begin{proof}
Let $z \in \ZZ^{\overline \mu}$. By \cite{Hump-coxeter-book}*{1.12 Theorem} 
there exists $\sigma \in W(\Psi)$ such that $\sigma(z) \in \Z(\vv)$. Since
$W(\Psi)$ is the stabilizer of $\vv$ it follows that $\DD{}{}{}(\Omega, \vv + 
z) = \DD{}{}{}(\Omega, \vv + \sigma(z))$ and hence $\module{\vv} = \sum_{z \in
\Z(\vv)} \DD{}{}{}(\Omega, \vv + z)$. 

We now prove that the sum is direct. Notice that the space $\DD{}{}{}(\Omega, 
\vv + z)$ consists of eigenvectors of $\Gamma = \CC[X_\mu]^{S_\mu}$ with 
eigenvalue $\ev_{\vv + z}$. Now if there exist $z, z' \in \Z(v)$ such that 
$\gamma(\vv + z) = \gamma(\vv + z')$ for all $\gamma \in \Gamma$ then $\vv + z 
= \sigma(\vv + z')$ for some $\sigma \in S_\mu$, or equivalently $\vv - 
\sigma(\vv) = \sigma(z') - z$. If $\sigma \notin W(\Psi)$ then there exist
$(k,i), (k,j) \in \II(\overline \mu)$ such that  $\vv_{k,i} \neq \vv_{k,j}$
and $\sigma(\vv)_{k,i} = \vv_{k,j}$. This implies that $\vv_{k,i} - \vv_{k,j}
= \sigma(z')_{k,j} - z_{k,j} \in \ZZ$, which contradicts the fact that $\vv$
is a seed. Thus $\sigma \in W(\Psi)$ and $z = z'$ and each space $\DD{}{}{}
(\Omega, \vv +z)$ consists of $\Gamma$-eigenvectors of different eigenvalue
and they are in direct sum. The basis statement now follows from item 
\ref{i:gamma-basis} of Proposition \ref{P:gamma-module-structure}.
\end{proof}

\paragraph
\about{A simplicity criterion.}
The following criterion for simplicity of a $U$-module, and its proof, are
completely analogous to \cite{EMV-orthogonal}*{Theorem 11}.

\begin{Theorem}\label{thm-simplicity}
Suppose $U$ is an integral Rational Galois algebra of type $A$ generated by
$\CC[X_\mu]^{S_\mu}$ and a finite set of elements $\mathcal X \subset 
(\CC(X_\mu) * \CC^\mu)^{S_\mu}$. Write each $X \in \mathcal X$ as
\begin{align*}
X 
	&=\sym_k \left( \frac{f_X}{\prod_{j=2}^{\mu_k} (x_{k,1} - x_{k,j})}
		t_{k,1}\right).
\end{align*}
If $\vv \in \CC^\mu$ is a seed such that $f_X(\sigma(\vv + z)) \neq 0$ for
all $X \in \mathcal X$ and all $z \in \Z(\vv)$ then $\module{\vv}$ is simple.
\end{Theorem}
% \begin{proof}
% We will prove that $\module{\vv} = U \cdot a$ for any $a \in \module{\vv}$.
% Since $\module{\vv} = \bigoplus_{z \in \Z(\vv)} \DD{}{}{}(\Omega, \vv+z)$ and
% each space consists of eigenvector of $\Gamma = \CC[X_\mu]^{S_\mu}$ with 
% different eigenvalue, we can assume that $a \in \DD{}{}{}(\Omega, \vv +z)$ for
% some $z \in \Z(\vv)$. Hence there exists a minimal $r \in \NN$ such that 
% $(\gamma - \gamma(\vv + z))^{r+1} a = 0$, which implies that 
% \end{proof}

\begin{bibdiv}
\begin{biblist}

\bib{ba} {article}{
author={ Bavula, Vladimir} 
title={Generalized Weyl algebras and their
representations}
journal=  {Algebra i Analiz} 
volume={4} 
date={1992} 
pages={75-92},
}


\bib{bavo}{article}{
author={ Bavula, Vladimir}
author={ Oystaeyen, Freddy} 
title={Simple Modules of the Witten-Woronowicz algebra}
journal={Journal of Algebra}
volume={ 271}
date={ 2004}
pages={827-845},
}




\bib{BGG-cohomology}{article}{
  author={Bern\v ste\u \i n, I. N.},
  author={Gel\cprime fand, I. M.},
  author={Gel\cprime fand, S. I.},
  title={Schubert cells, and the cohomology of the spaces $G/P$},
  language={Russian},
  journal={Uspehi Mat. Nauk},
  volume={28},
  date={1973},
  number={3(171)},
  pages={3--26},
}

\bib{Dem-schubert}{article}{
  author={Demazure, Michel},
  title={D\'esingularisation des vari\'et\'es de Schubert g\'en\'eralis\'ees},
  language={French},
  note={Collection of articles dedicated to Henri Cartan on the occasion of his 70th birthday, I},
  journal={Ann. Sci. \'Ecole Norm. Sup. (4)},
  volume={7},
  date={1974},
  pages={53--88},
}


\bib{dfo:gz}{article}{
author={Drozd, Yury}
author={Futorny, Vyacheslav} 
author={Ovsienko, Serge},
title={On Gelfand--Zetlin modules}, 
journal={Suppl. Rend. Circ. Mat.
Palermo}
volume= {26} 
date={1991} 
pages={143-147},

}





\bib{EMV-orthogonal}{article}{
  author={Early, Nick},
  author={Mazorchuk, Volodymir},
  author={Vyshniakova, Elizabetha},
  title={Canonical Gelfand-Zeitlin modules over orthogonal Gelfand-Zeitlin algebras},
  note={preprint, available online at \url {https://arxiv.org/abs/1709.01553}},
}




\bib{FGR1} {article}{
author={Futorny, Vyacheslav}
author={ Grantcharov, Dimitar}
author={ Ramirez, Luis Enrique}  
title={Irreducible generic Gelfand-Tsetlin modules of $\mathfrak{gl}(n)$}
journal={ Symmetry, Integrability and Geometry: Methods and Applications}
volume={018} 
date={2015}
}


\bib{FGR2}  {article}{
author={Futorny, Vyacheslav}
author={ Grantcharov, Dimitar}
author={ Ramirez, Luis Enrique} 
title={Singular Gelfand-Tsetlin modules for $\mathfrak{gl}(n)$}
journal={Adv. Math.}
volume={ 290} 
date={2016}
pages={ 453--482.}
}

\bib{FGR3}{article}{
author={Futorny, Vyacheslav}
author={ Grantcharov, Dimitar}
author={ Ramirez, Luis Enrique} 
title={New Singular Gelfand-Tsetlin modules of $gl(n)$ of index $2$}
journal={Communications in Mathematical Physics}
note={ to appear,
  arxiv:1606.03394.}
}


\bib{FGR4}{article}{
author={Futorny, Vyacheslav}
author={ Grantcharov, Dimitar}
author={ Ramirez, Luis Enrique}
title={Drinfeld category and  the classification  of singular Gelfand-Tsetlin $ \mathfrak{gl}_n$-modules}
journal={IMRN}
note={to appear; arXiv:1704.01209.}

}

\bib{FMO}{article}{
author={Futorny, Vyacheslav}
author={Molev, Alexander}
author={Ovsienko, Serge}
title={Gelfand-Kirillov conjecture and Gelfand-Tsetlin modules for finite $W$-algebras}
journal={ Adv. Math.}
volume={ 223}
date={ 2010}  
pages={773-796},
}




\bib{FO1-galois-orders}{article}{
  author={Futorny, Vyacheslav},
  author={Ovsienko, Serge},
  title={Galois orders in skew monoid rings},
  journal={J. Algebra},
  volume={324},
  date={2010},
  number={4},
  pages={598--630},
}

\bib{FO2}{article}{
author={Futorny, Vyacheslav}
author={Ovsienko, Serge} 
title={Fibers of characters in Gelfand-Tsetlin categories}
journal={ Trans. Ameri. Math. Soc.}
volume={366}
number= {8} 
date={2014}
pages={ 4173-4208.}
}









\bib{FRZ1}{article}{
author={Futorny, Vyacheslav}
author={Ramirez, Luis Enrique}
author={ Zhang, Jian}
title={Combinatorial construction of Gelfand-Tsetlin modules for $\gl_n$}
note={arXiv:1611.07908.}
}

\bib{FRZ2}{article}{
author={Futorny, Vyacheslav}
author={Ramirez, Luis Enrique}
author={Zhang, Jian}
title={Gelfand-Tsetlin modules of quantum $gl_n$ defined by admissible sets of relations }
journal={Journal of Algebra}
note={10.1016/j.jalgebra.2017.12.006}
}

\bib{Hart-rational-galois}{article}{
  author={Hartwig, Johnas},
  title={Principal Galois Orders and Gelfand-Zeitlin modules},
  note={preprint, available online at \url {http://arxiv.org/abs/1710.04186v1}},
}

\bib{Hiller-coxeter-book}{book}{
  author={Hiller, Howard},
  title={Geometry of Coxeter groups},
  series={Research Notes in Mathematics},
  volume={54},
  publisher={Pitman (Advanced Publishing Program), Boston, Mass.-London},
  date={1982},
  pages={iv+213},
}

\bib{Hump-coxeter-book}{book}{
  author={Humphreys, James E.},
  title={Reflection groups and Coxeter groups},
  series={Cambridge Studies in Advanced Mathematics},
  volume={29},
  publisher={Cambridge University Press, Cambridge},
  date={1990},
  pages={xii+204},
}

\bib{Lam-modules-book}{book}{
  author={Lam, T. Y.},
  title={Lectures on modules and rings},
  series={Graduate Texts in Mathematics},
  volume={189},
  publisher={Springer-Verlag, New York},
  date={1999},
  pages={xxiv+557},
}


\bib{M}{article}{
author={ Mazorchuk Volodymyr}
 title={Orthogonal Gelfand-Zetlin algebras, I}
 journal={ Beitr. Algebra Geom.}
 volume={ 40}
 number= { 2} 
 date={1999}
 pages= {399-415},
 }







\bib{PS-chains-bruhat}{article}{
  author={Postnikov, Alexander},
  author={Stanley, Richard P.},
  title={Chains in the Bruhat order},
  journal={J. Algebraic Combin.},
  volume={29},
  date={2009},
  number={2},
  pages={133--174},
}


\bib{MathOver}{misc}{ 
    title={A duality result for Coxeter groups},    
    author={Speyer, David},
    note={Answer available online at
    \href{https://mathoverflow.net/q/286744}
    	{https://mathoverflow.net/q/286744} 
    	(version: 2017-11-23)},
    organization={MathOverflow}  
}



\bib{O}{article}{
author={Ovsienko, Serge} 
title={Finiteness statements for
Gelfand--Tsetlin modules}, 
note={In: Algebraic structures and their
applications, Math. Inst., Kiev, 2002.}
}
  
  
\bib{RZ}{article}{ 
author={Ramirez Luis Enrique}
author={ Zadunaisky Pablo}
title={Gelfand-Tsetlin modules over $gl(n; C)$ with arbitrary characters}
note= {arXiv:1705.10731.}
}

\bib{V1}{article}{
author={Vishnyakova Elizaveta}
title={A geometric approach to $1$-singular Gelfand-Tsetlin $gl_n$-modules}, 
note={arXiv:1704.00170.}
}


\bib{V2}{article}{
author={Vishnyakova Elizaveta} 
title={Geometric approach to $p$-singular Gelfand-Tsetlin $gl_n$-modules}
note={ arXiv:1705.05793.}
}

\bib{Z}{article}{
author={Zadunaisky Pablo}
title={A new way to construct 1-singular Gelfand-Tsetlin modules}
journL={ Algebra Discrete Math.}
volume={23}
number={1}
date={2017}
pages={180-193.}
}

\end{biblist}
\end{bibdiv}

\end{document}