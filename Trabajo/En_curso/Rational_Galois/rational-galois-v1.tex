%%%%%%%%%%%%%%%%%%%%%% Generalities %%%%%%%%%%%%%%%%%%5
\documentclass[11pt,fleqn]{article}
\usepackage[paper=a4paper]
  {geometry}

\pagestyle{plain}
\pagenumbering{arabic}
%%%%%%%%%%%%%%%%%%%%%%%%%%%%%%%%
\usepackage{notas}
\usepackage{tikz}
\usepackage{mathdots}

%%%%%%%%%%%%%%%%%%%%%%%%%%% The usual stuff%%%%%%%%%%%%%%%%%%%%%%%%%
\newcommand\NN{\mathbb N}
\newcommand\CC{\mathbb C}
\newcommand\QQ{\mathbb Q}
\newcommand\RR{\mathbb R}
\newcommand\ZZ{\mathbb Z}
\renewcommand\k{\Bbbk}

\newcommand\A{\mathcal A}
\newcommand\B{\mathcal B}
\newcommand\F{\mathcal F}
\newcommand\V{\mathcal V}
\newcommand\D{\mathfrak D}
\newcommand\DD{\mathcal D}
\renewcommand\H{\mathcal H}
\renewcommand\O{\mathcal O}
%\newcommand\I{\mathcal I}
%\newcommand\J{\mathcal J}
\newcommand\Z{\mathsf Z}

\newcommand\maps{\longmapsto}
\newcommand\ot{\otimes}
\renewcommand\to{\longrightarrow}
\renewcommand\phi{\varphi}
\newcommand\Id{\mathsf{Id}}
\newcommand\im{\mathsf{im}}
\newcommand\coker{\mathsf{coker}}
%%%%%%%%%%%%%%%%%%%%%%%%% Specific notation %%%%%%%%%%%%%%%%%%%%%%%%%
\newcommand\g{\mathfrak g}
\newcommand\p{\mathfrak p}
\newcommand\m{\mathfrak m}
\newcommand\gl{\mathfrak{gl}}
\newcommand\gen{\mathsf{gen}}
\newcommand\std{\mathsf{std}}
\newcommand\sh{\mathsf{sh}}
\renewcommand\SS{\mathfrak S}
\newcommand\vv{\overline{v}}

\newcommand\vectspan[1]{\left\langle #1 \right\rangle}
\newcommand\interval[1]{\llbracket #1 \rrbracket}
\newcommand\Shuffle{\mathsf{Shuffle}}

\DeclareMathOperator\Frac{Frac}
\DeclareMathOperator\Specm{Specm}
\DeclareMathOperator\End{End}

\DeclareMathOperator\sym{sym}
\DeclareMathOperator\asym{asym}
\DeclareMathOperator\sg{sg}
\DeclareMathOperator\ev{\mathsf{ev}}
\DeclareMathOperator\ann{ann}
\DeclareMathOperator\supp{supp}


\newcommand\bigmodule{big GT module}

%%%%%%%%%%%%%%%%%%%%%%%%%%%%%%%%%%%%%% TITLES %%%%%%%%%%%%%%%%%%%%%%%%%%%%%%
\title{Rational Galois Algebras for real root systems}
%\author{[gamma-structure.tex]}
\date{}

\begin{document}
\maketitle
%\vspace{-2cm}

\section{Preelimnaries on reflections groups}
Let $V_\RR$ be a real vector space with a fixed inner product. Given $\alpha 
\in V_\RR^*$ we denote by $s_\alpha$ the orthogonal reflection through the 
hyperplane $\ker \alpha$, and by $s_\alpha^*$ the corresponding endomorphism of
$V_\RR^*$. A finite root system over $V_\RR$ is a finite set 
$\Phi \subset V_\RR^*$ such that for each $\alpha \in \Phi$ we have 
\emph{(R1)} $\Phi \cap \RR \alpha = \{\pm \alpha\}$ and \emph{(R2)} 
$s_\alpha^*(\Phi) \subset \Phi$. The Weyl group associated to $\Phi$ is the 
group $W(\Phi)$ generated by $\{s_\alpha \mid \alpha \in \Phi\}$. We do not 
assume the root system to be reduced or cristallographic, nor that $\Phi$ 
generates $V_\RR^*$, so $W(\Phi)$ may be decomposable. Any reflection group
$G \subset \mathsf{GL}(V)$ is the Weyl group of some root system $\Phi \subset
V^*$.

Root systems are usually defined as subsets of $V_\RR$, but since $V_\RR^*$ is 
canonically isomorphic to $V_\RR$ our definition is equivalent to the one 
found in classical references such as \cite{Hump-coxeter-book} or 
\cite{Hiller-coxeter-book}. We now review the basic features of this theory.

\paragraph
\about{Root systems and reflection groups}
Just as in the case of root systems for Lie algebras we can choose a subset 
$\Phi^+ \subset \Phi$ such that $\Phi = \Phi^+ \cup - \Phi^+$ and this 
determines a set of simple roots, or basis, $\Sigma \subset \Phi^+$ such that 
each $\alpha \in \Phi^+$ is of the form $\sum_{\alpha \in \Sigma} c_\alpha 
\alpha$ with $c_\alpha \in \RR_{\geq 0}$. If we fix a set of positive roots 
$\Phi^+$ with its corresponding set of simple roots $\Sigma$ then the set $S$ 
of reflections corresponding to simple roots is a minimal generating set of the
reflection group $G = W(\Phi)$, and hence $(G,S)$ is a finite Coxeter system. 
Each $s \in G$ of order two is of the form $s_\alpha$ for some $\alpha \in 
\Phi^+$; given $s \in G$ of order two we denote by $\alpha_s$ the 
corresponding positive root.

Once we fix a set of simple roots $\Sigma$, or equivalently a minimal 
generating set $S \subset G$, we obtain a notion of length of an element 
$\sigma \in G$, denoted by $\ell(\sigma)$ and defined as the least natural 
number $\ell$ such that $\sigma$ can be written as a composition of $\ell$ 
reflections in $S$. The group $G$ acts faithfully and transitively on the set 
of roots $\Phi$ and $\ell(\sigma)$ is also equal to the number 
$|\sigma(\Phi^+) \cap -\Phi^+|$, so $G$ has a unique longest longest word of 
length $|\Phi|$ which we will denote by $\omega_0$.

\paragraph
\about{Parabolic subgroups and root subsystems} 
\cite{Hump-coxeter-book}*{1.10}
Fix a root system $\Phi$ with simple roots $\Sigma$ and let $(G,S)$ be the 
corresponding Coxeter system. Set $I \subset S$ to be an arbitrary subset
and denote by $G_I$ the subgroup of $G$ generated by $I$. Then $(G_I, I)$
is also a Coxeter system and it determines a root system $\Phi_I \subset \Phi$
with simple roots $\Sigma_I = \{\alpha_s \mid s \in I\}$.

If $\sigma \in G_I$ then we can compute its length as an element of $G$ with 
respect to the generating set $S$ or as an element of $G_I$ with respect to 
the generating set $I$; however both lengths turn out to be equal, so we 
denote both by $\ell(\sigma)$. The groups $G_I$ and its conjugates are the 
parabolic subgroups of $G$. Since $G_I$ is also a Coxeter group it has
a unique element of maximal length which we will denote by $\omega_0(I)$.

The set $G^I = \{\sigma \in G \mid \ell(\sigma s) > \ell(\sigma) \mbox{ for 
all } s \in I\}$ is a set of representatives of the quotient $G/G_I$, and for 
each $\sigma \in G$ there exist unique elements $\sigma^I \in G^I$ and 
$\sigma_I \in G_I$ such that $\sigma = \sigma^I\sigma_I$ with $\ell(\sigma) = 
\ell(\sigma^I) + \ell(\sigma_I)$. The element $\sigma^I$ is the element of 
minimal length in the coclass $\sigma G_I$. It follows that $(\omega_0)_I
= \omega_0(I)$ and therefore $\omega_0^I = \omega_0 \omega_0(I)^{-1}$.

\paragraph
\about{Coinvariant spaces and their invariants}
Let $V = \CC \ot_\RR V_\CC$ and extend the bilinear form of $V_\RR$ to $V$.
Let $\Lambda = S(V)$ the symmetric algebra of $V$ over the complex numbers 
and set $L = \Frac(\Lambda)$. The action of $G$ on $V$ extends to an action of 
$G$ on $\Lambda$ and on $L$, and we set $\Gamma = \Lambda^G$ and $K = 
\Frac(\Gamma) = L^G$. The algebra $\Lambda$ has a natural $\NN$-grading with 
$\Lambda_1 = V^*$ and $\Gamma$ is a graded subalgebra. Ge denote by $J_G$ the 
ideal generated in $\Lambda$ by the elements of $\Gamma$ of positive degree. 
By the Chevalley-Shephard-Todd theorem $\Gamma$ is isomorphic to a polynomial 
algebra in $\dim_\CC V$ variables, $\Lambda$ is a free $\Gamma$-module of rank 
$|G|$, and a set $B \subset \Lambda$ is a basis of $\Lambda$ as a 
$\Gamma$-module if and only if its image in the quotient $\Lambda/J_G$ is a 
$\CC$-basis. Furthermore $\Lambda/J_G$ is naturally a graded $G$-module 
isomorphic to the regular representation of $G$ with Hilbert series 
$\sum_{\sigma \in G} t^{\ell(\sigma)}$. 

\paragraph
\about{Divided differences}
\label{L:dd-varia}
The smash product $L \# G$ is the $L$ vector space with basis $G$ and endowed 
with a product defined as $f \sigma \cdot g \tau = f \sigma(g) \sigma \tau$.
There is an embedding $L \# G \hookrightarrow \End_\CC(L)$ given by sending
$f \sigma \in L \# G$ to the function $l \in L \mapsto f\sigma(l)$.
In particular $L \# G$ acts on $L$ and given $X \in L\# G$ and $f \in L$ we 
must be careful to distinguish between the result of applying the operator 
corresponding to $X$ to $f$, which we denote by $X(f)$, and their product in 
the smash product $L \#G$, which is $X \cdot f = X(f) f$. 

Let $s\in G$ be a reflection and set
\begin{align*}
\nabla_s 
	&= \frac{1}{\alpha_s}(1-s) \in L \# G.
\end{align*}
Given $f \in L$ we have $\nabla_s \cdot f = \nabla_s(f) + s(f)\cdot \nabla_s$.
Notice that if $s(f) = f$ then $\nabla_s(f) = 0$. 
Given $\sigma \in G$ we take a reduced decomposition $\sigma = s_1 \cdots 
s_{\ell}$ and set $\partial_\sigma = \nabla_{s_1} \circ \cdots \circ 
\nabla_{s_\ell}$; this element is called the \emph{divided difference} 
corresponding to $\sigma$ and does not depend on the chosen reduced 
decomposition. 

Notice that an $L\# G$-module is the same as an $L$ vector space endowed with 
a $G$-module structure such that the action of $L$ is $G$-equivariant. A 
simple induction on the length of $\sigma$ shows that the divided operator
$\partial_\sigma$ defines a $K$-linear map over a $L\# G$-module $Z$. 
In particular $L$ is such a module and $\Lambda \subset L$ is closed under the 
action of divided differences.

\paragraph
\about{Schubert polynomials}
We recall the construction of a basis of $\Lambda/J_G$ by Schubert 
polynomials, due to Demazure \cite{Dem-schubert} and Bernstein, Gelfand and 
Gelfand \cite{BGG-cohomology} in the case where $G$ is a Weyl group, and
generalized to Coxeter groups by Hiller in 
\cite{Hiller-coxeter-book}*{chapter IV}. 

Let $\Phi$ be a root system with reflection group $G$, set $\Delta(\Phi) = 
\prod_{\alpha \in \Phi^+} \alpha$, and for each $\sigma \in G$ set 
$\SS_\sigma = \partial_{\sigma^{-1}\omega_0} \Delta(\Phi)$; notice that by 
definition $\deg \SS_\sigma = \ell(\sigma)$. 
The polynomials $\{\SS_\sigma \mid \sigma \in G\}$ are known as 
the \emph{Schubert polynomials} associated to $\Phi$ and form a basis 
of $\Lambda$ as a $\Gamma$-module, so their images form a basis of 
$\Lambda/J_G$ as a complex vector space. This implies that $L$ is a $K$ 
vector space of dimension $|G|$ and that $\{\SS_\sigma \mid \sigma \in G\}$ is 
also a basis of $L$ over $K$. Given $f \in L$ we will denote by $f_{(\sigma)}$ 
the coefficient of $\SS_\sigma$ in the expansion of $f$ in this basis, so $f 
= \sum_{\sigma \in G} f_{(\sigma)} \SS_\sigma$.

Since Schubert polynomials form a basis of $\Lambda / J_G$ there exist 
$c_{\sigma, \tau}^\rho \in \CC$ defined implicitly by the equation
\begin{align*}
\SS_\sigma \SS_\tau 
	&= \sum_{\rho \in G} c^\rho_{\sigma, \tau} \SS_\rho \mod J_G.
\end{align*}
The coefficients $c_{\sigma, \tau}^\rho$ are the \emph{generalized 
Littlewood-Richardson coefficients} of the root system $\Phi$. It follows from 
the definition that $c^\rho_{\sigma, \tau} = 0$ unless $\ell(\sigma) + 
\ell(\tau) = \ell(\rho)$. If $I \subset S$ then the space of $G_I$-invariants 
$(\Lambda/J_G)^{G_I}$ is generated by the set $\{\SS_\sigma \mid \sigma \in 
G^I\}$ \cite{Hiller-coxeter-book}*{Chapter IV (4.4)}, so in particular if 
$\sigma, \tau \in G^I$ then $c^{\rho}_{\sigma, \tau} \neq 0$ implies that 
$\rho \in G^I$. 

\paragraph
\about{Postnikov-Stanley operators}
\label{ps-operators}
Recall that $V$ inherits a bilinear form from $V_\RR$. This allows us to
identify $V$ with its complex dual $V^*$ and thus we get a bilinear form
$(-,-)$ on $V^*$ which is linear on the first variable.

Given $\alpha \in V^*$ there is a unique $\CC$-linear derivation 
$\Theta(\alpha): \Lambda
\to \Lambda$ such that $\Theta(\alpha)(\beta) = (\beta, \alpha)$ for each 
$\beta \in V^*$. This map extends uniquely to a morphism $\Theta: \Lambda \to 
\mathsf{Der}_\CC(\Lambda, \Lambda)$. Fixing an orthonormal basis $x_1, 
\ldots, x_n$ of $V^*$ so $S(V) \cong \CC[x_1, \ldots, x_n]$ we get that 
$\Theta(x_i) = \frac{\partial}{\partial x_i}$.

Let $(-,-)_\Theta: \Lambda \times \Lambda \to \CC$ be the bilinear form given 
by $(g,f) = \Theta(f)(g)(0)$. This is a nondegenerate bilinear form which 
identifies $\Lambda$ with its graded dual. For every graded ideal $J \subset 
\Lambda$ we write $\mathcal H_J = \{f \in \Lambda \mid (g,f)_\Theta = 0 
\mbox{ for all } f \in J\}$. Since the pairing $(-,-)_\Theta$ is nodegenerate 
the space $\mathcal H_J$ is naturally isomorphic to the graded dual 
$(\Lambda/J)^\circ$. We denote by $P_\sigma$ the unique element in $\mathcal
H_{J_G}$ such that $(P_\sigma, \SS_{\tau}) = \delta_{\sigma, \tau}$. It 
follows that the set $\{P_\sigma \mid \sigma \in G\}$ is a graded basis of 
$\mathcal H_{J_G}$, dual to the Demazure basis of $\Lambda/J_G$. Also for each
$I \subset S$ the set $\{P_\sigma \mid \sigma \in G^I\}$ is a graded basis of 
the dual of $(\Lambda/J_G)^{G_I}$.

The polynomials $P_\sigma$ were described by Postnikov and Stanley in terms of 
the Bruhat order of $G$ in \cite{PS-chains-bruhat} when $G$ is a Weyl 
group. For each covering relation $\sigma \preceq \sigma s_\alpha$ with 
$\alpha \in \Phi^+$ we set $\alpha = m(\sigma,\sigma s_\alpha) \in V^* = 
S(V)_1$, and for a saturated chain $C = (\sigma_1, \sigma_2, \ldots, 
\sigma_r)$ we denote by $m_C$ the product $\prod_{i=1}^{r-1} 
m_C(\sigma_i,\sigma_{i+1})$. Set
\begin{align*}
P_{\sigma, \tau} &= \frac{1}{(\ell(\tau) - \ell(\sigma))!}\sum_C m_C
\end{align*}
where the sum is taken over all saturated chains from $\sigma$ to $\tau$. 
Now according to \cite{PS-chains-bruhat}*{Corollary 6.9} if $\sigma \leq \tau$ 
in the Bruhat order then $P_{\sigma,\tau} = \sum_{\rho \in G} 
c^{\tau}_{\sigma,\rho} P_{\rho}$. This inspires the following definition.
\begin{Definition*}
Suppose $G$ is a Coxeter group. For each $\sigma, \tau \in G$ with $\sigma 
\leq \tau$ in the Bruhat order of $G$ we set $\D_\sigma = \Theta(P_\sigma)$ 
and $\D_{\tau,\sigma} = \sum_{\rho \in G} c^\tau_{\sigma, \rho} \D_\rho$.
\end{Definition*}
Notice that although by defintion $\D_{\tau, \sigma}$ is a differential 
operator on $\Lambda$ it has a well defined extension to the fraction field 
$L$, which we will denote by the same symbol. 

\paragraph
\label{P:leibniz-rule}
We denote by $\D_\sigma^0$ and $\D_{\tau, \sigma}^0$ the linear functional of 
$\Lambda$ obtained by applying the corresponding differential operator followed
by evaluation at $0$. It follows from the definition that $\D_\sigma^0
(\gamma f) = \gamma(0) \D_\sigma^0(f)$ for all $f \in \Lambda$ and $\gamma \in 
\Gamma$. The following proposition shows that this functional extends to the 
algebra of rational functions without poles at $0$ and gives a generalized 
Leibniz rule to compute the result of applying this operator to the product of 
two such functions. 

\begin{Proposition*}
Let $f,g \in L$ be regular at zero. 
\begin{enumerate}[(a)]
\item $f_{(\sigma)}$ is also regular at $0$ and furthermore $f_{(\sigma)}(0) 
= \D_\sigma(f)(0) = (\partial_{\sigma^{-1}} f)(0)$.

\item For each $\sigma \in G$ we have 
\begin{align*}
\D_\sigma^0(fg)
	&= \sum_{\rho \leq \sigma} \D_{\rho, \sigma}^0(f) \D_\rho^0(g)
	= \sum_{\rho \leq \sigma} \D_{\rho}^0(f) \D_{\rho, \sigma}^0(g)
\end{align*}
\end{enumerate}
\end{Proposition*}
\begin{proof}
Let $S \subset \Gamma$ be the set of $G$-invariant rational functions with 
nonzero constant term. This is clearly a $G$-invariant set and hence 
$S^{-1} \Gamma$ is a $G$-invariant subalgebra of $K = \Frac(\Gamma)$. 
Furthermore, if we denote by $A$ the algebra of rational functions in $L =
\Frac(\Lambda)$ which are regular at $0$ then the product map $S^{-1} \Gamma 
\ot \Lambda \to A$ is an isomorphism, since any fraction $p/q \in L$ can be
rewritten so that $g \in \Gamma$. Thus $A$ is a free $S^{-1}\Gamma$-module with
basis $\{\SS_\sigma \mid \sigma \in G\}$ and $f_{(\sigma)} \in A$ for all
$\sigma \in G$. 

It follows from the definition of $\D_\sigma$ that for each $\gamma \in 
\Gamma$ we have $\D_\sigma(\gamma f)(0) = \gamma(0) \D_\sigma(f)(0)$, and it
follows that the same holds if $\gamma \in A^G$. Thus
\begin{align*}
\D_\sigma(f)(0)
	&= \sum_{\tau} f_{(\tau)}(0) \D_\sigma(\SS_\tau)(0)
	= f_{(\sigma)}(0)
\end{align*}
as stated. Analogously 
\begin{align*}
(\partial_{\sigma} f)(0)
	&= \sum_{\tau} f_{(\tau)}(0) (\partial_{\sigma} 
		\partial_{\tau^{-1}\omega_0} \Delta(\Phi))(0)
\end{align*}
Now $\partial_{\sigma} \partial_{\tau^{-1}\omega_0} \Delta(\Phi)$ is 
zero unless $\ell(\tau \sigma) = \ell(\tau) + \ell(\sigma)$, in which case this
equals $\SS_{\tau\sigma^{-1}}$. Evaluation of this polynomial at $0$ is zero 
unless $\tau = \sigma$, in which case it evaluates to one, so 
$(\partial_{\sigma} f)(0) = f_{(\sigma)}(0)$.

Finally
\begin{align*}
\D_\sigma^0(fg)
	&= \sum_{\tau, \rho} f_{(\tau)}(0) g_{(\rho)}(0) 
		\D_\sigma(\SS_\tau \SS_\rho)(0) 
	= \sum_{\tau, \rho} c_{\tau, \rho}^\sigma \D_{\tau}^0(f) 
	\D_{\rho}^0(g)
\end{align*}
and the formulas in the last item follow from the fact that 
$c^\sigma_{\tau, \rho} = c^\sigma_{\rho, \tau}$.
\end{proof}

\paragraph
\about{Root subsystems and normal forms}
\label{L:normal-form}
Given $\Omega \subset \Sigma$ we denote by $\Phi(\Omega)$ the root subsystem 
generated by $\Omega$. We will call such subsystems \emph{standard}. If $\Psi 
\subset \Phi$ is an arbitrary subsystem then it contains a basis $\Omega 
\subset \Psi$, and that basis can be extended to a basis $\overline \Omega$ of 
$\Phi$. Since $G$ acts transitively on the bases of $\Phi$ it follows that for 
some $\sigma \in G$ we have $\sigma(\overline \Omega) = \Sigma$, and hence 
$\sigma(\Psi)$ is standard. Given $v \in V$ we denote by $\Phi_0(v)$ the set
of all roots in $\Phi$ such that $\alpha(v) = 0$, which is clearly a root
subsystem of $\Phi$. We will say that $v$ is $\Phi$-standard if $\Phi_0(v)$
is a standard subsystem of $\Phi$. Given $v \in V$ let $I \subset S$ be the 
set of simple reflections corresponding to $\Sigma \cap \Phi_0(v)$, i.e. the 
set of simple reflections fixing $v$. It follows from the definition that 
$v$ is standard if and only if $G_v$ is the parabolic subgroup corresponding to
$I$, and of course $\Phi_0(v)$ is a root system for this group.

Again take $\Omega \subset \Sigma$. We will say that an element $v \in V$ is 
\emph{$\Omega$-positive} if $\alpha(v) \in \RR_{\geq 0}$ for all $\alpha \in 
\Omega$. Now let $C = \left\{\frac{2(\alpha,\beta)}{(\alpha,\alpha)} \mid 
\alpha, \beta \in \Phi\right\} \subset \CC$ and let $R \subset \CC$ be an 
additive subgroup. For each $v \in V$ we denote by $\Phi_R(v)$ the set of all 
roots $\alpha \in \Phi$ such that $\alpha(v)$ lies in $\sum_{c \in C} Rc$. 
With this definition $\Phi_R(v)$ is a root subsystem of $\Phi$. We will say 
that $v$ is \emph{$R$-standard} if $\Phi_R(v)$ is standard. Furthermore if $R 
\subset \RR$ we set $\Sigma_R(v) = \Phi_R(v) \cap \Sigma$, and we will say 
that $v$ is in \emph{$R$-normal form} if it is $R$-standard and 
$\Sigma_R(v)$-positive. 

\begin{Lemma*}
Let $\Omega \subset \Sigma$ and let $v \in V$. The orbit $G \cdot v$ contains
an $\Omega$-positive element if and only if there exists $\sigma \in G$ such 
that $\alpha(v) \in \RR$ for all $\alpha \in \sigma(\Omega)$. In particular
given an additive subgroup $R \subset \RR$ there exists an element in the 
orbit $G \cdot v$ in $R$-normal form.
\end{Lemma*}
\begin{proof}
If $\sigma(v)$ is $\Omega$-positive then $\alpha(\sigma(v)) \in \RR_{\geq 0}$,
which is equivalent to the fact that each root in $\sigma^{-1}(\Omega)$ sends
$v$ to a positive real, so the only if part is clear.

For the if part, it is enough to show that if $\alpha(v) \in \RR$ for all 
$\alpha \in \Omega$ then there exists $\tau \in G$ such that $\alpha(\tau(v))
\in \RR_{\geq 0}$. Recall that $V = \CC \ot_\RR V_\RR$, and write $v = v_1 + 
i v_2$ with $v_1, v_2 \in V_\RR$. Since $\alpha(v) = \alpha(v_1) + i 
\alpha(v_2) \in \RR$ we see that $\alpha(v_2) = 0$ so the group $W = 
W(\Phi(\Omega))$ is generated by reflections that fix $v_2$. By 
\cite{Hump-coxeter-book}*{1.12} there exists $\tau \in W$ such that $\alpha
(\tau(v_1)) \geq 0$ for all $\alpha \in \Omega$, so $\tau(v) = \tau(v_1) + 
i v_2$ is $\Omega$-positive.
\end{proof}

\begin{Example*}
\begin{enumerate}
\item Consider the case where $\Phi$ is arbitrary. Then $\Phi_0(v)$ is the set 
of all roots in $\Phi$ such that $\alpha(v) = 0$, i.e. the set of all roots 
corresponding to reflections that fix $v$. In this case $\Phi_0(v)$ is 
$\Sigma$ standard if and only if there exists a subset $I$ of the basic 
reflections $S$ such that $G_v$ is generated by $I$.

\item Take $V = \CC^n$ and $\Phi = \{x_i - x_j \mid 1 \leq i,j \leq n\}$, 
where $x_i$ is the $i$-th element of the dual canonical basis. Then $G= S_n$ 
and $\Phi_\RR(v)$ is the set of all roots in $\Phi$ such that $\alpha(v) \in 
\RR$; in other words $x_i - x_j \in \Phi_\RR(v)$ if $v_i - v_j \in \RR$. If $v$
is in $\RR$-normal form then $v_i - v_{i+1}$ lies in $\RR$ if and only if it 
lies in $\RR_{\geq 0}$.

\item Take $V, \Phi, G$ as in the previous example. Then $\Phi_\ZZ(v)$ consists
of all roots such that $\alpha(v) \in \ZZ$, and $v$ is in $\ZZ$-normal form if
whenever $v_ i - v_{i+1} \in \ZZ$ then it lies in $\ZZ_{\geq 0}$.
\end{enumerate}
\end{Example*}



\newpage
\section{Rational Galois algebras}
We recall the definition of Rational Galois algebras, which were introduced 
by Hartwig in \cite{Hart-rational-galois}*{section 4}. As in the previous
section $V$ is a complex vector space, $\Lambda$ is its symmetric algebra and
$L$ is the fraction field of $\Lambda$.

\paragraph
Let $G \subset \mathsf{GL}(V)$ be a pseudo reflection group. As in the 
previous section, this induces actions of $G$ on $\Lambda$ and $L$. 

Given a character $\chi: G \to \CC^\times$, the space of relative invariants 
$\Lambda^G_\chi = \{p \in \Lambda \mid g \cdot p = \chi(g)p\} \subset 
\Lambda$ is a $\Lambda^G$ submodule of $\Lambda$, and by a theorem of Stanley 
\cite{Hiller-coxeter-book}*{(4.4) Proposition} $\Lambda^G_\chi$ is a free 
$\Lambda^G$-module of rank $1$. The generator of $\Lambda^G_\chi$ can be
presented explictly: it is $d_\chi = \prod_{H \in \A(G)} (\alpha_H)^{a_H}$,
where $\A(G)$ is the set of hyperplanes fixed pointwise by some element of 
$G$, each $\alpha_H$ is a linear form such that $\ker \alpha_H = H$, and 
$a_H \in \NN_0$ is minimal with respect to the property that 
$\det[s_H^*]^{a_H} = \chi(s_H)$, where $s_H$ is any generator of the 
stabilizer of $H$ in $G$. Notice that if $G$ is a Coxeter group as in the 
previous section then $a_H$ is either $1$ or $0$.

\paragraph
\label{L:translation}
Let $L \hookrightarrow \End_\CC(L)$ be the algebra map that sends any rational 
function $f \in L$ to the $\CC$-linear map $m_f: f' \in L \mapsto ff' \in L$.
This is an algebra map, and although $\End_\CC(L)$ is not a $L$-algebra it is
at least an $L$-vector space, with $f \cdot \phi = m_f \circ \phi$ for all
$\phi \in \End_\CC(L)$. Also $G$ acts on $\End_\CC(L)$ by conjugation and 
$\sigma \cdot m_f = \sigma \circ m_f \circ \sigma^{-1} = m_{\sigma(f)}$, so 
the map $f \mapsto m_f$ is $G$-equivariant. We will often simply write $f$ for 
the operator $m_f$.

Given $v \in V$ we define a map $a_v: V \to V$ given by $a_v(v') = v'+v$. This
in turn induces an endomorphism of $\Lambda$, which we denote by $t_v$, given  
bt $t_v(f) = f \circ a_{v}$; we sometimes write $f(x+v)$ for $t_v(f)$. Of
course each map $t_v$ can be extended to a $\CC$-linear operator in $L$ and
$t_v \circ t_{v'} = t_{v+v'}$, so we obtain a group morphism $v \in V \mapsto
t_v \in \End_\CC(L)$. The definitions imply that $\sigma \cdot t_v = 
t_{\sigma(v)}$ for each $\sigma \in G$ so this map is $G$-equivariant.

\begin{Lemma*}
Let $G$ and $V$ be as above. Let $A \subset K$ be a subalgebra and let $Z 
\subset V$ be an arbitrary subset. We denote by $A * Z$ the $A$-module generated by $\{t_z \mid z \in Z\}$ in $\End_\CC(L)$.
\begin{enumerate}
\item 
\label{i:translation-li}
The set $\{t_z \mid z \in Z\}$ is linearly independent over $L$.

\item 
\label{i:translation-algebra}
The subspace $L*V$ is a $\CC$-algebra with product given by $(ft_v)(f't_{v'}) 
= f t_v(f') t_{v+v'}$ for each $f,f' \in L, v,v' \in V$.

\item
\label{i:translation-subalgebra}
If $t_z(A) \subset A$ for each $z \in Z$ then $A * Z$ is a subalgebra of 
$L * V$. Furthermore if $A$ and $Z$ are stable by the action of $G$ then so is 
$A * Z$.
\end{enumerate}
\end{Lemma*}
\begin{proof}
Put $T = \sum_{i=1}^N f_i t_{z_i}$ where $f_i \in L^\times$ and each $z_i 
\in Z$, and assume $T = 0$. Given $p \in \Lambda$ we obtain that $0 = T(p) = 
\sum_i f_i p(x+z)$, or equivalently $p \sum_i f_i = \sum_i [p(x+z_i)- p] f_i$. 
Let $v \in V$ be arbitary and choose a polynomial $p$ of positive degree such 
that $p(v) = p(v + z_j)$ for all $j \neq i$ but $p(v + z_i) = p(v) + 1$. Then 
$0 = p(v+z_i) f_i(v)$ so $f_i(v) = 0$. Since $v$ was arbitrary this implies 
that $f_i = 0$ and we have proved item \ref{i:translation-li}. Item 
\ref{i:translation-algebra} is immediate from the definitions and item 
\ref{i:translation-subalgebra} is an easy consequence. 
\end{proof}

\begin{Definition}
A \emph{Rational Galois algebra} is a subalgebra $U \subset \End_\CC(L)$
generated by $\Gamma$ and a finite set of operators $\mathcal X \subset 
(L * V)^G$ such that for each $X \in \mathcal X$ there exists 
$\chi \in \hat G$ with $d_\chi X \in \Lambda * V$.
\end{Definition}
Since $(L*V)^G$ is an algebra of $G$-equivariant operators on $L$, the 
subfield $K = L^G$ is stable by the action of $(L * V)^G$. A Rational Galois
Algebra $U$ has the extra property that $\Gamma$ is also stable by its action.
Now since $\Gamma$ is a left $U$-module its dual $\Gamma^*$ is a right 
$U$-module. The algebra $L*V$ has an anti-isomorphism denoted by $\dagger$, 
defined by $(f t_v)^\dagger = t_v \circ f = t_v(f) t_v$, and if $U$ is a
Rational Galois algebra then so is $U^\dagger$. Thus $\Gamma^*$ can always
be seen as a left $U$-module. 


\newpage
\section{Canonical modules over Rational Galois Algebras}
Throughout this section $V_\RR$ is a real vector space and $\Phi \subset 
V_\RR^*$ is a fixed root system with a fixed set of positive roots $\Phi^+$
and the corresponding basis $\Sigma$. We also denote by $G \subset 
\mathsf{GL}_\RR(V_\RR)$ the finite reflection group associated to $\Phi$
and by $S$ the generating set corresponding to $\Sigma$. Also $V$ denotes
the complexification of $V_\RR$, with the induced $G$-action. We denote by
$\Lambda$ the symmetric algebra of $V$ and set $\Gamma = \Lambda^G$. Finally
we set $L = \Frac(\Lambda)$ and $K = L^G = \Frac(\Gamma)$.

\subsection*{The fundamental $\Gamma$-submodules}
Since $\Gamma$ is contained in any Rational Galois algebra, it is important
to study its action on $\Gamma^*$. This section is dedicated to the study of
a special class of $\Gamma$-submodules.

\begin{Definition}
\label{D:fundamental-subspace}
Let $v \in V$ and let $\sigma \in G$. We set $\D_\sigma^v = \D_\sigma^0 \circ
t_v$ and denote by $\DD^v_\sigma \in \Gamma^*$ the restriction of this 
functional to $\Gamma$. We also set $\Gamma^*(v) = \langle \DD_\sigma^v \mid 
\sigma \in G \rangle$.
\end{Definition}
The subspaces $\Gamma^*(v)$ will play a fundamental role in the sequel.
We will show that they are stable by the action of $\Gamma$ on $\Gamma^*$
and that they are the building blocks for Gelfand-Tsetlin representations of
a Rational Galois aglebra. We begin our study of them with the following lemma.

\begin{Lemma*}
\label{L:gamma-spaces}
Let $I \subset S$, let $W = G_I$ and let $v \in V$ be such that $G_v = W$.
\begin{enumerate}[(a)]
\item 
\label{i:translate-invariant}
Let $\pi_v = \pi \circ t_v: \Gamma \to \Lambda/J_G$. Then $\pi_v(\Gamma) = 
(\Lambda/J_G)^W$.

\item 
\label{i:ps-dual}
The set $\{\DD_\sigma^v \mid \sigma \in G^I\}$ is a basis of $\Gamma^*(v)$, 
and $\DD_\sigma^v = 0$ for all $\sigma \notin G^I$.

\item
\label{i:permute}
$\DD^{\tau(v)}_\sigma \in \Gamma^*(v)$ for each $\tau, \sigma \in G$.
\end{enumerate}
\end{Lemma*}
\begin{proof}
Item \ref{i:translate-invariant} is sometimes called ``Soergel's 
Endomorphismensatz'', though Soergel himself claims the result is folklore and 
even Weyl mentions that this is a classical result. Since we have not found a 
proof in the literature we provide one for completeness. 
The extension $L \mid K$ is Galois with $\mathsf{Gal}(L, K) = G$, so the field
$Kt_v(K)$ must be the fixed field for some $H \subset G$. Now if $\sigma \in 
H$ and $f \in K$ then $f(x+v) = \sigma(f(x+v)) = f(x+\sigma(v))$. Since $f$ is
arbitrary it follows that this happens if and only if $\sigma(v) = v$
so $W = H$. This implies that any $p \in \Lambda^{W}$ is a linear 
combination of rational functions of the form $f(x-v) g(x)$ with 
$f,g \in K$. Since $\Lambda$ is integrally closed in $L$ we must have $f, g 
\in \Gamma$ so $\Gamma t_v(\Gamma) = \Lambda^W$. This implies that 
$\pi_v(\Gamma) = \pi(\Lambda^W) = (\Lambda/J_G)^W$.

As mentioned in paragraph \ref{ps-operators} the set $\{\D_\sigma^0 \mid
\sigma \in G\}$ is the dual basis of the Schubert basis of $\Lambda/J_G$.
Furthermore $\{\SS_\sigma \mid \sigma \in G^I\}$ is a basis of 
$(\Lambda/J_G)^W$, which implies that $\{\D_\sigma^0 \mid \sigma \in G^I\}$
is the correspnding dual basis, while if $\tau \notin G^I$ then the
operator $\D_\tau^0$ is zero over the $W$-invariants of the algebra of 
coinvariants. This last statement, along with the previous item, implies that
$\DD_\tau^v = 0$. On the other hand, the previous item also implies that for 
each $\rho \in G^I$ there exists $\gamma_\rho \in \Gamma$ such that
$\pi_v(\gamma_\rho) = \pi(\SS_\rho)$, so $\DD_\sigma^v(\gamma_\rho) = 
\D_\sigma^0(\SS_\rho) = \delta_{\sigma, \rho}$. This implies the linear 
independence of the set $\{\DD_\sigma^v \mid \sigma \in G^I\}$.

Finally, since the algebra of coinvariants is a $G$-module, for each $\sigma, 
\tau \in G$ the functional $\D_\sigma^0 \circ \tau$ lies in the dual of 
$\Lambda/I_G$ and hence there exist $c_\nu \in \CC$ such that $\D_\sigma^0 
\circ \tau = \sum_\nu c_\nu \D_\nu^0$, so
\begin{align*}
\DD_\sigma^{\tau(v)}(\gamma) 
	&= \D_\sigma^0(\gamma(x + \tau(v))) 
	= \D_\sigma^0 \circ \tau(\gamma(x+v))\\
	&= \sum_\nu c_\nu \D_\nu^0(\gamma(x+v))
	= \sum_\nu c_\nu \DD_\nu^v(\gamma).
\end{align*}
and $\DD_\sigma^{\tau(v)} = \sum_\nu c_\nu \DD^v_\nu$.
\end{proof}

\paragraph
\label{P:gamma-action}
The following is a technical lemma. We thank MathOverflow user David Speyer
for suggesting it as a tool to prove the subsequent proposition.
\begin{Lemma*}
Let $A = A_0 \oplus A_1 \oplus \cdots \oplus A_d$ be a graded, commutative, 
finite-dimensional, self-injective $\CC$-algebra with $A_0 = \CC$. Then 
$A_d = \CC a$ for some $a \neq 0$, and the bilinear form mapping $(b,c) \in 
A \times A$ to the coefficient of $a$ in the product $bc$ is well defined and
non-degenerate.
\end{Lemma*}
\begin{proof}
By \cite{Lam-modules-book}*{(16.22) and (16.55)} $A$ is a symmetric algebra 
and there exists a nonsingular associative bilinear pairing $B: A \times A 
\to \CC$. Since $B$ is nondegenerate for each $a \in A_d$ there exists
$a' \in A$ such that $B(1,a'a) = B(a',a) = 1$. It follows that $a' \in A_0
= \CC$ and hence $A_d$ is one-dimensional, which implies that given $b,c \in 
A$ the coefficient of $a$ in the product $bc$ is well defined. Furthermore 
$B(b,c) = B(1,bc)$ must equal this coefficient.
\end{proof}

\begin{Proposition*}
Let $I \subset S$ and set $W= G_I$. Let $v \in V$ be such that $G_v = W$ and
set $\m_v = \{\gamma \in \Gamma \mid \gamma(v) = 0\}$.
\begin{enumerate}[(a)]
\item 
\label{i:gamma-action}
For each $\sigma \in G^I$ and each $\gamma \in \Gamma$ we have 
$\gamma \cdot \DD_\sigma^v = \gamma(v) \DD_\sigma^v + \sum_{\tau < \sigma} 
\DD_{\sigma, \tau}^v(\gamma) \DD_\tau^v$. 

\item
\label{i:gamma-cyclic} $\Gamma^*(v) = \Gamma \DD_{\omega_0^I}^v$.

\item 
\label{i:gamma-nilpotent}
For each $\sigma \in G^I$ we get $\m^{\ell(\sigma)+1} \DD_\sigma^v = 0$.
\end{enumerate}
\end{Proposition*}
\begin{proof}
For item \ref{i:gamma-action} we use Proposition 
\ref{P:leibniz-rule}, so given $\gamma, \gamma' \in \Gamma$
\begin{align*}
(\gamma \cdot \DD_\sigma^v)(\gamma')
	&= \D_\sigma^0(t_v(\gamma \gamma')) \\
	&= t_v(\gamma)(0) \D_\sigma^0(\gamma'(x-v)) 
		+ \sum_{\tau < \sigma} \D^0_{\tau, \sigma}(t_v(\gamma))
			\D_\tau^0(t_v(\gamma')) \\
	&= \gamma(v) \DD_\sigma^v(\gamma')
		+ \sum_{\tau < \sigma} \DD^v_{\tau, \sigma}(\gamma)
			\DD_\tau^v(\gamma'). 
\end{align*}

By the Chevalley-Shephard-Todd theorem, the algebras $\Gamma = \Lambda^G$ and 
$\Lambda^W$ are both isomorphic to polynomial algebras in $N = \dim V$ 
variables. Denote by $g_1, \ldots, g_N$ a set of algebraically independent 
generators of $\Gamma$ and by $h_1, \ldots, h_N$ a set of algebraically 
independent generators of $\Lambda^W$. Then $A = (\Lambda/I_G)^W = \Lambda^W / 
(I_G \cap \Lambda^W) = \CC[h_1, \ldots, h_N]/(g_1, \ldots, g_N)$ is a complete
intersection and hence falls inside the hypotheses of the previous lemma. Thus
the map $(\pi(f),\pi(g)) \mapsto \D_{\omega^I_0}(fg)(0)$ is a non-degenerate
bilinear form on $A$. 

In particular this tells us that for each $\sigma \in W^I$ there exists a 
polynomial $\SS_\sigma^*$ such that $\D_{\omega^I_0}(\pi(\SS_\sigma^*) 
\pi(\SS_\tau))(0) = 1$. Now choose $\gamma_\sigma \in \Gamma$ such that 
$\pi_v(\gamma) = \pi(\SS_\sigma^*)$, which exists by item 
\ref{i:translate-invariant} of Lemma \ref{D:fundamental-subspace}. Using item 
\ref{i:gamma-action} we obtain
\begin{align*}
\gamma_\sigma \cdot \DD_{\omega_0^I}^v
	&= \sum_{\tau \in G^I} \DD_{\tau, \omega^I_0}^v(\gamma) \DD_\tau^v
	= \sum_{\tau \in G^I} \D_{\omega^I_0}^0(t_v(\gamma)) \DD_\tau^v
	= \DD_\sigma^v
\end{align*}
which shows that $\Gamma^*(v)$ is generated over $\Gamma$ by 
$\D_{\omega^I_0}^v$. Item \ref{i:gamma-nilpotent} is an immediate consequence.
\end{proof}


\paragraph
\about{Jordan form of a generic element}
The action of $\gamma \in \Gamma$ over $\Gamma^*(v)$ depends only on its class
modulo $\ann(\Gamma^*(v))$, which we denote by $[\gamma]$. By item 
\ref{i:gamma-nilpotent} of Proposition \ref{P:gamma-action} $(\gamma-
\gamma(v))^{\ell(\sigma)+1}\DD_\sigma^v = 0$ for all $\sigma \in G^I$ so the 
minimal polynomial of $\gamma$ divides $(t-\gamma(v))^{\ell(\omega_0^I)+1}$. 
Now let $N \subset \Gamma/\ann(\Gamma^*(v))$ be the set of all $[\gamma]$ such 
that the minimal polynomial of $\gamma$ acting on $\Gamma^*(v)$ divides 
$(t-\gamma(v))^{\ell(\omega_0^I)}$. The action of $\Gamma$ on $\Gamma^*(v)$
induces a map $\Gamma/\ann(\gamma^*(v)) \hookrightarrow \End_\CC(\Gamma^*(z))$,
and with this identification it is clear that $N$ is a closed Zarisky 
subset of $\Gamma/\ann(\Gamma^*(v))$. 

\begin{Corollary*}
The set $N$ has positive codimension. In particular the Jordan form of a 
generic element of $\Gamma$ has a Jordan block of size $\ell(\omega_0^I)$.
\end{Corollary*}
\begin{proof}
Set $\gamma = t_{-v}(\sum_{s \in I} \SS_s) \in \Gamma$ and $r = 
\ell(\omega^I_0)$. We will show that $[\gamma] = \notin N$, which implies that 
$N$ has positive codimension. Since $\gamma(v) = 0$, Item 
\ref{i:gamma-nilpotent} of Proposition \ref{P:gamma-action} implies that 
$\gamma^i \cdot \DD_{\omega^I_0}^v$ is a linear combination of 
$\DD_\sigma^v$ with $\ell(\sigma) < r - i$ for each $i \geq 0$. In particular
$\gamma^r \cdot \DD_{\omega_0^I}^v = \D_{e, \omega_0^I}^0(\SS)\DD_e^v$, so all 
we have to do is show that $\D_{\omega_0^I}^0(\SS^r) \neq 0$.

Let $C= (e, \sigma_1, \cdots, \sigma_t)$ be a maximal chain from the 
identity element $e \in G$ to $\sigma_t \in G$, and recall that we 
denote by $m_C$ the product of all $\alpha_{s_i}$ with $\sigma_i s_i = 
\sigma_{i+1}$, that $\Theta(m_C)$ is the corresponding differential operator,
and that $\D_{e,\omega_0^I}$ is given by $\frac{1}{r!} \sum_C \Theta(m_C)$,
where the sum is over all saturated chains in $G$ from $e$ to $\omega^I_0$.
Repeated application of the Lebiniz rule shows that $\Theta(m_C)(\SS^r) =
\sum_{i = 1}^r \D_{s_i}(\SS)(0)$, which is zero unless each $s_i \in I$, and
this case the result is a positive real number. By 
\cite{BB-coxeter-book}*{Theorem 2.5.5} there is at least one chain from $e$
to $\omega_0^I$ such that each $s_i \in I$, so $\D_{e,\omega_0^I}^0(\SS^r) 
\neq 0$. 
\end{proof}

\newpage
\section{Modules associated to characters}
We now begin in earnest our study of $\Gamma^*$ as a module over a Rational
Galois algebra. We keep all the notation from the previous section, and
assume that we have fixed $U \subset (L*V)^G$ a Rational Galois algebra.
\begin{Definition}
Given $X = \sum_v f_v t_v \in L*V$, its \emph{support}, which we denote by 
$\supp X$, is the set of all $v \in V$ such that $f_v \neq 0$. Given a set 
$\mathcal X \subset L*V$ its support $\supp \mathcal X$ is the union of the 
supports of all elements of $\mathcal X$. 
\end{Definition}

\paragraph
Recall that we introduced divded differences as elements of the smash product
$L \# G$, so given any $G$-equivariant $L$ vector space $T$ it makes sense to 
take the divded difference of an element of $T$. In particular this applies to
$\End_\CC(L)$, so given $X \in \End_\CC(L)$ and $\sigma \in G$ we can obtain
a new operator $\partial_\sigma(X)$. Notice that in general this is \emph{not} 
equal to the composition of $\partial_\sigma$ (seen as an element of 
$\End_\CC(L)$) with $X$. The following lemma gathers some properties of these
operators.
\label{L:dd-varia}
\begin{Lemma*}
Let $X \in \End_\CC(X)$.
\begin{enumerate}[(a)]
\item 
\label{i:dd-on-operators}
For each $\sigma \in G$ and $f \in K$ we have $\partial_\sigma(X)(f) = 
\partial_\sigma(X(f))$.

\item 
\label{i:symmetrizing}
Let $\Psi \subset \Phi$ be a standard subsystem, let $G_I \subset G$ be 
the corresponding parabolic subgroup, let $\omega_0^I$ be the longest word
in $G^I$, and let $\Delta(\Phi)^\Psi = \Delta(\Phi)/\Delta(\Psi)$. If $X \in 
\End_\CC(L)^{G_I}$ then $\sum_{\sigma \in G} \sigma \cdot X = 
|G_I| \partial_{\omega_0^I} (X \Delta(\Phi)^\Psi)$.
\end{enumerate}
\end{Lemma*}
\begin{proof}
We will prove item \ref{i:dd-on-operators} by induction on the length of 
$\sigma$. If $\sigma$ is the identity then there is nothing to prove, so let
us assume that $\sigma = s \tau$ with $\ell(\sigma) = 1 + \ell(\tau)$ and 
$s \in S$, and that the statement holds for $\tau$. Putting $X' = 
\partial_\tau(X)$ we get
\begin{align*}
\partial_\sigma(X) (f)
	&= \partial_s(X')(f)
	= \frac{1}{\alpha_s} (X'(f) - s\circ X' \circ s (f)) 
	= \frac{1}{\alpha_s} (X'(f) - s(X'(f)))\\
	&= \partial_s(X'(f))
	= \partial_s(\partial_\tau(X)(f))
	= \partial_\sigma(X(f)).
\end{align*} 

The statement of \cite{Hiller-coxeter-book}*{Chpater IV (1.6)} implies that
$\partial_{\omega_0} = \frac{1}{\Delta(\Phi)} \sum_{\sigma \in G} 
(-1)^{\ell(\sigma)} \sigma$ as operators on $L$. Since the map from $L \# 
G$ to $\End_\CC(L)$ is injective, this equality holds in $L \# G$ and hence
there two elements act identically on any $L$-vector space with a 
$G$-equivariant operator, in particular $\End_\CC(L)$. Since $\sigma \cdot
\Delta(\Phi) = (-1)^{\ell(\sigma)} \Delta(\Phi)$ we can rewrite this as 
stating that for any $X \in \End_\CC(L)$ the equality $\sum_{\sigma \in G}
\sigma \cdot X = \partial_{\omega_0}(X \Delta(\Phi))$ holds. Of course
an analogous equality holds if we replace $G$ with $G_I$ and $\Phi$ with 
$\Psi$.

Let $\omega_1$ be the longest word in $G_I$. Then $\omega_0 \omega_1^{-1}
\in \omega_0 G_I$ and its length is equal to $\ell(\omega_0) - \ell(\omega_1)$,
which is clearly the smallest possible length of an element in the coset
$\omega_0 G_I$. Thus $\omega_0^I = \omega_0 \omega_1^{-1}$ and
\begin{align*}
\sum_{\sigma \in G} \sigma \cdot X 
	&=\partial_{\omega_0}(X \Delta(\Phi))
	= \partial_{\omega^I_0} \partial_{\omega_1}(X \Delta(\Psi) 
	\Delta(\Phi)^\Psi)
\end{align*}
Now both $\Delta(\Phi)^\Psi$ and $X$ are $G_I$-invariant, so the last 
expression equals
\begin{align*}
\partial_{\omega^I_0}(X \Delta(\Phi)^\Psi \partial_{\omega_1}(\Delta(\Psi)) )
= |G_I| \partial_{\omega^I_0}(X \Delta(\Phi)^\Psi).
\end{align*}
This completes the proof of item \ref{i:symmetrizing}.
\end{proof}

\paragraph
\label{P:form}
Let $\Psi \subset \Phi$ be a standard subsystem and let $W \subset G$ be the
corresponding standard parabolic subgroup. If $z \in V$ then by Lemma 
\ref{L:normal-form} there is some element in the orbit $W \cdot z$ whose
stabilizer is of the form $W_I$ for some $I \subset S \cap W$. The set $I$
is not unique, but any two such sets are conjugate by an element of $W$.
Given $Z \subset V$ stable by the action of $W$ we will say that $Y$ is a 
set of standard representatives of $Z/W$ if for each $z \in Z$ there is exactly
one element of $W \cdot z$ in $Y$, its stabilizer is a standard parabolic 
subgroup of $W$, and if any two elements in $Y$ have conjugate stabilizer 
groups then said groups are equal. It is immediate that such a set of 
representatives exists for any $Z$ that is $W$-stable.

\begin{Proposition*}
Let $X \in (L*V)^G$ and assume that there exists $\chi \in \hat G$ such that 
$d_\chi X \in \Lambda * V$. 
\begin{enumerate}[(a)]
\item 
\label{i:invariant-form}
For each $z \in \supp X$ there exists $f_z \in \Lambda^{G_z}$ such that
\begin{align*}
X
	&= \sum_{z \in \supp X} \frac{f_z}{d_{\chi}^z} t_z,
\end{align*}
where $d_{\chi}^z$ is the product of all $\alpha \in \Phi^+$ dividing $d_\chi$ 
such that $\alpha(z) \neq 0$. 

\item
\label{i:dd-form}
Let $\Psi \subset \Phi$ be a standard subsystem and let $W$ be the 
corresponding parabolic subgroup. Let $Y$ be a set of $W$-standard 
representatives of $\supp X / W$, and for each $z \in Y$ denote by $W^z$ the 
set of canonical coset representatives of $W/W_z$, by $\omega_0^z$ the 
longest element in $W^z$, and by $\Delta(\Psi)^z$ be the product of all roots 
in $\Psi^+$ with $\alpha(z) \neq 0$. Then  
\begin{align*}
X
	&= \sum_{z \in Y} \frac{1}{|W_z|}
		\partial_{\omega_0^{z}}\left(
			\frac{f_z \Delta(\Psi)^z}{d_{\chi}^z} t_z
		\right).
\end{align*}
\end{enumerate}
\begin{proof}
Fix $z \in \supp X$. Since $X$ is $G$-invariant we know that $\sigma \cdot X = 
X$ for any $\sigma \in G_z$. By item \ref{i:translation-li} of Lemma
\ref{L:translation} the coefficient of $t_z$ must be $G_z$ invariant, so 
writing this coefficient as $\frac{g}{d_\chi}$ we get
\begin{align*}
\frac{g}{d_\chi} 
	= \sigma \cdot \frac{g}{d_\chi} 
	=\frac{\sigma \cdot g}{\chi(g) d_\chi},
\end{align*}
so $\sigma \cdot g = \chi(\sigma) g$ for all $\sigma \in G_z$. 

Denote by $\chi'$ the restriction of $\chi$ to $G_z$. Now $G_z$ is the 
reflection group generated by the reflections fixing $z$ and it acts on 
$\Lambda$ by restriction, so by Stanley's theorem the space of relative 
invariants $\Lambda^{G_z}_{\chi'}$ is generated over $\Lambda^{G_z}$ by 
$d_{\chi'}$, and this polynimoial is the product of all roots $\alpha \in 
\Phi$ dividing $d_\chi$ such that $\alpha(z) = 0$. Thus $g = f_z d_{\chi}$
for some $f_z \in \Lambda^{G_z}$, and $\frac{g}{d_{\chi}} = 
\frac{f_z}{d_\chi/d_{\chi'}} = \frac{f_z}{d_{\chi}^z}$ and item 
\ref{i:invariant-form} is proved. 

Since $X$ is $G$-invariant, it is clear that
\begin{align*}
X 
	&= \frac{1}{|W|} \sum_{\sigma \in W} \sigma \cdot X
	=\sum_{z \in Y} \frac{1}{|W|}\sum_{\sigma \in W} \sigma \cdot 
		\left( \frac{f_z}{d_\chi^z} t_z\right).
\end{align*}
As we mentioned above the coefficient of $t_z$ is $G_z$-invariant and hence it 
is $W_z$ invariant, so applying item \ref{i:symmetrizing} of Lemma 
\ref{L:dd-varia} to $W$ we obtain
\begin{align*}
\sum_{\sigma \in W} \sigma \cdot
		\left(\frac{f_z}{d_\chi^z} t_z \right)
		&= |W^z| \partial_{\omega_0^{z}}\left(
			\frac{f_z \Delta(\Psi)^z}{d_\chi^z} t_z 
		\right)
\end{align*}
and the result follows.
\end{proof}
\end{Proposition*}

\paragraph
\begin{Corollary*}
Let $\Z = \supp U \subset V$ and let $v \in V$ be $G$-standard. Then 
$\sum_{z \in \Z} \Gamma^*(v+z) \subset \Gamma^*$ is a $U$-submodule.
\end{Corollary*}
\begin{proof}
Let $X$ be a generator of $U$, so there exists $\chi \in \hat G$ such that 
$d_\chi X \in \Lambda * V$. Let $\Psi = \{\alpha \in \Phi \mid \alpha(v) = 
0\}$, which is the standard root subsystem corresponding to $G_v$.

By item \ref{i:dd-form} of Proposition \ref{P:form} $X$ can be written as the 
sum of element of the form $\partial_{\omega_0^z}\left(F t_z\right)$ with 
$W_z$ a standard parabolic subgroup of $W$ and $F = 
\frac{f_z \Delta(\Psi)^z}{d_\chi^z}$. Recall that $d_\chi^z$ is the product of 
all roots $\alpha$ such that $\alpha(z) \neq 0$ and $\alpha \mid d_\chi$. In 
particular each root divides $d_\chi^z$ at most once, and hence 
$\Delta(\Psi)^z / d_\chi^z$ can be evaluated at $v$, which implies that $F$
can be evaluated at $v$, or equivalently $t_v(F)$ can be evaluated at $0$.
In particular $t_v(F) = \sum_{\tau \in G} t_v(F)_{(\tau)} \SS_\tau$ and
by Proposition \ref{P:leibniz-rule} $t_v(F)_{(\tau)}(0) = \D_\tau(F)(v)$.

Since $\omega_0^z \in W = G_v$ we have $t_v \circ \partial_{\omega_0^z} = 
\partial_{\omega_0^z} \circ t_v$, so for each $\sigma \in G^v$
\begin{align*}
	\DD_\sigma^v \circ \partial_{\omega_0^z} \left( F t_z \right) 
	&= \D_\sigma^0 \circ \partial_{\omega^z_0}(t_v(F) t_{v+z})
	=  \D_\sigma^0 \circ \sum_{\tau \in G} t_v(F)_{(\tau)}
		\partial_{\omega^z_0}(\SS_\tau t_{v+z}) \\
	&= \sum_{\tau \in G} \D_\tau^v (F) 
		(\D_\sigma^0 \circ \partial_{\omega^z_0}(\SS_\tau t_{v+z}) )
\end{align*} 
so $\DD_\sigma^v \circ X$ is a linear combination of operators of the form 
$\D_\sigma^0 \circ \partial_{\omega^z_0}(\SS_\tau t_{v+z})$. Applying
this to $\gamma \in \Gamma$ and using item \ref{i:dd-on-operators} of Lemma
\ref{L:dd-varia} we obtain
\begin{align*}
\D_\sigma^0 \circ \partial_{\omega^z_0}(\SS_\tau t_{v+z})(\gamma)
	&= \D_\sigma^0(\partial_{\omega^z_0}(\SS_\tau \gamma(x+v+z)))
\end{align*}
Writing $\gamma(x+v+z) = \sum_{\rho \in G} \gamma(x+v+z)_{(\rho)} 
\SS_\rho$ and using again Proposition \ref{P:leibniz-rule} we get
\begin{align*}
\D_\sigma^0(\partial_{\omega^z_0}(\SS_\tau \gamma(x+v+z)))
	&= \sum_{\rho \in G}
		\D_\sigma^0(\partial_{\omega^z_0}\SS_\tau \SS_\rho) 
		 \D_\rho^{v+z}(\gamma)\\
	&= \begin{cases}
		\sum_{\rho \in G} c^{\sigma \omega_0^z}_{\tau, \rho} 
		\DD_\rho^{v+z}(\gamma) 
			& \mbox{ if } \ell(\sigma \omega_0^z) 
				= \ell(\sigma) - \ell(\omega_0^z) \\
		0 & \mbox{otherwise}
	\end{cases}
\end{align*}
Thus the restriction of $\D_\sigma^0 \circ \partial_{\omega^z_0}(\SS_\tau 
t_{v+z})$ to $\Gamma$ is either $\sum_{\rho \in G} 
c^{\sigma \omega_0^z}_{\tau, \rho} \DD^{v+z}_\rho$ or $0$, and hence
\begin{align*}
\DD_\sigma^v \circ \partial_{\omega_0^z} \left(F t_z \right)
	&= 
	\begin{cases}
	\sum_{\rho \in G} \D_{\rho, \sigma\omega^z_0}^v(F) \DD_\rho^{v+z}
		& \mbox{ if } \ell(\sigma \omega_0^z) = 
			\ell(\sigma) - \ell(\omega_0^z) \\
	0 & \mbox{otherwise.}	
	\end{cases}
\end{align*}
In either case this element belongs to $\Gamma^*(v+z)$, so $\DD_\sigma^v 
\circ X \in \sum_{z \in \supp X} \Gamma^*(v+z)$, which implies the statement.
\end{proof}
The sum of spaces $\sum_{z \in \Z}\Gamma^*(v+z)$ is general not a direct sum.
Indeed it is easy to find cases where $v \in V$ is in normal form, and 
there exist $z, z' \in V$ which are not $G_v$-conjugate but $v+z$ and $v+z'$
are $G$-conjugate, so $\Gamma^*(v+z) = \Gamma^*(v+z')$. We can turn this into
a direct sum by choosing a unique normal form $[v+z]$ for each element in 
$v+\Z$, but we have not found a systematic way to do it. In the following 
sections we will consider a family of Rational Galois algebras for whom it is
possible, once $v$ is fixed, to choose a subset $\Z(v) \subset \Z$ such that
$\sum_{z \in \Z}\Gamma^*(v+z) = \bigoplus_{z \in \Z(v)} \Gamma^*(v+z)$. This 
will allow us to fix a basis for this space and give a more precise description
of its $U$-module structure.


\begin{bibdiv}
\begin{biblist}
\bibselect{biblio}
\end{biblist}
\end{bibdiv}

\end{document}