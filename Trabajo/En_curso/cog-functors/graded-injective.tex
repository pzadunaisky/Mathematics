%%%%%%%%%%%%%%%%%%%%%% Generalities %%%%%%%%%%%%%%%%%%5
\documentclass[11pt,fleqn]{article}
\usepackage[paper=a4paper]
  {geometry}

\pagestyle{plain}
\pagenumbering{arabic}
%%%%%%%%%%%%%%%%%%%%%%%%%%%%%%%%
% esto tiene que estar, en este orden...
\usepackage[small]{titlesec}
\usepackage{paragraphs}
\usepackage{hyperref}
\usepackage{amsthm,thmtools}

%\linespread{1.2}
\setlength{\parskip}{1.2ex}

\usepackage[utf8,latin1]{inputenc}
\usepackage[spanish,english]{babel}
\usepackage{enumerate}
\usepackage{mathpazo}
\usepackage{euler}
\usepackage[alphabetic,initials]{amsrefs}
\usepackage{amsfonts,amssymb,amsmath}
\usepackage{dsfont}
\usepackage{mathtools}
\usepackage{graphicx}
\usepackage[poly,arrow,curve,matrix]{xy}
\usepackage{wrapfig}
%\swapnumbers

% numbered versions
\declaretheoremstyle[headformat=swapnumber, spaceabove=\paraskip,
bodyfont=\itshape]{mystyle}
\declaretheorem[name=Lemma, sibling=para, style=mystyle]{Lemma}
\declaretheorem[name=Proposition, sibling=para, style=mystyle]{Proposition}
\declaretheorem[name=Theorem, sibling=para, style=mystyle]{Theorem}
\declaretheorem[name=Corolllary, sibling=para, style=mystyle]{Corollary}
\declaretheorem[name=Definition, sibling=para, style=mystyle]{Definition}

% unnumbered versions
\declaretheoremstyle[numbered=no, spaceabove=\paraskip,
bodyfont=\itshape]{mystyle-empty}
\declaretheorem[name=Lemma, style=mystyle-empty]{Lemma*}
\declaretheorem[name=Proposition, style=mystyle-empty]{Proposition*}
\declaretheorem[name=Theorem, style=mystyle-empty]{Theorem*}
\declaretheorem[name=Corollary, style=mystyle-empty]{Corollary*}
\declaretheorem[name=Definition, style=mystyle-empty]{Definition*}

% plain style
\declaretheoremstyle[
        headformat={{\bfseries\NUMBER.}{\itshape\NAME}\NOTE\ignorespaces},
        spaceabove=\paraskip, 
        headpunct={.},
        headfont=\itshape,
        bodyfont=\normalfont
        ]{mystyle-plain}
\declaretheorem[sibling=para, style=mystyle-plain]{Example}
\declaretheorem[sibling=para, style=mystyle-plain]{Remark}

% proofs, just as in amsthm but with adapted spacing
\makeatletter
\renewenvironment{proof}[1][\proofname]{\par
  \pushQED{\qed}%
  \normalfont \topsep.75\paraskip\relax
  \trivlist
  \item[\hskip\labelsep
        \itshape
    #1\@addpunct{.}]\ignorespaces
}{%
  \popQED\endtrivlist\@endpefalse
}
\makeatother

%%%%%%%%%%%%%%%%%%%%%%%%%%% The usual stuff%%%%%%%%%%%%%%%%%%%%%%%%%
\newcommand\NN{\mathbb N}
\newcommand\CC{\mathbb C}
\newcommand\QQ{\mathbb Q}
\newcommand\RR{\mathbb R}
\newcommand\ZZ{\mathbb Z}
\newcommand\KK{\mathbb K}

\newcommand\maps{\longmapsto}
\newcommand\ot{\otimes}
\renewcommand\to{\longrightarrow}
\renewcommand\phi{\varphi}
\newcommand\eps{\varepsilon}
\newcommand\uu[1]{\underline{#1}}

%%%%%%%%%%%%%%%%%%%%%%%%% Specific notation %%%%%%%%%%%%%%%%%%%%%%%%%

\newcommand\A{\mathcal A}
\newcommand\B{\mathcal B}
\renewcommand\L{\mathcal L}
\newcommand\R{\mathcal R}
\newcommand\F{\mathcal F}
\renewcommand\O{\mathcal O}
\newcommand\I{\uu I}
\renewcommand\b{\mathfrak{b}}
\newcommand\g{\mathfrak{g}}
\newcommand\h{\mathfrak h}
\newcommand\sq{\square}
\newcommand\n{\mathfrak{n}}
\newcommand\co{\mathsf{co}}
\DeclareMathOperator\Id{\mathsf{Id}}
\DeclareMathOperator\st{\mathsf{st}}

\newcommand\C{\mathsf C}
\renewcommand\k{\Bbbk}
\newcommand\D{\mathsf D}

\DeclareMathOperator\Mod{\mathsf{Mod}}
\DeclareMathOperator\Hom{\mathsf{Hom}}
\DeclareMathOperator\Ext{\mathsf{Ext}}
\DeclareMathOperator\Tor{\mathsf{Tor}}

\DeclareMathOperator\Gr{\mathsf{Gr}}
\DeclareMathOperator\GrHom{\underline{\mathsf{Hom}}}
\DeclareMathOperator\GrExt{\underline{\mathsf{Ext}}}
\DeclareMathOperator\GrTor{\underline{\mathsf{Tor}}}

\DeclareMathOperator\Ab{\mathsf{Ab}}
\DeclareMathOperator\ShHom{\mathcal Hom}
\DeclareMathOperator\ShExt{\mathcal Ext}
\DeclareMathOperator\ShTor{\mathcal Tor}

\DeclareMathOperator\pd{pd}
\DeclareMathOperator\id{id}
\DeclareMathOperator\rank{rank}
\DeclareMathOperator\Spec{Spec}
\DeclareMathOperator\supp{supp}
\DeclareMathOperator\ann{ann}
\DeclareMathOperator\im{Im}
\DeclareMathOperator\gr{gr}

%%%%%%%%%%%%%%%%%%%%%%%%%%%%%%%%%%%%%% TITLES %%%%%%%%%%%%%%%%%%%%%%%%%%%%%%
\title{On a pleasant exercise in homological algebra}
\date{[graded-injective.tex]}
\author{Pablo Zadunaisky}
\begin{document}

\maketitle
\begin{quote}
In \cite{vdB}, M. Van den Bergh states that if $A$ is a noetherian $\NN$-graded connected
algebra over a field $\k$ and $I$ is an injective object in the category of $\ZZ$-graded
$A$-modules, then $I$ has injective dimension at most one when considered as an
$A$-module. He leaves the proof of this fact as ``a pleasant exercise in homological
algebra''. An analogous result was proved for commutative algebras by Fossum and Foxby in
their article \cite{FF}. In this note we prove a general version of this result.
\end{quote}

\section{Introduction}
\paragraph
Throughout this document $\k$ denotes a commutative ring. All unadorned tensor products
are over $\k$. Given $A$-modules $I$ and $M$, we say that $M$ is $I$-acyclic if
$\Ext^i_A(M,I) = 0$ for $i > 0$.

\paragraph
Let us look at a possible solution to the exercise. Let $A$ be a $\ZZ$-graded algebra and
let $N$ be an $A$-module. We can turn $N \ot \k[t,t^{-1}]$ into a $\ZZ$-graded $A$-module,
with the action of an element $a \in A$ of degree $l \in \ZZ$ given by $a \cdot (n \ot t^r)
= an \ot t^{r+l}$ for all $n \in N$ and $r \in \ZZ$. There is an $A$-linear surjective map
$N \ot \k[t,t^{-1}] \to N$, defined by $n \ot t \mapsto n$, with kernel $N \ot (t-1)
\k[t,t^{-1}]$. Thus we obtain an exact sequence of $A$-modules
\begin{align*}
\xymatrix{
	0 \ar[r] & N \ot\k[t,t^{-1}] \ar[r]^-{1 \ot (t-1)} & N \ot\k[t,t^{-1}] \ar[r] &N \ar[r]
	&0.
}
\end{align*}
If we only consider the underlying vector spaces in this sequence, this is the tensor
product of $N$ with the minimal projective resolution of $\k \cong\k[t,t^{-1}]/(t-1)$ as a
$\k[t,t^{-1}]$-module.

Let $I$ be an injective object in the category of $\ZZ$-graded $A$-modules. Since $N \ot
\k[t,t^{-1}]$ is a graded module, it is natural to ask whether the fact that $I$ is
\emph{graded} injective implies that $\Ext_A(N \ot \k[t,t^{-1}]) = 0$; in Proposition
\ref{acyclic} we show that this holds if $A$ is
noetherian. Thus we have a resolution of length $1$ of $N$ by $I$-acyclic modules, which
proves that $\Ext_A^2(N,I) = 0$. Since $N$ is arbitrary, this proves that the injective
dimension of $I$ as an $A$-module is at most $1$.

\paragraph
In this note we try to put this result in a more general perspective, showing how
injective dimension changes when we change the grading group over a fixed algebra.
Surprisingly, the general case follows the same pattern as in the $\ZZ$-graded case, and
requires little more than general homological algebra.

\section{Graded algebras and modules:}
Let $G, H$ be two groups. The letters $g, g', g'' \ldots$ will always denote elements of
$G$, and $h, h', h''\ldots$ will always denote elements of $H$.
	
\paragraph
Let $A$ be a $\k$-algebra. We say that $A$ is $G$-graded if it can be decomposed as the
direct sum of subspaces $A_g$ indexed by $G$ such that $A_g A_{g'} \subset A_{gg'}$. A
$G$-graded module over $A$ is an $A$-module with a decomposition as a direct sum of
subspaces $M_g$ such that $A_g M_{g'} \subset M_{gg'}$. The vector space $M_g$ is called
the \emph{homogeneous component} of degree $g$. We consider $\k$ to be a graded algebra
with $\k_{1_G} =\k$.

An $A$-linear morphism between graded modules $N, M$ is said to be homogeneous of degree $g$
if we have $f(N_{g'}) \subset M_{g'g}$ for all $g'$. The category $\Mod^G_A$ has
$G$-graded $A$-modules as objects and homogeneous $A$-linear morphisms of degree $1_G$ as
morphisms. We write $\Hom^G_A(N,M)$ for $\Hom_{\Mod^G_A}(N,M)$, and for any $f \in
\Hom^G_A(N,M)$ we write $f_g$ for the $\k$-linear function $f_g: N_g \to M_g$ induced by
$f$ by restriction and correstriction.

Given a $G$-graded $A$-module $M$, we denote by $M[g]$ the $G$-graded $A$-module whose
homogeneous components are given by $M[g]_{g'} = M_{g'g}$. If $f: M \to M'$ is a morphism
in $\Mod^G_A$ then defining $f[g]: M[g] \to M'[g]$ to be the morphism whose underlying
function is the same as that of $f$, we obtain the \emph{$g$-suspension functor}, denoted by
$\Sigma_g: \Mod^G_A \to \Mod^G_A$. One can check that $\Sigma_g \circ \Sigma_{g^{-1}} =
\Id$, so the $g$-suspension functor is an automorphism of $\Mod^G_A$.

\paragraph\textbf{Change of grading functors.}
\label{change-of-grading-functors}
Let $H$ be another group and let $\phi: G \to H$ be a group morphism. Any $G$-graded
algebra $A$ can be given an $H$-grading by setting 
\[
  A_h = \bigoplus_{g \in \phi^{-1}(h)} A_g
\]
From this point on we fix such a morphism $\phi$, and we will consider $A$ to be an
$H$-graded algebra with the induced grading.

The morphism $\phi$ induces three functors 
\begin{align*}
  \phi_!, \phi_*: \Mod^G_A &\to \Mod^H_A & \phi^*: \Mod^H_A &\to \Mod^G_A
\end{align*}
Given a $G$-graded $A$-module $M$, the $H$-graded $A$-modules $\phi_!(M), \phi_*(M)$ have
homogeneous components of degree $h$ given by
\begin{align*}
	\phi_!(M)_h &= \bigoplus_{g \in \phi^{-1}(h)} M_g; & \phi_*(M)_h &= \prod_{g \in
\phi^{-1}(h)} M_g. 
\end{align*}
Notice that the underlying $A$-module of $\phi_!(M)$ is equal to that of $M$, so the
action of $A$ on $\phi_!(M)$ is simply the same as that on $M$, which is compatible with
the $H$-grading. In order to describe the action of $A$ on $\phi_*(M)$ it is enough to say
how a $G$-homogeneous element acts on an $H$-homogeneous element of the module and extend
the action by bilinearity: given $a \in A_g$ and $(m_{g'})_{g' \in \phi^{-1}(h)} \in
\phi_*(M)_h$, we set $a(m_g)_{g \in \phi^{-1}(h)} = (am_{g'})_{g' \in \phi^{-1}(h)}$.

If $f$ is a homogeneous morphism of $G$-graded modules (of arbitrary degree), then
$\phi_!(f)_h = \bigoplus\limits_{g \in \phi^{-1}(h)} f_g$ and $\phi_*(f)_h =
\prod\limits_{g \in \phi^{-1}(h)}f_g$.

Given $N$ an object of $\Mod^H_A$, we set 
\begin{align*}
  \phi^*(N) = \bigoplus_{g \in G} N_{\phi(g)} u_g.
\end{align*}
The action of a homogeneous element $a \in A_g$ over an element $n u_{g'} \in
\phi^*(N)_{g'}$ is given by $a(nu_{g'}) = (an) u_{gg'}$. Notice that $a \in A_{\phi(g)}$
and $n \in N_{\phi(g')}$, so $an \in N_{\phi(gg')}$ and the action is well defined. Given
a morphism of $H$-graded $A$-module $f: N \to N'$, we set $\phi^*(f): \phi^*(N) \to
\phi^*(N')$ to be the morphism given by the assignation $n u_g \mapsto f(n)u_g$.

Another way to understand this functor is as follows. Clearly $N \ot \k[G]$ is an $A \ot
\k[G]$-module, and since $A$ is $G$-graded there is a ring morphism $A \to A \ot \k[G]$
given by $a \in A_g \mapsto a \ot g$. Thus $N \ot \k[G]$ can be given an $A$-module
structure, and $\phi^*(N)$ can be identified with the $A$-submodule of $N \ot \k[G]$
generated by the elementary tensors of the form $n \ot g$ such that $\deg n = \phi(g)$.


\paragraph
\label{adjoint-functors}
The support of any module ${}_! M$ is clearly contained in the image of $\phi$, and
${}^*N$ depends only on the components of $N$ whose degree belongs to the image of
$\phi$. Hence from now on we assume $\phi$ to be surjective. Put $L = \ker \phi$.

\begin{Lemma*}
  Let $M$ be an object in $\Mod^G_A$. There exist natural isomorphisms
  \begin{align*}
	  \phi^* \circ \phi_!(M) &\cong \bigoplus_{l \in L} M[l]; &
	  \phi^* \circ \phi_*(M) &\cong \prod_{l \in L} M[l]
 \end{align*}
\end{Lemma*}
\begin{proof}
  Write $\equiv_L$ for congruence modulo $L$, that is, $g \equiv_L g'$ if and only if
  $\phi(g) = \phi(g')$. From the definitions we get
  \begin{align*}
	  (\phi^*(\phi_!(M))_g &= \phi_!(M)_{\phi(g)} = \bigoplus_{g' \equiv_L g} M_{g'}u_{g} =
	  \bigoplus_{g \in G} \bigoplus_{l \in L} M_{gl} u_{g},
  \end{align*}
  and a similar isomorphism for $\phi^* \circ \phi_*(M)$, so there are vector space
  isomorphisms as in the statement. It is routine to check that they are also
  $A$-linear and natural.
\end{proof}

\begin{Proposition*}
\label{P:adjoint}
  The functor $\phi^*$ is right adjoint to $\phi_!$ and left adjoint to $\phi_*$
\end{Proposition*}
\begin{proof}
	By the previous lemma there exists a natural transformation $\iota_M: M \to
	\phi^*(\phi_!(M))$, given by the inclusion of $M$ in the direct sum
	$\bigoplus\limits_{l \in L} M[l]$. Also there is a natural transformation
	$\eps_N: \phi_!(\phi^*(N)) \to N$ given by the assignation $n u_g \mapsto n$. These are
	respectively the unit and the counit of the adjoint pair $(\phi_!, \phi^*)$.
	There are also natural transformations $\tau_N: N \to \phi_*(\phi^*(N))$, given by $n
	\in N_h \mapsto (nu_g)_{g \in \phi^{-1}(h)} \in \phi_*(\phi^*(N))_h$, and $\pi_M:
	\phi^*(\phi_*(M)) \to M$,	given by the projection on to the factor $M$, which are the
	unit and the counit of the	adjoint pair $(\phi^*, \phi_*)$.
\end{proof}

\paragraph
By definition the functors $\phi_!, \phi_*$ and $\phi^*$ are exact. By the standard
properties of adjoint functors, $\phi_!$ sends projective objects to projective objects
and $\phi^*$ sends injective objects to injective objects. On a side note, $\phi^*$ also
has a right adjoint $\phi_*$; it is similar to $\phi_!$, but uses direct products instead
of direct sums.

\paragraph
\label{otimes-t}
We fix $t: H \to G$ a right inverse of $\phi$ as a function and define the morphism
\begin{align*}
N \ot\k[L] &\to {}^* N \\
n_h \ot l &\longmapsto n_hu_{t(h)l},
\end{align*}
which is clearly an isomorphism. We can pull back the $A$-module structure of ${}^*N$
through this morphism and give $N \ot\k[L]$ a $G$-graded $A$-module structure with the
action of $A$ given by
\[
  a_g(n_h \ot l) = a_g n_h \ot t(\phi(g)h)^{-1} g t(h) l
\]
Notice that $g^h := t(\phi(g)h)^{-1}g t(h) \in L$, and that since the action on $N \ot
k[L]$ is induced by the action on ${}^*N$, it is $(g'g)^h = g'^{\phi(g)h} g^h$. 

\paragraph
\label{otimes-t-def}
Now let $N$ be an object in $\Mod_A^H$. For any $\k[L]$-module $P$ we define a $G$-graded
$A$-module $N \ot^t P$. Its underlying vector space is $N \ot P$, with an element $a \in
A_g$ acting by
\[  
  a(n_h \ot p) = (a n_h) \ot g^h \cdot p.
\]
The homogeneous component of degree $g$ of $N \ot^t P$ is given by $N_{\phi(g)} \ot
P_{t(\phi(g))^{-1}g}$. Notice that by definition $N \ot^t\k[L] \cong {}^* N$.

For any $\k[L]$-linear morphism $f: P \to Q$ set $1 \ot^t f:= 1 \ot f : N \ot^t P \to N
\ot^t Q$. This morphism is not $G$-homogeneous, but we do have the following result. Its
proof follows immediately from the definitions.

\begin{Lemma*}
  For every object $N$ of $\Mod^H_A$ the assignation $\Mod\k[L] \to \Mod^H_A$ that maps
  every object $P$ to $\phi_!(N \ot^t P)$ and every morphism $f$ to $1 \ot^t f$ is an
  additive and exact functor.
\end{Lemma*}

We write $N \ot^t P$ instead of $\phi_!(N \ot^t P)$.

\paragraph
\label{acyclic}
We are now ready to generalize the argument given in the introduction. The first step is
proving the acyclicity result we mentioned there. Given objects $I, M$ of $\Mod^H_A$, we
say that $M$ is $I$-acyclic if $\R^i \Hom_A^H(M,I) = 0$ for all $i > 0$.

\begin{Proposition*}
  Assume $A$ is a noetherian $G$-graded $\k$-algebra and let $I$ be an injective object in
  $\Mod^G_A$. If $M$ is an object of $\Mod^G_A$, then $\phi_!(M)$ is $\phi_!(I)$-acyclic.
\end{Proposition*}
\begin{proof}
  By Proposition \ref{P:adjoint} there is a natural equivalence
  \[ 
    \Hom_A^H(\phi_!(-),\phi_!(I)) \cong \Hom^G_A(-,\phi^*(\phi_!(I)).
  \]
  Now by Lemma \ref{L:unit} ${}^* {}_! I = \bigoplus_{l \in L} I[l]$ is a direct sum of
  injective modules. Since $A$ is noetherian this is again an injective module, and so
  $\Hom_A^H(\phi_!(-),\phi_!(I))$ is an exact functor. On the other hand, since $\phi_!$
  is an exact functor that sends projective objects to projective objects,
  $(\R^i\Hom_A^H)(\phi_!(-),\phi_!(I)) \cong \R^i(\Hom_A^H(\phi_!(-),\phi_!(I))) = 0$.
\end{proof}

\begin{Theorem}
\label{injective-dimension}
  Let $A$ be a noetherian $G$-graded $\k$-algebra and let $I$ be an injective object of
  $\Mod^G_A$. The injective dimension of $\phi_!(I)$ is at most the projective dimension of
  $\k$ as a $\k[L]$-module with trivial structure.
\end{Theorem}
\begin{proof}
  Let $P_\bullet \to\k \to 0$ be a minimal projective resolution of $\k$ as a $\k[L]$-module
  and let $N$ be any object of $\Mod^H_A$. Applying $N \ot^t -$ to this resolution we get
  an exact complex in $\Mod^H_A$
  \begin{align}
  \label{c}
    N \ot^t P_\bullet \to N \ot^t\k \to 0
  \end{align}
  Since $P_\bullet$ is a projective $\k[L]$ module, it is a direct summand of a free
  $\k[L]$-module $F_\bullet$ of rank $n_\bullet$. Since $N \ot^t -$ is additive, this
  implies that $N \ot^t P_\bullet$ is a direct summand of $N \ot^t F_\bullet \cong ({}_!
  {}^*N)^{n_\bullet}$, which by Proposition \ref{acyclic} is $\phi_!(I)$-acyclic. Thus $N
  \ot^t P_\bullet$ is also $\phi_!(I)$-acyclic, and \ref{c} is a resolution of $N$ by
  $\phi_!(I)$-acyclic modules. Since its length is equal to the projective dimension of
  $\k$ as a $\k[L]$-module, the result follows.
\end{proof}

\begin{bibdiv}
\begin{biblist}

\bib{Doi}{article}{
   author={Doi, Yukio},
   title={Homological coalgebra},
   journal={J. Math. Soc. Japan},
   volume={33},
   date={1981},
   number={1},
   pages={31--50},
}

\bib{FF}{article}{
   author={Fossum, Robert},
   author={Foxby, Hans-Bj{\o}rn},
   title={The category of graded modules},
   journal={Math. Scand.},
   volume={35},
   date={1974},
   pages={288--300},
   issn={0025-5521},
   review={\MR{0379473 (52 \#378)}},
}

\bib{M}{book}{
   author={Montgomery, Susan},
   title={Hopf algebras and their actions on rings},
   series={CBMS Regional Conference Series in Mathematics},
   volume={82},
   publisher={Published for the Conference Board of the Mathematical
   Sciences, Washington, DC},
   date={1993},
   pages={xiv+238},
   isbn={0-8218-0738-2},
   review={\MR{1243637 (94i:16019)}},
}

\bib{vdB}{article}{
   author={van den Bergh, Michel},
   title={Existence theorems for dualizing complexes over non-commutative
   graded and filtered rings},
   journal={J. Algebra},
   volume={195},
   date={1997},
   number={2},
   pages={662--679},
}

\end{biblist}
\end{bibdiv}

%\end{document}

\section{An approach through Hopf algebras}
\label{hopf}
As is well known, a $G$-graded algebra is a comodule algebra over the Hopf algebra $\k[G]$,
and a $G$-graded module is a $(A,k[G])$-Hopf module. In this section we try to recast
Theorem \ref{injective-dimension} in terms of the theory of Hopf Algebras. For undefined
terms and general results on Hopf algebras we refer to \cite{M}.

\paragraph
Let $G$ be Hopf algebra. We denote by $\Mod^G$ the category of right $G$-comodules. If $A$
is a $G$-comodule algebra, we denote by $\Mod^G_A$ the category of $(A,G)$-Hopf modules.
Give a $G$-comodule $M$ we denote by $\rho^G_M: M \to M \ot G$ the comodule structure map
of $M$.

\paragraph
Let $M$ be a left $G$-comodule and $N$ a right $G$-comodule. Their \emph{cotensor product}
$N \square_G M$ is defined as the cokernel of the map 
\[
  \rho^G(N)\ot 1 - 1 \ot \rho^G(M): N \ot M \to N \ot G \ot M
\]
Given a morphism $f: M \to M'$ of $G$-comodules, $N \sq_G f: N \sq_G M \to N \sq_G M'$ is
the restriction of $1 \ot f$ to $N \square_G M$. The universal property of the
cokernel guarantees that this morphism is well defined. 

\paragraph
Let $G, H$ be Hopf algebras and let $\phi: G \to H$ be a Hopf-algebra morphism. In
particular $G$ is a left $H$-comodule with structure map $\rho^H_G = (\phi \ot 1) \circ
\rho_G^G$. In \cite{Doi}, Y. Doi associates to any such morphism two functors
\begin{align*}
\xymatrix{
 \Mod^G \ar@/^/[r]^{-_\phi} &\Mod^H \ar@/^/[l]^-{-^\phi}
}
\end{align*}
defined as follows. For every $G$-comodule $M$, we set $M_\phi$ to be the $H$-comodule
whose structure map is given by $\rho^H_{M_\phi} = (1 \ot \phi) \circ \rho^G_M$. On the
other hand, for every $H$-comodule $N$, the $G$-comodule $N^\phi$ is equal to the vector
space $N \sq_H G$, whose structure map $\rho^G_{N^\phi}$ is equal to the restriction $1
\sq_H \rho^G_G$. By \cite{Doi}*{Proposition 6}, $-_\phi$ is left adjoint to $-^\phi$.


\paragraph
If $A$ is a $G$-comodule algebra then $A_\phi$ is an $(A,H)$-comodule algebra. We will
usually omit the subindex. If $M$ is an object of $\Mod^G_A$, then $M_\phi$ has a natural
$(A,H)$-Hopf module structure with the same $A$-module structure. On the other hand, for any
object $N$ of $\Mod^H_A$, the $G$-comodule $N \sq_H G$ has the structure of an
$(A,G)$-Hopf module, with the action of $a$ on an element $n \ot g$ given by $a (n \ot g) =
a_0n \ot a_1 g$. Thus $-_\phi$ and $-^\phi$ induce functors
\begin{align*}
\xymatrix{
 \Mod^G_A \ar@/^/[r]^{-_\phi} &\Mod^H_A \ar@/^/[l]^-{-^\phi}
}
\end{align*}
If $G$ and $H$ are group algebras then $\phi$ induces a morphism between the underlying
groups, and $-_\phi = \phi_!$ and $-^\phi = \phi^*$.

\paragraph
Let $L = G^{\co H}$ and assume that the extension $L \subset G$ is \emph{cleft}, that is,
there exists $\gamma \in \Hom_k(H,G)$ which is convolution invertible and such that $\phi
\circ \gamma = \id_H$. This plays the rôle of the section $t$ from \ref{otimes-t}. In that
case, by \cite{Mon}*{Theorems 8.2.4, 8.5.6}, $G \cong H \ot L$ as a right
$H$-comodule and a left $L$-module, and the functor $\Mod_G^H \to \Mod L$ defined by the
assignation $T \mapsto T^{\co H}$ is an equivalence of categories. Thus for any
$(A,H)$-Hopf module $N$, we see that $N \sq_H G \cong N \ot L$ as vector spaces, and we
can pull back an $(A,G)$-Hopf module structure on $N \ot L$, or more generally, on any
$L$-module.

\paragraph
We now fix a projective resolution of $H$ in $\Mod^H_G$, say $P_\bullet \to H$, and apply
the functor $N \sq_H -$, thus obtaining an exact complex of $A$-modules $N \sq_H P_\bullet$. 
This is equivalent to the procedure described in Theorem \ref{injective-dimension}. If $H$
is cocommutative, then these are again $(A,H)$-Hopf modules. 

In order to repeat the proof of the theorem, we would need to prove that if $I$ is an
injective object of $\Mod^G_A$, then $I_\phi \sq_H G$ is again injective. This can be
easily proved if $L$ is isomorphic to a group algebra, which we leave as a pleasant
exercise in homological coalgebra.

\end{document}


\paragraph
As is well known, a $G$-grading over a vector space $V$ is the same as a $\k[G]$-comodule
structure over $V$. A $G$-graded $\k$-algebra is a $\k[G]$-comodule algebra and $G$-graded
$A$-modules are $(k[G],A)$-Hopf modules. If we replace the group algebra $\k[G]$ by a
Hopf algebra $G$, we have a category of $(G,A)$-Hopf modules $\Mod^G_A$. For any
object $M$ in $\Mod^G_A$ write $\rho_M: M \to M \ot G$ for the vector space map giving its
comodule structure.

A group homomorphism $\phi: G \to H$ induces a morphism $\phi:\k[G] \to\k[H]$ in an
obvious way. 
Given two Hopf algebras $G, H$ and a morphism $\phi: G \to H$ between them, we get two
functors $\phi_!: \Mod^G_A \to \Mod_A^H$ and $\phi^*: \Mod^H_A \to \Mod^G_A$, which we
describe in the following paragraphs. The construction is taken from \cite{Doi}.

\paragraph
Let $M$ be an object in $\Mod_k^G$.
The underlying vector space of ${}_!M$ is $M$. Its structure morphism is given by
$$ M \to M \ot G \stackrel{\id \ot \phi}{\to} M \ot H$$
In particular $A$ becomes a comodule algebra over $H$, and if $M$ had the extra structure
of a relative $(A,G)$-module, then ${}_!M$ with its $A$-module structure is a relative
$(A,H)$-module.

Now let $N$ be an object of $\Mod^H_A$. The underlying vector space of ${}^* N$ is
$$ N \square_H G = \ker (N \ot G \stackrel{\id \ot \phi - \rho_N \ot \id}{\to} N \ot H \ot G)$$
and its $A$-module structure is induced by that of $N$. This induces a relative
$(A,G)$-module structure on ${}^* N$. It is immidiate to check that this functors coincide
with the ones defined in \ref{change-of-grading-functors}.

\paragraph
The operation $\square_H$ is called the \emph{cotensor product} of $H$-comodules. It was
studied by Doi in the paper cited above. In Proposition 6. of this paper Doi proves that
$\phi_!$ and $\phi^*$ are adjoints when $A = k$ and $G, H$ are coalgebras. The same proof
works in this context since the isomorphisms he gives preserve $A$-linear morphisms. This
is analogous to Lemma \ref{adjoint}.

\paragraph
The morphism $\phi$ gives us a structure of $H$-comodule over $G$, and the structure of a
$G$-module over $H$. Thus it makes sense to consider $H$ as an object in the category
$\Mod_G^H$. If we put $L = G^H$ the subalgebra of coinvariants of $G$ by the action of $H$
then there is always a functor $\Mod^H_G \to \Mod L$ given by $M \mapsto M^H$. In certain
cases this is an equivalence of categories, for example when $G, H$ are group algebras or
envelopping algebras of Lie algebras. In this cases all Hopf algebra morphisms are induced
by morphisms between the original objects, and the coinvariants correspond to the Hopf
algebras associated to the\kernel of $\phi$.

\paragraph
Since $H^H = \k$, a projective resolution of $\k$ in $\Mod L$ corresponds to a projective
resolution of $H$ in $\Mod^H_G$. In that case the construction $N \ot^t - :
\Mod L \to \Mod^H_A$ is replaced by a \emph{twisted smash product} $N \#_\sigma -:
\Mod^H_G \to \Mod^H_A$ (see \cite{M}*{chapter 8}), which is also additive and exact.

\paragraph
Since we have no general carachterization of ${}^* {}_! I$ for an injective object in
$\Mod^G_A$, the best we can expect is that under the previous hypothesis the injective
dimension of ${}_! I$ is bounded by $\pd_G^H H + \id^G_A {}^*{}_! I$.













