%%%%%%%%%%%%%%%%%%%%%% Generalities %%%%%%%%%%%%%%%%%%5
\documentclass[11pt,fleqn]{article}
\usepackage[paper=a4paper]
  {geometry}

\pagestyle{plain}
\pagenumbering{arabic}
%%%%%%%%%%%%%%%%%%%%%%%%%%%%%%%%
\usepackage[small]{titlesec}
%\usepackage{paragraphs}

\usepackage{hyperref}

\usepackage{amsthm,thmtools}
%\usepackage{showlabels}
%\linespread{1.2}
\setlength{\parskip}{1.2ex}

\usepackage[utf8]{inputenc}
\usepackage[english]{babel}
\usepackage{enumerate}
\usepackage[osf,noBBpl]{mathpazo}
\usepackage[alphabetic,initials]{amsrefs}
\usepackage{amsfonts,amssymb,amsmath}
\usepackage{mathtools}
\usepackage{graphicx}
\usepackage[poly,arrow,curve,matrix]{xy}
\usepackage{wrapfig}
\usepackage{xcolor}
\usepackage{helvet}
\usepackage{stmaryrd}
\usepackage[normalem]{ulem}

\renewcommand\labelitemi{--}


%%%%%%%%%%%%Theorems, for paragraphs package%%%%%%%%%%%%%%%%%%%%%%%%%%
% numbered versions

\newskip\paraskip
\paraskip=0.75ex plus .2ex minus .2ex

\newcounter{para}[section]
\setcounter{para}{0}
\renewcommand\thepara{\thesection.\arabic{para}}
\def\paragraph{%
  \if@noskipsec \leavevmode \fi
  \par
  \if@nobreak
    \everypar{}%
  \else
    \addpenalty{\@secpenalty}%
    \addvspace{\paraskip}%
  \fi
  \noindent
  \refstepcounter{para}%
  \textbf{\thepara.}\hspace{1ex}%
  \@nobreakfalse
  \ignorespaces
}

\newcommand\pref[1]{\textbf{\ref{#1}}}

%\newcommand\theHpara{\theHsection.\arabic{para}}

\declaretheoremstyle[headformat=swapnumber, spaceabove=\paraskip,
bodyfont=\itshape]{mystyle}
\declaretheorem[name=Lemma, sibling=para, style=mystyle]{Lemma}
\declaretheorem[name=Proposition, sibling=para, style=mystyle]{Proposition}
\declaretheorem[name=Theorem, sibling=para, style=mystyle]{Theorem}
\declaretheorem[name=Corollary, sibling=para, style=mystyle]{Corollary}
\declaretheorem[name=Definition, sibling=para, style=mystyle]{Definition}
%\declaretheorem[name=Examples, sibling=para, style=mystyle]{Example}
%\declaretheorem[name=Remark, sibling=para, style=mystyle]{Remark}

% unnumbered versions
\declaretheoremstyle[numbered=no, spaceabove=\paraskip,
bodyfont=\itshape]{mystyle-empty}
\declaretheorem[name=Lemma, style=mystyle-empty]{Lemma*}
\declaretheorem[name=Proposition, style=mystyle-empty]{Proposition*}
\declaretheorem[name=Theorem, style=mystyle-empty]{Theorem*}
\declaretheorem[name=Corollary, style=mystyle-empty]{Corollary*}
\declaretheorem[name=Definition, style=mystyle-empty]{Definition*}
\declaretheorem[name=Examples, style=mystyle-empty]{Example*}
\declaretheorem[name=Remark, style=mystyle-empty]{Remark*}

% plain style
\declaretheoremstyle[
        headformat={{\bfseries\NUMBER.}{ \bfseries\NAME}},
        spaceabove=\paraskip, 
        headpunct={. },
        headfont=\bfseries,
        bodyfont=\normalfont
        ]{mystyle-plain}
\declaretheorem[name=Example, sibling=para, style=mystyle-plain]{Example}
\declaretheorem[name=Remark, sibling=para, style=mystyle-plain]{Remark}

\makeatletter
\renewenvironment{proof}[1][\textit{Proof}]{\par
  \pushQED{\qed}%
  \normalfont \topsep.75\paraskip\relax
  \trivlist
  \item[\hskip\labelsep
        \itshape
    #1\@addpunct{.}]\ignorespaces
}{%
  \popQED\endtrivlist\@endpefalse
}
\makeatother

\usepackage{tikz}
\usepackage{mathdots}
%%%%%%%%%%%%%%%%%%%%%%%%%%% The usual stuff%%%%%%%%%%%%%%%%%%%%%%%%%
\newcommand\NN{\mathbb N}
\newcommand\CC{\mathbb C}
\newcommand\QQ{\mathbb Q}
\newcommand\RR{\mathbb R}
\newcommand\ZZ{\mathbb Z}
\renewcommand\k{\Bbbk}

\newcommand\F{\mathcal F}
\newcommand\V{\mathcal V}
\newcommand\D{\overline D}
\newcommand\N{\mathcal N}
\renewcommand\H{\mathcal H}

\newcommand\maps{\longmapsto}
\newcommand\ot{\otimes}
\renewcommand\to{\longrightarrow}
\renewcommand\phi{\varphi}
\newcommand\id{\mathsf{id}}
\newcommand\im{\mathsf{im}}
\newcommand\coker{\mathsf{coker}}
%%%%%%%%%%%%%%%%%%%%%%%%% Specific notation %%%%%%%%%%%%%%%%%%%%%%%%%
\newcommand\g{\mathfrak g}
\newcommand\p{\mathfrak p}
\newcommand\m{\mathfrak m}
\newcommand\gl{\mathfrak{gl}}
\newcommand\gen{\mathsf{gen}}
\newcommand\std{\mathsf{std}}
\newcommand\sh{\mathsf{sh}}

\newcommand\vectspan[1]{\left\langle #1 \right\rangle}
\newcommand\interval[1]{\llbracket #1 \rrbracket}
\newcommand\Shuffle{\mathsf{Shuffle}}

\DeclareMathOperator\Frac{Frac}
\DeclareMathOperator\Specm{Specm}

\DeclareMathOperator\sym{sym}
\DeclareMathOperator\asym{asym}
\DeclareMathOperator\sg{sg}
%%%%%%%%%%%%%%%%%%%%%%%%%%%%%%%%%%%%%% TITLES %%%%%%%%%%%%%%%%%%%%%%%%%%%%%%
\title{Gelfand-Tsetlin modules over $\gl(n)$ with arbitrary characters}

\author{L.E. Ram\'irez\footnote{Universidade Federal do ABC, Santo Andr\'e-SP, 
Brasil\texttt{email:} luis.enrique@ufabc.edu.br} , 
P. Zadunaisky\footnote{Instituto de Matem\'atica e Estat\'istica, Universidade 
de S\~ao Paulo,  S\~ao Paulo SP, Brasil. \texttt{email:} pzadun@ime.usp.br.
The author is a FAPESP PostDoc Fellow, grant: 2016-25984-1 
S\~ao Paulo Research Foundation (FAPESP).}
}


\begin{document}
\maketitle

\begin{abstract}
A Gelfand-Tsetlin tableau $v$ induces a character $\chi_v$ of the 
Gelfand-Tsetlin subalgebra $\Gamma$ of $U(\gl(n,\CC))$. We show that for any 
tableaux there exists a Gelfand-Tsetlin module $V$ whose support is the set of
all characters of the form $\chi_{v+z}$, where $z$ is an integral tableau with 
top row $0$. We construct an explicit basis of $V$ and find the multiplicty of 
each character $\chi_{v+z}$ in terms of combinatorial invariants of the 
associated Gelfand-Tsetlin tableaux.
\end{abstract}
\noindent\textbf{MSC 2010 Classification:} 17B10.\\
\noindent\textbf{Keywords:} Gelfand-Tsetlin modules, Gelfand-Tsetlin bases,
tableaux realization.

\section{Introduction}
The notion of a Gelfand-Tsetlin module (see Definition \ref{D:gt-modules}) has 
its origin in the classical article \cite{GT-modules}, where I. Gelfand and M. 
Tsetlin gave a presentation of all the finite dimensional irreducible 
representations of $\g = \gl(n,\CC)$ in terms of certain tableaux, which have 
come to be known as \emph{Gelfand-Tsetlin tableaux}, or GT-tableaux for short. 
A GT-tableau is a triangular array of $\binom{n}{2}$ complex numbers, with $k$ 
entries in the $k$-th row; given a point $v \in \CC^{\binom{n}{2}}$ we denote 
the corresponding array by $T(v)$. The group $G = S_1 \times S_2 \times \cdots 
\times S_n$ acts on the set of all tableaux, with $S_k$ permuting the elements 
in the $k$-th row. Any finite dimensional irreducible representation of $\g$ 
has a basis parametrized by tableaux with integer coefficients satisfying 
certain relations. Identifying the elements of the basis with the 
corresponding tableaux, the action of an element of $\g$ over a 
tableau is given by rational functions in its entries. The coefficients of
this action are known as the Gelfand-Tsetlin formulas; their poles form an 
infinite hyperplane array in $\CC^{\binom{n}{2}}$.

The enveloping algebra $U = U(\g)$ contains a large (indeed, maximal) 
commutative subalgebra $\Gamma$ called the \emph{Gelfand-Tsetlin} subalgebra 
of $U$. A \emph{Gelfand-Tsetlin module} is a $U$-module that can be
decomposed as the direct sum of generalized eigenvector spaces for $\Gamma$.
The characters of $\Gamma$ are in one-to-one correspondence with tableaux 
modulo the action of $G$, and in the original construction of Gelfand and 
Tsetlin each tableaux $T(v)$ is an eigenvector of $\Gamma$ whose eigenvalue
is precisely the character $\chi_v: \Gamma \to \CC$ corresponding to $v$. 
Since no two tableaux in this construction are in the same $G$-orbit, the 
multiplicity of this character (i.e. the number of eigenvectors
of eigenvalue $\chi_v$) is one.

Starting from these observations, Y. Drozd, S. Ovsienko and V. Futorny 
introduced a large family of infinite dimensional $\g$-modules in 
\cite{DFO-GT-modules}. These are GT-modules have a basis parametrized by 
Gelfand-Tsetlin tableaux with complex coefficients such that no pattern is a 
pole for the rational functions appearing in the GT-formulae (such tableaux 
are called \emph{generic}, hence the name ``generic Gelfand-Tsetlin module''). 
While each character in the decomposition of a generic Gelfand-Tsetlin module
appears with multiplicity one, there are examples of non-generic GT-modules
with higher multiplicities. These examples were first encountered in 

In \cite{FGR-1-singular} V. Futorny, D. Grantcharov and the first named author
constructed a GT-module with $1$-singular characters, i.e. characters 
associated to tableaux over which the Gelfand-Tsetlin formulas may have 
singularities of order at most $1$. These modules have a basis in terms of 
so-called \emph{derived tableaux}, new objects which, according to the authors 
``are not new combinatorial objects'' but rather formal objects in a large
vector space that contains classical GT-tableaux. These construction was 
expanded and refined in the articles \cites{FGR-2-index, Zad-1-sing, 
V-geometric-singular-GT} for characters with more general singularities. The 
aim of this article is to extend this construction to \emph{arbitrary} 
characters and begin the study of the associated modules. In the process we 
give a combinatorial interpretation of generalized derived tableaux.

The general idea of our construction is the following. Denote by $V$ the vector
space of arbitrary integral GT-tableaux with coefficients in the field of 
rational functions over (the entries of) GT-tableaux; this is a $U$-module 
with the action of $\g$ given by the rational functions appearing in the 
Gelfand-Tsetlin formulas. These rational functions lie in the algebra $A$ of 
regular functions over generic tableaux, and hence the $A$-lattice $V_A$ whose 
$A$-basis is the set of all integral tableaux is a $U$-submodule of $V$; now 
given a generic tableau $v$, we can recover the corresponding generic module 
by specializing $V_A$ at $v$.

This idea breaks down if $v$ is a singular tableaux, and in that case we must
replace $A$ with an algebra $B \subset K$ such that $(1)$ evaluation at $v$ 
makes sense and $(2)$ there exists a $B$-lattice $L_B \subset V$ which is also
a $U$-submodule. It turns out that these conditions can be met, eventually by
replacing $v$ by $v+z$, where $z$ is an adequate tableaux with integral 
entries. Specializing $L_B$ at $v+z$ we obtain a GT-module which includes 
the character $\chi_v$ in its support.

Denote by $\overline \mu$ the partition of $\binom{n}{2}$ given by 
$(1,2,\ldots, n)$. Each point $v \in \CC^{\binom{n}{2}}$, or rather its class 
modulo $G$, defines a refinement $\eta(v)$ of $\mu$, and it turns out that the 
structure of the associated GT-module $V(T(v))$ deppends heavily on $\eta(v)$. 
For example, the multiplicity of every character of $V(T(v))$ is a divisor of 
$\eta(v)!$, and for most characters it is in fact equal to $\eta(v)!$. While 
it is possible to choose an algebra $B$ and a lattice $L_B$ that works for all 
$v$ simultaneously, it turns out to be more productive to fix a refinement 
$\eta$ and focus on characters with $\eta(v) = \eta$, thus obtaining an algebra
$B_\eta$ and a $B_\eta$-lattice $L_\eta$. Each $L_\eta$ has a basis of derived
tableaux, and changing $\eta$ changes this basis in an essential way.

While finishing this paper the article \cite{V-geometric-singular-GT} by 
E. Vishnyakova was uploaded to the ArXiv. The article gives the construction 
of $p$-singular GT-modules, where $p \in \NN_{\geq 2}$. This is a special 
class of singular modules, associated to classes $v \in \CC^{\binom{n}{2}}
/G$ where the partition $\eta(v)$ has only one nontrivial part, which is equal 
to $p$. Vishnyakova's argument is simmilar to ours in that she finds a large 
generic GT-module that reduces to a singular module by specialization, the 
main difference being that in her case the large module is constructed by a 
geometric argument, and differential operators play the role of divided 
differences. 

\bigskip

The article is organized as follows. In section \ref{preeliminaries} we 
set the notation used for the combinatorial invariants associated to tableaux.
In section \ref{modules-of-divided-differences} we develop a general framework 
for the study of actions by rational functions, and the use of symmetrization 
and divided differences operators to remove singularities. Finally section 
\ref{GT-arbitrary-characters} contains the main results of the article, giving
an explicit construction of singular GT-modules associated to arbitrary 
characters, along with explicit bases for the character spaces. 

\section{Preeliminaries on compositions and refinements}
\label{preeliminaries}
Let $n,m \in \NN$. We write $\interval{n,m} = \{k \in \NN \mid n \leq k \leq 
m\}$ and $\interval{n} = \interval{1,n}$. We denote by $S_n$ the symmetric 
group on $n$ elements, and set
\begin{align*}
\sym_n 
  &= \frac{1}{n!}\sum_{\sigma \in S_n} \sigma;
&\asym_n
  &= \frac{1}{n!} \sum_{\sigma \in S_n} \sg(\sigma) \sigma.
\end{align*}
These are idempotent elements of the group algebra $\CC[S_n]$ and given an 
$\CC[S_n]$-module $V$, multiplication by $\sym_n$, resp. $\asym_n$, is the 
projection onto the symmetric, resp. antisymmetric, component of $V$. 

Recall that for each $\sigma \in S_n$ the length of $\sigma$, denoted by 
$\ell(\sigma)$, is the number of inversions of $\sigma$, i.e. the number
of pairs $(i,j) \in \interval{n}^2$ such that $i<j$ but $\sigma(i) > 
\sigma(j)$. The length of a word is bounded by $\binom{n}{2}$, and there is 
exactly one word $w_0 \in S_n$ of maximal length. Also $\ell(\sigma) = 
\ell(\sigma^{-1})$, and $\ell(\sigma^{-1} w_0) = \ell(w_0) - \ell(\sigma)$. 
For each $i \in \interval{n-1}$ set $s_i = (i,i+1) \in S_n$. Then 
$\ell(\sigma)$ is the minimal number $l$ such that $\sigma$ can be 
written as $s_{i_1} s_{i_2} \cdots s_{i_l}$ with each $s_{i_j}$ a simple 
transposition; any such sequence $s_{i_1}, \ldots, s_{i_l}$ is called a 
reduced decomposition of $\sigma$. 

Let $\mu = (\mu_1, \ldots, \mu_r)$ be a composition of $n$, i.e. an $r$-uple 
of strictly postive integers whose sum is $n$. Each $\mu_k$ is called a 
\emph{part} of $\mu$. For each $k \in \interval{r}$ set $\alpha_k(\mu) = 
\alpha_k = \sum_{j=1}^{k-1} \mu_k$ and $\beta_k = \beta_k(\mu) = 
-1+\sum_{j=1}^{k+1} \mu_k$, so the interval $\interval{\alpha_k, \beta_k}$ has 
$\mu_k$ elements; we refer to this interval as the $k$-th block of $\mu$. 
Denote by $S^{(k)}$ the group of bijections of $\interval{\alpha_k, \beta_k}$, 
which is isomorphic to $S_{\mu_k}$, and set $S_\mu = S^{(1)} \times S^{(2)} 
\times \cdots \times S^{(r)} \subset S_n$. By definition $\sigma \in S_\mu$ 
is a product of the form $\sigma = \sigma^{(1)} \sigma^{(2)} \cdots 
\sigma^{(r)}$ with each $\sigma^{(j)} \in S^{(j)}$. The length 
of an element on $S_\mu$ is $\ell(\sigma) = \ell(\sigma^{(1)}) + \cdots + 
\ell(\sigma^{(r)})$, and $S_\mu$ has a unique longest word $w_\mu$ with 
$w_\mu^{(k)}$ the longest word of $S^{(k)}$. We set $\mu! = \mu_1! \mu_2! 
\cdots \mu_r!$, so $\mu!$ is the cardinal of $S_\mu$.

Denote by $\sym^{(k)}, \asym^{(k)}$ the symmetrizer and antisymmetrizer of the
group $S^{(k)}$, and let $\sym_\mu = \prod_{k=1}^r \sym^{(k)}$ and $\asym_\mu
= \prod_{k=1}^r \asym^{(k)}$. As before these are idempotent elements in the
group algebra $\CC[S_\mu]$ and for each $S_\mu$-module $V$, multiplication by
$\sym_\mu$, resp. $\asym_\mu$, is the projection to the symmetric, resp. 
antisymmetric component of $V$. Notice that $\sym^{(k)}$ commutes with 
$\sym^{(j)}$ and $\asym^{(j)}$ if $j \neq k$.

We say that $\sigma \in S_n$ is a $\mu$-shuffle if it is increasing in each 
interval $\interval{\alpha_k, \beta_k}$. Among the elements of a coclass 
$\sigma S_\mu \in S_n/S_\mu$ there is exactly one $\mu$-shuffle, and this is 
the unique element of minimal length in the coclass, since any other element
in it would invert as many elements as the shuffle plus some pair of 
elements in the same $\mu$-block. We denote by $j_\mu: S_n/S_\mu \to S_n$
the function that assigns to each coclass the unique $\mu$-shuffle in it. 
We denote the set of all $\mu$-shuffles by $\Shuffle_\mu$.

A \emph{refinement} of $\mu$ is a collection of compositions $\eta = 
(\eta^{(1)},\ldots, \eta^{(r)})$ with each $\eta^{(k)}$ a composition of 
$\mu_k$. If $\eta$ is a refinement of $\mu$ then the concatenation of the 
$\eta^{(k)}$'s is also a composition of $n$, which by abuse of notation we 
will also denote by $\eta$. A partition of $\mu$ is an ordered collection of 
$r$ partitions, with the $r$-th element in the collection a partition of 
$\mu$. Each refinement $\eta$ of $\mu$ induces a partition by forgetting the 
order of the parts of $\eta^{(k)}$.

If $\eta$ refines $\mu$ then $S_\eta \subset S_\mu$. We denote by $V_\eta^\mu$
the representation of $S_\mu$ obtained from the trivial representation of 
$S_\eta$. This is a representation of dimension $\mu!/\eta!$, and there exists
an element $v_\eta \in V_\eta^\mu$, unique up to a scalar multiple, that
generates $V_\eta^\mu$ as $\CC[S_\mu]$-module and whose stabilizer is $S_\eta$.
Each coclass of $S_\mu / S_\eta$ contains exactly one $\eta$-shuffle, and we
set $\Shuffle^\mu_\eta = S_\mu \cap \Shuffle_\eta$.

We write $\CC^{\mu} = \CC^{\mu_1} \oplus \CC^{\mu_2} \oplus \cdots 
\CC^{\mu_r}$; thus $v \in \CC^{\mu}$ is an $r$-uple of vectors $(v_1, \ldots,
v_r)$, with $v_k \in \CC^{\mu_k}$; we refer to the elements of $\CC^\mu$ as 
\emph{$\mu$-points}, or simply points if the partition $\mu$ is fixed. 
For each $k \in \interval{r}$ and each $i \in \interval{\mu_k}$ we write 
$v_{k,i}$ for the $i$-th coordinate of $v_k$. We refer to the $v_k$'s as the 
\emph{$\mu$-blocks} of $v$, and to the $v_{k,i}$'s as the \emph{entries} of 
$v$. We say that a $\mu$-point $v$ is \emph{integral} if all its entries lie 
in $\ZZ$. If $\eta$ is a refinement of $\mu$ then there exists an isomorphism 
$\CC^{\eta^{(k)}} \cong \CC^{\mu_k}$ which induces in turn an isomorphism 
$\CC^\eta \cong \CC^\nu$. Thus we can talk of the $\eta$-blocks of $v \in 
\CC^\mu$. The group $S_\mu$ acts on $\CC^\mu$ with $S^{(k)}$ acting on 
$\CC^{\mu_k}$. Since $\eta$ refines $\mu$ we have $S_\eta \subset S_\mu$ and 
so $S_\eta$ acts on $\CC^\mu$. The isomorphism $\CC^\eta \cong \CC^\mu$ is 
$S_\eta$ equivariant

\section{Divided-difference modules and symmetrization}
\label{modules-of-divided-differences}

\subsection{Symmetric group and polynomials}
In this subsection we recall some classical facts on the action of a symmetric
group on the algebra of polynomials.

Let $n \in \NN$, and let $\CC[X] = \CC[x_1, \ldots, x_n]$ with the usual 
grading. The symmetric group $S_n$ acts on $\CC[X]$ by permuting the 
variables, and each homogeneous component is preserved by this action. By 
classical results there exists a graded subspace $H \subset \CC[X]$ such that 
the map $\CC[X]^{S_n} \ot H \to \CC[X]$ induced by multiplication is an 
isomorphism. If we denote by $J$ the ideal of $\CC[X]$ generated by symmetric 
polynomials of positive degree then $\CC[X]/J$ is isomorphic to the regular 
representation of $S_n$, and $H \cong \CC[X]/J$ as graded vector spaces.
Thus, while there are many possible choices for $H$, the Hilbert series of the 
graded module $H$ is independent of the choice and equal to $\prod_{i=1}^n
\frac{1-t^i}{1-t}$. See for example \cite{OT-arrangements-book}*{Theorem 6.19}.

Let $\Delta = \prod_{1 \leq i < j \leq n} (x_i - x_j) \in \CC[X]$, usually 
called the \emph{Vandermonde determinant} of $\CC[X]$. For each 
$\sigma \in S_n$ we have $\sigma(\Delta) = \sg(\sigma) \Delta$, and every 
polynomial $f \in \CC[X]$ such that $\sigma f = \sg(\sigma) f$ is of the form 
$g \Delta$, with $g \in \CC[X]^{S_n}$. Thus given $f \in \CC[X]$ the polynomial
$\asym_n(f)$ is divisible by $\Delta$, and $\frac{\asym_n(f)}{\Delta} = 
\sym_n\left(\frac{f}{\Delta}\right)$ is a symmetric polynomial.

Recall that for each $i \in \interval{n-1}$ we have set $s_i = (i,i+1) \in 
S_n$. The divided difference $\partial_{i}$ of a polynomial is defined as
\begin{align*}
\partial_{i}(f) 
  &= \frac{f - s_i(f)}{x_i - x_{i+1}}.
\end{align*}
We now review some basic facts regarding these operators. The reader can find proofs for these claims in \cite{Man-symm-book}*{Section 2.3}.

It follows from the definition that $\partial_{i}(fg) = \partial_{i}(f) g + 
s_i(f) \partial_{i}(g)$. Thus $\partial_{i}$ is an $s_i$-derivation 
of degree $-1$. The kernel and cokernel of $\partial_{i}$ are both equal to 
the space of symmetric polynomials in $x_i, x_{i+1}$, and it follows that
$\partial_i$ is $\CC[X]^{S_n}$-linear.

The divided differences $\{\partial_{i} \mid 1 \leq i \leq n-1\}$ satisfy the 
braid relations, so it makes sense to define for each $\sigma \in S_n$ the 
divided difference $\partial_\sigma$ to be the composition of 
divided differences $\partial_{i_1} \circ \partial_{i_2} \circ \cdots \circ 
\partial_{i_{\ell(\sigma)}}$ with $s_{i_1}, s_{i_2}, \ldots 
s_{i_{\ell(\sigma)}}$ a reduced decomposition of $\sigma$. Thus 
$\partial_\sigma$ is a homogeneous $\CC[X]^{S_n}$-linear
operator of degree $-\ell(\sigma)$. In the special case of the longest word 
we have  
\begin{align*}
\partial_{w_0}(f) &= \frac{n!}{\Delta} \asym_n(f) 
  = n! \sym_n \left( \frac{f}{\Delta}\right).
\end{align*}
Also, $\partial_{\sigma} \circ \partial_{\tau} = \partial_{\sigma \tau}$ if and
only if $\ell(\sigma) + \ell(\tau) = \ell(\sigma\tau)$, otherwise the 
composition is zero. It follows that
\begin{align*}
0 
  &= \partial_{w_0}(\partial_{i}(fg)) 
  = \partial_{w_0}(\partial_{i}(f)g)  + 
      \partial_{w_0}(s_i(f) \partial_{i}(g)) \\
  &= \partial_{w_0}(\partial_{i}(f)g) -
      \partial_{w_0}(f s_i(\partial_{i}(g))) 
  = \partial_{w_0}(\partial_{i}(f)g) -
      \partial_{w_0}(f \partial_{i}(g)).
\end{align*}
From this and an inductive argument we get that 
$\partial_{w_0}(\partial_\sigma(f) g) = \partial_{w_0}(f 
\partial_{\sigma^{-1}}(g))$. We will often use the fact that
$\partial_{w_0}(f \partial_\sigma \Delta) = \partial_{w_0} (\Delta 
\partial_{\sigma^{-1}} (f)) = n!\sym_n(\partial_{\sigma^{-1}}(f))$.

For each $i \in \interval{n-1}$ set $d_i = x_i - x_{i+1}$. We denote by 
$\p$ the ideal generated by $\{d_i \mid i \in \interval{n-1}\}$, which is 
clearly a prime ideal, and write $\CC[D]$ for the algebra generated by the 
same set. Notice that $\CC[D]$ is stable by the action of $S_n$ and that any 
polynomial of positive degree in $\CC[D]$ lies in $\p$. A simple calculation 
shows that $\partial_i(d_j) \in \{0,\pm 1,2\}$ for all $i,j \in 
\interval{n-1}$, and by the twisted Leibniz rule the algebra $\CC[D]$ is 
stable by the action of divided differences. Now $\Delta = \prod_{1 \leq i < j 
\leq n}(d_i + d_{i+1} + \cdots + d_{j-1}) \in \CC[D]$, and hence 
$\partial_\sigma \Delta \in \CC[D]$ for each $\sigma \in S_n$. Now
$\partial_{w_0 \sigma^{-1}} \partial_\sigma \Delta = \partial_{w_0} \Delta = 1$
so $\partial_\sigma \Delta \in \p \setminus \{0\}$ whenever $\sigma 
\neq w_0$. Since $w_0$ is the longest word, this is equivalent to the statement
that $\partial_\sigma \Delta \equiv 0 \mod \p$ whenever $\ell(\sigma) < 
\ell(w_0)$.

\subsection{Multisymmetric actions}
Let $\mu = (\mu_1, \ldots, \mu_r)$ be a partition of $n$. For each $k \in 
\interval r$ we write $\CC[\mu]_k = \CC[\lambda_{k,j} \mid j \in \interval 
\mu_k$ and $\CC[\mu] = \CC[\mu]_1 \ot \CC[\mu]_2 \ot \cdots \ot \CC[\mu]_r$. 
Thus we can identify $\CC[\mu]$ with the symmetric algebra of $(\CC^\mu)^*$, 
and $\CC[\mu]_k$ with the symmetric algebra of $(\CC^{\mu_k})^*$, and hence we 
sometimes refer to elements of $\CC[\mu]$ as polynomial functions on 
$\mu$-points. We also set $\CC(\mu) = \Frac \CC[\mu]$, the field of rational
functions on $\mu$-points.

The action of $S_\mu$ on $\CC^\mu$ induces an action of $S_\mu$ on $\CC[\mu]$, 
with each $\CC[\mu]_k$ an $S_\mu$-subalgebra. We denote by $\p^{(k)} \subset 
\CC[\mu]_k$ the ideal generated by $\{\lambda_{k,i} - \lambda_{k,j} \mid 
j \in \interval \mu_k\}$, and by $\Delta_k$ the Vandermonde determinant in 
$\CC[\mu]_k$. We also set $\p \subset \CC[\mu]$ to be the ideal generated by 
the union of the $\p^{(k)}$'s, and $\Delta_\mu = \prod_{k=1}^r \Delta_k$. The 
zero set of $\p^{(k)}$ consists of $\mu$-points whose $k$-th block is of the 
form $(a,a, \ldots, a)$ for some $a \in \CC$, and the zero set of $\p$ is the 
set of $\mu$-points such that each $\mu$-block is of this form. 

Since $S_\mu \subset S_n$, for each $\sigma \in S_\mu$ we have an divided 
difference operator $\partial_\sigma$ that equals $\partial_{\sigma^{(1)}} 
\cdots \partial_{\sigma^{(r)}}$; to simplify notation we write 
$\partial_\sigma^{(k)}$ instead of $\partial_{\sigma^{(k)}}$. It follows that 
$\partial_\sigma^{(k)}$ is an endomorphism of $\CC[\mu]_k$ of degree 
$-\ell(\sigma^{(k)})$ that it acts as zero on $\CC[\mu]_j$ for $j \neq k$. 
Also $\partial_\sigma$ is a $\CC[\mu]^{S_\mu}$-linear endomorphism of 
$\CC[\mu]$ of degree $-\ell(\sigma)$. In particular $\partial_\sigma 
\Delta_\mu = \prod_{k=1}^r \partial_{\sigma}^{(k)} \Delta_k$, and so 
\begin{align*}
\partial_{w_\mu}(\partial_\sigma(f) g) 
  &= \prod_{k=1}^r \partial_{w^{(k)}} (\partial_{\sigma}^{(k)}(f), g)
  = \prod_{k=1}^r \partial_{w^{(k)}} (f \partial_{\sigma^{-1}}^{(k)}(g)) 
  = \partial_{w_\mu} (f \partial_{\sigma^{-1}}(g)). 
\end{align*}

The following result will be crucial in the sequel.
\begin{Proposition}
\label{P:multidivided-delta-basis}
For each $\sigma \in S_\mu$ there exists $c_\sigma \in \p^{S_\mu}$ such that 
for each $f \in \CC[\mu]$ we have 
\begin{align*}
f 
  &= \sum_{\sigma \in S_\mu} \sym_\mu(\partial_{\sigma} f) 
    (\partial_{\sigma^{-1}w_\mu} \Delta_\mu + c_{\sigma^{-1}w_\mu})
\end{align*}
In particular the set $\{\partial_\sigma \Delta_\mu \mid \sigma \in S_\mu\}$ 
is a basis of $\CC[\mu]$ as $\CC[\mu]^{S_\mu}$-module.
\end{Proposition}
\begin{proof}
Fix a total order $\leq$ on $S_\mu$ such that $\ell(\sigma) \leq \ell(\tau)$ 
implies $\sigma \leq \tau$. We can then label the rows and columns of $\mu! 
\times \mu!$ matrices with coefficients in $\CC(\mu)$ by elements of $S_\mu$. 
Given $T \in M_{\mu!}(\CC(\mu))$, we denote by $T_\sigma^\tau$ the element in 
the $\tau$-th row and the $\sigma$-th column.

Consider the matrices $X,Y \in M_{\mu!}(\CC(\mu))$ defined by $X^\sigma_\tau = 
\tau \left( \frac{\partial_\sigma \Delta_\mu}{\Delta_\mu} \right)$ and 
$Y^\rho_\nu = \frac{\rho(\partial_{\nu w_\mu} \Delta_\mu)}{\mu!}$. Then
\begin{align*}
(XY)_\nu^\sigma 
  &= \sum_{\tau \in S_\mu} \tau \left( \frac{\partial_\sigma \Delta_\mu
  \partial_{\nu w_\mu} \Delta_\mu }{n!\Delta_\mu} \right)
  = \frac{1}{n!}
    \partial_{w_\mu} (\partial_\sigma \Delta_\mu \partial_{\nu w_\mu} 
      \Delta_\mu)\\
  &= \frac{1}{n!}
    \partial_{w_\mu}(\Delta_\mu \partial_{\sigma^{-1}} \partial_{\nu w_\mu} 
      \Delta_\mu)
  = \sym_\mu(\partial_{\sigma^{-1}} \partial_{\nu w_\mu} \Delta_\mu)
\end{align*}
Now $\ell(\sigma^{-1}) + \ell(\nu w_\mu) = \ell(\sigma) + \ell(w_\mu) - 
\ell(\nu)$. If $\ell(\sigma) \geq \ell(\nu)$ then $\partial_{\sigma^{-1}} 
\partial_{\nu w_\mu} = 0$ unless equality holds and $\sigma^{-1}\nu w_\mu = 
w_\mu$, i.e. $\sigma = \nu$, in which case the composition equals 
$\partial_{w_\mu}$. Thus $XY$ is an upper triangular matrix with 
$(XY)^\nu_\nu = \frac{1}{\mu!}\partial_{w_\mu}(\Delta_\mu) = 1$ for each $\nu 
\in S_\mu$; furthermore, the nonzero elements in the upper-triangular part are 
of the form $\sym_\mu(\partial_\tau \Delta_\mu)$, with $\ell(\tau) < 
\ell(w_\mu)$, and hence lie in $\p^{S_\mu}$. From this we deduce that $X$ is 
invertible, and its inverse is of the form $Y + C$ with $C$ a matrix with 
entries in $\p^{S_\mu}$. 

Set $c_\sigma = C_\sigma^e$, where $e \in S_\mu$ is the identity.
Let $f \in \CC[X]$ and let $\mathbf f$ be the column vector $\mathbf f^\sigma
= \sigma(f)$. Now since $Y+C$ is the inverse of $X$ we get that $f = \mathbf
f^e = \sum_{\sigma \in S_\mu} (Y+C)_{\sigma^{-1}}^e 
(X \mathbf f)^{\sigma^{-1}}$. This is the same as the formula in the 
statement.
\end{proof}

\subsection{Modules of divided differences}
We keep the notation from the previous section. In particular $\mu$ is a 
partition of $n \in \NN$, and we have polynomial algebras $\CC[\mu], 
\CC[\mu]_k$, and ideals $\p_k \subset \CC[\mu]_k$, and $\p \subset 
\CC[\mu]$ generated by the union of the $\p_k$'s. We set $H$ to be the vector 
space generated by the set $\{\partial_\sigma \Delta_\mu \mid \sigma \in S_\mu
\}$. It follows that the map $\CC[\mu]^{S_\mu} \ot H \to \CC[\mu]$ induced by 
multiplication is an isomorphism of $\CC[\mu]^{S_\mu}$-modules.

Let $T \subset \CC[\mu]$ be a multiplicatively closed set with $T \cap \p = 
\emptyset$ and $\sigma(T) = T$ for all $\sigma \in S_\mu$. For the rest of this
section, we set $B = T^{-1} \CC[\mu]$ and $A = B[\Delta^{-1}]$. We set 
$\p^{(k)}_B = B\p^{(k)}$, the extension of the ideal $\p^{(k)} \subset 
\CC[\mu]_k$ to $B$, and $\p_B = B\p$ the extension of $\p$ to $B$. Since $T$ 
is $S_\mu$-invariant we can take $\overline T = T^{S_\mu}$ and $B \cong 
\overline T^{-1} \CC[\mu]^{S_k} \ot H$, where the map is induced by 
multiplication.

Let $V$ be an $A$-module with an equivariant $S_\mu$-action. Then it makes 
sense to apply divided differences to an element $v \in V$, by setting
\begin{align*}
\partial_{i}(v) = \frac{1}{\lambda_{k,i} - \lambda_{k,i+1}} (v - s_i(v)).
\end{align*}
for each $i \in \interval{\alpha^{(k)}(\mu), \beta^{(k)}(\mu) -1}$. These 
operators are analogous to the divided differences of polynomials, in 
particular they satisfy the braid relations, and the image and kernel of 
$\partial_{i}$ are equal to the subspace of invariant elements with respect to 
the action of the basic transposition $s_i$. Thus the same proof as in the 
case of polynomials shows that $\partial_{i}(fv) = \partial_{i}(f)v + s_i(f) 
\partial_{i}(v)$, and that $\partial_{w_\mu}(\partial_\sigma(f) v)= 
\partial_{w_\mu}(f \partial_{\sigma^{-1}}(v))$.

Since $T$ is $S_\mu$-invariant, any $f \in B$ can be written as $p/q$ with $q 
\in \CC[\mu]^{S_\mu}$, so $\partial_\sigma f = \partial_\sigma(p)/q \in B$
for any $\sigma \in S_\mu$. It follows that Proposition 
\ref{P:multidivided-delta-basis} extends to $B$, so
\begin{align*}
f \equiv \sum_{\sigma \in S_\mu} 
\sym_\mu(\partial_{\sigma} f) \partial_{\sigma^{-1} w_\mu} \Delta \mod 
\p_B^{S_\mu}
\end{align*}
for all $f \in B$. In particular $\partial_{w_\mu} f = \mu! \sym_\mu\left( 
\frac{f}{\Delta}\right) \in B^{S_\mu}$. 

As we mentioned before, the set of zeros of $\p$ consists of all $\mu$-points 
such that the $k$-th $\mu$-block is of the form $(a_k, a_k, \ldots, a_k)$ for
some $a_k \in \CC$. Hence if $a \in Z(\p)$
\begin{align*}
\sym_\mu(f)(a) 
  &= \frac{1}{\mu!} \sum_{\sigma \in S_\mu} f(\sigma^{-1}(a)) 
  = f(a)
\end{align*}
and so $\sym_k(f) \equiv f\mod \p$. Now the rational function $f/\Delta_\mu$ 
can not be evaluated at $a$, but its symmetrization can since it lies in $B$. 
Thus symmetrization can be seen as an operator that smooths out singularities 
of order $1$ lying in the hyperplane arrangement $Z(\p)$. This idea will recur 
through the rest of this article and inspires the following definition.

\begin{Definition}
Let $V$ be an $S_\mu$-module with an $S_\mu$-equivariant $B$-module structure.
We denote by $L_B(V)$ the $B$-submodule of $A \ot_B W$ generated by 
$\sym_\mu\left(\frac{1}{\Delta} V \right)$.
\end{Definition}
Suppose $W$ is an $S_\mu$-module with an equivariant $A$-module structure
and that $V \subset W$ is a $B$-submodule stable by the action of $S_\mu$.
Then the multiplication map $A \ot_B V \to W$ is injective and we identify
$A \ot_B V$ with its image, and $L_B(V) \subset A \ot_B V$ with its image 
inside $W$. 

\begin{Definition}
Let $V$ be a $B$-module with en equivariant $S_\mu$ action, and let $W = A 
\ot_B V$. For each $\sigma \in S_\mu$ and each $v \in V$ we set 
$D^\mu_\sigma(v) = \sym_\mu (\partial_\sigma v) \in W$.
\end{Definition}
Notice that in particular the definition applies to $B$, that $D^\mu_\sigma(B)
\subset B$, and that we can rewrite the formulas of Proposition 
\ref{P:multidivided-delta-basis} as 
\begin{align*}
f \equiv \sum_{\sigma \in S_\mu} D^\mu_\sigma (f) 
  \partial_{\sigma^{-1} w_\mu} \Delta_\mu \mod p^{S_\mu}_B.
\end{align*}
We show in the following lemma that this formula can be extended to $L_B(V)$.
\newcommand\W{\mathcal W}
\begin{Lemma}
\label{L:dds-generators}
Let $V$ be a $B$-module with an equivariant $S_\mu$ action. 
\begin{enumerate}[(a)]
\item 
\label{item:divided}
If $v \in V$ then $D^\mu_\sigma(v) \in L_B(V)$ for all $\sigma 
\in S_\mu$, and furthermore
\begin{align*}
v
  &=\sum_{\sigma \in S_\mu} (\partial_{\sigma^{-1}w_\mu}\Delta_\mu 
    + c_{\sigma^{-1}w_\mu}) D^\mu_\sigma(v),
\end{align*}
where the $c_\tau$ are the same as in Proposition 
\ref{P:multidivided-delta-basis}. In particular $V \subset L_B(V)$.

\item
\label{item:product-formula}
For each $\sigma, \nu \in S_\mu$ with $\ell(\nu) < \ell(\sigma)$ there exist 
operators $D_{\nu,\sigma}^\mu: B \to B$ of degree $\ell(\sigma) - \ell(\nu)$ 
such that for all $f \in B$ and $v \in V$ we have 
\begin{align*}
D^\mu_\sigma (f v)
  &\equiv \sym_\mu(f) D^\mu_\sigma(v) 
    + \sum_{\ell(\nu) < \ell(\sigma)} 
      D_{\nu,\sigma}^\mu(f)
      D^\mu_\nu (v) \mod \p_B^{S_\mu} L_B(V).
\end{align*}

\item
\label{item:generation}
Suppose $\mathcal V \subset V$ is such that $S_\mu( \mathcal V)$ generates $V$ 
as $B$-module. Then the set $G = \{D_\sigma^\mu(v) \mid v \in \mathcal V, 
\sigma \in S_\mu\}$ generates $L_B(V)$ over $B$.
\end{enumerate}
\end{Lemma}
\begin{proof}
To prove item (\ref{item:divided}) notice that
\begin{align*}
D^\mu_\sigma(v) 
  &= \frac{1}{\mu!}\partial_{w_\mu} \left(\Delta \partial_{\sigma} v \right)
  = \frac{1}{\mu!}\partial_{w_\mu} \left( 
    \partial_{\sigma^{-1}} (\Delta) w 
  \right) 
  = \sym_{\mu}\left( \frac{\partial_{\sigma^{-1}} \Delta}{\Delta} w\right)
  \in L_B(W).
\end{align*}
Recall that we have matrices $X, Y, C \in M_{\mu!}(\CC(X))$ with
\begin{align*}
X^\sigma_\tau
  &= \tau \left(\frac{\partial_\sigma \Delta}{\Delta} \right),
&Y^\rho_\nu 
  &=\frac{\rho(\partial_{\nu w_\mu} \Delta)}{n!},
\end{align*}
and $C$ with coefficients in $\p^{S_\mu} \subset \CC[\mu]$ such that $Y + C$
is the inverse of $X$. Thus
\begin{align*}
\sum_{\sigma \in S_\mu} &(\partial_{\sigma^{-1}w_\mu}\Delta_\mu 
    + c_{\sigma^{-1}w_\mu}) D^\mu_\sigma(v) \\
    &= \sum_{\sigma \in S_\mu} \sum_{\rho \in S_\mu} 
      (\partial_{\sigma^{-1}w_\mu}\Delta_\mu + c_{\sigma^{-1}w_\mu})
        \rho\left(\frac{\partial_{\sigma^{-1}} \Delta}{\Delta}\right) 
          \rho(v) \\
    &= \sum_{\rho \in S_\mu} \sum_{\sigma \in S_\mu} (Y+C)_{\sigma^{-1}}^e 
      X^{\sigma^{-1}}_\rho
    = \sum_{\rho \in S_\mu} (\mathsf{Id})_{\rho}^e \rho(v) = v.
\end{align*}

Since divided difference operators are $\CC[\mu]^{S_\mu}$-linear, for every 
$f,g \in B$ such that $f \equiv g \mod \p_B^{S_\mu}$ we obtain 
$\partial_\sigma(f v) \equiv \partial_\sigma(g v) \mod \p_B L_B(V)$. Using this
\begin{align*}
D^\mu_\sigma (fv)
  &\equiv D^\mu_\sigma \left(
      \sum_{\nu \in S_\mu} 
        \sum_{\tau \in S_\mu}
          \partial_{\nu^{-1}w_\mu}(\Delta_\mu)
            \partial_{\tau^{-1}w_\mu}(\Delta_\mu)
          D^\mu_\tau(f)D^\mu_\nu (v)
    \right)
    \mod \p_B L_B(V)
\end{align*}
Since $D_\sigma^\mu(v)$ and $D^\mu_\tau(f)$ are symmetric this last term equals
\begin{align*}
\sum_{\nu \in S_\mu} 
  \sum_{\tau \in S_\mu}
    D^\mu_\sigma (\partial_{\nu^{-1}w_\mu}\Delta_\mu
            \partial_{\tau^{-1}w_\mu}\Delta_\mu)
          D^\mu_\tau(f)D^\mu_\nu (v)    
\end{align*}
Let $d = \deg D^\mu_\sigma (\partial_{\tau^{-1}w_\mu} \Delta 
\partial_{\nu^{-1}w_\mu} \Delta) = \ell(\nu) + \ell(\tau) - \ell(\sigma)$. If 
$d > 0$ then this polynomial lies in $\p^{S_\mu}$, so these terms are zero
modulo $\p_B^{S_\mu} L_B(V)$. Now if $d = 0$ then $\ell(\sigma) = \ell(\nu) + 
\ell(\tau)$, which implies $\ell(\nu) \leq \ell(\sigma)$, and $\sigma = \nu$
implies $\tau = e$. Setting
\[
D_{\nu,\sigma}^\mu (f)
  = \sum_{\ell(\tau) = \ell(\sigma) - \ell(\nu)} 
    D^\mu_\sigma (\partial_{\nu^{-1}w_\mu}\Delta_\mu
            \partial_{\tau^{-1}w_\mu}\Delta_\mu)
          D^\mu_\tau(f)
\]
we get item (\ref{item:product-formula}).

Finally to prove item (\ref{item:generation}) it is enough to prove that for 
any $v \in \mathcal V, \sigma \in S_\mu$ and $f \in B$ the element $\sym_\mu
\left( \frac{f}{\Delta} \sigma (v)\right)$ lies in the $B$-module generated 
by $G$. Now
\begin{align*}
\sym_\mu \left(
  \frac{f}{\Delta_\mu} \sigma(v)
\right)
&= \sym_\mu \left(
  \frac{\sg(\sigma)\sigma(f)}{\Delta_\mu} v
\right)
\\
&=\sum_{\tau \in S_\mu}\left(
    \sym_\mu \sg(\sigma)D_\tau^\mu(\sigma(f)) \left(
      \frac{\partial_{\tau^{-1} w_\mu} \Delta_\mu}{\Delta_\mu} v
      \right)
    + c_\tau \sym_\mu \left( \frac{1}{\Delta_\mu} v\right)
    \right)  \\
&= \sum_{\tau \in S_\mu}\sg(\sigma)D_\tau^\mu(\sigma(f))\left(
    D^\mu_{\tau^{-1} w_\mu}(v)
    + c_\tau D^\mu_{w_\mu}(v)
    \right)
\end{align*}
and we are done.
\end{proof}

Let $\eta$ be a refinement of $\mu$. In that case we have inclusions $S_\eta 
\subset S_\mu \subset S_n$. Recall that we have set $\Shuffle_\eta^\mu = 
\Shuffle_\eta \cap S_\mu$, and that we denote by $V^\mu_\eta$ the 
$S_\mu$-module obtained by induction from the trivial $S_\eta$-module. By 
definition there exists $v_\eta \in V^\mu_\eta$ whose stabilizer is $S_\eta$ 
and such that $\CC[S_\mu]v_\eta = V^\mu_\eta$.

\begin{Lemma}
\label{L:dds-basis}
Let $\eta$ be a refinement of $\mu$. If $V = B \ot V^\mu_\eta$ then 
$L_B(V_B)$ is a free $B$-module with basis $\{D^\mu_\sigma(v_\eta) \mid \sigma 
\in \Shuffle_\eta^\mu\}$. Furthermore, if $\sigma \notin \Shuffle_\eta^\nu$ 
then $D^\mu_\sigma (v_\eta) = 0$.
\end{Lemma}
\begin{proof}
By item (\ref{item:generation}) of Lemma \ref{L:dds-generators} we know that
$\{D_\sigma^\mu(v_\eta) \mid \sigma \in S_\mu\}$ generates $L_B(V)$.
Suppose $\sigma \in S_\mu$ is not an $\eta$-shuffle. Then it can be 
written as $j_\eta(\sigma) s_{i_1} \cdots s_{i_t}$, with $\ell(\sigma) = 
\ell(j_\eta(\sigma)) + t$ and $s_{i_j} \in S_\eta$ for all $j$. Thus 
$\partial_{\sigma}(v) = 0$ for any $v$ which remains invariant by the action 
of $s_{i_t}$ and hence $D^\mu_{\sigma}(v_\eta) = 0$. It follows that
the set $\{D_\sigma^\mu(v) \mid \sigma \in \Shuffle^\mu_\eta\}$ generates 
$L_B(V)$ over $B$. 

By item (\ref{item:divided}) of Lemma \ref{L:dds-generators} $\CC(\mu) \ot_B  
L_B(V_B) = \CC(\mu) \ot V^\mu_\eta$. This is a $\CC(\mu)$-vector space of 
dimension $\mu!/\eta!$, and hence the generating set $\{1 \ot_B 
D_\sigma^\mu(v_\eta) \mid \sigma \in \Shuffle_\eta^\mu\}$ is a basis of $F \ot 
V^\mu_\eta$ since its cardinality is also $\mu!/\eta!$. In particular it is 
linearly independent over $B$ and hence $L_B(V_B)$ is a free $B$-module.
\end{proof}

\begin{Remark}
A more general approach can be taken in the study of modules of partial 
differentials, by considering the smash product $\CC(\mu) \# S_\mu$ and 
setting $\partial_i = \frac{1}{x_i - x_{i+1}} \# (\id_{S_\mu} - s_i)$. We 
could then define a divided difference module as a module over 
the algebra generated by $\CC[\mu][\partial_1, \ldots, \partial_{r-1}]$, and 
the properties of the corresponding operators would follow by calculations on 
the smash product algebra. However taking this route would lead us too far 
afield of our original purpose, and we stick to the definition given above, 
which will satisfy our needs.
\end{Remark}

\subsection{Equivariant families of modules}
Throughout this subsection $U$ denotes an associative algebra. Let $V$ be an 
$A$-module with an $S_\mu$-equivariant action, and suppose that $V$ has a 
$U$-module structure, and that $U$ acts by $A$-linear and $S_\mu$-equivariant 
operators. Then for each $\m \in \Specm A$ the $\CC$-vector space $A/\m \ot_A 
V_A$ inherits a $U$-module structure, and we obtain a family of $U$-modules 
parametrized by $\Specm A$. In this subsection we will show that under certain 
conditions, symmetrization operators allow us to obtain a larger family of 
submodules, parametrized by $\Specm B$.

\begin{Definition}
Let $V$ be an $B$-module. We say that $V$ \emph{supports a $\Delta$-singular 
action of $U$} if $A \ot_B V$ can be endowed with an $A$-linear $U$-module 
structure. We say that the action is \emph{weakly $\Delta$-singular} if 
$V$ has an $S_\mu$-action, the actions of $U$ and $B$ are $S_\mu$-equivariant, 
and $U V \subset L_B(V)$.
\end{Definition}

If $V$ is a $U$-module then it certainly supports a $\Delta$-singular action,
but the converse is not true. We have singled out the case of weakly 
$\Delta$-singular actions because in that case the converse holds if we 
replace $V$ by $L_B(V)$.

\begin{Proposition}
\label{P:weak-action-conditions}
Let $V$ be a $B$-module with an $S_\mu$-equivariant action. Then 
$V$ supports a weakly $\Delta$-singular action of $U$ if and only if there 
exists a set $\mathcal U$ of generators of $U$ such that $u V \subset L_B(V)$
for all $u \in \mathcal U$, and in that case $L_B(V)$ is a $U$-submodule of 
$A \ot_B V$. Furthermore, $A \ot_B L_B(V) = A \ot_B V$.
\end{Proposition}
\begin{proof}
Clearly the fact that $V$ supports a weakly $\Delta$-singular action implies
the existence of a set of generators as in the statement. Recall that $L_B(V)$ 
is generated by elements of the form $D^\mu_\sigma(v)$ with $v \in V$. Let 
$u \in \mathcal U$ and $v \in V$. By hypothesis there exists a finite set 
$\mathcal V \subset V$ and functions $f_{\sigma,v'} \in B$ for $\sigma \in 
S_\mu, v' \in \mathcal V$ such that
\begin{align*}
u v 
  &= \sum_{\tau \in S_\mu, v' \in \V} 
    f_{\sigma,v'} D^\mu_\tau (v')
\end{align*}
Since the action of $U$ is both $A$-linear and $S_\mu$-equivariant, we get
\begin{align*}
u D^\mu_\sigma (v)
  &= D^\mu_\sigma (u v)
  = \sum_{\tau \in S_\mu, v' \in \V} 
      D^\mu_{\sigma}(f_{\sigma,v'}) D^\mu_\tau (v') 
      \in L_B(V).
\end{align*}
Thus $L_B(V)$ is stable by the action of $U$, and since $V \subset L_B(V)$ by 
item (\ref{item:divided}) of Lemma \ref{L:dds-generators}, this shows that $u V
\subset L_B(V)$ for all $u \in \mathcal U$. The last assertion also follows 
from this.
\end{proof}

The following lemma will be useful to present concrete examples.
\begin{Lemma}
\label{L:induced}
Let $W = B \ot V$, and suppose that $V = \bigoplus_{i \in I} V_i$ as 
$S_\mu$-modules, with $I$ an index set and each $V_i$ isomorphic to some 
$V^\mu_{\eta_i}$ for $\eta_i$ a refinement of $\mu$. Let $v_i \in V_i$
be an element with stabilizer $S_{\eta_i}$ and $\CC[S_\mu] v_i = V_i$.
Suppose that $W$ supports a $\Delta$-singular action of $U$, and that
$U$ is generated by a set $\mathcal U$ such that 
$\Delta_{\eta_i} u v_i \subset W$ for each $u \in \mathcal U$. Then the action 
of $U$ is weakly $\Delta$-singular.
\end{Lemma}
\begin{proof}
By Proposition \ref{P:weak-action-conditions} it is enough to show that $u 
\sigma(v_i) \in L_B(W)$ for each $u \in \mathcal U, i \in I$ and $\sigma \in 
S_\mu$. Now the set $\{D^\mu_\tau (v_i) \mid \tau \in S_\mu, i \in I\} 
\setminus \{0\}$ is a $B$-basis of $L_R(W)$ by Lemma \ref{L:dds-basis}, so for 
each $\tau \in S_\mu$ and each $j \in I$ there exists a unique $f_{\tau,j} \in 
B$ such that
\begin{align*}
 u v_i
  &= \sum_{\tau \in S_\mu, j \in I} 
    \frac{f_{\tau,j}} {\Delta_{\eta_i}} D^\mu_\tau(v_j). 
\end{align*}
Since the action of $U$ is $S_\mu$-equivariant and both $v_i$ and 
$D^\mu_\sigma(v_j)$ are invariant by the action of $S_{\eta_i}$, the rational
function $\frac{f_{\sigma,j}} {\Delta_{\eta_i}}$ is also invariant by the 
action of $S_{\eta_i}$. Thus
\begin{align*}
\frac{f_{\sigma,j}} {\Delta_{\eta_i}}
  &=\sym_{\eta_i}\left(\frac{f_{\sigma,j}} {\Delta_{\eta_i}}\right) \in B
\end{align*}
so $u v_i \in L_B(W)$. Finally $u \tau(v_i) = \tau (u v_i) \in L_B(W)$ since
$L_B(W)$ is stable by the action of $S_\mu$.
\end{proof}

\section{Gelfand-Tsetlin modules over $\gl(n,\CC)$ with arbitrary characters}
\label{GT-arbitrary-characters}
Fix $n \in \NN_{\geq 2}$. Throughout this section $\overline \mu$ denotes the 
composition $(1,2, \ldots, n)$ of $N = \binom{n}{2}$, and $\mu$ its refinement 
$(1,2 ,\ldots, n-1, 1^n)$, where $1^n$ denotes the partition of $n$ given by 
$n$ ones. We also denote by $1^\mu$ the refinement $(1,1^2, \ldots, 1^n)$.

\subsection{Gelfand Tsetlin tableaux and associated partitions}
Let $r \in \NN$, and let $v \in \CC^r$. We say that $v$ is \emph{decreasing} if
$v_i - v_j \in \NN_0$ for each $1 \leq i < j \leq r$. We say that $v \in 
\CC^\mu$ is in \emph{normal form} if there is a refinement $\eta$ of $\mu$ 
such that for each $k \in \interval r$ the sequence $\eta^{(k)}$ is 
decreasing, the $\eta$-blocks of $v$ are decreasing, and if whenever two 
entries in the same $\mu$-block differ by an integer, they are in the same 
$\eta$-block. It follows from the definition that for each $v \in \CC^\mu$ 
there is at least one element in the $S_\mu$-orbit of $v$ in normal form, and
for any two such elements the refinement $\eta$ is the same. Any element of
the orbit of $v$ in normal form will be called a \emph{normalization} of $v$, 
and we will denote by $\eta(v)$ the corresponding refinement of $\mu$. 

The refinement $\eta(v)$ can be calculated as follows: for each $\mu$-block 
$v_k$ draw a graph $G_k$ whose vertices are labeled by $\interval{k}$, and 
with an edge between vertices $i$ and $j$ if and only if $v_{k,i} - v_{k,j} 
\in \ZZ$. Then $G_k$ is the disjoint union of a family of discrete graphs,
and the partition $\eta(v)^{(k)}$ is the sequence of numbers of vertices of
each complete graph, arranged in descending order. 

\begin{Definition}
Let $v \in \CC^{\overline \mu}$. The \emph{Gelfand-Tsetlin tableau} associated 
to $v$, denoted by $T(v)$, is the triangular array with $N$ complex entries of 
the form

\begin{tikzpicture}
\node (n1) at (-2,2.5) {$v_{n,1}$};
\node (n2) at (-1,2.5) {$v_{n,2}$};
\node (ndots) at (0,2.5) {$\cdots$};
\node (nn-1) at (1,2.5) {$v_{n,n-1}$};
\node (nn) at (2,2.5) {$v_{n,n}$};

\node (n-11) at (-1.5,2) {$v_{n-1,1}$};
\node (n-1dots) at (0,2) {$\cdots$};
\node (n-1n-11) at (1.5,2) {$v_{n-1,n-1}$};

\node (dots1) at (-1,1.625) {$\ddots$};
\node (dots2) at (0,1.5) {$\cdots$};
\node (dots3) at (1,1.625) {$\iddots$};

\node (21) at (-.5,1) {$v_{2,1}$};
\node (22) at (.5,1) {$v_{2,2}$};
\node (11) at (0,.5) {$v_{1,1}$};

\node (A) at (-3.5, 2.75) {};
\node (B) at (3.5, 2.75) {};
\node (C) at (0,0) {};
\end{tikzpicture}

Thus the $\overline \mu$-blocks of $v$ form the rows of $T(v)$. The action of 
$S_\mu$ on $\CC^\mu$ induces an action of $S_\mu$ on the set of all 
Gelfand-Tsetlin tableaux, with $S^{(k)}$ permuting the elements in the $k$-th 
row of $T(v)$. Notice that this action leaves the top row of the tableau fixed.
\end{Definition}
We will usually abreviate Gelfand-Tsetlin by GT.

Let $v \in \CC^{\mu}$ and let $\eta = \eta(v)$. We say that $v$ is 
\emph{generic} if $\eta(v)^{(k)} = 1^{\mu_k}$, i.e. if no two elements in the 
same $\mu$-block (or, in terms of the corresponding GT-tableau, no two entries 
in the same row) differ by an integer; if $\eta$ has at least one part larger 
than $1$ we say that $v$ is \emph{$\eta$-singular}. We also say that $v$ is 
\emph{$\eta$-critical} if any two entries in the same $\eta$-block are equal.

We denote by $\ZZ^\mu_0$ the set of all integral $\overline \mu$-points $v$ 
with $v_{n,i} = 0$. Let $\eta$ be a refinement of $\mu$, and let $\ZZ^\eta_0$ 
be the image of $\ZZ^\mu_0$ through the natural identification $\CC^\eta \cong 
\CC^\mu$. We say that $z \in \ZZ^\eta_0$ is in normal $\eta$-form if each 
$\eta$-block of $z$ is a decreasing sequence, and denote by $\N(\eta)$ the set
of all integral $\eta$-points in normal $\eta$-form. If $z \in \ZZ_0^\eta$
then there is a unique element in the $S_\eta$-orbit of $z$ which is in normal
$\eta$-form, which is called the $\eta$-normalization of $z$. For each $z \in 
\ZZ_0^\eta$ we set $\epsilon = \epsilon^\eta(z)$ to be the refinement of $\eta$
defined by the following property: if $z'$ is the $\eta$-normalization of $z$,
then two entries in the same $\eta$ block of $z'$ are equal if and only if
they are in the same $\epsilon$-block.



\subsection{Gelfand-Tsetlin modules}
As before we will denote by $\CC[\overline \mu]$ the algebra of polynomial 
functions on $\overline \mu$-points. For each $k \in \interval n$ and each
$i \in \interval k$ we denote by $\lambda_{k,i}$ the $i$-th coordinate 
function of $\CC[\overline \mu]_k$. Also set
\[
  \gamma_{k,j} = \sum_{i = 1}^k (\lambda_{k,i}+m-1)^j 
  \prod_{i \neq j} \left( 1 - \frac{1}{\lambda_{k,i} - \lambda_{k,j}}\right).
\]
Although it is not obvious, the $\gamma_{k,j}$ are algebraically independent 
polynomials and generate the subalgebra of invariant polynomials 
$\CC[\mu]^{S_\mu}$.

For each $k \in \NN$ set $U_k = U(\gl(k, \CC))$. We denote by $Z_k \subset U_k$
the center of $U_k$. Also we write $U$ for $U_n$. We get a chain of inclusions 
$U_1 \subset U_2 \subset \cdots \subset U_n$ induced by the maps sending 
standard generators $E_{i,j} \in \gl(k,\CC)$ to the corresponding $E_{i,j} 
\in \gl(k+1, \CC)$. The algebra $Z_k$ is a polynomial algebra on the generators
\[
  c_{k,j} = \sum_{(i_1, \ldots, i_j) \in \interval{k}^j} 
    E_{i_1,i_2} E_{i_2,i_3} \cdots E_{i_j, i_1} \qquad \qquad 1 \leq j \leq k,
\]
and there is an embedding $Z_k \to \CC[\mu]_k$ given by $c_{k,j} \mapsto 
\gamma_{k,j}$. We write $\Gamma = \CC[c_{k,j} \mid 1 \leq j \leq k \leq n] 
\subset U$, which is the algebra generated by $\bigcup_{k=1}^n Z_k$. The 
$c_{k,j}$ are algebraically independent and hence $\Gamma$ is isomorphic to
$\CC[\mu]^{S_\mu}$.

\begin{Definition}[\cite{DFO-GT-modules}*{section 2.1}]
\label{D:gt-modules}
The algebra $\Gamma$ is called the \emph{Gelfand-Tsetlin subalgebra} of 
$U$. A finitely generated $U$-module $M$ is called a 
\emph{Gelfand-Tsetlin module} if
\[
  M = \bigoplus_{\mathfrak m \in \Specm \Gamma} M [\mathfrak m],
\] 
where $M[\mathfrak m] = \{v \in M \mid \mathfrak m^k v = 0 \mbox{ for some } k 
\geq 0\}$.
\end{Definition}
Maximal ideals of $\Gamma$ are in natural bijection with its characters
$\chi: \Gamma \to \CC$. If $\chi$ is the character corresponding to $\m$ then
we also write $M[\chi] = M[\m]$. Now for each $v \in \CC^N$ we get a character 
$\chi_v: \Gamma \to \CC$ by setting $\chi_v(c_{k,j}) = \gamma_{k,j}(v)$, and
all characters of $\Gamma$ are of this form. Since $\gamma_{k,m}$ is a 
symmetric polynomial, $\chi_v = \chi_{\sigma(v)}$ for each $v \in 
\CC^{\overline \mu}$ and each $\sigma \in S_{\overline \mu}$, so characters of 
$\Gamma$ are in one to one correspondance with $\CC^N/ S_{\overline \mu}$. We 
will say that a character $\chi_v$ is generic, singular, critical or fully 
critical if $v$ is generic, singular, critical or fully critical, 
respectively. Notice that this is well defined since these properties are 
invariant by the action of $S_{\overline \mu}$. If $\chi = \chi_v$ then we set 
$\eta(\chi_v) = \eta(v)$. 

\subsection{Singular GT-modules}
\begin{Definition}
Let $C = \{\lambda_{k,i} - \lambda_{k,j} - z \mid 1 \leq i < j \leq k < n, 
z \in \ZZ \setminus \{0\}\}$, set $B = C^{-1}\CC[\mu]$. Given $\eta$ a
refinement of $\mu$, we set $B_\eta = B\left[\frac{\Delta_\eta}{\Delta_\mu}
\right]$, and denote by $\p_\eta$ the ideal generated by the $\lambda_{k,i} - 
\lambda_{k,j}$ with $i,j$ in the same $\eta^{(k)}$-block. Finally, set $A = 
B_{(1^N)} = B[\Delta_\mu^{-1}]$.
\end{Definition}
Notice that the generators given for $\p_\eta$ are the linear factors of
$\Delta_\eta/\Delta_\mu$. 

Let $\eta$ be a refinement of $\mu$. Any $v \in \CC^{\overline \mu}$ induces 
an automorphism in $\CC[\overline \mu]$ by setting $\lambda_{k,i} \mapsto 
\lambda_{k,i} + v_{k,i}$. This automorphism extends to an automorphism of 
$B_\eta$ if no two entries in the same $\eta$-block are equal, i.e. if  
$\epsilon^\eta(v) = 1^\eta$. Given $v$ with $\epsilon^\eta(v) = 1^\eta$ and 
$p \in B_\eta$ we denote $p(\lambda + v)$ or $p(\lambda^v)$ the image of $p$ 
by this automorphism. Notice that $\sigma(p(\lambda + v)) = 
\sigma(p)(\lambda + \sigma(v))$ for all $\sigma \in S_\eta$.

\begin{Definition}
Let $\eta$ be a refinement of $\mu$. We denote by $V_\CC$ the $\CC$ vector 
space generated by the set $\{T(z) \mid z \in \ZZ^\mu_0\}$, and set $V_\eta =
B_\eta \ot V_\CC$ with the diagonal $S_\eta$-action. 
\end{Definition} 
Since $S_\eta \subset S_\mu$, there is a natural action of $S_\eta$ on $V_\CC$,
and $V_\CC = \bigoplus_{z \in \N(\eta)} O(T(z))$, where $O(T(z))$ denotes the
vector space generated by the $S_\eta$-orbit of $T(z)$. Take $z \in \N(\eta)$ 
and let $\epsilon = \epsilon^\eta(z)$. By definition the stabilizer of $z$ 
under the action of $S_\eta$ is $S_{\epsilon}$, and so $O(z) \cong O(T(z))
\cong V_\epsilon^\eta$. Thus $V_\eta = \bigoplus_{z \in \N(\eta)} B_\eta \ot 
O(T(z))$ as $S_\eta$-modules. 

For each $1 \leq i \leq k \leq n$ set
\begin{align*}
p_{k,i}^\pm(\lambda) 
  &= \prod_{j = 1}^{k\pm1}(\lambda_{k,i} -\lambda_{k\pm1,j}); &
q_{k,i}(\lambda)
  &= \prod_{j \neq i} (\lambda_{k,i} - \lambda_{k,j}); 
& e_{k,i}^\pm(\lambda) = \frac{p_{k,i}^\pm}{q_{k,i}},
\end{align*}
and also $|\lambda|_k = \lambda_{k,1} + \lambda_{k,2} + \cdots \lambda_{k,k}$.
Then for each $\sigma \in S_\mu$ we have $\sigma (e^\pm_{k,i}(\lambda)) = 
e^\pm_{k,\sigma^{(k)}(i)}(\lambda)$ and $\sigma(|\lambda|_k) = |\lambda|_k$.

The following result is essentially due 
to Drozd, Futorny and Ovsienko \cite{DFO-GT-modules}, but was proved in the 
form below in \cite{Zad-1-sing}*{Proposition}.
\begin{Proposition}
\label{P:universal-generic-GT-module}
Let $V_A$ be the free $A$-module with basis $\{T(z) \mid z \in \ZZ^N_0\}$.
The $A$-module $V_A$ can be endowed with the structure of a $U$-module 
with the action of the canonical generators given by
\begin{align*}
E_{k,k+1} T(z) 
  &= - \sum_{i=1}^k e^+_{k,i}(\lambda^z) T(z + \delta^{k,i}); \\
E_{k+1,k} T(z) 
  &= \sum_{i=1}^k e^-_{k,i}(\lambda^z) T(z - \delta^{k,i}); \\
E_{k,k} T(z)
  &= (|\lambda^z|_k - |\lambda^z|_{k-1} + k -1) T(z).
\end{align*}
With this definition, the action of $U$ is $S_\mu$-equivariant.
Furthermore, for each $1 \leq k \leq m \leq n$, we have $c_{m,k} T(z) = 
\gamma_{m,k}(\lambda^z) T(z)$.
\end{Proposition}
Let $v \in \CC^\mu$ be a generic point. Then for each $f \in A$ it makes sense 
to consider the evaluation of $f$ at $v$, and this induces a $1$-dimensional 
representation $\CC_v$ of $A$. As a consequence of the previous proposition, 
the vector space $\CC_v \ot_A V_A$ is a $U$-module. Setting $V(T(v))$ to be 
the vector space with basis $\{T(v+z) \mid z \in \ZZ^N_0\}$, we obtain a 
linear isomorphism $\CC_v \ot_A V_A \to V(T(v))$ given by $1 \ot_A T(z) 
\mapsto T(v+z)$, and so $V(T(v))$ inherits a $U$-module structure. 
Furthermore, $c_{m,k} T(v+z) =\gamma(v+z) T(v+z)$ and hence $V(T(v))$ is a 
Gelfand-Tsetlin module. Since $v$ is generic, there is no element of the form 
$T(v+z')$ in the $S_{\overline \mu}$-orbit of $T(v+z)$ and so $V(T(v)) = 
\bigoplus_{z \in \ZZ^N_0} V(T(v))[\chi_{v+z}]$, and each component is 
$1$-dimensional.

Now this construction breaks down if $v$ is not generic, or equivalently if 
$\eta = \eta(v) \neq 1^\mu$. However if $v$ is $\eta$-critical then evaluation 
at $v$ does define a one-dimensional representation of $B_\eta$, and so it 
makes sense to look at the action of $U$ on $V_\eta$. While $V_\eta$ is not a 
$U$-submodule of $V_A$, we do have the following result.
\begin{Proposition}
\label{P:GT-weak-singular}
Let $\eta$ be a refinement of $\mu$. The action of $U$ on $V_\eta$ is weakly 
$\Delta_\eta$-singular.
\end{Proposition}
\begin{proof}
As stated above $V_\eta = \bigoplus_{z \in \N(\eta)} O(T(z))$, and each 
$O(T(z))$ is isomorphic to $V^\eta_{\epsilon(z)}$. It follows from the 
definition of the action of $U$ that for each $z \in \N(\eta)$ the elements
$\Delta_{\epsilon(z)} E_{k,k+1} T(z), \Delta_{\epsilon(z)} E_{k+1,k} T(z)$ and 
$E_{k,k} T(z)$ all lie in $V_\eta$. The statement now follows from Lemma 
\ref{L:induced}. 
\end{proof}
Let $v \in \CC^\mu$ be an $\eta$-singular point. This implies that there 
exists $z \in \ZZ^\eta_0$ such that $v+z$ is $\eta$-critical and hence 
evaluation at $v+z$ defines a $1$-dimensional representation of $B_\eta$, 
which we denote by $\CC_{v+z}$. Hence we obtain a $U$-module by taking 
$\CC_{v+z} \ot_{B_\eta} V_\eta$. We will show that this $U$-module is a 
GT-module whose support consists of characters of the form $\chi_{v+z'}$ with 
$z' \in \ZZ^\mu_0$, and we will find the multiplicities of these characters.


\subsection{The action of $U$ on $L_\eta$}
For the rest of this section fix $\eta$ a refinement of $\mu$, and
set $A_\eta = B_\eta / \p_\eta$. By definition $A_\eta$ is the algebra
of polynomial functions over the hyperplane of $\eta$-critical points,
and hence is a polynomial algebra in variables $y_i^{(k)}$, one for each
$\eta^{(k)}$-block. If $v$ is $\eta$-critical then $\CC_v \ot_{B_\eta} 
L_B(V_\eta) \cong \CC_v \ot_{A_\eta} A_\eta \ot_{B_\eta} L_B(V_\eta)$.

Let us write $D_\sigma^\eta(z) = D_\sigma^\eta(T(z))$ for each $\sigma \in 
S_\eta$ and each $z \in \ZZ^\eta_0$; we will refer to this element as the 
$\sigma$-derived tableau of $T(z)$. We know by Lemma \ref{L:dds-basis} and the 
discussion in the previous section that the set 
\begin{align*}
\{D_\sigma^\eta(z) \mid z \in \N(\eta), 
  \sigma \in \Shuffle_{\epsilon(z)}^\eta\}
\end{align*}
forms a $B_\eta$-basis of $L_{B_\eta}(V_\eta)$.
\begin{Definition}
The \emph{universal $\eta$-singular GT-module} is defined as $A_\eta
\ot_{B_\eta} L_{B_\eta}(V_\eta)$, and denoted by $L_\eta$. For each $z \in 
\ZZ^\eta_0$ and $\sigma \in S_\eta$ we denote by $\D_\sigma(z)$ the image of 
$D^\eta_\sigma(z)$ in $L_\eta$.
\end{Definition}
This subsection will be dedicated to the study of this module. A first 
obervation is that $L_\eta$ is a free $A_\eta$-module with basis 
$\{\D_\sigma^\eta(z) \mid z \in \N(\eta), \sigma \in 
\Shuffle_{\epsilon(z)}^\eta\}$.

Fix $z \in \ZZ^\mu_0$ in normal $\eta$-form, and let $\epsilon = 
\epsilon^\eta(v)$. We fix $k \in \interval{n-1}$ and set $\alpha_i = 
\alpha_i^{(k)}(\epsilon), \beta_i = \beta_i^{(k)}(\epsilon)$. Then given
$j \in \interval {k}$ the point $z + \delta^{k,j}$ is in normal $\eta$-form if 
and only if $j = \alpha_i$ for some $i$; analogously $z - \delta^{k,j}$ is in 
normal form if and only if $j = \beta_i$ for some $i$. Let us write 
$\alpha(j) = \alpha_i, \beta(j) = \beta_i$ determined by the condition that
$j \in \interval{\alpha_i, \beta_i}$.

By definition the transposition $(j,\alpha(j))$ is in the stabilizer of $z$, 
and hence $(j,\alpha(j)) (z+\delta^{k,j}) = z + \delta^{k,\alpha(j)}$ is in 
$\eta$-normal form. It follows that
\begin{align*}
E_{k,k+1} T(z)
  &= E_{k,k+1} \sym_\epsilon T(z)
  = - \sum_{j=1}^k \sym_\epsilon \left( e_{k,j}^+(\lambda + z) 
    T(z + \delta^{k,j})\right) \\
  &= - \sum_{j=1}^k \sym_\epsilon (j,\alpha(j))\left( e_{k,\alpha(j)}^+
    (\lambda + z) T(z + \delta^{k,\alpha(j)})\right) \\
  &= - \sum_{i=1}^t \epsilon^{(k)}_i \sym_\epsilon\left( 
    e_{k,\alpha(j)}^+(\lambda + z) T(z + \delta^{k,\alpha_i})
  \right) \\
  &= - \sum_{i=1}^t \sum_{\nu \in S_\eta} 
  \epsilon^{(k)}_i \sym_\epsilon\left( 
    e_{k,\alpha(j)}^+(\lambda + z) (\partial_{\nu^{-1}w_\eta}\Delta_\eta 
      + c_\nu) 
  \right)  
  D_\nu^\eta (z + \delta^{k,\alpha_i}),
\end{align*}
where we have used item (\ref{item:divided}) of Lemma \ref{L:dds-generators} to
write $T(z+\delta^{k,\alpha_i})$ in terms of its derived tableaux. An 
analogous reasoning shows that 
\begin{align*}
E_{k+1,k} T(z)
  &= \sum_{i=1}^t \sum_{\nu \in S_\eta} 
  \epsilon^{(k)}_t \sym_\epsilon\left( 
    e_{k,\alpha(j)}^-(\lambda + z) (\partial_{\nu^{-1}w_\eta}\Delta_\eta 
      + c_\nu) 
  \right)  
  D_\nu^\eta (z - \delta^{k,\beta_i}).
\end{align*}

Since $U$ acts by $S_\eta$-equivariant and $B_\eta$-linear operators, we have
$E_{k,k+1} D_\sigma^\eta(z) = D_\sigma^\eta(E_{k,k+1} T(z))$. In order to find
an explicit expression for this element, we apply the operator $D_\sigma^\eta$
to the expression in the previous display. Now derived tableaux are invariant
by the action of $S_\eta$ and hence the operator acts only on the coefficients
of the tableaux. Now since $\Delta_\epsilon e_{k,\alpha(j)}^-(\lambda + z) 
\in B_\eta$, we can use Proposition \ref{P:multidivided-delta-basis} and
thus obtain
\begin{align*}
D_{\sigma}^\eta &\left(
  \sym_\epsilon\left( 
    e_{k,\alpha(j)}^+(\lambda + z) 
      (\partial_{\nu^{-1}w_\eta}\Delta_\eta + c_\nu) 
    \right)
  \right) \\
  &=\sum_{\rho \in S_\eta}D_{\sigma}^\eta \left(
    \sym_\epsilon\left( 
      \frac{
        (\partial_{\rho^{-1}w_\eta}\Delta_\eta + c_\rho)
        (\partial_{\nu^{-1}w_\eta}\Delta_\eta + c_\nu)
      }{\Delta_\epsilon} \right) 
    \right) 
    D^\eta_\rho \left(
      e_{k,\alpha(j)}^+(\lambda + z) \Delta_\epsilon)
    \right).
\end{align*}
Using that $c_\nu \in \p_B^{S_\mu}$ we see that this last sum is congruent 
$\mod \p_B$ to
\begin{align*}
\sum_{\rho \in S_\eta}D_{\sigma}^\eta \left(
    \sym_\epsilon\left( 
      \frac{
        (\partial_{\rho^{-1}w_\eta}\Delta_\eta)
        (\partial_{\nu^{-1}w_\eta}\Delta_\eta)
      }{\Delta_\epsilon} \right) 
    \right) 
    D^\eta_\rho \left(
      e_{k,\alpha(j)}^+(\lambda + z) \Delta_\epsilon)
    \right).
\end{align*}
Since the operators $D^\eta_{\sigma}$ are linear combinations of divided
differences, the coefficient $D_{\sigma}^\eta \left(
    \sym_\epsilon\left( 
      \frac{
        (\partial_{\rho^{-1}w_\eta}\Delta_\eta)
        (\partial_{\tau^{-1}w_\eta}\Delta_\eta)
      }{\Delta_\epsilon} \right) 
    \right) $
is a polynomial, and if it is of positive degree then it lies in 
$\p^{S_\mu}$. Now its degree is $d = \ell(\rho) + \ell(\nu) - \ell(w_\epsilon)
- \ell(\sigma)$, and the previous sum is congruent modulo $\mod \p_B$ to
\begin{align*}
\sum_{\ell(\rho) = \ell(\sigma) - \ell(\nu) + \ell(w_\epsilon)}
  D_{\sigma}^\eta \left(
    \sym_\epsilon\left( 
      \frac{
        (\partial_{\rho^{-1}w_\eta}\Delta_\eta)
        (\partial_{\nu^{-1}w_\eta}\Delta_\eta)
      }{\Delta_\epsilon} \right) 
    \right) 
    D^\eta_\rho \left(
      e_{k,\alpha(j)}^+(\lambda + z) \Delta_\epsilon)
    \right).
\end{align*}
Writing
\begin{align*}
c_{\nu,\sigma,\rho}^{\eta,\epsilon} 
= D_{\sigma}^\eta \left(
    \sym_\epsilon\left( 
      \frac{
        (\partial_{\rho^{-1}w_\eta}\Delta_\eta)
        (\partial_{\nu^{-1}w_\eta}\Delta_\eta)
      }{\Delta_\epsilon} \right) 
    \right) 
\end{align*}
and putting together all this information we get that the coefficient of 
$D_{\nu}^\eta(z+\delta^{k,\alpha_i})$ in $E_{k,k+1} D_\sigma(z)$ is congruent
to
\begin{align*}
\epsilon^{(k)}_i 
    \sum_{\ell(\rho) = \ell(\sigma) - \ell(\nu) + \ell(w_\epsilon)} 
          c_{\nu,\sigma,\rho}^{\eta,\epsilon} D^\eta_\rho \left(
      e_{k,\alpha_i}^+(\lambda + z) \Delta_\epsilon
    \right) \mod \p_B L_B.
\end{align*}
Analogously the coefficient of $D_{\nu}^\eta(z-\delta^{k,\beta_i})$ in 
$E_{k+1,k} D_\sigma(z)$ is congruent to
\begin{align*}
\epsilon^{(k)}_i 
  \sum_{\ell(\rho) = \ell(\sigma) - \ell(\nu) + \ell(w_\epsilon)} 
          c_{\nu,\sigma,\rho}^{\eta,\epsilon} D^\eta_\rho \left(
      e_{k,\alpha(j)}^-(\lambda + z) \Delta_\epsilon
    \right) \mod \p_B L_B.
\end{align*}
This shows that derived tableux of degree higher than $\ell(\sigma)$ appear
in $E_{k,k+1} D^\eta_\sigma(z)$ only if $z$ is singular, and that this degree
is bounded by $\ell(\sigma) + \ell(w_\epsilon)$.

Meanwhile, a similar though much easier computation shows that $E_{k,k} 
D_\sigma^\eta(z) = (|\lambda^z|_k - |\lambda^z|_{k-1} + k -1) 
D_\sigma^\eta(z)$. Finally, using item (\ref{item:product-formula}) of Lemma
\ref{L:dds-generators} we see that the generators of the Gelfand-Tsetlin 
algebra $c_{k,j}$ act on a derived tableaux by
\begin{align*}
c_{k,j} D_\sigma^\eta (z) 
  &\equiv \sym_\eta (\gamma_{k,j}(\lambda^z)) D_\sigma^\eta(z)
   + \sum_{\ell(\nu) < \ell(\sigma)} 
      D_{\nu, \sigma}^\eta(\gamma_{k,j}(\lambda^z)) D_\nu^\eta(z)
\end{align*}
We summarize our findings in the following proposition.

\begin{Proposition}
\label{P:generic-eta-module}
Let $\eta$ be a partition of $\mu$. For each $f \in B_\eta$ write $\overline 
f$ for its image in $A_\eta$.

For each $\sigma, \nu \in S_\mu$ there exist $g^+_{\sigma,\nu}, 
g^-_{\sigma,\nu} \in B_\eta^{S_\eta}$ such that the action of $U$ on the 
module $L_\eta$ is given by
\begin{align*}
  E_{k,k+1} \D_\sigma(z) 
    &= \sum_{\ell(\nu) \leq \ell(\sigma) + \ell(w_{\epsilon(z)})}  
    \overline g^+_{\sigma, \nu}(\lambda + z)
      \D_\nu(z + \delta^{k,\alpha_i^{(k)}(\epsilon(z))});\\
  E_{k+1,k} \D_\sigma(z) 
    &= \sum_{\ell(\nu) \leq \ell(\sigma) + \ell(w_{\epsilon(z)})} 
      \overline g^-_{\sigma, \nu}(\lambda + z) \D_\nu(z-\delta^{k,
        \beta_i^{(k)}(\epsilon(z))});\\
  E_{k,k} \D_\sigma(z) 
    &= (|\overline \lambda^z|_k - |\overline \lambda^z|_{k-1} + k -1) 
      \D_\sigma(z).
\end{align*} 
Furthermore $c_{k,j} \D_\sigma(z) = \overline \gamma_{k,j}(\lambda + z) 
\D_\sigma(z) + \sum_{\ell (\nu) < \ell(\sigma)} \overline{D_{\nu, \sigma}
(\gamma_{k,i}(\lambda + z))} \D_\nu(z)$.
\end{Proposition}
\begin{Remark}
Suppose $\epsilon(z) = 1^\eta$, so $S_\epsilon$ is the trivial group. Then the 
coefficient of $\overline D_{\sigma}^\eta(z+\delta^{k,i})$ in $E_{k,k+1} 
\overline D_{\sigma}^\eta(z)$ is $e_{k,i}^+(\overline \lambda + z)$. However
it may happen that $z + \delta^{k+i}$ is singular, and hence some of its 
derived tableaux may be zero.
\end{Remark}

\begin{Definition}
Let $v \in \CC^\mu$ be an $\eta$-critical point. We define $V(T(v))$ to be the
$U$-module $\CC_v \ot_{A_\eta} L_\eta$, and for each $z \in \ZZ^\mu_0$ we set 
$D_\sigma(v+z) = 1 \ot_{A_\eta} \D_\sigma(z)$.
\end{Definition}
We are now ready to prove the main result of this article.

\begin{Theorem}
Let $\eta$ be a refinement of $\mu$, and let $v$ be an $\eta$-critical point.
The module $V(T(v))$ is a Gelfand-Tsetlin module, and its support consists of 
all characters of the form $\chi_{v+z}$ with $z \in \ZZ^\eta_0$. Furthermore 
the character $\chi_{v+z}$ appears with multiplicity $\eta!/\epsilon(z)!$ in
$V(T(v))$.
\end{Theorem}
\begin{proof}
By definition the set $\{D_\sigma(v+z) \mid z \in \N(\eta), \sigma \in 
\Shuffle_{\epsilon(z)}^\eta\}$ forms a basis of $V(T(v))$. The action of 
$\Gamma$ on $V(T(v))$ is given by the formulas of Proposition 
\ref{P:generic-eta-module} evaluated at $v$, and from this it follows that
$(c_{k,i} - \gamma_{k,i}(v+z))^{\ell(\sigma)}D_\sigma(v+z) = 0$, so 
$D_\sigma(v+z)$ is a generalized eigenvector of eigenvalue $\gamma_{k,i}(v+z)$.
It follows that $V(T(v)) = \bigoplus_{z \in \N(\eta)} V(T(v))[\chi_{v+z}]$
and that $V(T(v))[\chi_{v+z}]$ has the set $\{D_\sigma(v+z) \mid \sigma \in 
\Shuffle_{\epsilon(z)}^\eta\}$ as a basis, so the multiplicity of $\chi_{v+z}$
is $|\Shuffle_{\epsilon(z)}^\eta| = |S_\eta/S_{\epsilon(z)}| = 
\eta!/\epsilon(z)!$, as stated.
\end{proof}

\subsection{Example: the $3$-singular case}
In this subsection we fix $n = 4$ and study the case $\eta = (1,1^2, 3, 1^4)$.
The following table shows the nonzero derived tableaux of $T(v+z)$, classified
according to the partition $\epsilon^\eta(z)$; we always assume $a > b > c$.

\begin{tabular}{|c|c|l|}
\hline
$z_3$ & $\epsilon^\eta(z)^{(3)}$ & Nonzero derived tableaux \\
\hline
$(a,b,c)$
  & $(1,1,1)$
  & \parbox[c]{9cm}
    {$D_e(v+z), D_{(12)}(v+z), D_{(23)}(v+z),$\\ $D_{(123)}(v+z), 
    D_{(132)}(v+z), D_{(13)}(v+z)$} \\
\hline
$(a,a,b)$
  & $(2,1)$
  & $D_e(v+z), D_{(23)}(v+z), D_{(123)}(v+z)$ \\
\hline
$(a,b,b)$
  & $(1,2)$ 
  & $D_e(v+z), D_{(12)}(v+z), D_{(132)}(v+z)$ \\
\hline
$(a,a,a)$
  & $(3)$
  & $D_e(v+z)$ \\
\hline
\end{tabular}

Let us calculate as an example $E_{3,4} D_e(v)$, with $v_3 = (a,a,a)$. 
Now $E_{3,4} D_e(v)$ is a linear combination of derived tableaux $D_\nu
(v + \delta^{3,1})$, and according to the table above we can only have $\nu \in
\{e, (12), (132)\}$. In this case $\epsilon^\eta(v) = \eta$, and hence 
\begin{align*}
c_{\nu,e,\rho}^{\eta,\epsilon} 
&= \sym_\eta \left(
        (\partial_{\rho^{-1}(13)}\Delta_\eta)
        (\partial_{\nu^{-1}(13)}\Delta_\eta)
    \right) \\
& = \sym_\eta (\partial_{(13)\rho} \partial_{\nu^{-1}(13)} \Delta_\eta)\equiv
    \begin{cases}
    1 & \mbox{ if $\rho = (13) \nu$} \\
    0 & \mbox{otherwise}
    \end{cases}
    \mod \p_B 
\end{align*}
It follows that the coefficient of $D_e(v+\delta^{k,1})$ is
\begin{align*}
3 D_{(13))}^\eta(e_{3,1}^+(\lambda) \Delta)
  &= 3 \sym_{S_3} (e_{3,1}^+(\lambda))
  = \frac{1}{2} \frac{\partial^2 p_{3,1}^+}{\partial \lambda_{3,1}^2}.
\end{align*}

By a similar analysis the coefficient of $D_{(12)}(v+\delta^{3,1})$ is
\begin{align*}
3 \left( D_{(123)}^\eta(e_{3,1}^+(\lambda) \Delta_\eta)\right)
    = 6 \frac{\partial p_{3,1}^+}{\partial \lambda_{3,1}},
\end{align*}
and that of $D_{(132)}(v+\delta^{3,1})$
\begin{align*}
3 \left(D_{(23)}^\eta(e_{3,1}^+(\lambda) \Delta_\eta)\right)
    = 6 p_{3,1}^+,
\end{align*}
so
\begin{align*}
E_{3,4} \D_e(v) 
  = \frac{1}{2} \frac{\partial^2 p_{3,1}^+}{\partial \lambda_{3,1}^2}(v) 
    \D_e(v)
  + 6 \frac{\partial p_{3,1}^+}{\partial \lambda_{3,1}}(v) 
    \D_{(12)}(v+\delta^{3,1})
  + 6 p_{3,1}^+(v) \D_{(132)}(v+\delta^{3,1}).
\end{align*}

For a different example, we can look at $E_{4,3} D_{(13)}(v+z)$ with $z_3 = 
(1,0,-1)$ and all other entries equal to zero. In this case $\epsilon(z) = 
1^\mu$, so
\begin{align*}
c_{\nu,(13),\rho}^{\eta,\epsilon} 
  &= \sym_{S_3} \left( \frac{
    \partial_{\rho^{-1}(13)}\Delta \partial_{\nu^{-1}(13)} \Delta}{\Delta}
    \right) = \sym_{S_3}(\partial_{(13)\rho}\partial_{\nu^{-1}(13)} \Delta)
\end{align*}
so once again we only consider $\rho = (13)\nu$, and so the coefficient of
$D_{\nu}(v+\delta^{k,i})$ is given by
\begin{align*}
D_{(13)\nu}^\eta(e^-_{k,i}(\lambda + z)) 
  &\equiv \partial_{(13)\nu}(e^-_{k,i}(\lambda + z))(v+z) \mod \p_B.
\end{align*}
With this, a direct computation shows that
\begin{align*}
E_{4,3} &\D_{(13)}(v+z)
  = \frac{1}{2} \frac{\partial p_{3,1}^-}{\partial \lambda_{1,3}}(v+z)
    \D_{(123)}(v+z-\delta^{3,1}) + \frac{1}{2} \D_{(12)}(v+z-\delta^{3,1})\\
    &+ \frac{1}{4} \D_{e} (v+z-\delta^{3,1}) -2 \left( 3 p^-_{2,3}(v+z) - 
    1\right) \D_{(13)}(v+z-\delta^{3,2})\\
    &+ \D_{(12)}(v+z-\delta^{3,2}) - 
    \frac{\partial p_{3,2}^-}{\partial\lambda_{3,2}}(v+z) \D_e(v+z-
    \delta^{3,2})\\
    &+ \frac{1}{2}p_{3,3}^-(v+z) \D_{(13)}(v+z-\delta^{3,3}) 
    + \frac{1}{4}p_{3,3}^-(v+z) \D_{(123)}(v+z-\delta^{3,3})  \\
    &+ \frac{1}{2}\frac{\partial p_{3,3}^-}{\partial \lambda_{3,3}}(v+z) 
      \D_{(132)}(v+z-\delta^{3,3}) + \frac{1}{4}
      \frac{\partial p_{3,3}^-}{\partial \lambda_{3,3}}(v+z) 
      \D_{(23)}(v+z-\delta^{3,3}) \\
    &+ \frac{1}{2}\D_{(12)}(v+z-\delta^{3,3}) + \frac{1}{4}
    \D_{e}(v+z-\delta^{3,3})
\end{align*}

Notice that six derived tableaux are missing in this expression, for 
instance $\D_{(23)}(v+z-\delta^{2,2})$. This is not because the corresponding
coefficient is zero but becuase the tableaux itself is zero; this is due to 
the fact that the entries in $z$ are consecutive integers and hence the 
stabilizer of $v+z-\delta^{2,2}$ is not trivial. If we took 
$z' \in \ZZ^\eta_0$ with $z'_3 = (0, 3, 6)$, the expression for $E_{4,3} 
\D(v+z)$ would involve six more terms than the ones above.
\begin{bibdiv}
\begin{biblist}
\bibselect{biblio}
\end{biblist}
\end{bibdiv}
\end{document}
