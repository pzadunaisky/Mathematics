%%%%%%%%%%%%%%%%%%%%%% Generalities %%%%%%%%%%%%%%%%%%5
\documentclass[11pt,fleqn]{article}
\usepackage[paper=a4paper]
  {geometry}

\pagestyle{plain}
\pagenumbering{arabic}
%%%%%%%%%%%%%%%%%%%%%%%%%%%%%%%%
\usepackage[small]{titlesec}
%\usepackage{paragraphs}

\usepackage{hyperref}

\usepackage{amsthm,thmtools}
%\usepackage{showlabels}
%\linespread{1.2}
\setlength{\parskip}{1.2ex}

\usepackage[utf8]{inputenc}
\usepackage[english]{babel}
\usepackage{enumerate}
\usepackage[osf,noBBpl]{mathpazo}
\usepackage[alphabetic,initials]{amsrefs}
\usepackage{amsfonts,amssymb,amsmath}
\usepackage{mathtools}
\usepackage{graphicx}
\usepackage[poly,arrow,curve,matrix]{xy}
\usepackage{wrapfig}
\usepackage{xcolor}
\usepackage{helvet}
\usepackage{stmaryrd}
\usepackage[normalem]{ulem}

\renewcommand\labelitemi{--}


%%%%%%%%%%%%Theorems, for paragraphs package%%%%%%%%%%%%%%%%%%%%%%%%%%
% numbered versions

\newskip\paraskip
\paraskip=0.75ex plus .2ex minus .2ex

\newcounter{para}[section]
\setcounter{para}{0}
\renewcommand\thepara{\thesection.\arabic{para}}
\def\paragraph{%
  \if@noskipsec \leavevmode \fi
  \par
  \if@nobreak
    \everypar{}%
  \else
    \addpenalty{\@secpenalty}%
    \addvspace{\paraskip}%
  \fi
  \noindent
  \refstepcounter{para}%
  \textbf{\thepara.}\hspace{1ex}%
  \@nobreakfalse
  \ignorespaces
}

\newcommand\pref[1]{\textbf{\ref{#1}}}

%\newcommand\theHpara{\theHsection.\arabic{para}}

\declaretheoremstyle[headformat=swapnumber, spaceabove=\paraskip,
bodyfont=\itshape]{mystyle}
\declaretheorem[name=Lemma, sibling=para, style=mystyle]{Lemma}
\declaretheorem[name=Proposition, sibling=para, style=mystyle]{Proposition}
\declaretheorem[name=Theorem, sibling=para, style=mystyle]{Theorem}
\declaretheorem[name=Corollary, sibling=para, style=mystyle]{Corollary}
\declaretheorem[name=Definition, sibling=para, style=mystyle]{Definition}
%\declaretheorem[name=Examples, sibling=para, style=mystyle]{Example}
%\declaretheorem[name=Remark, sibling=para, style=mystyle]{Remark}

% unnumbered versions
\declaretheoremstyle[numbered=no, spaceabove=\paraskip,
bodyfont=\itshape]{mystyle-empty}
\declaretheorem[name=Lemma, style=mystyle-empty]{Lemma*}
\declaretheorem[name=Proposition, style=mystyle-empty]{Proposition*}
\declaretheorem[name=Theorem, style=mystyle-empty]{Theorem*}
\declaretheorem[name=Corollary, style=mystyle-empty]{Corollary*}
\declaretheorem[name=Definition, style=mystyle-empty]{Definition*}
\declaretheorem[name=Examples, style=mystyle-empty]{Example*}
\declaretheorem[name=Remark, style=mystyle-empty]{Remark*}

% plain style
\declaretheoremstyle[
        headformat={{\bfseries\NUMBER.}{ \bfseries\NAME}},
        spaceabove=\paraskip, 
        headpunct={. },
        headfont=\bfseries,
        bodyfont=\normalfont
        ]{mystyle-plain}
\declaretheorem[name=Ejemplo, sibling=para, style=mystyle-plain]{Example}
\declaretheorem[name=Observaci\'on, sibling=para, style=mystyle-plain]{Remark}

\makeatletter
\renewenvironment{proof}[1][\textit{Proof}]{\par
  \pushQED{\qed}%
  \normalfont \topsep.75\paraskip\relax
  \trivlist
  \item[\hskip\labelsep
        \itshape
    #1\@addpunct{.}]\ignorespaces
}{%
  \popQED\endtrivlist\@endpefalse
}
\makeatother

\usepackage{tikz}
\usepackage{mathdots}
%%%%%%%%%%%%%%%%%%%%%%%%%%% The usual stuff%%%%%%%%%%%%%%%%%%%%%%%%%
\newcommand\NN{\mathbb N}
\newcommand\CC{\mathbb C}
\newcommand\QQ{\mathbb Q}
\newcommand\RR{\mathbb R}
\newcommand\ZZ{\mathbb Z}
\renewcommand\k{\Bbbk}

\newcommand\F{\mathcal F}
\newcommand\V{\mathcal V}
\newcommand\D{\mathcal D}
\renewcommand\H{\mathcal H}
\newcommand\I{\mathcal I}
\newcommand\p{\mathfrak p}


\newcommand\maps{\longmapsto}
\newcommand\ot{\otimes}
\renewcommand\to{\longrightarrow}
\renewcommand\phi{\varphi}
\newcommand\Id{\mathsf{Id}}
\newcommand\im{\mathsf{im}}
\newcommand\coker{\mathsf{coker}}
%%%%%%%%%%%%%%%%%%%%%%%%% Specific notation %%%%%%%%%%%%%%%%%%%%%%%%%
\newcommand\g{\mathfrak g}
\newcommand\gl{\mathfrak{gl}}
\newcommand\gen{\mathsf{gen}}
\newcommand\std{\mathsf{std}}

\newcommand\vectspan[1]{\left\langle #1 \right\rangle}
\newcommand\interval[1]{\llbracket #1 \rrbracket}

\DeclareMathOperator\End{End}
\DeclareMathOperator\Specm{Specm}

\DeclareMathOperator\sym{sym}
\DeclareMathOperator\asym{asym}
\DeclareMathOperator\sg{sg}
\DeclareMathOperator\Frac{Frac}

%%%%%%%%%%%%%%%%%%%%%%%%%%%%%%%%%%%%%% TITLES %%%%%%%%%%%%%%%%%%%%%%%%%%%%%%
\title{Singular Gelfand-Tsetlin modules
\footnote{
  This is a DRAFT for a section to be included in a future article with 
  Futorny et al. This section will contain the construction of what I think is 
  the universal Gelfand-Tsetlin module associated to an arbitrary GT-tableaux, 
  and all the information we can obtain about it.
  }
}

\author{Pablo Zadunaisky
\footnote{Instituto de Matem\'atica e Estat\'istica, Universidade de S\~ao
Paulo,  S\~ao Paulo SP, Brasil. \texttt{email:} pzadun@ime.usp.br.
PZ is a FAPESP PostDoc Fellow, grant: 2016-25984-1 
S\~ao Paulo Research Foundation (FAPESP).}
}

\date{05/05/2017}

\begin{document}
\maketitle

\section*{Notation}

All the notation included here should be in the final article.

Fix $n \in \NN_{\geq 2}$ and $N = \binom{n}{2}$.

$U = U(\gl(n,\CC))$, $\Gamma \subset U$ the Gelfand-Tsetlin algebra associated
with inclusion in the top right corner. $c_{k,i}$ the usual generators of 
$\Gamma$, $\gamma_{k,i}$ the symmetric polynomials that give the action.

$e_{k,i}^\pm, e_k$ the rational functions appearing in the GT-formulas.

$\CC(\lambda) = \CC[\lambda_{k,i} \mid 1 \leq i \leq k \leq n]$, and 
$\CC(\lambda_m) = \CC[\lambda_{m,i} \mid 1 \leq i \leq m] \subset \Lambda$
for each $m \in \interval{n}$.

$G = S_1 \times S_2 \times \cdots \times S_n$.

$G$ acts on $\CC(\lambda)$ as follows: an element $\sigma \in S_m$ permutes 
the variables in $\CC(\lambda_m)$ in the obvious way and fixes all others. 
Given $\sigma \in S_m$ and $f \in \CC(\lambda_m)$, set $f^\sigma = \sigma 
\cdot f$.

$\ZZ^N$ acts on $\CC(\lambda)$ as follows: for each $z \in \ZZ^N$, set $(z 
\cdot f)(\lambda) = f(\lambda + z)$. Sometimes we write $f(\lambda^z) = z \cdot
f$.

$\Lambda_m = \CC(\lambda_m)^{S_m}$. 

$J_m \subset \CC(\lambda_m)$ is the ideal generated by symmetric polynomials of
positive degree in $\CC(\lambda_m)$.

We identify $\CC^N$ with the space of all tableaux. Write $\ZZ^N$ for integral 
tableaux, $\ZZ^N_0$ for integral tableaux with top row $0$, $\CC^N_\gen$ for 
generic tableaux, $\CC^N_\std$ for standard tableaux.

\section{The universal Gelfand-Tsetlin module associated to a tableaux}

Let $v \in \CC^N$ be an arbitrary tableaux, and let $\chi$ be the character of 
$\Gamma$ associated to $v$. Our objective is to build a Gelfand-Tsetlin module 
$M$ such that $M(\chi) \neq 0$. 

\begin{Definition}
We denote by $V_\CC$ the $\CC$ vector-space generated by the set 
\[
  \{T(z) \mid z \in \ZZ^N_0\}.
\]
For any $\CC$-algebra $R$ we write $V_R = R \ot_\CC V_\CC$.
\end{Definition} 

Let $A$ be the localization of $\CC(\lambda)$ at the infinite set 
$\{\lambda_{k,i} - \lambda_{k,j} - a \mid 1 \leq i < j \leq k < n, a \in 
\ZZ\}$. This is the algebra of rational functions over $\CC^N_\gen$, Notice 
that the algebra $A$ contains the rational functions $e_{k,i}^\pm, e_k$. 
For each $v \in \CC_\gen$ there is a $1$-dimensional representation of $A$ 
which we denote as $\CC_v$; we fix a generator of this $1$-dimensional module, 
which we denote by $1_v$. The free $A$-module $V_A$ can be endowed with the 
structure of a $U$-module. The following result is 
\cite{Zad-1-sing}*{Proposition 1}.

\begin{Proposition}
\label{P:universal-generic-GT-module}
The $A$-module $V_A$ can be endowed with the structure of a $U$-module 
with the action of the canonical generators given by
\begin{align*}
E_{k,k+1} T(z) 
  &= - \sum_{i=1}^k e_{k,i}^+(\lambda^z) T(z + \delta^{k,i}); \\
E_{k+1,k} T(z) 
  &= \sum_{i=1}^k e_{k,i}^-(\lambda^z) T(z - \delta^{k,i}); \\
E_{k,k} T(z)
  &= e_k(\lambda^z) T(z),
\end{align*}
and for each $1 \leq i \leq k \leq n$, we have $c_{k,i} T(z) = 
\gamma_{k,i}(\lambda^z) T(z)$. 
\end{Proposition}

Given $v \in \CC^N_\gen$, the Gelfand-Tsetlin module associated to the generic 
tableaux $T(v)$ was constructed by Drozd, Futorny and Ovsienko in 
\cite{DFO-GT-modules}. This module, denoted by $V(T(v))$, is a $\CC$-vector 
space generated by all tableaux $\{T(v+z) \mid z \in \ZZ^N_0\}$, and the 
action of $U$ is given by the Gelfand-Tsetlin formulae. Now the $A$-module 
$V_A$, along with its $U$-module structure, can be thought of as a generic 
version of a module associated to a generic tableauxin the following sense. 
The $\CC$-vector space $V_A \ot_A \CC_v$ inherits a $U$-module structure from 
$V_A$, and it has a $\CC$-basis consisting of all vectors $T(z) \ot_A 1_v$. 
By definition 
\[
  c_{k,i} T(z) \ot_A 1_v 
    = \gamma_{k,i}(\lambda^z) T(z) \ot_A 1_v 
    = \gamma_{k,i}(v + z) T(z) \ot_A 1_v,
\]
so $T(z) \ot 1_v$ is an eigenvector of eigenvalue $v + z$. The assignation 
$T(z) \ot 1_v \in V_A \ot_A \CC_v \mapsto T(v+z) V(T(v))$ is an isomorphism of 
$U$-modules. \textcolor{red}{There is a unique irreducible GT-module such that 
$V(\chi_v) \neq 0$, which thus must be a subquotient of $V(T(v))$, and thus 
all these modules can be obtained through $V_A$ by specialization and taking 
subquotients.}

An ideal objective would be to obtain a universal GT-module $V$, such that for
all tableaux $v \in \CC^N$ we could obtain a module $V(T(v))$ by some form of
specialization at $v$. We will give a construction of such a module in this 
section, and realize it as a submodule of $V_A$.

\begin{Definition}
We denote by $B$ the localization of $\CC(\lambda)$ at the infinite set
$\{\lambda_{k,i} - \lambda_{k,j} - a \mid 1 \leq i < j \leq k < n, a \in 
\ZZ \setminus \{0\}\}$.
\end{Definition}
Let $v \in \CC^N$. We will say that $v$ is $\emph{fully-singular}$ if whenever
$v_{k,i} - v_{k,j} \in \ZZ$ for some $1 \leq i < j \leq k < n$, then $v_{k,i}
- v_{k,j} = 0$. Notice that if $v \in \CC^N$ is arbitrary there always exists
$z \in \ZZ^N_0$ such that $v+z$ is fully singular.

If $v$ is fully-singular and $f \in B$ then $f(v)$ is well defined, and so we 
get a one dimensional representation of $B$, which we will also denote by 
$\CC_v$; extending the previous notation, we fix a generator $1_v \in \CC_v$. 
If $v \in \CC^N_\gen$ then it is fully singular and the $B$-module $\CC_v$ is 
the restriction of the $A$-module $\CC_v$, so the slight abuse of notation in 
this definition is well justified. We will find our universal module $L_B$ as 
a complete $B$-lattice of $V_A$, and for each fully singular $v \in \CC^N$
we will set $V(T(v)) = L_B \ot_B \CC_v$. The support of this module will
contain all characters of the form $\chi_{v+z}$ with $z \in \ZZ^N_0$, and hence
every character will appear in one of these modules.

\subsection{The action of the symmetric group}
Recall that $G = S_1 \times S_2 \times \cdots \times S_n$; we will see $S_k$ 
as a subgroup of $G$ for all $k \in \interval{n}$. For any $g \in G$ and any 
$k \in \interval n$ we write $g[k]$ for the projection of $g$ on $S_k$; also
given a subgroup $E < G$ we will write $E_k$ for $E \cap S_k$, so $E = E_1 
\times E_2 \times \cdots \times E_n$.

Set $\I = \{(k,i) \mid 1 \leq i \leq k \leq n\}$. For each $\sigma \in S_m$ set
$\sigma(m,i) = (m,\sigma(i))$ for all $i \in \interval m$ and $\sigma(k,i) = 
(k,i)$ whenever $k \neq m$ and $i \in \interval k$. This defines an action of 
$G$ on $\I$. Thus $G$ acts on the algebra $\CC(\lambda)$ by $g\cdot 
\lambda_{k,i} = \lambda_{g(k,i)}$ (we are making a slight abuse of notation 
by setting $\lambda_{(k,i)} = \lambda_{k,i}$). This action extends
to the field of rational functions $F = \Frac(\CC(\lambda))$, and the 
subalgebras $A$ and $B$ defined above are stable by the action of $G$. We 
sometimes write $f^g$ for $g \cdot f$; with this notation we get $(f^g)^{g'} = 
f^{g'g}$.

The group $G$ also acts on the space of all tableaux by $g \cdot \delta^{k,i} 
= \delta^{g(k,i)}$, and this action extends to an action on $V_\CC$. Finally,
since $V_A = A \ot_\CC V_\CC$, it follows that $G$ acts diagonally on $V_A$,
i.e. $g \cdot f \ot T(z) = f^g \ot T(g \cdot z)$. The next lemma gathers
some useful facts relating all these actions.

\begin{Lemma}
The following hold.
\label{L:G-equivariant}
\begin{enumerate}[(a)]
\item \label{G-on-shifts}
For each $g \in G, f \in \Frac(\CC(\lambda))$ and $v \in \CC^N$ we have
$f^g (\lambda^z) = (g \cdot f)(\lambda^{g(z)})$.

\item \label{G-on-GT-functions}
For each $g \in G$ and each $(k,i) \in \I$, we have that $g \cdot 
e_{k,i}^\pm = e_{g(k,i)}$ and $g \cdot e_k = e_k$.

\item \label{G-U-equivariant}
The action of $U$ on $V_A$ is $G$-equivariant.
\end{enumerate}
\end{Lemma}
\begin{proof}
To prove items (\ref{G-on-shifts}) and (\ref{G-on-GT-functions}) it is enough 
to show that the formulas hold for $g = \sigma \in S_k$ for some $k \in 
\interval n$, and in both cases the verification is direct. 

Now for any $v \in \CC^N, g \in G$ and $k \in \interval n$ we have
\begin{align*}
g \cdot (E_{k,k+1} T(v))
  &= - \sum_{i = 1}^k g \cdot 
    \left(e_{k,i}(\lambda^v) T(v + \delta^{k,i}) \right)
    = -\sum_{i=1}^k e_{g[k](k,i)} T((g \cdot v)+\delta^{g[k](k,i)}).
\end{align*}
The element $g[k]$ is a permutation of all $(k,i)$ with $k$ fixed, so this 
equals
\begin{align*}
- \sum_{i=1}^k e_{k,i}(\lambda^{g \cdot v}) T((g \cdot v)+\delta^{k,i}) = 
E_{k,k+1} T(g \cdot v).
\end{align*}
A similar proof works for $E_{k+1,k}$ and $E_{k,k}$.
\end{proof}

\subsection{Divided diferences}
Fix $m \in \interval{n}$ and write $x_i = \lambda_{m,i}$. By definition $\CC[x]
= \CC[x_1, \ldots, x_m] \subset \CC[\lambda]$. The symmetric group $S_m$ acts 
on $\CC[x]$ in the obvious way. We denote by $\Delta$ the polynomial
$\prod_{i<j} x_i - x_j$. Clearly for each $i \in \interval m$ we have 
$\Delta^{(i,i+1)} = - \Delta$, so $\Delta^\sigma = (\sg \sigma) \Delta$, and
if $f \in \CC[x]$ is such that $f^\sigma = (\sg \sigma) f$ then $f = p \Delta$,
where $p \in \CC[x]^{S_m}$.

Let $F = \CC(x) = \Frac(\CC[x])$; clearly the action of $S_m$ extends to $F$
For each $i \in \interval{m-1}$ let $s_i = (i,i+1) \in S_m$ and let $\partial_i
= \frac{1}{x_i - x_{i+1}}(1 - s_i) \in F \# S_m$. Given $\sigma \in 
S_m$, define the divided difference operator $\partial_\sigma$ as in 
\cite{Man-symm-book}*{Defintion 2.3.1}. Notice that $\CC[x] \# S_m$ is a 
subalgebra of $F \# S_m$; we denote by $\CC[x, \partial]$ the $\CC[x] \# S_m$
subalgebra of $F \# S_m$ generated by divided differences.

Let $M$ be an $S_m$-equivariant $\CC[x]$-module. We will say that $M$ is 
divided difference $\CC[x]$-module if $M$ is a $\CC[x,\partial]$-module.
For example, $\CC[x]$ is a divided difference $\CC[x]$-module, and so is the
$U$-module $V_A$ described above. If $S \subset \CC[x]$ is a multiplicatively
closed subset and is also closed by the action of $S_m$ then $S^{-1}\CC[x]$
is a divided difference $\CC[x]$-module.

Set
\begin{align*}
\sym 
  &= \frac{1}{m!}\sum_{\sigma \in S_m} \sigma; 
  &\asym 
  &= \frac{1}{m!}\sum_{\sigma \in S_m} \sg(\sigma) \sigma.
\end{align*}
These elements lie in $\CC[x] \# S_m$. If $M$ is a $\CC[x]$-module with an 
equivariant $S_m$-action, then multiplication by $\sym$ is the projection to 
the symmetric component of $M$ and multiplication by $\asym$ is the projection
to its antysimmetric component. If $R$ is a $\CC[x]$-subalgebra of $F$ stable 
by the action of $S_m$ then for each $f \in R$ we have $\asym(f)/\Delta \in R$.

For the rest of this section we denote by $\p$ the ideal generated by the 
elements $p_i = x_i - x_{i+1}$ for $1 \leq i \leq m-1$. Clearly this is a prime
ideal. The localization $\CC[x]_\p$ consists of all rational functions in $F$
that can be evaluated at a point of the form $(a,a, \ldots, a)$.

\begin{Lemma}
Let $f,g \in \CC[x]_\p$ and let $\sigma, \tau \in S_m$. The following hold.
\begin{enumerate}
\item $\sym(f) \equiv f \mod I$.

\item $\asym((\partial_\sigma f) g) = \asym(f (\partial_{\sigma^{-1}}g))$.

\item $\asym(\partial_{\sigma^{-1}} \Delta \partial_\tau \Delta) = 
\begin{cases}
\Delta \sym(\partial_{\sigma \tau} \Delta) & \mbox{ if } 
\ell(\sigma) + \ell(\tau) = \ell(\sigma\tau) \\
0 & \mbox{otherwise}.
\end{cases}$

\item $\partial_\sigma \Delta \equiv \delta_{\sigma, w_0} \mod I$.
\end{enumerate}
\end{Lemma}
\begin{proof}
Let $t = (a, a, \ldots, a)$, so $\tau \cdot t = t$. Then $\sym(f)(t) = f(\sym 
\cdot t) = f(t)$, so $\sym(f) - f$ is zero over the variety $V(\p)$, which 
implies $\sym(f) \equiv f \mod \p$. 

Notice that if $f^{(i,i+1)} = f$ then $\asym(f) = \frac{\asym(f) + 
\asym(f^{(i,i+1)})}{2} = 0$, so in particular $\asym(\partial_{s_i}(f)) = 0$.
Now $\partial_{s_i}(fg) = \partial_{s_i}(f)g + f^{s_i}\partial_{s_i})g)$, so
\begin{align*}
0 
  &= \asym(\partial_i(fg)) 
  = \asym(\partial_i(f) g) + \asym(f^{s_i}\partial_i(g))\\
  &= \asym(\partial_i(f) g) - \asym((f^{s_i}\partial_i(g))^{s_i})\\
  &= \asym(\partial_i(f) g) - \asym(f\partial_i(g))
\end{align*}
so $\asym(\partial_i(f)g) = \asym(f \partial_i(g))$. The result follows by 
induction on the length of $\sigma$, and this immediately implies that 
$\asym(\partial_{\sigma^{-1}} \Delta \partial_\tau \Delta) = \asym(\Delta 
\partial_\sigma \partial_{\tau} \Delta$, and using the fact that 
$\Delta$ is antisymmetric we get that this equals $\Delta 
\sym(\partial_{\sigma^{-1}})\partial_{\tau} \Delta)$. The result follows
from the definition of symmetric differences.

Finally, Let $\CC[\p]$ be the subalgebra of $\CC$ generated by the ideal $\p$.
It is easy to check that $\CC[\p]$ is stable by $\partial_i$ and that this is
an operator of degree $-1$. Since $\Delta \in \CC[\p]$, so do all of its 
derived differences. Now $\partial_\sigma \Delta$ is a nonzero homogeneous 
polynomial of degree $\binom{n}{2} - \ell(\sigma)$, so it lies in $\p$ unless
$\sigma = w_0$.
\end{proof}

\subsection{Derived tableaux}
For each $i \in \interval{m-1}$ define $\partial_i: V_A \to V_A$ as the linear
extension of the assignation $\partial_i(T(v)) = \frac{T(v) - T(s_i \cdot v)}
{x_i - x_{i+1}}$. For each $\sigma \in S_m$ we define $\partial_\sigma$ by...

\begin{Definition}
Given $\sigma \in S_m$ and $v \in \ZZ_0^N$, we set $D_\sigma(v) = 
\sym(\partial_\sigma T(v)) \in V_A$.
\end{Definition}

\begin{Lemma}
For each $\sigma \in S_m$ we have
\begin{align*}
D_\sigma(v) 
  &= \sym \left(
    \frac{\partial_{\sigma^{-1}} \Delta}{\Delta} T(v)
  \right).
\end{align*}
Also for each $\tau \in S_m$ there exist $c_{\nu} \in \CC[x]$ such 
that $T(\rho(v)) = \sum_{\sigma \in S_m} c^\rho_\nu D_\nu(v)$.
Furthermore, $c^{\rho}_\nu \equiv (\partial_\nu \Delta)^\rho \mod 
\p^{S_m}$.
\end{Lemma}
\begin{proof}
The same proof as above shows that for each element $T \in V_A$ and each $f \in
A$ we have $\asym(f \partial_\sigma T) = \asym(\partial_{\sigma^{-1}}(f) T)$.
Thus
\begin{align*}
\sym(\partial_\sigma T(v))
  &= \frac{1}{\Delta}\asym(\Delta \partial_\sigma T(v))
  = \frac{1}{\Delta} \asym(\partial_{\sigma^{-1}}\Delta T(v)) 
  = \sym\left(\frac{\partial_{\sigma^{-1}}\Delta}{\Delta} T(v)\right).
\end{align*}

We denote by $R$ the ring of square matrices with coefficients in $F$ and rows 
and columns indexed by $S_m$; for each $\sigma, \rho \in S_m$ and each $X \in 
R$ we denote by $X^\sigma_\rho$ the entry in row $\sigma$ and column $\rho$ of 
$X$, so if $Y \in R$ then $(X Y)^\sigma_\rho = \sum_\nu X^\sigma_\nu 
Y^\nu_\rho$.

Let $X, Y \in R$ be the matrices defined by $X_\sigma^\rho = \left( 
\frac{\partial_{\sigma^{-1}} \Delta}{\Delta} \right)^\rho$, and $Y_\tau^\nu = 
(\partial_{\nu^{-1} w_0} \Delta)^\tau$. Then
\begin{align*}
(YX)^\nu_\sigma 
  &= \sum_\rho Y_\rho^\nu X^\rho_\sigma
  = \sum_{\rho} (\partial_{\nu^{-1} w_0} \Delta)^\rho 
    \left( 
      \frac{\partial_{\sigma^{-1}} \Delta}{\Delta} 
    \right)^\rho
  = \frac{\asym(\partial_{\nu^{-1} w_0} \Delta\partial_{\sigma^{-1}} \Delta)}
    {\Delta}
  = \sym(\partial_\sigma \partial_{\nu^{-1} w_0} \Delta) 
\end{align*}
If $\ell(\sigma) + \ell(\nu^{-1} w_0) > \ell(w_0)$, or equvalently 
$\ell(\sigma) > \ell(\nu)$, then $(YX)^{\nu}_\sigma = 0$. On the other hand, 
if $\ell(\sigma) + \ell(\nu^{-1} w_0) = \ell(w_0)$ then $\partial_\sigma 
\partial_{\nu^{-1} w_0} = 0$ unless $\sigma = \nu$. Thus if we order the 
elements of $S_m$ by length, then $YX$ is an upper triangular matrix with
$1$'s in the diagonal and polynomials of the form $\sym(\partial_\rho \Delta)$
in the upper triangular part; since $\partial_\rho \Delta \in \p$ and 
$\p$ is invariant by the action of $S_m$, it follows that the strictly
upper triangular part of $YX$ lies in $\p^{S_m}$. Thus $X$ is invertible in 
$R$, and its inverse is of the form $Y + Z$ with $Z$ a strictly upper 
triangular matrix with coefficients in $\p^{S_m}$.

Put $c_\sigma = (Y+Z)^\sigma_e$, where $e \in S_m$ is the identity element.
Since $\rho\cdot X^e_\sigma = X^{\rho}_\sigma$, we have that
$(Y+Z)^\nu_\rho = \rho \cdot (Y+Z)^\nu_e = c_\nu^\rho$. It follows 
that $c_\nu^\rho \equiv (\partial_{\nu^{-1} w_0}(\Delta))^\rho \mod \p^{S_m}$, 
and that
\begin{align*}
\sum_{\nu} c_\nu^\rho D_\nu(T(v))
  &= \sum_\nu c_\nu^\rho 
    \sum_\tau \left(
      \frac{\partial_{\nu^{-1}} \Delta}{\Delta}
    \right)^\tau T(\tau(v))\\
  &= \sum_\tau \left(
      \sum_\nu (Y+Z)^\nu_\rho X_\nu^\tau
    \right) T(\tau(v))\\
  &= \sum_\tau(\delta_{\rho,\tau})T(\tau(v)) 
  = T(\rho(v))
\end{align*}
\end{proof}


Let $E \subset S_m$ be the stabilizer of $v$. Then $D_\sigma(v) \neq 0$ if and 
only if $\sigma$ is the element of minimal length in $\sigma E$.

The set $\{D_\sigma(v) \mid \sigma \in S_m\} \setminus \{0\}$ is an
$A$-basis of the $A$-module with basis $\{T(\sigma(v)) \mid \sigma \in S_m\}$,
and the coefficients of $T(\sigma(v))$ in the basis of derived tableaux lie in
$\CC[x]$.

We denote by $B$ the localization of $\CC[x]$ at the inifinte set $\{x_i - x_j 
- z \mid 1 \leq i < j \leq m, z \in \ZZ \setminus \{0\}\}$. We denote by $I$ 
the ideal generated by $x_i - x_{i+1}$ in $B$ and set $\overline B = B/I$.

\begin{Definition}
Write $L_B(v)$ for the $B$-module generated by $\{D_\sigma(v) \mid \sigma \in 
S_m\}$, and $L_B = \bigoplus_{v \in \ZZ^N_0/S_m} L_B(v)$. Finally we set
$L = L_B \ot_B \overline B = L_B/I L_B = \bigoplus_v L_B(v) / I L_B(v)$.
\end{Definition}

Write $\overline T(v)$ and $\overline D_\sigma(v)$ for the image of 
$T(v)$ and $D_\sigma(v)$ in $L$, respectively.

\begin{Lemma}
$\overline T(\tau(v)) = \sum_\sigma (\partial_\sigma \Delta)^\tau \overline 
D_\sigma(v)$
\end{Lemma}

\begin{Lemma}
$L_B$ is a $U$-module.
\end{Lemma}
\begin{proof}
The only ones hard to check are $E^+ = E_{m,m+1}$ and $E^- = E_{m+1,m}$.
\begin{align*}
E^+ D_\sigma(v)
  &= \sym (\partial_\sigma E T(v))
  = \sym\left(
    \partial_\sigma 
      \left(
        \sum_j e_{m,j}^+(\lambda + v) T(v + \delta^{m,j})
      \right)
    \right)\\
  & \equiv \sym \left(
      \partial_\sigma \left(
        \sum_j e_{m,j}^+(\lambda + v) 
        \sum_\tau (\partial_\tau \Delta) D_\tau (v + \delta^{m,j})
      \right)
    \right) \\
  &= \sum_{j,\tau} \sym(\partial_\sigma (e_{m,j}^+(\lambda+v)) 
    \partial_{\tau} \Delta) D_\tau(v+\delta^{m,j}) \\
  &\equiv \sum_{\tau, j}  \partial_{\tau^{-1}} 
      \partial_\sigma (e_{m,j}^+(\lambda + v) \Delta) D_\tau(v+\delta^{m,j}).
\end{align*}
Since $e_{m,j}^+(\lambda + v) \Delta \in B$, so does 
$\partial_\sigma(e_{m,j}^+(\lambda + v) \Delta)$ for any $\sigma \in S_m$.
This completes the proof for $E^+$, the one for $E^-$ is analogous.
\end{proof}

\begin{bibdiv}
\begin{biblist}
\bib{DFO-GT-modules}{article}{
   author={Drozd, Yu. A.},
   author={Futorny, V. M.},
   author={Ovsienko, S. A.},
   title={Harish-Chandra subalgebras and Gel$\prime$fand-Zetlin modules},
   conference={
      title={Finite-dimensional algebras and related topics},
      address={Ottawa, ON},
      date={1992},
   },
   book={
      series={NATO Adv. Sci. Inst. Ser. C Math. Phys. Sci.},
      volume={424},
      publisher={Kluwer Acad. Publ., Dordrecht},
   },
   date={1994},
   pages={79--93},
}

\bib{Man-symm-book}{book}{
   author={Manivel, Laurent},
   title={Symmetric functions, Schubert polynomials and degeneracy loci},
   series={SMF/AMS Texts and Monographs},
   volume={6},
   note={Translated from the 1998 French original by John R. Swallow;
      Cours Sp\'ecialis\'es [Specialized Courses], 3},
   publisher={American Mathematical Society, Providence, RI; Soci\'et\'e
      Math\'ematique de France, Paris},
   date={2001},
   pages={viii+167},
}    

\bib{Zad-1-sing}{article}{
  author={Zadunaisky, Pablo},
  title={A new way to construct $1$-singular Gelfand-Tsetlin modules},
  journal={Alg. and Discrete Math.},
  volume={23},
  number={1},
  date={2017},
}


\end{biblist}
\end{bibdiv}
\end{document}