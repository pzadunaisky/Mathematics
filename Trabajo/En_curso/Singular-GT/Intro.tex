\documentclass[11pt,fleqn]{article}
\usepackage[paper=a4paper]
  {geometry}

\pagestyle{plain}
\pagenumbering{arabic}
%%%%%%%%%%%%%%%%%%%%%%%%%%%%%%%%
\usepackage[small]{titlesec}
%\usepackage{paragraphs}

\usepackage{hyperref}

\usepackage{amsthm,thmtools}
%\usepackage{showlabels}
%\linespread{1.2}
\setlength{\parskip}{1.2ex}

\usepackage[utf8]{inputenc}
\usepackage[english]{babel}
\usepackage{enumerate}
\usepackage[osf,noBBpl]{mathpazo}
\usepackage[alphabetic,initials]{amsrefs}
\usepackage{amsfonts,amssymb,amsmath}
\usepackage{mathtools}
\usepackage{graphicx}
\usepackage[poly,arrow,curve,matrix]{xy}
\usepackage{wrapfig}
\usepackage{xcolor}
\usepackage{helvet}
\usepackage{stmaryrd}
\usepackage[normalem]{ulem}

\renewcommand\labelitemi{--}


%%%%%%%%%%%%Theorems, for paragraphs package%%%%%%%%%%%%%%%%%%%%%%%%%%
% numbered versions

\newskip\paraskip
\paraskip=0.75ex plus .2ex minus .2ex

\newcounter{para}[section]
\setcounter{para}{0}
\renewcommand\thepara{\thesection.\arabic{para}}
\def\paragraph{%
  \if@noskipsec \leavevmode \fi
  \par
  \if@nobreak
    \everypar{}%
  \else
    \addpenalty{\@secpenalty}%
    \addvspace{\paraskip}%
  \fi
  \noindent
  \refstepcounter{para}%
  \textbf{\thepara.}\hspace{1ex}%
  \@nobreakfalse
  \ignorespaces
}

\newcommand\pref[1]{\textbf{\ref{#1}}}

%\newcommand\theHpara{\theHsection.\arabic{para}}

\declaretheoremstyle[headformat=swapnumber, spaceabove=\paraskip,
bodyfont=\itshape]{mystyle}
\declaretheorem[name=Lemma, sibling=para, style=mystyle]{Lemma}
\declaretheorem[name=Proposition, sibling=para, style=mystyle]{Proposition}
\declaretheorem[name=Theorem, sibling=para, style=mystyle]{Theorem}
\declaretheorem[name=Corollary, sibling=para, style=mystyle]{Corollary}
\declaretheorem[name=Definition, sibling=para, style=mystyle]{Definition}
%\declaretheorem[name=Examples, sibling=para, style=mystyle]{Example}
%\declaretheorem[name=Remark, sibling=para, style=mystyle]{Remark}

% unnumbered versions
\declaretheoremstyle[numbered=no, spaceabove=\paraskip,
bodyfont=\itshape]{mystyle-empty}
\declaretheorem[name=Lemma, style=mystyle-empty]{Lemma*}
\declaretheorem[name=Proposition, style=mystyle-empty]{Proposition*}
\declaretheorem[name=Theorem, style=mystyle-empty]{Theorem*}
\declaretheorem[name=Corollary, style=mystyle-empty]{Corollary*}
\declaretheorem[name=Definition, style=mystyle-empty]{Definition*}
\declaretheorem[name=Examples, style=mystyle-empty]{Example*}
\declaretheorem[name=Remark, style=mystyle-empty]{Remark*}

% plain style
\declaretheoremstyle[
        headformat={{\bfseries\NUMBER.}{ \bfseries\NAME}},
        spaceabove=\paraskip, 
        headpunct={. },
        headfont=\bfseries,
        bodyfont=\normalfont
        ]{mystyle-plain}
\declaretheorem[name=Example, sibling=para, style=mystyle-plain]{Example}
\declaretheorem[name=Remark, sibling=para, style=mystyle-plain]{Remark}

\makeatletter
\renewenvironment{proof}[1][\textit{Proof}]{\par
  \pushQED{\qed}%
  \normalfont \topsep.75\paraskip\relax
  \trivlist
  \item[\hskip\labelsep
        \itshape
    #1\@addpunct{.}]\ignorespaces
}{%
  \popQED\endtrivlist\@endpefalse
}
\makeatother

\usepackage{tikz}
\usepackage{mathdots}
%%%%%%%%%%%%%%%%%%%%%%%%%%% The usual stuff%%%%%%%%%%%%%%%%%%%%%%%%%
\newcommand\NN{\mathbb N}
\newcommand\CC{\mathbb C}
\newcommand\QQ{\mathbb Q}
\newcommand\RR{\mathbb R}
\newcommand\ZZ{\mathbb Z}
\renewcommand\k{\Bbbk}

\newcommand\F{\mathcal F}
\newcommand\V{\mathcal V}
\newcommand\D{\overline D}
\newcommand\N{\mathcal N}
\renewcommand\H{\mathcal H}

\newcommand\maps{\longmapsto}
\newcommand\ot{\otimes}
\renewcommand\to{\longrightarrow}
\renewcommand\phi{\varphi}
\newcommand\id{\mathsf{id}}
\newcommand\im{\mathsf{im}}
\newcommand\coker{\mathsf{coker}}
%%%%%%%%%%%%%%%%%%%%%%%%% Specific notation %%%%%%%%%%%%%%%%%%%%%%%%%
\newcommand\g{\mathfrak g}
\newcommand\p{\mathfrak p}
\newcommand\m{\mathfrak m}
\newcommand\gl{\mathfrak{gl}}
\newcommand\gen{\mathsf{gen}}
\newcommand\std{\mathsf{std}}
\newcommand\sh{\mathsf{sh}}

\newcommand\vectspan[1]{\left\langle #1 \right\rangle}
\newcommand\interval[1]{\llbracket #1 \rrbracket}
\newcommand\Shuffle{\mathsf{Shuffle}}

\DeclareMathOperator\Frac{Frac}
\DeclareMathOperator\Specm{Specm}

\DeclareMathOperator\sym{sym}
\DeclareMathOperator\asym{asym}
\DeclareMathOperator\sg{sg}
%%%%%%%%%%%%%%%%%%%%%%%%%%%%%%%%%%%%%% TITLES %%%%%%%%%%%%%%%%%%%%%%%%%%%%%%
\title{Gelfand-Tsetlin modules over $\gl(n)$ with arbitrary characters}

\author{L.E. Ram\'irez\footnote{Universidade Federal do ABC, Santo Andr\'e-SP, 
Brasil\texttt{email:} luis.enrique@ufabc.edu.br} , 
P. Zadunaisky\footnote{Instituto de Matem\'atica e Estat\'istica, Universidade 
de S\~ao Paulo,  S\~ao Paulo SP, Brasil. \texttt{email:} pzadun@ime.usp.br.
The author is a FAPESP PostDoc Fellow, grant: 2016-25984-1 
S\~ao Paulo Research Foundation (FAPESP).}
}


\begin{document}
\maketitle

\begin{abstract}
Associated with any Gelfand-Tsetlin character $\chi$, Ovsienko's Theorem (see \cite{Ov}?????) guarantees the existence of a Gelfand-Tsetlin module with $\chi$ on it's support. For some families of characters such modules have been constructed (finite dimensional \cite{GT-modules}, generic \cite{DFO-GT-modules} , $1$-singular \cite{FGR-1-singular}, index $2$ \cite{FGR-2-index}, among others), but not explicit construction of such modules was known for arbitrary characters. In this article we construct for any $\chi$ a Gelfand-Tsetlin module $V$ with explicit basis of tableaux that unify the previous known constructions, and give the multiplicity of each character in terms of combinatorial invariants of the associated Gelfand-Tsetlin tableaux.
\end{abstract}
\noindent\textbf{MSC 2010 Classification:} 17B10.\\
\noindent\textbf{Keywords:} Gelfand-Tsetlin modules, Gelfand-Tsetlin bases,
tableaux realization.

\section{Introduction}
The notion of a Gelfand-Tsetlin module (see Definition \ref{D:gt-modules}) has 
its origin in the classical article \cite{GT-modules}, where I. Gelfand and M. 
Tsetlin gave an explicit presentation of all finite dimensional irreducible 
representations of $\g = \gl(n,\CC)$ in terms of certain combinatorial 
objects, which have come to be known as \emph{Gelfand-Tsetlin tableaux}, or 
GT-tableaux for short. A GT-tableau is a triangular array of $\binom{n}{2}$ 
complex numbers, with $k$ entries in the $k$-th row; given a point $v \in 
\CC^{\frac{n(n+1)}{2}}$ we denote the corresponding array by $T(v)$. The group 
$G = S_1 \times S_2 \times \cdots \times S_n$ acts on the set of all tableaux, 
with $S_k$ permuting the elements in the $k$-th row. Gelfand-Tsetlin Theorem establish that any finite dimensional irreducible representation of $\g$ 
has a basis parameterized by GT-tableaux with integer entries satisfying certain betweenness relations. Identifying the elements of the basis with the 
corresponding GT-tableaux, the action of an element of $\g$ over a 
tableau is given by rational functions in its entries. The coefficients of
this action are known as the Gelfand-Tsetlin formulas; their poles form an 
infinite hyperplane array in $\CC^{\binom{n}{2}}$.

The enveloping algebra $U = U(\g)$ contains a large (indeed, maximal) 
commutative subalgebra $\Gamma$ called the \emph{Gelfand-Tsetlin} subalgebra 
of $U$. A \emph{Gelfand-Tsetlin module} is a $U$-module that can be
decomposed as the direct sum of generalized eigenvector spaces for $\Gamma$.
The characters of $\Gamma$ are in one-to-one correspondence with GT-tableaux modulo the action of $G$ (see \cite{Zh}???), and in the original construction of Gelfand and 
Tsetlin each tableau $T(v)$ is an eigenvector of $\Gamma$ whose eigenvalue is precisely the character $\chi_v: \Gamma \to \CC$ corresponding to $v$. Since no two tableaux in this construction are in the same $G$-orbit, the multiplicity of this character (i.e. the number of eigenvectors
of eigenvalue $\chi_v$) is one.

Starting from these observations, Y. Drozd, S. Ovsienko and V. Futorny 
introduced a large family of infinite dimensional $\g$-modules in 
\cite{DFO-GT-modules}. These GT-modules have a basis parameterized by Gelfand-Tsetlin tableaux with complex coefficients such that no pattern is a pole for the rational functions appearing in the GT-formulae (such tableaux 
are called \emph{generic}, hence the name ``generic Gelfand-Tsetlin module''). 
While each character in the decomposition of a generic Gelfand-Tsetlin module
appears with multiplicity one, there are examples of non-generic GT-modules
with higher multiplicities. These examples were first encountered in \cites{Fut1, Fut2} for $\mathfrak{sl}(3)$.

In \cite{FGR-1-singular} V. Futorny, D. Grantcharov and the first named author
constructed a GT-module with $1$-singular characters, i.e. characters 
associated to tableaux over which the Gelfand-Tsetlin formulas may have 
singularities of order at most $1$. These modules have a basis in terms of 
so-called \emph{derived tableaux}, new objects which, according to the authors 
``are not new combinatorial objects'' but rather formal objects in a large
vector space that contains classical GT-tableaux. These construction was 
expanded and refined in the articles \cites{FGR-2-index, Zad-1-sing, 
V-geometric-singular-GT} for characters with more general singularities. The 
aim of this article is to extend this construction to \emph{arbitrary} 
characters and begin the study of the associated modules. In the process we 
give a combinatorial interpretation of generalized derived tableaux.

The general idea of our construction is the following. Denote by $V$ the vector
space of arbitrary integral GT-tableaux with coefficients in the field of 
rational functions over (the entries of) GT-tableaux; this is a $U$-module 
with the action of $\g$ given by the rational functions appearing in the 
Gelfand-Tsetlin formulas. These rational functions lie in the algebra $A$ of 
regular functions over generic tableaux, and hence the $A$-lattice $V_A$ whose 
$A$-basis is the set of all integral tableaux is a $U$-submodule of $V$; now 
given a generic tableau $T(v)$, we can recover the corresponding generic module by specializing $V_A$ at $v$.

This idea breaks down if $T(v)$ is a singular tableau, and in that case we must
replace $A$ with an algebra $B \subset K$ such that $(1)$ evaluation at $v$ 
makes sense and $(2)$ there exists a $B$-lattice $L_B \subset V$ which is also
a $U$-submodule. It turns out that these conditions can be met, eventually by
replacing $v$ by $v+z$, where $z$ is an adequate tableau with integral 
entries. Specializing $L_B$ at $v+z$ we obtain a GT-module which includes 
the character $\chi_v$ in its support.

Denote by $\overline \mu$ the partition of $\binom{n}{2}$ given by 
$(1,2,\ldots, n)$. Each point $v \in \CC^{\binom{n}{2}}$, or rather its class 
modulo $G$, defines a refinement $\eta(v)$ of $\mu$, and it turns out that the 
structure of the associated GT-module $V(T(v))$ depends heavily on $\eta(v)$. 
For example, the multiplicity of every character of $V(T(v))$ is a divisor of 
$\eta(v)!$, and for most characters it is in fact equal to $\eta(v)!$. While 
it is possible to choose an algebra $B$ and a lattice $L_B$ that works for all 
$v$ simultaneously, it turns out to be more productive to fix a refinement 
$\eta$ and focus on characters with $\eta(v) = \eta$, thus obtaining an algebra
$B_\eta$ and a $B_\eta$-lattice $L_\eta$. Each $L_\eta$ has a basis of derived
tableaux, and changing $\eta$ changes this basis in an essential way.

While finishing this paper the article \cite{V-geometric-singular-GT} by 
E. Vishnyakova was uploaded to the ArXiv. Using a geometric approach, the article gives the construction 
of $p$-singular GT-modules, where $p \in \NN_{\geq 2}$ . This is a special 
class of singular modules, associated to classes $v \in \CC^{\binom{n}{2}}
/G$ where the partition $\eta(v)$ has only one nontrivial part, which is equal 
to $p$.
\bigskip

The article is organized as follows. In section \ref{preeliminaries} we 
set the notation used for the combinatorial invariants associated to tableaux.
In section \ref{modules-of-divided-differences} we develop a general framework 
for the study of actions by rational functions, and the use of symmetrization 
and divided differences operators to remove singularities. Finally section 
\ref{GT-arbitrary-characters} contains the main results of the article, giving
an explicit construction of singular GT-modules associated to arbitrary 
characters, along with explicit bases for the character spaces. 



% [Fut1] V. Futorny, A generalization of Verma modules, and irreducible representations of the Lie algebra $\mathfrak{sl}(3)$, {Ukrainskii Matematicheskii Zhurnal}, 38 (1986),  492--497.

% [Fut2] V. Futorny, Weight $sl(3)$-modules generated by semiprimitive elements, {Ukrainskii Matematicheskii Zhurnal} {43} (1991), 281--285.

% [Ov] S. Ovsienko, Finiteness statements for Gelfand–Zetlin modules, in: Algebraic Structures and Their Applications, Math. Inst., Kiev, 2002, pp. 323–328


% [Zh] D.P. Zhelobenko, Compact Lie groups and their representation, Translation of Mathematical Monographs, Vol. 40, American MAthematical Society, Providence, R.I., 1973.

\end{document}