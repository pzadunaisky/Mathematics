%%%%%%%%%%%%%%%%%%%%%% Generalities %%%%%%%%%%%%%%%%%%5
\documentclass[11pt,fleqn]{article}
\usepackage[paper=a4paper]
  {geometry}

\pagestyle{plain}
\pagenumbering{arabic}
%%%%%%%%%%%%%%%%%%%%%%%%%%%%%%%%
\usepackage{notas}
\usepackage{tikz}
\usepackage{mathdots}

%%%%%%%%%%%%%%%%%%%%%%%%%%% The usual stuff%%%%%%%%%%%%%%%%%%%%%%%%%
\newcommand\NN{\mathbb N}
\newcommand\CC{\mathbb C}
\newcommand\QQ{\mathbb Q}
\newcommand\RR{\mathbb R}
\newcommand\ZZ{\mathbb Z}
\renewcommand\k{\Bbbk}

\newcommand\B{\mathcal B}
\newcommand\F{\mathcal F}
\newcommand\V{\mathcal V}
\newcommand\D{\mathcal D}
\newcommand\DD{\mathfrak D}
\renewcommand\H{\mathcal H}
\renewcommand\O{\mathcal O}
\newcommand\I{\mathcal I}
\newcommand\J{\mathcal J}

\newcommand\maps{\longmapsto}
\newcommand\ot{\otimes}
\renewcommand\to{\longrightarrow}
\renewcommand\phi{\varphi}
\newcommand\Id{\mathsf{Id}}
\newcommand\im{\mathsf{im}}
\newcommand\coker{\mathsf{coker}}
%%%%%%%%%%%%%%%%%%%%%%%%% Specific notation %%%%%%%%%%%%%%%%%%%%%%%%%
\newcommand\g{\mathfrak g}
\newcommand\p{\mathfrak p}
\newcommand\m{\mathfrak m}
\newcommand\gl{\mathfrak{gl}}
\newcommand\gen{\mathsf{gen}}
\newcommand\std{\mathsf{std}}
\newcommand\sh{\mathsf{sh}}
\newcommand\OTheta{\overline \Theta}
\newcommand\vv{\overline{v}}

\newcommand\vectspan[1]{\left\langle #1 \right\rangle}
\newcommand\interval[1]{\llbracket #1 \rrbracket}
\newcommand\Shuffle{\mathsf{Shuffle}}

\DeclareMathOperator\Frac{Frac}
\DeclareMathOperator\Specm{Specm}

\DeclareMathOperator\sym{sym}
\DeclareMathOperator\asym{asym}
\DeclareMathOperator\sg{sg}
\DeclareMathOperator\st{\mathsf{st}}
\DeclareMathOperator\n{norm}

\newcommand\bigmodule{big GT module}

%%%%%%%%%%%%%%%%%%%%%%%%%%%%%%%%%%%%%% TITLES %%%%%%%%%%%%%%%%%%%%%%%%%%%%%%
\title{On Canonical Representations of Rational Galois Algebras}
%\author{[gamma-structure.tex]}
\date{}

\begin{document}
\maketitle
%\vspace{-2cm}

\section{Notation}
We denote by $V$ a complex vector space of finite dimension $N$, and by $G
\subset \mathsf{GL}(V)$ a finite group of reflections acting on $V$. We also
denote by $\Lambda$ the symmetric algebra over $V$, which is a polynomial
algebra in $N$ variables. The action of $G$ on $V$ induces an action on 
$\Lambda$, and we denote by $\Gamma$ the fixed ring $\Lambda^G$. Finally we set
$K$ to be the fraction field of $\Gamma$ and $L$ to be the fraction field of
$\Gamma$.

\section{Rational Galois Algebras}

\subsection{Canonical representations associated to characters}

\newpage
\section{Postnikov-Stanley differential operators}
Throughout this section we assume that $G$ is the Weyl group associated to a 
root system $\Phi$. We do not assume $G$ or $\Phi$ to be irreducible, though
of course the proof of most results will reduce to this case. Since $G$ is a
finite Coxeter group, for each element $\sigma \in G$ there is a well defined
notion of length, which we denote by $\ell(\sigma)$. The group $G$ is also
endowed with the Bruhat order.


\newpage
\section{The structure of canonical representations}

\end{document}