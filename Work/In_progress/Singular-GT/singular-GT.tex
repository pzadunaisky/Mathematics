%%%%%%%%%%%%%%%%%%%%%% Generalities %%%%%%%%%%%%%%%%%%5
\documentclass[11pt,fleqn]{article}
\usepackage[paper=a4paper]
  {geometry}

\pagestyle{plain}
\pagenumbering{arabic}
%%%%%%%%%%%%%%%%%%%%%%%%%%%%%%%%
\usepackage{notas}
\usepackage{tikz}
%%%%%%%%%%%%%%%%%%%%%%%%%%% The usual stuff%%%%%%%%%%%%%%%%%%%%%%%%%
\newcommand\NN{\mathbb N}
\newcommand\CC{\mathbb C}
\newcommand\QQ{\mathbb Q}
\newcommand\RR{\mathbb R}
\newcommand\ZZ{\mathbb Z}
\renewcommand\k{\Bbbk}

\newcommand\F{\mathcal F}
\newcommand\V{\mathcal V}
\newcommand\D{\mathcal D}
\renewcommand\H{\mathcal H}

\newcommand\maps{\longmapsto}
\newcommand\ot{\otimes}
\renewcommand\to{\longrightarrow}
\renewcommand\phi{\varphi}
\newcommand\Id{\mathsf{Id}}
\newcommand\im{\mathsf{im}}
\newcommand\coker{\mathsf{coker}}
%%%%%%%%%%%%%%%%%%%%%%%%% Specific notation %%%%%%%%%%%%%%%%%%%%%%%%%
\newcommand\g{\mathfrak g}
\newcommand\gl{\mathfrak{gl}}
\newcommand\gen{\mathsf{gen}}
\newcommand\std{\mathsf{std}}

\DeclareMathOperator\End{End}
\DeclareMathOperator\Sym{Sym}
\DeclareMathOperator\Alt{Alt}
\DeclareMathOperator\sg{sg}
%%%%%%%%%%%%%%%%%%%%%%%%%%%%%%%%%%%%%% TITLES %%%%%%%%%%%%%%%%%%%%%%%%%%%%%%
\title{Singular Gelfand Tsetlin modules over $\gl$ à la Zadunaisky}
\date{[singular-GT.tex]}
\author{Pablo Zadunaisky}
\begin{document}
\maketitle

The objective of these notes is to understand the construction of Singular
Gelfand-Tsetlin modules due to Futorny, Grantcharov and Ramirez found in 
\cite{FGR}. 

\section{GT Tableaux and GT algebras}
References for all results can be found in \cite{FGR}*{section 3}.

\paragraph
Fix $n \in \NN_{\geq 2}$, and set $N = \frac{n(n+1)}{2}$. For each $m \in \NN$ 
set $U_m = U(\gl(m, \CC))$ and $Z_m \subset U_m$ its center, and set $U = 
U_n$. We get a chain of inclusions $U_1 \subset U_2 \subset \cdots \subset U_n$
induced by the maps sending standard generators $E_{i,j} \in \gl(k,\CC)$ to 
the corresponding $E_{i,j} \in \gl(k+1, \CC)$. 

\paragraph
\label{HC-morphism}
For each $m \in \NN$ the algebra $Z_m$ is a polynomial algebra on the 
generators
\[
	c_{m,k} = \sum_{(i_1, \ldots, i_k) \in [m]^k} E_{i_1,i_2} E_{i_2,i_3} 
		\cdots E_{i_k, i_1} \qquad \qquad 1 \leq k \leq m.
\]
Let $\Lambda_m = \CC[\lambda_{m,k} \mid 1 \leq k \leq m]$ be a polynomial
algebra in $m$ variables, and set
\[
	\gamma_{m,k} = \sum_{i = 1}^m (\lambda_{m,i}+m-1)^k 
	\prod_{i \neq j} \left( 1 - \frac{1}{\lambda_{m,i} - \lambda_{m,j}}\right).
\]
Although it is not obvious, the $\gamma_{m,k}$ are algebraically independent 
polynomials and they are invariant by the obvious action of the symmetric 
group $S_m$ on $\Lambda$; in fact $\Lambda_m^{S_m} = \CC[\gamma_{m,k} \mid 
1 \leq k \leq m]$, and there is an embedding $Z_m \to \Lambda$ given by 
$c_{m,k} \mapsto \gamma_{m,k}$. 

\paragraph
\label{GT-algebra}
We write $\Gamma = \CC[c_{m,k} \mid 1 \leq k \leq m \leq n] = 
\subset U$, which is the algebra generated by $\bigcup_{k=1}^n Z_k$. The 
$c_{m,k}$ are algebraically independent and hence this is isomorphic to a 
polynomial algebra in $N$ generators.

Let $\Lambda = \CC[\lambda_{i,j} \mid 1 \leq i \leq j \leq n]$ be a polynomial
algebra in $N$ variables. The group $S_m$ acts on $\Lambda$ 
permuting the variables $\lambda_{m,k}$ with $1 \leq k \leq m$ and fixing the 
rest. This induces an action of the group $G = S_n \times S_{n-1} \times 
\cdots \times S_1$ on $\Lambda$. Composing with the embedding from 
\ref{HC-morphism} we get that $\Gamma$ is isomorphic to $\Lambda^G$.

\note{I suppose this is just the Harish-Chandra morphism of each $Z_k$ pasted
together. Actually, is there \emph{a} HC morphism? In any case restriction to 
$Z_k$ gives an isomorphism with invariants in a polynomial algebra.}

\paragraph
\label{GT-tableaux}
Recall we have fixed $n \in \NN$ and set $N = \frac{n(n+1)}{2}$. A 
\newterm{Gelfand-Tsetlin 
tableau} of size $n$ is a vector $v \in \CC^{N}$, whose coordinates are 
indexed by $\{(i,j) \mid 1 \leq i \leq j \leq n\}$. We can represent this 
graphically as [INSERT IMAGE]. For ease of reference we will sometimes write 
\[
v = (v_{n,1}, v_{n,2}, \ldots v_{n,n} 
	\mid v_{n-1,1}, \ldots, v_{n-1,n-1} 
	\mid \cdots 
	\mid v_{1,1}).
\]

\paragraph
\label{various-tableau}
We say that a GT-tableau $v$ is 
\begin{itemize}
\item \newterm{integral} if $v \in \ZZ^{N}$. 

\item \newterm{standard} if for all $1 \leq i < j \leq k < n$ the difference 
$v_{k,i}-v_{k-1,i} \in \ZZ_{\geq 0}$ and $v_{k-1,i}-v_{k,i+1} \in \ZZ_{>0}$; 
we denote the set of standard tableaux by $\CC^N_\std$.

\item \newterm{generic} if for all $1 \leq i < j \leq k 
< n$ the difference $v_{k,i}-v_{k,j}$ is not in $\ZZ$; we denote the set of 
generic tableaux by $\CC^N_\gen$. 

\item \newterm{singular} if it is non-generic tableau, i.e. there is a pair of 
entries in the same line differing by an integer. If there is exactly one 
such pair then the tableau is called \newterm{1-singular}. Finally if two 
entries in the same row are equal the tableau is called \newterm{critical}.
\end{itemize}

\paragraph
Set $\ZZ^N_0 = \{v \in \ZZ^N \mid v_{n,i} = 0 \mbox{ for } 1 \leq i \leq n\}$.
For any ring $A$ we denote by $V_A$ the free $A$-module generated by all 
tableaux of the form $T(v)$ with $v \in \ZZ_0^N$. The group $G = S_n \times 
S_{n-1} \times \cdots \times S_1$ acts on $V_A$ in an obvious fashion.


\section{GT-modules}

\paragraph
\label{GT-formulas}
Set for each $1 \leq i \leq k \leq n$
\begin{align*}
p_{k,i}^\pm(\lambda) 
	&= \prod_{j = 1}^{k\pm1}(\lambda_{k,i} -\lambda_{k\pm1,j}); &
q_{k,i}(\lambda)
	&= \prod_{j \neq i} (\lambda_{k,i} - \lambda_{k,j}). \\
|\lambda|_k &= \lambda_{k,1} + \lambda_{k,2} + \cdots \lambda_{k,k}.
\end{align*}
Notice that $p_{k,i}^\pm(v)$ depends only on the $k, k \pm 1$-th rows of 
$v \in \CC^N$, while $q_{k,i}(v)$ depends only on the $k$-th row.

\paragraph
If $w$ is a GT tableau set 
\[
V(w) = \langle T(w+z) \mid z \in \ZZ^N_0 \rangle_\CC.
\]
The $T$ is just a symbol to remind us that we have a basis indexed by 
tableaux; in particular $T(v+z) \neq T(v) + T(z)$.

The following is a classical theorem which we quote without proof.
\begin{Theorem}[Gelfand-Tsetlin, '50]
\label{GT}
Let $\lambda = (\lambda_1, \ldots, \lambda_n)$ be a dominant weight of 
$\gl(n,\CC)$, and let 
\begin{align*}
V(\lambda) 
	&= \langle T(v) \mid v \in \CC^N_{\std} 
		\mbox{ and } v_{n,1} = \lambda_1, v_{n,2} = \lambda_2 - 1, \ldots, 
		v_{n,n} = \lambda_{n} - n +1
	\rangle_\CC
\end{align*}
(by convention, if $v$ is non-standard then $T(v) = 0$ in $V(\lambda)$).
Then
\begin{enumerate}[(a)]
\item \label{GT-structure}
$V(\lambda)$ is a $U$-module with the action of the canonical generators 
given by
\begin{align*}
E_{k,k+1} T(v) 
	&= - \sum_{i=1}^k \frac{p^+_{k,i}(v)}{q_{k,i}(v)} T(v + \delta^{k,i}) \\
E_{k+1,k} T(v) 
	&= \sum_{i=1}^k \frac{p^-_{k,i}(v)}{q_{k,i}(v)} T(v - \delta^{k,i}) \\
E_{k,k} T(v)
	&= (|v|_k - |v|_{k-1} + k -1) T(v)
\end{align*}


\item \label{GT-action} 
For each $1 \leq k \leq m \leq n$, we have $c_{m,k} T(v) 
= \gamma_{m,k}(v) T(v)$.

\item With this structure, $V(\lambda)$ is a finite dimensional representation
of maximal weight $\lambda$.
\end{enumerate}
\end{Theorem}

\paragraph
\label{GT-generic}
We will now reprove the following result due to Drozd, Futorny and Ovsienko.
\begin{Theorem*}[Futorny-Droz-Ovsyenko]
Let $w \in \CC^N$. If $w$ is generic then replacing $V(\lambda)$ with $V(w)$, 
items (\ref{GT-structure}) and (\ref{GT-action}) of \ref{GT} hold mutatis 
mutandis (in particular the condition on $v$ being standard is void).
\end{Theorem*}

We give a proof of this result. In order to do that we need a lemma.

\begin{Lemma*}
Let $W \subset \ZZ^N_0$ be a finite set and let $S_W = \bigcap_{w \in W} 
\CC^N_\std - w$. Let $F \in \CC(\lambda_{k,i} \mid 1 \leq i \leq k \leq n)$ be 
a rational function which is well defined in a set containing $S_W$. If $F(v) 
= 0$ for all $v \in S_W$, then $F = 0$.
\end{Lemma*}
\begin{proof}
Explicitly, $v$ lies in $S_W$ if and only if
\begin{align*}
v_{k,i} - v_{k-1,i} - \max_{w \in W}\{w_{k-1,i} - w_{k,i}, 0\} +1 
	&\in \ZZ_{> 0}, \\
v_{k-1,i} - v_{k,i+1} - \max_{w \in W}\{w_{k,i+1} - w_{k-1,i}\}.
	&\in \ZZ_{>0} 
\end{align*}
We enumerate the coordinate functions of a tableau as follows: start by 
$\lambda_{n,n}$, which we denote by $x_1$; then, looking at all entries with 
second coordinate $n-1$, we enumerate them by taking $x_2 = 
\lambda_{n-1,n-1}$, then $x_3 = \lambda_{n,n-1}$; next we take the elements 
with second coordinate $n-2$ starting by $\lambda_{n-2,n-2}$ and moving in the 
northwest direction. Explicitly, setting $\phi(i,j) = (i-j+1) + 
\frac{(n-j)(n-j+1)}{2}$ we write $x_{\phi(i,j)} = \lambda_{i,j}$. The 
following figure shows the enumeration corresponding to $n = 3$; two entries 
are joined by an edge if and only if they appear in one of the inequalities 
defining $S_W$, and in all cases the leftmost appears with a plus sing and the 
rightmost with a minus.

\begin{tikzpicture}
\node (31) at (0,3) {$\lambda_{3,1}$};
\node (32) at (2,3) {$\lambda_{3,2}$};
\node (33) at (4,3) {$\lambda_{3,3}$};
\node (21) at (1,2) {$\lambda_{2,1}$};
\node (22) at (3,2) {$\lambda_{2,2}$};
\node (11) at (2,1) {$\lambda_{1,1}$};

\node (31a) at (6,3) {$x_6$};
\node (32a) at (8,3) {$x_3$};
\node (33a) at (10,3) {$x_1$};
\node (21a) at (7,2) {$x_5$};
\node (22a) at (9,2) {$x_2$};
\node (11a) at (8,1) {$x_4$};

\draw (33) -- (22) -- (11)  (32) -- (21);
\draw (22) -- (32)  (11) -- (21) -- (31);

\draw (33a) -- (22a) -- (11a)  (32a) -- (21a);
\draw (22a) -- (32a)  (11a) -- (21a) -- (31a);
\end{tikzpicture}

With this enumeration, all the equations defining $S_W$ read as $x_r(v) - 
x_s(v) - t_{r,s} \in \ZZ_{>0}$ for some $t_{r,s} \in \ZZ$, and in each case 
$r > s$. Thus if we put $x_1(v) = 0$ then by induction for each $s = 1, 
\ldots, N$ there is always an $x_s(v)$ satisfying all equations involving 
$x_r$ with $r\leq s$. In this way we find a tableau in $S_W$, so in particular 
$S_W$ is not empty. 

For each $s \in [N]$, let $w_s \in \ZZ^N$ be the tableau such that $x_t(w_s) 
= 1$ if $t \geq s$, and $x_t(w_s) = 0$ if $t < s$. Notice that $v \in S_W$ 
implies $v + r w_s \in S_z$ for all $r \in \NN$. 
Let $f/g$ be the reduced expression of $F$. The hypothesis implies that if 
$v \in S_W$ then $f(v) = 0$ and $g(v) \neq 0$, and all we have to prove is 
that $f = 0$; we will do this by induction on the highest $t$ such that $x_t$ 
appears in $f$. If $f$ is a polynomial on $x_1$, then taking $v \in S_z$, our 
hypothesis implies that $f(v + rw_1) = 0$ for all $r \in \NN$, which can only 
happen if $f = 0$. Assuming the result holds for all numbers less than $t$, we 
write $f$ as a polynomial in $x_t$ with coefficients in the ring of polynomial 
functions on $x_1, \ldots, x_{t-1}$, so $f = f_0 + f_1 x_t + \cdots + f_l 
x_t^l$. Fix $v \in S_z$. For each $r \in \NN$ the value of $f_i(v + rw_t)$ is 
independent of $r$, since the entries below the $t$-th are fixed, so 
$f(v+rw_t)$ is a polynomial in $r$ whose $i$-th coefficient equals $f(v)$.
Since $f(v + rw_s) = 0$ for all $r \in \NN$, we see that $f_i(v) = 0$. Now $v$ 
was arbitrary, so each $f_i$ is zero over $S_W$, which implies that $f_i = 0$ 
for all $i$ by the inductive hypothesis.
\end{proof}

\paragraph
Consider the action of $\ZZ^N$ over $\CC^N$, given by $v^z = v+z$. This induces
an action on $\Lambda$, given by $(z \cdot f)(v) = f(v^z)$.
This action extends naturally to the fraction field $K = 
\operatorname{Frac}(\Lambda) = \CC(\lambda_{k,i} \mid 1 \leq i \leq k \leq 
n)$. 

For the rest of this section let $\H$ be the infinite hyperplane arrangement 
in $\CC^N$ consisting of all hyperplanes defined by the equations 
$\lambda_{k,i} - \lambda_{k,j} - r$ for all $1 \leq i < j \leq k \leq n$ and 
all $r \in \ZZ$, and let $A \subset K$ be the algebra of rational functions 
whose poles are contained in $\H$. Notice that $A$ is stable by the action of
$\ZZ^N$.


\begin{Proposition}
Let $V_A$ be the $A$-module with basis $\{T(z) \mid z \in \ZZ^n_0\}$, and let 
$V_K = K \ot_A V_A$. We see $V_A$ as $A$-submodule of $V_K$ in the obvious way.

\begin{enumerate}[(a)]
\item \label{generic-GT-structure}
The vector space $V_K$ can be endowed with the structure of a $U$-module 
with the action of the canonical generators given by
\begin{align*}
E_{k,k+1} T(z) 
	&= - \sum_{i=1}^k \frac{p^+_{k,i}(\lambda^z)}{q_{k,i}(\lambda^z)} 
		T(z + \delta^{k,i}); \\
E_{k+1,k} T(z) 
	&= \sum_{i=1}^k \frac{p^-_{k,i}(\lambda^z)}{q_{k,i}(\lambda^z)} 
		T(z - \delta^{k,i}); \\
E_{k,k} T(z)
	&= (|\lambda^z|_k - |\lambda^z|_{k-1} + k -1) T(z).
\end{align*}

\item For each $1 \leq k \leq m \leq n$, we have $c_{m,k} T(z) = 
\gamma_{m,k}(\lambda^z) T(z)$.

\item The $A$-module $V_A$ is a sub $U$-module of $V_K$.
\end{enumerate}
\end{Proposition}

\begin{proof}
Let $F$ be the free $\CC$-algebra generated by $X_{k,k+1}, X_{k+1,k}$ for 
$1 \leq k < n$, and $X_{k,k}$ for $1 \leq k \leq n$; there is an obvious map
$\phi: F \to U$. The formulas in item \ref{generic-GT-structure} define an 
$F$-module structure on $V_K$, so there is an algebra map $F \to \End_K V_K$, 
and we must show that this map factors through $U$.

Let $a \in F$ be a linear combination of monomials of length at most $r \geq 
0$. Then for each $w \in \ZZ^N_0$ there exists a rational function $f_a(-,w) 
\in K$ such that
\[
	aT(z) = \sum_{w \in \ZZ^N_0} f_a(\lambda^z,w) T(z+w)
\]
for all $z \in \ZZ^N_0$. Furthermore if the sum of the absolute values of the
entries of $w$ is greater than $r$ then $f_a(-,w) = 0$, so in fact the sum is 
over a finite set $W$. By Lemma \ref{GT-generic} the rational function 
$f_a(-,w)$ is determined by its values on $S_W = \bigcap_{w \in W} 
\CC_\std^N-w$.

Fix $v \in S_W$ and let $\lambda = (v_{n,1}, v_{n,1} + 1, \ldots, v_{n,n} + 
n -1)$. Since $v$ is standard $\lambda$ is a weight of $\gl(n,\CC)$, so we can
consider the representation $V(\lambda)$ as defined in \ref{GT}. By 
construction the set $\{T(v+w) \mid w \in W\} \subset V(\lambda)$ consists of 
nonzero linearly independent elements, and by construction
\[
	\phi(a) T(v) = \sum_{w \in W} f_a(v, w) T(v+w).
\]
Thus if $\phi(a) = 0$ then $f_a(v,z) = 0$ for all $v \in S_W$ and hence 
$f_a(-,w) = 0$, and the formulas in item \ref{generic-GT-structure} indeed 
define a $U$-module structure on $V_K$. Furthermore, if $\phi(a) = c_{m,k}$ for
some $k$ and $m$ then $f_a(v,z) = 0$ for all $z \neq 0$ and $f_a(v,0) = 
\gamma_{m,k}(v)$, and again by Lemma \ref{GT-generic} we get item (b). Item 
(c) follows from the definition.
\end{proof}

\paragraph
\label{GT-generic-proof}
\begin{proof}[Proof of Theorem \ref{GT-generic}]
If $w \in \CC_\gen^N$ then the map $f \in A \mapsto f(w) \in \CC$ is well
defined, and induces a one-dimensional representation of $A$ which we denote
$\CC_w$. Now $\CC_w \ot_A V_A$ is a $U$-module, with the action of $U$ induced
by its action on $V_A$, and furthermore it is isomorphic to $V(w)$ as 
$\CC$-vector space through the assignation $1 \ot_A T(z) \mapsto T(w+z)$. 
This provides the desired action of $U$ on $V(w)$.
\end{proof}

\section{$1$-singular GT-modules}
We extend the construction from Theorem \ref{GT-generic} to the case were
$w$ is $1$-critical.

\paragraph
Let $\Sigma = \{(k,i,j) \mid 1 \leq i < j \leq k \leq n\}$, and fix $(t,r,s) 
\in \Sigma$; write $x = \lambda_{t,r}, y = \lambda_{t,s}$. Let $B \subset A$ 
be the algebra of all rational functions in $A$ without poles in the hyperplane
$x-y$, i.e. $B$ consists of all functions in $A$ that can be evaluated in a 
$1$-critical tableau $v$ whose critical entries are precisely $x(v)$ and 
$y(v)$.

\paragraph
Let $\tau \in S_t$ be the permutation $(rs)$. We see $\tau $ as an element of
$G = S_1 \times S_2 \times \cdots \times S_n$ in the obvious way. This element
induces a linear transformation $\tau : \CC^N \to \CC^N$, which on the 
canonical basis is defined by 
\begin{align*}
\tau(\delta^{k,i})
	&= \begin{cases}
		\delta^{t,\tau(i)} & \mbox{ if } k = t; \\
		\delta^{k,i} & \mbox{ otherwise. }
	\end{cases}
\end{align*}
Clearly $\tau^2 = \Id$.

Recall that $K = \CC(\lambda_{k,i} \mid 1 \leq k \leq i \leq n)$ and that $V_K$
is the $K$-vector space with basis $\{T(z) \mid \ZZ^N_0\}$. For each $z \in 
\ZZ_0^N$ we define   
\begin{align*}
S(z) 
	&= \frac{T(z) + T(\tau(z))}{2},
&A(z)
	&= \frac{T(z) - T(\tau(z))}{2(x-y)}.
\end{align*}
Notice that $S(\tau(z)) = S(z)$ and $A(\tau(z)) = - A(z)$, so if $x(z) = y(v)$,
or equivalently $\tau(z) = z$, then $S(z) = T(z)$ and $A(z) = 0$. Also notice
that $T(z) = S(z) + (x-y)A(z)$, so the set $\{S(z), A(z) \mid z \in \ZZ_0^N\}$
generates $V_K$, and removing $0$ it is in fact a $K$-basis.

\paragraph
\label{GT-singular}
Our aim is to prove the following proposition.
\begin{Proposition*}
Let $V_B$ be the $B$-module generated by $\{S(z), A(z) \mid z \in \ZZ_0^N\}$.
Then $V_B \subset V_K$ is a sub-$U$-module.
\end{Proposition*}
Suppose the proposition is proved, and let $v \in \CC^N$ be a $1$-critical 
tableau with critical entries $v_{t,r}, v_{t,s}$. Then $v$ induces a 
$1$-dimensional representation of $B$ denoted by $\CC_v$, and we can recover
the $1$-critical module of \cite{FGR} taking $\CC_v \ot_B V_B$. In order to 
prove the proposition, we have to show that for all adequate $k$ the
$B$-module $V_B$ is stable by the action of $E_{k,k}, E_{k,k+1}$ and 
$E_{k+1,k}$. This is long and tedious, so we go step by step. We begin with
a lemma that gathers some properties of the relevant functions.

\begin{Lemma*}
\begin{enumerate}
\item \label{depends}
The polynomial $p_{k,i}^\pm$ depends only on $\lambda_{k,i}$ and 
$\lambda_{k \pm 1,j}$ with $1 \leq j \leq k \pm 1$. 
The polynomial $q_{k,i}$ depends only on $\lambda_{k,j}$ with $1 \leq j \leq 
k$.

\item \label{x-y-q}
We have $(x-y) \mid q_{k,i}(\lambda^z)$ if and only if $(k,i) \in 
\{(t,r), (t,s)\}$ and $z = \tau(z)$, and in that case $(x-y)^2$ does not
divide $q_{k,i}(\lambda^z)$.

\item \label{cosito}
For each $z \in \ZZ^N_0$ 
\[
F^\pm_z = \frac{p^\pm_{t,r}(\lambda^z)}{q_{k,r}(\lambda^z)} 
+ \frac{p^\pm_{t,s}(\lambda^{z})}{q_{k,s}(\lambda^{z})}\ \in B,
\]

\item \label{cosito2}
For each $z \in \ZZ^N_0$
\[
D_z^\pm =
\frac{1}{x-y}\left(\frac{p_{k,i}^\pm(\lambda^z)}{q_{k,i}(\lambda^z)} 
- \frac{p_{k,i}^\pm(\lambda^{\tau(z)})}{q_{k,i}(\lambda^{\tau(z)})}\right) 
\in B
\]
\end{enumerate}
\end{Lemma*}
\begin{proof}
The first two items are obvious by inspection, and the third follows from the 
second except if $\tau(z) = z$. Now $q_{t,r} = (x-y) \prod_{i \neq r,s}(x^z - 
\lambda_{t,i})$ and $q_{t,s} = (y-x) \prod_{i \neq r,s}(y^z - \lambda_{t,i})$.
Set
\[
G_z^\pm 
= \frac{p^\pm_{t,r}(\lambda^z)}{\prod_{i \neq r,s}(x^z - \lambda_{t,i}^z)} 
- \frac{p^\pm_{t,s}(\lambda^{z})}{\prod_{i \neq r,s}(y^z - \lambda_{t,i}^z)}
\]
Now $G_z^\pm \in B$ and $F_z = G_z/(x-y)$, so it is enough to show that 
$(x-y) \mid G_z^\pm$ in $B$, or in other words that the image of $G_z^\pm$ in 
the quotient ring $B/(x-y)$ is zero. Clearly the image of both denominators in 
the quotient is the same, and direct inspection shows that the same happens 
with the numerators. 

For the fourth item, notice that $z = \tau(z)$ implies $D_z = 0$, so we have 
$E_z^\pm = (x-y)D_z^\pm \in B$ for all $z$. Direct calculations show that
$E_z \equiv 0 \mod x-y$.
\end{proof}

\paragraph
\about{Action of $E_{k,k}$}

Since $|\lambda^z|_k = |\lambda^{\tau(z)}|_k = |\lambda|_k + |z|_k$
for all $k$, we obtain
\begin{align*}
E_{k,k} S(z) 
	&= (|\lambda^z|_k - |\lambda^z|_{k-1} + k -1) S(z) \\
E_{k,k} A(z) 
	&= (|\lambda^z|_k - |\lambda^z|_{k-1} + k -1) A(z)
\end{align*}
Thus $V_B$ is stable by the action of $E_{k,k}$.

\paragraph
\about{Action of $E_{k,k+1}$ on $S(z)$ for $k \neq t$}

The hypothesis that $k \neq t$ implies that $\tau(z) + \delta^{k,i} = 
\tau(z + \delta^{k,i})$ and also that $q_{k,i}(\lambda^z) = 
q_{k,i}(\lambda^{\tau(z)})$ for all $i$. Thus by definition
\begin{align*}
E_{k,k+1} S(z)
	&= - \frac{1}{2}
		 	\sum_{i=1}^k \frac{1}{q_{k,i}(\lambda^z)} 
			\left(
			p_{k,i}^+(\lambda^z) T(z+\delta^{k,i})
			+ p_{k,i}^+(\lambda^{\tau(z)}) T(\tau(z + \delta^{k,i}))
		\right) 
\end{align*}
By item \ref{x-y-q} of Lemma \ref{GT-singular} $1/q_{k,i}(\lambda^z) \in B$.
Since $T(z) = S(z) + (x-y)A(z) \in V_B$ the right hand side of the equation 
lies in $V_B$. 

Let us refine that result. If we assume that $k \neq t,t-1$ then 
$p_{k,i}(\lambda^z) = p_{k,i}(\lambda^{\tau(z)})$, so the previous formula 
becomes
\begin{align}
\label{k-k+1-S}
E_{k,k+1} S(z) 
	&= - \sum_{i = 1}^k \frac{p^+_{k,i}(\lambda^z)}{q_{k,i}(\lambda^z)}
	S(z+\delta^{k,i}).
\end{align}
If we take $k = t-1$ then we can replace $T(z)$ by $S(z) + (x-y)A(z)$ and
obtain
\begin{align}
\label{t-1-t-S}
E_{t-1,t} S(z)
	&= -\frac{1}{2}\sum_{i=1}^{t-1} \frac{p^+_{t-1,i}(\lambda^z) + 
	p^+_{t-1,i}(\lambda^{\tau(z)})}{q_{t-1,i}(\lambda^z)} 
	S(z + \delta^{t-1,i}) + \\
	&\qquad \quad + \frac{p^+_{t-1,i}(\lambda^z) - 
	p^+_{t-1,i}(\lambda^{\tau(z)})}{q_{t-1,i}(\lambda^z)} (x-y)
	A(z + \delta^{t-1,i}).
\end{align}

\paragraph
\about{Action of $E_{t,t+1}$ on $S(z)$}
Using the equality $T(z) = S(z) + (x-y)A(z)$, and the fact that $S(\tau(z) + 
\delta^{k,i}) = S(z+\tau(\delta^{k,i}))$ and $A(\tau(z) + \delta^{k,i}) = - 
A(z+\tau(\delta^{k,i}))$ we get
\begin{align*}
E_{t,t+1} S(z)
	&= -\frac12\sum_{i=1}^t \frac{p^+_{t,i}(\lambda^z)}{q_{k,i}(\lambda^z)}
		(S(z + \delta^{t,i}) + (x-y)A(z + \delta^{t,i})) + \\
		&\qquad \qquad 
		+ \frac{p^+_{t,i}(\lambda^{\tau(z)})}{q_{k,i}(\lambda^{\tau(z)})}
		\Big(S(z + \tau(\delta^{t,i})) - (x-y)A(z + \tau(\delta^{t,i}))\Big)
\end{align*}
By definition of $\tau$, we get
\begin{align}
E_{t,t+1}S(z)
	&=-\frac12\sum_{i = 1}^t 
	\left(\frac{p^+_{t,i}(\lambda^z)}{q_{t,i}(\lambda^z)} 
+ \frac{p^+_{t,\tau(i)}(\lambda^{\tau(z)})}{q_{t,\tau(i)}
(\lambda^{\tau(z)})}\right)S(z + \delta^{t,i}) + \\
& \qquad \qquad +
\left(\frac{p^+_{t,i}(\lambda^z)}{q_{t,i}(\lambda^z)} 
- \frac{p^+_{t,\tau(i)}(\lambda^{\tau(z)})}{q_{t,\tau(i)}
(\lambda^{\tau(z)})}\right)(x-y)A(z+\delta^{t,i}).
\end{align}
The coefficients of the $S$'s and the $A$'s' in this formula lie in $B$ unless
$z = \tau(z)$, in which case there might be problems when $i = r$ or $i = s$.
In that case the coefficient of the $A$'s lie in $B$ because $(x-y)$ divides 
$q_{t,r}$ and $q_{t,s}$ exactly once. Thus all we have to prove is that
if $z = \tau(z)$ then the coefficients of $S(z+\delta^{t,r})$ and 
$S(z+\delta^{t,s})$ lie in $B$ but this is precisely item \ref{cosito} of 
Lemma \ref{GT-singular}.

\paragraph
\about{Action of $E_{k+1,k}$ on $S(z)$}
Similar reasoning works for this case. We only state the results.
\begin{align*}
E_{k+1,k}S(z) 
	&= \sum_{i=1}^k \frac{p_{k,i}^-(\lambda^z)}{q_{k,i}(\lambda^z)}
	S(z - \delta^{k,i}) & (k \neq t,t+1)\\
E_{t+2,t+1}S(z)
	&= \frac12\sum_{i=1}^{t+1} 
	\frac{p^-_{t+1,i}(\lambda^z) + p^-_{t+1,i}(\lambda^{\tau(z)})}
	{q_{t+1,i}(\lambda^z)} S(z - \delta^{t+1,i}) + \\
	&\qquad \quad 
	+ \frac{p^-_{t+1,i}(\lambda^z) - p^-_{t+1,i}(\lambda^{\tau(z)})}
	{q_{t+1,i}(\lambda^z)} (x-y) A(z - \delta^{t+1,i}) \\
E_{t+1,t} S(z)
	&= \frac12 \sum_{i=1}^t \left(
	\frac{p^-_{t,i}(\lambda^z)}{q_{t,i}(\lambda^z)} 
	+ \frac{p^-_{t,\tau(i)}(\lambda^{\tau(z)})}
	{q_{t,\tau(i)}(\lambda^{\tau(z)})}\right) S(z - \delta^{t,i}) + \\
	& \qquad \qquad +
		\left(\frac{p^-_{t,i}(\lambda^z)}{q_{t,i}(\lambda^z)} - 
		\frac{p^-_{t,\tau(i)}(\lambda^{\tau(z)})}{q_{t,\tau(i)}
		(\lambda^{\tau(z)})}\right)(x-y)A(z - \delta^{t,i})
\end{align*}

\paragraph
\about{The action on $A(z)$}
Arguments similar to those above show that we get
\begin{align*}
E_{k,k+1} A(z) 
	&= - \sum_{i = 1}^k \frac{p^+_{k,i}(\lambda^z)}{q_{k,i}(\lambda^z)}
	A(z) & (k \neq t,t-1)\\
E_{t-1,t} A(z)
	&= -\frac12 \sum_{i=1}^{t-1} 
	\frac{p_{t-1,i}^+(\lambda^z) - p_{t-1,i}^+(\lambda^{\tau(z)})}
	{(x-y)q_{t-1,i}(\lambda^z)} S(z + \delta^{t-1,i}) \\
	&\qquad \qquad + \frac{p_{t-1,i}^+(\lambda^z) + p_{t-1,i}^+(\lambda^{\tau(z)})}
	{q_{t-1,i}(\lambda^z)}A(z + \delta^{t-1,i}) \\
E_{t,t+1} A(z)
	&= -\frac12 \sum_{i=1}^t \frac{1}{x-y}\left( 
		\frac{p_{t,i}^+(\lambda^z)}{q_{t,i}(\lambda^z)} - 
		\frac{p_{\tau(t,i)}^+(\lambda^{\tau(z)})}{q_{\tau(t,i)}(\lambda^{\tau(z)})}\right) 
		S(z + \delta^{t,i}) \\
	& \qquad \qquad 
	+ \left(\frac{p_{t,i}^+(\lambda^z)}{q_{t,i}(\lambda^z)} 
	+ \frac{p_{\tau(t,i)}^+(\lambda^{\tau(z)})}{q_{\tau(t,i)}(\lambda^{\tau(z)})}\right) 
	A(z + \delta^{t,i}).
\end{align*}

Similarly we get
\begin{align*}
E_{k+1,k} A(z) 
	&= \sum_{i=1}^k \frac{p_{k,i}^-(\lambda^{z})}{q_{k,i}(\lambda^{z})}
	A(z - \delta^{k,i}) & (k \neq t,t+1) \\
E_{t+2,t+1} A(z)
	&= \frac12 \sum_{i=1}^{t+1} \frac{p_{t+1,i}^-(\lambda^z) - 
	p_{t+1,i}^-(\lambda^{\tau(z)})}
	{(x-y)q_{t+1,i}(\lambda^z)}S(z - \delta^{t+1,i}) \\
	&\qquad \qquad + \frac{p_{t+1,i}^-(\lambda^z) + 
	p_{t+1,i}^-(\lambda^{\tau(z)})}
	{q_{t+1,i}(\lambda^z)}A(z-\delta^{t+1,i}) \\
E_{t,t+1} A(z)
	&= -\frac12 \sum_{i=1}^t \frac{1}{x-y}\left( 
		\frac{p_{t,i}^-(\lambda^z)}{q_{t,i}(\lambda^z)} - 
		\frac{p_{t,\tau(i)}^-(\lambda^{\tau(z)})}{q_{t,\tau(i)}(\lambda^{\tau(z)})}\right) 
		S(z - \delta^{t,i}) \\
	& \qquad \qquad 
	+ \left(\frac{p_{t,i}^-(\lambda^z)}{q_{t,i}(\lambda^z)} 
	+ \frac{p_{t,\tau(i)}^-(\lambda^{\tau(z)})}{q_{t,\tau(i)}(\lambda^{\tau(z)})}\right) 
	A(z - \delta^{k,i}).
\end{align*}

\paragraph
\about{The action of $\Gamma$ on $S(z), A(z)$}
By carrying things out as before we get
\begin{align*}
c_{m,k} S(z)
	&= \gamma_{m,k}(\lambda^z)S(z) 
&c_{m,k} A(z)
	&= \gamma_{m,k}(\lambda^z)A(z) 
	&(m \neq t),
\end{align*}
while for $k = t$
\begin{align*}
c_{t,k} S(z)
	&= \frac{\gamma_{m,k}(\lambda^z) + \gamma_{m,k}(\lambda^{\tau(z)})}{2} S(z)
		+ (x-y)\frac{\gamma_{m,k}(\lambda^z) - \gamma_{m,k}(\lambda^{\tau(z)})}
		{2}A(z) \\
c_{t,k} A(z)
	&= \frac{\gamma_{m,k}(\lambda^z) + \gamma_{m,k}(\lambda^{\tau(z)})}{2}A(z)
		+ \frac{\gamma_{m,k}(\lambda^z) - \gamma_{m,k}(\lambda^{\tau(z)})}
		{2(x-y)}S(z)
\end{align*}
Notice that this makes it less obvious whether $V_B$ is a GT-module.

\paragraph
\about{The generic $1$-singular $U$-module}
Set $B' = B/(x-y)$, and $V' = B' \ot_B V_B$. Clearly for any $w \in \CC^N$ 
with $x(w) = y(w)$ the $B$-module $\CC_w$ is also a $B'$-module. Thus to study 
the generic $U$-module over a $1$-critical tableau we look at the action of
$U$ on $V'$.

\begin{align*}
E_{k,k+1} S(z)
	&\equiv -\sum_{i=1}^k \frac{p_{k,i}^+(\lambda^z)}{q_{k,i}(\lambda^z)} 
	S(z+\delta^{k,i}) \mod (x-y) &(z \neq \tau(z) \mbox{ 0 } k \neq t); \\
E_{t,t+1} S(z)
	&\equiv -\sum_{i=1}^t \frac{p_{t,i}^+(\lambda^z)}{q_{t,i}(\lambda^z)} 
	S(z+\delta^{t,i}) \\&\qquad \qquad + 2\frac{p^+_{t,r}(\lambda^z)}{\hat q_{t,r}(\lambda^z)} A(z+\delta^{t,r}) \mod (x-y) &(z = \tau(z)) \\
E_{k+1,k} S(z)
	&\equiv \sum_{i=1}^k \frac{p_{k,i}^-(\lambda^z)}{q_{k,i}(\lambda^z)} 
	S(z-\delta^{k,i}) \mod (x-y) &(z \neq \tau(z) \mbox{ 0 } k \neq t); \\
E_{t,t+1} S(z)
	&\equiv \sum_{i=1}^t \frac{p_{t,i}^-(\lambda^z)}{q_{t,i}(\lambda^z)} 
	S(z-\delta^{t,i}) \\&\qquad \qquad + 2\frac{p^-_{t,r}(\lambda^z)}{\hat q_{t,r}(\lambda^z)} A(z-\delta^{t,r}) \mod (x-y) &(z = \tau(z))
\end{align*}
where $\hat q_{t,r} = q_{t,r}/(x-y)$.
\begin{align*}
E_{k,k+1} A(z)
	&\equiv -\sum_{j=1}^k \frac{p_{k,i}^+(\lambda^z)}{q_{k,i}(\lambda^z)}
	A(z+\delta^{k,i}) \mod (x-y)&(k \neq t,t-1) \\
E_{t-1,t} A(z) 
	&\equiv -\sum_{j=1}^{t-1} \frac{p_{t-1,i}^+(\lambda^z)}
	{q_{t-1,i}(\lambda^z)} A(z+\delta^{t-1,i}) \\
	&\qquad \qquad + \frac{(b-a)}{2}\frac{\hat p^+_{t-1,i}(\lambda^z)}
	{q_{t-1,i}(\lambda^z)} S(z + \delta^{t-1,i})\\
E_{t,t+1} A(z)
	&\equiv -\sum_{j=1}^{t} \frac{p_{t,i}^+(\lambda^z)}
	{q_{t,i}(\lambda^z)} A(z+\delta^{t,i}) \\
	&\qquad \qquad + \frac{a-b}{2} \frac{p_{t,i}^+(\lambda^z)}
	{mcm(q_{t,i}(\lambda^z),q_{t,i}(\lambda^{\tau(z)}))} S(z + \delta^{t,i})
\end{align*}
where $\displaystyle \hat p_{t \mp 1,i}^\pm = \prod_{j \neq r,s}^{t\mp 1} 
(\lambda_{t\mp 1, i} - \lambda_{t,j})$.


\newpage

\section{The search for the general formula}

I want to find out how to construct a module associated with a $2$-critical
tableau. I work with tableaux of heigth $4$, and write $x = \lambda_{3,1}, y 
= \lambda_{3,2}, z = \lambda_{3,3}$. I will deal with the case $x = y = z$.

The idea is the following. Let $B = \{f/g \in A \mid g \mbox{ is not divisible 
by any of } (x-y), (x-z), (y-z)\}$, so a $2$-critical tableau $v$ with $x(v) = 
y(v) = z(v)$ still defines a $1$-dimensional $B$-module $\CC_v$. We wish to 
find a $B$-lattice $V_B \subset V_K$ which is stable by the action of 
$\gl(4,\CC)$, so $\CC_v \ot_B V_B$ defines a module associated to the tableau.

\paragraph
If $v \in \CC^6$ I will write $T(x(v), y(v), z(v))$ for $T(v)$.

\paragraph
The group $S_3$ acts naturally on $\ZZ^3$, so it acts on the third row of 
tableaux and induces an action of $S_3$ over $V_K$, for example $(123) 
T(a,b,c) = T(c,a,b)$ (yes, that is the right one, I checked. Go ahead, check 
again). Since $S_3$ is just one of the many symmetric groups acting, let us
write 
\begin{align*}
\sigma(k,i)
	&\begin{cases}
		(k,i) & k \neq 3 \\
		(3,\sigma(i)) & k = 3.
	\end{cases}
\end{align*}

The orbit of $T(a,b,c)$ by this action is generated by the following vectors
\begin{align*}
\Sym(a,b,c) 
	&= \sum_{\sigma \in S_3} \sigma T(a,b,c) \\
\Alt(a,b,c) 
	&= \sum_{\sigma \in S_3} \sg(\sigma) \sigma T(a,b,c) \\
v_1(a,b,c) 
	&= \sum_{i = 0}^2 \omega^i (123)^iT(a,b,c) &
v_2(a,b,c) 
	&= \sum_{i = 0}^2 \omega^i (12)(123)^iT(a,b,c) \\
w_1(a,b,c) 
	&= \sum_{i = 0}^2 \overline \omega^i (123)^iT(a,b,c) &
w_2(a,b,c) 
	&= \sum_{i = 0}^2 \overline \omega^i (12)(123)^iT(a,b,c) \\
\end{align*}
with $\omega$ a cubic root of unity. Vectors on the same line generate a
stable subspace.

\paragraph
The action of $S_3$ induces an action on $K$, with $(\sigma \cdot p)
(x,y,z)= p(\sigma^{-1}(x,y,z))$. Thus
\begin{align*}
 \sigma \cdot p^\pm_{3,j}
	& = p^\pm_{3,\sigma(j)}
	& \sigma \cdot p^\pm_{k,j} 
	& = p^\pm_{k,j} \ \ (k \neq 3) \\
\sigma \cdot q_{3,j} 
	& = q_{3,\sigma(j)} 
	& \sigma \cdot q_{k,j} 
	& = q_{k,j} \ \ (k \neq 3)
\end{align*}
Also $\sigma \cdot p(\lambda^{v}) = (\sigma \cdot p)(\lambda^{\sigma(v)})$.
Using our previous notation we get $\sigma \cdot p^{\pm}_{k,i} = 
p_{\sigma(k,i)}$ and $\sigma \cdot q_{k,i} = q_{\sigma(k,i)}$.

From this it follows that
\begin{align*}
E_{k,k+1} T(\sigma (v))
	&= \sum_{i=1}^k 
		\frac{p_{k,i}^+(\lambda^{\sigma(v)})}{q_{k,i}(\lambda^{\sigma(v)})}
		T(\sigma(v)+\delta^{k,i})\\
	&= \sum_{j=1}^k 
		\frac{p_{\sigma(k,j)}^+(\lambda^{\sigma(v)})}{q_{\sigma(k,j)}
		(\lambda^{\sigma(v)})} T(\sigma(v+\delta^{k,j})) \\
	&= \sum_{j=1}^k 
		\left(\sigma \cdot 
			\frac{p_{k,j}^+(\lambda^{v})}{q_{k,j}(\lambda^{v})} 
		\right) T(\sigma(v + \delta^{k,j})) \\
	&= \sigma\left(E_{k,k+1}T(v)\right)
\end{align*}

\paragraph
We put a scalar product over $V_K$, with $T(v)$ being an orthonormal basis.
Clearly $S_3$ acts by unitary operators


IDEA: Take $V_\CC$ as the $\CC$-space generated by the tableaux $\ZZ_0^N$, and 
let $S_k$ act on it. Now $S_k$ also acts on $K = \CC(\lambda)$, and hence on 
$V_K = K \ot_\CC V_\CC$, and the $E_{j,j+1}$ are compatible with the 
$S_n$-action. Let $\Delta = \prod_i(x_{k,i} - x_{k,j})$ and let $c_\lambda$ 
be the Schur projectors. I claim that
\begin{align*}
B c_1(V_\CC) 
	\oplus 
		\bigoplus_{\lambda \vdash n}\frac{1}{\Delta} \cdot Bc_\lambda(V_\CC)
\end{align*}
is $\gl(n,\CC)$-stable. Maybe changing $\Delta$ for $\Delta_\lambda$?


\newpage

\section{}

\paragraph
Fix $l < n$. The group $S_l$ acts on $\ZZ_0^N$, by permuting the entries in
the $l$-th row of the tableaux. It also acts on $K = \CC(\lambda)$ by 
permuting the generators $\lambda_{l,i}$ and fixing the others. Given $\sigma
\in S_l$ we set 
\begin{align*}
\sigma(k,i)
	&= \begin{cases}
		(k,i) & k \neq l \\
		(l,\sigma(i)) & k = l.
	\end{cases}
\end{align*}
Thus $\sigma(\lambda_{k,j}) = \lambda_{\sigma(k,j)}$. Also we get that
$\sigma \cdot p(v) = p(\sigma^{-1}(v))$ for each $\sigma \in S_l, p \in K$ and 
$v \in \ZZ_0^N$ whenever this makes sense.

\begin{Lemma*}
For each $1 \leq k \leq j \leq n$ and each $\sigma \in S_l$ the following 
formulas hold.
\begin{align*}
\sigma \cdot p_{k,j}^\pm 
	&= p_{\sigma(k,j)} 
	&\sigma \cdot q_{k,j} = q_{\sigma(k,j)}.
\end{align*}
Also $\sigma \cdot (p(\lambda^v)) = (\sigma \cdot p)(\lambda^{\sigma(v)})$
for each $v \in \CC^N$.
\end{Lemma*}

\begin{Proposition}
The operators $E_{k,k+1}, E_{k,k}, E_{k+1,k}$ are $S_l$-equivariant.
\end{Proposition}
\begin{proof}
Fix $\sigma \in S_l$. We have
\begin{align*}
E_{k,k+1}(\sigma\cdot T(v))
	&= -\sum_{j=1}^k 
		\frac{p_{k,j}^+(\lambda^{\sigma(v)})}{q_{k,j}(\lambda^{\sigma(v)})}
			T(\sigma(v) + \delta^{k,j}).
\end{align*}
Taking $(k,i) = \sigma^{-1}(k,j)$ the last sum equals
\begin{align*}
-\sum_{i=1}^k 
	\frac{p_{\sigma(k,i)}^+(\lambda^{\sigma(v)})}{q_{\sigma(k,i)}
		(\lambda^{\sigma(v)})}	T(\sigma(v + \delta^{k,i}))
	&= -\sum_{i=1}^k \sigma \cdot \left( 
		\frac{p_{k,i}^+(\lambda^{v})}{q_{k,i}(\lambda^{v})} T(v+\delta^{k,i})
		\right) = \sigma (E_{k,k+1} T(v)).
\end{align*}
The proofs that $E_{k+1,k}$ is equivariant is similar. In the case of $E_{k,k}$
the result follows immediately from the definitions.
\end{proof}

\paragraph
We take the notation from \cite{FH-rt-book}*{chapter 4}. For each 
partition $\lambda \vdash k$ we set $e_\lambda$ to be the corresponding 
normalized Young symmetrizer, i.e. $e_\lambda^2 = e_\lambda$. We claim that
For each partition $\lambda$ there exists $\Delta_{\lambda} \in 
\CC[\lambda_{k,i}]$ such that 
\begin{align*}
\bigoplus_{\lambda \vdash k} 
	\frac{1}{\Delta_\lambda} B \ot_\CC e_\lambda(V_\CC)
\end{align*}
is a sub $U$-module of $V_K$.

\newpage
\begin{bibdiv}
\begin{biblist}
\bib{FGR}{article}{
   author={Futorny, Vyacheslav},
   author={Grantcharov, Dimitar},
   author={Ramirez, Luis Enrique},
   title={Singular Gelfand-Tsetlin modules of ${\germ{gl}}(n)$},
   journal={Adv. Math.},
   volume={290},
   date={2016},
   pages={453--482},
}

\bib{FH-rt-book}{book}{
   author={Fulton, William},
   author={Harris, Joe},
   title={Representation theory},
   series={Graduate Texts in Mathematics},
   volume={129},
   note={A first course;
   Readings in Mathematics},
   publisher={Springer-Verlag, New York},
   date={1991},
   pages={xvi+551}
}

\end{biblist}
\end{bibdiv}
\end{document}