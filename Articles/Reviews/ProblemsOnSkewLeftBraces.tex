%%%%%%%%%%%%%%%%%%%%%% Generalities %%%%%%%%%%%%%%%%%%5
\documentclass[11pt,fleqn]{article}
\usepackage[paper=a4paper]
  {geometry}

\pagestyle{plain}
\pagenumbering{arabic}
%\linespread{1.2}
\setlength{\parskip}{1.2ex}


\usepackage[utf8]{inputenc}
\usepackage[spanish]{babel}
\usepackage{enumerate}
\usepackage[osf,noBBpl]{mathpazo}
\usepackage[alphabetic,initials]{amsrefs}
\usepackage{amsfonts,amssymb,amsmath}
\usepackage{mathtools}

\newcommand\CC{\mathbb C}
\newcommand\NN{\mathbb N}
\newcommand\ZZ{\mathbb Z}
\newcommand\RR{\mathbb R}

\begin{document}
Observaciones, typos, comentarios, etc. sobre ``Problems on skew left braces''.

\textbf{Página 1, linea 2} El ``would'' de la segunda frase debería ser 
''will''.

\textbf{Página 1, Linea -4} Definir $\mathbb S_X$? No es más claro decir 
simplemente que las funciones son biyectivas?

\textbf{Página 2, segunda linea después de (4)} ``Examples can be found for 
example...''

\textbf{Página 2, primer teorema} ``If $A$ is a skew left brace $A$''.

\textbf{Página 2, segundo teorema} Hay que cambiar el ``If'' del principio por 
un ``Let''. Ya dijiste que las soluciones son todas non-degenerate.

\textbf{Página 2, última línea antes de Problems} El ``could'' en esa última 
linea debería ser ``can''. 

\textbf{Página 2, segunda linea de problems} ``cycle set'' debería estar en 
plural.

\textbf{Página 3, problema 1} Esto es puro gusto mío, pero no sería mejor 
formularlo como pregunta? ``Is there a free cycle set over a set X? If so, give 
an explicit presentation''

\textbf{Página 3, debajo del problema 1} ``Tables'' debería estar en singular. 
Decís ``size'' en lugar de ``cardinality'', que yo ví que lo hacen ESS, pero por 
ahí confunde si no estás en tema.

\textbf{Página 3, antes del problema 3} Estaría bueno aclarar que $x \cdot x = 
x$ implica que $r(x,x) = (x,x)$, para no perder de vista las soluciones de YBE.

\textbf{Página 3, antes del problema 5} ``the number of solutions grows fast'' 
debería decir ``for small $n$'', ¿no? Si hay cotas inferiores para el número de 
soluciones estaría bueno mencionarlas!

\textbf{Página 3, después del problema 5} La primera frase suena rara en 
inglés. Mejor ``Much less is known for non-involutive solutions''.

\textbf{Página 4, segundo párrafo} Otra vez, dar vuelta la primera frase.

\textbf{Página 4, antes del problema 10} Solutions debería estar en singular.

\textbf{Página 4, problema 11} ``When \emph{does} the multiplicative group 
$G(X,r)$ \emph{have} torsion?'' 

\textbf{Página 4, problema 12} ``When \emph{does} the multiplicative group 
$G(X,r)$ \emph{admit} a left ordering?'' 

\textbf{Página 4, problema 14} Algunos? Todos?

\textbf{Página 4, después del problema 14} Suena raro hablar de resultados 
parciales de ese problema. Falta una ``a'' antes de ``reasonable time''. Me pasó 
algo gracioso con la última frase, no entendí si es dificil porque hay pocos 
(entonces es dificil encontrar ejemplos) o porque hay muchos (entonces es 
dificil encontrarlos todos).

\textbf{Página 5, abajo de la tabla} Los números de la forma $p^n m$ con $m$ 
coprimo con $p$ son todos!

\textbf{Página 5, última línea} Nejabati es un gran nombre, pero estás 
llamando a la gente solo por apellido.

\textbf{Página 6, abajo del problema 16} ``It should be fairly easy to solve 
problem 16 using the results (techniques??) from [13].''

\textbf{Página 6, antes del problema 20} ``In full generality, conjecture is 
known to be false'', falta un the después de la coma.

\textbf{Página 6, problema 21} Me llama la atención que le pongas
''Gateva-Ivanona pairs'' a los contraejemplos de la conjetura de 
Gateva-Ivanova. Yo les pondría Vendramin pairs, o Anti-Tatianic pairs.

\textbf{Pagina 7, segunda línea} No es claro cuál es la solución inducida.

\textbf{Página 7, antes del problema 23} ``... for all skew braces, despite in 
general''. En lugar de despite debería decir ``although''.

\textbf{Página 8, línea 5} ``turns'' debería ser ``turn''.

\textbf{Página 9, problema 30} Minimal en qué sentido? Si decís ``the minimal''
parece que ya sabés que es único.

\textbf{Página 9, antes del problema 35} En lugar de ``produces with'' yo pondría
''obtains from''. Usás la frase ``Biquandels are useful'' en dos oraciones 
seguidas.

\textbf{Página 9, después del problema 35} ``strengthen'' debería ser 
''strengthened''.

\textbf{Página 9, problema 36} ``Is \emph{it} possible...''

\textbf{Página 9, antes del problema 37} ``These definitions extend...''

\textbf{Página 10, problema 44} No entendí el enunciado. ¿Ya hay una 
definición de ideal nil para braces, y preguntás si vale Köthe? ¿O estás 
pidiendo una definición de ideal nil que extienda la de anillos radicales,
para poder formular Köthe para braces?

\textbf{Página 10, problema 48} Así como está formulado ¡parece que las 
soluciones de YBE estuvieran teniendo un orgasmo! Mejor ``Does every 
indecomposable solution of YBE come from a one-generator skew left brace?''

\textbf{Página 11, primera línea} ``question'' debería ir en plural.

Entiendo que no va del todo con el espíritu de la lista, pero un problema
para pensar es: ¿Es inyectivo el funtor que asigna a cada solución conjuntista
un espacio vectorial trenzado? Y si no, ¿cuándo dos soluciones conjuntistas 
diferentes inducen el mismo espacio vectorial trenzado? ¿Cuál es la imagen de 
este funtor? Otra, ¿se puede ``torcer'' ese funtor, para obtener nuevas 
soluciones?
\end{document}
