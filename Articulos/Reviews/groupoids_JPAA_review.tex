%%%%%%%%%%%%%%%%%%%%%% Generalities %%%%%%%%%%%%%%%%%%5
\documentclass[11pt,fleqn]{article}
\usepackage[paper=a4paper]
  {geometry}

\pagestyle{plain}
\pagenumbering{arabic}


\usepackage[utf8]{inputenc}
\usepackage[english]{babel}
\usepackage{enumerate}
%\usepackage[osf,noBBpl]{mathpazo}
\usepackage[alphabetic,initials]{amsrefs}
\usepackage{amsfonts,amssymb,amsmath}
\usepackage{mathtools}

\title{Review of 
``The Groupoid Galois Extension and the Partial
Isomorphisms Groupoid''}
\author{Wagner Cortes and Tha\'isa Tamusiunas}

\begin{document}
\maketitle

I am recommending the article to be rejected, both on mathematical grounds and
for a poor exposition of the results. The results as stated in the 
introduction seem to be outside the aim and scope of the Journal of Pure and 
Applied Algebra. I can make no claims regarding the validity of said results, 
since the exposition of the proofs was obscure and several of the statements
made no obvious sense. Let me give four examples.

First, it was very dificult to make sense of the definitions in section 2. 
The condition that the morphisms $\beta_g: E_{g^{-1}} \to E_g$ between ideals 
be ``isomorphisms of $K$-algebras'' (page 3, middle of the page) and the 
presence of a mysterious element $1_g$ in condition (ii) of the
definition of a \emph{Galois extension} only made sense after skipping them 
and reading section 3, where it is stated that the ideals must be 
generated by a local unit, a restriction not present in [1] (where algebras
are allowed to be non-unital). 

Second, I was unable to understand the statement of Proposition 3.1. The 
$R$-algebra $S$ is a direct sum of the form $\bigoplus_{e \in G_0} E_e$, but
in that statement an element $s \in S$ is decomposed as $\sum_{g \in G} s_g$,
without clarification. Also the statement claims that it is possible to 
calculate a certain image $\phi_g(s_g)$ in terms of images of $s_{g^{-1}}$, 
which only makes sense if $s_g$ determines $s_{g^{-1}}$. Since no explanation
is given as to what exactly is $s_{g^{-1}}$, it is impossible to decide if
this assertion is correct. Reading the proof did not help clarify this matter.
A similar statement appears in the proof of Corollary 3.2.

Third, right after the proof of Proposition 3.1 the authors begin discussing 
the star product $S \star G$ without ever defining it, and look at a map 
$j: S \star G \to End_R(S)$. These definitions are again taken from [1], 
without clarification, but one may assume that this star 
product is an analogue of the star product between an 
algebra and a group acting on it, which begs the question of what the authors 
mean by $E_h \star {}_g \mathcal G_g$ since the set on the right is not a 
groupoid unless $g \in G_0$.

Fourth, the proof of Theorem 3.5 states that a groupoid can be written as a 
disjoint union of sets ${}_g \mathcal G_g$ over $g \in G$. However if $g$ and
$h$ have the same source and target in $G_0$ then ${}_h \mathcal G_h = {}_g 
\mathcal G_g$, so either there must be some extra hypothesis on the groupoid 
or the proof needs to be ammended.

\bigskip

\noindent [1] D. Bagio; A. Paques, \emph{Partial groupoid actions: 
globalization, Morita theory and Galois theory}, Comm. Algebra (2012).
\end{document}