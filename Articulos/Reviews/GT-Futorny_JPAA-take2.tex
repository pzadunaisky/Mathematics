%%%%%%%%%%%%%%%%%%%%%% Generalities %%%%%%%%%%%%%%%%%%5
\documentclass[11pt,fleqn]{article}
\usepackage[paper=a4paper]
  {geometry}

\pagestyle{plain}
\pagenumbering{arabic}
%\linespread{1.2}
\setlength{\parskip}{1.2ex}


\usepackage[utf8]{inputenc}
\usepackage[english]{babel}
\usepackage{enumerate}
\usepackage[osf,noBBpl]{mathpazo}
\usepackage[alphabetic,initials]{amsrefs}
\usepackage{amsfonts,amssymb,amsmath}
\usepackage{mathtools}

\newcommand\CC{\mathbb C}
\newcommand\NN{\mathbb N}
\newcommand\ZZ{\mathbb Z}
\newcommand\RR{\mathbb R}

\begin{document}

\noindent\textbf{Review of 
``Irreducible subquotients of generic Gelfand-Tsetlin modules over 
$U_q(\mathfrak{gl}_n)$''}, by V. Futorny, L.E. Ramírez, J. Zhiang

\noindent\textbf{Recommendation:} Accept with minor changes.

All problems with the previous text have been addressed, so I recommend the 
article for publication. I add some comments, mostly on typos and small 
corrections.

\bigskip

Page 2, line 7. The $h$ in the exponent should be regular, not fraktur. It 
should also be defined somewhere. Let me insist that $q^x$ does not make sense 
in general and must be explicitly defined as notation for $e^{hx}$.

Page 3, lines 3-6. The statement that the center is generated by the elements
$c_{mk}$ needs a reference.

Page 4, proof of Theorem 4.3. There are still some $\sigma_i$'s which should be
$\sigma(i)$'s.

Page 5, proof of Proposition 5.3. There's no need to do this by contradiction,
the proof is straightforward.

Page 5, Theorem 5.4. $\Gamma$ should be $\Gamma_q$.

Page 7, Definition 6.1. The definition of $v_{p,s,u}(T(R))$ is missing.

Page 10, Theorem 7.7. If I understand correctly, the basis of $M$ consists of
tableaux which differ from $T(v)$ in only one entry, and the difference is 
between $0$ and $d$. Such a basis has cardinality $\frac{n(n-1)}{2}d$. I 
suppose the authors mean that the basis consists of tableaux $T(w)$ with 
$w_{ni} = v_{ni}$ and $0 \leq w_{ki} - v_{ki} < d$ for all $k < n$.
\end{document}