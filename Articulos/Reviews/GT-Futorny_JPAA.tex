%%%%%%%%%%%%%%%%%%%%%% Generalities %%%%%%%%%%%%%%%%%%5
\documentclass[11pt,fleqn]{article}
\usepackage[paper=a4paper]
  {geometry}

\pagestyle{plain}
\pagenumbering{arabic}
%\linespread{1.2}
\setlength{\parskip}{1.2ex}


\usepackage[utf8]{inputenc}
\usepackage[english]{babel}
\usepackage{enumerate}
\usepackage[osf,noBBpl]{mathpazo}
\usepackage[alphabetic,initials]{amsrefs}
\usepackage{amsfonts,amssymb,amsmath}
\usepackage{mathtools}

\newcommand\CC{\mathbb C}
\newcommand\NN{\mathbb N}
\newcommand\ZZ{\mathbb Z}
\newcommand\RR{\mathbb R}

\title{Review of 
``Irreducible subquotients of generic Gelfand-Tsetlin modules over 
$U_q(\mathfrak{gl}_n)$''}
\author{V. Futorny, L.E. Ramírez, J. Zhiang}
\date{}
\begin{document}
\maketitle

\noindent\textbf{Recommendation:} Revise and resubmit.

Fix $n \in \NN$. Let $U$ be the enveloping algebra of $gl_n(\CC)$ and let 
$\Gamma \subset U$ be the Gelfand-Tsetlin subalgebra of $U$. A Gelfand-Tsetlin
module is a $U$-module on which the action of $\Gamma$ is locally finite. The 
class of Gelfand-Tsetlin modules has been of great interest in the last thirty 
years, and the first author and his collaborators have made large 
contributions to our understanding of this class. Since an analogous notion 
of Gelfand-Tsetlin modules exists for quantized enveloping algebras, it is 
natural to pursue a similar study in the quantum context and this article
aims to begin with this project. 

The results presented in this article are very interesting and certainly 
within the aims and scope of the JPAA; also, they would represent the first 
step for a very ambitious and interesting research program. I am 
particularly intrigued by applications to representations of quantized 
enveloping algebras at roots of unity. However the present document suffers 
from far too many typos, careless writing and missing arguments to be 
considered more than a draft. It needs a thorough revision before I can 
recommend it for publication, which I will gladly do once the authors take 
care of the problems discussed below.


\bigskip

\noindent\textbf{Comments, corrections and suggestions}

Page 1, line 3. The phrase 'a classical Gelfand-Tsetlin basis' should say
'classical Gelfand-Tsetlin bases'

Page 1, line 10. Delete the 'the' in the phrase 'among the others'.

Page 1, lines 14-17. Reference [11] is unpublished and yet is spoken of in the 
past tense. In any case it is unclear whether the authors mean [11] or the 
paper under review by the phrase 'this paper'.

Page 2, lines 7-9. The authors fix the proverbial $q \in \CC$, and set $1(q) 
= \{x \in \CC \mid q^x = 1\}$. Since complex exponentiation is not uniquely 
defined, this set is not well-defined and yet the authors make no comment 
regarding it. This is a common omission in the literature on quantum groups,
and whole books have been written on the subject without giving this problem 
any thought, since one usually considers only integral powers of $q$ in which 
case there is no ambiguity. However the set $1(q)$ plays a central role in 
this article, and several statements are made obscure by this omission, in 
particular Definition 6.1 and Theorem 6.2 (the main results!). This is not 
really a problem if, as in [22], one first fixes $h \in \CC$ and sets $q^x = 
\mathsf{exp}(hx)$, in particular $q = \mathsf{exp}(h)$. Only once this is done 
does the definition of $1(q)$ makes sense, and $1(q) = \{\frac{2 k \pi}{h}i 
\mid k \in \ZZ\}$. Notice in particular that unless $q$ is a root of unity, a 
different choice of $h$ will change the set $1(q)$. Since most of the contents 
of this article hinge on the property that certain numbers lie in 
$\frac{1(q)}{2} + \ZZ$, this should be addressed at some point. See also the
comment about Definition 6.1 and Theorem 6.2.

Page 2, line 10. The 'n' after 'rank' should be between \$'s.

Page 2, line 14. $U_q$ is 'the' algebra and not 'a' algebra defined by the 
relations (1)-(7). Also the index $i$ should only go up to $n-1$.

Page 2, line 25. The word 'representation' should be 'presentation'.

NOTE: Reference [13] is only available in Russian, or in an awful
English translation where the numeration has been changed and statements 
abbreviated. Is there any alternative source, or alternative translation?

Page 3, lines 3-4. There should be no point between '$\mathfrak{gl_m}$' and 
'We'. The word 'denotes' should be 'denote'.

Page 3, line 6. Theorem 3.2.
As far as I can tell the statement in [11] is about $sl_n$ and not 
$gl_n$, and $k = 1, \ldots, n-1$ there, so the statement does not seem to
follow immediately. Also, what does it mean when $l_{i,j}^+$ or $l_{j,i}^-$ 
with $i > j$ appear in this formula? Finally, does the result hold with $q$ a 
root of unity? (This is the first of several instances in which this question
is ignored in statements, but used in the proofs.)

Page 3, line 16. Proof of Lemma 3.4. The proof in [9] is short and the idea is
quite simple: the algebra $\Gamma$ separates elements with different 
characters. A short comment, or better yet a copy of the short proof, would 
make this part of the article five lines longer and much more readable.

Page 3, line 22. The choice of notation is a bit unfortunate, as we now have
$l_{i,j}^\pm$ and $l_{i,j}$, which mean completely different things.

Page 4, line 1. As stated in [22], the theorem was originally proved in M. 
Jimbo, ``Quantum $R$ matrix related to the generalized Toda system: an 
algebraic approach'', section 5. Also the formulation found here is more like
the one found in [19] than the one found in [25]. Finally in this case $q$
must certainly not be a root of unity, and this has to be clearly stated.
(More on this later.) 

Page 4, line 9. Proposition 4.3 and proof. At some point $\sigma(i)$ becomes 
$\sigma_i$ without any clarification. The permutations described in the 
statement are usually called $(k,n-k)$-shuffles. Let us denote the set of 
$(k,n-k)$-shuffles by $Sh$. The last string of equalities would be easier to 
follow with a comment such as ``Since $\sum_{\sigma \in S_n} q^{\ell(\sigma)} 
= (n)_q$ and since $S_n = \bigsqcup_{\tau \in Sh} (S_k \times S_{n-k})\tau$, 
the following equality holds
\begin{align*}
\sum_{\sigma \in S_n} q^{-2\ell(\sigma)} 
	= \sum_{\sigma' \in S_k} q^{-2\ell(\sigma')} 
	\sum_{\sigma'' \in S_{n-k}} q^{-2\ell(\sigma'')}
	\sum_{\tau \in Sh} q^{-2\ell(\tau)}
	= (k)_{q^{-2}}! (n-k)_{q^{-2}}! \sum_{\tau \in Sh} q^{-2\ell(\tau)}
\end{align*}
so...''

Page 5, Corollary 4.4. It is not clear a priori that the eigenvalues of 
$c_{mk}$ depend only on the $m$-th row, or that $c_{mk}$ is diagonalizable at 
all, since $c_{mk}$ involves $l_{ij}^\pm$ with $1 \leq i,j \leq m$. I believe
a more correct statement would be that $c_{mk} \in (U_m)_q \subset U_q$, and 
the action of $(U_m)_q$ on the bottom $m$-rows is given by the Gelfand-Tsetlin
formulas, so $c_{mk}$ acts as if the table $T(R)$ was of height $m$, ignoring
all rows above $m$.

Page 5, Definition 5.1. As stated above, this depends not just on $q$ but on 
$h$, and equivalently states that
\begin{align*}
l_{ij} - l_{ik} \notin \{\frac{2r\pi}{h}i + s \mbox{ with } r,s \in \ZZ\}.
\end{align*}

Page 5, Theorem 5.2. $q$ should not be a root of unity.

Page 5, Proposition 5.3. 'separate' should be 'separates'. Also the proof 
assumes that $q$ is not a root of unity.

Page 6, Definition 5.5. Add 'irreducible' after the word 'unique'.

Page 6, Notation 5.6. In the statement $L \in \CC^{n(n+1)/2}$ but $z \in 
\ZZ^{\frac{n(n-1)}{2}}$, so what is $L+z$?. 

Page 6, Definition 5.7. 'exist' should be 'exists'.

Page 6, Theorem 5.9. Point (i) is equivalent to saying that $W(T(R))$ is a 
submodule of $V(T(R))$, and this seems clearer. In 
page 7 line 4, the proof begins by assuming $e_k T(S) \notin W(T(R))$ but ends
proving that $e_k T(S) \in W(T(R))$ without using the hypothesis. Finally, 
at the bottom of page 7, the inductive step is almost incomprehensible and 
should be rewritten.

Page 8, Definition 6.1. Since the last few results are only true if $q$ is not
a root of unity, I am assuming that this hypothesis is also in place here. 
Notice that this implies that $\frac{1(q)}{2} + \ZZ = \ZZ\frac{\pi}{h}i \oplus 
\ZZ$. Now fix $1 \leq u < p \leq n$. By the definition of a 
$q$-generic tableaux there is at most one $1 \leq s \leq p$ such that
$(p,s,u) \in \Omega(T(R))$, and since the sum above is direct $d_{p,u}$ is 
either $1$ or $0$, depending on whether $\Omega(T(R))$ contains an element of 
the form $(p,s,u)$ or not.

Page 8, Theorem 6.2. In the statement, there is an 'a' missing before 
'$q$-generic tableau'. Also item (i) is not really clear, since by Theorem 5.9 
$\mathcal I(T(R))$ does not generate a submodule of $V(T(L))$ in general; the 
proof makes it clear that one should see tableaux $T(S)$ with $\Omega^+(T(R))
\subsetneq \Omega^+(T(S))$ as zero. Finally, by the previous comment the number
of irreducible modules in the block associated with $T(L)$ is $2^\omega$, with
$\omega = | \Omega(T(L))|$.

Page 9, Remark 7.2. The first claim only holds if $T(L)$ is standard. 
The claim that Theorem 5.2 holds when $q$ is a root of unity deserves at least 
a separate statement as a lemma and a proof. The argument that generic 
Gelfand-Tsetlin modules are indeed modules in the non-root-of-unity case is 
done by arguing that every standard tableaux appears in a finite dimensional 
representation, which does not happen in the present context.

Page 9, Theorem 7.3. The equality should be a congruence. In the first line of 
the proof, delete the word 'two'. The last line of the proof states that since
two numbers are different modulo $e$, they are equal.

Page 9, Proposition 7.4. Is $T(L)$ generic? Also, the second to last line of 
the proof mentions $W(T(R))$, which is not defined yet.

Page 10, line 23. $W_{ij}(R)$ is defined as a submodule, but it is never stated
a submodule of what.

Page 10, proof of Theorem 7.7. There is no Proposition 7.3.

Page 10, Definition 7.9. What is $v_{i+1,s}$? Without this definition I was 
unable to understand Definition 7.10 and Theorem 7.11.

Page 11, Example. Set $n = 2$. This should also be made explicit in 
Theorem 7.12.

\noindent\textbf{References.}

\noindent [11] V. Futorny,J. Hartwig, M. Rosso, Gelfand-Tsetlin characters for 
quantum $gl(n)$. In progress.

\noindent [13] L. Faddeev, N. Reshetikhin and L. Takhtadzhyan, Quantization of 
Lie groups and Lie algebras, Leningrad Math. J. 1 (1990), 193–225.

\noindent [19] A.Klimyc, K. Schmüdgen, Quantum groups and their 
representations, Springer-Verlag, Berlin Heidelberg, 1997.

\noindent [22] V. Mazorchuk, L. Turowska, On Gelfand-Tsetlin modules over 
$U_q (gl(n))$, Czechoslovak Journal of Physics, (2000), 139–141

\noindent [25] K. Ueno, Y. Shibukawa, T. Takebayashi, Construction of 
Gelfand-Tsetlin Basis for $U_q(gl(N +1))$-modules. Publ. RIMS, Kyoto Univ. 26 
(1990), 667-679.
\end{document}